\documentclass{amsart}

\title{multline}

\csname noTeXMLhistory\endcsname

\def\theequation{EQ\arabic{equation}}

\begin{document}

\maketitle

\section{one}

\tracingmacros=1

\begin{equation}\label{EQ}
e = mc^2
\end{equation}

\texttt{multline} is odd in that there can only be one equation tag.
If you specify a custom \texttt{tag}, it can go on any line, but the
tag still gets moved to the top (for leqno) or bottom (for reqno)
lines.

\texttt{texml} will move all \texttt{tag}s to the last line.

\begin{multline}
  XXX\tag{a}\\YYY\\ZZZ
\end{multline}

\begin{multline}
  XXX\\YYY\tag{b}\\ZZZ
\end{multline}

\begin{multline}
  XXX\\YYY\label{YYY}\\ZZZ\tag{c}
\end{multline}

EQ \eqref{EQ}

YYY \eqref{YYY}

\end{document}
