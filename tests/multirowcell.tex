%% AMS prddvilualatex

\documentclass{amsart}

\usepackage{multirow}
\usepackage{makecell}
\usepackage{float}

\title{multirowcell}

\csname noTeXMLhistory\endcsname

\begin{document}

\maketitle

hello, world

\section*{multirow(3)}

\begin{table}[H]
\caption{}
\begin{tabular}{cccc}
1 & 1                  & 1 & 1\\
2 & \multirow{3}{*}{R2--4} & 2 & 2\\
3 &                    & 3 & 3\\
4 &                    & 4 & 4\\
5 & 5                  & 5 & 5\\
6 & 6                  & 6 & 6
\end{tabular}

\end{table}

\section*{multicolumn(2), multirowcell(3)}

\begin{table}[H]
\caption{}
\begin{tabular}{llll}
A &   B & C                & D\\
E &   \multicolumn{2}{c}{\multirowcell{3}{R2--4, C2--3}} &H\\
I &                        & & J \\
K &                        & & L \\
M &   N & O                & P\\
Q &   R & S                & T
\end{tabular}

% 0   1  2  3
% 4   5  5  6
% 7   5  5  8
% 9   5  5 10
% 11 12 13 14
% 15 16 17 18
\end{table}

\section*{multicolumn(3), multirowcell(3)}

\begin{table}[H]
\caption{Cf.\ table 3 in \texttt{muticolumn}}
\begin{tabular}{lllll}
A &   B & C                & D & D\\
E &   \multicolumn{3}{c}{\multirowcell{3}{R2--4XXXXXXXXXXXXXXXXXXXXX, C2--3}} &H\\
I &                        & & & J \\
K &                        & & & L \\
M &   N & O                & P & P\\
Q &   R & S                & T & T
\end{tabular}
\end{table}

\end{document}
