\documentclass{amsart}

%\nofiles

\usepackage{hyperref}

\usepackage{cleveref}

\newtheorem{theorem}{Theorem}

\newtheorem{definition}[theorem]{Definition}

\begin{document}

See [section cref] \cref{sec:intro}.

See [section Cref] \Cref{sec:intro}.

See [theorem cref] \cref{thm:1}

See [theorem Cref] \Cref{thm:1}

See [equation cref] \cref{ref:eq}

See [equation Cref] \Cref{ref:eq}

See [ref definition] \ref{def:1}

See [definition] \cref{def:1}

See [two definitions] \cref{def:1,def:2}

See [three definitions] \cref{def:1,def:2,def:3}

See [many definitions] \cref{def:1,def:2,def:3,def:4,def:5}

See [nonsequential definitions] \cref{def:1,def:3,def:5}

See [definition crefrange] \crefrange{def:1}{def:5}

See [definition Crefrange] \Crefrange{def:1}{def:5}

[labelcref] \labelcref{def:1}

[namecref] \namecref{def:1}

[nameCref] \nameCref{def:1}

[lcnamecref] \lcnamecref{def:1}

[namecrefs] \namecrefs{def:1}

[nameCrefs] \nameCrefs{def:1}

[lcnamecrefs] \lcnamecrefs{def:1}

\section{Introduction}
\label{sec:intro}

This is \cref{sec:intro}.

\begin{equation}
\label{ref:eq}
1 + 1 = 2
\end{equation}

\begin{theorem}\label{thm:1}
Here is a theorem.
\end{theorem}

\begin{definition}
\label{def:1}
Here is a definition.
\end{definition}

\begin{definition}
\label{def:2}
Here is another definition.
\end{definition}

\begin{definition}
\label{def:3}
Here is another definition.
\end{definition}

\begin{definition}
\label{def:4}
Here is another definition.
\end{definition}

\begin{definition}
\label{def:5}
Here is another definition.
\end{definition}

\end{document}
