\documentclass{amsart}

\ifx\noAMSmetadata\undefined
    \usepackage{algorithmic}
\else
    \LoadPackage{Algorithmic}
    \noAMSmetadata
\fi

\title{algorithmic}

\parindent0pt

\begin{document}

\maketitle

hello, world

% \tracingmacros=1

\section{Basic forms}

\subsection{The Simple Statement}

\begin{algorithmic}[1]
\STATE $S \leftarrow O$
\end{algorithmic}

\begin{algorithmic}[1]
\STATE $S \leftarrow O$ \COMMENT{comment}
\end{algorithmic}

\subsection{The body}

\begin{algorithmic}[1]
\BODY[comment]
    \STATE something 1
    \STATE something 2
\ENDBODY
\end{algorithmic}

\subsection{The \emph{if-then-else} Statement}

\tracingmacros=1

\begin{algorithmic}[3]
\IF[comment]{some condition is true}
    \STATE do some processing
\ELSIF[comment]{some other condition is true}
    \STATE do some different processing
\ELSIF[comment]{some even more bizarre condition is met}
    \STATE do something else
\ELSE[comment]
    \STATE do the default actions
\ENDIF
\end{algorithmic}

\subsection{The \emph{for} Loop}

\begin{algorithmic}[1]
\FOR[comment]{$i=0$ to $10$}
    \STATE carry out some processing 
\ENDFOR
\end{algorithmic}

\begin{algorithmic}[1]
\FORALL[comment]{$i$ such that $0\leq i\leq 10$}
    \STATE carry out some processing 
\ENDFOR
\end{algorithmic}

\begin{algorithmic}[1]
\FOR[comment]{$i=0$ \TO $10$}
    \STATE carry out some processing 
\ENDFOR
\end{algorithmic}

\subsection{The \emph{while} Loop}

\begin{algorithmic}[1]
\WHILE[comment]{some condition holds}
    \STATE carry out some processing 
\ENDWHILE
\end{algorithmic}

\subsection{The \emph{repeat-until} Loop}

\begin{algorithmic}[1]
\REPEAT[comment]
    \STATE carry out some processing 
\UNTIL{some condition is met}
\end{algorithmic}

\subsection{The Infinite Loop}

\begin{algorithmic}[1]
\LOOP[comment]
    \STATE this processing will be repeated forever
\ENDLOOP
\end{algorithmic}

\subsection{The Precondition}

\begin{algorithmic}[1]
\REQUIRE $x \neq 0$ and $n \geq 0$
\end{algorithmic}

\subsection{The Postcondition}

\begin{algorithmic}[1]
    \ENSURE $x \neq 0$ and $n \geq 0$
\end{algorithmic}

\subsection{Globals}

\begin{algorithmic}[1]
    \GLOBALS $x$, $y$
\end{algorithmic}

\subsection{Inputs}

\begin{algorithmic}[1]
    \INPUTS[comment]
        \STATE $x$, $y$
    \ENDINPUTS
\end{algorithmic}

\subsection{Outputs}

\begin{algorithmic}[1]
    \OUTPUTS[comment]
        \STATE $x$, $y$
    \ENDOUTPUTS
\end{algorithmic}

\subsection{Returning Values}

\begin{algorithmic}[1]
\RETURN $(x+y)/2$
\end{algorithmic}

\subsection{Printing Messages}

\begin{algorithmic}[1]
\PRINT \texttt{``Hello, World!''}
\end{algorithmic}

\subsection{Comments}

\begin{algorithmic}[1]
\STATE do something \COMMENT{this is a comment}
\end{algorithmic}

\section{Some longer examples}

\begin{algorithmic}
    \STATE $a \leftarrow 1$
    \IF{$a$ is even}
        \PRINT ``$a$ is even''
    \ELSIF{$a$ is odd}
        \PRINT ``$a$ is odd''
    \ELSE
        \PRINT ``$a$ is really weird''
    \ENDIF
\end{algorithmic}

\begin{algorithmic}
    \REQUIRE $n \geq 0$
    \ENSURE $y = x^n$
    \STATE $y \leftarrow 1$
    \STATE $X \leftarrow x$
    \STATE $N \leftarrow n$
    \WHILE{$N \neq 0$}
        \IF{$N$ is even}
            \STATE $X \leftarrow X \times X$
            \STATE $N \leftarrow N / 2$
        \ELSE[$N$ is odd]
            \STATE $y \leftarrow y \times X$
            \STATE $N \leftarrow N - 1$
        \ENDIF
    \ENDWHILE
\end{algorithmic}

\subsection{mcom3655}

\begin{algorithmic}
\STATE {\textbf{Input:}  Choose an arbitrary $\alpha_{{{L}}}^{0}\in\mathbb{C}^{{{(L+1)^2}}}$,  $\{\epsilon^{k}\}$ with $\epsilon^{k}>0$. Set $k=0$.}
\WHILE{a termination criterion is not met,}
\STATE { 1) Solve the weighted $\ell_{2}$ minimization problem
\begin{equation}\label{alpha}
\alpha^{k+1}_{l\cdot}=\arg\min\limits_{\alpha_{l\cdot}\in\mathbb{C}^{2l+1}} \left\{ \frac{1}{2}\|\alpha_{l\cdot}-\alpha_{l\cdot}^{\circ}\|_{2}^{2}+\frac{1}{2}\lambda p\beta_{l}w_{l}^{k}\|\alpha_{l\cdot}\|_{2}^{2}\right\},{{l=0,1,\ldots,L}},
\end{equation}
\quad  where $w_{l}^{k}=(\|\alpha^{k}_{l\cdot}\|_{2}^{2}+\epsilon^{k})^{\frac{p}{2}-1}$.}

\STATE {2) Set $k\leftarrow k+1$ and go to step 1).}
\ENDWHILE
\end{algorithmic}


\subsection{mcom3356}

	\begin{algorithmic}[1]
		\REQUIRE A lower trapezoidal $n\times (n-1)$ matrix $\mathbf{H}=(h_{i,j})$ with $h_{i,j}=0$ if $j>i$ and $h_{j,j}\ne 0$.
		\ENSURE A unimodular matrix $\mathbf D$ such that $\mathbf H:=\mathbf D\cdot \mathbf H = (h_{i,j})$ satisfying  $|h_{i,j}|\le |h_{j,j}|/2$ for $1\le j<i\le n$.

		\STATE $\mathbf{D} := \mathbf{I}_n$.
		\FOR {$i$ from $2$ to $n$}
		\FOR {$j$ from $i-1$ to $1$ by stepsize $-1$}
		\STATE {$q :=\lfloor h_{i,j}/h_{j,j}+0.5\rfloor$.}
		\FOR {$k$ from $1$ to $n$}
		\STATE {$d_{i,k}:= d_{i,k}-qd_{j,k}$.}
		\ENDFOR
		\ENDFOR
		\ENDFOR
	\end{algorithmic}

\subsection{mcom3363}

	\begin{algorithmic}[1]
		\REQUIRE A polynomial $h$  with irreducible factors of degree $d$ over $k=\mathbb{F}_q[X]/f(X)$.
		\ENSURE An irreducible factor of $h$ over $k$.
		\STATE If $\deg h = d$ return $h$.
		\STATE Take a random polynomial $a_0\in k[Z]$ of degree less than $\deg h$,
		\STATE\label{alg:ks-pseudotrace} Compute $a_1
		\leftarrow \sum_{i=0}^{md-1} a_0^{q^i} \mod h$,
		\IF{$q$ is an even power $q=2^e$}
		\STATE\label{alg:ks:even} Compute $a_2 \leftarrow
		\sum_{i=0}^{e-1} a_1^{2^i}\mod h$
		\ELSE
		\STATE\label{alg:ks:odd} Compute $a_2 \leftarrow a_1^{(q-1)/2}\mod h$
		\ENDIF
		\STATE\label{alg:ks:gcd} Compute $h_0\leftarrow\gcd(a_2,h)$ and
		$h_1\leftarrow\gcd(a_2-1,h)$ and $h_{-1}\leftarrow h/(h_0h_1)$,
		\STATE Apply recursively to the smallest non-constant polynomial among
		$h_0,h_1,h_{-1}$.
	\end{algorithmic}

\subsection{mcom3385 (with endtags=no)}

\makeatletter
\ALC@noendtrue
\makeatother

  \begin{algorithmic}[1]
    \REQUIRE A rational number $\alpha=a/b$.
    \ENSURE The algorithm tells whether the RCF expansion of $\alpha$ is finite or periodic.
    \STATE $x:=\alpha$
    \STATE $B_1:=\max\left(\frac{\log b}{\log\ell},2\right)$
    \FOR{$i=1$ to $B_1$}
        \IF{$x<0$}
            \RETURN{The expansion is periodic.}
        \ENDIF
        \STATE $y:=x-\lfloor x\rfloor_\ell$
        \IF{$y==0$}
            \RETURN{The expansion is finite.}
        \ENDIF
        \STATE $x:=1/y$
    \ENDFOR
   \end{algorithmic}

\end{document}
