\documentclass{amsart}

\title{tabular}

\begin{document}

\maketitle

hello, world

\begin{table}[h]
\caption{}
\begin{tabular}{@{}|cc}
A & B
\end{tabular}
\end{table}

\begin{table}[h]
\caption{}
\begin{tabular}{||c|c||}
A & B
\end{tabular}
\end{table}

\begin{table}[h]
\caption{The \texttt{hfill} in the third row of this and the following
  table is lost.  I'm not going to try to replicate that in texml; use
  \texttt{multicolumn} instead.}
\begin{tabular}{l}
    XXXXXXXXXXX\\
    X\\
    \hfill X\\
\end{tabular}
\end{table}

\begin{table}[h]
\caption{The width specifier is ignored: \texttt{tabular*} is treated
  identically to \texttt{tabular} by \texttt{texml}.  Should that change?}
\begin{tabular*}{5in}{l}
    XXXXXXXXXXX\\
    X\\
    \hfill X\\
\end{tabular*}
\end{table}

\begin{table}[h]
\caption{}
\begin{tabular}{ll@{\extracolsep{10pt}}llllllll}
    1&2&3&4&5&6
\end{tabular}
\end{table}

\begin{table}[h]
\caption{}
\begin{tabular}{ll@{\hskip20pt}l}
    1&2&3
\end{tabular}
\end{table}

\begin{table}[h]
\caption{The improper \texttt{bf} in the E2 column currently ``works''
but should not be relied upon.}
\tracingmacros=1
\begin{tabular}{|l|c||r|p{1in}|}
\hline
    AAA1&AAA2&AAA3&\\\cline{2-3}  % 1
    B1&B2&B3&\\[3\jot]\cline{1-1}\cline{3-3}   % 2
    C1&C2&&\\\hline     % 3
      &\multicolumn{2}{r|}{D2}&\\
    \textit{E1}&E\bf E2 &E3 & some flowing text that takes up more than one line so
    we can see how parboxes work.\\\hline
\end{tabular}
\end{table}

\begin{table}[h]
\caption{Note that the X inserted via the @-specifier appears at the
  far left of the column in the \TeX\ output, but shoved over the
  right in the HTML.  There's probably no good way to fix that, so
  it's another example of something not to do.}
\begin{tabular}{|l@{X}|@{\extracolsep{10pt}}c||r|p{1in}|}
\hline
    AAA1&AAA2&AAA3&\\\cline{2-3}  % 1
    B1&B2&B3&\\[3\jot]\cline{1-1}\cline{3-3}   % 2
    C1&C2&&\\\hline     % 3
      &\multicolumn{2}{r|}{D2}&\\
    \textit{E1}&E\bf E2 &E3 & some flowing text that takes up more than one line so
    we can see how parboxes work.\\\hline
\end{tabular}
\end{table}

\begin{table}[h]
\caption{}
\begin{tabular}{l@{X}@{\extracolsep{10pt}}crp{1in}}
\hline
    AAA1&AAA2&AAA3&\\   % 1
    B1&B2&B3&\\[3\jot]  % 2
    C1&C2&&\\           % 3
      &\multicolumn{2}{r}{D2}&\\
    \textit{E1}&E\bf E2 &E3 & some flowing text that takes up more than one line so
    we can see how parboxes work.
\end{tabular}
\end{table}

\end{document}

\def\wta{\widetilde a}

\begin{tabular}{ |c|c|c|c|c|}
\hline
$n$ & $\wta_n^{(k)}$ &  $A_1(n,k)/\wta_n^{(k)}$ & $A_2(n,k)/\wta_n^{(k)}$ & sum \\
\hline
80 & $3.517 \,\times\,10^{19}$ & 1.00000000  & 0.00000000 & 1.00000000 \\
85 & $1.909 \,\times\,10^{14}$ & 0.99999944  & 0.00000056 & 1.00000000 \\
90 & $4.732 \,\times\,10^8$    & 0.91404303  & 0.08595697 & 1.00000000 \\
91 & $6.347 \,\times\,10^7$    & 0.45791582  & 0.54208418 & 1.00000000 \\
92 & $3.119 \,\times\,10^7$    & 0.06059598 &  0.93940402 & 1.00000000 \\
93 & $2.524 \,\times\,10^7$    & 0.00471615  & 0.99528385 & 1.00000000 \\
94 & $2.167 \,\times\,10^7$    & 0.00033500  & 0.99966500 & 1.00000000 \\
95 & $1.879 \,\times\,10^7$    & 0.00002283 &  0.99997717 & 1.00000000 \\
100 & $9.945 \,\times\,10^6$   & 0.00000000  & 1.00000000 & 1.00000000 \\
\hline
\end{tabular}

\end{document}
