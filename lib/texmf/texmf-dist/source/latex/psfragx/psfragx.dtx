% \iffalse
%<*gobble>
% %%%%%%%%%%%%%%%%%%%%%%%%%%%%%%%%%%%%%%%%%%%%%%%%%%%%%%%%%%%%%%%%%
% %%%%%%%%%%%%%%%%%%%%%%%%%%%%%%%%%%%%%%%%%%%%%%%%%%%%%%%%%%%%%%%%%
%
%
%
% Run LaTeX on this file and read the generated file readmePFX.txt.
%
%
%
% Copyright 2003-2012, Pascal Kockaert
% This file is distributed under the LPPL license.
% (see http://www.latex-project.org/lppl.txt for details)
%
%
% You can check the integrity of this file by looking at the
% character table below.
% %%%%%%%%%%%%%%%%%%%%%%%%%%%%%%%%%%%%%%%%%%%%%%%%%%%%%%%%%%%%%%%%%
% %%%%%%%%%%%%%%%%%%%%%%%%%%%%%%%%%%%%%%%%%%%%%%%%%%%%%%%%%%%%%%%%%
% \CharacterTable
%  {Upper-case    \A\B\C\D\E\F\G\H\I\J\K\L\M\N\O\P\Q\R\S\T\U\V\W\X\Y\Z
%   Lower-case    \a\b\c\d\e\f\g\h\i\j\k\l\m\n\o\p\q\r\s\t\u\v\w\x\y\z
%   Digits        \0\1\2\3\4\5\6\7\8\9
%   Exclamation   \!     Double quote  \"     Hash (number) \#
%   Dollar        \$     Percent       \%     Ampersand     \&
%   Acute accent  \'     Left paren    \(     Right paren   \)
%   Asterisk      \*     Plus          \+     Comma         \,
%   Minus         \-     Point         \.     Solidus       \/
%   Colon         \:     Semicolon     \;     Less than     \<
%   Equals        \=     Greater than  \>     Question mark \?
%   Commercial at \@     Left bracket  \[     Backslash     \\
%   Right bracket \]     Circumflex    \^     Underscore    \_
%   Grave accent  \`     Left brace    \{     Vertical bar  \|
%   Right brace   \}     Tilde         \~}
%</gobble>
%<*license>
% This package IS NOT psfrag.sty. It is an eXtension to psfrag.sty.
%
% This file may be distributed and/or modified under the
% conditions of the LaTeX Project Public License, either version 1.2
% of this license or (at your option) any later version.
% The latest version of this license is in
%    http://www.latex-project.org/lppl.txt
% and version 1.2 or later is part of all distributions of LaTeX
% version 1999/12/01 or later.
%</license>
%<*readme>
\begin{filecontents*}{readmePFX.txt}
Readme file associated to the psfragx package.

This file describes how to install the package.

All you need is the file psfragx.dtx. If the file psfragx.ins does
not exist, run LaTeX on the file psfragx.dtx. This should
generate the file psfragx.ins.

Run LaTeX on the file psfragx.ins. This generates all the files of
the package. The documentation will be generated if you run LaTeX
on psfragx.drv.

In order to create the index, you should run makeindex like this.
> makeindex -s gind psfragx.idx
> makeindex -s gglo -o psfragx.gls psfragx.glo

Do not forget to re-run LaTeX on psfragx.drv two times.

You can now read the file psfragx.dvi or transform it into
psfragx.ps or psfragx.pdf.

If you want to modify the default behaviour of psfragx, do not
edit the code, but create a file psfragx.cfg to this end. This
file will be input after the package psfragx is loaded.

Finally, if your TeX implementation uses a database to find files,
you should rebuild this database. fpTeX and teTeX systems use
> mktexlsr

Happy TeXing.

Pascal Kockaert

\end{filecontents*}
%</readme>
%<*gobble>
% Next lines create file psfragx.ins.
% Run it through LaTeX in order to install the package.
\begin{filecontents*}{psfragx.ins}
%</gobble>
%<*ins>
% This file will generate fast loadable files and documentation
% driver files from the doc files in this package when run through
% LaTeX or TeX.
%
% Copyright (C) 2003-2012 Pascal Kockaert
% Pascal.Kockaert ad ulb.ac.be
% Note that ''ad�� is often replaced by the ligature: ''@��.
% ----------------------------------------------------------
%
% This package may be distributed and/or modified under the
% conditions of the LaTeX Project Public License, either version 1.2
% of this license or (at your option) any later version.
% The latest version of this license is in
%    http://www.latex-project.org/lppl.txt
% and version 1.2 or later is part of all distributions of LaTeX
% version 1999/12/01 or later.
%
% --------------- start of docstrip commands ------------------
%
\input docstrip

\keepsilent\askonceonly

\usedir{tex/latex/psfragx}

\preamble

This is a generated file.


Copyright (C) 2003-2012 Pascal Kockaert
Pascal.Kockaert ad ulb.ac.be
Note that ''ad�� is often replaced by the ''@�� sign.
----------------------------------------------------------

This package may be distributed and/or modified under the
conditions of the LaTeX Project Public License, either version 1.2
of this license or (at your option) any later version.
The latest version of this license is in
   http://www.latex-project.org/lppl.txt
and version 1.2 or later is part of all distributions of LaTeX
version 1999/12/01 or later.

\endpreamble

\generate{\file{psfragx.drv}{\from{psfragx.dtx}{driver}}}
\generate{\file{psfragx.sty}{\from{psfragx.dtx}{package,version}}}
\generate{\file{psfragx.ins}{\from{psfragx.dtx}{ins}}}
\generate{\file{psfragx.cfg}{\from{psfragx.dtx}{cfg,version}}}
\generate{\nopreamble\nopostamble%
          \file{psfragx.m}  {\from{psfragx.dtx}{matlab}}}
\generate{\nopreamble\nopostamble%
          \file{pfxprint.m}  {\from{psfragx.dtx}{pfxprint}}}
%\generate{\file{Example.eps}{\from{psfragx.dtx}{example}}}

\Msg{***********************************************************}
\Msg{* To complete the installation of psfragx, you should}
\Msg{* move psfragx.sty to a directory seen by TeX.}
\Msg{*}
\Msg{* The file psfragx.cfg should be adapted to your needs}
\Msg{* and put into a personal directory (HOME rather than LOCAL)}
\Msg{* You can adapt the format of the documentation file by}
\Msg{* editing the file psfragx.drv, before compiling it.}
\Msg{*}
\Msg{***************************************************************}
\Msg{* This message is transcripted at the end of  psfragx.log}
\Msg{***************************************************************}

\endbatchfile
%</ins>
%<*gobble>
\end{filecontents*}
%</gobble>
%<*gobble>
              \ProvidesFile{psfragx.dtx}
%</gobble>
%<package>\NeedsTeXFormat{LaTeX2e}[2001/06/01]
%<package>\ProvidesPackage{psfragx}
%<driver>\ProvidesFile{psfragx.drv}
%<cfg>\ProvidesFile{psfragx.cfg}
%<*version>
% %\fi
%             \ProvidesFile{psfragx.dtx}
                           [2012/05/02 v1.1 psfragx (Pascal Kockaert)]
% % \iffalse
%</version>
%<*gobble>
\csname endinput\endcsname\csname @@end\endcsname\csname end\endcsname
%</gobble>
%<*driver>
\documentclass[a4paper,11pt]{ltxdoc}
\usepackage[ansinew]{inputenc}
\usepackage[T1]{fontenc}
\usepackage[francais,english]{babel}
%% Uncomment the following line if you don't want to include a
%% source-code listing.
%%\OnlyDescription
\CodelineIndex
%%
%%% Babel french option
\NoAutoSpaceBeforeFDP
%%%
\begin{document}
       \DocInput{psfragx.dtx}
\end{document}
%</driver>
% %\fi
%
% \GetFileInfo{psfragx.dtx}
%
% \title{PSfragX: one graphic in one file%
%         \thanks{This file has version~\fileversion.
%                 It was processed on ~\filedate.}}
% \author{Pascal Kockaert}
% \date{\filedate}
%
% \CheckSum{646}
% \RecordChanges
% \MakeShortVerb{\"}
%
% %%%%%%%%%%%%%%%%%%%%%%%%%%%%%%%%%%%%%%%%%%%%%%%%%%%%%%%%%%%%%%%%%%%
% \newcommand{\EPS}{\textsc{eps}}
% \newcommand{\PFX}{\textsc{pfx}}
% \newcommand{\OVP}{\textsc{ovp}}
% \maketitle
% \begin{abstract}
% "\usepackage["\textit{options}"]{pfragx}" inputs the packages "psfrag" and
% "graphicx", and adds essentially one \LaTeX{} command, which is
% "\includegraphicx" (with an 'x� at the end).
%
% This command differs from "\includegraphics" in the fact that it
% inputs "\psfrag" replacements contained into the included {\EPS} file
% itself.
%
% If the {\EPS} files contains those replacements at the right place,
% "\usepackage[sub]{psfragx}"
% will substitute "\includegraphicx" to "\includegraphics"
% automatically.
% At the same time, it is possible to include overpic commands into the
% {\EPS} file, and they will be automatically processed.
%
% The {\EPS} file can be written by a matlab script, so that the user
% needs only to call the script in order to print the matlab figure.
% No additional work will be necessary.
% \end{abstract}
%
% \tableofcontents
%
% \section{Warning}
% Some options of this package allow to overwrite some files ending
% in .pfx and in .ovp. Be sure to understand how these options work
% before using them.
%
% The text below assumes that you are used to
% "\includegraphics" from the "graphicx" package, and
% also to "\psfrag" from the "psfrag" package. Reading
% the documentation of "overpic" could also help to
% understand what follows.
%
% \section{Motivation}
%
% Using graphics drawn by mathematical softwares is very convenient
% but does not offer all the flexibility of \TeX{} and \LaTeX{} when
% it comes to write labels.
%
% Some solutions exist, like the matlab "laprint.m" function
% (http://www.uni-kassel.de/fb16/rat/matlab/laprint/) to print from
% matlab(TM)
% into {\EPS} files suited to be easily handled by "psfrag". All the
% labels (including numbers on the axis) are converted into strings
% like `x01� that should be replaced by their values, like `3.141'.
%
% Though the result is pleasant, it is \textsc{mandatory} to keep track
% of the substitutions. This is why the "laprint.m" function takes care
% to write a \TeX{} file that contains all the "\psfrag" commands
% necessary to obtain the original labels. This \TeX{} file can be
% edited to modify the "\psfrag" commands. This scheme works well, but
% has some limitations. You must obviously take care to move the ".tex"
% and the ".eps" file together. But in addition, you must input the
% graphic using an "\input" command. If you intend to modify its size
% or change some "\psfrag" replacements, you need to open and modify
% the original ".tex" file. If you want to use packages like
% "overpic", you must modify the ".tex" file output by "laprint" or
% copy all the "psfrag" replacements that it contains to your main
% \TeX{} file. In case you would choose to copy the psfrag replacements
% into your main \TeX{} file, you will end up with a lot of lines like
% \begin{verbatim}
% \psfrag{x01}[B][B][1][0]{3.141}
% \psfrag{x02}[B][B][1][0]{6.283}
% \psfrag{x03}[B][B][1][0]{9.425}
% \psfrag{x04}[B][B][1][0]{12.566}
% \end{verbatim}
% into your \LaTeX{} document.
%
% In fact, you \textsc{do not need} to see all these lines, and should
% never see them, except, for example, if you want to replace them with
% \begin{verbatim}
% \psfrag{x01}[B][B][1][0]{$\pi$}
% \psfrag{x02}[B][B][1][0]{$\2\,\pi$}
% \end{verbatim}
% and so on.
%
% What is said here about "\psfrag" commands can be transposed to the
% "overpic" environment that allows to put "picture" objects over a
% graphic.
%
% The authors of "psfrag" have designed a mechanism that allows to
% embed "\psfrag" commands into the {\EPS} file itself\footnote
%        {This was kindly reminded to me by Michael C. Grant, one of the
%        authors of "psfrag", that I would like to thank here.%
%        }.
% Though this mechanism can be convenient, it presents some drawbacks
% that are described into the documentation of "psfrag".
%
% The package "psfragX" aims to circumvent these drawbacks, as well as
% to introduce more flexibility into the automatic inclusion mechanism.
% For example, "psfragX" allows to define different "\psfrag"
% replacements for different languages. If "babel" is used,
% replacements will be selected according to the current language of
% the document. It also allows to make use of "color" commands that are
% ignored if the "color" package is not loaded.
%
% \section{How PSfragX works}
%
% \subsection{PSfrag, \textsc{pfx}, overpic and \textsc{ovp}}
%
% The package "psfragx" allows to embed "\psfrag" commands into the
% {\EPS} file, as well as "picture" objects in "overpic" environments.
% In order to simplify the description, we will refer only to
% "\psfrag" inclusions here below.
% The "overpic" inclusion mechanism works in the same way and will
% not be described. The differences between "psfrag" and "overpic"
% inclusions will appear in the syntax of some commands. We use the
% three letters "pfx" to prefix things that relate to "psfrag" and
% "ovp" to prefix things that relate to "overpic".
%
% The psfragx mechanism can be divided into two parts that are
% described separately.
%
% \subsection{Merging and separating the \TeX{} and {\EPS} documents}
%
% We use the result of "laprint.m" as an example, but all the {\EPS} file
% could be processed in the same way. "laprint" outputs
% "Fig1.tex" \&\ "Fig1.eps" files that can be converted into a single
% file that we call Figure1.eps, which is a copy of Fig1.eps, with
% additional comments that contain all the interesting lines of
% Fig1.tex. At this stage, you can throw the original files Fig1.tex
% and Fig1.eps. \textsc{Be careful to make a backup!!!}
%
% The added comments are not read by the PostScript interpreter and
% should not affect the resulting {\EPS} file. As far as I know, the
% comments used conform to the Adobe(TM) Document Structuring
% Convention (ADSC). These comments can be added by hand or using
% the matlab script "psfragx.m" that should be accompanying this file.
% Their structure also conforms to the DocStrip convention (see
% "docstrip.dtx"): the part to be copied starts with a comment "%<*pfx>"
% and ends with "%</pfx>". All the lines between these marks will be
% taken into account by "psfragx".
%
% \begin{verbatim}
% %<*pfx>
% %\psfrag{x01}[B][B][1][0]{3.141}
% %\psfrag{x02}[B][B][1][0]{6.283}
% %\psfrag{x03}[B][B][1][0]{9.425}
% %\psfrag{x04}[B][B][1][0]{12.566}
% %</pfx>
% \end{verbatim}
% PSfragx looks for these lines into the {\EPS} file and outputs them
% into a file with the same name, but a ".pfx" extension. In other
% words, "Figure1.pfx" is created with the comments of "Figure1.eps".
% This file is normally created only once, though there is an option
% to overwrite it. This means that you could edit it by hand without
% loosing your work next time you run LaTeX.
%
% In the same manner, it is possible to include picture commands using
% \begin{verbatim}
% %<*ovp>
% %\put(50,50){Middle of the graphic}
% %</ovp>
% \end{verbatim}
% These lines will go into the file "Figure1.ovp".
%
% The process of seeking for "pfx" and "ovp" environments stops
% as soon as a line starting with "%\endinput" is found. Including
% such a line will speed the things up.
%
% \subsection{Input the right file at the right moment}
%
% Once the "Figure1.pfx" file exists, the command "\includegraphicx"
% includes it and uses the conventional "\includegraphics" from
% "graphicx" with the same arguments. The result is that all the
% "\psfrag" replacements are processed before the
% "Figure1.eps" file is included.
%
% The "\psfrag" commands do not appear into the \TeX{} file.
% The {\PFX} file could be deleted, and all the replacements would still
% be performed, because the {\PFX} file would be re-generated on the fly.
%
% Now that all the labels on the axes are perfectly drawn, we could
% still want to replace the value "3.141" by the tag "$\pi$". This
% is why the "\includegraphicx" command has a second facultative
% argument. Inside this argument, you should issue all the "\psfrag"
% commands that you want to perform after the inclusion of the {\PFX}
% file.
%
% \begin{verbatim}
% \includegraphicx[width=\linewidth]
%                 (\psfrag{x01}[B][B]{$\pi$}%
%                  \psfrag{x02}[B][B]{$2\,\pi$})
%                 {Figure1.eps}
% \end{verbatim}
%
% The second optional argument is defined with "()" to avoid an
% interaction between the brackets of the "\psfrag" command and
% those of the "\includegraphicx" command.
%
% All that is said about {\PFX} files and "psfrag" replacements can be
% transposed to {\OVP} files and "overpic" "picture" commands.
%
% If we want to add "overpic" commands before or after the inclusion
% of the {\OVP} file, we can use the two other optional arguments of
% "\includegraphicx":
% \begin{verbatim}
% \includegraphicx[width=\linewidth]
%                 <\put(50,50){Foreground object}>
%                 [\put(0,0){Background object}]
%                 {Figure1.eps}
% \end{verbatim}
%
%
%
% \section{Usage}
%
% \subsection{Package options}
%
% The package is input by
% "\usepackage["\textit{options}"]{psfragx}"
%
%
% The \textit{options} are
% \begin{description}
% \item[sub, nosub] substitute or do not substitute
%                   "\includegraphicx" to "\includegraphics";
% \item[allcom,selcom] copy all or
% only selected MetaComments from the {\EPS} file to the {\PFX} file
% (if you do not understand what this means, you can safely ignore it);
% \item[ovp, noovp] makes use of overpic to (automatically) put picture
% objects over the graphics.
% \end{description}
%
% \subsection{Two new commands}
%
% \DescribeMacro{\includegraphicx}
% \DescribeEnv{overpix}
% \begin{itemize}
% \item \textsc{The} command of the package is:
%        \begin{tabular}{rl}
%        "\includegraphicx"   &"["\textit{keys}"]"\\
%                             &"("\textit{psfrags}")"\\
%                             &"<"\textit{foreground overpic}">"\\
%                             &"["\textit{background overpic}"]"\\
%                             &"{"\textit{file.eps}"}"\\
%        \end{tabular}
% \item \textsc{The} environment of the package is:
%        \begin{tabular}{rl}
%        "\begin{overpix}"   &"["\textit{keys}"]"\\
%                             &"<"\textit{foreground overpic}">"\\
%                             &"["\textit{background overpic}"]"\\
%                             &"{"\textit{file.eps}"}"\\
%        "\end{overpix}"&\\
%        \end{tabular}
% \end{itemize}
%
% The meaning of the keys is explained in table~\ref{Tab:mean:keys}.
% \begin{table}{h!}
% \centering
% \caption{Meaning of the keys for \texttt{\string\includegraphics}
%          and \texttt{overpix}}
% \label{Tab:mean:keys}
% \footnotesize
% \begin{tabular}{ccl}
% \hline
%  key          &acceptable values          &action\\
% \hline
% \textit{graphicx keys}
%               & \textit{usual values}
%                                           &\textit{usual meanings}\\
%  pfx          &"true"/"false"             &allows/disallows the
%                                            inclusion of the
%                                            {\PFX} file\\
%  overwritepfx &"true"/"false"             &allows/disallows to
%                                            overwrite an existing
%                                            {\PFX} file\\
%  pfxadd       &\textit{psfrags}           &"\psfrag" commands to be
%                                            processed after the
%                                            inclusion of the
%                                            {\PFX} file\\
%  ovp          &"true"/"false"             &allows/disallows the
%                                            inclusion of the
%                                            {\OVP} file\\
%  overwriteovp &"true"/"false"             &allows/disallows to
%                                            overwrite an existing
%                                            {\OVP} file\\
%  ovpbgd       &\textit{picture commands}  &"picture" commands to be
%                                            processed before the
%                                            inclusion of the
%                                            {\OVP} file\\
%  ovpfgd       &\textit{picture commands}  &"picture" commands to be
%                                            processed after the
%                                            inclusion of the
%                                            {\OVP} file\\
% \hline
% \end{tabular}
% \end{table}
%
% The item denoted by \textit{psfrags} should consist only in
% "\psfrag{A}[b][c][d][e]{F}" commands, where A,b,c,d,e,F can be
% anything. In addition, the "\psfrag" commands can be selectively
% included according to the current language of the document
% (at the point of inclusion). Two commands are provided.
% The commands "\iflanguage" is explained in the "babel"
% documentation. If "babel" is not loaded, "\iflanguage" is redefined
% to match the definition of "babel", that is:\\
% "\iflanguage{"\textit{languagename}"}"%
%            "{"\textit{true case}"}"%
%            "{"\textit{false case}"}".
% The configuration file "psfragxcfg" given below, as an example,
% redefines the main commands of the "color" package so that no
% error occurs if the \textit{psfrags} contains color commands and
% the package "color" is not loaded.
%
% The item denoted by \textit{picture commands} should consist only in
% commands that are allowed in the usual "picture" environment of
% \LaTeX{}. You can also make use of "\iflanguage" and color commands,
% provided that the configuration file given below is used.
%
% \textsc{You cannot put blank lines}, that is lines that would 
% consist only in one ``\%'' sign. If you insert such lines, the 
% ``\%'' sign will be removed and some space will be added in front of
% the included figure.
% 
%
% \subsection{Other new commands}
%
%  Though they are not needed in a normal use of "psfragx",
%  the following commands are available:
%  "\allmetacomments",
%  "\selectedmetacomments",
%  "\copypfxfromto{<EPS file>}{<PFX file>}",
%  "\setpfxinput{<File>}",
%  "\setpfxoutput{<File>}",
%  "\copypfxlines",
%  "\pfxinput",
%  "\ovpinput".
%  Their usage can be deduced from the commented source code.
%
% The other commands are internal and start with "\pfx@" or "\ovp@".
%
% \section{Configuration file}
%
% The file "psfragx.cfg" will be input by psfrags, if it exists.
% This file can contain new commands of general use, or commands that
% must appear just before or just after the inclusion of the {\PFX}/{\OVP}
% file occurs. To this aim, four commands can be defined. Their names
% are "\Beforepfxinput", "\Afterpfxinput",
% "\Beforeovpinput", and "\Afterovpinput". They can be used as
% in the example below.
%    \begin{macrocode}
%<*cfg>
% Example of configuration file for psfragx.sty
% The macros \Beforepfxinput, \Afterpfxinput
% \Beforeovpinput, and \Afterovpinput are executed
% into a group. They should not define global commands to
% avoid side effects.
%
%
% The command \providecolorcommands defines commands that
% take the same arguments as the mains commands of the
% color package, in case this package is not loaded.
%
\newcommand{\providecolorcommands}
           {\def\pfx@gobble@two##1##2{\typeout{Some psfragx
                                       replacement would appear
                                       in color ##1{##2}
                                       if the color package was
                                       loaded!!!}}%
            \def\pfx@gobble@three@fbox##1##2##3{\typeout{Some psfragx
                                                replacement would
                                                appear
                                                in color ##1{##2}
                                                and others
                                                in color ##1{##3}
                                                if the color package
                                                was loaded!!!}%
                                                \fbox}%
            \def\pfx@fm@to@mm##1##2##{\csname ##1\endcsname{##2}}%
            \expandafter\ifx\csname textcolor\endcsname\relax
              \def\textcolor{\pfx@fm@to@mm{pfx@gobble@two}}\fi
            \expandafter\ifx\csname color\endcsname\relax
              \def\color{\pfx@fm@to@mm{pfx@gobble@two}}\fi
            \expandafter\ifx\csname colorbox\endcsname\relax
              \def\colorbox{\pfx@fm@to@mm{pfx@gobble@two}}\fi
            \expandafter\ifx\csname fcolorbox\endcsname\relax
              \def\fcolorbox{\pfx@fm@to@mm{pfx@gobble@three@fbox}}\fi
            }
% The name of the next four commands are specific to psfragx
\def\Beforepfxinput{\providecolorcommands}
\def\Afterpfxinput{}
\def\Beforeovpinput{\providecolorcommands}
\def\Afterovpinput{}
%</cfg>
%    \end{macrocode}
% \section{Example of tagged {\EPS} file}
%
% We provide here below an example of {\EPS} that uses
% the language and color features... According to the
% �Adobe Document Structuring Convention� (ADSC), comments
% starting with two percent signs have a special meaning.
% You should therefore avoid to put \textsc{exactly} two
% percents signs at the beginning of a line. If you respect this
% rule and avoid very long lines, you should never broke your
% {\EPS} file.
%
%    \begin{macrocode}
%<*example>
 %%!PS-Adobe-2.0 EPSF-1.2
 %%Creator: Adobe Illustrator(TM) 1.2d4
 %%Title: tiger.eps
 %%CreationDate: 4/12/90 3:20 AM
 %%BoundingBox: 17 171 567 739
 %<*pfx>
 %\psfrag{T}[B][B]{\fcolorbox{white}{black}{\color{white}Title}}
 %\psfrag{t}[t][t]{time (s)}
 %\psfrag{I}[b][b]{I (W)}
 %\iflanguage{french}
 %           {\psfrag{T}[B][B]{Title}%
 %            \psfrag{t}[t][t]{temps (s)}}
 %           {}
 %\psfrag{T}[B][B]{Title}
 %</pfx>
 %<*ovp>
 %\put(0,80){(a)}
 %</ovp>
 %\endinput
 %%EndComments
 %%
 %% [The code of the {\EPS} file should come HERE]
 %%
 %% End
%</example>
%    \end{macrocode}
%
% Full examples should be provided with this package. They are
% not included into "psfrags.dtx".
%
% \section{Associated matlab scripts}
%
% The script "psfragx.m" is written for matlab and can be used in
% conjunction with "laprint.m" (see URL above) in order to benefit from
% the advantages of "laprint.m" and mix the resulting .tex and .eps
% files into a file that contains all the information.
%
% The scripts "pfxprint.m" can be used with the same syntax as 
% "laprint.m" (see documentation of "laprint"). This script invoques 
% "laprint" with the settings contained in the file "laprpfx.mat", and 
% immediately after, it merges the generated \EPS{} and \TeX{} files. 
% Therefore, you should ensure that the files "laprint.m", "laprpfx.mat", and 
% "psfragx.m" are in a directory searched by matlab before using the 
% "pfxprint" command.
%
% At the time of writing, the current version of "laprint" is "3.16". 
% This version works well with "pfxprint.m" and "psfragx.sty".
% 
%
% \section{Credits}
%
% All the code to extract the comments from the {\EPS} file is inspired
% from "docstrip". The set of commands was reduced to its
% minimum, and a "\pfx@" prefix was added to all the commands, in
% order to avoid any interaction with other packages.
%
% %%%%%%%%%%%%%%%%%%%%%%%%%%%%%%%%%%%%%%%%%%%%%%%%%%%%%%%%%%%%%%%%%%%
%
% \StopEventually{\PrintChanges\PrintIndex}
%
% %%%%%%%%%%%%%%%%%%%%%%%%%%%%%%%%%%%%%%%%%%%%%%%%%%%%%%%%%%%%%%%%%%%
%
% \section{Mise en �uvre}
%
% \changes{v0.1}{2003/08/30}{First dtx version}
%
%    \begin{macrocode}
%<*package>
%    \end{macrocode}
%
% Almost all the internal commands start with "\pfx@", "\ovp@",
% "\ifpfx@", or "\ifovp@".
% Two exceptions are "\@..@overpix" and "\@..@igx",
% where "@..@" can be "@", "@@", "@@@" or "@@@@".
%
% \subsection{Required packages and options}
%
% We offer the option to substitute the new "\includegraphicx" command
% to the usual "\includegraphics", and optionally, the overpix
% environment to the usual overpic one. This could broke things but
% allows to use "psfragx" with existing documents almost transparently.
% \begin{macro}{\pfx@subfalse}
% \begin{macro}{\pfx@subtrue}
%    \begin{macrocode}
\DeclareOption{sub}{\pfx@subtrue}
\DeclareOption{nosub}{\pfx@subfalse}
%    \end{macrocode}
% \end{macro}
% \end{macro}
% The next option was of some help to debug this package.
% With "allcom", all the lines of the {\EPS} file starting with "%%" are
% copied to the {\PFX} and {\OVP} files. Otherwise, these lines are not
% copied if they are out of a tagged environment.
%    \begin{macrocode}
\DeclareOption{allcom}{\allmetacomments}
\DeclareOption{selcom}{\selectedmetacomments}
%    \end{macrocode}
% The next option specifies that the overpic environment will be used.
% Therefore, the overpic package should be loaded.
%
% \begin{macro}{\pfx@ovptrue}
% \begin{macro}{\pfx@ovpfalse}
%    \begin{macrocode}
\DeclareOption{ovp}{\pfx@ovptrue}
\DeclareOption{noovp}{\pfx@ovpfalse}
%    \end{macrocode}
% \end{macro}
% \end{macro}
%
% We define the new commands needed to process the options.
% \begin{macro}{\allmetacomments}
% \begin{macro}{\selectedmetacomments}
%    \begin{macrocode}
       \newif\ifpfx@sub\pfx@subfalse
       \newif\ifpfx@ovp\pfx@ovptrue
       \newif\ifpfx@metacomments
       \pfx@metacommentsfalse
       \def\allmetacomments{\pfx@metacommentstrue}
       \def\selectedmetacomments{\pfx@metacommentsfalse}
%    \end{macrocode}
% \end{macro}
% \end{macro}
%
% Finally, default options are defined.
%    \begin{macrocode}
\ExecuteOptions{sub,ovp,selcom}
\ProcessOptions*
%    \end{macrocode}
%
% Now, we load the other packages.
%    \begin{macrocode}
\RequirePackage{graphicx}
\RequirePackage{psfrag}
%    \end{macrocode}
%
% The overpic package is not loaded if this was required by the user.
% Otherwise, we load this package. To ensure proper placement of the
% objects put into the picture environment, we must always use the
% same option when loading "overpic". We choose this option to be "percent".
%
%    \begin{macrocode}
\ifpfx@ovp \RequirePackage[percent]{overpic} \fi
%    \end{macrocode}
%
% \subsection{Reading the {\EPS} file and writing {\PFX} or {\OVP} files}
% The code that follows is highly inspired from that of docstrip.tex.
%
% \subsubsection{Copying selected lines from the {\EPS} file}
% Below, we write the code to copy specific lines contained in the
% {\EPS} file into an auxiliary file. Comments (single "%" sign)
% in front of these lines are automatically removed.
%
% First, we define a few macros of general use.
%
% \begin{macro}{\pfx@gobble}
% \begin{macro}{\pfx@percent}
% \begin{macro}{\pfx@doublepercent}
%    \begin{macrocode}
\def\pfx@gobble#1{}
{\catcode`\%=12
 \gdef\pfx@percent{%}
 \gdef\pfx@doublepercent
}
%    \end{macrocode}
% \end{macro}
% \end{macro}
% \end{macro}
%
% Here we define the extension of the auxiliary file, and the name of
% the tag associated to this file. The "metaprefix" replaces double
% percent signs found into the original {\EPS} file.
% \begin{macro}{\pfx@ext}
% \begin{macro}{\pfx@tag}
% \begin{macro}{\pfx@metaprefix}
%    \begin{macrocode}
\let\pfx@metaprefix\pfx@doublepercent
\def\pfx@tag{pfx}
\def\pfx@ext{pfx}
\def\pfx@tmp{}
%    \end{macrocode}
% \end{macro}
% \end{macro}
% \end{macro}
%
% We also need to define the string after which we will stop to scan
% the {\EPS} file. This string must appear at the beginning of a line.
% If this string is not present into the {\EPS} file, the file will be
% scanned up to the end. This string is defined to be "%\endinput".
% \begin{macro}{\pfx@endinput}
%    \begin{macrocode}
\edef\pfx@endinput
     {\pfx@percent\expandafter\pfx@gobble\string\\endinput}
%    \end{macrocode}
% \end{macro}
%
% And now, we copy the needed code from "docstrip", with some
% modifications to throw the leading percent sign when we copy
% the lines that appear between two tags "<*pfx>...</pfx>".
%
% We define a command to change catcodes,
%    \begin{macrocode}
\def\pfx@makeother#1{\catcode`#1=12\relax}
%    \end{macrocode}
% another to copy a given token,
%    \begin{macrocode}
\def\pfx@iden#1{#1}
%    \end{macrocode}
% and a few boolean variables.
%    \begin{macrocode}
\newif\ifpfx@continue
\newif\ifpfx@outputtofile
%    \end{macrocode}
%
% The names of the input and output files are contained into the
% internal variables "\pfx@infile" and "\pfx@outfile". These named
% can be accessed from the document through the two commands
% "\setpfxinput" and "\setpfxoutput".
% \begin{macro}{\setpfxinput}
% \begin{macro}{\setpfxoutput}
%    \begin{macrocode}
\def\setpfxinput#1{\gdef\pfx@infile{#1}}
\def\setpfxoutput#1{\gdef\pfx@outfile{#1}}
\gdef\pfx@infile{} \gdef\pfx@outfile{}
%    \end{macrocode}
% \end{macro}
% \end{macro}
%
% Two streams are reserved by "psfragx". I do not know if I should
% use them locally rather than globally.
%    \begin{macrocode}
\newread\pfx@in
\newwrite\pfx@out
%    \end{macrocode}
%
% The macro "\copypfsfromto" \textsc{does not check}
% that the input file exists.
% \begin{macro}{\copypfxfromto}
%    \begin{macrocode}
\def\copypfxfromto#1#2{%
\setpfxinput{#1}%
\setpfxoutput{#2}%
\copypfxlines%
}
%    \end{macrocode}
% \end{macro}
% The macro "\copypfxlines" does the real job.
% See "docstrip" to understand how it works.
% \begin{macro}{\copypfxlines}
%    \begin{macrocode}
\def\pfx@ignorespaces{\ignorespaces}%
\def\copypfxlines{% input and output files are global names
\immediate\openin\pfx@in\pfx@infile\relax \ifeof\pfx@in
       \errmessage{psfragx tried to read from a file that
                   does not exist. This seems to be a bug!}%
\else
       \immediate\openout\pfx@out=\pfx@outfile\relax
       \immediate\write\pfx@out{\pfx@ignorespaces}
       \ifeof\pfx@out
             \begingroup
             \pfx@makeother\ \pfx@makeother\\\pfx@makeother\$%
             \pfx@makeother\#\pfx@makeother\^\pfx@makeother\^^K%
             \pfx@makeother\_\pfx@makeother\^^A\pfx@makeother\%%
             \pfx@makeother\~\pfx@makeother\{\pfx@makeother\}%
             \pfx@makeother\&\endlinechar-1\relax
             \loop
                     \read\pfx@in to \pfx@inline
                     \ifx\pfx@inline\pfx@endinput
                            \pfx@continuefalse
                            \typeout{psfragx: \pfx@percent
                                     \expandafter\pfx@gobble
                                     \string\\endinput was
                                     found in \pfx@infile.}%
                     \else
                            \ifeof\pfx@in
                                   \pfx@continuefalse
                            \typeout{psfragx: End of file
                                     \pfx@infile was reached.}%
                            \else
                                   \pfx@continuetrue
                                   \expandafter\pfx@processline
                                   \pfx@inline\pfx@endline
                            \fi%
                     \fi%
             \ifpfx@continue
             \repeat
             \endgroup
       \else
              \errmessage{psfragx: output file already exists!}%
       \fi %\pfx@out
       \immediate\closeout\pfx@out
\fi %\pfx@in
\immediate\closein\pfx@in
}
%    \end{macrocode}
% At this stage, all "<pfx>" and "<*pfx>...</pfx>" lines from
% "\pfx@infile" should be in "\pfx@outfile".
% \end{macro}
%
% Each time a new line is found by the previous macro, the line is
% processed using "\pfx@processline". This macro scans the beginning
% of the line and defers the treatment to the right macro.
% In the "docstrip" code, normal lines are copied without change.
% In our code, the leading percent of copied lines is removed.
%
% \begin{macro}{\pfx@processline}
%    \begin{macrocode}
\def\pfx@normalline#1\pfx@endline{%
       \def\pfx@inline{#1}%
       \ifpfx@outputtofile%
              \immediate\write\pfx@out{\pfx@inline}%
       \fi%
}
%
\def\pfx@removecomment#1\pfx@endline{%
       \def\pfx@inline{#1}%
       \ifpfx@outputtofile%
              \immediate\write\pfx@out{\pfx@inline}%
       \fi%
}
%
\bgroup\catcode`\%=12 \pfx@iden{\egroup
\def\pfx@putmetacomment%}#1\pfx@endline{%
       \edef\pfx@inline{\pfx@metaprefix#1}%
       \ifpfx@metacomments
              \immediate\write\pfx@out{\pfx@inline}%
       \else
              \ifpfx@outputtofile
                     \immediate\write\pfx@out{\pfx@inline}%
              \fi
      \fi
}
%
\begingroup
\catcode`\%=12 \catcode`\*=14 \gdef\pfx@processline#1{*
       \ifx%#1*
              \expandafter\pfx@processlinex
       \else
              \expandafter\pfx@normalline
       \fi
#1}
\endgroup
%
\begingroup
\catcode`\%=12 \catcode`\*=14
\gdef\pfx@processlinex%#1{*
       \ifcase\ifx%#10\else
              \ifx<#11\else2\fi\fi\relax
       \expandafter\pfx@putmetacomment\or
       \expandafter\pfx@checkoption\or
       \expandafter\pfx@removecomment\fi
#1}
\endgroup
%
\def\pfx@checkoption<#1{%
       \ifcase\ifx*#10\else
              \ifx/#11\else2\fi\fi\relax
       \expandafter\pfx@staroption\or
       \expandafter\pfx@slashoption\or
       \expandafter\pfx@tagoption\fi
#1}
%
\def\pfx@staroption*#1>#2\pfx@endline{%
       \def\pfx@tmp{#1}%
       \ifx\pfx@tmp\pfx@tag
              \pfx@outputtofiletrue
       \fi
}
%
\def\pfx@slashoption/#1>#2\pfx@endline{%
       \def\pfx@tmp{#1}%
       \ifx\pfx@tmp\pfx@tag\relax
              \pfx@outputtofilefalse
       \fi
}
%
\def\pfx@tagoption#1>#2\pfx@endline{%
       \def\pfx@tmp{#1}%
       \ifx\pfx@tmp\pfx@tag\relax
              \def\pfx@inline{#2}%
              \immediate\write\pfx@out{\pfx@inline}%
       \fi
}
%    \end{macrocode}
% \end{macro}
% This ends the code to read {\EPS} file and write {\PFX} file. It is
% clear that only "\pfx@tag" and "\pfx@ext" should be changed from
% "pfx"to "ovp" in order to process "overpic" inclusions rather than
% "psfrag" replacements.
%
% \subsection{Code that inputs the {\PFX} and {\OVP} files}
%
% This code will add commands to input the {\PFX} and {\OVP} files if
% they exist. If they do not, they will be created on the fly and read
% just after. An option allows to ignore existing files and generate
% {\PFX} and {\OVP} files from the {\EPS} file each time the {\EPS}
% file is included.
%
% At first, we define commands related to {\PFX} files.
% Later on, we will adapt them to {\OVP} files.
%
% \subsubsection{Saving and providing commands of other packages}
%
% We save the commands that could be redefined later
%    \begin{macrocode}
\let\pfx@includegraphics=\includegraphics
\let\pfx@overpic=\overpic
\let\pfx@endoverpic=\endoverpic
%    \end{macrocode}
%
% Even if "overpic" is not loaded, the "overpic" environment should
% exist. In this case, the {\OVP} files will not be processed, and
% no "picture" element should be put over the graphics.
% Nonetheless, "\includegraphicx"
% is defined to always use the "overpic" environment.
%
% Therefore, we provide a definition of the "overpic" environment
% that is partially copied from overpic.sty. We have removed all the
% code that makes computations about the size and the position of the
% grid.
% \begin{macro}{\pfx@overpic}
% \begin{macro}{\pfx@endoverpic}
%    \begin{macrocode}
\@ifundefined{pfx@overpic}{%
                       \newcommand*{\pfx@overpic}[2][]
                       {\sbox{\z@}{\includegraphics[#1]{#2}}%
                       \settodepth{\@tempcnta}{\usebox{\z@}}%
                       \settoheight{\@tempcntb}{\usebox{\z@}}%
                       \advance\@tempcntb\@tempcnta%
                       \settowidth{\@tempcnta}{\usebox{\z@}}%
                       \begin{picture}(\@tempcnta,\@tempcntb)%
                       \put(0,0){\makebox(0,0)[bl]{\usebox{\z@}}}}%
                      }{}
\@ifundefined{pfx@endoverpic}{\def\pfx@endoverpic{\end{input}}}{}
%    \end{macrocode}
% \end{macro}
% \end{macro}
%
% \DescribeMacro{\iflanguage}
% We also have to provide "\iflanguage" command, in case "babel"
% is not loaded. We could have simplified the code, because
% "\pfx@iflanguage" should always expand to "\@secondoftwo"
% if babel is not loaded. Because this code was also copied from
% the babel package with some changes, we use it even if "babel"
% is loaded. This could cause problems if the internal command
% "l@"\textit{language} of "babel" was redefined. Though we redefine
% a "babel" command, this should cause no major problem, because
% the command "\iflanguage" will be provided only at time of the
% file inclusion. This means that "\psfrag" replacements should
% contain no reference to "\iflanguage" as these commands will
% be evaluated  after the file is read. The command "\iflanguage"
% should be evaluated at time of inclusion, in order to decide
% which "psfrag" or "picture" commands are to be taken into account.
%
% \DescribeMacro{\onlylanguage}
% \DescribeMacro{\endonlylanguage}
% In order to simplify the writing of multilingual {\EPS} files, we
% also provide the command "\onlylanguage" "{"\textit{language}"}..."
% "\endonlylanguage", which argument is read only if the current
% language of the document is \textit{language}.
%
% In order to simplify the writing of multilingual {\EPS} files, we
% also provide the command "\onlylanguage" "{"\textit{language}"}..."
% "\endonlylanguage", which argument is read only if the current
% language of the document is \textit{language}.
% \begin{macro}{\pfx@iflanguage}
% \begin{macro}{\iflanguage}
% \begin{macro}{\onlylanguage}
% \begin{macro}{\endonlylanguage}
% \begin{macro}{\pfx@save@iflanguage}
% \begin{macro}{\pfx@restore@iflanguage}
% \begin{macro}{\pfx@firstoftwo}
% \begin{macro}{\pfx@secondoftwo}
%    \begin{macrocode}
\long\def\pfx@firstoftwo#1#2{#1\ignorespaces}%
\long\def\pfx@secondoftwo#1#2{#2\ignorespaces}%
\def\pfx@iflanguage#1{%
    \ifnum\csname l@#1\endcsname=\language
      \expandafter\pfx@firstoftwo
    \else
      \expandafter\pfx@secondoftwo
    \fi}
\long\def\onlylanguage#1#2\endonlylanguage{\pfx@iflanguage{#1}{#2}{}\ignorespaces}
\def\pfx@save@iflanguage{\let\save@pfx@iflanguage=\iflanguage%
                         \let\iflanguage=\pfx@iflanguage}
\def\pfx@restore@iflanguage{\let\iflanguage=\save@pfx@iflanguage}
%    \end{macrocode}
% \end{macro}
% \end{macro}
% \end{macro}
% \end{macro}
% \end{macro}
% \end{macro}
% \end{macro}
% \end{macro}
% The two commands "\pfx@save@iflanguage" and
% "\pfx@restore@iflanguage" will be called just before
% and after the inclusion of the {\PFX} file.
%
% \subsubsection{New commands to read and write files}
%
% We start with some declarations (new commands and new if)
% The names of the "ifGin" series are chosen to be easily processed
% through the "keyval" package mechanism. "Gin" is the prefix used
% by the "graphicx" package.
% \begin{macro}{\ifpfx@generate}
% \begin{macro}{\ifovp@generate}
% \begin{macro}{\pfx@add}
% \begin{macro}{\ovp@add@bgd}
% \begin{macro}{\ovp@add@fgd}
% \begin{macro}{\ifGin@pfx}
% \begin{macro}{\ifGin@overwritepfx}
% \begin{macro}{\ifGin@ovp}
% \begin{macro}{\ifGin@overwriteovp}
%    \begin{macrocode}
\newif\ifpfx@generate
\newif\ifovp@generate
\newcommand*\pfx@add{}
\newcommand*\ovp@add@bgd{}
\newcommand*\ovp@add@fgd{}
\newif\ifGin@pfx
\newif\ifGin@overwritepfx
\newif\ifGin@ovp
\newif\ifGin@overwriteovp
%    \end{macrocode}
% \end{macro}
% \end{macro}
% \end{macro}
% \end{macro}
% \end{macro}
% \end{macro}
% \end{macro}
% \end{macro}
% \end{macro}
% As the names indicate, these macros are attached to {\PFX} or
% {\OVP} inclusions. They allow to save information to know if a
% {\PFX}/{\OVP} file is to be generated, if the {\PFX}/{\OVP}
% automatic inclusion mechanism is to be used and if existing
% {\PFX}/{\OVP} files should be overwritten. Finally, three commands
% will contain the "\psfrag" commands ("\pfx@add") to be issued
% after the inclusion of the {\PFX} file, as well as "picture"
% commands to be issued before ("\ovp@add@bgd") and after
% ("\ovp@add@fgd") the {\OVP} file inclusion.
%
% The new keys will be available through the optional arguments of
% "\includegraphicx".
% This is why they are defined as belonging to the same group as the
% "graphicx" keys: "Gin".
%
% The role that we have described for the previous commands is
% assigned here below. As is common, the boolean keys are set
% to be true if they are invoked without argument.
%    \begin{macrocode}
\define@key{Gin}{pfx}[true]%
           {\lowercase{\Gin@boolkey{#1}}{pfx}}
\define@key{Gin}{overwritepfx}[true]%
           {\lowercase{\Gin@boolkey{#1}}{overwritepfx}}
\define@key{Gin}{pfxadd}[]%
           {\def\pfx@add{#1}}
\define@key{Gin}{ovp}[true]%
           {\lowercase{\Gin@boolkey{#1}}{ovp}}
\define@key{Gin}{overwriteovp}[true]%
           {\lowercase{\Gin@boolkey{#1}}{overwriteovp}}
\define@key{Gin}{ovpbgd}[]%
           {\def\ovp@add@bgd{#1}}
\define@key{Gin}{ovpfgd}[]%
           {\def\ovp@add@fgd{#1}}
%    \end{macrocode}
% We will define a handy syntax for the "\includegraphicx" command.
% This command will mainly convert some of its optional arguments
% to keys "pfxadd={"\textit{argument}"}",
%         "ovpbgd={"\textit{argument}"}",
%     and "ovpfgd={"\textit{argument}"}".
%
% It is now time to define the commands that will test for the
% existence of the input and output files and decide if an output
% file is to be generated. This command makes use of values
% defined previously for {\PFX} files. This is why we prefix the
% command with "\pfx".
% To understand the code below, it is important to know that the
% command "\filename@parse{}" defines three commands that are
% "\filename@area", "\filename@base" and "\filename@ext".
% \begin{macro}{\pfxinput}
%    \begin{macrocode}
\newcommand*{\pfxinput}[1]{%
  \filename@parse{#1}%
  \IfFileExists{\filename@base.\pfx@ext}
               {\pfx@generatefalse}
               {\pfx@generatetrue}%
  \ifGin@overwritepfx\pfx@generatetrue\fi
  \IfFileExists{#1}{}{\pfx@generatefalse}%
  \ifpfx@generate%
       \copypfxfromto{\filename@area\filename@base.\filename@ext}
                     {\filename@base.\pfx@ext}%
  \fi%
  \pfx@save@iflanguage
  \csname Before\pfx@tag input\endcsname
  \InputIfFileExists{\filename@base.\pfx@ext}
                    {\typeout{psfragx: reading commands from
                                       \filename@base.\pfx@ext}}
                    {\typeout{psfragx: I was not able to read psfrag
                              definitions from
                              \filename@base.\pfx@ext}}%
  \csname After\pfx@tag input\endcsname
  \pfx@restore@iflanguage
}
%    \end{macrocode}
% \end{macro}
% As was announced, we determine if the output file exists.
% In case this file exists we decide not to generate the output file.
% If the user required that the output file be overwritten, we
% ask to generate the output file in any case.
% Then we test if the input file exists. If not, we cannot generate
% the ouptut file.
%
% Now that the existence of the input file has been checked, we can
% call the low level command "\copypfxfromto". This completes the
% first step.
%
% The second step is to input the {\PFX} file if it exists. The file
% is input inside a "\pfx@save@iflanguage" "\pfx@restore@iflanguage"
% pair. In addition, the commands "\Beforepfxinput" and
% "\Afterpfxinput" are issued if they exist. Otherwise, they expand
% to "\relax". These commands should be defined into the "psfragx.cfg"
% file in order to customise the behaviour of "psfragx".
%
% Now, we define variations of "\pfx@" commands in order to work with
% the "overpic" environment rather than with "psfrag" replacements.
%
% We redefine the tag and the extension, copy the overwrite
% permission and call "\pfxinput".
% \begin{macro}{\ovpinput}
%    \begin{macrocode}
\newcommand*{\ovpinput}[1]{%
\begingroup
       \def\pfx@ext{ovp}%
       \def\pfx@tag{ovp}%
       \ifGin@overwriteovp\Gin@overwritepfxtrue
                     \else\Gin@overwritepfxfalse\fi
       \pfxinput{#1}%
\endgroup}
%    \end{macrocode}
% \end{macro}
%
% \subsection{The main command of this package}
%
%  Here comes the definition of the main command of this package,
% as seen by the user : "\includegraphicx".
%  This new command will make use of the new environment called
% "overpix", in order to include the graphics.
%
% \subsubsection{Internal commands}
%
% First, we define two internal commands that perform the
% required task. Then we define external commands with optional
% arguments.
% \begin{macro}{\pfx@includegraphicx}
% The macro "\pfx@includegraphics" is just a shortcut to acces
% the "overpix" environment. We do not call
% "\begin{pfx@overpix}..." "\end{pfx@overpix}", in order to save time.
%    \begin{macrocode}
\def\pfx@includegraphicx#1#2{%
       \mbox{\pfx@overpix{#1}{#2}\endpfx@overpix}}
%    \end{macrocode}
% \end{macro}
%
% \begin{macro}{\ovp@box@tmp}
% In what follows, we need a temporary box. This is called "\ovp@box@tmp".
%    \begin{macrocode}
\newbox{\ovp@box@tmp}%
%    \end{macrocode}
% \end{macro}
%
% \begin{macro}{\pfx@overpix}
% \begin{macro}{\endpfx@overpix}
% The environment "overpix" has the same syntax as the "overpic" one.
% You can notice that the original version of "\includegraphics" is
% used. This is important if we decide, later, to let
% "\includegraphics" be equivalent to "\includegraphicx".
% The "\psfrag" commands and the "picture" commands are processed
% inside this environment. To avoid side effects of command
% redefinitions inside the included files, we enclose the contents of
% "overpix" inside a "\begingroup" "\endgroup" pair.
%
% The graphic is included via the original or the lightened version
% of the "overpic" environment.
%
% All we do is to evaluate the keys of the first argument, then
% input the {\PFX} file, and process the contents of "\pfx@add".
% Thereafter, we call the original or lightened version of the
% "overpic" environment. We read the keys again and add the
% background layer of the picture environment, then the layer
% contained into the {\OVP} file.
%
% When the "\pfx@overpix" command is issued, we end up into a
% "picture" environment that constitutes yet another layer.
% Finally, the "picture" environment should be closed by an
% "\endpfx@overpix" command. Before doing so, the foreground
% layer of the picture environment is drawn.
%
% It is mandatory to take care that not spurious space is added
% at the end of the lines. A percent sign should appear each time
% the line is ending with something else than a command name.
%
%    \begin{macrocode}
\def\pfx@overpix#1#2{%
       \begingroup%
       \begin{lrbox}{\ovp@box@tmp}%
       \let\includegraphics=\pfx@includegraphics%
       \Gin@pfxtrue%
       \Gin@overwritepfxfalse%
       \def\pfx@add{}%
       \setkeys{Gin}{#1}%
       \ifGin@pfx%
          \pfxinput{#2}%
       \fi%
       \pfx@add
       \pfx@overpic[#1]{#2}
              \Gin@ovptrue
              \Gin@overwriteovpfalse
              \def\ovp@add@bgd{}%
              \def\ovp@add@fgd{}%
              \setkeys{Gin}{#1}%
              \ovp@add@bgd
              \ifGin@ovp
                     \ovpinput{#2}%
              \fi
       }% \pfx@overpix
%
\def\endpfx@overpix{%
              \ovp@add@fgd%
       \pfx@endoverpic
       \end{lrbox}%
       \usebox{\ovp@box@tmp}%
       \endgroup%
       }%
%    \end{macrocode}
% \end{macro}
% \end{macro}
%
%
% \subsubsection{External commands}
%
% The definitions here below ensure that the optional arguments
% are optional.
% \begin{macro}{\overpix}
% \begin{macro}{\endoverpix}
% The syntax of "overpix" is as follows.\\
% \begin{tabular}{rl}
% "\overpix"  &"["\textit{keys}"]"\\
%             &"<"\textit{foreground layer}">"\\
%             &"["\textit{background layer}"]"\\
%             &"{"\textit{file.eps}"}"
% \end{tabular}
%    \begin{macrocode}
\def\overpix{\@ifnextchar[{\@overpix}%
                          {\@overpix[]}}%
\def\@overpix[#1]{\@ifnextchar<{\@@overpix[#1]}%
                               {\@@overpix[#1]<>}}%
\def\@@overpix[#1]<#2>%
              {\@ifnextchar[{\@@@overpix[#1]<#2>}%
                            {\@@@overpix[#1]<#2>[]}}%
\def\@@@overpix[#1]<#2>[#3]#4%
               {\pfx@overpix{#1,ovpfgd={#2},ovpbgd={#3}}{#4}}
\def\endoverpix{\endpfx@overpix}
%    \end{macrocode}
% This set of commands converts the optional arguments into keys.
% \end{macro}
% \end{macro}
% \begin{macro}{\includegraphicx}
% The syntax of "\includegraphicx" is as follows.\\
% \begin{tabular}{rl}
% "\includegraphicx" &"["\textit{keys}"]"\\
%                    &"("\textit{psfrag replacements}")"\\
%                    &"<"\textit{foreground layer}">"\\
%                    &"["\textit{background layer}"]"\\
%                    &"{"\textit{file.eps}"}"
% \end{tabular}
%    \begin{macrocode}
\def\includegraphicx{\@ifnextchar[{\@igx}%
                                  {\@igx[]}}%
\def\@igx[#1]{\@ifnextchar({\@@igx[#1]}%
                           {\@@igx[#1]()}}%
\def\@@igx[#1](#2){\@ifnextchar<{\@@@igx[#1](#2)}%
                                {\@@@igx[#1](#2)<>}}%
\def\@@@igx[#1](#2)<#3>{\@ifnextchar[{\@@@@igx[#1](#2)<#3>}%
                                     {\@@@@igx[#1](#2)<#3>[]}}%
\def\@@@@igx[#1](#2)<#3>[#4]#5%
    {\pfx@includegraphicx{#1,pfxadd={#2},ovpfgd={#3},ovpbgd={#4}}{#5}}
%    \end{macrocode}
% This set of commands converts the optional arguments into keys.
% \end{macro}
%
% \subsection{Overloading includegraphics and overpic}
%
% If the user requires so, we let "\includegraphics" and the "overpic"
% environment act as their counterparts ending in "x".
% Though this substitution was tested, it could broke things and
% should be used with care.
%
%
% \begin{macro}{\includegraphics}
% \begin{macro}{\overpic}
% \begin{macro}{\endoverpic}
%    \begin{macrocode}
\ifpfx@sub
       \let\includegraphics=\includegraphicx
       \ifpfx@ovp
              \let\overpic=\overpix
              \let\endoverpic=\endoverpix
       \fi
\fi
%    \end{macrocode}
% \end{macro}
% \end{macro}
% \end{macro}
%
% \subsection{Configuration file}
%
% Finally, we input the configuration file if it exists.
%    \begin{macrocode}
\InputIfFileExists{psfragx.cfg}{}{}
%    \end{macrocode}
% This ends the code of "psfragx.sty".
%    \begin{macrocode}
%</package>
%    \end{macrocode}
%
% \section{Code of the matlab script}
%
% Note that the lines containing the "rm" and "mv" commands should be
% replaced by their equivalents on the operating system 
% on wich matlab is running. For example, under \textsc{dos} and its 
% successors the replacements are "del" and "ren".
% 
%
%    \begin{macrocode}
%<*matlab>
%% psfragx.m [2012/05/02 v0.4 Pascal Kockaert]
%%
%% function psfragx(TeXname,EPSname,Outname)
%% nargin=1 -> EPSname=TeXname
%% nargin=2 -> Outname=EPSname
%%
%% Interleaves lines of TeXname.tex and EPSname.eps
%% so that all the lines of EPSname.eps are copied in Outname.tex and 
%% lines from TeXname.tex starting with
%% \psfrag 
%% and 
%% %<pfx>
%% are written in Outname.eps, as a comment following the 
%% %%BoundigBox 
%% line.

%% This work may be distributed and/or modified under the
%% conditions of the LaTeX Project Public License, either version 1.3
%% of this license or (at your option) any later version.
%% The latest version of this license is in
%%   http://www.latex-project.org/lppl.txt
%% and version 1.3 or later is part of all distributions of LaTeX
%% version 2005/12/01 or later.
%%
%% This file is currently maintained by Pascal Kockaert

function psfragx(TeXname,EPSname,Outname)
TMPname='psfragx_tmp';
if nargin<2, EPSname=TeXname; end
if nargin<3, Outname=EPSname; end
if Outname==EPSname, 
   eval(['!rm ',TMPname,'.eps'])    
   eval(['!mv ',EPSname,'.eps ',TMPname,'.eps'])
   EPSname=TMPname;
end
TeXName=([TeXname,'.tex']);
EPSName=([EPSname,'.eps']);
OutName=([Outname,'.eps']);

%% Uncomment for debugging purposes
%% eval(['ls ',TeXName])
%% eval(['ls ',EPSName])
%% eval(['ls ',OutName])


BeginInput ='%%BoundingBox:';
BeginPSFRAG='%<pfx>\pfxbegin[1.0]{laprint}%';
EndPSFRAG  ='%<pfx>\pfxend';
StartPFX   ='%<*pfx> Inserted where \begin{psfrags}% occured';
StopPFX    ='%</pfx> Inserted where \end{psfrags}% occured';
EndInput  ='%\endinput';
EndOfFile ='%%EOF';
ResizeBox ='%<pfx>\def\naturalwidth';
StopOn    ={'\psfrag{','<pfx>','\begin{psfrags}','\end{psfrags}','\resizebox'};

TeXFile=fopen(TeXName,'r');
if (TeXFile==-1)
        error(['I was not able to open ',TeXName,'!']);
end
EPSFile=fopen(EPSName,'r');
if (EPSFile==-1)
        error(['I was not able to open ',EPSName,'!']);
end
OutFile=fopen(OutName,'w');
if (OutFile==-1)
        error(['I was not able to open ',OutName,'!']);
end

[sEPS,llEPS,iEPS]=CopyUntil(EPSFile,OutFile,{BeginInput});
       if sEPS~=1, error(['No line contains ',BeginInput]); 
       else
                 fprintf(OutFile,'%s\n',llEPS);       
       end

%%%
%%% Write preamble
%%% 
fprintf(OutFile,'%%<*pfx> Begin Preamble\n');
fprintf(OutFile,'%%\\providecommand*{\\pfxbegin}[2][]{}%%\n');
fprintf(OutFile,'%%\\providecommand{\\pfxend}{}%%\n');
fprintf(OutFile,'%%</pfx> End Preamble\n');
%%%
%%% Copy interesting lines
%%%
while 1
       [sTeX,llTeX,iTeX]=ReadUntil(TeXFile,StopOn);
              if sTeX~=1, break; end
       switch iTeX
          case 1,    %   \psfrag
                 fprintf(OutFile,'%%%s\n',llTeX);
          case 2,    %   %<pfx>
                 fprintf(OutFile,'%s\n',llTeX);
          case 3,    %   \begin{psfrags}
                 fprintf(OutFile,'%s\n',BeginPSFRAG);
                 fprintf(OutFile,'%s\n',StartPFX);          
          case 4,    %   \end{psfrags}
                 fprintf(OutFile,'%s\n',StopPFX);                    
                 fprintf(OutFile,'%s\n',EndPSFRAG);
          case 5,    %   \resizebox   
                 tmpbeg=findstr(llTeX,'{');
                 tmpend=findstr(llTeX,'}');       
                 if (length(tmpbeg)>0)&(length(tmpend)>0)
                     if (tmpbeg(1)<tmpend(1))
                        fprintf(OutFile,'%s%s%%\n',ResizeBox,llTeX(tmpbeg(1):tmpend(1)));
                     end
                 end
       otherwise
              error('Otherwise should never happen !')
       end
end
%%%
%%% Write postamble
%%%
fprintf(OutFile,'%s\n',EndInput);
%%%
%%% Copy to the end of file
%%%
[sEPS,llEPS,iEPS]=CopyUntil(EPSFile,OutFile,{''});
%%%
%%% Close files
%%%
fclose(OutFile);
fclose(TeXFile);
fclose(EPSFile);
return

function [OK,lastline,elt]=CopyUntil(fidIn,fidOut,linebeg);
sl=length(linebeg);
if sl==0, OK=-2; return, end
llb=zeros(sl);
for ii=1:sl
       llb(ii)=length(linebeg{ii});
end
lastline='';
OK=0;
elt=0;
while 1
       Line=fgetl(fidIn);
       if ~isstr(Line), 
              OK=-1;
              return, 
       end  %EndOfFile
       for ii=1:sl
%%%              fprintf('Seeking for line starting with %s.\n',linebeg{ii});
              if llb==0,    %%% Copying to the end of file
              else
                     if length(Line)>=llb(ii)
%%%                            fprintf('This line counts more than %i chars.\n',llb(ii));
                            if Line(1:llb(ii))==linebeg{ii},
                                   OK=1;
                                   elt=ii;
                                   lastline=Line;
                                   break
                            end
                     end
              end
       end    %%% No matching string
       if OK==1, break, end
       if ~isempty(fidOut)
              fprintf(fidOut,'%s\n',Line);
       end
end
return

function [OK,lastline,elt]=ReadUntil(fidIn,linebeg);
         [OK,lastline,elt]=CopyUntil(fidIn,[],linebeg);
return
%</matlab>
%<*pfxprint>
%%% pfxprint [ -*- Matlab -*- ] Time-stamp: <2004-08-12 18:20:57 Pascal Kockaert>
%%%
% function pfxprint(fig,name,'optA','valA','optB','valB',...)
%
% TO USE THIS FUNCTION, THE FILE laprpfx.mat SHOULD BE IN THE MATLAB PATH
% 
% This function is to be used like laprint.m
% The EPS and TeX files resulting from the call to laprint with the given arguments
% are automatically merged into one EPS file that contains the 
% psfrags replacements as comments.
% These comments can be automatically used in LaTeX, with the help of the psfragx package.
%
% This file is subject to the LPPL licence (see other files in the source archive or www.ctan.org)
% Copyright 2004, Pascal Kockaert
%

function pfxprint(fig,name,varargin)
 deftxtint=get(0,'DefaultTextInterpreter');
 set(0,'DefaultTextInterpreter','none');

  laprint(fig,name,'options','laprpfx',varargin{:});
  psfragx(name);

 set(0,'DefaultTextInterpreter',deftxtint)

% Default options are
% LAPRINTOPT =
%              figno: 2
%           filename: 'laprint'
%              width: 12
%             factor: 0.8
%         scalefonts: 1
%      keepfontprops: 0
%         asonscreen: 0
%     keepticklabels: 0
%     mathticklabels: 0
%               head: 0
%            comment: 'Test de laprint'
%            caption: ''
%       extrapicture: 0
%             nzeros: 3
%            verbose: 'off'
%            figcopy: 1
%           printcmd: 'print('-f<figurenumber>','-depsc2','<filename.eps>')'
%            package: 'graphicx'
%              color: 0
%         createview: 0
%       viewfilename: 'unnamed_'
%        processview: 0
%               cmd1: 'latex -halt-on-error -interaction nonstopmode <viewfile>.tex'
%               cmd2: 'dvips -D600 -E* -o<viewfile>.eps <viewfile>.dvi'
%               cmd3: 'epstool --bbox --copy --output <filename>_final.eps <viewfile>.eps'
%               cmd4: 'rm <viewfile>.eps <viewfile>.dvi <viewfile>.aux <viewfile>.log <viewfile>.pfg'
%               cmd5: 'ghostview <filename>_final.eps&'
%               cmd6: ''
%               cmd7: ''
%               cmd8: ''
%</pfxprint>
%    \end{macrocode}
%\typeout{^^J***************************************************}
%\typeout{In order to generate index files, run^^J}
%\typeout{makeindex -s gglo -o psfragx.gls psfragx.glo}
%\typeout{makeindex -s gind psfragx.idx^^J}
%\typeout{***************************************************^^J}
%\Finale
\endinput
