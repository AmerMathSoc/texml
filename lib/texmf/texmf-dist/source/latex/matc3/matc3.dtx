% \iffalse meta-comment
% Copyright (C) 2013 by Dimitrios Vrettos - d.vrettos@gmail.com
%
% This file may be distriubuted and/or modified under the
% conditions of the LaTeX Project Public Licence, either
% version 1.3c of this licence or (at your option) any later
% version. The latest version of this licence is in:
%
%	http://www.latex-project.org/lppl.txt
%
% and version 1.3c  or later is part of all distriubutions of 
% LaTeX version 2008/05/04 or later.
%
% This work has the LPPL maintenance status `maintained'.
% and consists of the files listed in the README file. 
% \fi
%
%\iffalse
%<*driver>
\ProvidesFile{matc3.dtx}
%</driver>
%<package>\NeedsTeXFormat{LaTeX2e}[1999/12/01]
%<package>\ProvidesPackage{matc3}
%<*package> 
    [2013/04/06 v1.0.1 Pacchetto di comandi per i libri MatematicaC3]
%</package>
%
%<*driver>
\RequirePackage{amsmath, amsfonts, marvosym}
\RequirePackage[usenames,dvipsnames]{xcolor}
\documentclass[italian, 10pt]{ltxdoc}
\usepackage[T1]{fontenc}
\usepackage[utf8x]{inputenc}
\usepackage[italian]{babel}
\usepackage{doc}
\usepackage{matc3}
\usepackage{calc}
\usepackage{amsmath}
\usepackage{indentfirst}
\usepackage{pifont}
\usepackage{booktabs}
\EnableCrossrefs
\CodelineIndex
\RecordChanges
\GlossaryPrologue{\section*{Cronologia delle  modifiche}}
\IndexPrologue{\section*{Indice analitico}I numeri scritti in corsivo si riferiscono alla pagina in 
cui la voce corrispondente \`e
descritta; i numeri sottolineati si riferiscono alla riga del codice della definizione; i numeri in
tondo si riferiscono alle linee del codice in cui viene utilizzata la voce.}
\setcounter{IndexColumns}{2}
\newcounter{chapter}
\setcounter{chapter}{0}
\makeindex
\DoNotIndex{\@ifstar,\ensuremath, \left, \lr@valass, \mathbb, \mathclose, \mathopen, \n@valass, \right}
\DoNotIndex{\mathfrak, \DeclareMathOperator, \mathcal, \mathbf, \mathrm, \dag, \circ, \superscript}
\DoNotIndex{\providecommand, \ast,\newsavebox, \savebox, \usebox, \boxa, \fbox, \boxar, \boxas}
\DoNotIndex{\boxb, \boxc, \boxd, \boxdf, \boxdp, \boxe, \boxf, \boxi, \boxno, \boxp, \boxr, \boxs}
\DoNotIndex{\boxsi, \boxt, \boxv, \newcommand, \paragraph, \color, \ding, \vspace, \fontsize}
\DoNotIndex{\Writinghand, \,, \itemsep, \parskip, \setlength, \emph, \alph, \begin, \end, \hspace}
\DoNotIndex{\labelsep, \labelwidth, \textcolor, \vskip, \hrule, \renewcommand, \arraystretch}
\begin{document}
 \DocInput{matc3.dtx}
\end{document}
%</driver>
%\fi
%
%\CheckSum{64}
% \CharacterTable
%  {Upper-case    \A\B\C\D\E\F\G\H\I\J\K\L\M\N\O\P\Q\R\S\T\U\V\W\X\Y\Z
%   Lower-case    \a\b\c\d\e\f\g\h\i\j\k\l\m\n\o\p\q\r\s\t\u\v\w\x\y\z
%   Digits        \0\1\2\3\4\5\6\7\8\9
%   Exclamation   \!     Double quote  \"     Hash (number) \#
%   Dollar        \$     Percent       \%     Ampersand     \&
%   Acute accent  \'     Left paren    \(     Right paren   \)
%   Asterisk      \*     Plus          \+     Comma         \,
%   Minus         \-     Point         \.     Solidus       \/
%   Colon         \:     Semicolon     \;     Less than     \<
%   Equals        \=     Greater than  \>     Question mark \?
%   Commercial at \@     Left bracket  \[     Backslash     \\
%   Right bracket \]     Circumflex    \^     Underscore    \_
%   Grave accent  \`     Left brace    \{     Vertical bar  \|
%   Right brace   \}     Tilde         \~}
% 
% \changes{v1.0}{2013/04/05}{Primo rilascio pubblico}
% \changes{v1.0.1}{2013/04/06}{Bug corretto}  
% \GetFileInfo{matc3.sty}
%
% \title{Il pacchetto \textsf{matc3}\thanks{Questo documento
% corrisponde a \textsf{matc3}~\fileversion,
% data~\filedate.}}
% \author{Dimitrios Vrettos \\ \texttt{d.vrettos@gmail.com}}
%
% \maketitle
%
% \tableofcontents
% \section{Introduzione}
%
% Questo pacchetto \`e stato creato per soddisfare le esigenze dei testi di MatematicaC3
% (rilasciati con licenza \textit{Creative Commons} e scaricabili gratuitamente dal
% sito di matematicamente.it). 
% Si tratta di una raccolta di comandi personalizzati per lo più di carattere matematico.
% Ovviamente il pacchetto viene rilasciato anche per altri usi e scopi, non obbligatoriamente
% correlati al progetto di MatematicaC3.
% 
% La documentazione presenta la sezione dell'installazione del pacchetto \textsf{matc3},
% nonch\'e le sue dipendenze da altri pacchetti. Successivamente, vengono presi in esame
% le varie macro che offre il pacchetto. In fine, c'\`e l'implentazione commentata del codice
% sorgente.
% 
% Il pacchetto \textsf{matc3} viene rilasciato con la 
% licenza \LaTeX\ \textit{Project Public Licence, version 1.3c or later}\footnote{L'ultima versione
% della licenza
% \`e disponibile su \texttt{http://www.latex-project.org/lppl.txt}.}.

%
% \section{Installazione}
% \subsection{Usando Make}
%
% La distribuzione del pacchetto contiene un~\texttt{Makefile}. Dando il comando
% \begin{verbatim}
% 	$ make help
% \end{verbatim}
% vengono mostrate le varie opzioni disponibili. A titolo informativo ne vengono presentate due:
% \begin{verbatim}
% 	$ make install
% \end{verbatim}
% che compila i sorgenti e successivamente  installa il pacchetto e la documentazione e aggiorna
% il database.
%
% Se qualcosa, invece, non \`e andato a buon fine, si proceda con la disinstallazione completa:
% \begin{verbatim}
% 	$ make uninstall
% \end{verbatim}
%
% \section{Dipendenze}
% I pacchetti richiesti per far funzionare il pacchetto \textsf{matc3} sono:
% \begin{itemize}
%  \item \textsf{amsmath} per la matematica;
%  \item \textsf{amsfonts} per i simboli degli insiemi numerici;
%  \item \textsf{marvosym} per altri simboli.
% \end{itemize}

% \section{Uso}
% \subsection{Macro}
% \subsubsection{Insiemi numerici}
% I seguenti comandi vanno inseriti in un ambiente matematico.
% 
% \DescribeMacro{\insN}
% Il simbolo dei numeri naturali.
% 
% \DescribeMacro{\insZ}
% Il simbolo dei numeri interi.
% 
% \DescribeMacro{\insQ}
% Il simbolo dei numeri razionali.
% 
% \DescribeMacro{\insJ}
% Il simbolo dei numeri irrazionali.
% 
% \DescribeMacro{\insR}
% Il simbolo dei numeri reali.
% 
% \DescribeMacro{\insC}
% Il simbolo dei numeri complessi.
% 
% \DescribeMacro{\insD}
% Il simbolo dei numeri dispari.
%  
% \DescribeMacro{\insP}
% Il simbolo dei numeri pari.
% 
% La tabella~\ref{tab:insiemi} riporta un riepilogo delle macro appena descritte.
% \begin{table}[ht]
% \centering
% \caption{Insiemi numerici}
% \label{tab:insiemi}
% 
% \begin{tabular}{lcc}
% \toprule
%  \textit{Insieme}	& \textit{Comando}	& \textit{Simbolo}\\
% \midrule
%  Naturali		& |\insN|			& $\insN$ \\
%  Interi		& |\insZ|			& $\insZ$ \\
%  Razionali		& |\insQ|			& $\insQ$ \\
%  Irrazionali		& |\insJ|			& $\insJ$ \\
%  Reali	  	& |\insR|			& $\insR$ \\
%  Complessi		& |\insC|			& $\insC$ \\
%  Dispari		& |\insD|			& $\insD$ \\
%  Pari	  		& |\insP|			& $\insP$ \\
% \bottomrule
% \end{tabular}
% \end{table}
% 
% \subsubsection{Simboli e operatori matematici}
% 
% \DescribeMacro{\var}
% Macro utilizzata per la varianza nella statistica. 
% 
% \DescribeMacro{\cfvar}
% Coefficiente di variazione.
%
% \DescribeMacro{\cvar}
% Il campo di variazione (statistica). 
% 
% \DescribeMacro{\spV}
% \`E il simbolo usato per gli spazi vettoriali. Esempio d'uso:~|$\spV$| $\spV$.
% 
% \DescribeMacro{\Kor}
% Corrispondenza fra due insiemi. Scrivendo~|$\Kor$| si ottiene~$\Kor$.
% 
% \DescribeMacro{\Rel}
% Il simbolo usato per indicare una relazione tra insiemi. Se, ad esempio,~$A$ e~$B$ sono due insiemi,
% un'eventuale relazione fra di loro viene espressa con la formula~|$A \Rel B$| e il risultato ottenuto
% \`e~$A\Rel B$.
% 
% \DescribeMacro{\Dom}
% Il dominio di una funzione. Esempio:~|$\Dom$| $\Dom$.
% 
% \DescribeMacro{\Cod}
% Viene usato per indicare il codominio di una funzione. Come prima~|$\Cod$| $\Cod$.
% 
% \DescribeMacro{\divint}
% Si tratta del simbolo testuale della divisione. Esempio:~|$5 \divint 3$| dar\`a come 
% risultato~$5 \divint 3$.
% 
% \DescribeMacro{\mcd}
% Il massimo comune divisore:~|$\mcd (a,b)$| $\mcd (a,b)$.
% 
% \DescribeMacro{\mcm}
% Il minimo comune multiplo. Esempio:~|$\mcm (a,b)$| $\mcm (a,b)$.
%
% \DescribeMacro{\card}
% La cardinalit\`a di un insieme. Il risultato della formula~|$\card A$| \`e~$\card A$.
% 
% \DescribeMacro{\CE}
% Questa macro indica le condizioni di esistenza di un'espressione. |$\CE$| $\CE$.
% 
% \DescribeMacro{\ID}
% \`E l'insieme delle definizioni:~|$\ID$| $\ID$.
% 
% \DescribeMacro{\IS}
% L'insieme delle soluzioni:~|$\IS$| $\IS$.
% 
% \DescribeMacro{\IM}
% \`E l'insieme delle immagini:~|$\IM$| $\IM$.
% 
% \DescribeMacro{\Area}
% L'area di una superficie. Ad esempio~|$\Area (ABCD)$| $\Area(ABCD)$.
% 
% \DescribeMacro{\media}
% La media usata nella statistica. |$\media (2,3,4)$| $\media (2,3,4)$.
% 
% \DescribeMacro{\mediana}
% La mediana usata nella statistica. |$\mediana (2,3,4)$| $\mediana (2,3,4)$.
% 
% \DescribeMacro{\moda}
% La moda usata nella statistica. |$\moda (2,3,4)$| $\moda (2,3,4)$.
% 
% \DescribeMacro{\valass}
% Il valore assoluto di un numero; ad esempio:~|$\valass{-12}$| $\valass{-12}$. 
% 
% \DescribeMacro{\longarray}
% Permette di scrivere un'array raddoppiando l'interlinea.
% \subsubsection{Altri comandi}
% 
% \DescribeMacro{\osservazione}
% Intitola un nuovo capoverso come ``osservazione'', ponendo all'inizio il simbolo~``\ding{113}''.
% 
% \DescribeMacro{\conclusione}
% Come nel caso precedente. Stavolta il simbolo usato è ``\ding{109}''.
% 
% \DescribeMacro{\vspazio}
% Aggiunge una riga aggiuntiva dopo la fine di un capoverso.
% 
% \DescribeMacro{\risolvi}
% Questo comando serve per indicare l'esercizio che c'è da risolvere usando un riferimento.
% Ad esempio |\risolvi{\ref{<nome esercizio>}}|.
% 
% \DescribeMacro{\risolvii}
% Se invece gli esercizi da risolvere sono più di uno, allora viene usato:
% 
% |\risolvii{\ref{<nome esercizio 1>} \ref{<nome esercizio 2>}}|.
% 
% \DescribeMacro{\Ast}
% Crea un asterisco in posizione di apice. Il codice |\Ast| da come risultato~``{\Ast}''.
% 
% \DescribeMacro{\croce}
% Come nel caso dell'asterisco, si crea una croce in posizione di apice: |\croce| ``{\croce}''.
% 
% \DescribeMacro{\grado}
% Stampa il simbolo del grado. |$10\grado$| $10\grado$. 
% 
% \DescribeMacro{\aC}
% Crea la sigla ``\aC'' (\textit{avanti Cristo}).
% 
%
% \subsubsection{Lettere in scatola}
% 
% \DescribeMacro{\boxA}
% \boxA
% 
% \DescribeMacro{\boxAR}
% \boxAR
% 
% \DescribeMacro{\boxAS}
% \boxAS
% 
% \DescribeMacro{\boxB}
% \boxB
% 
% \DescribeMacro{\boxC}
% \boxC
% 
% \DescribeMacro{\boxD}
% \boxD
% 
% \DescribeMacro{\boxDF}
% \boxDF
% 
% \DescribeMacro{\boxDP}
% \boxDP
% 
% \DescribeMacro{\boxE}
% \boxE
% 
% \DescribeMacro{\boxF}
% \boxF
%
% \DescribeMacro{\boxI}
% \boxI
% 
% \DescribeMacro{\boxNo}
% \boxNo
% 
% \DescribeMacro{\boxP}
% \boxP
% 
% \DescribeMacro{\boxR}
% \boxR
% 
% \DescribeMacro{\boxS}
% \boxS
% 
% \DescribeMacro{\boxSi}
% \boxSi
% 
% \DescribeMacro{\boxT}
% \boxT
% 
% \DescribeMacro{\boxV}
% \boxV
% 
% 
% \section{Implementazione}
%
% \begin{macro}{\insN}
% Definizione del simbolo dei numeri naturali:
% 
%    \begin{macrocode}
\newcommand{\insN}{\ensuremath{\mathbb{N}}}
%    \end{macrocode}
% \end{macro}
% \begin{macro}{\insZ}
% Definizione del simbolo dei numeri interi:
%    \begin{macrocode}
\newcommand{\insZ}{\ensuremath{\mathbb{Z}}}
%    \end{macrocode}
% \end{macro}
% \begin{macro}{\insQ}
% Definizione del simbolo dei numeri razionali:
%    \begin{macrocode}
\newcommand{\insQ}{\ensuremath{\mathbb{Q}}}
%    \end{macrocode}
% \end{macro}
% \begin{macro}{\insJ}
% Definizione del simbolo dei numeri irrazionali:
%    \begin{macrocode}
\newcommand{\insJ}{\ensuremath{\mathbb{J}}}
%    \end{macrocode}
% \end{macro}
% \begin{macro}{\insR}
% Definizione del simbolo dei numeri reali:
%    \begin{macrocode}
\newcommand{\insR}{\ensuremath{\mathbb{R}}}
%    \end{macrocode}
% \end{macro}
% \begin{macro}{\insC}
% Definizione del simbolo dei numeri complessi:
%    \begin{macrocode}
\newcommand{\insC}{\ensuremath{\mathbb{C}}}
%    \end{macrocode}
% \end{macro}
% \begin{macro}{\insD}
% Definizione del simbolo dei numeri dispari:
%    \begin{macrocode}
\newcommand{\insD}{\ensuremath{\mathbb{D}}}
%    \end{macrocode}
% \end{macro}
% \begin{macro}{\insP}
% Definizione del simbolo dei numeri pari:
%    \begin{macrocode}
\newcommand{\insP}{\ensuremath{\mathbb{P}}}
%    \end{macrocode}
% \end{macro}
% \begin{macro}{\var}
% Varianza. 
%    \begin{macrocode}
\newcommand{\var}[1]{\ensuremath{\mathrm{Var}{#1}}}
%    \end{macrocode}
% \end{macro}
% \begin{macro}{\cfvar}
% Coefficiente di variazione. 
%    \begin{macrocode}
\newcommand{\cfvar}[1]{\ensuremath{\mathrm{CV}{#1}}}
%    \end{macrocode}
% \end{macro}
% \begin{macro}{\cvar}
% Campo di varianza.
%    \begin{macrocode}
\newcommand{\cvar}[1]{\ensuremath{\mathrm{CVar}{#1}}}
%    \end{macrocode}
% \end{macro}
% \begin{macro}{\spV}
% Definizione del simbolo degli spazi vettoriali:
%    \begin{macrocode}
\newcommand{\spV}{\ensuremath{\mathbf{V}}}
%    \end{macrocode}
% \end{macro}
% \begin{macro}{\Kor}
% Definizione del simbolo di una corrispondenza fra due insiemi:
%    \begin{macrocode}
\newcommand{\Kor}{\ensuremath{\mathbf{K}}}
%    \end{macrocode}
% \end{macro}
% \begin{macro}{\Rel}
% Definizione del simbolo della realazione tra insiemi:
%    \begin{macrocode}
\newcommand{\Rel}{\ensuremath{\mathfrak{R}}}
%    \end{macrocode}
% \end{macro}
% \begin{macro}{\Dom}
% Definizione del simbolo del dominio di una funzione:
%    \begin{macrocode}
\newcommand{\Dom}{\ensuremath{\mathcal{D}}}
%    \end{macrocode}
% \end{macro}
% \begin{macro}{\Cod}
% Definizione del simbolo impiegato per indicare il codominio di una funzione:
%    \begin{macrocode}
\newcommand{\Cod}{\ensuremath{\mathcal{C}}}
%    \end{macrocode}
% \end{macro}
%    \begin{macro}{\divint}
%  Definizione del simbolo testuale della divisione:
% \begin{macrocode}
\DeclareMathOperator{\divint}{div}
%    \end{macrocode}
% \end{macro}
% \begin{macro}{\mcd}
% Definizione del massimo comune divisore:
%    \begin{macrocode}
\DeclareMathOperator{\mcd}{MCD}
%    \end{macrocode}
% \end{macro}
% \begin{macro}{\mcm}
% Definizione del minimo comune multiplo:
%    \begin{macrocode}
\DeclareMathOperator{\mcm}{mcm}
%    \end{macrocode}
% \end{macro}
% \begin{macro}{\card}
% Definizione della cardinalit\`a di un insieme:
%    \begin{macrocode}
\DeclareMathOperator{\card}{card}
%    \end{macrocode}
% \end{macro}
% \begin{macro}{\CE}
% Definizione del comando della condizione di esistenza:
%    \begin{macrocode}
\DeclareMathOperator{\CE}{C.E.}
%    \end{macrocode}
% \end{macro}
% \begin{macro}{\ID}
% L'insieme delle definizioni:
%    \begin{macrocode}
\DeclareMathOperator{\ID}{I.D.}
%    \end{macrocode}
% \end{macro}
% \begin{macro}{\IS}
% Definizione dell'insieme delle soluzioni:
%    \begin{macrocode}
\DeclareMathOperator{\IS}{I.S.}
%    \end{macrocode}
% \end{macro}
% \begin{macro}{\IM}
% Definizione dell'insieme delle immagini:
%    \begin{macrocode}
\DeclareMathOperator{\IM}{IM.}
%    \end{macrocode}
% \end{macro}
% \begin{macro}{\Area}
% Definizione dell'area di una superficie:
%    \begin{macrocode}
\DeclareMathOperator{\Area}{Area}
%    \end{macrocode}
% \end{macro}
% \begin{macro}{\media}
% Definizione della media:
%    \begin{macrocode}
\DeclareMathOperator{\media}{Media}
%    \end{macrocode}
% \end{macro}
% \begin{macro}{\mediana}
% Definizione della mediana:
%    \begin{macrocode}
\DeclareMathOperator{\mediana}{Mediana}
%    \end{macrocode}
% \end{macro}
% \begin{macro}{\moda}
% Definizione della moda: 
%    \begin{macrocode}
\DeclareMathOperator{\moda}{Moda}
%    \end{macrocode}
% \end{macro}
% \begin{macro}{\valass}
% Definizione del valore assoluto:
%    \begin{macrocode}
\newcommand\valass{\@ifstar\lr@valass\n@valass}
\newcommand\lr@valass[1]{\left|#1\right|}
\newcommand\n@valass[2][]{\mathopen{#1|}#2\mathclose{#1|}}     
%    \end{macrocode}
% \end{macro}
% \begin{macro}{\longarray}
% Definizione dell'array a doppia interlinea:
%    \begin{macrocode}
\newcommand{\longarray}{\renewcommand{\arraystretch}{2}}
%    \end{macrocode}
% \end{macro}
% \begin{macro}{\osservazione}
% Definizione:
%    \begin{macrocode}
\newcommand{\osservazione}{%
  \paragraph{%
    {\color{Mahogany}\ding{113}} Osservazione%
  }%
}
%    \end{macrocode}
% \end{macro}
% \begin{macro}{\conclusione}
% Definizione:
%    \begin{macrocode}
\newcommand{\conclusione}{%
  \paragraph{%
    {\color{Mahogany}\ding{109}} Conclusione%
  }%
}
%    \end{macrocode}
% \end{macro}
% \begin{macro}{\vspazio}
% Definizione del comando:
%    \begin{macrocode}
\newcommand{\vspazio}{\vspace{1ex}}
%    \end{macrocode}
% \end{macro}
% \begin{macro}{\risolvi}
% Si utilizza il simbolo~``\Writinghand '' impostandolo a dimensione di~\texttt{12pt} e dando il titolo ``Esercizio proposto: ''.
%    \begin{macrocode}
\newcommand{\risolvi}{%
  {\fontsize{12pt}{0pt}%
    \Writinghand\,} \emph{Esercizio proposto: %
  }%
}
%    \end{macrocode}
% \end{macro}
% \begin{macro}{\risolvii}
% Come nel caso precedente, ma stavolta viene stampato ``Esercizi proposti: ''.
%    \begin{macrocode}
\newcommand{\risolvii}{%
  {\fontsize{12pt}{0pt}%
    \Writinghand\,} \emph{Esercizi proposti: %
  }%
} 
%    \end{macrocode}
% \end{macro}
% Per primo viene definito il comando~|\superscript|, che sar\`a utilizzato
% nelle definizioni dei comandi~|\ast| e~|\croce|.
%    \begin{macrocode}
\providecommand{\superscript}[1]{\ensuremath{^{#1}}}
%    \end{macrocode}
% 
% \begin{macro}{\Ast}
% Definizione dell'asterisco:
%    \begin{macrocode}
\newcommand{\Ast}{\superscript{\ast}}
%    \end{macrocode}
% \end{macro}
% \begin{macro}{\croce}
% Definizione della croce:
%    \begin{macrocode}
\newcommand{\croce}{\superscript{\dag}}
%    \end{macrocode}
% \end{macro}
% \begin{macro}{\grado}
% Definizione del simbolo del grado:
%    \begin{macrocode}
\newcommand{\grado}{\ensuremath{{}^{\circ}}}
%    \end{macrocode}
% \end{macro}
% \begin{macro}{\aC}
% Definizione della sigla `avanti Cristo'.
%    \begin{macrocode}
\newcommand{\aC}{a.C.}
%    \end{macrocode}
% \end{macro}
% \begin{macro}{\boxA}
% 
%    \begin{macrocode}
\newsavebox{\boxa}
\savebox{\boxa}[12pt][c]{\fbox{A}}
\newcommand{\boxA}{\usebox{\boxa}}
%    \end{macrocode}
% \end{macro}
% \begin{macro}{\boxAR}
% 
%    \begin{macrocode}
\newsavebox{\boxar}
\savebox{\boxar}[12pt][c]{\fbox{AR}}
\newcommand{\boxAR}{\usebox{\boxar}}
%    \end{macrocode}
% \end{macro}
% \begin{macro}{\boxAS}
% 
%    \begin{macrocode}
\newsavebox{\boxas}
\savebox{\boxas}[12pt][c]{\fbox{AS}}
\newcommand{\boxAS}{\usebox{\boxas}}
%    \end{macrocode}
% \end{macro}
% \begin{macro}{\boxB}
% 
%    \begin{macrocode}
\newsavebox{\boxb}
\savebox{\boxb}[12pt][c]{\fbox{B}}
\newcommand{\boxB}{\usebox{\boxb}}
%    \end{macrocode}
% \end{macro}
% \begin{macro}{\boxC}
% 
%    \begin{macrocode}
\newsavebox{\boxc}
\savebox{\boxc}[12pt][c]{\fbox{C}}
\newcommand{\boxC}{\usebox{\boxc}}
%    \end{macrocode}
% \end{macro}
% \begin{macro}{\boxD}
% 
%    \begin{macrocode}
\newsavebox{\boxd}
\savebox{\boxd}[12pt][c]{\fbox{D}}
\newcommand{\boxD}{\usebox{\boxd}}
%    \end{macrocode}
% \end{macro}
% \begin{macro}{\boxDF}
% 
%    \begin{macrocode}
\newsavebox{\boxdf}
\savebox{\boxdf}[12pt][c]{\fbox{DF}}
\newcommand{\boxDF}{\usebox{\boxdf}}
%    \end{macrocode}
% \end{macro}
% \begin{macro}{\boxDP}
% 
%    \begin{macrocode}
\newsavebox{\boxdp}
\savebox{\boxdp}[12pt][c]{\fbox{DP}}
\newcommand{\boxDP}{\usebox{\boxdp}}
%    \end{macrocode}
% \end{macro}
% \begin{macro}{\boxE}
% 
%    \begin{macrocode}
\newsavebox{\boxe}
\savebox{\boxe}[12pt][c]{\fbox{E}}
\newcommand{\boxE}{\usebox{\boxe}}
%    \end{macrocode}
% \end{macro}
% \begin{macro}{\boxF}
% 
%    \begin{macrocode}
\newsavebox{\boxf}
\savebox{\boxf}[12pt][c]{\fbox{F}}
\newcommand{\boxF}{\usebox{\boxf}}
%    \end{macrocode}
% \end{macro}
% \begin{macro}{\boxI}
% 
%    \begin{macrocode}
\newsavebox{\boxi}
\savebox{\boxi}[12pt][c]{\fbox{I}}
\newcommand{\boxI}{\usebox{\boxi}}
%    \end{macrocode}
% \end{macro}
% \begin{macro}{\boxNo}
% 
%    \begin{macrocode}
\newsavebox{\boxno}
\savebox{\boxno}[12pt][c]{\fbox{No}}
\newcommand{\boxNo}{\usebox{\boxno}}
%    \end{macrocode}
% \end{macro}
% \begin{macro}{\boxP}
% 
%    \begin{macrocode}
\newsavebox{\boxp}
\savebox{\boxp}[12pt][c]{\fbox{P}}
\newcommand{\boxP}{\usebox{\boxp}}
%    \end{macrocode}
% \end{macro}
% \begin{macro}{\boxR}
% 
%    \begin{macrocode}
\newsavebox{\boxr}
\savebox{\boxr}[12pt][c]{\fbox{R}}
\newcommand{\boxR}{\usebox{\boxr}}
%    \end{macrocode}
% \end{macro}
% \begin{macro}{\boxS}
% 
%    \begin{macrocode}
\newsavebox{\boxs}
\savebox{\boxs}[12pt][c]{\fbox{S}}
\newcommand{\boxS}{\usebox{\boxs}}
%    \end{macrocode}
% \end{macro}
% \begin{macro}{\boxSi}
% 
%    \begin{macrocode}
\newsavebox{\boxsi}
\savebox{\boxsi}[12pt][c]{\fbox{Sì}}
\newcommand{\boxSi}{\usebox{\boxsi}}
%    \end{macrocode}
% \end{macro}
% \begin{macro}{\boxT}
% 
%    \begin{macrocode}
\newsavebox{\boxt}
\savebox{\boxt}[12pt][c]{\fbox{T}}
\newcommand{\boxT}{\usebox{\boxt}}
%    \end{macrocode}
% \end{macro}
% \begin{macro}{\boxV}
% 
%    \begin{macrocode}
\newsavebox{\boxv}
\savebox{\boxv}[12pt][c]{\fbox{V}}
\newcommand{\boxV}{\usebox{\boxv}}
%    \end{macrocode}
% \end{macro}
% \clearpage
% \Finale
%
% \addcontentsline{toc}{section}{Indice analitico}
% \PrintIndex
% \addcontentsline{toc}{section}{Cronologia delle modifiche}
% \PrintChanges 
\endinput
