% \iffalse Meta-Kommentar
%
% Copyright 1997, 1998, 1998, 1999
% Thomas Leineweber
% Lehrstuhl VI, Fachbereich Informatik,
% Universitaet Dortmund
% (leineweb@ls6.cs.uni-dortmund.de)
%
%        This package is based on work by J. Wahlmann and R. Garmann.
%        I took their package and enhanced it.
%
% Enhancements provided by
%   Jean-Pierre Drucbert
% 
% This is the documented macrocode for the generation and
% documentation of the package `method'.
%
% This file may be distributed under the terms of the LaTeX Project Public
% License, as described in lppl.txt in the base LaTeX distribution.
% Either version 1.0 or, at your option, any later version.
%
% \fi
%
% \CheckSum{787}
%% \CharacterTable
%%  {Upper-case    \A\B\C\D\E\F\G\H\I\J\K\L\M\N\O\P\Q\R\S\T\U\V\W\X\Y\Z
%%   Lower-case    \a\b\c\d\e\f\g\h\i\j\k\l\m\n\o\p\q\r\s\t\u\v\w\x\y\z
%%   Digits        \0\1\2\3\4\5\6\7\8\9
%%   Exclamation   \!     Double quote  \"     Hash (number) \#
%%   Dollar        \$     Percent       \%     Ampersand     \&
%%   Acute accent  \'     Left paren    \(     Right paren   \)
%%   Asterisk      \*     Plus          \+     Comma         \,
%%   Minus         \-     Point         \.     Solidus       \/
%%   Colon         \:     Semicolon     \;     Less than     \<
%%   Equals        \=     Greater than  \>     Question mark \?
%%   Commercial at \@     Left bracket  \[     Backslash     \\
%%   Right bracket \]     Circumflex    \^     Underscore    \_
%%   Grave accent  \`     Left brace    \{     Vertical bar  \|
%%   Right brace   \}     Tilde         \~}
%
% \title{\LaTeX-package for an easy declaration of functions and variables}
% \author{Thomas Leineweber}
% \maketitle
%
% \changes{v1.0}{????/??/??}{First version by J. Wahlmann}
% \changes{v1.1}{1995/10/17}{Facelifting by R. Garmann}
% \changes{v1.1}{1995/10/18}{Facelifting by R. Garmann}
% \changes{v1.1}{1995/10/19}{New environment data for variables}
% \changes{v1.1}{1995/10/19}{init and del for the environment data added}
% \changes{v1.1}{1995/10/25}{Test for the use of the special commands
% outside the environments}
% \changes{v1.2}{1997/06/25}{Change documentation to a dtx-file}
% \changes{v1.3}{1997/06/29}{Documentation changes}
% \changes{v1.6}{1999/01/04}{Documentation changes}
% \changes{v1.7}{1999/01/28}{Documentation changes}
% \changes{v1.7}{1999/01/28}{First parts for internationalization}
% \changes{v1.8}{1999/02/04}{Change standard to english}
% \changes{v2.0}{1999/02/05}{First public version with 
% internationalization and localization for english and german}
% \changes{v2.0b}{1999/03/25}{Change licencse to lppl}
%
% \section{Overview}
% The package method can be used to easily format method- and
% variabledeclarations with \LaTeX.
% It is based on work by J. Wahlmann and Robert Garmann.
% 
%\section{Usage}
% The package is used as usual:
%\begin{quote}
% |\usepackage[<language>]{method}|
%\end{quote}
% It defines two new environments: method and data. Method is used to
% typeset method-declarations, data for variable-declarations.
% At the moment the two options \textsf{english} and \textsf{german}
% are defined. With these options it is possible to select the language
% used to typeset the declarations. In the future some other languages
% will be added.
%
%\section{The environment method}
% \DescribeEnv{method}
% Within the environment
% method the following commands are defined:
%
%\begin{itemize}
%\item \DescribeMacro{\head}
%  \cs{head}\texttt{\{\emph{Header}\}:} The header of the method.
%\item \DescribeMacro{\para}
%  \cs{para}\texttt{\{\emph{Name}\}\{\emph{Description}\}:}
%  Name and description of a parameter.
%\item \DescribeMacro{\precond}
%  \cs{precond}\texttt{\{\emph{Precondition}\}:}
%  Description of a precondition of the method.
%\item \DescribeMacro{\descr}
%  \cs{descr}\texttt{\{\emph{Description}\}:}
%  Description of the method itself.
%\item \DescribeMacro{\postcond}
%  \cs{postcond}\texttt{\{\emph{postcondition}\}:}
%  Description of a postcondition of the method.
%\item \DescribeMacro{\error}
%  \cs{error}\texttt{\{\emph{Exception}\}:} Error and exceptions.
%\item \DescribeMacro{\return}
%  \cs{return}\texttt{\{\emph{Return value}\}:}
%  Description of the data returned by the method.
%\item \DescribeMacro{\see}
%  \cs{see}\texttt{\{\emph{where}\}:} Cross-References.
%\end{itemize}
%
%These commands have the following in common:
%\begin{itemize}
%\item All parameters are simple texts.
%\item The sequence of the commands inside the method-environment is
%  not relevant. The parts are typeset automatically.
%\item Up to 26 \cs{para}-commands are allowed inside a method
%  environment. When there are more, a warning will be issued und the
%  following parameters will be ignored.
%\item The header of the method and the parameters are typeset in a
%typewriter font.
%\item If the header is extremely long, it can be typeset in more
%lines with the following macros (an example is given further down):
%\begin{verbatim}
%       \headtabbed{<functionname>}     Name of the function
%       \headpara{<parametername>}      one or more parameter
%\end{verbatim}
%\end{itemize}
%
%\section{The environment data} \DescribeEnv{data}
% The environment data equals to the environment method. The macros 
% \cs{head}, \cs{descr} and \cs{see} can also be used inside a
% data-environment.
% Further macros inside a data-environment are:
%\begin{itemize}
%\item \DescribeMacro{\init}
%  \cs{init}\texttt{\{\emph{Info}\}:} Information about the generation
%  of the objekt.
%\item \DescribeMacro{\del}
%  \cs{del}\texttt{\{\emph{Info}\}:} Information about the release of
%  the object.
%\end{itemize}
%
%\section{Examples}
%In this section some examples for the usage of the environments
%method and data are shown.
%\begin{verbatim}     
%      \begin{method}
%       \head{int div(int a, int b, double \&c);}
%       \para{a}{dividend}
%       \para{b}{divisor}
%       \para{\&c}{result of the division}
%       \precond{no preconditions}
%       \descr{Divides \texttt{a} by \texttt{b} and gives the result
%       in \texttt{c}}
%       \postcond{no postconditions}
%       \error{no errors}
%       \return{\texttt{-1}, when  \texttt{b==0}, else \texttt{0}}
%       \see{your favourite mathematics book}
%      \end{method}
%
%      \begin{method}
%       \headtabbed{PrimObject(}
%         \headpara{const Matrix transformation,}
%         \headpara{AbstGeometry *geometry = 0,}
%         \headpara{MaterialApplication *material = 0,}
%         \headpara{AbstBumpMap *bumpMap = 0,}
%         \headpara{Distribution *distribution = 0);}
%       \para{transformation}{Transformation matrix}
%       \para{*geometry}{\ldots}
%       \descr{\ldots}
%      \end{method}
%
%      \begin{data}
%       \head{char *name}
%       \descr{Name of the user}
%      \end{data}
%      \begin{data}
%       \head{char *no}
%       \descr{Telephone-number of the user}
%       \see{Telephone Book}
%      \end{data}
%\end{verbatim}
%
% \StopEventually
%
% \section{Identification und documentation}
%
%    This package can only be used with \LaTeXe. Therefore make sure,
%    we use no other \TeX-format.
%    \begin{macrocode}
\NeedsTeXFormat{LaTeX2e}
%    \end{macrocode}
%
%    Show the name of the package and its version
%    \begin{macrocode}
%<+method>\ProvidesPackage{method}
%<+method>              [1999/03/25 v2.0b
%<+method>               LaTeX-package for method- and
%<+method>               data-descriptions (TL)]
%    \end{macrocode}
%
%
%    We have a specialized class for the documentation.
%    \begin{macrocode}
%<*driver>
\documentclass[a4paper]{ltxdoc}
%    \end{macrocode}
%    Set the specific options for the documentation of the package.
%    \begin{macrocode}
\DoNotIndex{\",\\,\addtolength,\begin,\CodelineIndex,\CodelineNumbered}
\DoNotIndex{\def,\DocInput,\documentclass,\DoNotIndex,\EnableCrossrefs}
\DoNotIndex{\end,\fbox,\fboxrule,\hfill,\hspace,\ifcase,\or,\fi}
\DoNotIndex{\ifnum,\fi,\item,\itemindent,\labelsep,\labelwidth}
\DoNotIndex{\leftmargin,\listparindent,\NeedsTeXFormat,\newcommand}
\DoNotIndex{\newcount,\newcounter,\newenvironment,\newlength,\sloppy}
\DoNotIndex{\nopagebreak,\PackageError,\parbox,\parindent,\stepcounter}
\DoNotIndex{\PrintChanges,\PrintIndex,\ProvidesPackage,\RecordChanges}
\DoNotIndex{\setcounter,\setlength,\textbf,\texttt,\usepackage,\vspace}
\DoNotIndex{\settowidth,\textwidth,\topsep}
\CodelineNumbered
\CodelineIndex
\EnableCrossrefs
\RecordChanges
\setcounter{StandardModuleDepth}{1}
\usepackage[T1]{fontenc}
\usepackage[latin1]{inputenc}
%    \end{macrocode}
%    Give all details.
%    \begin{macrocode}
\begin{document}
\DocInput{method.dtx}
\PrintIndex
\PrintChanges
\end{document}
%</driver>
%    \end{macrocode}
%
%
%\section{Package internals}
%
% At the start of a method-environment the actual textwidth is read
% and saved for the layout of the description.
%
% The commands for the parts, namely \cs{head}, \cs{para},
% \cs{precond}, ..., define
% internal commands with the names \cs{meth@head}, \cs{meth@pa},
% \cs{meth@pb}, ..., \cs{meth@pz},
% \cs{meth@precond}, ...  which are defined with the actual
% parameters.
%
% At the end of a method-environment all these internal saved data is
% typeset in a (hopefully) fashionable way.

%
% \section{Helping commands}
% \begin{macro}{\meth@paranum}
% \begin{macro}{\meth@headparanum}
% The counter \cs{meth@paranum} counts the number of
% \cs{para}-commands within a
% method-environment:
%    \begin{macrocode}
\newcounter{meth@paranum}
%    \end{macrocode}
% The counter \cs{meth@headparanum} stores how many 
% \cs{headpara}-commands are given within a method-environment:
%    \begin{macrocode}
\newcounter{meth@headparanum}
%    \end{macrocode}
% \end{macro}
% \end{macro}
%
%\begin{macro}{\meth@totwid}
% The header will be typeset inside a framed minipage with a width of
% \cs{@totwid} (= \cs{textwidth} - 6mm):
%    \begin{macrocode}
\newlength{\meth@totwid}
%    \end{macrocode}
%\end{macro}
%
%\begin{macro}{\meth@indent}
%\begin{macro}{\meth@listdecl}
% \changes{v1.3}{1997/06/29}{Minimale �nderung.}
% The descriptions are organized as lists with the following parameters:
%    \begin{macrocode}
\def\meth@indent{3.5cm}
\def\meth@listdecl{\labelwidth3cm \labelsep0.5cm
                   \itemindent0cm \leftmargin\meth@indent
                   \topsep0cm \listparindent0cm}
%    \end{macrocode}
%\end{macro}
%\end{macro}
%
% \begin{macro}{\meth@righttotwid}
% The right part of the list has a width of \cs{meth@righttotwid} 
% (= \cs{meth@totwid} - \cs{meth@indent}):
%    \begin{macrocode}
\newlength{\meth@righttotwid}
%    \end{macrocode}
%\end{macro}
%
%\begin{macro}{\meth@namewid}
%\begin{macro}{\meth@nameindent}
%\begin{macro}{\meth@rightnamewid}
% The following lengths are used for the macro \cs{headtabbed}:
%    \begin{macrocode}
\newlength{\meth@namewid}
\newlength{\meth@nameindent}
\newlength{\meth@rightnamewid}
%    \end{macrocode}
%\end{macro}
%\end{macro}
%\end{macro}
%
%\section{Options}
%\begin{macro}{\textsee}
%\begin{macro}{\textinit}
%\begin{macro}{\textdel}
%\begin{macro}{\textreturn}
%\begin{macro}{\textprecond}
%\begin{macro}{\textpostcond}
%\begin{macro}{\textdescr}
%\begin{macro}{\texterror}%
% Now we have the option-processing. The option defines the language,
% which will be used to print the textual parts of the
% descriptions. At the moment only the languages english and german
% are defined.
%
% First, define the parts for german descriptions
%    \begin{macrocode}
\DeclareOption{german}{\def\textsee{Siehe auch:}
                       \def\textinit{Erzeugung:}
                       \def\textdel{Freigabe:}
                       \def\textreturn{R\"uckgabewert:}
                       \def\textprecond{Vorbed.:}
                       \def\textpostcond{Nachbed.:}
                       \def\textdescr{Beschreibung:}
                       \def\texterror{Ausnahmebeh.:}}

%    \end{macrocode}
% Now for the english descriptions:
%    \begin{macrocode}
\DeclareOption{english}{\def\textsee{see also:}
                       \def\textinit{initialisation:}
                       \def\textdel{disposal:}
                       \def\textreturn{return value:}
                       \def\textprecond{precondition:}
                       \def\textpostcond{postcondition:}
                       \def\textdescr{description:}
                       \def\texterror{exceptions:}}
%    \end{macrocode}
% \changes{v2.0a}{1999/02/18}{French localization (thanks to
% Jean-Pierre Drucbert)}
% The french descriptions, provided by Jean-Pierre Drucbert:
%    \begin{macrocode}
\DeclareOption{french}{\def\textsee{voir aussi:}
                       \def\textinit{initialisation:}
                       \def\textdel{lib\'eration:}
                       \def\textreturn{valeur de retour:}
                       \def\textprecond{pr\'econdition:}
                       \def\textpostcond{postcondition:}
                       \def\textdescr{description:}
                       \def\texterror{exceptions:}}
%    \end{macrocode}
% Make the english version the default version and process the options.
%    \begin{macrocode}
\ExecuteOptions{english}
\ProcessOptions\relax
%    \end{macrocode}
%\end{macro}
%\end{macro}
%\end{macro}
%\end{macro}
%\end{macro}
%\end{macro}
%\end{macro}
%\end{macro}
%
%\section{Error-detection}
%\begin{macro}{\meth@where}
%\begin{macro}{\meth@checkdoubleopen}
%\begin{macro}{\meth@checknotopen}
%\changes{v1.3}{1997/06/29}{Show error messages with \cs{PackageError}}
% The macro \cs{head} can be used both in the environments method and
% data. The following value is used to differentiate if method or data
% is active. If none is active, the counter is set to 99. Method sets
% it to 0, data to 1.
%    \begin{macrocode}
\newcount\meth@where  \meth@where=99
%    \end{macrocode}
% Now define the error messages:
%    \begin{macrocode}
\def\meth@checkdoubleopen{
  \ifnum\meth@where<99
     \PackageError{method}%
       {There is an method.sty-environment open!}%
       {}
  \fi
}
\def\meth@checknotopen{
  \ifnum\meth@where=99
     \PackageError{method}%
       {There is no method.sty-environment open!}%
       {}
  \fi
}
%    \end{macrocode}
%\end{macro}
%\end{macro}
%\end{macro}
%
%\section{The environment method}
%\begin{environment}{method}
% Now we define the environment method.
%    \begin{macrocode}
\newenvironment{method}
{
%    \end{macrocode}
% First we check, wether a method or data-environment is open. After
% that we set \cs{meth@where} to 0, which shows, that we are inside a
% method-environment.
%    \begin{macrocode}
  \meth@checkdoubleopen
  \meth@where=0
%    \end{macrocode}
% Define the lengths used for the typesetting of the method.
% \changes{v1.1}{????/??/??}{Abzug von \cs{meth@totwid} von 4~mm auf 6~mm erhh�ht}
%    \begin{macrocode}
  \setlength{\meth@totwid}{\textwidth}
  \addtolength{\meth@totwid}{-6mm}
  \setlength{\meth@righttotwid}{\meth@totwid}
  \addtolength{\meth@righttotwid}{-\meth@indent}
%    \end{macrocode}
% The right column is not very wide. Therfore use \cs{sloppy}.
%    \begin{macrocode}
  \sloppy
%    \end{macrocode}
% All parts are defined to nothing.
%\changes{v1.5}{1998/01/19}{extend to 26 parameters}
%    \begin{macrocode}
  \def\meth@head{}
  \def\meth@headtabbed{}
  \setcounter{meth@headparanum}{0}
   \def\meth@hpa{}\def\meth@hpb{}\def\meth@hpc{}\def\meth@hpd{}
   \def\meth@hpe{}\def\meth@hpf{}\def\meth@hpg{}\def\meth@hph{}
   \def\meth@hpi{}\def\meth@hpj{}\def\meth@hpk{}\def\meth@hpl{}
   \def\meth@hpm{}\def\meth@hpn{}\def\meth@hpo{}\def\meth@hpp{}
   \def\meth@hpq{}\def\meth@hpr{}\def\meth@hps{}\def\meth@hpt{}
   \def\meth@hpu{}\def\meth@hpv{}\def\meth@hpw{}\def\meth@hpx{}
   \def\meth@hpy{}\def\meth@hpz{}
  \setcounter{meth@paranum}{0}
   \def\meth@pa{}\def\meth@pb{}\def\meth@pc{}\def\meth@pd{}
   \def\meth@pe{}\def\meth@pf{}\def\meth@pg{}\def\meth@ph{}
   \def\meth@pi{}\def\meth@pj{}\def\meth@pk{}\def\meth@pl{}
   \def\meth@pm{}\def\meth@pn{}\def\meth@po{}\def\meth@pp{}
   \def\meth@pq{}\def\meth@pr{}\def\meth@ps{}\def\meth@pt{}
   \def\meth@pu{}\def\meth@pv{}\def\meth@pw{}\def\meth@px{}
   \def\meth@py{}\def\meth@pz{}
  \def\meth@precond{}
  \def\meth@descr{}
  \def\meth@postcond{}
  \def\meth@error{}
  \def\meth@return{}
  \def\meth@see{}
%    \end{macrocode}
% Now for the end of the environment. The first line in a paragraph
% is not indented and a small space is made above the header.
%    \begin{macrocode}
}{
  \parindent0cm
  \vspace{2mm}
%    \end{macrocode}
% now a sorted list of all given parts inside the environment. The
% user has to see, that only one of 
%  \cs{meth@head} and \cs{meth@headtabbed} is used.
% \changes{v1.3}{1997/06/29}{Minimal change in the format of typesetting}%
% \changes{v1.5}{1998/01/19}{extend to 26 parameters}
%    \begin{macrocode}
  \meth@head\meth@headtabbed
  \nopagebreak[4]
  \meth@pa \meth@pb \meth@pc \meth@pd \meth@pe \meth@pf \meth@pg
  \meth@ph \meth@pi \meth@pj \meth@pk \meth@pl \meth@pm \meth@pn
  \meth@po \meth@pp \meth@pq \meth@pr \meth@ps \meth@pt \meth@pu
  \meth@pv \meth@pw \meth@px \meth@py \meth@pz
  \meth@precond
  \meth@descr
  \meth@postcond
  \meth@error
  \meth@return
  \meth@see
%    \end{macrocode}
% Now set \cs{meth@where} back to 99.
%    \begin{macrocode}
  \meth@where=99
%    \end{macrocode}
% This is the end of the definition of the environment method.
%    \begin{macrocode}
}
%    \end{macrocode}
%\end{environment}
%
%\section{The environment data}
%\begin{environment}{data}
% The environment data is nearly equivalent to the environment method.
% \changes{v1.3}{1997/06/29}{Change \cs{meth@totwid} from 4~mm to 6~mm.}
%    \begin{macrocode}
\newenvironment{data}
{
  \meth@checkdoubleopen
  \meth@where=1
  \setlength{\meth@totwid}{\textwidth}
  \addtolength{\meth@totwid}{-6mm}
  \sloppy
  \def\meth@head{}
  \def\meth@descr{}
  \def\meth@init{}
  \def\meth@del{}  
  \def\meth@see{}
}{
  \parindent0cm
  \vspace{2mm}
  \meth@head
  \nopagebreak
  \meth@descr
  \meth@init
  \meth@del
  \meth@see
  \meth@where=99
}
%    \end{macrocode}
%\end{environment}
%
%\section{Macros for the parts inside the environments}
%
% The definitions for the parts of the environments data and method
% \subsection{\cs{head}}
% \begin{macro}{\head}
% The macro \cs{head} is used to typeset the header of the method or
% the definition of the variable.
%    \begin{macrocode}
\newcommand{\head}[1]{
%    \end{macrocode}
% First check, if the environment is active at the moment.
%    \begin{macrocode}
  \meth@checknotopen
%    \end{macrocode}
% If \cs{meth@where} is set to 0, the environment method is active.
%    \begin{macrocode}
  \ifnum\meth@where=0
    \def\meth@head{
%    \end{macrocode}
% The code to typeset the header.
% \changes{v1.3}{1997/06/29}{\cs{tt} replaced by \cs{texttt}.}
%    \begin{macrocode}
      {\setlength{\fboxrule}{0.2mm}%
       \fbox{\parbox{\meth@indent}{\hfill}
             \begin{minipage}{\meth@righttotwid}
               {\parindent-\meth@indent \texttt{#1}}
             \end{minipage}
            }}
    }
  \fi%
%    \end{macrocode}
% If \cs{meth@where} is set to 1, the environment data is active.
%    \begin{macrocode}
  \ifnum\meth@where=1
    \def\meth@head{
%    \end{macrocode}
% The code to typeset the header.
% \changes{v1.3}{1997/06/29}{\cs{tt} replaced by \cs{texttt}.}
%    \begin{macrocode}
      {\setlength{\fboxrule}{0.1mm}%
       \fbox{\hspace{2mm}\begin{minipage}{\meth@totwid}
         \texttt{#1}
       \end{minipage}}
      }
    }
  \fi%
}
%    \end{macrocode}
% \end{macro}
%
% \subsection{\cs{headtabbed}}
% \begin{macro}{\headtabbed}
% \cs{headtabbed} takes care of the first line, which will be
% formatted in the header of the method and defines the macro
% \cs{meth@headtabbed}, which does the output.
% \changes{v1.3}{1997/06/29}{\cs{tt} replaced by \cs{texttt}.}
%    \begin{macrocode}
\newcommand{\headtabbed}[1]{
  \meth@checknotopen
  \def\meth@headtabbed{
     \setlength{\fboxrule}{0.2mm}%
     \settowidth{\meth@namewid}{\texttt{#1}}%
     \setlength{\meth@rightnamewid}{\meth@totwid}
     \addtolength{\meth@rightnamewid}{-\meth@namewid}
     \setlength{\meth@nameindent}{\meth@namewid}
     \addtolength{\meth@nameindent}{2mm}
     \fbox{\parbox{\meth@nameindent}{\hfill}%
             \begin{minipage}{\meth@rightnamewid}
                \parindent-\meth@namewid
                \texttt{#1\meth@hpa
                  \meth@hpb
                  \meth@hpc
                  \meth@hpd
                  \meth@hpe
                  \meth@hpf
                  \meth@hpg
                  \meth@hph
                  \meth@hpi
                  \meth@hpj
                  \meth@hpk
                  \meth@hpl
                  \meth@hpm
                  \meth@hpn
                  \meth@hpo
                  \meth@hpp
                  \meth@hpq
                  \meth@hpr
                  \meth@hps
                  \meth@hpt
                  \meth@hpu
                  \meth@hpv
                  \meth@hpw
                  \meth@hpx
                  \meth@hpy
                  \meth@hpz}%
             \end{minipage}%
          }
  }
}
%    \end{macrocode}
% \end{macro}
%
% \subsection{\cs{headpara}}
% \begin{macro}{\meth@defheadpara}
% \changes{v1.5}{1997/01/19}{extend to 26 parameters}
% \cs{meth@defheadpara} searches for the first empty macro of
%  \cs{meth@hpa}, \ldots, \cs{meth@hpz}. In this macro the new line can
%  be saved. This macro is used by \cs{headpara}.
%    \begin{macrocode}
\newcommand{\meth@defheadpara}[1]{
  \ifcase\value{meth@headparanum}
    \def\meth@hpa{#1}  \or
    \def\meth@hpb{\\#1}  \or
    \def\meth@hpc{\\#1}  \or
    \def\meth@hpd{\\#1}  \or
    \def\meth@hpe{\\#1}  \or
    \def\meth@hpf{\\#1}  \or
    \def\meth@hpg{\\#1}  \or
    \def\meth@hph{\\#1}  \or
    \def\meth@hpi{\\#1}  \or
    \def\meth@hpj{\\#1}  \or
    \def\meth@hpk{\\#1}  \or
    \def\meth@hpl{\\#1}  \or
    \def\meth@hpm{\\#1}  \or
    \def\meth@hpn{\\#1}  \or
    \def\meth@hpo{\\#1}  \or
    \def\meth@hpp{\\#1}  \or
    \def\meth@hpq{\\#1}  \or
    \def\meth@hpr{\\#1}  \or
    \def\meth@hps{\\#1}  \or
    \def\meth@hpt{\\#1}  \or
    \def\meth@hpu{\\#1}  \or
    \def\meth@hpv{\\#1}  \or
    \def\meth@hpw{\\#1}  \or
    \def\meth@hpx{\\#1}  \or
    \def\meth@hpy{\\#1}  \or
    \def\meth@hpz{\\#1}  \or
    \PackageError{method}%
       {Too many parameters in method-environment !}{}
  \fi
  \stepcounter{meth@headparanum}
}
%    \end{macrocode}
% \end{macro}
%
% \begin{macro}{\meth@defheadpara}
% Now for the definition of the macro \cs{headpara}. Check first, if
% the correspondent environment is open. Then use the macro
% \cs{meth@defheadpara}.
%    \begin{macrocode}
\newcommand{\headpara}[1]{
  \meth@checknotopen
  \meth@defheadpara{#1}
}
%    \end{macrocode}
% \end{macro}
%
% \subsection{\cs{para}}
% \begin{macro}{\meth@defpara}
%    \begin{macrocode}
\newcommand{\meth@defpara}[1]{
  \ifcase\value{meth@paranum}
    \def\meth@pa{#1}  \or
    \def\meth@pb{#1}  \or
    \def\meth@pc{#1}  \or
    \def\meth@pd{#1}  \or
    \def\meth@pe{#1}  \or
    \def\meth@pf{#1}  \or
    \def\meth@pg{#1}  \or
    \def\meth@ph{#1}  \or
    \def\meth@pi{#1}  \or
    \def\meth@pj{#1}  \or
    \def\meth@pk{#1}  \or
    \def\meth@pl{#1}  \or
    \def\meth@pm{#1}  \or
    \def\meth@pn{#1}  \or
    \def\meth@po{#1}  \or
    \def\meth@pp{#1}  \or
    \def\meth@pq{#1}  \or
    \def\meth@pr{#1}  \or
    \def\meth@ps{#1}  \or
    \def\meth@pt{#1}  \or
    \def\meth@pu{#1}  \or
    \def\meth@pv{#1}  \or
    \def\meth@pw{#1}  \or
    \def\meth@px{#1}  \or
    \def\meth@py{#1}  \or
    \def\meth@pz{#1}  \or
    \PackageError{method}%
    {Too many parameters in method.sty-environment !}
  \fi
  \stepcounter{meth@paranum}
}
%    \end{macrocode}
% \end{macro}
%
% \begin{macro}{\para}
% Here the definition for \cs{para}:
% \changes{v1.3}{1997/06/29}{\cs{tt} replaced by \cs{texttt}.}
% \changes{v1.4}{1997/06/29}{insert braces, which were missing before}
%    \begin{macrocode}
\newcommand{\para}[2]{
  \meth@checknotopen
  \meth@defpara{\begin{list}{\texttt{#1}}{\meth@listdecl}
    \item #2    
    \end{list}}
}
%    \end{macrocode}
% \end{macro}
%
% \subsection{The other macros}
% \begin{macro}{\precond}
% The other macros are very simple. The create a list-environment and
% put their data inside of this list.
% \changes{v1.3}{1997/06/29}{\cs{bf} replaced by \cs{textbf}.}
% \changes{v1.4}{1997/06/29}{insert missing braces.}
%    \begin{macrocode}
\newcommand{\precond}[1]{
  \meth@checknotopen
  \def\meth@precond{\begin{list}{\textbf{\textprecond}}{\meth@listdecl}
    \item #1
    \end{list}}
}
%    \end{macrocode}
% \end{macro}
% \begin{macro}{\postcond}
% \changes{v1.3}{1997/06/29}{\cs{bf} replaced by \cs{textbf}.}
% \changes{v1.4}{1997/06/29}{insert missing braces.}
%    \begin{macrocode}
\newcommand{\postcond}[1]{
  \meth@checknotopen
  \def\meth@postcond{\begin{list}{\textbf{\textpostcond}}{\meth@listdecl}
      \item #1
    \end{list}}
}
%    \end{macrocode}
% \end{macro}
% \begin{macro}{\descr}
% \changes{v1.3}{1997/06/29}{\cs{bf} replaced by \cs{textbf}.}
% \changes{v1.4}{1997/06/29}{insert missing braces.}
%    \begin{macrocode}
\newcommand{\descr}[1]{
  \meth@checknotopen
  \def\meth@descr{\begin{list}{\textbf{\textdescr}}{\meth@listdecl}
      \item #1
    \end{list}}
}
%    \end{macrocode}
% \end{macro}
% \begin{macro}{\error}
% \changes{v1.3}{1997/06/29}{\cs{bf} replaced by \cs{textbf}.}
% \changes{v1.4}{1997/06/29}{insert missing braces.}
%    \begin{macrocode}
\newcommand{\error}[1]{
  \meth@checknotopen
  \def\meth@error{\begin{list}{\textbf{\texterror}}{\meth@listdecl}
      \item #1
    \end{list}}
}
%    \end{macrocode}
% \end{macro}
% \begin{macro}{\return}
% \changes{v1.3}{1997/06/29}{\cs{bf} replaced by \cs{textbf}.}
% \changes{v1.4}{1997/06/29}{insert missing braces.}
%    \begin{macrocode}
\newcommand{\return}[1]{
  \meth@checknotopen
  \def\meth@return{\begin{list}{\textbf{\textreturn}}{\meth@listdecl}
    \item #1
    \end{list}}
}
%    \end{macrocode}
% \end{macro}
% \begin{macro}{\see}
% \changes{v1.3}{1997/06/29}{\cs{bf} replaced by \cs{textbf}.}
% \changes{v1.4}{1997/06/29}{insert missing braces.}
%    \begin{macrocode}
\newcommand{\see}[1]{
  \meth@checknotopen
  \def\meth@see{\begin{list}{\textbf{\textsee}}{\meth@listdecl}
    \item #1
    \end{list}}
}
%    \end{macrocode}
% \end{macro}
% \begin{macro}{\init}
% \changes{v1.3}{1997/06/29}{\cs{bf} replaced by \cs{textbf}.}
% \changes{v1.4}{1997/06/29}{insert missing braces.}
%    \begin{macrocode}
\newcommand{\init}[1]{
  \meth@checknotopen
  \def\meth@init{\begin{list}{\textbf{\textinit}}{\meth@listdecl}
    \item #1
    \end{list}}
}
%    \end{macrocode}
% \end{macro}
% \begin{macro}{\del}
% \changes{v1.3}{1997/06/29}{\cs{bf} replaced by \cs{textbf}.}
% \changes{v1.4}{1997/06/29}{insert missing braces.}
%    \begin{macrocode}
\newcommand{\del}[1]{
  \meth@checknotopen
  \def\meth@del{\begin{list}{\textbf{\textdel}}{\meth@listdecl}
    \item #1
    \end{list}}
}
%    \end{macrocode}
%\end{macro}
% \Finale
%
\endinput
%
%
% \Finale
%
\endinput
