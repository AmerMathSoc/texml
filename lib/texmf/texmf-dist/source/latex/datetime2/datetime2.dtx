%\iffalse
% datetime2.dtx generated using makedtx version 1.1 (c) Nicola Talbot
% Command line args:
%   -doc "datetime2-manual.tex"
%   -section "chapter"
%   -author "Nicola Talbot"
%   -src "datetime2.sty\Z=>datetime2.sty"
%   -src "datetime2-calc.sty\Z=>datetime2-calc.sty"
%   datetime2
% Created on 2016/6/5 13:21
%\fi
%\iffalse
%<*package>
%% \CharacterTable
%%  {Upper-case    \A\B\C\D\E\F\G\H\I\J\K\L\M\N\O\P\Q\R\S\T\U\V\W\X\Y\Z
%%   Lower-case    \a\b\c\d\e\f\g\h\i\j\k\l\m\n\o\p\q\r\s\t\u\v\w\x\y\z
%%   Digits        \0\1\2\3\4\5\6\7\8\9
%%   Exclamation   \!     Double quote  \"     Hash (number) \#
%%   Dollar        \$     Percent       \%     Ampersand     \&
%%   Acute accent  \'     Left paren    \(     Right paren   \)
%%   Asterisk      \*     Plus          \+     Comma         \,
%%   Minus         \-     Point         \.     Solidus       \/
%%   Colon         \:     Semicolon     \;     Less than     \<
%%   Equals        \=     Greater than  \>     Question mark \?
%%   Commercial at \@     Left bracket  \[     Backslash     \\
%%   Right bracket \]     Circumflex    \^     Underscore    \_
%%   Grave accent  \`     Left brace    \{     Vertical bar  \|
%%   Right brace   \}     Tilde         \~}
%</package>
%\fi
% \iffalse
% Doc-Source file to use with LaTeX2e
% Copyright (C) 2016 Nicola Talbot, all rights reserved.
% \fi
% \iffalse
%<*driver>
\documentclass[widecs,report]{nlctdoc}

\usepackage{metalogo}
\usepackage{graphicx}
\usepackage[utf8]{inputenc}
\usepackage[T1]{fontenc}
\usepackage[calc,english,en-GB,useregional=false]{datetime2}
\usepackage[colorlinks,
            bookmarks,
            hyperindex=false,
            pdfauthor={Nicola L.C. Talbot},
            pdftitle={datetime2: date and time formats}]{hyperref}

\CheckSum{2374}
\RecordChanges
\DisableCrossrefs

\renewcommand*{\usage}[1]{\hyperpage{#1}}
\renewcommand*{\main}[1]{\hyperpage{#1}}
\IndexPrologue{\chapter*{\indexname}\markboth{\indexname}{\indexname}%
\addcontentsline{toc}{chapter}{Index}}
\setcounter{IndexColumns}{2}

\newcommand{\style}[1]{\texttt{#1}%
 \index{#1=\texttt{#1} style|main}%
 \index{styles:>#1=\texttt{#1}|main}}

\newcommand*{\module}[2][]{%
 \ifblank{#1}%
 {%
   \href{http://ctan.org/pkg/datetime2-#2}{\texttt{#2}}%
 }%
 {%
   \href{http://ctan.org/pkg/datetime2-#1}{\texttt{#2}}%
 }%
}

\doxitem{Option}{option}{options (definitions)}

\DTMsavetimestamp{gmt}{2015-03-01T15:35:09+00:00}
\DTMsavetimestamp{bst}{2015-04-01T08:55:39+01:00}

\newcommand*{\inpkg}[1]{\llap{\styfmt{#1}}}
\newcommand*{\oldcs}[1]{\cs{#1}%
 \index{datetime commands:=\styfmt{datetime} package commands>#1=\cs{#1}|main}%
}

\setlength\marginparwidth{70pt}

\begin{document}
\DocInput{datetime2.dtx}
\end{document}
%</driver>
%\fi
%
%\MakeShortVerb{"}
%
%\title{datetime2 v1.5.1:
%date and time formats}
%\author{Nicola L. C. Talbot\\\url{http://www.dickimaw-books.com/}}
%
%\date{2016-06-05}
%\maketitle
%
%\begin{abstract}
%The \styfmt{datetime2} package replaces the \styfmt{datetime}
%package. Languages and regional variations are dealt with by the
%\styfmt{datetime2} language modules which are independently
%maintained and installed. Make sure that when you install
%\styfmt{datetime2} you also install the required \styfmt{datetime2}
%language modules.
%\end{abstract}
%
%\tableofcontents
%
%\chapter{Introduction}
%\label{sec:intro}
%
%I~wrote the original \styfmt{datetime} package back in the 1990s as
%an alternative to the \sty{ukdate} package, which had dropped out of
%some of the TeX distributions, so it was designed specifically for
%UK date formats.\footnote{Of course, \texttt{ukdatetime} would've
%been a better name, but the 8 dot 3 filename restriction was a concern back
%then.} However some users found the time formats useful
%and the ability to save dates for later use, so when \sty{babel}
%came along I~had a number of requests to make \styfmt{datetime}
%compatible with \sty{babel} so that the regional date formats were
%preserved but the other \styfmt{datetime} functions could be used. 
%Then PDF\TeX\ came into existence and its \cs{pdfcreationdate} now
%provided a~way of obtaining the seconds and time zone, which can't
%be obtained from \TeX's \cs{time} primitive.  Over time, the
%continual updates to the package has started to put a~strain on the
%original na\"ive \LaTeX\ code of my first package, as it's been
%stretched well past its intended design.
%
%The other problem with \styfmt{datetime} is that some of the commands
%aren't expandable but some users want to be able to use them in
%expandable contexts (such as, for example, PDF bookmarks or writing
%a date stamp to an external file) or they
%want to be able to upper case the first letter if the date comes at
%the start of a~sentence. This isn't an issue in English, as the
%weekday and month names are proper nouns and so automatically start
%with an upper case letter, regardless of where they appear in
%a~sentence. Users who don't know the history and original purpose of
%the \styfmt{datetime} package are puzzled as to why the defaults are
%all UK English or some styles were hard-coded, and some users are
%confused as to the ordering of the day, month and year parameters.
%In addition, some command names were incompatible with other
%date-related packages, but renaming those commands would break
%compatibility with older documents.
%
%In order to address all these issues, a~replacement package is
%necessary. Your old documents that use \styfmt{datetime} should
%still be able to compile, but for your new documents, you may prefer
%the improved \styfmt{datetime2} package instead.
%
%\chapter{Example Usage}
%\label{sec:examples}
%
%\begin{verbatim}
%\documentclass{article}
%\usepackage{datetime2}
%\begin{document}
%This PDF was created on \today.
%\end{document}
%\end{verbatim}
%In the above example, the date is displayed in the form:
%\begin{display}
%\DTMsetstyle{default}\DTMusedate{gmt}
%\end{display}
%This is the \style{default} style.
%
%\begin{verbatim}
%\documentclass{article}
%\usepackage{datetime2}
%\begin{document}
%This PDF was created on \DTMnow.
%\end{document}
%\end{verbatim}
%In the above example, the full date, time and time zone is displayed
%in the form
%\begin{display}
%\DTMsetstyle{default}\DTMuse{gmt}
%\end{display}
%or
%\begin{display}
%\DTMsetstyle{default}\DTMuse{bst}
%\end{display}
%\emph{unless} you are using \XeLaTeX\ in which case the seconds and
%time zone are omitted. (\XeLaTeX\ doesn't provide this information.)
%Alternatively you can hide the seconds and zone
%using the package options \pkgopt[false]{showseconds} and
%\pkgopt[false]{showzone}. If you want UTC+0 to be displayed
%numerically instead of using a Z you can use the
%\pkgopt[false]{showisoZ} package option.
%
%\begin{verbatim}
%\documentclass[british]{article}
%\usepackage{babel}
%\usepackage{datetime2}
%\begin{document}
%This PDF was created on \DTMnow.
%\end{document}
%\end{verbatim}
%This has the same default numerical output as the previous example,
%but there are now two additional styles available: \style{en-GB} and
%\style{en-GB-numeric}. The \cs{datebritish} command provided by
%\sty{babel} is redefined to prevent \sty{babel} from overriding your
%preferred date style. The regional style can be enabled with the
%\pkgopt{useregional} option.
%
%\begin{verbatim}
%\documentclass[british]{article}
%\usepackage{babel}
%\usepackage[useregional]{datetime2}
%\begin{document}
%This PDF was created on \DTMnow.
%\end{document}
%\end{verbatim}
%The full date, time and zone are displayed in the form
%\begin{display}
%\DTMsetstyle{en-GB}\DTMuse{gmt}
%\end{display}
%or
%\begin{display}
%\DTMsetstyle{en-GB}\DTMuse{bst}
%\end{display}
%
%\begin{verbatim}
%\documentclass[british]{article}
%\usepackage{babel}
%\usepackage[useregional=numeric]{datetime2}
%\begin{document}
%This PDF was created on \DTMnow.
%\end{document}
%\end{verbatim}
%The full date, time and zone are displayed in the form
%\begin{display}
%\DTMsetstyle{en-GB-numeric}\DTMuse{gmt}
%\end{display}
%or
%\begin{display}
%\DTMsetstyle{en-GB-numeric}\DTMuse{bst}
%\end{display}
%
%\begin{verbatim}
%\documentclass[english]{article}
%\usepackage{babel}
%\usepackage[useregional=numeric]{datetime2}
%\begin{document}
%This PDF was created on \DTMnow.
%\end{document}
%\end{verbatim}
%The full date, time and zone are displayed in the form
%\begin{display}
%\DTMsetstyle{default}\DTMuse{gmt}
%\end{display}
%This is because no \emph{regional dialect} has been specified. The
%language name \texttt{english} is
%ambiguous, so the \style{default} style is used.
%
%\begin{verbatim}
%\documentclass[english]{article}
%\usepackage{babel}
%\usepackage[useregional]{datetime2}
%\begin{document}
%This PDF was created on \today.
%\end{document}
%\end{verbatim}
%The date is now displayed as
%\begin{display}
%\DTMsetstyle{english}\DTMusedate{gmt}
%\end{display}
%since that's \LaTeX's default behaviour for \cs{today}. For other
%languages, you should check the language module documentation to
%find out what happens when the region can't be determined from the
%language name.
%
%\begin{verbatim}
%\documentclass[british]{article}
%\usepackage{babel}
%\usepackage[useregional,showdow]{datetime2}
%
%\DTMlangsetup[en-GB]{abbr}
%
%\begin{document}
%This PDF was created on \today.
%\end{document}
%\end{verbatim}
%The date is now displayed as
%\begin{display}
%\DTMsetstyle{en-GB}\DTMlangsetup[en-GB]{abbr}\DTMsetup{showdow}%
%\DTMusedate{gmt}
%\end{display}
%The options used in \cs{DTMlangsetup} (such as \texttt{abbr}) are
%provided by the language modules and should be described in the
%module's documentation. Different languages may have different
%options so \texttt{abbr} may not be available for some of them. The
%\pkgopt{showdow} (show day of week) option is a package-wide option
%even though it's a language setting. (This is because the
%\styfmt{datetime2-calc} package is also needed if the day of the week
%should be displayed. As a package option, \pkgopt{showdow} can
%automatically load the required package.)
%
%\begin{important}
%You need to check the documentation for the relevant language module
%or package to find out which styles check
%the \pkgopt{showdow} setting as not all of them do. For example, the
%\module[english]{datetime2-english module} documentation indicates
%which of the English date styles support this setting.
%(The \styfmt{datetime} package had a similar limitation with its
%\pkgoptfmt{dayofweek} package option.)
%\end{important}
%
%If for some reason you can't load \sty{babel} before \styfmt{datetime2}
%and can only load it afterwards, then you need to explicitly load
%the module for each \sty{babel} dialect with:
%\begin{definition}[\DescribeMacro\DTMusemodule]
%\cs{DTMusemodule}\marg{language}\marg{module-name}
%\end{definition}
%where \meta{language} is the language (or dialect) name used with
%\sty{babel} (for example, \texttt{british}) and \meta{module-name} 
%is the name of the \styfmt{datetime2} module (for example, \texttt{en-GB}).
%
%Example:
%\begin{verbatim}
%\documentclass{article}
%
%\usepackage{datetime2}
%
%\usepackage[british,irish]{babel}
%
%\DTMusemodule{british}{en-GB}
%\DTMusemodule{irish}{ga-IE}
%
%\end{verbatim}
%The \meta{language} argument should match the
%\cs{date\meta{language}} command provided by \sty{babel}. Similarly
%for \sty{polyglossia}:
%\begin{verbatim}
%\documentclass{article}
%
%\usepackage{datetime2}
%
%\usepackage{polyglossia}
%\setmainlanguage[variant=uk]{english}
%\setotherlanguage{irish}
%
%\DTMusemodule{english}{en-GB}
%\DTMusemodule{irish}{ga-IE}
%
%\end{verbatim}
%This ensures that not only is the required \styfmt{datetime2}
%regional module loaded but also the date switching mechanism
%\cs{date\meta{language}} is modified to prevent \sty{babel} or
%\sty{polyglossia} from interfering with \styfmt{datetime2}.
%
%Remember that you need to switch on the regional style, if required:
%\begin{verbatim}
%\DTMsetup{useregional}
%\end{verbatim}
%
%\chapter{Displaying the Date and Time}
%\label{sec:display}
%
%A specific date can be displayed using:
%\begin{definition}[\DescribeMacro\DTMdisplaydate]
%\cs{DTMdisplaydate}\marg{year}\marg{month}\marg{day}\marg{dow}
%\end{definition}
%The format used to display the date is governed by the 
%\termdef{display style}.
%
%The arguments are all \emph{numerical}: \meta{year} is the year,
%\meta{month} is the month number (starting from~1 for January),
%\meta{day} is the day of the month and \meta{dow} is the day of the
%week number starting from~0 for Monday. The day of week number may
%be \texttt{-1}, which indicates that the style should ignore it.
%(Some styles always ignore the day of week, regardless of its
%value.) This command is intended for use in expandable contexts
%(such as writing the date to another file or using the date in the
%bookmarks) and is used by \cs{today}. The date styles must ensure
%that any fragile content is protected. (This is why \meta{dow}
%isn't an optional argument otherwise the command wouldn't be
%expandable.)
%
%\begin{important}
%The \meta{dow} argument must always be an integer from $-1$ to $6$.
%It should not be left blank or set to any other value. In some
%cases using an incorrect value may not cause a problem, but in other
%cases it will. So it's best to get into the habit of always setting
%it correctly.
%\end{important}
%
%If you want the \meta{dow} value automatically calculated from the date,
%you can use \cs{DTMdate} (described below) instead with the
%\pkgopt{showdow} package option. Note that \cs{DTMdate} is
%\emph{robust}. If you require an expandable alternative, the
%\meta{dow} value must be calculated first. The simplest way to do
%this is to first save the date and then use it (see
%\sectionref{sec:store}).
%
%Examples (with the \pkgopt{showdow} package option set):
%\begin{itemize}
%\item Ignore day of week:
%\begin{verbatim}
%\section{\DTMdisplaydate{2016}{2}{10}{-1}}
%\end{verbatim}
%This overrides the \pkgopt{showdow} option in this specific
%instance.
%
%\item Save the date first and then use it:
%\begin{verbatim}
%\DTMsavedate{mydate}{2016-02-10}
%\section{\DTMusedate{mydate}}
%\end{verbatim}
%(See \sectionref{sec:store}.)
%
%\item Another expandable alternative (but less convenient and more
%prone to error since the date has to be repeated):
%\begin{verbatim}
%\DTMcomputedayofweekindex{2016-02-10}{\dowindex}
%\section{\DTMdisplaydate{2016}{2}{10}{\dowindex}}
%\end{verbatim}
%(See \sectionref{sec:calc}.)
%
%\item Robust version won't work in PDF bookmarks or case-changing
%contexts (such as page headers):
%\begin{verbatim}
%\section{\DTMdate{2016-02-10}}
%\end{verbatim}
%Note that if there is a table of contents, this will mean that the
%day of week index has to be calculated \emph{twice}. Once in the
%table of contents and once in the actual section title. If the
%section title is also used in the page header, then the day of week
%index will additionally be calculated on every page that has this
%section title in the header.
%\end{itemize}
%(See the accompanying \texttt{datetime2-sample-journal.tex} sample
%file for more examples of using dates in section titles.)
%
%Some styles may start the date with a word (such as the day of the
%week name or the month name). In English, proper nouns are
%capitalised regardless of where they appear in a~sentence but some
%languages use lower case month or day of week names. In this event,
%if the initial letter needs to be capitalised then you can use:
%\begin{definition}[\DescribeMacro\DTMDisplaydate]
%\cs{DTMDisplaydate}\marg{year}\marg{month}\marg{day}\marg{dow}
%\end{definition}
%which is analogous to \cs{DTMdisplaydate}. Styles that are
%unaffected by this issue (for example, numerical or English dates)
%set \cs{DTMDisplaydate} to just \cs{DTMdisplaydate}. As with
%\cs{DTMdisplaydate} the style needs to ensure that any fragile
%content is protected in the event that \cs{DTMDisplaydate} is used
%in an expandable context. (Note that for this reason, I~don't
%recommend the use of the commands provided by the \sty{mfirstuc}
%package as they're not expandable.)
%
%If you want the \meta{dow} value automatically calculated from the date,
%you can use \cs{DTMDate} (described below) instead with the
%\pkgopt{showdow} package option.
%
%The current date is displayed using
%\begin{definition}[\DescribeMacro\today]
%\cs{today}
%\end{definition}
%This uses \cs{DTMdisplaydate} to format the date so it will match
%the currently selected date style. There's also a first letter
%upper case version that uses \cs{DTMDisplaydate}:
%\begin{definition}[\DescribeMacro\Today]
%\cs{Today}
%\end{definition}
%
%Since there are other classes and packages that redefine \cs{today},
%as from version 1.4, the \styfmt{datetime2} package provides
%\begin{definition}[\DescribeMacro\DTMtoday]
%\cs{DTMtoday}
%\end{definition}
%and
%\begin{definition}[\DescribeMacro\DTMToday]
%\cs{DTMToday}
%\end{definition}
%The package now assigns \cs{today} and \cs{Today} to \cs{DTMtoday}
%and \cs{DTMToday}, respectively. If your document loads another
%package or class that modifies \cs{today} at the beginning of the
%document, you can switch it back to \styfmt{datetime2}'s definition
%using
%\begin{verbatim}
%\let\today\DTMtoday
%\end{verbatim}
%after the start of the document or use the \cs{AtBeginDocument}
%hook:
%\begin{verbatim}
%\AtBeginDocument{\let\today\DTMtoday}
%\end{verbatim}
%
%\begin{important}
%If you use \sty{babel} or \sty{polyglossia} you must make sure 
%you have the relevant \styfmt{datetime2} language modules installed.
%(See \sectionref{sec:lang}.) You also need to make sure that
%\styfmt{datetime2} is loaded \emph{after}
%\sty{babel}\slash\sty{polyglossia} otherwise \cs{today} will be
%redefined so that it no longer uses \cs{DTMdisplaydate}.
%\end{important}
%
%As mentioned above, \emph{some styles} allow the day of the week to be
%displayed. This requires the \styfmt{datetime2-calc} package which will
%automatically be loaded if you set \pkgopt{showdow} in the
%\styfmt{datetime2} package option list or if you set
%\pkgopt{showdow} in \ics{DTMsetup} \emph{in the preamble}. The
%package option \pkgopt{calc} will also load \styfmt{datetime2-calc} or
%you can load it explicitly using \cs{usepackage} after
%\styfmt{datetime2} has been loaded. (You may use \pkgopt[true]{showdow} in
%the \env{document} environment if the \styfmt{datetime2-calc}
%package has been loaded in the preamble either explicitly or through
%the \pkgopt{calc} option.)
%
%When \styfmt{datetime2-calc} is loaded, it computes the current day of
%the week (using commands provided by the \sty{pgfcalendar} package)
%which can then be used by \cs{today} or \cs{Today}. If 
%\styfmt{datetime2-calc} isn't loaded then neither \cs{today} nor
%\cs{Today} will display the day of the week, regardless of the
%current style.
%
%If you would like a more convenient syntax and don't care about
%expansion, there is also a~robust \emph{non-expandable} command that
%can be used to display a~particular date:
%\begin{definition}[\DescribeMacro\DTMdate]
%\cs{DTMdate}\marg{date}
%\end{definition}
%As before there's also a capitalised version:
%\begin{definition}[\DescribeMacro\DTMDate]
%\cs{DTMDate}\marg{date}
%\end{definition}
%In these cases the date should be provided as
%\meta{YYYY}\texttt{-}\meta{MM}\texttt{-}\meta{DD} in the argument
%\meta{date}. For example:
%\begin{verbatim}
%\DTMdate{2015-03-23}
%\end{verbatim}
%
%Note that hyphens must always be used, regardless of
%the separator options. Take care that the category code of the
%hyphen hasn't changed when you use this syntax.
%
%\begin{important}
%The year \meta{YYYY} can't be negative\footnotemark\ in \cs{DTMdate} or
%\cs{DTMDate}. Use \cs{DTMdisplaydate} or 
%\cs{DTMDisplaydate} instead.
%\end{important}\footnotetext{Well, actually it can if you put it in braces
%and don't use \styfmt{datetime2-calc}.}
%
%These commands internally use \cs{DTMdisplaydate} and
%\cs{DTMDisplaydate}, respectively. If the \styfmt{datetime2-calc}
%package has been loaded, the day of the week will be computed,
%otherwise the day of the week will be set to \texttt{-1}. Another
%benefit of the \styfmt{datetime2-calc} package is that it allows
%additional formats permitted by the \sty{pgfcalendar} package:
%\begin{itemize}
%\item \meta{YYYY}\texttt{-}\meta{MM}\texttt{-last} (the last day of
%the given month).
%
%\item
%\meta{YYYY}\texttt{-}\meta{MM}\texttt{-}\meta{DD}\texttt{+-}\meta{n}
%(\meta{n} days before the given date).
%
%\item
%\meta{YYYY}\texttt{-}\meta{MM}\texttt{-}\meta{DD}\texttt{+}\meta{n}
%(\meta{n} days after the given date).
%
%\item \meta{YYYY}\texttt{-}\meta{MM}\texttt{-last+-}\meta{n}
%(\meta{n} days before the last day of
%the given month).
%
%\item \meta{YYYY}\texttt{-}\meta{MM}\texttt{-last+}\meta{n}
%(\meta{n} days after the last day of
%the given month).
%
%\end{itemize}
%See the \sty{pgfcalendar} package for further details.
%
%If you want to be able to use a date in an expandable context that
%can perform these calculations, consider first saving the date using
%one of the commands described in \sectionref{sec:store} and then use
%one of the expandable commands such as \cs{DTMuse} to display the
%date.
%
%An error or unexpected results may occur if you try using one of
%these extended formats without loading the \styfmt{datetime2-calc}
%package. An example that only works with \styfmt{datetime2-calc}:
%\begin{verbatim}
%\DTMdate{2015-03-last}
%\end{verbatim}
%An example that works with or without \styfmt{datetime2-calc}:
%\begin{verbatim}
%\DTMdate{2015-03-31}
%\end{verbatim}
%In this second case, you'll only notice a difference in the output
%if the style should show the day of the week.
%
%The style of the date is the same as for \cs{DTMdisplaydate} and
%\cs{DTMDisplaydate} (which \cs{DTMdate} and \cs{DTMDate} internally
%use, as mentioned above).
%
%A time can be displayed using
%\begin{definition}[\DescribeMacro\DTMdisplaytime]
%\cs{DTMdisplaytime}\marg{hour}\marg{minute}\marg{sec}
%\end{definition}
%where the arguments are all numerical (using 24 hours). The 
%\termdef{time style} currently in effect determines how the time is formatted.
%The command is designed to be used in an expandable context so the
%styles should take care to protect any fragile commands.
%
%Note that this command doesn't display the time zone. To display the
%time zone, you need to use
%\begin{definition}[\DescribeMacro\DTMdisplayzone]
%\cs{DTMdisplayzone}\marg{TZh}\marg{TZm}
%\end{definition}
%where \meta{TZh} is the hour offset and \meta{TZm} is the minute
%offset. The display is governed by the \termdef{zone style}.
%Again, the style should protect any fragile commands in case this is
%used in an expandable context.
%
%The current time (as set at the start of the document build)
%can be displayed using
%\begin{definition}[\DescribeMacro\DTMcurrenttime]
%\cs{DTMcurrenttime}
%\end{definition}
%This internally just uses \cs{DTMdisplaytime} and so is designed for
%use in an expandable context.
%
%The current zone can be displayed using
%\begin{definition}[\DescribeMacro\DTMcurrentzone]
%\cs{DTMcurrentzone}
%\end{definition}
%This internally just uses \cs{DTMdisplayzone} and so is designed for
%use in an expandable context.
%
%\begin{important}
%If the PDF\TeX\ primitive \ics{pdfcreationdate} is defined, the
%current time information is obtained from that, which includes the
%seconds and time zone. Lua\TeX\ also defines this command but
%\XeTeX\ doesn't, and in that case the only way to determine the
%current time is from \TeX's \ics{time} primitive which only contains
%the number of minutes since midnight, which means that the seconds
%and time zone are unavailable. Therefore if \XeTeX\ is used, the
%\pkgopt{showseconds} and \pkgopt{showzone} options are automatically
%switched off.
%\end{important}
%
%There is also a~non-expandable robust command to display the time:
%\begin{definition}[\DescribeMacro\DTMtime]
%\cs{DTMtime}\marg{tm}
%\end{definition}
%where \meta{tm} must be in the 24 hour format 
%\meta{hh}\texttt{:}\meta{mm}\texttt{:}\meta{ss}
%(colon-separated numerical arguments). Take care if you use \sty{babel}
%with a~language setting that makes the colon character active. You
%will have to switch off the shorthands in order to use this command
%correctly.
%
%The full date, time and zone (if available) can be displayed using
%\begin{definition}[\DescribeMacro\DTMdisplay]
%\cs{DTMdisplay}\marg{year}\marg{month}\marg{day}\marg{dow}\marg{hh}\marg{mm}\marg{ss}\marg{TZh}\marg{TZm}
%\end{definition}
%The arguments are all numerical.  The way the information is
%displayed in the document is governed by the \termdef{full style}
%(or \termdef{date-time style}).
%Typically the full style will redefine this command to use 
%\cs{DTMdisplaydate}, \cs{DTMdisplaytime} and (optionally)
%\cs{DTMdisplayzone}. The \pkgopt{showzone} setting may govern whether
%or not to display the time zone (although a~style may ignore this
%setting). The separators between the date and time and between the time
%and zone are governed by the style.
%
%There is also an analogous version if capitalisation is required:
%\begin{definition}[\DescribeMacro\DTMDisplay]
%\cs{DTMDisplay}\marg{year}\marg{month}\marg{day}\marg{dow}\marg{hh}\marg{mm}\marg{ss}\marg{TZh}\marg{TZm}
%\end{definition}
%Some styles may simply make this equivalent to \cs{DTMdisplay}.
%Other styles may use a~similar format to \cs{DTMdisplay} but replace
%\cs{DTMdisplaydate} with \cs{DTMDisplaydate}.
%
%The full current date, time and (optionally) zone can be displayed
%using:
%\begin{definition}[\DescribeMacro\DTMnow]
%\cs{DTMnow}
%\end{definition}
%which uses \cs{DTMdisplay} or
%\begin{definition}[\DescribeMacro\DTMNow]
%\cs{DTMNow}
%\end{definition}
%which uses \cs{DTMDisplay}.
%
%\chapter{Storing and Using Dates and Times}
%\label{sec:store}
%
%Date, time and zone information can be saved for later use. Note
%that the information is always saved numerically. The style is only
%applied when the information is later used. The commands that save
%the information are robust and not expandable. The commands that use the
%data are typically expandable although there may be some exceptions. 
%Take care that the colon (\texttt{:}) and hyphen (\texttt{-})
%characters haven't had their normal category code changed. (For
%example, through \sty{babel}'s shortcuts.) In the commands below,
%the \meta{name} (no active characters) is a name that uniquely
%identifies the information.
%
%Dates are saved using
%\begin{definition}[\DescribeMacro\DTMsavedate]
%\cs{DTMsavedate}\marg{name}\marg{date}
%\end{definition}
%where \meta{date} is in the same format as for \cs{DTMdate}.
%As with \cs{DTMdate} (and \cs{DTMDate}) the format can be extended
%with the \styfmt{datetime2-calc} package. If you want to access the day
%of week, you must make sure that \styfmt{datetime2-calc} has been
%loaded \emph{before you save the date}. (Remember that the \pkgopt{calc}
%and \pkgopt{showdow} package options will automatically load
%\styfmt{datetime2-calc}.) If \styfmt{datetime2-calc} has been loaded, the
%day of week number will be calculated and saved. Whether or not it
%is displayed in the document when the date is later used depends on
%the settings when the date is displayed not when it's saved.
%
%This command will override any previously defined date saved with
%this \meta{name}. If a time or zone hasn't been defined with this
%\meta{name}, the time and zone elements will all be set to 0
%otherwise they will remain unchanged.
%
%Note that you can't have a negative year in \meta{date}. There's an
%alternative command you can use instead that doesn't try parsing
%\meta{date}:
%\begin{definition}[\DescribeMacro\DTMsavenoparsedate]
%\cs{DTMsavenoparsedate}\marg{name}\marg{YYYY}\marg{MM}\marg{DD}\marg{dow}
%\end{definition}
%The day of week \meta{dow} may be \texttt{-1} if unknown. This
%command doesn't calculate the day of week, even if
%\styfmt{datetime2-calc} has been loaded.
%
%Times are saved using
%\begin{definition}[\DescribeMacro\DTMsavetime]
%\cs{DTMsavetime}\marg{name}\marg{time}
%\end{definition}
%where the \meta{time} is in the same format as for \cs{DTMtime}.
%
%This command will override any previously defined time saved with
%this \meta{name}. If a date or
%zone hasn't been defined with this \meta{name}, the date and zone
%elements will all be set to 0 (or \texttt{-1} for the day of week) otherwise they will remain unchanged.
%
%Times and zone are saved using
%\begin{definition}[\DescribeMacro\DTMsavetimezn]
%\cs{DTMsavetimezn}\marg{name}\marg{time and zone}
%\end{definition}
%where the \meta{time and zone} is in the form
%\begin{display}
%\meta{hh}:\meta{mm}:\meta{ss} \meta{TZh}:\meta{TZm}
%\end{display}
%(Note the space between the seconds and the hour offset.)
%
%This command will override any previously defined time and zone
%saved with this \meta{name}. If a date hasn't been defined with this
%\meta{name}, the year, month and day will be set to zero and the day
%of the week to \texttt{-1} otherwise they will remain unchanged.
%
%All date, time and zone information can be saved at the same time
%using:
%\begin{definition}[\DescribeMacro\DTMsavetimestamp]
%\cs{DTMsavetimestamp}\marg{name}\marg{data}
%\end{definition}
%where \meta{data} is in the format:
%\begin{display}
%\meta{YYYY}-\meta{MM}-\meta{DD}T\meta{hh}:\meta{mm}:\meta{ss}\meta{zone}
%\end{display}
%The \meta{zone} may either be \texttt{Z} or in the form
%\meta{TZh}\texttt{:}\meta{TZm} (for example, \texttt{-03:00}
%or \texttt{-3:0}). This will override any date, time or zone data
%previously saved with this \meta{name}.
%
%The current date and time can be saved using:
%\begin{definition}[\DescribeMacro\DTMsavenow]
%\cs{DTMsavenow}\marg{name}
%\end{definition}
%
%There is also a command that can be used to save the modification
%date of a file, but it's not available for some \TeX\ engines:
%\begin{definition}[\DescribeMacro\DTMsavefilemoddate]
%\cs{DTMsavefilemoddate}\marg{name}\marg{file name}
%\end{definition}
%where \meta{file name} is the name of the file (remember to use
%forward slashes \verb|/| for the directory divider). If you build
%your document using PDF\LaTeX, this command will use the PDF\TeX\
%primitive \cs{pdffilemoddate}. If you use Lua\TeX\ this command will
%attempt to use \texttt{os.date} but it uses \texttt{\%z} for the
%time zone, which may not work on some operating systems. If you use
%\XeLaTeX\ this command will generate a warning and will assume a
%date of 0000-00-00T00:00:00Z.
%
%The above commands are all localised to the current scope. If the
%data is required after the end of the scope, you can make the
%assignments global using:
%\begin{definition}[\DescribeMacro\DTMmakeglobal]
%\cs{DTMmakeglobal}\marg{name}
%\end{definition}
%For example:
%\begin{verbatim}
%\DTMsavenow{mydate}\DTMmakeglobal{mydate}
%\end{verbatim}
%
%A previously saved date can be displayed using the current style
%with
%\begin{definition}[\DescribeMacro\DTMusedate]
%\cs{DTMusedate}\marg{name}
%\end{definition}
%This just uses \cs{DTMdisplaydate}. An error will occur if
%\meta{name} hasn't been defined. Alternatively for the capitalised
%version:
%\begin{definition}[\DescribeMacro\DTMUsedate]
%\cs{DTMUsedate}\marg{name}
%\end{definition}
%which uses \cs{DTMDisplaydate} instead.
%
%A previously saved time can be displayed using the current style
%with
%\begin{definition}[\DescribeMacro\DTMusetime]
%\cs{DTMusetime}\marg{name}
%\end{definition}
%This just uses \cs{DTMdisplaytime}. An error will occur if
%\meta{name} hasn't been defined.
%
%A previously saved zone can be displayed using the current style
%with
%\begin{definition}[\DescribeMacro\DTMusezone]
%\cs{DTMusezone}\marg{name}
%\end{definition}
%This just uses \cs{DTMdisplayzone}. An error will occur if
%\meta{name} hasn't been defined.
%
%The entire date, time and zone can be displayed in the current style
%with
%\begin{definition}[\DescribeMacro\DTMuse]
%\cs{DTMuse}\marg{name}
%\end{definition}
%This uses \cs{DTMdisplay}. An error will occur if
%\meta{name} hasn't been defined. Alternatively,
%\begin{definition}[\DescribeMacro\DTMUse]
%\cs{DTMUse}\marg{name}
%\end{definition}
%will use \cs{DTMDisplay} instead.
%
%You can determine if a given \meta{name} has been defined
%using
%\begin{definition}[\DescribeMacro\DTMifsaveddate]
%\cs{DTMifsaveddate}\marg{name}\marg{true}\marg{false}
%\end{definition}
%
%The individual numerical elements can be fetched using one of the
%following commands. These don't check if the given data identified
%by \meta{name} has been defined and will expand to \cs{relax} if the
%name isn't recognised.
%
%\begin{definition}[\DescribeMacro\DTMfetchyear]
%\cs{DTMfetchyear}\marg{name}
%\end{definition}
%This expands to the year.
%
%\begin{definition}[\DescribeMacro\DTMfetchmonth]
%\cs{DTMfetchmonth}\marg{name}
%\end{definition}
%This expands to the month number.
%
%\begin{definition}[\DescribeMacro\DTMfetchday]
%\cs{DTMfetchday}\marg{name}
%\end{definition}
%This expands to the day of the month.
%
%\begin{definition}[\DescribeMacro\DTMfetchdow]
%\cs{DTMfetchdow}\marg{name}
%\end{definition}
%This expands to the day of the week number (\texttt{-1} if unknown).
%
%\begin{definition}[\DescribeMacro\DTMfetchhour]
%\cs{DTMfetchhour}\marg{name}
%\end{definition}
%This expands to the hour.
%
%\begin{definition}[\DescribeMacro\DTMfetchminute]
%\cs{DTMfetchminute}\marg{name}
%\end{definition}
%This expands to the minute.
%
%\begin{definition}[\DescribeMacro\DTMfetchsecond]
%\cs{DTMfetchsecond}\marg{name}
%\end{definition}
%This expands to the second.
%
%\begin{definition}[\DescribeMacro\DTMfetchTZhour]
%\cs{DTMfetchTZhour}\marg{name}
%\end{definition}
%This expands to the hour offset.
%
%\begin{definition}[\DescribeMacro\DTMfetchTZminute]
%\cs{DTMfetchTZminute}\marg{name}
%\end{definition}
%This expands to the minute offset.
%
%\chapter{Styles}
%\label{sec:styles}
%
%If you want to just change the \term{date style} use:
%\begin{definition}[\DescribeMacro\DTMsetdatestyle]
%\cs{DTMsetdatestyle}\marg{name}
%\end{definition}
%where \meta{name} identifies the style. For example:
%\begin{verbatim}
%\DTMsetdatestyle{iso}
%\end{verbatim}
%This will just change the date style (\cs{DTMdisplaydate} and
%\cs{DTMDisplaydate}), not the time or zone styles. Note that
%\cs{DTMdisplay} typically uses \cs{DTMdisplaydate} so this will also
%change the date element of \cs{DTMdisplay}.
%
%If you want to just change the \term{time style} use:
%\begin{definition}[\DescribeMacro\DTMsettimestyle]
%\cs{DTMsettimestyle}\marg{name}
%\end{definition}
%where \meta{name} identifies the style. For example:
%\begin{verbatim}
%\DTMsettimestyle{iso}
%\end{verbatim}
%This will just change the time style (\cs{DTMdisplaytime}), not the
%date or zone styles. Note that \cs{DTMdisplay} typically uses
%\cs{DTMdisplaytime} so this will also change the time element of
%\cs{DTMdisplay}.
%
%If you want to just change the \term{zone style} use:
%\begin{definition}[\DescribeMacro\DTMsetzonestyle]
%\cs{DTMsetzonestyle}\marg{name}
%\end{definition}
%where \meta{name} identifies the style. For example:
%\begin{verbatim}
%\DTMsetzonestyle{iso}
%\end{verbatim}
%This will just change the zone style (\cs{DTMdisplayzone}), not the
%date or time styles. Note that \cs{DTMdisplay} typically uses
%\cs{DTMdisplayzone} so this will also change the zone element of
%\cs{DTMdisplay}.
%
%If you want to change the \term{full style} use:
%\begin{definition}[\DescribeMacro\DTMsetstyle]
%\cs{DTMsetstyle}\marg{name}
%\end{definition}
%where \meta{name} identifies the style. For example:
%\begin{verbatim}
%\DTMsetstyle{iso}
%\end{verbatim}
%
%Note that in this case this does more than simply
%\begin{verbatim}
%\DTMsetdatestyle{iso}\DTMsettimestyle{iso}\DTMsetzonestyle{iso}
%\end{verbatim}
%as it also changes \cs{DTMdisplay} and \cs{DTMDisplay}.
%If the style \meta{name} is only a \term{partial style}, a~warning
%will be issued for any partial styles that aren't defined for the
%given name as well as a warning for the undefined full style.
%An error will occur if there are neither partial nor full styles
%with the given \meta{name}.
%
%The predefined styles listed in \sectionref{sec:predefinedfull} are all
%\emph{full styles}. This means that they change the date, time, zone
%and full format, so any of them can be used in \cs{DTMsetdatestyle},
%\cs{DTMsettimestyle}, \cs{DTMsetzonestyle} or \cs{DTMsetstyle}.
%However it's possible for a style to be only a \emph{partial style},
%such as those described in \sectionref{sec:predefinedtime}.
%
%For example, if \texttt{foo} is a \term{date style} and
%a \term{time style} but isn't a \term{zone style} or 
%a \term{full style} then you can use
%\begin{verbatim}
%\DTMsetdatestyle{foo}
%\end{verbatim}
%and
%\begin{verbatim}
%\DTMsettimestyle{foo}
%\end{verbatim}
%but you can't use \cs{DTMsetzonestyle}. You can use
%\begin{verbatim}
%\DTMsetstyle{foo}
%\end{verbatim}
%but this will now only be equivalent to
%\begin{verbatim}
%\DTMsetdatestyle{foo}\DTMsettimestyle{foo}
%\end{verbatim}
%and while \cs{DTMdisplay} and \cs{DTMDisplay} will typically use these date
%and time settings, the way that the date, time and zone are arranged
%will be governed by the full style setting that was already in
%effect before the date and time style changed.
%
%The style changes are all local and so are affected by the current
%scope.
%
%\section{Predefined Styles}
%\label{sec:predefinedstyles}
%
%The base \styfmt{datetime2} package provides a number of predefined
%numerical styles. \sectionref{sec:predefinedfull} lists the full styles,
%which can be used with \cs{DTMsetstyle}, \cs{DTMsetdatestyle},
%\cs{DTMsettimestyle} and \cs{DTMsetzonestyle}.
%\sectionref{sec:predefinedtime} lists the predefined (partial) times
%styles, which can be used with \cs{DTMsettimestyle} and
%\cs{DTMsetstyle}.
%
%\subsection{Full Styles}
%\label{sec:predefinedfull}
%
%The following are predefined full styles that are provided by the
%base \styfmt{datetime2} package.  Additional styles are available
%through the language modules (see \sectionref{sec:lang}).
%
%\begin{description}
%\item[\style{default}] The \style{default} style displays the date in the form
%\begin{display}
%\meta{YYYY}\meta{YMsep}\meta{MM}\meta{MDsep}\meta{DD}
%\end{display}
%where the month \meta{MM} and day of the month \meta{DD} numbers are
%formatted as two digits. The separators \meta{YMsep} and
%\meta{MDsep} default to a hyphen but can be changed using the
%options \pkgopt{yearmonthsep}, \pkgopt{monthdaysep} or
%\pkgopt{datesep} (either through the package options or using
%\ics{DTMsetup}).
%
%The time is displayed in the form:
%\begin{display}
%\meta{hh}\meta{HMsep}\meta{mm}\meta{MSsep}\meta{ss}
%\end{display}
%where the hour, month and seconds are formatted as two digits. The
%final \meta{MSsep}\meta{ss} is omitted if the option
%\pkgopt{showseconds} has been set to \texttt{false}. The separators
%\meta{HMsep} and \meta{MSsep} default to a~colon (\texttt{:}) but
%these may be changed using the options \pkgopt{hourminsep},
%\pkgopt{minsecsep} or \pkgopt{datetimesep}.
%
%The zone is displayed in form
%\begin{display}
%\meta{TZh}\meta{HMsep}\meta{TZm}
%\end{display}
%or just \texttt{Z} if the option \pkgopt{showisoZ} is set to
%\texttt{true} and both \meta{TZh} and \meta{TZm} are zero.
%The separator \meta{HMsep} is the same as used for the time format.
%The final \meta{HMsep}\meta{TZm} is omitted if the option
%\pkgopt{showzoneminutes} is set to \texttt{false}. The hour offset
%\meta{TZh} is formatted as two digits proceeded by either \texttt{+}
%or \texttt{-} and the minute offset is formatted as two digits.
%Note that since one of the main purposes of this package is to
%provide expandable date commands that can be used to write
%information to external files, no attempt is made to convert the
%hyphen \texttt{-} (for negative offsets) into a minus sign. If you
%want it rendered correctly in your document, consider placing the
%time zone command in math mode and adjust the separators as
%necessary.
%
%The full style is in the form
%\begin{display}
%\meta{date}\meta{DTsep}\meta{time}\meta{TZsep}\meta{zone}
%\end{display}
%The \meta{date}\meta{DTsep} part is omitted if the option
%\pkgopt{showdate} is set to \texttt{false}, and
%the \meta{TZsep}\meta{zone} part is omitted if the option
%\pkgopt{showzone} is set to \texttt{false}. The separator between
%the date and time \meta{DTsep} defaults to \cs{space} but may be
%changed using the \pkgopt{datetimesep} option. The separator between
%the time and zone \meta{TZsep} defaults to nothing but may be
%changed using the \pkgopt{timezonesep} option.
%
%\item[\style{iso}] The \style{iso} style is like the \style{default}
%style but the separators can't be changed. The separators used in
%the date format are fixed as hyphens and the separators used in the
%time and zone formats are fixed as colons. In the full format, the separator between the
%date and time is fixed as \texttt{T} and there's no separator between the
%time and zone. The only options that can change the \style{iso}
%style are \pkgopt{showseconds}, \pkgopt{showdate}, \pkgopt{showzone},
%\pkgopt{showzoneminutes} and \pkgopt{showisoZ}.
%
%\item[\style{yyyymd}] This is like the \style{default} style
%except that the month and date aren't forced into a two-digit
%format.
%
%\item[\style{ddmmyyyy}] This is like the \style{default} style
%except that the date is formatted in the reverse order
%\begin{display}
%\meta{DD}\meta{MDsep}\meta{MM}\meta{YMsep}\meta{YYYY}
%\end{display}
%The day and month are displayed as two-digits and the separators are
%as for the \style{default} style. The options that modify the
%\style{default} style similarly modify this style.
%
%\item[\style{dmyyyy}] This is like the \style{ddmmyyyy} style except
%that it doesn't force the day and month into a~two-digit format.
%The options that modify the
%\style{default} style similarly modify this style.
%
%\item[\style{dmyy}] This is like the \style{dmyy} style except
%that it only displays the final two digits of the year.
%The options that modify the
%\style{default} style similarly modify this style.
%
%\item[\style{ddmmyy}] This is like the \style{default} style
%except that the date is formatted in the reverse order
%\begin{display}
%\meta{DD}\meta{MDsep}\meta{MM}\meta{YMsep}\meta{YY}
%\end{display}
%The day, month and year are displayed as two-digits and the separators are
%as for the \style{default} style. The options that modify the
%\style{default} style similarly modify this style.
%
%\item[\style{mmddyyyy}] This is like the \style{ddmmyyyy} style
%except the day and month numbers are reversed. The separator between
%the month and day is still given by the \pkgopt{monthdaysep} or 
%\pkgopt{datesep} options. The separator between the day and year is given by
%the \pkgopt{dayyearsep} or \pkgopt{datesep} options.
%
%\item[\style{mmddyy}] This is like the \style{ddmmyy} style
%except the day and month numbers are reversed. The separator between
%the month and day is still given by the \pkgopt{monthdaysep} or 
%\pkgopt{datesep} options. The separator between the day and year is given by
%the \pkgopt{dayyearsep} or \pkgopt{datesep} options.
%
%\item[\style{mdyyyy}] This is like the \style{mmddyyyy} style except
%that it doesn't force the day and month into a~two-digit format.
%
%\item[\style{mdyy}] This is like the \style{mdyyyy} style except
%that the year only has the final two digits displayed.
%
%\item[\style{pdf}] This formats the date, time and zone so that the
%full style is in the form required by the date settings in
%\ics{pdfinfo}. The date format is
%\begin{display}
%D:\meta{YYYY}\meta{MM}\meta{DD}
%\end{display}
%where the month and day numbers are displayed as two digits.
%
%The time format is
%\begin{display}
%\meta{hh}\meta{mm}\meta{ss}
%\end{display}
%where the numbers are displayed as two digits.
%
%The zone format is
%\begin{display}
%\meta{hh}'\meta{mm}'
%\end{display}
%or Z for zero time offset if the option \pkgopt{showisoZ} is
%used. (The \pkgopt{showisoZ} option is the only option that modifies
%the \style{pdf} style.) The hour and minutes are displayed as two
%digits where the hour has the sign present (either \texttt{+} or
%\texttt{-}).
%
%The full style is a concatenation of the date, time and zone.
%\begin{display}
%D:\meta{YYYY}\meta{MM}\meta{DD}\meta{hh}\meta{mm}\meta{ss}\meta{hh}'\meta{mm}'
%\end{display}
%
%\end{description}
%
%\subsection{Time Styles}
%\label{sec:predefinedtime}
%
%There's only one predefined time (partial) style provided
%by the base \styfmt{datetime2} package. This style can be used to
%override the time format part of full styles. For example, to use
%the \style{default} full style with the \style{hmmss} time style:
%\begin{verbatim}
%\DTMsetstyle{default}\DTMsettimestyle{hmmss}
%\end{verbatim}
%
%\begin{description}
%\item[\style{hmmss}] The \style{hmmss} style is like the time style
%provided by the full \style{default} style except that the hour
%isn't forced into two digits.
%
%\end{description}
%
%\subsection{Zone Styles}
%\label{sec:predefinedzone}
%
%The following are predefined zone (partial) styles that are provided
%by the base \styfmt{datetime2} package. These styles can be used to
%override the zone format part of full styles. For example, to use
%the \style{default} full style with the \style{map} zone style:
%\begin{verbatim}
%\DTMsetstyle{default}\DTMsetzonestyle{map}
%\end{verbatim}
%
%\begin{description}
%\item[\style{map}] The \style{map} style uses
%\cs{DTMusezonemapordefault} to display the mapping, if one exists,
%or use the default style, if a mapping doesn't exist. For example:
%\begin{verbatim}
%\DTMNatoZoneMaps
%\DTMsetzonestyle{map}
%\end{verbatim}
%This first defines the NATO mappings and then switches to the
%\style{map} style.
%
%\item[\style{hhmm}] The \style{hhmm} style displays the time zone in
%the form
%\begin{display}
%\meta{TZh}\meta{HMsep}\meta{TZm}
%\end{display}
%where \meta{HMsep} is given by the \pkgopt{hourminsep} option.
%This style honours the \pkgopt{showzoneminutes} option
%but ignores the \pkgopt{showisoZ} option. The hour is always
%prefixed by the sign.
%
%\end{description}
%
%\section{Defining New Styles}
%\label{sec:newstyle}
%
%A new \term{date style} can be defined using:
%\begin{definition}[\DescribeMacro\DTMnewdatestyle]
%\cs{DTMnewdatestyle}\marg{name}\marg{definition}
%\end{definition}
%This defines a \term{partial style} that should only modify 
%\cs{DTMdisplaydate} and \cs{DTMDisplaydate}. The redefinition of
%these commands should be placed in \meta{definition}.
%
%A new \term{time style} can be defined using:
%\begin{definition}[\DescribeMacro\DTMnewtimestyle]
%\cs{DTMnewtimestyle}\marg{name}\marg{definition}
%\end{definition}
%This defines a \term{partial style} that should only modify 
%\cs{DTMdisplaytime}. The redefinition should be placed in 
%\meta{definition}.
%
%A new \term{zone style} can be defined using:
%\begin{definition}[\DescribeMacro\DTMnewzonestyle]
%\cs{DTMnewzonestyle}\marg{name}\marg{definition}
%\end{definition}
%This defines a \term{partial style} that should only modify 
%\cs{DTMdisplayzone}. The redefinition should be placed in 
%\meta{definition}.
%
%A new \term{full style} can be defined using:
%\begin{definition}[\DescribeMacro\DTMnewstyle]
%\cs{DTMnewstyle}\marg{name}\marg{date style definition}\marg{time
%style definition}\marg{zone style definition}\marg{full style definition}
%\end{definition}
%This does
%\begin{display}
%\cs{DTMnewdatestyle}\marg{name}\marg{date style definition}\newline
%\cs{DTMnewtimestyle}\marg{name}\marg{time style definition}\newline
%\cs{DTMnewzonestyle}\marg{name}\marg{zone style definition}\newline
%\end{display}
%and finally \meta{full style definition} should redefine
%\cs{DTMdisplay} and \cs{DTMDisplay}.
%
%\begin{important}
%Remember to use a double-hash to reference the parameters (\verb|##1|,
%\verb|##2| etc) within \meta{definition} in all the above.
%In each case \meta{name} is the label identifying the style
%and shouldn't contain active characters.
%\end{important}
%
%As from version 1.2, you can redefine existing styles with the following 
%commands.
%
%\begin{definition}[\DescribeMacro\DTMrenewdatestyle]
%\cs{DTMrenewdatestyle}\marg{name}\marg{definition}
%\end{definition}
%This redefines the named date style. The original may be either a
%partial or a full style.
%
%\begin{definition}[\DescribeMacro\DTMrenewtimestyle]
%\cs{DTMrenewtimestyle}\marg{name}\marg{definition}
%\end{definition}
%This redefines the named time style. The original may be either a
%partial or a full style.
%
%\begin{definition}[\DescribeMacro\DTMrenewzonestyle]
%\cs{DTMrenewzonestyle}\marg{name}\marg{definition}
%\end{definition}
%This redefines the named time zone style. The original may be either a
%partial or a full style.
%
%\begin{definition}[\DescribeMacro\DTMrenewstyle]
%\cs{DTMrenewstyle}\marg{name}\marg{definition}
%\end{definition}
%This redefines the named full style. The original style must also
%be a full style.
%
%There are also commands analogous to \cs{providecommand} that will
%define styles that don't already exist.
%
%\begin{definition}[\DescribeMacro\DTMprovidedatestyle]
%\cs{DTMprovidedatestyle}\marg{name}\marg{definition}
%\end{definition}
%This defines the named date style. This won't do anything if either
%a partial date style or a full style with the given name already
%exists.
%
%\begin{definition}[\DescribeMacro\DTMprovidetimestyle]
%\cs{DTMprovidetimestyle}\marg{name}\marg{definition}
%\end{definition}
%This defines the named time style. This won't do anything if either
%a partial time style or a full style with the given name already
%exists.
%
%\begin{definition}[\DescribeMacro\DTMprovidezonestyle]
%\cs{DTMprovidezonestyle}\marg{name}\marg{definition}
%\end{definition}
%This defines the named zone style. This won't do anything if either
%a partial zone style or a full style with the given name already
%exists.
%
%\begin{definition}[\DescribeMacro\DTMprovidestyle]
%\cs{DTMprovidestyle}\marg{name}\marg{definition}
%\end{definition}
%This defines the named full style if the named full style doesn't
%already exist. This internally uses the previous
%three commands for the partial elements of the full style, so it a
%partial style with this name already exists, it won't be changed.
%
%There are some helper commands provided that you might want to use
%in the style definitions.
%
%\begin{definition}[\DescribeMacro\DTMtwodigits]
%\cs{DTMtwodigits}\marg{number}
%\end{definition}
%This displays \meta{number} so that it has \emph{exactly} two digits.
%Unlike \LaTeX's \cs{two@digits} this will check for a negative
%number and will trim a number whose absolute value is greater than
%or equal to 100. This command is expandable.
%
%\begin{definition}[\DescribeMacro\DTMcentury]
%\cs{DTMcentury}\marg{year}
%\end{definition}
%This converts \meta{year} to the century. If \meta{year} is negative it does:
%\begin{display}
%\texttt{-\cs{DTMcentury}\{-\meta{year}\}}
%\end{display}
%Example:
%\begin{verbatim}
%\DTMcentury{1945}
%\end{verbatim}
%expands to \DTMcentury{1945} (not 19). Note that
%\begin{verbatim}
%\DTMcentury{1900}
%\end{verbatim}
%expands to \DTMcentury{1900}.
%
%\begin{definition}[\DescribeMacro\DTMdivhundred]
%\cs{DTMdivhundred}\marg{number}
%\end{definition}
%This expands to $\lfloor\meta{number}/100\rfloor$ (integer division
%by 100 rounded down). For example:
%\begin{verbatim}
%\DTMdivhundred{1945}
%\end{verbatim}
%expands to \DTMdivhundred{1945} and
%\begin{verbatim}
%\DTMdivhundred{1900}
%\end{verbatim}
%expands to \DTMdivhundred{1900}.
%
%\begin{definition}[\DescribeMacro\DTMtexorpdfstring]
%\cs{DTMtexorpdfstring}\marg{\TeX}\marg{PDF}
%\end{definition}
%If \sty{hyperref} is loaded, this is equivalent to
%\cs{texorpdfstring} otherwise it just does the first argument and
%ignores the second. (The check for \sty{hyperref} is deferred until
%the start of the \env{document} environment, so it doesn't matter if
%\sty{hyperref} is loaded after \styfmt{datetime2}.) This command may
%be used to provide alternative text to use if the date\slash
%time\slash zone is displayed in the PDF bookmarks.
%
%\begin{definition}[\DescribeMacro\DTMsep]
%\cs{DTMsep}\marg{tag}
%\end{definition}
%This accesses the value of the \meta{tag}\texttt{sep} base package
%option. (Not the language module options.)
%For example
%\begin{verbatim}
%\DTMsep{yearmonth}
%\end{verbatim}
%expands to the value supplied by the \pkgopt{yearmonthsep} package option.
%
%\begin{definition}[\DescribeMacro\DTMusezonemap]
%\cs{DTMusezonemap}\marg{TZh}\marg{TZm}
%\end{definition}
%This expands to the time zone abbreviation or \cs{relax} if no
%mapping has been set for the given time zone.
%
%You can define a time zone mapping using
%\begin{definition}[\DescribeMacro\DTMdefzonemap]
%\cs{DTMdefzonemap}\marg{TZh}\marg{TZm}\marg{abbr}
%\end{definition}
%For example
%\begin{verbatim}
%\DTMdefzonemap{00}{00}{GMT}
%\DTMdefzonemap{01}{00}{BST}
%\end{verbatim}
%Note that \styfmt{datetime2} doesn't know anything about daylight
%saving, so this is only really designed for dates and times 
%in a~specific location. This overwrites any previous mapping
%for this time zone.
%
%The base \styfmt{datetime2} package provides
%\begin{definition}[\DescribeMacro\DTMNatoZoneMaps]
%\cs{DTMNatoZoneMaps}
%\end{definition}
%This defines the military\slash NATO mappings from A (Alpha time) to
%Z (Zulu time). You can use this command if you want these time zones
%(but remember to set an appropriate time zone style that uses
%the zone mappings).
%
%The language modules may provide mappings that are enabled
%when you switch to that style. For example, the \module[english]{en-GB}
%language module provides the \pkgopt{mapzone} option which, if set
%to \texttt{true}, will map +00:00 to GMT and +01:00 to BST. See the
%documentation for the language module for further details.
%
%\begin{definition}[\DescribeMacro\DTMclearmap]
%\cs{DTMclearmap}\marg{TZh}\marg{TZm}
%\end{definition}
%Clears the time zone mapping. The regional time zone styles should
%use
%\begin{definition}[\DescribeMacro\DTMresetzones]
%\cs{DTMresetzones}
%\end{definition}
%before applying any regional mappings. This defaults to nothing
%which means that any mappings previously defined by other styles
%won't be cleared. You can redefine this command if you want to clear
%any mappings that aren't relevant for other regions.
%
%You can test if a~mapping is defined using
%\begin{definition}[\DescribeMacro\DTMhaszonemap]
%\cs{DTMhaszonemap}\marg{TZh}\marg{TZm}\marg{true}\marg{false}
%\end{definition}
%This will do \meta{true} if there is a mapping defined for that time
%zone or \meta{false} otherwise.
%
%\begin{definition}[\DescribeMacro\DTMusezonemapordefault]
%\cs{DTMusezonemapordefault}\marg{TZh}\marg{TZm}
%\end{definition}
%This will use the mapping if its defined otherwise it 
%will expand to the format
%\meta{TZh}\meta{HMsep}\meta{TZm}
%where \meta{HMsep}\meta{TZm} is omitted if the option
%\pkgopt{showzoneminutes} is set to \texttt{false}. The separator
%\meta{HMsep} is as given by the \pkgopt{hourminsep} option. (The
%\pkgopt{showisoZ} option isn't used here so UTC+00:00 will be
%displayed as +00:00 or +00 if there's no mapping.)
%
%Here's an example of a simple date style that just displays the year
%and month as two digits but uses the \pkgopt{yearmonthsep} option:
%\begin{verbatim}
%\newdatestyle
% {mmyy}% label
% {% definitions
%   \renewcommand*{\DTMdisplaydate}[4]{%
%     \DTMtwodigits{##2}\DTMsep{yearmonth}\DTMtwodigits{##1}}%
%   \renewcommand*{\DTMDisplaydate}{\DTMdisplaydate}%
% }
%\end{verbatim}
%
%If you want to distribute your new styles, just put the definitions
%in a package and upload it to CTAN. For example (replace
%\texttt{mystylename} with something more appropriate, and also
%change the date in the \cs{ProvidesPackage} line):
%\begin{verbatim}
% \NeedsTeXFormat{LaTeX2e}
% \ProvidesPackage{mystylename}[2014/03/24 v1.0]
% \RequirePackage{datetime2}
%
% % style definitions here
%
% \endinput
%\end{verbatim}
%Save the file as \texttt{mystylename.sty}, add some documentation
%about the style (or styles) provided and read the instructions at
%\url{http://www.ctan.org/upload} and
%\url{http://www.ctan.org/file/help/ctan/CTAN-upload-addendum}. The upload location
%for additions to the \styfmt{datetime2} package (either for
%packages defining new styles or for language modules) should be
%\texttt{/macros\slash latex\slash contrib\slash datetime2-contrib\slash mystylename} (remember to
%replace \texttt{mystylename} as appropriate).
%
%\chapter{Multi-Lingual Support}
%\label{sec:lang}
%
%If you want to use \styfmt{datetime2} with \sty{babel} or
%\sty{polyglossia}, make sure you load
%\sty{babel}\slash\sty{polyglossia} \emph{before} you load
%\styfmt{datetime2} otherwise their \cs{date}\meta{language} will
%overwrite \cs{datetime}'s definition of \cs{today}.
%\emph{Additionally} you need to make sure you install the relevant
%\styfmt{datetime2} language modules. These modules are automatically
%loaded, if required, by \styfmt{datetime2} but only if they are
%already installed. Remember that if you use \XeLaTeX\ you won't have
%the seconds or time zone available for the current date and time.
%
%\begin{important}
%If the required language modules aren't installed or
%\styfmt{datetime2} is loaded before
%\sty{babel}\slash\sty{polyglossia} then \styfmt{datetime2}'s
%definition of \cs{today} will be overridden and may no longer match
%the currently selected date style.
%\end{important}
%
%Each language module defines a textual style (where the month is
%displayed as a word) for that language or region
%which can be used in the argument of \cs{DTMsetstyle},
%\cs{DTMsetdatestyle}, \cs{DTMsettimestyle} or \cs{DTMsetzonestyle}.
%The language module may also define a numeric style. In the
%ambiguous cases where the language name alone doesn't indicate the
%region (for example, \texttt{english} instead of \texttt{UKenglish} or
%\texttt{USenglish}) the module should use the \style{default} numeric
%style (see \sectionref{sec:predefinedfull}).
%
%The textual style provided by the module will automatically be
%set using \cs{DTMsetstyle} \emph{if the \pkgopt{useregional}
%option is set to \texttt{text}}. By default \pkgopt{useregional} is
%\texttt{false}, unless the language\slash region is passed via the
%\styfmt{datetime2} package option list. (The \pkgopt{useregional}
%option is unaffected if the setting is passed through the document
%class option list.) The numeric style provided by the module will
%automatically be set if the \pkgopt{useregional} option is set to
%\texttt{numeric}. See the descriptions for the \pkgopt{useregional}
%and \pkgopt{style} options in \sectionref{sec:pkgopt}.
%
%\begin{important}
%Be careful not to mix the language\slash region options between the document
%class option list and the \sty{babel}\slash\sty{polyglossia}
%interface. For example, don't do:
%\begin{verbatim}
%\documentclass[en-GB]{article}
%\usepackage[canadien,british]{babel}
%\end{verbatim} 
%\end{important}
%The above example will prevent the \sty{tracklang} package from picking up the
%\sty{babel} setting and it will only detect the \texttt{en-GB}
%option. Use only the document class options or only the \sty{babel}
%package option list or duplicate \emph{all} the \sty{babel} package options with
%analogous \sty{tracklang} options in the document class. For example
%\begin{verbatim}
%\documentclass[canadien,british]{article}
%\usepackage{babel}
%\end{verbatim} 
%or
%\begin{verbatim}
%\documentclass{article}
%\usepackage[canadien,british]{babel}
%\end{verbatim} 
%or
%\begin{verbatim}
%\documentclass[fr-CA,en-GB]{article}
%\usepackage[canadien,british]{babel}
%\end{verbatim} 
%
%Language modules may be used without \sty{babel} or \sty{polyglossia}.
%For example:
%\begin{verbatim}
%\documentclass{article}
%\usepackage[en-GB]{datetime2}
%\begin{document}
%\today
%\end{document}
%\end{verbatim} 
%If you have more than one language or region you will need to switch
%styles using \cs{DTMsetstyle} etc if you aren't using \sty{babel} or 
%\sty{polyglossia}:
%\begin{verbatim}
%\documentclass{article}
%\usepackage[en-GB,en-CA]{datetime2}
%\begin{document}
%\DTMsetstyle{en-GB}\today.
%\DTMsetstyle{en-CA}\today
%\end{document}
%\end{verbatim} 
%
%If you want to change the number separators for the \emph{regional}
%numeric styles, you need to use \cs{DTMlangsetup}. If you want to
%change the number separators for the base \styfmt{datetime2}
%predefined numeric styles (see \sectionref{sec:predefinedstyles})
%then you need to use \cs{DTMsetup} or the package options. You
%therefore need to use \cs{DTMsetup} for the ambiguous regionless
%language numeric settings since they just use the \style{default}
%style. Check the module documentation to find out if the
%\style{default} style is used.
%
%Examples of use:
%\begin{enumerate}
%\item Language option specified through the document class and
%picked up by \sty{tracklang} (which is loaded by \styfmt{datetime2}).
%This setting is also picked up by \sty{babel} which is loaded before
%\styfmt{datetime2}.
%\begin{verbatim}
%\documentclass[british]{article}
%
%\usepackage{babel}
%\usepackage{datetime2}
%
%\begin{document}
%\today
%\end{document}
%\end{verbatim}
%The date is displayed in the default format 2015-03-01.
%
%In this case, the \texttt{en-GB} language module is loaded which
%defines the text style \style{en-GB} and the numeric style
%\style{en-GB-numeric}. Since \pkgopt{useregional} hasn't been set,
%\cs{today} uses \styfmt{datetime2}'s \style{default} numerical format.
%If \sty{babel} was loaded after \styfmt{datetime2}, the
%\sty{babel}'s hook management system would overwrite
%\styfmt{datetime2}'s definition of \cs{today} so that it no longer used
%\ics{DTMdisplaydate}. A similar result is obtained if in the above
%example \sty{babel} is replaced with \sty{polyglossia} (where the
%language is set in the document class option).
%
%You can change the \pkgopt{useregional} setting either through
%\styfmt{datetime2}'s package options or using \ics{DTMsetup} however
%it will only have an effect during the module loading (when the
%value is changed via the package option) and when
%\cs{date}\meta{language} is used.
%
%For example, in the document below, the date is displayed using the
%\style{default} numeric format because \pkgopt{useregional} has been
%changed \emph{after} \sty{babel} uses \cs{datebritish} to set the
%language at the start of the document.
%\begin{verbatim}
%\documentclass[british]{article}
%
%\usepackage{babel}
%\usepackage{datetime2}
%
%\begin{document}
%\DTMsetup{useregional}
%\today
%\end{document}
%\end{verbatim}
%So here \cs{today} again displays the date in the form 2015-03-01. 
%
%If the setting is moved to the preamble:
%\begin{verbatim}
%\documentclass[british]{article}
%
%\usepackage{babel}
%\usepackage{datetime2}
%
%\DTMsetup{useregional}
%\begin{document}
%\today
%\end{document}
%\end{verbatim}
%then the \pkgopt{useregional} setting is checked at the beginning of
%the document when \sty{babel} uses \cs{datebritish}. So in this case
%\cs{today} will display the date in the form 1st March 2015.
%
%\item Language setting specified through \sty{babel}'s package
%option list:
%\begin{verbatim}
%\documentclass{article}
%
%\usepackage[british]{babel}
%\usepackage{datetime2}
%
%\begin{document}
%\today
%\end{document}
%\end{verbatim}
%This has the same result as placing \texttt{british} in the document
%class option list, so the date is again displayed in the default
%format 2015-03-01 but, as in the previous example, the \style{en-GB}
%and \style{en-GB-numeric} styles are both defined if required.
%
%However a problem occurs if \sty{babel} is replaced by 
%\sty{polyglossia}:
%\begin{verbatim}
%\documentclass{article}
%
%\usepackage{fontspec}
%\usepackage{polyglossia}
%\setdefaultlanguage[variant=uk]{english}
%
%\usepackage{datetime2}
%
%\begin{document}
%\today
%\end{document}
%\end{verbatim}
%In this case \sty{tracklang} is unable to pick up the variant and
%can only detect the root language, so it will load the generic
%\module{english} module instead of the \texttt{en-GB} module. This
%means that the \style{en-GB} and \style{en-GB-numeric} styles are no longer
%available. However, since \pkgopt{useregional} is \texttt{false} the
%date is still displayed using the \texttt{default} numeric style
%in the form 2015-03-01.
%
%\item As mentioned above neither \sty{babel} nor \sty{polyglossia} are required in
%order to use the \styfmt{datetime2} language modules. You can simply
%supply the language setting in the package option list:
%\begin{verbatim}
%\documentclass{article}
%
%\usepackage[british]{datetime2}
%
%\begin{document}
%\today
%\end{document}
%\end{verbatim}
%This additionally sets \pkgopt[true]{useregional} (since the
%language is in the package option list not the document class option
%list) so the date produced by \cs{today} now uses the \texttt{en-GB}
%date style in the form 1st March 2015.
%
%\item The regional numeric format can be used instead if
%\pkgopt{useregional} is set to \texttt{numeric}:
%\begin{verbatim}
%\documentclass{article}
%
%\usepackage[british,useregional=numeric]{datetime2}
%
%\begin{document}
%\today
%\end{document}
%\end{verbatim}
%This now displays the date in the form 1/3/2015.
%\end{enumerate}
%
%Many of the language options have synonyms. In addition to the
%\sty{babel} synonyms (such as \pkgopt{british} or
%\pkgopt{UKenglish}) the \sty{tracklang} package provides options in
%ISO form, such as \pkgopt{en-GB}. Note that the style name provided
%by each language module is independent of the package option used
%to select that style. So regardless of whether you use
%\pkgopt{british}, \pkgopt{UKenglish} or \pkgopt{en-GB}, the text style
%name is \style{en-GB} and the numeric style name is
%\style{en-GB-numeric}. If just \pkgopt{english} is used, the text
%style name is \style{english} but the numeric style is
%\style{default}.
%
%Languages where the region is automatically implied, such
%as \pkgopt{scottish}, provide a text style with the root language
%name (\style{scottish} in this instance) and a numeric style in the
%form \meta{language}\texttt{-numeric} (such as \style{scottish-numeric}).
%Note that the \module{irish} module has regionless styles \texttt{irish} and
%\texttt{irish-numeric} but also has regional styles
%\texttt{ga-IE} and \texttt{ga-IE-numeric} (for the Republic of
%Ireland) and \texttt{ga-GB} and \texttt{ga-GB-numeric} (for Northern
%Ireland). In this case the regionless style has a numeric style
%instead of using the \style{default} style since both
%\texttt{ga-IE-numeric} and \texttt{ga-GB-numeric} are the same so
%there's no ambiguity.  The only difference in the three modules
%\texttt{datetime2-irish}, \texttt{datetime2-ga-IE} and
%\texttt{datetime2-ga-GB} is the time zone mappings.
%
%The language or regional modules may provide additional settings that can be
%applied using
%\begin{definition}[\DescribeMacro\DTMlangsetup]
%\cs{DTMlangsetup}\oarg{module-name list}\marg{options}
%\end{definition}
%where \meta{module-name list} is a comma-separated list of 
%modules that have previously been loaded (such as \texttt{en-GB,en-US})
%and \meta{options} is a \meta{key}=\meta{value} list of options.
%
%\begin{important}
%Note that the names in the \meta{module-name list} are 
%the identifying names of the module (such as \module[english]{en-GB} or 
%\module{english})
%which aren't necessarily the same as the language name supplied 
%to whatever language package you are using (such as \sty{babel} or
%\sty{polyglossia}). The code for each module should be in the file
%\texttt{datetime2-\meta{module-name}.ldf}, which should be in TeX's
%path.
%\end{important}
%
%If \meta{module-name list} is omitted, then the list of all loaded
%modules is assumed. There is also a starred version
%of this command (as from v1.3) which suppresses the warning
%if the given \meta{options} aren't available for any of
%the modules named in \meta{module-name list}. You may prefer
%to use the starred version if you omit \meta{module-name list}
%to skip the warnings from the base modules that don't support the
%given options.
%
%The modules may also provided 
%user commands to further customise the style. These settings should
%all be described in the module's documentation, which should be
%accessible via \texttt{texdoc datetime2-\meta{language}} where
%\meta{language} is the root language name in lower case 
%(such as \texttt{english}).
%
%Note that although I~maintain the \styfmt{datetime2} English
%language module, I~don't maintain the other modules. If you have an
%issue with one of the other modules, please contact the module
%maintainer. If there is no maintainer, feel free to volunteer to
%take over the maintenance (\href{http://www.dickimaw-books.com/contact.html}{send me a
%message}). If there's no module for your language
%you can create your own module and upload it to CTAN in the 
%\texttt{/macros\slash latex\slash contrib\slash datetime2-contrib\slash
%datetime2-\meta{language}} directory.
%
%You can use the English or Irish modules as a template for a
%language with multiple regions. Just download the English source
%files
%\href{http://mirror.ctan.org/macros/latex/contrib/datetime2-contrib/datetime2-english/datetime2-english.dtx}{\texttt{datetime2-english.dtx}} and
%\href{http://mirror.ctan.org/macros/latex/contrib/datetime2-contrib/datetime2-english/datetime2-english.ins}{\texttt{datetime2-english.ins}} or the Irish source files
%\href{http://mirror.ctan.org/macros/latex/contrib/datetime2-contrib/datetime2-irish/datetime2-irish.dtx}{\texttt{datetime2-irish.dtx}}
%and
%\href{http://mirror.ctan.org/macros/latex/contrib/datetime2-contrib/datetime2-irish/datetime2-irish.ins}{\texttt{datetime2-irish.ins}} from
%CTAN and make the appropriate modifications. Alternatively you can
%use the Scottish module as a template for a single-region language.
%Just download the Scottish source files 
%\href{http://mirror.ctan.org/macros/latex/contrib/datetime2-contrib/datetime2-scottish/datetime2-scottish.dtx}{\texttt{datetime2-scottish.dtx}} and
%\href{http://mirror.ctan.org/macros/latex/contrib/datetime2-contrib/datetime2-scottish/datetime2-scottish.ins}{\texttt{datetime2-scottish.ins}} from CTAN and make the appropriate
%modifications. (Don't forget to provide a README file.)
%
%Each language module should be in a file named
%\texttt{datetime2-\meta{lang}.ldf} where \meta{lang} is either the language
%name or in the form \meta{language ISO code}\texttt{-}\meta{country ISO code}. (See the
%\sty{tracklang} documentation for further details of the naming
%scheme.)
%
%A regional module may load a base module for the same language using
%\begin{definition}[\DescribeMacro\RequireDateTimeModule]
%\cs{RequireDateTimeModule}\marg{name}
%\end{definition}
%This will input the file \texttt{datetime2-\meta{name}.ldf}. This command
%should not be used outside the \styfmt{datetime2} language module files.
%If you are creating a package that explicitly needs to load one of these files,
%then you can use:
%\begin{definition}[\DescribeMacro\DTMusemodule]
%\cs{DTMusemodule}\marg{language}\marg{name}
%\end{definition}
%where \meta{language} is the \sty{babel} or \sty{polyglossia} language or
%dialect name that identifies the relevant \cs{date\meta{language}}
%macro (for example, \texttt{english}) and \meta{name} is the same as
%above (for example, \texttt{en-GB}).
%
%Note that \cs{RequireDateTimeModule} (which is also internally used
%by \cs{DTMusemodule}) stores a mapping from the language name
%and the module name. You can determine what module was loaded
%for a given dialect name using
%\begin{definition}[\DescribeMacro\DTMdialecttomodulemap]
%\cs{DTMdialecttomodulemap}\marg{dialect}
%\end{definition}
%This expands to the required module name or \cs{relax} if the 
%given dialect name wasn't used to load a module.
%For example:
%\begin{verbatim}
%\documentclass{article}
%\usepackage[british]{datetime2}
%
%\begin{document}
%
%british map: \DTMdialecttomodulemap{british}.
%english map: \DTMdialecttomodulemap{english}.
%
%\end{document}
%\end{verbatim}
%This produces:
%\begin{quote}
%british map: en-GB.
%english map: .
%\end{quote}
%In the above, the second instance expands to \cs{relax}.
%
%\begin{important}
%If you want to provide a language module don't assume all users want
%to use the same input encoding or \sty{babel} shorthands as you. Use \LaTeX\
%commands for non-ASCII characters (and remember to use \cs{protect}
%where necessary).
%\end{important}
%
%As an addendum to the above warning, Lua\TeX\ and \XeTeX\ support
%UTF-8 characters without the need to make them active, so
%I~recommend you provide two files: one with the \LaTeX\ commands,
%such as \cs{c}, for (PDF)\LaTeX\ users, and one with UTF-8 characters
%for Lua\LaTeX\ and \XeLaTeX\ users. For example, the \texttt{fr-FR} module
%could start with:
%\begin{verbatim}
%\ProvidesDateTimeModule{fr-FR}
%
%\RequirePackage{ifxetex,ifluatex}
%
%\ifxetex
% \RequireDateTimeModule{french-utf8}
%\else
% \ifluatex
%   \RequireDateTimeModule{french-utf8}
% \else
%   \RequireDateTimeModule{french-ascii}
% \fi
%\fi
%\end{verbatim}
%
%This helps provide fully expandable dates for Lua\LaTeX\ and
%\XeLaTeX\ users. (See the Scottish or Irish modules for examples.)
%
%\chapter{Standalone Month or Weekday Names}
%\label{sec:names}
%
%\begin{important}
%If you want the month name or weekday name to appear in a section
%or chapter heading, it's best to use the expandable commands
%provided by the language modules rather than the robust commands
%provided by \styfmt{datetime2-calc}. Remember that you can't use robust
%commands in PDF bookmarks and such commands may prevent case-changing in 
%headers for page styles that use \cs{MakeUppercase}.
%\end{important}
%
%The language or regional modules described in 
%\sectionref{sec:lang} typically provide an expandable command 
%\begin{definition}
%\cs{DTM\meta{root-language}monthname}\marg{n}
%\end{definition}
%which takes a numerical argument that indicates the month number.
%This command is used in the date style which ensures that even if
%the document language has switched but not the date style, then
%the month name will be in the correct language for that style.
%(Otherwise you could end up with a mix of style from one dialect
%using names from another language, which was one of the problems
%with the original \styfmt{datetime} package.)
%
%For example, if the \module{english} module is loaded (which is
%automatically loaded by the English dialect modules, such as
%\module[english]{en-GB}) then the command \cs{DTMenglishmonthname} is
%defined. So if you're writing in English and you want to display
%just the
%month name, then you can do:
%\begin{verbatim}
%\DTMenglishmonthname{1}
%\end{verbatim} 
%Some language modules, where month names aren't automatically
%capitalised, may additionally define a version that
%has the first letter in upper case. For example, the \module{french}
%module defines \cs{DTMfrenchMonthname} in addition to
%\cs{DTMfrenchmonthname}.
%
%Some of the modules may have other alternatives. For example,
%the \module{serbian} module provides Cyrillic
%(\cs{DTMserbiancyrmonthname}) and Latin
%(\cs{DTMserbianlatinmonthname})
%month names. It also provides \cs{DTMserbianmonthname}, which
%defaults to \cs{DTMserbiancyrmonthname} but can be redefined
%using \cs{DTMlangsetup}.
%
%To find out the available commands for the module you are using,
%see that module's documentation.
%
%If you are writing a document that uses multiple languages
%and you simply want to display the month name in the currently
%selected language, then you can use the robust command 
%\begin{definition}
%\ics{DTMmonthname}\marg{n}
%\end{definition}
%provided by the \styfmt{datetime2-calc} package,
%described in \sectionref{sec:calc}. Remember that the
%\styfmt{datetime2-calc} package also loads the \sty{pgfcalendar}
%package, which provides the expandable command
%\cs{pgfcalendarmonthname}. (The \sty{pgfcalendar} package provides
%multilingual support via the \sty{translator} package.)
%
%Some of the language modules additionally provide a command
%that displays the first letter in upper case. This isn't provided
%for languages where the month name is always displayed with
%a capital first letter, such as in English. For example,
%the \module{serbian} module also defines
%\cs{DTMserbiancyrMonthname} and \cs{DTMserbianlatinMonthname},
%with \cs{DTMserbianMonthname} initially defined to use
%the Cyrillic version.
%
%So, if you specifically want to display the Serbian Cyrillic 
%month name with the first letter in upper case, you need to
%make sure the \module{serbian} module is loaded
%and then use the provided \cs{DTMserbiancyrMonthname} command.
%
%If you want the month name to vary according to the current language
%setting in the document, you can use the robust command
%\begin{definition}
%\cs{DTMMonthname}\marg{n}
%\end{definition}
%provided by \styfmt{datetime2-calc}, which will first
%attempt \cs{DTM\meta{language}Monthname} and then
%\cs{DTM\meta{language}monthname} before falling back on the
%\cs{pgfcalendarmonthname}, see \sectionref{sec:calc} for further
%details.
%
%Some, but not all, language modules provide a command (or commands)
%for month name abbreviations. It's up to the maintainer of the
%module to add these if they currently aren't provided. You will
%need to check the module documentation to find out if abbreviations
%are supported. If they are supported, they should be in
%the form
%\begin{definition}
%\cs{DTM\meta{root-language}shortmonthname}\marg{n}
%\end{definition}
%Again there may be variations, such as new and old styles or alternative
%alphabets. For example, the \module{english} module
%provides \cs{DTMenglishshortmonthname}.
%
%As with the full form, if you want to display the abbreviation
%in a specific language (or variation), then use the command
%provided by the relevant language module. If you want the
%abbreviation to pick up the current language, then you can
%use the robust command
%\begin{definition}
%\cs{DTMshortmonthname}\marg{n}
%\end{definition}
%provided by the \styfmt{datetime2-calc} package. This will fall back on
%\cs{pgfcalendarmonthshortname} if
%\cs{DTM\meta{language}shortmonthname} isn't defined. See
%\sectionref{sec:calc} for further details.
%
%Modules may additionally provide a version of the abbreviated
%form that starts with a capital letter. This should typically
%be in the form
%\begin{definition}
%\cs{DTM\meta{root-language}shortMonthname}\marg{n}
%\end{definition}
%Check the module documentation to see if this is provided.
%Again, the \styfmt{datetime2-calc} package provides a robust command
%\begin{definition}
%\cs{DTMshortMonthname}\marg{n}
%\end{definition}
%that attempts to determine the relevant
%module command from the language name.
%
%Language modules may or may not provide a command that displays
%the weekday name. As with the month name, there may or may not
%be an abbreviated version or capital first letter version
%or variations such as new\slash old styles.
%Check the module documentation for further details. If
%the module doesn't provide a weekday name macro, then the 
%provided styles won't support the \pkgopt{showdow} option.
%
%\emph{If} the module provides weekday name support, then
%the name will typically be provided by a macro in the form
%\begin{definition}
%\cs{DTM\meta{root-language}weekdayname}\marg{n}
%\end{definition}
%where \meta{n} is an integer from 0 (Monday) to 6 (Sunday).
%For example, the \module{english} module provides
%\cs{DTMenglishweekdayname}.
%
%Again, the \styfmt{datetime2-calc} package provides \emph{robust}
%commands that attempt to find the relevant module-provided 
%command based on the current language. If not found, 
%the fallback commands are those provided by the \sty{pgfcalendar}
%package. See \sectionref{sec:calc} for further details.
%
%\chapter{Package Options}
%\label{sec:pkgopt}
%
%The following package options are provided. Most of these are
%\meta{key}=\meta{value} options, unless stated otherwise.
%
%Settings that govern the predefined numerical styles (not including
%the fixed styles \style{iso} and \style{pdf}):
%\begin{description}
%\item[\pkgopt{yearmonthsep}] This sets the separator between the
%year and month for the big-endian and little-endian styles.
%Default: \texttt{-} (hyphen).
%Note that if you want a space as a separator you need to use \cs{space}. 
%If you simply use a space character (for example, \verb|yearmonthsep={ }|)
%then the separator will be discarded. The same applies for the other separators described below.
%
%\item[\pkgopt{monthdaysep}] This sets the separator between the
%month and day.
%Default: \texttt{-} (hyphen).
%
%\item[\pkgopt{dayyearsep}] This sets the separator between the
%day and year for the middle-endian styles.
%Default: \texttt{-} (hyphen).
%
%\item[\pkgopt{datesep}] This sets the separators between the day and
%month, the month and year, and the day and year.  Example:
%\begin{verbatim}
%\usepackage[datesep=/]{datetime2}
%\end{verbatim}
%This is equivalent to:
%\begin{verbatim}
%\usepackage[yearmonthsep=/,monthdaysep=/,dayyearsep=/]{datetime2}
%\end{verbatim}
%
%\item[\pkgopt{hourminsep}] This sets the separator between the hour
%and minute. (Both for the time and for the zone.)
%Default: \texttt{:} (colon).
%
%\item[\pkgopt{minsecsep}] This sets the separator between the minute
%and seconds.
%Default: \texttt{:} (colon).
%
%\item[\pkgopt{timesep}] This sets the separators between the hour
%and minute and between the minute and seconds. 
%Example:
%\begin{verbatim}
%\usepackage[timesep=:]{datetime2}
%\end{verbatim}
%This is equivalent to:
%\begin{verbatim}
%\usepackage[hourminsep=:,minsecsep=:]{datetime2}
%\end{verbatim}
%\end{description}
%
%The following settings are used by the predefined numerical styles when 
%displaying the full date, time and zone (excluding the fixed styles \style{iso}
%and \style{pdf}) with commands that use \cs{DTMdisplay} or
%\cs{DTMDisplay}.
%\begin{description}
%\item[\pkgopt{datetimesep}] Sets the separator between the date and
%time.
%Default: \cs{space}.
%
%\item[\pkgopt{timezonesep}] Sets the separator between the time and
%zone.
%Default: empty.
%
%\end{description}
%
%The following settings are used by the predefined styles and may
%also be used by the language modules.
%\begin{description}
%\item[\pkgopt{showseconds}] Boolean key to determine whether or not
%to show the seconds when the time is displayed. The \style{iso}
%style honours this setting but the \style{pdf} style ignores it.
%Default: \texttt{true} unless \XeTeX\ is used.
%
%\item[\pkgopt{showdate}] Boolean key to determine whether or not to
%show the date with commands that use \cs{DTMdisplay} or
%\cs{DTMDisplay}. (Some styles may ignore this.) The \style{iso}
%style honours this setting but the \style{pdf} style ignores it.
%Default: \texttt{true} unless \XeTeX\ is used.
%
%\item[\pkgopt{showzone}] Boolean key to determine whether or not to
%show the time zone with commands that use
%\cs{DTMdisplay} or \cs{DTMDisplay}. (Some styles may ignore this.) The \style{iso} style honours this setting but the
%\style{pdf} style ignores it.
%Default: \texttt{true} unless \XeTeX\ is used.
%
%\item[\pkgopt{showzoneminutes}] Boolean key to determine whether or
%not to show the zone offset minutes. The \style{iso} style honours
%this setting but the \style{pdf} style ignores it. This setting is
%ignored if \pkgopt{showzone} is \texttt{false}.
%Default: \texttt{true}.
%
%\item[\pkgopt{showisoZ}] Boolean key to determine whether or not to
%show UTC+00:00 as \texttt{Z} instead of numerically. This option
%may be ignored by zone styles that use the zone mappings. If you
%want all the time zones in military form, you can use
%\cs{DTMNatoZoneMaps} to set up the time zone abbreviations and then
%use a zone style that uses the mappings.
%Default: \texttt{true}.
%
%\end{description}
%
%General settings:
%\begin{description}
%\item[\pkgopt{useregional}] Allowed values: \texttt{false},
%\texttt{text} or \texttt{numeric}. You may also use \texttt{num} as
%an abbreviation for \texttt{numeric}. If no value is supplied
%\texttt{text} is assumed.
%
%This key determines whether or not to \emph{use} the loaded regional 
%settings and, if the regional setting should be used, it determines
%whether the text style (months as words) or numeric style should be
%used. If you haven't loaded \sty{babel} or \sty{polyglossia}, this
%key only has an effect when used as a package option. If you have
%loaded one of those packages, the change comes into effect at module
%load time and whenever \cs{date}\meta{language} is used (which
%includes at the beginning of the \env{document} environment). If you want
%to switch the style at any other time, you need to use
%\cs{DTMsetstyle} but unless \pkgopt[false]{useregional} the next
%instance of \cs{date}\meta{language} will change the style.
%
%Note that setting this option to \texttt{false} doesn't prevent the modules 
%from being loaded. It just prevents them from automatically setting the style
%and prevents \cs{date}\meta{language} from changing the style if you
%are using \sty{babel} or \sty{polyglossia}.
%
%The default value is \texttt{false} unless the language or region is
%passed to the \styfmt{datetime2} package option list. However, using \pkgopt{style}
%will set \pkgopt{useregional} to \texttt{false}.
%
%Examples:
%\begin{verbatim}
%\documentclass[british]{article}
%\usepackage{babel}
%\usepackage{datetime2}
%\end{verbatim}
%In the above \pkgopt{useregional} is \texttt{false}.
%
%\begin{verbatim}
%\documentclass{article}
%\usepackage[british]{datetime2}
%\end{verbatim}
%In the above \pkgopt{useregional} is \texttt{text}.
%
%\begin{verbatim}
%\documentclass{article}
%\usepackage[british,style=iso]{datetime2}
%\end{verbatim}
%In the above \pkgopt{useregional} is \texttt{false}. (The
%\pkgopt{british} option implements \pkgopt[text]{usenumerical}
%but the \pkgopt{style} option then implements
%\pkgopt[false]{usenumerical}.)
%
%\begin{verbatim}
%\documentclass{article}
%\usepackage[style=iso,british]{datetime2}
%\end{verbatim}
%In the above \pkgopt{useregional} is \texttt{text}. (The
%\pkgopt{style} option implements \pkgopt[false]{usenumerical}
%but the \pkgopt{british} option then implements
%\pkgopt[text]{usenumerical}.)
%
%\item[\pkgopt{style}] Sets the current style using \cs{DTMsetstyle}
%when the \styfmt{datetime2} package has finished loading. This also sets
%\pkgopt[false]{useregional} but that setting can be overridden later
%in the option list.
%
%Default value: empty (use the default style or the regional style,
%according to the value of \pkgopt{useregional}).
%
%This key isn't available in \cs{DTMsetup}. Use \cs{DTMsetstyle} instead.
%
%\item[\pkgopt{calc}] Load the \styfmt{datetime2-calc} package. This
%will allow the day of week to be computed and allow you to use the
%\sty{pgfcalendar} offset style date formats in commands like
%\cs{DTMdate} as well as defining the commands described in
%\sectionref{sec:calc}. This option doesn't take a value. It can't be switched
%off. This option can't be used in \cs{DTMsetkeys}. The default is to
%not load \styfmt{datetime2-calc}.
%
%\item[\pkgopt{showdow}] This is a boolean key that determines
%whether or not to show the day of week in styles that support this.
%Note that \pkgopt[true]{showdow} will automatically load
%\styfmt{datetime2-calc} so
%\begin{verbatim}
%\usepackage[showdow]{datetime2}
%\end{verbatim}
%is equivalent to 
%\begin{verbatim}
%\usepackage[showdow,calc]{datetime2}
%\end{verbatim}
%
%This option may be used in \cs{DTMsetup}, but if you attempt to
%switch it on in the \env{document} environment you'll get an error
%if the \styfmt{datetime2-calc} package hasn't been loaded.
%\textbf{Not all styles support this setting.}
%Default: \texttt{false}. 
%
%This option is actually a language-dependent option and isn't used
%by the base package, but it's implemented as a package option as the
%\styfmt{datetime2-calc} package is also needed if the day of the week
%should be displayed. As a package option, \pkgopt{showdow} can
%automatically load the required package.
%
%\begin{important}
%You need to check the documentation to find out which styles check
%the \pkgopt{showdow} setting as not all of them do.
%\end{important}
%
%\item[\pkgopt{warn}] This is a boolean key. If true (default)
%\styfmt{datetime2} warnings will be displayed. If false, the warnings
%will be suppressed.
%Default: \texttt{true}.
%
%\end{description}
%
%Any additional option passed to the \styfmt{datetime2} package (not
%through \cs{DTMsetup}) will be considered a \sty{tracklang} option
%and will be passed to \cs{TrackPredefinedDialect}. (See the
%\sty{tracklang} documentation for further details of that command.)
%
%Apart from \pkgopt{calc}, \pkgopt{style} and the regional options, all the above
%options can also be set using:
%\begin{definition}[\DescribeMacro\DTMsetup]
%\cs{DTMsetup}\marg{option list}
%\end{definition}
%
%The language modules may additionally provide options which can be
%set using:
%\begin{definition}[\DescribeMacro\DTMlangsetup]
%\cs{DTMlangsetup}\oarg{module-name list}\marg{option list}
%\end{definition}
%
%This will set the \meta{option list} for each module listed in
%\meta{module-name list}. Unknown options will generate a~warning rather
%than an error message. The default value of \meta{module-name list} is
%the list of all loaded modules.
%
%Example:
%\begin{verbatim}
%\documentclass{article}
%\usepackage[british]{datetime2}
%\DTMlangsetup{mapzone}
%\end{verbatim}
%The module list here is \texttt{english-base},\texttt{en-GB} and
%since the \texttt{english-base} doesn't have a \texttt{mapzone}
%option, this will result in a warning:
%\begin{verbatim}
%Package datetime2 Warning: Region `english-base' has ignored 
%(datetime2)                the following settings:
%(datetime2)                mapzone
%\end{verbatim}
%You can either ignore the warning or use the optional argument to
%exclude the \texttt{english-base} module:
%\begin{verbatim}
%\documentclass{article}
%\usepackage[british]{datetime2}
%\DTMlangsetup[en-GB]{mapzone}
%\end{verbatim}
%Alternatively you can use the starred version:
%\begin{verbatim}
%\documentclass{article}
%\usepackage[british]{datetime2}
%\DTMlangsetup*{mapzone}
%\end{verbatim}
%
%Note that some modules may have options with the same name as the
%above listed package options, but the keys are defined in different
%families (see \sty{xkeyval} documentation) so you need to take care
%to use \cs{DTMsetup} for package-wide settings and \cs{DTMlangsetup}
%for the module-specific settings.
%
%For example, the \pkgopt{datesep} package option described above is
%used by the predefined numerical styles but regional modules that
%provide their own numerical styles may use a different date
%separator that matches their region so they may also provide a
%\texttt{datesep} option independent of the base \pkgopt{datesep}
%option.
%
%Examples:
%\begin{verbatim}
%\documentclass[british]{article}
%\usepackage[datesep=.]{datetime2}
%\begin{document}
%\today
%\end{document}
%\end{verbatim}
%The above displays the date in the form
%{\DTMsetup{datesep=.}\DTMsetdatestyle{default}\today}
%since the default style is in use and \pkgopt{datesep} is used as a
%package option.
%
%\begin{verbatim}
%\documentclass[british]{article}
%\usepackage{datetime2}
%\DTMsetup{datesep=.}
%\begin{document}
%\today
%\end{document}
%\end{verbatim}
%The above displays the date in the form
%{\DTMsetup{datesep=.}\DTMsetdatestyle{default}\today}
%since the default style is in use and \pkgopt{datesep} is used in
%\cs{DTMsetup}.
%
%\begin{verbatim}
%\documentclass[british]{article}
%\usepackage{datetime2}
%\DTMlangsetup{datesep=.}
%\begin{document}
%\today
%\end{document}
%\end{verbatim}
%The above displays the date in the form
%{\DTMlangsetup[en-GB]{datesep=.}\DTMsetdatestyle{default}\today}
%since the default style is in use but \pkgopt{datesep} is used in
%\cs{DTMlangsetup}, which only influences the \style{en-GB-numeric}
%style, which isn't the current style.
%
%\begin{verbatim}
%\documentclass[british]{article}
%\usepackage[useregional=numeric]{datetime2}
%\DTMlangsetup{datesep=.}
%\begin{document}
%\today
%\end{document}
%\end{verbatim}
%The above displays the date in the form
%{\DTMlangsetup[en-GB]{datesep=.}\DTMsetdatestyle{en-GB-numeric}\today}
%since the \style{en-GB-numeric} style is in use and \pkgopt{datesep}
%is used in \cs{DTMlangsetup}.
%
%\begin{verbatim}
%\documentclass[british]{article}
%\usepackage[useregional]{datetime2}
%\DTMlangsetup{datesep=.}
%\begin{document}
%\today
%\end{document}
%\end{verbatim}
%The above displays the date in the form
%{\DTMlangsetup[en-GB]{datesep=.}\DTMsetdatestyle{en-GB}\today}
%since the \style{en-GB} style is in use and \pkgopt{datesep} is used
%in \cs{DTMlangsetup}, which only influences the
%\style{en-GB-numeric} style.
%
%\begin{verbatim}
%\documentclass[british]{article}
%\usepackage[useregional=numeric]{datetime2}
%\DTMsetup{datesep=.}
%\begin{document}
%\today
%\end{document}
%\end{verbatim}
%The above displays the date in the form
%{\DTMsetup{datesep=.}\DTMsetdatestyle{en-GB-numeric}\today}
%since the \style{en-GB-numeric} style is in use but \pkgopt{datesep}
%is used in \cs{DTMsetup} which influences the base predefined
%numeric styles not the regional styles.
%
%\chapter{The \styfmt{datetime2-calc} Package}
%\label{sec:calc}
%
%The \styfmt{datetime2-calc} package can be loaded after \styfmt{datetime2} in the 
%usual way:
%\begin{verbatim}
%\usepackage{datetime2}
%\usepackage{datetime2-calc}
%\end{verbatim}
%or using the \pkgopt{calc} package option to \styfmt{datetime2}:
%\begin{verbatim}
%\usepackage[calc]{datetime2}
%\end{verbatim}
%or by using \pkgopt[true]{showdow}:
%\begin{verbatim}
%\usepackage[showdow]{datetime2}
%\end{verbatim}
%
%This package loads the \sty{pgfcalendar} package which provides a
%way of computing the day of week from a given date. Once
%\styfmt{datetime2-calc} has been loaded, you can enable or disable the
%weekday in dates where the style supports this, but note that \emph{not
%all styles support this}, even if the \styfmt{datetime2-calc} package
%has been loaded.
%
%As with the commands in \sectionref{sec:store}, the commands
%described below that save date\slash time information will
%\emph{overwrite} any previously defined date\slash time data with
%the same identifying \meta{name}. However, they may only overwrite
%specific elements of the data (for example, just the year, month,
%day and day of week elements) and leave the other elements
%unchanged. Where the remaining elements are undefined they'll be set to
%zero, except for the day of week element, which will be set to
%\texttt{-1}.
%
%In addition to enabling the weekday calculations, the \styfmt{datetime2-calc} 
%package also provides the following commands:
%\begin{definition}[\DescribeMacro\DTMsavejulianday]
%\cs{DTMsavejulianday}\marg{name}\marg{number}
%\end{definition}
%This uses \cs{pgfcalendarjuliantodate} to obtain the year, month and
%day from the given Julian day number and uses
%\cs{pgfcalendarjuliantoweekday} to obtain the day of week and then
%saves it. The date can later be used with commands such as
%\cs{DTMuse}\marg{name} described in \sectionref{sec:store}.
%Example:
%\begin{verbatim}
%\DTMsavejulianday{mydate}{2457023}
%\end{verbatim}
%
%\begin{definition}[\DescribeMacro\DTMsaveddatetojuliandate]
%\cs{DTMsaveddatetojulianday}\marg{name}\marg{register}
%\end{definition}
%This uses \cs{pgfcalendardatetojulian} to convert a previously saved
%date (identified by \meta{name}) to a Julian day. The result is
%stored in \meta{register} which should be a count register (not a
%\LaTeX\ counter name).
%Example:
%\begin{verbatim}
%\newcount\myct
%\DTMsaveddatetojulianday{mydate}{\myct}
%\end{verbatim}
%
%\begin{definition}[\DescribeMacro\DTMsaveddateoffsettojuliandate]
%\cs{DTMsaveddateoffsettojulianday}\marg{name}\marg{offset}\marg{register}
%\end{definition}
%This is like the previous command but converts the date obtained by
%incrementing the saved date with \meta{offset}. The result is stored
%in \meta{register}. This is equivalent to
%\begin{quote}\ttfamily
%\cs{pgfcalendardatetojulian}\{\meta{y}-\meta{m}-\meta{d}+\meta{offset}\}\marg{register}
%\end{quote}
%where \meta{y}, \meta{m} and \meta{d} are the year, month and day
%fetched from the saved date. A negative \meta{offset} indicates an earlier
%date.
%Example:
%\begin{verbatim}
%\DTMsaveddateoffsettojulianday{mydate}{2}{\myct}
%\end{verbatim}
%or
%\begin{verbatim}
%\DTMsaveddateoffsettojulianday{mydate}{-7}{\myct}
%\end{verbatim}
%
%\begin{definition}[\DescribeMacro\DTMifdate]
%\cs{DTMifdate}\marg{name}\marg{test}\marg{true}\marg{false}
%\end{definition}
%This is just a convenient interface to \cs{pgfcalendarifdate}
%for a saved date (identified by \meta{name}). The remaining
%arguments are the same as the final three arguments of 
%\cs{pgfcalendarifdate}. Note that the
%\texttt{equals}, \texttt{at least}, \texttt{at most} and
%\texttt{between} keywords available in \meta{test} need to be in the
%format specified by the \styfmt{pgf} manual, but remember that you can
%use commands like \cs{DTMfetchyear}.
%Example:
%\begin{verbatim}
%Is \texttt{mydate2} (\DTMusedate{mydate2}) before
%\texttt{mydate} (\DTMusedate{mydate})?
%\DTMifdate
% {mydate2}
% {at most=
%   \DTMfetchyear{mydate}-\DTMfetchmonth{mydate}-\DTMfetchday{mydate}}
% {yes}{no}.
%\end{verbatim}
%
%\begin{definition}[\DescribeMacro\DTMsaveddatediff]
%\cs{DTMsavedatediff}\marg{name1}\marg{name2}\marg{register}
%\end{definition}
%Computes the difference (in days) between two saved dates and stores
%the result in the given count register. The first date is identified
%by \meta{name1} and the second date is identified by \meta{name2}.
%The dates are converted to their respective Julian day numbers
%\meta{J1} and \meta{J2} and the result is given by
%\meta{J1}$-$\meta{J2}.
%
%\begin{important}
%Note that the time and zone are not taken into account, even if they
%were provided when the dates were stored.
%\end{important}
%Example:
%\begin{verbatim}
%\DTMsaveddatediff{mydate}{mydate2}{\myct}
%
%\DTMusedate{mydate} is
%\ifnum\myct=0
% the same day as
%\else
%  \ifnum\myct<0
%   \number-\myct\space day\ifnum\myct<-1s\fi\space before
%  \else
%   \number\myct\space day\ifnum\myct>1s\fi\space after
%  \fi
%\fi
%\DTMusedate{mydate2}.
%\end{verbatim}
%
%The \styfmt{datetime2-calc} package also provides commands that convert
%a datetime instance into Zulu\footnote{That's Zulu as in the NATO
%alphabet representation of the letter Z.}\ time (UTC+00:00).
%
%\begin{definition}[\DescribeMacro\DTMsaveaszulutime]
%\cs{DTMsaveaszulutime}\marg{name}\marg{YYYY}\marg{MM}\marg{DD}\marg{hh}\marg{mm}\newline\marg{ss}\marg{TZh}\marg{TZm}
%\end{definition}
%This converts the given datetime instance into UTC+00:00 and saves
%the result. You can then use the date with commands like \cs{DTMuse}
%described in \sectionref{sec:store}. The \meta{name} argument is the
%label identifying the saved data. The other arguments are all
%numbers.
%Example:
%\begin{verbatim}
%\DTMsaveaszulutime{mydate}{2014}{6}{3}{20}{45}{0}{6}{0}
%\end{verbatim}
%
%\begin{definition}[\DescribeMacro\DTMtozulu]
%\cs{DTMtozulu}\marg{name1}\marg{name2}
%\end{definition}
%Uses \cs{DTMsaveaszulutime} to convert the datetime stored in
%\meta{name1} and saves it to \meta{name2}.
%Example:
%\begin{verbatim}
%\DTMsavetimestamp{mydate}{2014-05-01T03:55:00 -06:00}
%Original date: \DTMuse{mydate}.
%
%\DTMtozulu{mydate}{mydate2}
%UTC+00:00: \DTMuse{mydate2}.
%\end{verbatim}
%The above produces (using the \style{default} format):
%\DTMsetstyle{default}%
%\begin{quote}
%\DTMsavetimestamp{mydate}{2014-05-01T03:55:00 -06:00}
%Original date: \DTMuse{mydate}.
%
%\DTMtozulu{mydate}{mydate2}
%UTC+00:00: \DTMuse{mydate2}.
%\end{quote}
%
%The \sty{pgfcalendar} package also provides a variety of useful
%date-related commands. See the documentation (part of the
%\styfmt{pgf} manual) for further details. Note that the language
%modules don't use \sty{pgfcalendar} month and weekday names as the
%\sty{pgfcalendar} package isn't loaded by default and the styles
%need to match the language with the syntax. However, since the
%\styfmt{datetime2-calc} package automatically loads the
%\sty{pgfcalendar} package, as from v1.3 the \styfmt{datetime2-calc}
%provides robust month name and weekday name commands that may be
%used outside of date styles, which fallback on the commands provided
%by \sty{pgfcalendar}.
%
%\begin{important}
%The following commands, which are all robust, should not be used in date 
%styles, since each language style must use the name macro for that specific 
%language to match the style. Make sure you have the relevant
%language module installed and loaded to allow these commands to work
%correctly. See also \sectionref{sec:names}. Remember that instead of
%these robust commands, you can simply just use the commands provided
%by the \sty{pgfcalendar} package. (See the \styfmt{pgf} manual for
%further details.)
%\end{important}
%
%\begin{definition}[\DescribeMacro\DTMmonthname]
%\cs{DTMmonthname}\marg{n}
%\end{definition}
%This checks if \cs{DTM\meta{lang}monthname} exists where \meta{lang}
%is given by \cs{languagename}. If so, that macro is used. For example:
%\begin{verbatim}
%\documentclass{article}
%\usepackage[english]{babel}
%\usepackage[calc]{datetime2}
%\begin{document}
%\DTMmonthname{12}
%\end{document}
%\end{verbatim}
%Here \cs{languagename} is \texttt{english}, so this
%uses \cs{DTMenglishmonthname} which is defined by the
%\module{english} module.
%
%If the test with \cs{languagename} didn't work, \cs{DTMmonthname}
%then tries with \meta{lang} set to
%\begin{verbatim}
%\TrackedLanguageFromDialect{\languagename}
%\end{verbatim}
%(\cs{TrackedLanguageFromDialect} is provided by the \sty{tracklang}
%package.) If \cs{DTM\meta{lang}monthname} exists, then this command is used. 
%For example:
%\begin{verbatim}
%\documentclass{article}
%\usepackage[british]{babel}
%\usepackage[calc]{datetime2}
%\begin{document}
%\DTMmonthname{12}
%\end{document}
%\end{verbatim}
%Here \cs{languagename} is \texttt{british}, so this again uses
%\cs{DTMenglishmonthname} as the root language is obtained from the
%dialect to language mapping.
%
%\begin{important}
%Note that this won't work if you confuse \sty{tracklang} by using an 
%alternative dialect name in the class option or by directly loading 
%\sty{tracklang} with different dialect labels.
%\end{important}
%
%In the event that neither of those commands exist,
%\cs{DTMmonthname} will fallback on \cs{pgfcalendarmonthname}
%(provided by the \sty{pgfcalendar} package). This will also issue a
%warning. For example:
%\begin{verbatim}
%\documentclass{article}
%\usepackage[calc]{datetime2}
%\begin{document}
%\DTMmonthname{12}
%\end{document}
%\end{verbatim}
%This produces the warning message:
%\begin{verbatim}
%Can't find underlying language macro for
%\DTMmonthname (language: english); 
%using pgfcalendar macro instead
%\end{verbatim}
%and uses \cs{pgfcalendarmonthname} instead of
%\cs{DTMenglishmonthname} (which hasn't been defined because the
%\module{english} module \emph{hasn't been loaded}).
%
%You can switch off the warning by setting the \pkgopt{warn} option
%to false or by redefining \cs{dtmnamewarning} to ignore its
%argument.
%
%\begin{definition}[\DescribeMacro\DTMMonthname]
%\cs{DTMMonthname}\marg{n}
%\end{definition}
%This checks if \cs{DTM\meta{lang}Monthname} exists. First where
%\meta{lang} is given by \cs{languagename} and then where \meta{lang}
%is given by
%\begin{verbatim}
%\TrackedLanguageFromDialect{\languagename}
%\end{verbatim}
%If neither of those values of \meta{lang} match a defined command,
%\cs{DTMMonthname} then tests for non-case-changing versions
%\cs{DTM\meta{lang}monthname}.
%This is because not all language modules provide a macro for use at the
%start of a sentence since some languages always start month names
%with a capital letter. For example, the
%\module{english} module provides \cs{DTMenglishmonthname} but
%doesn't provide an upper case alternative, since English month names
%always start with a capital. Therefore:
%\begin{verbatim}
%\documentclass{article}
%\usepackage[english]{babel}
%\usepackage[calc]{datetime2}
%\begin{document}
%\DTMMonthname{12}
%\end{document}
%\end{verbatim}
%just uses \cs{DTMenglishmonthname}.
%
%As before, if the relevant command can't be detected for any case
%of \meta{lang} for either \cs{DTM\meta{lang}Monthname} or
%\cs{DTM\meta{lang}monthname} (for example, the required language module hasn't
%been loaded) then \cs{DTMMonthname} will use
%\cs{pgfcalendarmonthname} and attempt to convert the first letter to
%upper case.
%
%Since some \emph{but not all} language modules also provide month
%name abbreviations, the \styfmt{datetime2-calc} package also provides:
%\begin{definition}[\DescribeMacro\DTMshortmonthname]
%\cs{DTMshortmonthname}\marg{n}
%\end{definition}
%This behaves in a similar way to \cs{DTMmonthname} but tries to
%determine if \cs{DTM\meta{lang}shortmonthname} exists (where
%\meta{lang} is either \cs{languagename} or obtained from
%\cs{languagename} using \sty{tracklang}'s dialect to language
%mapping). If no command can be found, the fallback uses
%\cs{pgfcalendarmonthshortname} provided by the \sty{pgfcalendar}
%package. For example, if the language module hasn't been loaded or
%if the language module doesn't provide an abbreviated version.
%
%Similarly there is a version for the start of a sentence for
%languages that normally use lower case month names:
%\begin{definition}[\DescribeMacro\DTMshortMonthname]
%\cs{DTMshortMonthname}\marg{n}
%\end{definition}
%
%There are also analogous commands for the weekday names, where
%\meta{n} is an integer from 0 (Monday) to 6 (Sunday). This index can
%be computed using:
%\begin{definition}[\DescribeMacro\DTMcomputedayofweekindex]
%\cs{DTMcomputedayofweekindex}\marg{date}\marg{cs}
%\end{definition}
%where \meta{date} is in the form \meta{YYYY}-\meta{MM}-\meta{DD} and
%\meta{cs} is a control sequence in which to store the result.
%Remember that date styles automatically access the day of week index
%from the fourth argument of \cs{DTMdisplaydate}, so this command
%shouldn't be used within a date style.
%
%\begin{definition}[\DescribeMacro\DTMweekdayname]
%\cs{DTMweekdayname}\marg{n}
%\end{definition}
%This checks if \cs{DTM\meta{lang}weekdayname} exists where
%\meta{lang} is given by \cs{languagename}.
%If it does, that macro is used. For example:
%\begin{verbatim}
%\documentclass{article}
%\usepackage[english]{babel}
%\usepackage[calc]{datetime2}
%\begin{document}
%\DTMweekdayname{6}
%\end{document}
%\end{verbatim}
%This uses \cs{DTMenglishweekdayname}, which is provided by the
%\module{english} module.
%
%If \cs{DTM\meta{lang}weekdayname} doesn't exist with \meta{lang} set
%to \cs{languagename}, \cs{DTMweekdayname} then tests with
%\meta{lang} set to:
%\begin{verbatim}
%\TrackedLanguageFromDialect{\languagename}
%\end{verbatim}
%If the command \cs{DTM\meta{lang}weekdayname} exists in this case, that 
%command is used.  For example:
%\begin{verbatim}
%\documentclass{article}
%\usepackage[british]{babel}
%\usepackage[calc]{datetime2}
%\begin{document}
%\DTMweekdayname{6}
%\end{document}
%\end{verbatim}
%This uses \cs{DTMenglishweekdayname} as it can determine (through
%\sty{tracklang}) that \texttt{british} has been defined as a dialect
%of \texttt{english}.
%
%If this second test fails, then \cs{DTMweekdayname} will issue a
%warning and fallback on \cs{pgfcalendarweekdayname}, provided by the
%\sty{pgfcalendar} package. For example:
%\begin{verbatim}
%\documentclass{article}
%\usepackage[calc]{datetime2}
%\begin{document}
%\DTMweekdayname{6}
%\end{document}
%\end{verbatim}
%This produces the warning message:
%\begin{verbatim}
%Can't find underlying language macro for
%\DTMweekdayname (language: english);
%using pgfcalendar macro instead
%\end{verbatim}
%As before, this warning is produced with \cs{dtmnamewarning}.
%
%The first letter upper case version is:
%\begin{definition}[\DescribeMacro\DTMWeekdayname]
%\cs{DTMWeekdayname}\marg{n}
%\end{definition}
%As with \cs{DTMMonthname}, if no upper case version can be found in
%the relevant language module this will use the non-case-changing
%version. The fallback \sty{pgfcalendar} macro is again
%\cs{pgfcalendarweekdayname} with an attempt to convert the first
%letter to upper case.
%
%Again, abbreviations may or may not be supported by language
%modules. If they're not supported, the fallback is
%\cs{pgfcalendarweekdayshortname}.
%
%\begin{definition}[\DescribeMacro\DTMshortweekdayname]
%\cs{DTMshortweekdayname}\marg{n}
%\end{definition}
%which will attempt to use \cs{DTM\meta{lang}shortweekdayname} and
%\begin{definition}[\DescribeMacro\DTMshortWeekdayname]
%\cs{DTMshortWeekdayname}\marg{n}
%\end{definition}
%which will attempt to use \cs{DTM\meta{lang}shortWeekdayname} or
%\cs{DTM\meta{lang}shortweekdayname}.
%
%For completeness, there's also a language-sensitive date ordinal command:
%\begin{definition}[\DescribeMacro\DTMordinal]
%\cs{DTMordinal}\marg{n}
%\end{definition}
%where \meta{n} is a number from 1 to 31. Again, this shouldn't be
%used in date styles, but only if a standalone date ordinal is required.
%For most languages, this only has a suffix for the first day of the
%month (that is where \meta{n} is 1) or the suffix may simply be a
%full stop (period). It should not be confused with \sty{fmtcount}'s
%\cs{ordinalnum} command, which is for general ordinals rather than
%date-specific ordinals.
%
%\chapter{Migrating from \styfmt{datetime}}
%\label{sec:migrate}
%
%This section is for users who want to switch over from the old
%\styfmt{datetime} package.
%
%Note that \styfmt{datetime2} is modularised for improved efficiency both in
%terms of package overheads and spreading the maintenance load. This
%means that you only need to install \styfmt{datetime2} if you only want
%the base numeric styles, but it you want multilingual or regional
%support, you need to additionally load the required language module
%or modules.
%
%For example, consider the following document that uses
%\styfmt{datetime}:
%\begin{verbatim}
%\documentclass{article}
%\usepackage[T1]{fontenc}
%\usepackage[utf8]{inputenc}
%\usepackage[french]{babel}
%\usepackage{datetime}
%\begin{document}
%\today.
%\end{document}
%\end{verbatim}
%This just needs to have the \styfmt{datetime} package installed, which
%includes the necessary file \texttt{dt-french.def}.
%
%This example can be adjusted for \styfmt{datetime2}:
%\begin{verbatim}
%\documentclass{article}
%\usepackage[T1]{fontenc}
%\usepackage[utf8]{inputenc}
%\usepackage[french]{babel}
%\usepackage[useregional]{datetime2}
%\begin{document}
%\today.
%\end{document}
%\end{verbatim}
%This requires both \styfmt{datetime2} and the \module{french} module,
%so you need to ensure that both have been installed.
%
%The above example also draws attention to another change from
%\styfmt{datetime} and that concerns the default package behaviour.  The
%\styfmt{datetime} package defaults to a British date style, unless
%\sty{babel} has been loaded first, whereas the \styfmt{datetime2}
%package defaults to an ISO numeric style, unless language or
%regional settings are provided in the class option. As illustrated
%in the above, you need the \pkgopt{useregional} option if you want
%\cs{datefrench} (or equivalent) to switch the date style.
%
%\section{\styfmt{datetime} package options}
%
%The \styfmt{datetime} package provides the following options, which can
%be emulated with \styfmt{datetime2} or through one of its dependent
%modules or packages:
%
%\begin{description}
%\item[\pkgoptfmt{long}] This option was designed for full British
%dates, and is the default if \sty{babel} isn't loaded. Example:
%\begin{verbatim}
%\documentclass{article}
%\usepackage[long]{datetime}
%\begin{document}
%\today
%\end{document}
%\end{verbatim}
%This produces the date in the form: 
%Wednesday 20\textsuperscript{th} January, 2016.
%
%This can be achieved with \styfmt{datetime2} and the \module{english}
%module as follows:
%\begin{verbatim}
%\documentclass{article}
%\usepackage[en-GB,showdow]{datetime2}
%\DTMlangsetup[en-GB]{ord=raise,monthyearsep={,\space}}
%\begin{document}
%\today
%\end{document}
%\end{verbatim}
%
%\item[\pkgoptfmt{short}] This option was designed for abbreviated
%British dates. For example:
%\begin{verbatim}
%\documentclass{article}
%\usepackage[short]{datetime}
%\begin{document}
%\today
%\end{document}
%\end{verbatim}
%This produces the date in the form:
%Wed 20\textsuperscript{th} Jan, 2016
%
%This can be achieved with \styfmt{datetime2} and the \module{english}
%module as follows:
%\begin{verbatim}
%\documentclass{article}
%\usepackage[en-GB,showdow]{datetime2}
%\DTMlangsetup[en-GB]{abbr,ord=raise,monthyearsep={,\space}}
%\begin{document}
%\today
%\end{document}
%\end{verbatim}
%
%\item[\pkgoptfmt{iso}] This option was designed for
%\meta{YYYY}-\meta{MM}-\meta{DD} dates. For example:
%\begin{verbatim}
%\documentclass{article}
%\usepackage[iso]{datetime}
%\begin{document}
%\today
%\end{document}
%\end{verbatim}
%This produces the date in the form: 2016-01-20. This is default for
%\styfmt{datetime2}:
%\begin{verbatim}
%\documentclass{article}
%\usepackage{datetime2}
%\begin{document}
%\today
%\end{document}
%\end{verbatim}
%If you don't want the style to depend on the separator settings, you
%can use the \style{iso} style:
%\begin{verbatim}
%\documentclass{article}
%\usepackage[style=iso]{datetime2}
%\begin{document}
%\today
%\end{document}
%\end{verbatim}
%
%\item[\pkgoptfmt{yyyymmdd}] This option was designed for
%\meta{YYYY}\slash\meta{MM}\slash\meta{DD} dates. For example:
%\begin{verbatim}
%\documentclass{article}
%\usepackage[yyyymmdd]{datetime}
%\begin{document}
%\today
%\end{document}
%\end{verbatim}
%This produces the date in the form: 2016/01/20. This is very nearly
%in the same form as the default \styfmt{datetime2} style. All that
%needs changing are the separators between the year, month and day:
%\begin{verbatim}
%\documentclass{article}
%\usepackage[datesep=/]{datetime2}
%\begin{document}
%\today
%\end{document}
%\end{verbatim}
%
%\item[\pkgoptfmt{ddmmyyyy}] This option was designed for
%\meta{DD}\slash\meta{MM}\slash\meta{YYYY} dates. For example:
%\begin{verbatim}
%\documentclass{article}
%\usepackage[ddmmyyyy]{datetime}
%\begin{document}
%\today
%\end{document}
%\end{verbatim}
%This produces the date in the form: 20/01/2016. This can be changed
%to \styfmt{datetime2} with a couple of package options:
%\begin{verbatim}
%\documentclass{article}
%\usepackage[datesep=/,style=ddmmyyyy]{datetime}
%\begin{document}
%\today
%\end{document}
%\end{verbatim}
%
%\item[\pkgoptfmt{dmyyyy}] This option was designed for
%\meta{D}\slash\meta{M}\slash\meta{YYYY} dates. For example:
%\begin{verbatim}
%\documentclass{article}
%\usepackage[ddmmyyyy]{datetime}
%\begin{document}
%\today
%\end{document}
%\end{verbatim}
%This produces the date in the form: 20/1/2016.  This can be changed
%to \styfmt{datetime2} with a couple of package options:
%\begin{verbatim}
%\documentclass{article}
%\usepackage[datesep=/,style=dmyyyy]{datetime}
%\begin{document}
%\today
%\end{document}
%\end{verbatim}
%
%\item[\pkgoptfmt{ddmmyy}] This option was designed for
%\meta{DD}\slash\meta{MM}\slash\meta{YY} dates. For example:
%\begin{verbatim}
%\documentclass{article}
%\usepackage[ddmmyy]{datetime}
%\begin{document}
%\today
%\end{document}
%\end{verbatim}
%This produces the date in the form: 20/01/16.  This can be changed
%to \styfmt{datetime2} with a couple of package options:
%\begin{verbatim}
%\documentclass{article}
%\usepackage[datesep=/,style=ddmmyy]{datetime}
%\begin{document}
%\today
%\end{document}
%\end{verbatim}
%
%\item[\pkgoptfmt{dmyy}] This option was designed for
%\meta{D}\slash\meta{M}\slash\meta{YY} dates. For example:
%\begin{verbatim}
%\documentclass{article}
%\usepackage[dmyy]{datetime}
%\begin{document}
%\today
%\end{document}
%\end{verbatim}
%This produces the date in the form: 20/1/16.  This can be changed
%to \styfmt{datetime2} with a couple of package options:
%\begin{verbatim}
%\documentclass{article}
%\usepackage[datesep=/,style=dmyy]{datetime}
%\begin{document}
%\today
%\end{document}
%\end{verbatim}
%
%\item[\pkgoptfmt{text}] This option was designed for a full UK textual
%date. For example:
%\begin{verbatim}
%\documentclass{article}
%\usepackage[text]{datetime}
%\begin{document}
%\today
%\end{document}
%\end{verbatim}
%This produces the date in the form: 
%Wednesday the Twentieth of January, Two Thousand and Sixteen
%
%This document can be changed to \styfmt{datetime2} through the
%\sty{datetime2-en-fulltext} package, which needs to be installed
%separately:
%\begin{verbatim}
%\documentclass{article}
%\usepackage{datetime2-en-fulltext}
%\DTMsetdatestyle{en-FullText}
%\begin{document}
%\today
%\end{document}
%\end{verbatim}
%
%\item[\pkgoptfmt{us}] This option was designed for the standard US
%date. For example:
%\begin{verbatim}
%\documentclass{article}
%\usepackage[us]{datetime}
%\begin{document}
%\today
%\end{document}
%\end{verbatim}
%This produces the date in the form: 
%January 20, 2016
%
%This can be achieved with \styfmt{datetime2} and the \module{english}
%module as follows:
%\begin{verbatim}
%\documentclass{article}
%\usepackage[en-US]{datetime2}
%\begin{document}
%\today
%\end{document}
%\end{verbatim}
%
%\item[\pkgoptfmt{mmddyyyy}] This option was designed for
%\meta{MM}\slash\meta{DD}\slash\meta{YYYY} dates. For example:
%\begin{verbatim}
%\documentclass{article}
%\usepackage[mmddyyyy]{datetime}
%\begin{document}
%\today
%\end{document}
%\end{verbatim}
%This produces the date in the form: 01/20/2016.  This can be changed
%to \styfmt{datetime2} with a couple of package options:
%\begin{verbatim}
%\documentclass{article}
%\usepackage[datesep=/,style=mmddyyyy]{datetime}
%\begin{document}
%\today
%\end{document}
%\end{verbatim}
%
%\item[\pkgoptfmt{mdyyyy}] This option was designed for
%\meta{M}\slash\meta{D}\slash\meta{YYYY} dates. For example:
%\begin{verbatim}
%\documentclass{article}
%\usepackage[mmddyyyy]{datetime}
%\begin{document}
%\today
%\end{document}
%\end{verbatim}
%This produces the date in the form: 1/20/2016.  This can be changed
%to \styfmt{datetime2} with a couple of package options:
%\begin{verbatim}
%\documentclass{article}
%\usepackage[datesep=/,style=mdyyyy]{datetime}
%\begin{document}
%\today
%\end{document}
%\end{verbatim}
%
%\item[\pkgoptfmt{mmddyy}] This option was designed for
%\meta{MM}\slash\meta{DD}\slash\meta{YY} dates. For example:
%\begin{verbatim}
%\documentclass{article}
%\usepackage[mmddyy]{datetime}
%\begin{document}
%\today
%\end{document}
%\end{verbatim}
%This produces the date in the form: 01/20/16.  This can be changed
%to \styfmt{datetime2} with a couple of package options:
%\begin{verbatim}
%\documentclass{article}
%\usepackage[datesep=/,style=mmddyy]{datetime}
%\begin{document}
%\today
%\end{document}
%\end{verbatim}
%
%\item[\pkgoptfmt{mdyy}] This option was designed for
%\meta{M}\slash\meta{D}\slash\meta{YY} dates. For example:
%\begin{verbatim}
%\documentclass{article}
%\usepackage[mdyy]{datetime}
%\begin{document}
%\today
%\end{document}
%\end{verbatim}
%This produces the date in the form: 1/20/16.  This can be changed
%to \styfmt{datetime2} with a couple of package options:
%\begin{verbatim}
%\documentclass{article}
%\usepackage[datesep=/,style=mdyy]{datetime}
%\begin{document}
%\today
%\end{document}
%\end{verbatim}
%
%\item[\pkgoptfmt{raise}] This option was designed to make the ordinal
%st,nd,rd,th appear as a subscript. It was originally just intended for British
%dates. This is one of the default settings for \styfmt{datetime}. For example:
%\begin{verbatim}
%\documentclass{article}
%\usepackage[raise]{datetime}
%
%\begin{document}
%\today
%\end{document}
%\end{verbatim}
%This produces the date in the form: 
%Wednesday 20\textsuperscript{th} January, 2016
%
%With \styfmt{datetime2}, this setting \emph{may} be provided by a language
%module, where appropriate. For example:
%\begin{verbatim}
%\documentclass{article}
%\usepackage[en-GB,showdow]{datetime2}
%\DTMlangsetup[en-GB]{ord=raise}
%
%\begin{document}
%\today
%\end{document}
%\end{verbatim}
%
%\item[\pkgoptfmt{level}] This option was designed to make the ordinal
%st,nd,rd,th appear level with the rest of the text (to counteract
%the previous option). For example:
%\begin{verbatim}
%\documentclass{article}
%\usepackage[level]{datetime}
%
%\begin{document}
%\today
%\end{document}
%\end{verbatim}
%This produces the date in the form: Wednesday 20th January, 2016
%
%With \styfmt{datetime2} this setting may be provided by a language
%module, where appropriate. This is, in fact, the default setting for
%the \style{en-GB} style:
%\begin{verbatim}
%\documentclass{article}
%\usepackage[en-GB,showdow]{datetime2}
%
%\begin{document}
%\today
%\end{document}
%\end{verbatim}
%but can be explicitly set using:
%\begin{verbatim}
%\DTMlangsetup[en-GB]{ord=level}
%\end{verbatim}
%
%\item[\pkgoptfmt{dayofweek}] This option was designed to show the
%weekday name for those styles that supported it. This is one of the
%default \styfmt{datetime} settings. For example:
%\begin{verbatim}
%\documentclass{article}
%\usepackage[dayofweek]{datetime}
%
%\begin{document}
%\today
%\end{document}
%\end{verbatim}
%This produces the date in the form: 
%Wednesday 20\textsuperscript{th} January, 2016
%
%With \styfmt{datetime2} this setting \emph{may} be provided by a language
%module, where supported. For example:
%\begin{verbatim}
%\documentclass{article}
%\usepackage[en-GB,showdow]{datetime2}
%
%\begin{document}
%\today
%\end{document}
%\end{verbatim}
%Some language modules don't support this option (just as some
%language settings with \styfmt{datetime} also don't support the
%\pkgoptfmt{dayofweek} option.)
%
%\item[\pkgoptfmt{nodayofweek}] This option was designed to hide the
%weekday name for those styles that supported it (to counteract the
%previous option). For example:
%\begin{verbatim}
%\documentclass{article}
%\usepackage[nodayofweek]{datetime}
%
%\begin{document}
%\today
%\end{document}
%\end{verbatim}
%This produces the date in the form: 
%20\textsuperscript{th} January, 2016
%
%This setting is the default for \styfmt{datetime2}.
%However, if it \pkgopt{showdow} has been switched on, it can later
%be switched off using
%\begin{verbatim}
%\DTMsetup{showdow=false}
%\end{verbatim}
%
%\item[\pkgoptfmt{hhmmss}] This option was designed for time formats in
%the style \meta{HH}:\meta{MM}:\meta{SS}. For example:
%\begin{verbatim}
%\documentclass{article}
%\usepackage[hhmmss]{datetime}
%
%\begin{document}
%\currenttime
%\end{document}
%\end{verbatim}
%This produces the time in the form: 17:28:52.
%
%This is the default for \styfmt{datetime2}:
%\begin{verbatim}
%\documentclass{article}
%\usepackage{datetime2}
%
%\begin{document}
%\DTMcurrenttime
%\end{document}
%\end{verbatim}
%
%\item[\pkgoptfmt{24hr}] This option was designed for 24 hour time formats in
%the style \meta{HH}:\meta{MM}. For example:
%\begin{verbatim}
%\documentclass{article}
%\usepackage[24hr]{datetime}
%
%\begin{document}
%\currenttime
%\end{document}
%\end{verbatim}
%This produces the time in the form: 17:28.
%
%This can be achieved using the default \styfmt{datetime2} time style
%with the seconds suppressed:
%\begin{verbatim}
%\documentclass{article}
%\usepackage[showseconds=false]{datetime2}
%
%\begin{document}
%\DTMcurrenttime
%\end{document}
%\end{verbatim}
%
%\item[\pkgoptfmt{12hr}] This option was designed for 12 hour time
%formats with \qt{am} or \qt{pm} suffixes. For example:
%\begin{verbatim}
%\documentclass{article}
%\usepackage[12hr]{datetime}
%
%\begin{document}
%\currenttime
%\end{document}
%\end{verbatim}
%This produces the time in the form: 5:28pm.
%
%This can be achieved through a \styfmt{datetime2} language module that
%supports this format. For example, the \module{english} module:
%\begin{verbatim}
%\documentclass{article}
%\usepackage[en-GB]{datetime2}
%
%\begin{document}
%\DTMcurrenttime
%\end{document}
%\end{verbatim}
%
%\item[\pkgoptfmt{oclock}] This option was designed for a UK-style full text
%time. For example:
%\begin{verbatim}
%\documentclass{article}
%\usepackage[oclock]{datetime}
%
%\begin{document}
%\currenttime
%\end{document}
%\end{verbatim}
%This produces the time in the form:
%Twenty minutes past Six in the afternoon
%
%This can be changed to \styfmt{datetime2} through the
%\sty{datetime2-en-fulltext} package, which needs to be installed
%separately:
%\begin{verbatim}
%\documentclass{article}
%\usepackage{datetime2-en-fulltext}
%\DTMsettimestyle{en-FullText}
%\begin{document}
%\DTMcurrenttime
%\end{document}
%\end{verbatim}
%
%\end{description}
%
%The \styfmt{datetime} package option \pkgoptfmt{nodate} was provided
%before multilingual support was added to allow \sty{babel} users to
%use the time commands without causing a conflict with the date. Once
%support for the \sty{babel} package was added, this option became
%superfluous.
%
%\section{Time and Date Commands}
%
%\begin{definition}
%\ics{today}
%\end{definition}
%This is a robust command in later versions of \styfmt{datetime}. In
%earlier versions is was fragile and had to be protected when used in
%moving arguments. With \styfmt{datetime2}, this command is designed to
%be expandable and so therefore is not robust and shouldn't need
%protecting. (The date styles should take care of any fragile
%commands, such as \cs{textsuperscript}, where necessary.)
%
%The current time is displayed with \styfmt{datetime} using
%\begin{definition}[\inpkg{datetime}]
%\oldcs{currenttime}
%\end{definition}
%The \styfmt{datetime2} equivalent is:
%\begin{definition}[\inpkg{datetime2}]
%\ics{DTMcurrenttime}
%\end{definition}
%Again, the command provided by \styfmt{datetime} is robust and the
%equivalent command provided by \styfmt{datetime2} is designed to be
%expandable.
%
%A specific date is display by \styfmt{datetime}'s 
%\begin{definition}[\inpkg{datetime}]
%\oldcs{formatdate}\marg{DD}\marg{MM}\marg{YYYY}
%\end{definition}
%command. The arguments are in little-endian (UK) order with the day,
%month and year. With \styfmt{datetime2} you can use either:
%\begin{definition}[\inpkg{datetime2}]
%\ics{DTMdisplay}\marg{YYYY}\marg{MM}\marg{DD}\marg{dow}
%\end{definition}
%(expandable) or
%\begin{definition}[\inpkg{datetime2}]
%\ics{DTMdate}\{\meta{YYYY}-\meta{MM}-\meta{DD}\}
%\end{definition}
%(robust) where \meta{YYYY} is the year, \meta{MM} is the month
%number, \meta{DD} is the day of month number and \meta{dow} is the
%day of week number (starting from 0 for Monday) or \texttt{-1} to
%disregard it.
%
%A specific time is displayed by \styfmt{datetime}'s
%\begin{definition}[\inpkg{datetime}]
%\oldcs{formattime}\marg{hh}\marg{mm}\marg{ss}
%\end{definition}
%command, which has three arguments: the hour (24) the minutes past
%the hour and the seconds past the minute. With \styfmt{datetime2} you
%can use either:
%\begin{definition}[\inpkg{datetime2}]
%\ics{DTMdisplaytime}\marg{hh}\marg{mm}\marg{ss}
%\end{definition}
%(expandable) or
%\begin{definition}[\inpkg{datetime2}]
%\ics{DTMtime}\{\meta{hh}:\meta{mm}:\meta{ss}\}
%\end{definition}
%(robust) where \meta{hh} is the hour, \meta{mm} is the minutes and
%\meta{ss} is the seconds.
%
%The date separator used by the predefined \styfmt{datetime} styles is
%given by
%\begin{definition}[\inpkg{datetime}]
%\oldcs{dateseparator}
%\end{definition}
%which needs to be redefined if
%required. With \styfmt{datetime2}, the date separator for the base
%numeric styles (except the fixed \style{iso} style) can be changed
%through the \pkgopt{datesep} package option. For example:
%\begin{verbatim}
%\usepackage[datesep={.}]{datetime2}
%\end{verbatim}
%or
%\begin{verbatim}
%\DTMsetup{datesep={.}}
%\end{verbatim}
%Some of the language modules may also provide a similar option. For
%example:
%\begin{verbatim}
%\usepackage[en-GB]{datetime2}
%\DTMlangsetup[en-GB]{datesep={.}}
%\end{verbatim}
%
%The time separator for \styfmt{datetime} is given by
%\begin{definition}[\inpkg{datetime}]
%\oldcs{timeseparator}
%\end{definition}
%With \styfmt{datetime2}, the time separator for the
%basic numeric styles (not including the fixed \style{iso} style) can be
%changed through the \pkgopt{timesep} package option. For example:
%\begin{verbatim}
%\usepackage[timesep={.}]{datetime2}
%\end{verbatim}
%or
%\begin{verbatim}
%\usepackage{datetime2}
%\DTMsetup{timesep={.}}
%\end{verbatim}
%Some of the language modules may also provide a similar option. For
%example:
%\begin{verbatim}
%\usepackage[en-GB]{datetime2}
%\DTMlangsetup[en-GB]{timesep={.}}
%\end{verbatim}
%
%\begin{definition}[\inpkg{datetime}]
%\oldcs{pdfdate}
%\end{definition}
%The \cs{pdfdate} command provided by \styfmt{datetime} for use within
%\ics{pdfinfo} became redundant with the introduction of
%\ics{pdfcreationdate} to PDF\TeX\ version 1.30.0.
%
%Old style (using \styfmt{datetime}):
%\begin{verbatim}
%\pdfinfo{
%  /Author (Me)
%  /Title (A Sample Document)
%  /CreationDate (D:20040501215500)
%  /ModDate (D:\pdfdate)
%}
%\end{verbatim}
%New style (simply using PDF\TeX):
%\begin{verbatim}
%\pdfinfo{
%  /Author (Me)
%  /Title (A Sample Document)
%  /CreationDate (D:20040501215500)
%  /ModDate (\pdfcreationdate)
%}
%\end{verbatim}
%Alternatively you can use the \style{pdf} style:
%\begin{verbatim}
%\DTMsetstyle{pdf}
%\pdfinfo{
%  /Author (Me)
%  /Title (A Sample Document)
%  /CreationDate (\DTMdisplay{2004}{05}{01}{-1}{21}{55}{00}{00}{00})
%  /ModDate (\DTMnow)
%}
%\end{verbatim}
%
%The \styfmt{datetime} command to display the month name is
%\begin{definition}[\inpkg{datetime}]
%\oldcs{monthname}\oarg{n}
%\end{definition}
%which can't be expanded. With
%\styfmt{datetime2}, the month name for a specific language can be
%obtained from a command provided by the relevant module. For
%example, the \module{english} module provides
%\begin{definition}[\inpkg{datetime2-english}]
%\ics{DTMenglishmonthname}\marg{n}
%\end{definition}
%which is expandable. These types of commands are designed for use
%within language-dependent date styles. This ensures that the name
%matches the style. (One of the failings of \styfmt{datetime} was that
%the original date styles provided when the package was originally
%only intended for British dates produced weird hybrid styles when
%multilingual support was later added and styles such as \pkgoptfmt{long} 
%were used with another language.) These types of commands provided
%by the \styfmt{datetime2} language modules are analogous to the
%\cs{monthname\meta{language}} commands provided by \styfmt{datetime}'s
%supporting language files (for example, 
%\begin{definition}[\inpkg{datetime-defaults}]
%\oldcs{monthnameenglish}\oarg{n}
%\end{definition}
%in \styfmt{datetime-defaults} or 
%\begin{definition}[\inpkg{dt-french.def}]
%\oldcs{monthnamefrench}\oarg{n}
%\end{definition}
%defined in \styfmt{dt-french.def}) except that the \styfmt{datetime} commands have
%an optional argument which means they're not expandable.
%
%If you load the \styfmt{datetime2-calc} package,
%either explicitly or through the \pkgopt{calc} or \pkgopt{showdow}
%\styfmt{datetime2} package options, then \sty{pgfcalendar} will also be
%loaded. In which case you can use
%\begin{definition}[\inpkg{pgfcalendar}]
%\ics{pgfcalendarmonthname}\marg{n}
%\end{definition}
%even if none of the \styfmt{datetime2} language modules have been
%loaded. This command requires the \sty{translator} package to
%provide multilingual support. See the \styfmt{pgf} manual for further
%details.
%
%Another possibility if you want the month name alone using the
%current language is to use the robust command
%\begin{definition}[\inpkg{datetime2-calc}]
%\ics{DTMmonthname}\marg{n}
%\end{definition}
%defined by \styfmt{datetime2-calc}. This is the closest match to
%\styfmt{datetime}'s \oldcs{monthname} command but note that the argument
%isn't optional. Remember that you can use \ics{month} for the current
%month number, which is the value of \meta{n} when omitted in
%\oldcs{monthname}.
%
%Old style (\styfmt{datetime}):
%\begin{verbatim}
%\documentclass{article}
%
%\usepackage[T1]{fontenc}
%\usepackage[utf8]{inputenc}
%\usepackage[french,english]{babel}
%
%\usepackage{datetime}
%
%\begin{document}
%\selectlanguage{french}
%\monthname
%
%\selectlanguage{english}
%\monthname
%\end{document}
%\end{verbatim}
%New style (\styfmt{datetime2} with the \module{french} and
%\module{english} modules):
%\begin{verbatim}
%\documentclass{article}
%
%\usepackage[T1]{fontenc}
%\usepackage[utf8]{inputenc}
%\usepackage[french,english]{babel}
%
%\usepackage[calc]{datetime2}
%
%\begin{document}
%\selectlanguage{french}
%\DTMmonthname{\month}
%
%\selectlanguage{english}
%\DTMmonthname{\month}
%\end{document}
%\end{verbatim}
%
%The abbreviated month name is given by 
%\begin{definition}[\inpkg{datetime}]
%\oldcs{shortmonthname}\oarg{n} 
%\end{definition}
%in
%\styfmt{datetime}. Similar to above, language modules may provide an
%expandable command to produce the abbreviated name in that specific
%language. For example, the \module{english} module provides
%\begin{definition}[\inpkg{datetime2-english}]
%\ics{DTMenglishshortmonthname}\marg{n}
%\end{definition}
%As with \cs{DTMenglishmonthname}, this is designed for use in
%English date styles.
%
%Again, if \styfmt{datetime2-calc} is loaded, the \sty{pgfcalendar}
%command is also available:
%\begin{definition}[\inpkg{pgfcalendar}]
%\ics{pgfcalendarmonthshortname}\marg{n}
%\end{definition}
%Alternative you can use the robust command defined by
%\styfmt{datetime2-calc}:
%\begin{definition}[\inpkg{datetime2-calc}]
%\ics{DTMshortmonthname}\marg{n}
%\end{definition}
%
%Old style (\styfmt{datetime}):
%\begin{verbatim}
%\documentclass{article}
%
%\usepackage[T1]{fontenc}
%\usepackage[utf8]{inputenc}
%\usepackage[french,english]{babel}
%
%\usepackage{datetime}
%
%\begin{document}
%\selectlanguage{french}
%\shortmonthname
%
%\selectlanguage{english}
%\shortmonthname
%\end{document}
%\end{verbatim}
%New style (\styfmt{datetime2} with the \module{french} and
%\module{english} modules):
%\begin{verbatim}
%\documentclass{article}
%
%\usepackage[T1]{fontenc}
%\usepackage[utf8]{inputenc}
%\usepackage[french,english]{babel}
%
%\usepackage[calc]{datetime2}
%
%\begin{document}
%\selectlanguage{french}
%\DTMshortmonthname{\month}
%
%\selectlanguage{english}
%\DTMshortmonthname{\month}
%\end{document}
%\end{verbatim}
%
%See \sectionref{sec:calc} and \sectionref{sec:names} for further
%details.
%
%The weekday names in \styfmt{datetime} are more complicated 
%and not all the \texttt{dt-\meta{lang}.def} files provide translations.  The ones
%that do support the weekday provide 
%\begin{definition}[\inpkg{dt-\meta{lang}.def}]
%\oldcs{dayofweeknameid\meta{lang}}\marg{n}
%\end{definition}
%which takes a single argument that's an integer from 1 (Sunday) to 7 (Saturday).
%For example, 
%\begin{definition}[\inpkg{datetime-defaults}]
%\oldcs{dayofweeknameidenglish}\marg{n}
%\end{definition}
%or
%\begin{definition}[\inpkg{dt-french.def}]
%\oldcs{dayofweeknameidfrench}\marg{n}
%\end{definition}
%
%The \styfmt{datetime2} language modules that support the weekday name
%provide 
%\begin{definition}[\inpkg{datetime2-\meta{lang}}]
%\cs{DTM\meta{lang}weekdayname}\marg{dow}
%\end{definition}
%which takes a single argument that's an integer from 0 (Monday) to 6
%(Sunday). As with \styfmt{datetime}, not all of the \styfmt{datetime2}
%language modules support the weekday.
%
%Old style (\styfmt{datetime}):
%\begin{verbatim}
%\documentclass{article}
%\usepackage[T1]{fontenc}
%\usepackage[utf8]{inputenc}
%\usepackage[french,english]{babel}
%\usepackage{datetime}
%
%\begin{document}
%\dayofweeknameidfrench{1}
%
%\dayofweeknameidenglish{1}
%\end{document}
%\end{verbatim}
%New style (\styfmt{datetime2} with the \module{english} and
%\module{french} modules):
%\begin{verbatim}
%\documentclass{article}
%\usepackage[T1]{fontenc}
%\usepackage[utf8]{inputenc}
%\usepackage[french,english]{babel}
%\usepackage{datetime2}
%
%\begin{document}
%\DTMfrenchweekdayname{6}
%
%\DTMenglishweekdayname{6}
%\end{document}
%\end{verbatim}
%
%As with the month name, date styles should explicitly use the
%weekday macro (if provided) for the specific language to display 
%the weekday name rather than using a macro that varies according to
%the current language.
%
%The \styfmt{datetime} package provides 
%\begin{definition}[\inpkg{datetime}]
%\oldcs{dayofweeknameid}\marg{n}
%\end{definition}
%to provide the weekday name for the current language. If the current language
%doesn't provide a translation for the weekday names, then the
%English names are used. This command basically attempts to use
%\cs{dayofweeknameid\meta{lang}} (where \meta{lang} is given
%by \cs{languagename}) if it exists otherwise it uses
%\oldcs{dayofweeknameidenglish}. This language-sensitive macro is a fragile 
%command that requires protection in moving arguments.
%
%As before, if the \styfmt{datetime2-calc} package is loaded, the
%\sty{pgfcalendar} package's commands are also available including
%\begin{definition}[\inpkg{pgfcalendar}]
%\ics{pgfcalendarweekdayname}\marg{dow}
%\end{definition}
%where the argument \meta{dow} is an integer from 0 (Monday) to 6
%(Sunday). Multilingual support is provided through the
%\sty{translator} package.
%
%The \styfmt{datetime2-calc} package also provides a robust
%language-sensitive command:
%\begin{definition}[\inpkg{datetime2-calc}]
%\ics{DTMweekdayname}\marg{dow}
%\end{definition}
%This attempts to use \cs{DTM\meta{lang}weekdayname} if it exists,
%where \meta{lang} is either \cs{languagename} or obtained from the
%dialect-to-language mapping provided by \sty{tracklang}. If both
%attempts fail, \cs{DTMweekdayname} will fallback on
%\cs{pgfcalendarweekdayname} (with a warning through
%\ics{dtmnamewarning}).
%
%Old style (\styfmt{datetime}):
%\begin{verbatim}
%\documentclass{article}
%\usepackage[T1]{fontenc}
%\usepackage[utf8]{inputenc}
%\usepackage[french,english]{babel}
%
%\usepackage{datetime}
%
%\begin{document}
%
%\selectlanguage{french}
%\dayofweeknameid{1}
%
%\selectlanguage{english}
%\dayofweeknameid{1}
%\end{document}
%\end{verbatim}
%New style (\styfmt{datetime2} with the \module{english} and
%\module{french} modules):
%\begin{verbatim}
%\documentclass{article}
%\usepackage[T1]{fontenc}
%\usepackage[utf8]{inputenc}
%\usepackage[french,english]{babel}
%
%\usepackage[calc]{datetime2}
%
%\begin{document}
%
%\selectlanguage{french}
%\DTMweekdayname{6}
%
%\selectlanguage{english}
%\DTMweekdayname{6}
%\end{document}
%\end{verbatim}
%
%The \styfmt{datetime} package also provides the command
%\begin{definition}[\inpkg{datetime}]
%\oldcs{dayofweekname}\marg{DD}\marg{MM}\marg{YYYY}
%\end{definition}
%that has three arguments: the day of the month,
%the month number and the year. This calculates the day of week
%index \meta{n}, an integer from 1 (Sunday) to 7 (Saturday), and then uses the
%result in the argument of \oldcs{dayofweeknameid}\marg{n}. The
%\cs{dayofweekname} command was provided by \styfmt{datetime} as a
%convenient shortcut for use in date styles. (Similarly for
%\oldcs{shortdayofweekname}.) With the \styfmt{datetime2} date styles, the
%day-of-week index is automatically available through the fourth
%argument of \ics{DTMdisplaydate} (and \ics{DTMDisplaydate}), so there
%is little use for an equivalent command. Additionally, if a date has
%already been saved with \styfmt{datetime2}, the weekday can be
%extracted from the saved data through \ics{DTMfetchdow}, which can be
%used in the argument of commands like \ics{DTMweekdayname} or
%\ics{DTMenglishweekdayname}.
%
%However, if a calculation is required for some reason, it can be
%obtained using the \sty{pgfcalendar} commands
%\ics{pgfcalendardatetojulian} and \ics{pgfcalendarjuliantoweekday},
%which are described in the \styfmt{pgf} manual.
%
%The \styfmt{datetime} package provides the conditional
%\begin{definition}[\inpkg{datetime}]
%\oldcs{ifshowdow}
%\end{definition}
%which can be used to determine whether or not styles should display
%the weekday name (if supported). The \styfmt{datetime2} package has the
%analogous conditional
%\begin{definition}[\inpkg{datetime2}]
%\ics{ifDTMshowdow}
%\end{definition}
%
%The \styfmt{datetime} package provides the command 
%\begin{definition}[\inpkg{datetime}]
%\oldcs{ordinaldate}\marg{n}
%\end{definition}
%as a date-type ordinal where the argument should be an integer from 1
%to 31. For English, this just uses \sty{fmtcount}'s 
%non-expandable \cs{ordinalnum} command. For Breton, Welsh and French
%a suffix is only added when the argument is 1. For all other
%languages, this command just displays the number.
%
%With \styfmt{datetime2}, the language modules may or may not provide a
%command to display the ordinal but most of them do. For example, the 
%\module{english} module provides 
%\begin{definition}[\inpkg{datetime2-english}]
%\ics{DTMenglishordinal}\marg{n}
%\end{definition}
%This displays the suffix using
%\begin{definition}[\inpkg{datetime2-english}]
%\ics{DTMenglishfmtordsuffix}\marg{suffix}
%\end{definition}
%The definition of this command is changed by the styles provided by
%the English regional modules. For example, the \style{en-US} style
%redefines \cs{DTMenglishfmtordsuffix} to ignore its argument.
%See the documentation for the \module{english} module for further
%details.
%
%The \module{french} module provides:
%\begin{definition}[\inpkg{datetime2-french}]
%\ics{DTMfrenchordinal}\marg{n}
%\end{definition}
%This displays \meta{n} and if \meta{n} is 1, a suffix is appended.
%See the \module{french} module documentation for further details.
%
%The \module{breton} module provides:
%\begin{definition}[\inpkg{datetime2-breton}]
%\ics{DTMbretonordinal}\marg{n}
%\end{definition}
%which similarly appends a suffix if \meta{n} is 1 but not for other
%values. In this case, the suffix for the first day is formatted using
%\begin{definition}[\inpkg{datetime2-breton}]
%\ics{DTMbretonfmtordinal}\marg{suffix}
%\end{definition}
%which is redefined by the \csopt{DTMlangsetup}{ord} option.
%See the \module{breton} module documentation for further details.
%
%The \module{welsh} module similarly provides:
%\begin{definition}[\inpkg{datetime2-welsh}]
%\ics{DTMwelshordinal}\marg{n}
%\end{definition}
%and
%\begin{definition}[\inpkg{datetime2-welsh}]
%\ics{DTMwelshfmtordinal}\marg{suffix}
%\end{definition}
%See the \module{welsh} module documentation for further details.
%
%Other modules may simply define \cs{DTM\meta{lang}ordinal} to just
%display its argument. For example, the \module{german} module
%provides
%\begin{definition}[\inpkg{datetime2-german}]
%\ics{DTMgermanordinal}\marg{n}
%\end{definition}
%which just displays \meta{n} (the day of month number).
%
%Alternatively, modules may define \cs{DTM\meta{lang}ordinal} to
%display its argument followed by a full stop (period). For example,
%the \module{norsk} module provides
%\begin{definition}[\inpkg{datetime2-norsk}]
%\ics{DTMnorskordinal}\marg{n}
%\end{definition}
%which displays \meta{n} followed by a full stop.
%
%As before, the date styles should explicitly use the ordinal macro
%that matches the style. However, if you have some need to display
%the day of month independent of any of the styles, you can use
%\begin{definition}[\inpkg{datetime2-calc}]
%\ics{DTMordinal}\marg{n}
%\end{definition}
%which is provided by \styfmt{datetime2-calc}. This attempts to use 
%\cs{DTM\meta{lang}ordinal} if it exists, otherwise it just displays
%\meta{n}.
%
%Old style (\styfmt{datetime}):
%\begin{verbatim}
%\documentclass{article}
%\usepackage[T1]{fontenc}
%\usepackage[utf8]{inputenc}
%\usepackage[french,english]{babel}
%
%\usepackage{datetime}
%
%\begin{document}
%
%\selectlanguage{french}
%\ordinaldate{1}
%
%\selectlanguage{english}
%\ordinaldate{1}
%\end{document}
%\end{verbatim}
%This displays 1\textsuperscript{er} in the first case (through 
%\cs{ordinaldatefrench}) and
%1\textsuperscript{st} in the second case (through \sty{fmtcount}'s
%\cs{ordinalnum}).
%
%New style (\styfmt{datetime2} with the \module{english} and
%\module{french} modules):
%\begin{verbatim}
%\documentclass{article}
%\usepackage[T1]{fontenc}
%\usepackage[utf8]{inputenc}
%\usepackage[french,english]{babel}
%
%\usepackage[calc]{datetime2}
%
%\begin{document}
%
%\selectlanguage{french}
%\DTMordinal{1}
%
%\selectlanguage{english}
%\DTMordinal{1}
%\end{document}
%\end{verbatim}
%This displays 1\textsuperscript{er} in the first case and
%just 1 in the second case, because the regionless \module{english}
%module doesn't use a suffix.
%
%To achieve the same result as the \styfmt{datetime} example, a few
%modifications are needed:
%\begin{verbatim}
%\documentclass{article}
%\usepackage[T1]{fontenc}
%\usepackage[utf8]{inputenc}
%\usepackage[french,british]{babel}
%
%\usepackage[calc,useregional]{datetime2}
%\DTMlangsetup[en-GB]{ord=raise}
%
%\begin{document}
%
%\selectlanguage{french}
%\DTMordinal{1}
%
%\selectlanguage{british}
%\DTMordinal{1}
%\end{document}
%\end{verbatim}
%
%The \styfmt{datetime} package provides the command
%\begin{definition}[\inpkg{datetime}]
%\oldcs{twodigit}\marg{n}
%\end{definition}
%for use in date styles that require two-digit
%numbers. The \styfmt{datetime2} package provides
%\begin{definition}[\inpkg{datetime2}]
%\ics{DTMtwodigits}\marg{n}
%\end{definition}
%
%\section{Saving Dates}
%
%The \styfmt{datetime} package provides some commands for saving a date
%for later use. (There are no equivalent commands for saving a time
%in \styfmt{datetime}.)
%With \styfmt{datetime}, a date is saved using
%\begin{definition}[\inpkg{datetime}]
%\oldcs{newdate}\marg{name}\marg{DD}\marg{MM}\marg{YYYY}
%\end{definition}
%With \styfmt{datetime2}, a date can be saved using
%\begin{definition}[\inpkg{datetime2}]
%\ics{DTMsavedate}\marg{name}\marg{date}
%\end{definition}
%where \meta{date} is in the form
%\texttt{\meta{YYYY}-\meta{MM}-\meta{DD}}.
%
%A previously saved date can be displayed with \styfmt{datetime} using:
%\begin{definition}[\inpkg{datetime}]
%\oldcs{displaydate}\marg{name}
%\end{definition}
%With \styfmt{datetime2}, the saved date can be displayed using:
%\begin{definition}[\inpkg{datetime2}]
%\ics{DTMusedate}\marg{name}
%\end{definition}
%
%Old style (\styfmt{datetime}):
%\begin{verbatim}
%\documentclass{article}
%\usepackage{datetime}
%
%\begin{document}
%
%\newdate{mydate}{20}{1}{2016}
%\displaydate{mydate}
%
%\end{document}
%\end{verbatim}
%New style (\styfmt{datetime2}):
%\begin{verbatim}
%\documentclass{article}
%\usepackage[en-GB,showdow]{datetime2}
%
%\begin{document}
%
%\DTMsavedate{mydate}{2016-01-20}
%\DTMusedate{mydate}
%
%\end{document}
%\end{verbatim}
%
%Individual elements of the date can be extracted with the
%\styfmt{datetime} commands:
%\begin{definition}[\inpkg{datetime}]
%\oldcs{getdateday}\marg{name}
%\end{definition}
%for the day of the month,
%\begin{definition}[\inpkg{datetime}]
%\oldcs{getdatemonth}\marg{name}
%\end{definition}
%for the month number, and
%\begin{definition}[\inpkg{datetime}]
%\oldcs{getdateyear}\marg{name}
%\end{definition}
%for the year.
%
%These elements can be fetched in \styfmt{datetime2} using:
%\begin{definition}[\inpkg{datetime2}]
%\ics{DTMfetchday}\marg{name}
%\end{definition}
%for the day of the month,
%\begin{definition}[\inpkg{datetime2}]
%\ics{DTMfetchmonth}\marg{name}
%\end{definition}
%for the month number, and
%\begin{definition}[\inpkg{datetime2}]
%\ics{DTMfetchyear}\marg{name}
%\end{definition}
%for the year. Additionally you can fetch the day of week index if it
%has been computed:
%\begin{definition}[\inpkg{datetime2}]
%\ics{DTMfetchdow}\marg{name}
%\end{definition}
%
%Old style (\styfmt{datetime}):
%\begin{verbatim}
%\documentclass{article}
%
%\usepackage{datetime}
%
%\begin{document}
%
%\newdate{mydate}{20}{1}{2016}
%
%Year: \getdateyear{mydate}.
%Month: \getdatemonth{mydate}.
%Day: \getdateday{mydate}.
%
%\end{document}
%\end{verbatim}
%New style (\styfmt{datetime2}):
%\begin{verbatim}
%\documentclass{article}
%
%\usepackage{datetime2}
%
%\begin{document}
%
%\DTMsavedate{mydate}{2016-01-20}
%
%Year: \DTMfetchyear{mydate}.
%Month: \DTMfetchmonth{mydate}.
%Day: \DTMfetchday{mydate}.
%
%\end{document}
%\end{verbatim}
%
%With \styfmt{datetime2}, you can also save the current time (as in the
%time of the document build) with
%\begin{definition}[\inpkg{datetime2}]
%\ics{DTMsavenow}\marg{name}
%\end{definition}
%and then access each field of the date, time and zone.
%
%Old style (\styfmt{datetime}):
%\begin{verbatim}
%\documentclass{article}
%
%\usepackage{datetime}
%
%\begin{document}
%
%Year: \number\year.
%Month: \number\month.
%Day: \number\day.
%DOW: \computedayofweek{\day}{\month}{\year}\number\dayofweek.
%
%Hour: \number\currenthour.
%Minute: \number\currentminute.
%Second: \number\currentsecond.
%
%\end{document}
%\end{verbatim}
%New style (\styfmt{datetime2}):
%\begin{verbatim}
%\documentclass{article}
%
%\usepackage[calc]{datetime2}
%
%\begin{document}
%\DTMsavenow{now}
%
%Year: \DTMfetchyear{now}.
%Month: \DTMfetchmonth{now}.
%Day: \DTMfetchday{now}.
%DOW: \DTMfetchdow{now}.
%
%Hour: \DTMfetchhour{now}.
%Minute: \DTMfetchminute{now}.
%Second: \DTMfetchsecond{now}.
%
%Time Zone Hour: \DTMfetchTZhour{now}.
%Time Zone Minute: \DTMfetchTZminute{now}.
%
%\end{document}
%\end{verbatim}
%(Note that the day of week index is different as \styfmt{datetime2}
%uses the same indexing system as \sty{pgfcalendar}.)
%
%\section{Multilingual Support}
%
%The \styfmt{datetime} package comes with the following files (in
%addition to \texttt{datetime.sty} and
%\texttt{datetime-defaults.sty}):
%
%\begin{center}
%\begin{tabular}{llll}
%\texttt{dt-american.def} & \texttt{dt-dutch.def} & \texttt{dt-lsorbian.def} & \texttt{dt-slovak.def}\\
%\texttt{dt-australian.def} & \texttt{dt-esperanto.def} & \texttt{dt-magyar.def} & \texttt{dt-slovene.def}\\
%\texttt{dt-austrian.def} & \texttt{dt-estonian.def} & \texttt{dt-naustrian.def} & \texttt{dt-spanish.def}\\
%\texttt{dt-bahasa.def} & \texttt{dt-finnish.def} & \texttt{dt-newzealand.def} & \texttt{dt-swedish.def}\\
%\texttt{dt-basque.def} & \texttt{dt-french.def} & \texttt{dt-ngerman.def} & \texttt{dt-turkish.def}\\
%\texttt{dt-breton.def} & \texttt{dt-galician.def} & \texttt{dt-norsk.def} & \texttt{dt-UKenglish.def}\\
%\texttt{dt-british.def} & \texttt{dt-german.def} & \texttt{dt-polish.def} & \texttt{dt-ukraineb.def}\\
%\texttt{dt-bulgarian.def} & \texttt{dt-greek.def} & \texttt{dt-portuges.def} & \texttt{dt-USenglish.def}\\
%\texttt{dt-canadian.def} & \texttt{dt-hebrew.def} & \texttt{dt-romanian.def} & \texttt{dt-usorbian.def}\\
%\texttt{dt-catalan.def} & \texttt{dt-icelandic.def} & \texttt{dt-russian.def} & \texttt{dt-welsh.def}\\
%\texttt{dt-croatian.def} & \texttt{dt-irish.def} & \texttt{dt-samin.def}\\
%\texttt{dt-czech.def} & \texttt{dt-italian.def} & \texttt{dt-scottish.def}\\
%\texttt{dt-danish.def} & \texttt{dt-latin.def} & \texttt{dt-serbian.def}
%\end{tabular}
%\end{center}
%
%These \texttt{dt-\meta{lang}.def} files provide the code to integrate
%\styfmt{datetime} with \sty{babel} for each language given by \meta{lang}.
%This means that if you have \styfmt{datetime} installed and there's a
%\texttt{.def} file that matches the language you are using with
%\sty{babel}, then all you need to do is load \sty{babel} before
%\styfmt{datetime}.
%
%With \styfmt{datetime2}, language support is provided in separate
%independently-maintained modules. The actual \styfmt{datetime2} package
%itself just comes with two files: \texttt{datetime2.sty} and
%\texttt{datetime2-calc.sty}. This means that if you only want to use
%the basic numeric styles and aren't using \sty{babel} or
%\sty{polyglossia}, then that's all you need. (Although you'll
%obviously need to install dependent packages, such as \styfmt{pgf}
%which provides the \sty{pgfcalendar} package used by
%\styfmt{datetime2-calc}. However \sty{fmtcount}, which is required by
%\styfmt{datetime}, isn't required by \styfmt{datetime2}.)
%
%If you want to use \styfmt{datetime2} with language support, then you
%only need to install the modules for the required language. For
%example, if you only use English, you can just install the
%\module{english} module and if you only use French, you can just
%install the \module{french} module.
%
%At the time of writing, the following modules are available on CTAN:
%\begin{center}
%\begin{tabular}{lll}
%\module[bahasai]{datetime2-bahasai} & \module[galician]{datetime2-galician} & \module[russian]{datetime2-russian} \\
%\module[basque]{datetime2-basque} & \module[german]{datetime2-german} & \module[samin]{datetime2-samin} \\
%\module[breton]{datetime2-breton} & \module[greek]{datetime2-greek} & \module[scottish]{datetime2-scottish} \\
%\module[bulgarian]{datetime2-bulgarian} & \module[hebrew]{datetime2-hebrew} & \module[serbian]{datetime2-serbian} \\
%\module[catalan]{datetime2-catalan} & \module[icelandic]{datetime2-icelandic} & \module[slovak]{datetime2-slovak} \\
%\module[croatian]{datetime2-croatian} & \module[irish]{datetime2-irish} & \module[slovene]{datetime2-slovene} \\
%\module[czech]{datetime2-czech} & \module[italian]{datetime2-italian} & \module[spanish]{datetime2-spanish} \\
%\module[danish]{datetime2-danish} & \module[latin]{datetime2-latin} & \module[swedish]{datetime2-swedish} \\
%\module[dutch]{datetime2-dutch} & \module[lsorbian]{datetime2-lsorbian} & \module[turkish]{datetime2-turkish} \\
%\module[english]{datetime2-english} & \module[magyar]{datetime2-magyar} & \module[ukrainian]{datetime2-ukrainian} \\
%\module[esperanto]{datetime2-esperanto} & \module[norsk]{datetime2-norsk} & \module[usorbian]{datetime2-usorbian} \\
%\module[estonian]{datetime2-estonian} & \module[polish]{datetime2-polish} & \module[welsh]{datetime2-welsh} \\
%\module[finnish]{datetime2-finnish} & \module[portuges]{datetime2-portuges} \\
%\module[french]{datetime2-french} & \module[romanian]{datetime2-romanian}
%\end{tabular}
%\end{center}
%Some of these only support the root language but some, such as
%\module{english}, provide support for different regions. There is
%also a supplementary package \sty{datetime2-en-fulltext} that
%replicates \styfmt{datetime}'s \texttt{text} and \texttt{oclock}
%styles (and requires \sty{fmtcount}).
%
%Old style (\styfmt{datetime}):
%\begin{verbatim}
%\documentclass{article}
%\usepackage[T1]{fontenc}
%\usepackage[utf8]{inputenc}
%\usepackage[french,british]{babel}
%
%\usepackage{datetime}
%
%\begin{document}
%\selectlanguage{french}
%\today
%
%\selectlanguage{british}
%\today
%
%\end{document}
%\end{verbatim}
%This produces:
%\begin{quote}
%21 janvier 2016\par
%Thursday 21\textsuperscript{st} January, 2016
%\end{quote}
%
%New style (\styfmt{datetime2}, \sty{datetime2-english} and \sty{datetime2-french}):
%\begin{verbatim}
%\documentclass{article}
%\usepackage[T1]{fontenc}
%\usepackage[utf8]{inputenc}
%\usepackage[french,british]{babel}
%
%\usepackage[showdow,useregional]{datetime2}
%
%\begin{document}
%\selectlanguage{french}
%\today
%
%\selectlanguage{british}
%\today
%
%\end{document}
%\end{verbatim}
%This produces:
%\begin{quote}
%21 janvier 2016\par
%Thursday 21st January 2016
%\end{quote}
%There's a slight difference in the appearance of the British date.
%An exact reproduction of the \styfmt{datetime} format can be achieved
%by modifying the \style{en-GB} options:
%\begin{verbatim}
%\documentclass{article}
%\usepackage[T1]{fontenc}
%\usepackage[utf8]{inputenc}
%\usepackage[french,british]{babel}
%
%\usepackage[showdow,useregional]{datetime2}
%\DTMlangsetup[en-GB]{ord=raise,monthyearsep={,\space}}
%
%\begin{document}
%\selectlanguage{french}
%\today
%
%\selectlanguage{british}
%\today
%
%\end{document}
%\end{verbatim}
%Note that neither \styfmt{dt-french.def} from \styfmt{datetime} nor 
%\sty{datetime2-french} support
%the show day of week option.
%
%The \styfmt{datetime} package provides:
%\begin{definition}[\inpkg{datetime}]
%\oldcs{setdefaultdate}\marg{date declaration}
%\end{definition}
%to always use the date style given by \meta{date declaration}
%instead of letting \sty{babel} switch the date format every time the
%language changes.
%
%In \styfmt{datetime2}, the default is the reverse: the style won't
%change when the language changes unless the languages (or regions)
%have been listed in the \styfmt{datetime2} package options. If the
%regional styles have been enabled, allowing \sty{babel} to change
%the date style whenever the language changes, then you can switch
%this behaviour off by setting the \pkgopt{useregional} option to
%\pkgoptfmt{false}.
%
%Old style (\styfmt{datetime}):
%\begin{verbatim}
%\documentclass{article}
%\usepackage[T1]{fontenc}
%\usepackage[utf8]{inputenc}
%\usepackage[french,british]{babel}
%
%\usepackage{datetime}
%
%\yyyymmdddate
%
%\begin{document}
%\selectlanguage{french}
%\today
%
%\selectlanguage{british}
%\today
%
%\end{document}
%\end{verbatim}
%This produces:
%\begin{quote}
%21 janvier 2016\par
%Thursday 21\textsuperscript{st} January, 2016
%\end{quote}
%(The numeric date style has been overridden.)
%\begin{verbatim}
%\documentclass{article}
%\usepackage[T1]{fontenc}
%\usepackage[utf8]{inputenc}
%\usepackage[french,british]{babel}
%
%\usepackage{datetime2}
%
%\DTMsetdatestyle{iso}
%
%\begin{document}
%\selectlanguage{french}
%\today
%
%\selectlanguage{british}
%\today
%
%\end{document}
%\end{verbatim}
%This produces:
%\begin{quote}
%2016-01-21\par
%2016-01-21
%\end{quote}
%(The ISO date style overrides the language setting.)
%
%Compare this to:
%\begin{verbatim}
%\documentclass{article}
%\usepackage[T1]{fontenc}
%\usepackage[utf8]{inputenc}
%\usepackage[french,british]{babel}
%
%\usepackage[en-GB,fr-FR]{datetime2}
%
%\DTMsetdatestyle{iso}
%
%\begin{document}
%\selectlanguage{french}
%\today
%
%\selectlanguage{british}
%\today
%
%\end{document}
%\end{verbatim}
%This produces:
%\begin{quote}
%21 janvier 2016\par
%21st January 2016
%\end{quote}
%
%Examples that explicitly suppress the language-sensitive dates
%follow. First with \styfmt{datetime}:
%\begin{verbatim}
%\documentclass{article}
%\usepackage[T1]{fontenc}
%\usepackage[utf8]{inputenc}
%\usepackage[french,british]{babel}
%
%\usepackage{datetime}
%
%\setdefaultdate{\yyyymmdddate}
%
%\begin{document}
%\selectlanguage{french}
%\today
%
%\selectlanguage{british}
%\today
%
%\end{document}
%\end{verbatim}
%This produces:
%\begin{quote}
%2016/01/21\par
%2016/01/21
%\end{quote}
%
%Now with \styfmt{datetime2}:
%\begin{verbatim}
%\documentclass{article}
%\usepackage[T1]{fontenc}
%\usepackage[utf8]{inputenc}
%\usepackage[french,british]{babel}
%
%\usepackage[en-GB,fr-FR,useregional=false]{datetime2}
%
%\DTMsetdatestyle{iso}
%
%\begin{document}
%\selectlanguage{french}
%\today
%
%\selectlanguage{british}
%\today
%
%\end{document}
%\end{verbatim}
%This produces:
%\begin{quote}
%2016-01-21\par
%2016-01-21
%\end{quote}
%
%\section{Predefined Date Formats}
%
%The \styfmt{datetime} date styles were set with declarations such as
%\cs{longdate} (or package options that used the associated
%declaration, such as \pkgoptfmt{long}).
%With \styfmt{datetime2}, date styles are set using
%\begin{definition}[\inpkg{datetime2}]
%\ics{DTMsetdatestyle}\marg{style name}
%\end{definition}
%(which just changes the date style without changing the time style) or
%\begin{definition}[\inpkg{datetime2}]
%\ics{DTMsetstyle}\marg{style name}
%\end{definition}
%which sets the full date-time style.
%
%This section lists the \styfmt{datetime} declarations and how the
%same style can be set through \styfmt{datetime2}.
%
%\begin{definition}[\inpkg{datetime}]
%\oldcs{yyyymmdddate}
%\end{definition}
%By default this style produces a date in the form 2016/01/20. (The
%separator is governed by \oldcs{dateseparator}.) This is the default
%style for \styfmt{datetime2} with the exception of the separator,
%which defaults to a hyphen. To reproduce the \styfmt{datetime}
%format, you can set the date style to \style{default} (if it has
%been previous changed from the default) and change the separator
%with the \pkgopt{datesep} option:
%\begin{verbatim}
%\DTMsetdatestyle{default}
%\DTMsetup{datesep={/}}
%\end{verbatim}
%
%\begin{definition}[\inpkg{datetime}]
%\oldcs{longdate}
%\end{definition}
%This is the default date format for \styfmt{datetime} and this style produces
%the date in the form: Wednesday 8\textsuperscript{th} March, 2000.
%This is actually a regional style for some of the English dialects,
%so with \styfmt{datetime2} this additionally needs the
%\module{english} module installed. To exactly replicate this date
%format, including the day of week name, the superscript ordinal
%suffix and the comma after the month name, you need the
%\pkgopt{showdow} and \pkgopt{en-GB} package options and 
%the \csopt{DTMlangsetup}{ord} and \csopt{DTMlangsetup}{monthyearsep}
%options for the \style{en-GB} style. For example:
%\begin{verbatim}
%\usepackage[showdow,en-GB]{datetime2}
%\DTMlangsetup[en-GB]{ord=raise,monthyearsep={,\space}}
%\end{verbatim}
%Another possibility is:
%\begin{verbatim}
%\usepackage[british]{babel}
%\usepackage[showdow]{datetime2}
%\DTMlangsetup[en-GB]{ord=raise,monthyearsep={,\space}}
%\DTMsetdatestyle{en-GB}
%\end{verbatim}
%There are other regional styles in the \module{english} module that
%produce the same format, such as \style{en-GG}.
%
%\begin{definition}[\inpkg{datetime}]
%\oldcs{shortdate}
%\end{definition}
%This is similar to \cs{longdate} but uses abbreviated names to
%produce a date in the form: Wed 8\textsuperscript{th} Mar, 2000.
%With \styfmt{datetime2}, this is like the above but additionally
%needs the \csopt{DTMlangsetup}{abbr} option for the \style{en-GB}
%style:
%\begin{verbatim}
%\usepackage[showdow,en-GB]{datetime2}
%\DTMlangsetup[en-GB]{ord=raise,monthyearsep={,\space},abbr}
%\end{verbatim}
%or
%\begin{verbatim}
%\usepackage[british]{babel}
%\usepackage[showdow]{datetime2}
%\DTMlangsetup[en-GB]{ord=raise,monthyearsep={,\space},abbr}
%\DTMsetdatestyle{en-GB}
%\end{verbatim}
%
%\begin{definition}[\inpkg{datetime}]
%\oldcs{ddmmyyyydate}
%\end{definition}
%This produces a date in the form 08/03/2000. This can be reproduced
%with just \styfmt{datetime2} using the \style{ddmmyyyy} style with
%the date separator set to a slash:
%\begin{verbatim}
%\DTMsetdatestyle{ddmmyyyy}
%\DTMsetup{datesep=/}
%\end{verbatim}
%
%\begin{definition}[\inpkg{datetime}]
%\oldcs{dmyyyydate}
%\end{definition}
%This produces a date in the form 8/3/2000. This can be reproduced
%with just \styfmt{datetime2} using the \style{dmyyyy} style with
%the date separator set to a slash:
%\begin{verbatim}
%\DTMsetdatestyle{dmyyyy}
%\DTMsetup{datesep=/}
%\end{verbatim}
%This format is also the style of some of the regional numeric date
%styles. For example, with the \module{english} module:
%\begin{verbatim}
%\usepackage[british]{babel}
%\usepackage{datetime2}
%\DTMsetdatestyle{en-GB-numeric}
%\end{verbatim}
%
%\begin{definition}[\inpkg{datetime}]
%\oldcs{ddmmyydate}
%\end{definition}
%This produces a date in the form 08/03/00. This can be reproduced
%with just \styfmt{datetime2} using the \style{ddmmyy} style with
%the date separator set to a slash:
%\begin{verbatim}
%\DTMsetdatestyle{ddmmyy}
%\DTMsetup{datesep=/}
%\end{verbatim}
%
%\begin{definition}[\inpkg{datetime}]
%\oldcs{dmyydate}
%\end{definition}
%This produces a date in the form 8/3/00. This can be reproduced
%with just \styfmt{datetime2} using the \style{dmyy} style with
%the date separator set to a slash:
%\begin{verbatim}
%\DTMsetdatestyle{dmyy}
%\DTMsetup{datesep=/}
%\end{verbatim}
%
%\begin{definition}[\inpkg{datetime}]
%\oldcs{textdate}
%\end{definition}
%This style is designed to produce a full text British date in the
%form: Wednesday the Eighth of March, Two Thousand. The
%\sty{datetime2-en-fulltext} package is required to reproduce this
%style:
%\begin{verbatim}
%\usepackage[showdow]{datetime2-en-fulltext}
%\DTMsetdatestyle{en-FullText}
%\end{verbatim}
%Note that with both \styfmt{datetime} and
%\styfmt{datetime2-en-fulltext} this style should not be used if the
%current language isn't English.
%
%\begin{definition}[\inpkg{datetime}]
%\oldcs{usdate}
%\end{definition}
%This style is designed to produce \TeX's default US date format in
%the form March 8, 2000. This style can be reproduced with
%\styfmt{datetime2} and the \module{english} module:
%\begin{verbatim}
%\usepackage[en-US]{datetime2}
%\end{verbatim}
%or
%\begin{verbatim}
%\usepackage[USenglish]{babel}
%\usepackage[useregional]{datetime2}
%\DTMsetdatestyle{en-US}
%\end{verbatim}
%
%\begin{definition}[\inpkg{datetime}]
%\oldcs{mmddyyyydate}
%\end{definition}
%This style produces a date in a middle-endian format in the form:
%03/08/2000. This style can be reproduced with \styfmt{datetime2} using
%the \style{mmddyyyy} style with the date separator set to a slash:
%\begin{verbatim}
%\DTMsetup{datesep=/}
%\DTMsetdatestyle{mmddyyyy}
%\end{verbatim}
%
%\begin{definition}[\inpkg{datetime}]
%\oldcs{mdyyyydate}
%\end{definition}
%This style produces a date in a middle-endian format in the form:
%3/8/2000. This style can be reproduced with \styfmt{datetime2} using
%the \style{mdyyyy} style with the date separator set to a slash:
%\begin{verbatim}
%\DTMsetup{datesep=/}
%\DTMsetdatestyle{mdyyyy}
%\end{verbatim}
%This format is also the style of some of the regional numeric date
%styles. For example, with the \module{english} module:
%\begin{verbatim}
%\usepackage[USenglish]{babel}
%\usepackage{datetime2}
%\DTMsetdatestyle{en-US-numeric}
%\end{verbatim}
%
%\begin{definition}[\inpkg{datetime}]
%\oldcs{mmddyydate}
%\end{definition}
%This style produces a date in a middle-endian format in the form:
%03/08/00. This style can be reproduced with \styfmt{datetime2} using
%the \style{mmddyy} style with the date separator set to a slash:
%\begin{verbatim}
%\DTMsetup{datesep=/}
%\DTMsetdatestyle{mmddyy}
%\end{verbatim}
%
%\begin{definition}[\inpkg{datetime}]
%\oldcs{mdyydate}
%\end{definition}
%This style produces a date in a middle-endian format in the form:
%3/8/00. This style can be reproduced with \styfmt{datetime2} using
%the \style{mdyy} style with the date separator set to a slash:
%\begin{verbatim}
%\DTMsetup{datesep=/}
%\DTMsetdatestyle{mdyy}
%\end{verbatim}
%
%\section{Predefined Time Formats}
%
%The time formats are set in \styfmt{datetime} using
%\begin{definition}[\inpkg{datetime}]
%\oldcs{settimeformat}\marg{style-name}
%\end{definition}
%With \styfmt{datetime2}, the time styles are set using
%\begin{definition}[\inpkg{datetime2}]
%\ics{DTMsettimestyle}\marg{style-name}
%\end{definition}
%(which just changes the time style without changing the date style) or
%\begin{definition}[\inpkg{datetime2}]
%\ics{DTMsetstyle}\marg{style name}
%\end{definition}
%which sets the full date-time style.
%
%The \styfmt{datetime} package provides the following time styles:
%\begin{description}
%\item[xxivtime] This style produces the time in twenty-four hour
%format in the form 09:28. (The default time style for
%\styfmt{datetime}.) Note that this style has no seconds. The format
%can be reproduced in \styfmt{datetime2} with the \style{default}
%style and the \pkgopt{showseconds} option set to \pkgoptfmt{false}:
%\begin{verbatim}
%\DTMsettimestyle{default}
%\DTMsetup{showseconds=false}
%\end{verbatim}
%
%\item[hhmmsstime] This is like the previous style but includes the
%seconds. This is the default time style for \styfmt{datetime2}:
%\begin{verbatim}
%\DTMsettimestyle{default}
%\end{verbatim}
%
%\item[ampmtime] This style produces the time in twelve hour format
%in the form 9:28am or 7:54pm. This style is available in some of the
%regional modules for \styfmt{datetime2}. For example, using the
%\style{en-GB} style in the \module{english} module:
%\begin{verbatim}
%\usepackage[en-GB]{datetime2}
%\end{verbatim}
%or
%\begin{verbatim}
%\usepackage[british]{babel}
%\usepackage{datetime2}
%\DTMsettimestyle{en-GB}
%\end{verbatim}
%
%\item[oclock] This style is designed for a full text English time
%format in the form: Twenty-Eight minutes past Ten in the afternoon.
%This style can be reproduced with the \style{en-FullText} style
%provided by the \sty{datetime2-en-fulltext} package:
%\begin{verbatim}
%\usepackage{datetime2-en-fulltext}
%\DTMsettimestyle{en-FullText}
%\end{verbatim}
%
%\end{description}
%
%\section{Defining a New Date Format}
%
%The \styfmt{datetime} package provides:
%\begin{definition}[\inpkg{datetime}]
%\oldcs{newdateformat}\marg{name}\marg{format}
%\end{definition}
%to define a new date style. Within \meta{format}, the placeholder
%commands \oldcs{THEDAY}, \oldcs{THEMONTH} and \oldcs{THEYEAR} are
%used to represent the relevant day, month and year values. There are
%also counter placeholders \texttt{DAY}, \texttt{MONTH} and
%\texttt{YEAR}, which may be used instead.
%
%A necessary consequence of allowing placeholder commands in
%\meta{format} means that these commands must be set as appropriate
%before the date can be formatted. This means that the formatted date
%can't be expanded and the date commands must be made robust to
%protect them in moving arguments. This is an inherent problem with the
%\styfmt{datetime} package that can't be fixed without breaking
%backwards compatibility and is one of the main reasons for
%introducing the replacement \styfmt{datetime2} package.
%
%The \styfmt{datetime2} package provides a better way of providing
%date styles, which are defined using:
%\begin{definition}[\inpkg{datetime2}]
%\ics{DTMnewdatestyle}\marg{name}\marg{definition}
%\end{definition}
%Instead of using placeholder commands, the date styles
%simply redefine the date formatting macros within \meta{definition}.
%There are two principle date formatting commands that the date style
%must define although styles may redefine additional helper commands
%if necessary.
%
%The two main date formatting commands are:
%\begin{definition}[\inpkg{datetime2}]
%\ics{DTMdisplaydate}\marg{YYYY}\marg{MM}\marg{DD}\marg{dow}
%\end{definition}
%for use where no case-changing is required and
%\begin{definition}[\inpkg{datetime2}]
%\ics{DTMDisplaydate}\marg{YYYY}\marg{MM}\marg{DD}\marg{dow}
%\end{definition}
%for use where the date must begin with an upper case letter, for
%example if the date occurs at the start of a sentence. For styles
%where the case-change is irrelevant (for example, numeric styles or
%styles that always start with an upper case letter), the
%\cs{DTMDisplaydate} command may simply be set to
%\cs{DTMdisplaydate}.
%
%The \styfmt{datetime} manual provides some examples of new date
%styles. The first is simply a numeric little-endian style with a hyphen 
%separating each number:
%\begin{verbatim}
%\newdateformat{mydate}{\THEDAY-\THEMONTH-\THEYEAR}
%\end{verbatim}
%
%To convert this into a \styfmt{datetime2} format, the new style
%simply needs to redefine \cs{DTMdisplaydate} so that it has the
%following definition:
%\begin{verbatim}
%\renewcommand{\DTMdisplaydate}[4]{#3-#2-#1}
%\end{verbatim}
%Note that this command must always have four arguments, even if one
%or more of them are ignored. So here \cs{THEYEAR} is just the first
%argument \verb|#1|, \cs{THEMONTH} is the second argument \verb|#2|
%and \cs{THEDAY} is the third argument \verb|#3|. One other thing to note is
%that the arguments may be supplied with a leading zero. If you want
%to trim this off, you can use \TeX's \ics{number} primitive. For
%example:
%\begin{verbatim}
%\renewcommand{\DTMdisplaydate}[4]{\number#3-\number#2-\number#1 }
%\end{verbatim}
%This is the better method to allow for, say, registers used in any
%of the arguments.
%
%Therefore this new date style can be defined for \styfmt{datetime2}
%as follows:
%\begin{verbatim}
%\DTMnewdatestyle{mydate}{%
% \renewcommand{\DTMdisplaydate}[4]{\number##3-\number##2-\number##1 }%
% \renewcommand{\DTMDisplaydate}{\DTMdisplaydate}%
%}
%\end{verbatim}
%(Note the need to double the \verb|#| in the parameters, as is usual
%when redefining a command within another command in this manner.)
%
%For example, using \styfmt{datetime}:
%\begin{verbatim}
%\documentclass{article}
%\usepackage{datetime}
%
%\newdateformat{mydate}{\THEDAY-\THEMONTH-\THEYEAR}
%\mydate
%
%\begin{document}
%\today.
%\end{document}
%\end{verbatim}
%Now using \styfmt{datetime2}:
%\begin{verbatim}
%\documentclass{article}
%\usepackage{datetime2}
%
%\DTMnewdatestyle{mydate}{%
% \renewcommand{\DTMdisplaydate}[4]{\number##3-\number##2-\number##1 }%
% \renewcommand{\DTMDisplaydate}{\DTMdisplaydate}%
%}
%\DTMsetdatestyle{mydate}
%
%\begin{document}
%\today.
%\end{document}
%\end{verbatim}
%
%Another example provided in the \styfmt{datetime} manual ensures two
%digits for the month and day of month:
%\begin{verbatim}
%\newdateformat{dashdate}{\twodigit{\THEDAY}-\twodigit{\THEMONTH}-\THEYEAR}
%\end{verbatim}
%
%This is a minor modification to the previous example. The two digit
%number format can be obtained through \styfmt{datetime2}'s
%\ics{DTMtwodigits} command:
%\begin{verbatim}
%\DTMnewdatestyle{dashdate}{%
% \renewcommand{\DTMdisplaydate}[4]{%
%   \DTMtwodigits{##3}-\DTMtwodigits{##2}-\number##1 }%
% \renewcommand{\DTMDisplaydate}{\DTMdisplaydate}%
%}
%\end{verbatim}
%
%A complete example follows for \styfmt{datetime}:
%\begin{verbatim}
%\documentclass{article}
%\usepackage{datetime}
%
%\newdateformat{dashdate}{\twodigit{\THEDAY}-\twodigit{\THEMONTH}-\THEYEAR}
%\dashdate
%
%\begin{document}
%\today.
%\end{document}
%\end{verbatim}
%And for \styfmt{datetime2}:
%\begin{verbatim}
%\documentclass{article}
%\usepackage{datetime2}
%
%\DTMnewdatestyle{dashdate}{%
% \renewcommand{\DTMdisplaydate}[4]{%
%  \DTMtwodigits{##3}-\DTMtwodigits{##2}-\number##1 }%
% \renewcommand{\DTMDisplaydate}{\DTMdisplaydate}%
%}
%\DTMsetdatestyle{dashdate}
%
%\begin{document}
%\today.
%\end{document}
%\end{verbatim}
%
%You may have noticed that there's no equivalent of
%\styfmt{datetime}'s counter placeholders, but there's no real need
%for them and any attempt at implementing them would return to the
%original problem of preventing an expandable date format. \LaTeX\
%counters all internally use \TeX\ count registers and counter
%formatting commands have to access those registers to determine the
%counter value.
%
%To illustrate this, the \styfmt{datetime} manual provides the
%example:
%\begin{verbatim}
%\newdateformat{usvardate}{\monthname[\THEMONTH] \ordinal{DAY}, \THEYEAR}
%\end{verbatim}
%This uses the \texttt{DAY} counter placeholder since \cs{ordinal}
%(provided by the \sty{fmtcount} package) requires a counter name as
%the argument. However, \cs{ordinal} internally uses \cs{ordinalnum}
%with the internal count register as the argument, so the format
%could just as easily be defined as:
%\begin{verbatim}
%\newdateformat{usvardate}{\monthname[\THEMONTH] \ordinalnum{\THEDAY}, \THEYEAR}
%\end{verbatim}
%
%So how can this style be reproduced with \styfmt{datetime2}? A
%na\"ive approach is:
%\begin{verbatim}
%\DTMnewdatestyle{usvardate}{%
% \renewcommand{\DTMdisplaydate}[4]{%
%  \DTMmonthname{##2} \ordinalnum{##2}, \number##1 }%
% \renewcommand{\DTMDisplaydate}{\DTMdisplaydate}%
%}
%\end{verbatim}
%This requires both \styfmt{datetime2-calc} (for \ics{DTMmonthname})
%and \sty{fmtcount} (for \cs{ordinalnum}).
%
%A complete example that uses it:
%\begin{verbatim}
%\documentclass{article}
%
%\usepackage{fmtcount}
%\usepackage[calc]{datetime2}
%
%\DTMnewdatestyle{usvardate}{%
% \renewcommand{\DTMdisplaydate}[4]{%
%  \DTMmonthname{##2} \ordinalnum{##2}, \number##1 }%
% \renewcommand{\DTMDisplaydate}{\DTMdisplaydate}%
%}
%\DTMsetdatestyle{usvardate}
%
%\begin{document}
%\today.
%\end{document}
%\end{verbatim}
%This produces a warning from \cs{DTMmonthname} because
%\cs{DTMenglishmonthname} hasn't been defined, so it uses 
%\cs{pgfcalendarmonthname} instead. The document displays the text:
%\begin{quote}
%January 1\textsuperscript{st}, 2016.
%\end{quote}
%This seems to produce the correct output, but let's see what happens
%if we make a minor modification to the example:
%\begin{verbatim}
%\documentclass{article}
%
%\usepackage[english,french]{babel}
%\usepackage{fmtcount}
%\usepackage[calc]{datetime2}
%
%\DTMnewdatestyle{usvardate}{%
% \renewcommand{\DTMdisplaydate}[4]{%
%  \DTMmonthname{##2} \ordinalnum{##2}, \number##1 }%
% \renewcommand{\DTMDisplaydate}{\DTMdisplaydate}%
%}
%\DTMsetdatestyle{usvardate}
%
%\begin{document}
%\today.
%\end{document}
%\end{verbatim}
%This now produces:
%\begin{quote}
%janvier 1\textsuperscript{er}, 2016.
%\end{quote}
%which doesn't match the rationale of the style being a variation of
%the standard US date style nor does it match the little-endian
%syntax of French dates.
%
%Now let's make another minor change to the example:
%\begin{verbatim}
%\documentclass{article}
%
%\usepackage{fmtcount}
%\usepackage[calc]{datetime2}
%
%\DTMnewdatestyle{usvardate}{%
% \renewcommand{\DTMdisplaydate}[4]{%
%  \DTMmonthname{##2} \ordinalnum{##2}, \number##1 }%
% \renewcommand{\DTMDisplaydate}{\DTMdisplaydate}%
%}
%\DTMsetdatestyle{usvardate}
%
%\begin{document}
%\section{\today: an example}
%
%\today.
%\end{document}
%\end{verbatim}
%This document can't compile properly and causes the error:
%\begin{verbatim}
%! Argument of \@sect has an extra }.
%<inserted text> 
%                \par 
%\end{verbatim}
%This is because the style definition has made \cs{today} fragile
%because it uses an unprotected fragile command. This can be fixed by
%protecting \cs{ordinalnum} in the style definition.
%
%Let's make another modification:
%\begin{verbatim}
%\documentclass{article}
%
%\usepackage{fmtcount}
%\usepackage[calc]{datetime2}
%
%\DTMnewdatestyle{usvardate}{%
% \renewcommand{\DTMdisplaydate}[4]{%
%  \DTMmonthname{##2} \protect\ordinalnum{##2}, \number##1 }%
% \renewcommand{\DTMDisplaydate}{\DTMdisplaydate}%
%}
%\DTMsetdatestyle{usvardate}
%
%\pagestyle{headings}
%
%\begin{document}
%\section{\today: an example}
%
%\today.
%\end{document}
%\end{verbatim}
%This compiles without error now that \cs{ordinalnum} has been
%protected, but the page header appears as:
%\begin{quote}
%\emph{1\quad January 1\textsuperscript{st}, 2016: AN EXAMPLE}\hfill 1
%\end{quote}
%The date hasn't been rendered in upper case so the header doesn't
%look right.
%
%If \sty{hyperref} is added, another problem becomes evident:
%\begin{verbatim}
%\documentclass{article}
%
%\usepackage{fmtcount}
%\usepackage[calc]{datetime2}
%\usepackage[colorlinks]{hyperref}
%
%\DTMnewdatestyle{usvardate}{%
% \renewcommand{\DTMdisplaydate}[4]{%
%  \DTMmonthname{##2} \protect\ordinalnum{##2}, \number##1 }%
% \renewcommand{\DTMDisplaydate}{\DTMdisplaydate}%
%}
%\DTMsetdatestyle{usvardate}
%
%\pagestyle{headings}
%
%\begin{document}
%\section{\today: an example}
%
%\today.
%\end{document}
%\end{verbatim}
%The PDF bookmarks for this document show the section title as:
%\begin{quote}
%01 , 2016: an example
%\end{quote}
%
%Both the header and the bookmark problem are caused by
%non-expandable elements of the date style. \emph{The same problems
%occur with \styfmt{datetime}}, and is one of the main reasons for
%developing a replacement package since these problems can't be fixed
%by \styfmt{datetime}. In fact, the \styfmt{datetime} version of this
%example looks even worse in the bookmarks:
%\begin{verbatim}
%\documentclass{article}
%
%\usepackage{datetime}
%\usepackage[colorlinks]{hyperref}
%
%\newdateformat{usvardate}{\monthname[\THEMONTH] \ordinalnum{\THEDAY}, \THEYEAR}
%\usvardate
%
%\pagestyle{headings}
%
%\begin{document}
%\section{\today: an example}
%
%\today.
%\end{document}
%\end{verbatim}
%The bookmark now looks like:
%\begin{quote}
%===[0], 0: an example
%\end{quote}
%(The page header is the same as for the \styfmt{datetime2} example
%above, with the date not matching the rest of the heading case.)
%
%A more appropriate way of defining this style with
%\styfmt{datetime2} package is to use the \emph{expandable} commands
%provided by the \module{english} module:
%\begin{verbatim}
%\documentclass{article}
%
%\usepackage[calc,en-GB]{datetime2}
%\usepackage[colorlinks]{hyperref}
%
%\DTMnewdatestyle{usvardate}{%
% \renewcommand*{\DTMenglishfmtordsuffix}{\DTMenGBfmtordsuffix}%
% \renewcommand{\DTMdisplaydate}[4]{%
%  \DTMenglishmonthname{##2} \DTMenglishordinal{##2}, \number##1 }%
% \renewcommand{\DTMDisplaydate}{\DTMdisplaydate}%
%}
%\DTMsetdatestyle{usvardate}
%
%\pagestyle{headings}
%
%\begin{document}
%\section{\today: an example}
%
%\today.
%\end{document}
%\end{verbatim}
%This fixes both the header and the bookmarks.
%
%If you don't want to rely on remembering to use the \pkgopt{en-GB}
%option to load the \style{en-GB} style (which defines
%\cs{DTMenGBfmtordsuffix}) you can use this alternative:
%\begin{verbatim}
%\documentclass{article}
%
%\usepackage[calc]{datetime2}
%\usepackage[colorlinks]{hyperref}
%
%\DTMusemodule{english}{en-GB}
%
%\DTMnewdatestyle{usvardate}{%
% \renewcommand*{\DTMenglishfmtordsuffix}{\DTMenGBfmtordsuffix}%
% \renewcommand{\DTMdisplaydate}[4]{%
%  \DTMenglishmonthname{##2} \DTMenglishordinal{##2}, \number##1 }%
% \renewcommand{\DTMDisplaydate}{\DTMdisplaydate}%
%}
%\DTMsetdatestyle{usvardate}
%
%\pagestyle{headings}
%
%\begin{document}
%\section{\today: an example}
%
%\today.
%\end{document}
%\end{verbatim}
%This is a useful method to employ if you want to define the new
%style in a package without forcing the package options used to load
%\styfmt{datetime2}.
%
%Note that this new style definition is unaffected by language
%changes, so the other problem with the \styfmt{datetime} version of
%this style is also eliminated.
%
%\section{Defining a New Time Format}
%
%The \styfmt{datetime} package provides:
%\begin{definition}[\inpkg{datetime}]
%\oldcs{newtimeformat}\marg{name}\marg{format}
%\end{definition}
%to define a new time style. Within \meta{format}, the placeholder
%commands \oldcs{THEHOUR}, \oldcs{THEMINUTE}, \oldcs{THESECOND},
%\oldcs{THEHOURXII}, \oldcs{THETOHOUR}, \oldcs{THETOMINUTE} may be
%used. There are also corresponding placeholder counters:
%\texttt{HOUR}, \texttt{MINUTE}, \texttt{SECOND}, \texttt{HOURXII},
%\texttt{TOHOUR} and \texttt{TOMINUTE}.
%
%This placeholder style of format for the time has the same problems
%as that for the date styles described in the previous section.
%
%Time styles are defined in \styfmt{datetime2} using:
%\begin{definition}[\inpkg{datetime2}]
%\ics{DTMnewtimestyle}\marg{name}\marg{definition}
%\end{definition}
%
%As with the date styles, the \styfmt{datetime2} package provides
%styles that redefine a formatting macro. In this case, the time
%(without the zone) is simply formatted with
%\begin{definition}[\inpkg{datetime2}]
%\ics{DTMdisplaytime}\marg{hh}\marg{mm}\marg{ss}
%\end{definition}
%so the \meta{definition} part of \cs{DTMnewtimestyle} needs to
%redefine this command.
%
%The placeholder commands \cs{THEHOUR}, \cs{THEMINUTE}
%and \cs{THESECOND} from \styfmt{datetime} can now be represented by
%the first, second and third arguments of \cs{DTMdisplaytime}. The
%other placeholders are more complicated as they need to be
%calculated which may prevent the style from being an expandable
%format.
%
%The \styfmt{datetime} manual provides a simple example of defining a
%new time style:
%\begin{verbatim}
%\newtimeformat{dottime}{\twodigit{\THEHOUR}.\twodigit{\THEMINUTE}}
%\end{verbatim}
%This style is then set using:
%\begin{verbatim}
%\settimeformat{dottime}
%\end{verbatim}
%
%An equivalent time style can be defined with \styfmt{datetime2}:
%\begin{verbatim}
%\DTMnewtimestyle{dottime}{%
% \renewcommand*\DTMdisplaytime[3]{\DTMtwodigits{##1}.\DTMtwodigits{##2}}}
%\end{verbatim}
%
%A complete example using \styfmt{datetime}:
%\begin{verbatim}
%\documentclass{article}
%
%\usepackage{datetime}
%
%\newtimeformat{dottime}{\twodigit{\THEHOUR}.\twodigit{\THEMINUTE}}
%\settimeformat{dottime}
%
%\begin{document}
%\currenttime.
%\end{document}
%\end{verbatim}
%A complete example using \styfmt{datetime2}:
%\begin{verbatim}
%\documentclass{article}
%
%\usepackage{datetime2}
%
%\DTMnewtimestyle{dottime}{%
% \renewcommand*\DTMdisplaytime[3]{\DTMtwodigits{##1}.\DTMtwodigits{##2}}}
%\DTMsettimestyle{dottime}
%
%\begin{document}
%\DTMcurrenttime.
%\end{document}
%\end{verbatim}
%
%Note that the \styfmt{datetime} example has a problem with PDF
%bookmarks if the time is used in a sectioning command, for the same
%reasons as those discussed above in the previous section. This can
%be seen with a slight modification to the example:
%\begin{verbatim}
%\documentclass{article}
%
%\usepackage{datetime}
%\usepackage[colorlinks]{hyperref}
%
%\newtimeformat{dottime}{\twodigit{\THEHOUR}.\twodigit{\THEMINUTE}}
%\settimeformat{dottime}
%
%\begin{document}
%\section{\currenttime: an example}
%
%\currenttime.
%\end{document}
%\end{verbatim}
%The bookmark appears as:
%\begin{quote}
%====by 1=by 12by 12 by -=by -60by -1=0.00: an example
%\end{quote}
%
%The \styfmt{datetime2} example works fine with a similar
%modification:
%\begin{verbatim}
%\documentclass{article}
%
%\usepackage{datetime2}
%\usepackage[colorlinks]{hyperref}
%
%\DTMnewtimestyle{dottime}{%
% \renewcommand*\DTMdisplaytime[3]{\DTMtwodigits{##1}.\DTMtwodigits{##2}}}
%\DTMsettimestyle{dottime}
%
%\begin{document}
%\section{\DTMcurrenttime: an example}
%
%\DTMcurrenttime.
%\end{document}
%\end{verbatim}
%
%There is one slight drawback with this example that's irrelevant to the choice
%of package. The bookmark (and table of contents, if present) will always show
%the time from the previous \LaTeX\ run.  This is only a problem when using the
%current time, especially if seconds are required or the document build takes
%longer than a minute. If the example has a specific time instead of one that
%changes for every \LaTeX\ run, then this issue is eliminated.
%
%If you want a time style that requires the other placeholder
%commands provided by \styfmt{datetime}, then it's more complicated
%to convert to \styfmt{datetime2} as the values that
%\styfmt{datetime} conveniently computes and stores in the
%placeholder commands \oldcs{THEHOURXII}, \oldcs{THETOHOUR} and
%\oldcs{THETOMINUTE} now need to be calculated, preferably in an
%expandable way.
%
%An expandable twelve hour time format can be illustrated with the
%\style{englishampm} style provided by the \module{english}
%module:\footnote{Bug fix from \styfmt{datetime2-english} v1.03
%included}
%\begin{verbatim}
%\DTMnewtimestyle
% {englishampm}% label
% {%
%    \renewcommand*\DTMdisplaytime[3]{%
%      \ifnum##2=0
%        \ifnum##1=12
%          \DTMtexorpdfstring
%            {\DTMenglishampmfmt{\DTMenglishnoon}}%
%            {\DTMenglishnoon}%
%        \else
%          \ifnum##1=0
%            \DTMtexorpdfstring
%            {\DTMenglishampmfmt{\DTMenglishmidnight}}%
%            {\DTMenglishmidnight}%
%          \else
%            \ifnum##1=24
%              \DTMtexorpdfstring
%              {\DTMenglishampmfmt{\DTMenglishmidnight}}%
%              {\DTMenglishmidnight}%
%            \else
%              \ifnum##1<12
%                \number##1
%                \DTMtexorpdfstring
%                {\DTMenglishampmfmt{\DTMenglisham}}%
%                {\DTMenglisham}%
%              \else
%                \number\numexpr##1-12\relax
%                \DTMtexorpdfstring
%                {\DTMenglishampmfmt{\DTMenglishpm}}%
%                {\DTMenglishpm}%
%              \fi
%              \fi
%            \fi
%          \fi
%        \fi
%      \else
%        \ifnum##1<13
%          \ifnum##1=0
%            12%
%          \else
%            \number##1
%          \fi
%          \DTMenglishtimesep\DTMtwodigits{##2}%
%          \ifnum##1=12
%            \DTMtexorpdfstring
%            {\DTMenglishampmfmt{\DTMenglishpm}}%
%            {\DTMenglishpm}%
%          \else
%            \DTMtexorpdfstring
%            {\DTMenglishampmfmt{\DTMenglisham}}%
%            {\DTMenglisham}%
%          \fi
%        \else
%          \number\numexpr##1-12\relax
%          \DTMenglishtimesep\DTMtwodigits{##2}%
%          \ifnum##1=24
%            \DTMtexorpdfstring
%            {\DTMenglishampmfmt{\DTMenglisham}}%
%            {\DTMenglisham}%
%          \else
%            \DTMtexorpdfstring
%            {\DTMenglishampmfmt{\DTMenglishpm}}%
%            {\DTMenglishpm}%
%          \fi
%        \fi
%      \fi
%    }%
% }%
%\end{verbatim}
%
%This is certainly more complicated than the
%\texttt{ampmtime} format provided by \styfmt{datetime}, which is
%defined as:
%\begin{verbatim}
%\newtimeformat{ampmtime}%
%{%
% \ifthenelse{\value{HOUR}=0}{12}{\THEHOURXII}%
% \timeseparator
% \twodigit{\THEMINUTE}%
% \ifthenelse{\value{HOUR}<12}{\amname}%
% {%
%   \ifthenelse{\value{HOUR}=12}{ \noon}{\pmname}%
% }%
%}
%\end{verbatim}
%However the \styfmt{datetime2} version produces better results,
%especially where expansion is required (for example, in bookmarks or
%writing a time stamp to an external file). The \styfmt{datetime2}
%version also performs extra checks for midnight where the
%\styfmt{datetime} version produces ambiguous text. So the
%\texttt{ampmtime} definition is simple but buggy.
%
%The basic algorithm for \style{englishampm} is:
%\begin{description}
%\item If the minute value is 0 (\verb|\ifnum##2=0|):
% \begin{description}
%  \item If the hour value is 12 (\verb|\ifnum##1=12|):
%   \begin{description}
%    \item Print \qt{noon}
%   \end{description}
%
%  \item Otherwise (not noon but on the hour)
%
%   \begin{description}
%    \item If the hour value is 0 (\verb|\ifnum##1=0|)
%     \begin{description}
%      \item Print \qt{midnight}
%     \end{description}
%
%    \item Otherwise (not 12:00 or 00:00 but on the hour)
%
%     \begin{description}
%      \item If the hour value is 24 (\verb|\ifnum##1=24|)
%       \begin{description}
%        \item Print \qt{midnight}
%       \end{description}
%
%      \item Otherwise (not 12:00 or 00:00 or 24:00 but on the hour)
%       \begin{description}
%        \item If the hour value is less than 12 (\verb|\ifnum##1<12|)
%          \begin{description}
%           \item Print the hour value (\verb|\number##1|)
%           \item Print \qt{am}
%          \end{description}
%        \item Otherwise
%          \begin{description}
%           \item Print the hour value less 12 
%                 (\verb|\number\numexpr##1-12|)
%           \item Print \qt{pm}
%          \end{description}
%       \end{description}
%     \end{description}
%   \end{description}
% \end{description}
%
% \item Otherwise (minute value isn't zero)
%
% \begin{description}
%  \item If the hour value is less than 13 (\verb|\ifnum##1<13|)
%
%  \begin{description}
%   \item If the hour value is 0 (\verb|\ifnum##1=0|)
%    \begin{description}
%     \item Print \qt{12}
%    \end{description}
%   \item Otherwise
%    \begin{description}
%      \item Print the hour (\verb|\number##1|)
%    \end{description}
%
%   \item Print the separator between the hour and minute
%         (\verb|\DTMenglishtimesep|)
%
%   \item Print minute value (\verb|\DTMtwodigits{##2}|)
%
%   \item If the hour value is 12 (\verb|\ifnum##=12|)
%    \begin{description}
%      \item Print \qt{pm}
%    \end{description}
%   \item Otherwise (hour value less than 12)
%    \begin{description}
%      \item Print \qt{am}
%    \end{description}
%  \end{description}
%
%  \item Otherwise (hour value $\ge13$)
%
%  \begin{description}
%   \item Print the hour value less 12 
%         (\verb|\number\numexpr##1-12|)
%   \item Print the time separator
%
%   \item Print minute value (\verb|\DTMtwodigits{##2}|)
%
%   \item If the hour value is 24 (\verb|\ifnum##1=24|)
%   \begin{description}
%    \item Print \qt{am}
%   \end{description}
%   \item Otherwise
%   \begin{description}
%    \item Print \qt{pm}
%   \end{description}
%  \end{description}
% \end{description}
%\end{description}
%Most of the complication in this style isn't trying to determine the
%equivalent value of \cs{THEHOURXII} (\verb|\number\numexpr##1-12|
%if \verb|##1>12|) but in improving the algorithm to catch special
%cases that the \styfmt{datetime} style misses.  Additionally,
%\ics{DTMtexorpdfstring} is used to prevent any font formatting
%commands from being added to the bookmarks.
%
%The \sty{datetime2-en-fulltext} package provides an example of a
%time style that requires calculating the minutes to the hour and the
%next hour. Note that fragile commands are protected in this style.
%Since it contains fragile commands that require protection, it can't
%be used in an expandable context.
%
%Remember that most of the styles provided by \styfmt{datetime2} and its
%associated modules and dependent packages are configurable, so if
%your preferred style is only marginally different to a predefined
%style, you may be able to tweak that style to fit your requirements.
%For example, the dot time format can be obtained using the
%\style{default} style with the seconds suppressed and the separator
%change to a dot:
%\begin{verbatim}
%\DTMsettimestyle{default}
%\DTMsetup{timesep={.},showseconds=false}
%\end{verbatim}
%
%The twelve hour format provided by the \module{english} module can also be 
%adjusted. This style honours the package-wide time-related option
%\pkgopt{hourminsep} and the \qt{am}, \qt{pm}, \qt{midnight} or
%\qt{noon} part is formatted according to:
%\begin{definition}[\inpkg{datetime2-english}]
%\ics{DTMenglishampmfmt}\marg{text}
%\end{definition}
%
%So a twelve hour format that uses a dot instead of a colon and has
%the text part in small caps can be obtained using:
%\begin{verbatim}
%\usepackage[english,hourminsep={.}]{datetime2}
%\renewcommand*{\DTMenglishampmfmt}[1]{\textsc{#1}}
%\end{verbatim}
%
%\StopEventually{\PrintChanges\PrintIndex}
%
%
%\chapter{The Code}
%\iffalse
%    \begin{macrocode}
%<*datetime2.sty>
%    \end{macrocode}
%\fi
%\section{datetime2.sty code}
%\changes{1.0}{2015-03-24}{Initial release}
%    \begin{macrocode}
\NeedsTeXFormat{LaTeX2e}
\ProvidesPackage{datetime2}[2016/06/05 v1.5.1 (NLCT) date and time formats]
%    \end{macrocode}
% Use \sty{tracklang} to find out what languages have been loaded.
%    \begin{macrocode}
\RequirePackage{tracklang}
%    \end{macrocode}
% Also require \sty{etoolbox}.
%    \begin{macrocode}
\RequirePackage{etoolbox}
%    \end{macrocode}
% Need \sty{xkeyval} for \meta{key}=\meta{value} interface.
%    \begin{macrocode}
\RequirePackage{xkeyval}[2006/11/18]
%    \end{macrocode}
%
%\begin{macro}{\dtm@pdfcreationdate}
%\changes{1.5}{2016-06-04}{new}
%The \sty{luatex85} package defines \ics{pdfcreationdate} in terms of
%\cs{pdffeedback}, but the parsing commands need a command whose
%replacement text is directly in the PDF date time format, so define a command
%with the full replacement text that can be used instead. This will allow
%for any possible future changes of \cs{pdfcreationdate} that 
%require deeper levels of expansion.
%    \begin{macrocode}
\ifdef\pdfcreationdate
{%
  \edef\dtm@pdfcreationdate{\pdfcreationdate}%
}%
{%
%    \end{macrocode} 
%\changes{1.5.1}{2016-06-05}{added check for \cs{pdffeedback}}
% Check if newer version of Lua\TeX\ is being used but
% \sty{luatex85} hasn't been loaded.
%    \begin{macrocode}
  \ifdef\pdffeedback
  {%
     \edef\dtm@pdfcreationdate{\pdffeedback creationdate}%
  }%
  {}%
}
%    \end{macrocode} 
%\end{macro}
%
%\begin{macro}{\dtm@yearmonthsep}
% Separator between year and month for numeric dates.
%    \begin{macrocode}
\newcommand*{\dtm@yearmonthsep}{-}
%    \end{macrocode}
%\end{macro}
%
%\begin{macro}{\dtm@monthdaysep}
% Separator between month and day for numeric dates.
%    \begin{macrocode}
\newcommand*{\dtm@monthdaysep}{-}
%    \end{macrocode}
%\end{macro}
%
%\begin{macro}{\dtm@dayyearsep}
% Separator between day and year for numeric middle-endian dates.
%    \begin{macrocode}
\newcommand*{\dtm@dayyearsep}{-}
%    \end{macrocode}
%\end{macro}
%
%\begin{macro}{\dtm@hourminsep}
% Separator between the hour and minute for times.
%    \begin{macrocode}
\newcommand*{\dtm@hourminsep}{:}
%    \end{macrocode}
%\end{macro}
%
%\begin{macro}{\dtm@minsecsep}
% Separator between the minute and second for times.
%    \begin{macrocode}
\newcommand*{\dtm@minsecsep}{:}
%    \end{macrocode}
%\end{macro}
%
%\begin{macro}{\dtm@timezonesep}
% Separator between the date and time.
%    \begin{macrocode}
\newcommand*{\dtm@datetimesep}{\space}%
%    \end{macrocode}
%\end{macro}
%
%\begin{macro}{\dtm@timezonesep}
% Separator between the time and time zone.
%    \begin{macrocode}
\newcommand*{\dtm@timezonesep}{}
%    \end{macrocode}
%\end{macro}
%
%\begin{option}{datesep}
% Set year/month and month/day separator.
%    \begin{macrocode}
\define@key{datetime2.sty}{datesep}{%
  \renewcommand*{\dtm@yearmonthsep}{#1}%
  \renewcommand*{\dtm@monthdaysep}{#1}%
  \renewcommand*{\dtm@dayyearsep}{#1}%
}
%    \end{macrocode}
%\end{option}
%
%\begin{option}{yearmonthsep}
% Set year/month separator.
%    \begin{macrocode}
\define@key{datetime2.sty}{yearmonthsep}{%
  \renewcommand*{\dtm@yearmonthsep}{#1}%
}
%    \end{macrocode}
%\end{option}
%
%\begin{option}{monthdaysep}
% Set month/day separator.
%    \begin{macrocode}
\define@key{datetime2.sty}{monthdaysep}{%
  \renewcommand*{\dtm@monthdaysep}{#1}%
}
%    \end{macrocode}
%\end{option}
%
%\begin{option}{dayyearsep}
% Set day/year separator for middle-endian dates.
%    \begin{macrocode}
\define@key{datetime2.sty}{dayyearsep}{%
  \renewcommand*{\dtm@dayyearsep}{#1}%
}
%    \end{macrocode}
%\end{option}
%
%\begin{option}{timesep}
% Set hour/minute and minute/second separator.
%    \begin{macrocode}
\define@key{datetime2.sty}{timesep}{%
  \renewcommand*{\dtm@hourminsep}{#1}%
  \renewcommand*{\dtm@minsecsep}{#1}%
}
%    \end{macrocode}
%\end{option}
%
%\begin{option}{hourminsep}
% Set hour/minute separator.
%    \begin{macrocode}
\define@key{datetime2.sty}{hourminsep}{%
  \renewcommand*{\dtm@hourminsep}{#1}%
}
%    \end{macrocode}
%\end{option}
%
%\begin{option}{minsecsep}
% Set minute/second separator.
%    \begin{macrocode}
\define@key{datetime2.sty}{minsecsep}{%
  \renewcommand*{\dtm@minsecsep}{#1}%
}
%    \end{macrocode}
%\end{option}
%
%\begin{option}{timezonesep}
% Set separator between the time and the time zone (used in
% \cs{DTMnow}).
%    \begin{macrocode}
\define@key{datetime2.sty}{timezonesep}{%
  \renewcommand*{\dtm@timezonesep}{#1}%
}
%    \end{macrocode}
%\end{option}
%
%\begin{option}{datetimesep}
% Set separator between the date and the time (used in
% \cs{DTMnow}).
%    \begin{macrocode}
\define@key{datetime2.sty}{datetimesep}{%
  \renewcommand*{\dtm@datetimesep}{#1}%
}
%    \end{macrocode}
%\end{option}
%
%\begin{option}{showseconds}
% Boolean key to determine whether or not to show the seconds.
%    \begin{macrocode}
\define@boolkey{datetime2.sty}[DTM]{showseconds}[true]{}
%    \end{macrocode}
%\end{option}
%
%\begin{option}{showdate}
% Boolean key to determine whether or not to show the date in
% \cs{DTMdisplay} and \cs{DTMDisplay}.
%    \begin{macrocode}
\define@boolkey{datetime2.sty}[DTM]{showdate}[true]{}
\DTMshowdatetrue
%    \end{macrocode}
%\end{option}
%
%\begin{option}{showzone}
% Boolean key to determine whether or not to show the time zone in
% \cs{DTMdisplay} and \cs{DTMDisplay}.
%    \begin{macrocode}
\define@boolkey{datetime2.sty}[DTM]{showzone}[true]{}
%    \end{macrocode}
%\end{option}
%
%\begin{option}{showisoZ}
% Boolean key to determine whether or not to use \texttt{Z} instead
% of \texttt{+00:00} for UTC in the \style{default}, \style{iso} or
% \style{pdf} styles. (Other styles may also use this.)
%    \begin{macrocode}
\define@boolkey{datetime2.sty}[DTM]{showisoZ}[true]{}
\DTMshowisoZtrue
%    \end{macrocode}
%\end{option}
%
% Switch off seconds and time zone if \cs{dtm@pdfcreationdate}
% isn't defined, otherwise switch on.
%    \begin{macrocode}
\ifdef\dtm@pdfcreationdate
{%
  \DTMshowsecondstrue
  \DTMshowzonetrue
}%
{%
  \DTMshowsecondsfalse
  \DTMshowzonefalse
}%
%    \end{macrocode}
%
%\begin{option}{showzoneminutes}
% Boolean key to determine whether or not to show the time zone
% minutes. (If \cs{DTMshowzonefalse} then this option is irrelevant.)
%    \begin{macrocode}
\define@boolkey{datetime2.sty}[DTM]{showzoneminutes}[true]{}
\DTMshowzoneminutestrue
%    \end{macrocode}
%\end{option}
%
%\begin{macro}{\DTMifcaseregional}
%\begin{definition}
%\cs{DTMifcaseregional}\marg{false}\marg{text}\marg{numeric}
%\end{definition}
% Determines if the user wants the language modules to set the
% regional format. The first argument \meta{false} indicates that
% they don't want the regional format set, the second argument
% \meta{text} indicates they want the textual format (e.g. 1st
% March, 2015 or March 1, 2005) and the third argument \meta{numeric}
% indicates they want the numeric format (e.g. 1/3/2015 or
% 3/1/2015). A change in the setting will only have an affect when
% the module is loaded and when \cs{date}\meta{language} is used to
% set the style. The default is false.
%    \begin{macrocode}
\newcommand*{\DTMifcaseregional}[3]{#1}
%    \end{macrocode}
%\end{macro}
%
%\begin{option}{useregional}
% Setting to determine whether or not to use the regional
% settings (if any are loaded).
%    \begin{macrocode}
\define@choicekey{datetime2.sty}{useregional}[\val\nr]%
 {false,text,numeric,num}[text]%
 {%
   \ifcase\nr\relax
     \renewcommand*{\DTMifcaseregional}[3]{##1}%
   \or
     \renewcommand*{\DTMifcaseregional}[3]{##2}%
   \or
     \renewcommand*{\DTMifcaseregional}[3]{##3}%
   \or
     \renewcommand*{\DTMifcaseregional}[3]{##3}%
   \fi
 }
%    \end{macrocode}
%\end{option}
%
%\begin{macro}{\@dtm@setusecalc}
%    \begin{macrocode}
\newcommand*{\@dtm@setusecalc}{%
  \renewcommand*{\@dtm@usecalc}{\RequirePackage{datetime2-calc}}%
}
%    \end{macrocode}
%\end{macro}
%\begin{macro}{\@dtm@usecalc}
%    \begin{macrocode}
\newcommand*{\@dtm@usecalc}{}
%    \end{macrocode}
%\end{macro}
% Disable attempt to load \sty{datetime2-calc} in the document.
%    \begin{macrocode}
\AtBeginDocument{%
  \@ifpackageloaded{datetime2-calc}%
  {%
    \renewcommand*{\@dtm@setusecalc}{}%
  }% 
  {%
    \renewcommand*{\@dtm@setusecalc}{%
      \PackageError{datetime2}{You must load `datetime2-calc'
      package to use option `showdow'}{Try one of the following:^^J
      pass `calc' option to `datetime2' package when you load it^^J
      or move `showdow' option to `datetime2' package option list^^J
      or move \string\DTLsetup\space to the preamble.}%
    }%
  }%
}
%    \end{macrocode}
%
%\begin{option}{calc}
% This option will load the \sty{datetime2-calc} which uses the
% \sty{pgfcalendar} package to compute the day of week and offsets.
% The package is loaded at the end of this one.
%    \begin{macrocode}
\DeclareOptionX{calc}{\@dtm@setusecalc}
%    \end{macrocode}
%\end{option}
%
%\begin{option}{showdow}
% Boolean key to determine whether or not to show the day of week
% for the styles that can show the day of week. If this is switched
% on, then \sty{datetime2-calc} is required. If this key is set later
% in the document with \cs{DTMsetup}, then the \sty{datetime2-calc}
% package must previously be loaded for it to have an effect.
%    \begin{macrocode}
\define@boolkey{datetime2.sty}[DTM]{showdow}[true]{%
  \ifDTMshowdow \@dtm@setusecalc \fi
}
\DTMshowdowfalse
%    \end{macrocode}
%\end{option}
%
%\begin{macro}{\@dtm@warning}
% Warning messages.
%    \begin{macrocode}
\newcommand*{\@dtm@warning}[1]{%
  \if@dtm@warn
    \PackageWarning{datetime2}{#1}%
  \fi
}
%    \end{macrocode}
%\end{macro}
%
%\begin{option}{warn}
% Allow user to suppress package warnings.
%    \begin{macrocode}
\define@boolkey{datetime2.sty}[@dtm@]{warn}[true]{}
\@dtm@warntrue
%    \end{macrocode}
%\end{option}
%
%\begin{macro}{\@dtm@initialstyle}
%    \begin{macrocode}
\newcommand*{\@dtm@initialstyle}{}
%    \end{macrocode}
%\end{macro}
%\begin{option}{style}
% Set the style. This automatically sets
% \pkgopt[false]{useregional}.
%    \begin{macrocode}
\define@key{datetime2.sty}{style}{%
  \renewcommand*{\@dtm@initialstyle}{#1}%
  \ifstrempty{#1}%
  {}%
  {%
    \renewcommand*{\DTMifcaseregional}[3]{##1}%
  }%
}
%    \end{macrocode}
%\end{option}
%
% Pass any unknown options to \sty{tracklang}.
% This will automatically switch the "useregional" setting to
% \texttt{text}.
%\changes{1.3}{2016-01-22}{added predefined dialect test}
%    \begin{macrocode}
\DeclareOptionX*{%
  \ifcsundef{@tracklang@add@\CurrentOption}%
  {%
    \PackageError{datetime2}{`\CurrentOption' is not a recognised dialect.
    \MessageBreak Perhaps you have misspelt it or the 
    \MessageBreak named dialect may be unsupported or
    \MessageBreak perhaps you forgot to use the `style' key}%
    {Any options that aren't described in the manual are assumed
    \MessageBreak to be language or dialect names.}%
  }%
  {%
    \TrackPredefinedDialect{\CurrentOption}%
    \renewcommand*{\DTMifcaseregional}[3]{#2}%
  }%
}
%    \end{macrocode}
%
%Process options passed to this package:
%    \begin{macrocode}
\ProcessOptionsX
%    \end{macrocode}
%
% Disable \texttt{calc} option. If it's required, just load
% \sty{datetime2-calc} with \cs{usepackage}.
%    \begin{macrocode}
\disable@keys{datetime2.sty}{calc}
%    \end{macrocode}
% Disable \texttt{style} option. If it's required, just 
% use \cs{DTMsetup}.
%    \begin{macrocode}
\disable@keys{datetime2.sty}{style}
%    \end{macrocode}
%
% Provide a way to set options after package has been loaded.
%\begin{macro}{\DTMsetup}
%    \begin{macrocode}
\newcommand*{\DTMsetup}[1]{%
 \def\@dtm@usecalc{}%
 \setkeys{datetime2.sty}{#1}%
 \@dtm@usecalc
}
%    \end{macrocode}
%\end{macro}
%
%\subsection{Defaults}
% This section sets up the defaults.
%
%\begin{macro}{\@dtm@parsedate}
% Parse date in the format \meta{year}-\meta{month}-\meta{day}.
% The arguments are expanded. (This is redefined by
% \sty{datetime2-calc}.)
%    \begin{macrocode}
\def\@dtm@parsedate#1-#2-#3\@dtm@endparsedate{%
  \edef\@dtm@year{\number#1}%
  \edef\@dtm@month{\number#2}%
  \edef\@dtm@day{\number#3}%
  \def\@dtm@dow{-1}%
}
%    \end{macrocode}
%\end{macro}
%
%\begin{macro}{\@dtm@parsetime}
% Define command to parse time in the format
% \meta{h}:\meta{m}:\meta{s}. The results are stored in
% \cs{@dtm@hour}, \cs{@dtm@minute} and \cs{@dtm@second}.
% The arguments are expanded.
%    \begin{macrocode}
\def\@dtm@parsetime#1:#2:#3\@dtm@endparsetime{%
  \edef\@dtm@hour{\number#1}%
  \edef\@dtm@minute{\number#2}%
  \edef\@dtm@second{\number#3}%
}
%    \end{macrocode}
%\end{macro}
%
%\begin{macro}{\@dtm@parsetimezn}
% Define command to parse time in the format
% \meta{h}:\meta{m}:\meta{s} \meta{znh}:\meta{znm}. The results are stored in
% \cs{@dtm@hour}, \cs{@dtm@minute}, \cs{@dtm@second},
% \cs{@dtm@timezonehour} and \cs{@dtm@timezoneminute}.
% The arguments are expanded.
%    \begin{macrocode}
\def\@dtm@parsetimezn#1:#2:#3 #4\@dtm@endparsetimezn{%
  \@dtm@parsetime#1:#2:#3\@dtm@endparsetime
  \@dtm@parsezone{#4}%
}
%    \end{macrocode}
%\end{macro}
%
%\begin{macro}{\@dtm@parsezone}
% Define command to parse time zone in the format
% \texttt{Z} or \meta{znh}:\meta{znm}. The results are stored in
% \cs{@dtm@timezonehour} and \cs{@dtm@timezoneminute}.
% The arguments are expanded in the event that registers are used.
%    \begin{macrocode}
\newcommand*{\@dtm@parsezone}[1]{%
  \ifstrequal{#1}{Z}%
  {%
    \def\@dtm@timezonehour{+00}%
    \def\@dtm@timezoneminute{00}%
  }%
  {%
    \@dtm@parse@zone#1\@dtm@endparse@zone
  }%
}
\def\@dtm@parse@zone#1:#2\@dtm@endparse@zone{%
  \edef\@dtm@timezonehour{\number#1}%
  \edef\@dtm@timezoneminute{\number#2}%
}
%    \end{macrocode}
%\end{macro}
%
%\begin{macro}{\@dtm@parsetimestamp}
% Parse date and time in ISO format
%\meta{YYYY}-\meta{MM}-\meta{DD}T\meta{hh}:\meta{mm}:\meta{sec}\meta{time
%zone}
% where \meta{time zone} may be \texttt{Z} or in the form
% \meta{hh}:\meta{mm} (where \meta{hh} includes the sign).
%    \begin{macrocode}
\def\@dtm@parsetimestamp#1-#2-#3T#4:#5:#6#7#8\@dtm@endparsetimestamp{%
  \@dtm@parsedate#1-#2-#3\@dtm@endparsedate
  \@dtm@parsetime#4:#5:#6#7\@dtm@endparsetime
  \@dtm@parsezone{#8}%
}
%    \end{macrocode}
%\end{macro}
%
%\begin{macro}{\DTMsavefilemoddate}
% Not available for some engines.
%    \begin{macrocode}
\newcommand*{\DTMsavefilemoddate}[2]{%
  \@dtm@warning{Your TeX engine doesn't support accessing
  file modification dates}%
  \cslet{@dtm@#1@year}{0}%
  \cslet{@dtm@#1@month}{0}%
  \cslet{@dtm@#1@day}{0}%
  \cslet{@dtm@#1@dow}{-1}%
  \cslet{@dtm@#1@hour}{0}%
  \cslet{@dtm@#1@minute}{0}%
  \cslet{@dtm@#1@second}{0}%
  \cslet{@dtm@#1@TZhour}{0}%
  \cslet{@dtm@#1@TZminute}{0}%
}
%    \end{macrocode}
%\end{macro}
%
% Find out the current time. If \cs{dtm@pdfcreationdate} is defined,
% it can be fetched from that.
%    \begin{macrocode}
\ifdef\dtm@pdfcreationdate
{%
%    \end{macrocode}
% Define commands to parse \cs{dtm@pdfcreationdate}
%    \begin{macrocode}
  \def\@dtm@parsepdfdatetime#1:#2#3#4#5#6#7#8#9{%
    \def\@dtm@year{#2#3#4#5}%
    \def\@dtm@month{#6#7}%
    \def\@dtm@day{#8#9}%
    \@dtm@parsepdftime
  }
  \def\@dtm@parsepdftime#1#2#3#4#5#6#7\@dtm@endparsepdfdatetime{%
    \def\@dtm@hour{#1#2}%
    \def\@dtm@minute{#3#4}%
    \def\@dtm@second{#5#6}%
    \ifstrequal{#7}{Z}%
    {%
      \def\@dtm@timezonehour{00}%
      \def\@dtm@timezoneminute{00}%
    }%
    {%
      \@dtm@parsepdftimezone#7%
    }%
  }
  \def\@dtm@parsepdftimezone#1'#2'{%
    \def\@dtm@timezonehour{#1}%
    \def\@dtm@timezoneminute{#2}%
  }%
%    \end{macrocode}
% Now parse \cs{dtm@pdfcreationdate}
%    \begin{macrocode}
  \expandafter\@dtm@parsepdfdatetime\dtm@pdfcreationdate\@dtm@endparsepdfdatetime
%    \end{macrocode}
% Save the values.
%    \begin{macrocode}
  \let\@dtm@currentyear\@dtm@year
  \let\@dtm@currentmonth\@dtm@month
  \let\@dtm@currentday\@dtm@day
  \let\@dtm@currenthour\@dtm@hour
  \let\@dtm@currentminute\@dtm@minute
  \let\@dtm@currentsecond\@dtm@second
  \let\@dtm@currenttimezonehour\@dtm@timezonehour
  \let\@dtm@currenttimezoneminute\@dtm@timezoneminute
%
%    \end{macrocode}
% Lua\TeX\ doesn't provide \cs{pdffilemoddate}.
%    \begin{macrocode}
  \ifdef\pdffilemoddate
  {%
    \renewcommand*{\DTMsavefilemoddate}[2]{%
      \expandafter\@dtm@parsepdfdatetime\pdffilemoddate{#2}\@dtm@endparsepdfdatetime
      \cslet{@dtm@#1@year}{\@dtm@year}%
      \cslet{@dtm@#1@month}{\@dtm@month}%
      \cslet{@dtm@#1@day}{\@dtm@day}%
      \cslet{@dtm@#1@dow}{\@dtm@dow}%
      \cslet{@dtm@#1@hour}{\@dtm@hour}%
      \cslet{@dtm@#1@minute}{\@dtm@minute}%
      \cslet{@dtm@#1@second}{\@dtm@second}%
      \cslet{@dtm@#1@TZhour}{\@dtm@timezonehour}%
      \cslet{@dtm@#1@TZminute}{\@dtm@timezoneminute}%
    }
  }%
  {%
%    \end{macrocode}
% Lua time zone information provided by \verb|%z| is OS dependent,
% so this might not work.
%    \begin{macrocode}
    \ifdef\directlua
    {
      \renewcommand*{\DTMsavefilemoddate}[2]{%
        \expandafter\@dtm@parseluadatetime
          \directlua{tex.print(os.date(
             "\expandafter\@gobble\string\%Y-%
              \expandafter\@gobble\string\%m-%
              \expandafter\@gobble\string\%d-%
              \expandafter\@gobble\string\%w
              \expandafter\@gobble\string\%H:%
              \expandafter\@gobble\string\%M:%
              \expandafter\@gobble\string\%S
              \expandafter\@gobble\string\%z",
            lfs.attributes("#2").modification))}%
         \@dtm@endparseluadatetime
         \cslet{@dtm@#1@year}{\@dtm@year}%
         \cslet{@dtm@#1@month}{\@dtm@month}%
         \cslet{@dtm@#1@day}{\@dtm@day}%
         \cslet{@dtm@#1@dow}{\@dtm@dow}%
         \cslet{@dtm@#1@hour}{\@dtm@hour}%
         \cslet{@dtm@#1@minute}{\@dtm@minute}%
         \cslet{@dtm@#1@second}{\@dtm@second}%
         \cslet{@dtm@#1@TZhour}{\@dtm@TZhour}%
         \cslet{@dtm@#1@TZminute}{\@dtm@TZminute}%
      }
      \def\@dtm@parseluadatetime#1-#2-#3-#4 #5:#6:#7 #8\@dtm@endparseluadatetime{%
         \edef\@dtm@year{\number#1}%
         \edef\@dtm@month{\number#2}%
         \edef\@dtm@day{\number#3}%
         \edef\@dtm@dow{\number#4}%
         \edef\@dtm@hour{\number#5}%
         \edef\@dtm@minute{\number#6}%
         \edef\@dtm@second{\number#7}%
         \@dtm@parseluatimezone#8000000\@dtm@endparseluatimezone
      }
      \def\@dtm@parseluatimezone#1#2#3#4#5#6{%
         \ifstrequal{#1}{+}%
         {%
           \def\@dtm@TZhour{#1#2#3}%
           \ifstrequal{#4}{:}%
           {%
              \def\@dtm@TZminute{#5#6}%
           }%
           {%
              \def\@dtm@TZminute{#4#5}%
           }%
         }%
         {%
           \ifstrequal{#1}{-}%
           {%
              \def\@dtm@TZhour{#1#2#3}%
              \ifstrequal{#4}{:}%
              {%
                 \def\@dtm@TZminute{#5#6}%
              }%
              {%
                 \def\@dtm@TZminute{#4#5}%
              }%
           }%
           {%
             \ifstrequal{#1}{Z}%
             {%
               \def\@dtm@TZhour{0}%
               \def\@dtm@TZminute{0}%
             }%
             {%
               \def\@dtm@TZhour{#1#2}%
               \ifstrequal{#3}{:}%
               {%
                  \def\@dtm@TZminute{#4#5}%
               }%
               {%
                  \def\@dtm@TZminute{#3#4}%
               }%
             }%
           }%
         }%
         \@@dtm@parseluatimezone
      }
      \def\@@dtm@parseluatimezone#1\@dtm@endparseluatimezone{%
      }
    }
    {}
  }%
}%
{%
%    \end{macrocode}
% \cs{pdfcreationdate} not defined. By a process of elimination, the
% \TeX\ engine is either \XeTeX\ or it's very old. (Or it may be a
% new version of Lua\TeX\ without \sty{luatex85}.) In this case, the
% seconds and time zone can't be obtained. The hour and minute need
% to be calculated from \TeX's \cs{time} primitive.
%    \begin{macrocode}
   \count@=\time\relax
   \divide\count@ by 60\relax
   \edef\@dtm@currenthour{\number\count@}%
   \multiply\count@ by -60\relax
   \advance\count@ by \time\relax
   \edef\@dtm@currentminute{\number\count@}%
   \newcommand*{\@dtm@currentsecond}{00}%
   \newcommand\@dtm@currenttimezonehour{00}%
   \newcommand\@dtm@currenttimezoneminute{00}%
%    \end{macrocode}
% Get the day, month and year from \TeX's primitives.
%    \begin{macrocode}
   \edef\@dtm@currentyear{\number\year}%
   \edef\@dtm@currentmonth{\number\month}%
   \edef\@dtm@currentday{\number\day}%
}
%    \end{macrocode}
% Make \cs{DTMsavefilemoddate} robust.
%    \begin{macrocode}
\robustify\DTMsavefilemoddate
%    \end{macrocode}
%
% Current day of week defaults to \texttt{-1} (that is, ignore it).
%\begin{macro}{\@dtm@currentdow}
%    \begin{macrocode}
\newcommand*{\@dtm@currentdow}{-1}
%    \end{macrocode}
%\end{macro}
%
% Allow \XeLaTeX\ users a way of manually setting the current time zone.
%\begin{macro}{\DTMsetcurrentzone}
%    \begin{macrocode}
\newcommand*{\DTMsetcurrentzone}[2]{%
  \renewcommand\@dtm@currenttimezonehour{#1}%
  \renewcommand\@dtm@currenttimezoneminute{#2}%
}
%    \end{macrocode}
%\end{macro}
%
%\begin{macro}{\DTMtoday}
%\changes{2016-02-11}{1.4}{new}
% Provided in case of conflict to obtain \styfmt{datetime2}'s
% version of \cs{today}.
%    \begin{macrocode}
\newcommand*{\DTMtoday}{%
 \DTMdisplaydate 
  {\@dtm@currentyear}%
  {\@dtm@currentmonth}%
  {\@dtm@currentday}%
  {\@dtm@currentdow}%
}
%    \end{macrocode}
%\end{macro}
%
%\begin{macro}{\today}
%\changes{2016-02-11}{1.4}{no longer using \cs{renewcommand}}
% Version 1.4 dropped \cs{renewcommand} in case
% \cs{today} hasn't already been defined.
%    \begin{macrocode}
\let\today\DTMtoday
%    \end{macrocode}
%\end{macro}
%
% The \cls{scrlttr2} class redefines \cs{today} at the start of the
% document, so check for this.
%    \begin{macrocode}
\@ifclassloaded{scrlttr2}{\AtBeginDocument{\let\today\DTMtoday}}{}
%    \end{macrocode}
%
%\begin{macro}{\DTMToday}
% First letter upper case version.
%\changes{2016-02-11}{1.4}{new}
% Added to v1.4 to provide \sty{datetime2}'s version in case of
% conflict.
%    \begin{macrocode}
\newcommand*{\DTMToday}{%
 \DTMDisplaydate 
  {\@dtm@currentyear}%
  {\@dtm@currentmonth}%
  {\@dtm@currentday}%
  {\@dtm@currentdow}%
}
%    \end{macrocode}
%\end{macro}
%
%\begin{macro}{\Today}
%    \begin{macrocode}
\let\Today\DTMToday
%    \end{macrocode}
%\end{macro}
%
%\begin{macro}{\DTMdisplaydate}
%\begin{definition}
%\cs{DTMdisplaydate}\marg{year}\marg{month}\marg{day}\marg{day of
%week}
%\end{definition}
% Display the given date. If the day of week is negative, ignore it.
% The default style ignores it regardless.
%    \begin{macrocode}
\newcommand*\DTMdisplaydate[4]{%
  \number#1\dtm@yearmonthsep\DTMtwodigits{#2}\dtm@monthdaysep\DTMtwodigits{#3}%
}%
%    \end{macrocode}
%\end{macro}
%
%\begin{macro}{\DTMDisplaydate}
% First letter upper case version. Defaults to \cs{DTMdisplaydate}.
%    \begin{macrocode}
\newcommand*{\DTMDisplaydate}{\DTMdisplaydate}
%    \end{macrocode}
%\end{macro}
%
%\begin{macro}{\DTMdate}
% Display date where date is specified in the format
%\meta{yyyy}-\meta{mm}-\meta{dd}. Use \cs{expandafter} in case
%argument is a control sequence containing the date.
% This command isn't expandable
%    \begin{macrocode}
\newrobustcmd*{\DTMdate}[1]{%
  \expandafter\@dtm@parsedate#1\@dtm@endparsedate
  \DTMdisplaydate{\@dtm@year}{\@dtm@month}{\@dtm@day}{\@dtm@dow}%
}
%    \end{macrocode}
%\end{macro}
%
%\begin{macro}{\DTMDate}
% Upper case version.
%    \begin{macrocode}
\newrobustcmd*{\DTMDate}[1]{%
  \expandafter\@dtm@parsedate#1\@dtm@endparsedate
  \DTMDisplaydate{\@dtm@year}{\@dtm@month}{\@dtm@day}{\@dtm@dow}%
}
%    \end{macrocode}
%\end{macro}
%
%\begin{macro}{\DTMcurrenttime}
% Display the current time.
%    \begin{macrocode}
\newcommand*{\DTMcurrenttime}{%
 \DTMdisplaytime
  {\@dtm@currenthour}%
  {\@dtm@currentminute}%
  {\@dtm@currentsecond}%
}
%    \end{macrocode}
%\end{macro}
%
%\begin{macro}{\DTMdisplaytime}
%\begin{definition}
%\cs{DTMdisplaytime}\marg{hour}\marg{minute}\marg{sec}
%\end{definition}
% Display the given time.
%    \begin{macrocode}
\newcommand*\DTMdisplaytime[3]{%
  \DTMtwodigits{#1}\dtm@hourminsep\DTMtwodigits{#2}%
  \ifDTMshowseconds\dtm@minsecsep\DTMtwodigits{#3}\fi
}%
%    \end{macrocode}
%\end{macro}
%
%\begin{macro}{\DTMtime}
% Display date where time is specified in the format
%\meta{hour}:\meta{minute}:\meta{seconds}. This uses \cs{expandafter} in case
%argument is a control sequence containing the time. Not expandable.
%    \begin{macrocode}
\newrobustcmd*{\DTMtime}[1]{%
  \@dtm@parsetime#1\@dtm@endparsetime
  \DTMdisplaytime{\@dtm@hour}{\@dtm@minute}{\@dtm@second}%
}
%    \end{macrocode}
%\end{macro}
%
%\begin{macro}{\DTMcurrentzone}
% Display the current time zone.
%    \begin{macrocode}
\newcommand*{\DTMcurrentzone}{%
 \DTMdisplayzone
  {\@dtm@currenttimezonehour}%
  {\@dtm@currenttimezoneminute}%
}
%    \end{macrocode}
%\end{macro}
%
%\begin{macro}{\DTMdisplayzone}
% Display time zone.
%    \begin{macrocode}
\newcommand*{\DTMdisplayzone}[2]{%
 \ifboolexpe
 { bool{DTMshowisoZ}
   and test{\ifnumequal{#1}{0}}
   and test{\ifnumequal{#2}{0}}
 }%
 {%
   Z%
 }%
 {%
  \ifnum#1<0\else+\fi\DTMtwodigits{#1}%
  \ifDTMshowzoneminutes\dtm@hourminsep\DTMtwodigits{#2}\fi
 }%
}
%    \end{macrocode}
%\end{macro}
%
%\begin{macro}{\DTMnow}
% Current date, time and time zone.
%    \begin{macrocode}
\newcommand*{\DTMnow}{%
 \DTMdisplay
 {\@dtm@currentyear}
 {\@dtm@currentmonth}
 {\@dtm@currentday}
 {\@dtm@currentdow}
 {\@dtm@currenthour}%
 {\@dtm@currentminute}%
 {\@dtm@currentsecond}%
 {\@dtm@currenttimezonehour}%
 {\@dtm@currenttimezoneminute}%
}
%    \end{macrocode}
%\end{macro}
%
%\begin{macro}{\DTMNow}
% Current date, time and time zone.
%    \begin{macrocode}
\newcommand*{\DTMNow}{%
 \DTMDisplay
 {\@dtm@currentyear}
 {\@dtm@currentmonth}
 {\@dtm@currentday}
 {\@dtm@currentdow}
 {\@dtm@currenthour}%
 {\@dtm@currentminute}%
 {\@dtm@currentsecond}%
 {\@dtm@currenttimezonehour}%
 {\@dtm@currenttimezoneminute}%
}
%    \end{macrocode}
%\end{macro}
%
%\begin{macro}{\DTMdisplay}
%\begin{definition}
%\cs{DTMdisplay}\marg{YYYY}\marg{MM}\marg{DD}\marg{DOW}\marg{hh}\marg{mm}\marg{ss}\newline\marg{TZh}\marg{TZm}
%\end{definition}
% Display the date and time.
%    \begin{macrocode}
\newcommand*{\DTMdisplay}[9]{%
 \ifDTMshowdate
   \DTMdisplaydate{#1}{#2}{#3}{#4}%
   \dtm@datetimesep
 \fi
 \DTMdisplaytime
  {#5}%
  {#6}%
  {#7}%
 \ifDTMshowzone
  \dtm@timezonesep
  \DTMdisplayzone
   {#8}%
   {#9}%
 \fi
}
%    \end{macrocode}
%\end{macro}
%
%\begin{macro}{\DTMDisplay}
%\begin{definition}
%\cs{DTMDisplay}\marg{YYYY}\marg{MM}\marg{DD}\marg{DOW}\marg{hh}\marg{mm}\marg{ss}\newline\marg{TZh}\marg{TZm}
%\end{definition}
% First letter upper case version. Defaults to \cs{DTMdisplay}.
%    \begin{macrocode}
\newcommand*{\DTMDisplay}{\DTMdisplay}
%    \end{macrocode}
%\end{macro}
%
%\subsection{Styles}
%
% Provide user level commands for displaying number as two digits.
% (Truncate if over 99, to allow for converting year to two digits).
%\begin{macro}{\DTMtwodigits}
%    \begin{macrocode}
\newcommand*{\DTMtwodigits}[1]{%
 \ifnum#1<0
  -\DTMtwodigits{-#1}%
 \else
   \ifnum#1<100
     \ifnum#1<10
       0\number#1
     \else
       \number#1
     \fi
   \else
%    \end{macrocode}
% \cs{numexpr} rounds rather than truncates integer division, which
% is a little awkward to get around in an expandable context.
%    \begin{macrocode}
     \ifnum\numexpr#1-(#1/100)*100<0
       \number\numexpr#1-((#1/100)-1)*100\relax
     \else
       \number\numexpr#1-(#1/100)*100\relax
     \fi
   \fi
 \fi
}
%    \end{macrocode}
%\end{macro}
%
%\begin{macro}{\DTMcentury}
% Expands to the given number divided by 100 rounded upwards (in
% absolute terms).
% Provided in case the user just wants the century.
%    \begin{macrocode}
\newcommand*{\DTMcentury}[1]{%
  \ifnum#1<0
    -\DTMcentury{-#1}%
  \else
    \ifnum\numexpr#1-(#1/100)*100<1
       \number\numexpr#1/100\relax
    \else
       \number\numexpr(#1/100)+1\relax
    \fi
  \fi
}
%    \end{macrocode}
%\end{macro}
%
%\begin{macro}{\DTMdivhundred}
% Expands to the given number divided by 100.
%    \begin{macrocode}
\newcommand*{\DTMdivhundred}[1]{%
  \ifnum#1<0
    -\DTMdivhundred{-#1}%
  \else
    \ifnum\numexpr#1-(#1/100)*100<0
       \number\numexpr(#1)/100-1\relax
    \else
       \number\numexpr((#1)/100)\relax
    \fi
  \fi
}
%    \end{macrocode}
%\end{macro}
%
%\begin{macro}{\DTMtexorpdfstring}
%Provide user with a command that will use \sty{hyperref}'s
%\cs{texorpdfstring} if \sty{hyperref} has been loaded. If
%\sty{hyperref} isn't loaded it just does the first argument.
%    \begin{macrocode}
\newcommand*{\DTMtexorpdfstring}[2]{#1}
\AtBeginDocument{%
  \@ifpackageloaded{hyperref}%
  {%
    \renewcommand*{\DTMtexorpdfstring}{\texorpdfstring}%
  }%
  {}%
}
%    \end{macrocode}
%\end{macro}
%
% Access separator:
%\begin{macro}{\DTMsep}
%    \begin{macrocode}
\newcommand*{\DTMsep}[1]{\csname dtm@#1sep\endcsname}
%    \end{macrocode}
%\end{macro}
%
% Date-only styles are stored internally as
% \cs{@dtm@datestyle@}\meta{label}, time-only styles are stored
% internally as \cs{@dtm@timestyle@}\meta{label}, zone-only styles
% are stored internally as \cs{@dtm@zonestyle@}\meta{label}.
%
%\begin{macro}{\DTMnewdatestyle}
% Define a new date-only style. This should only redefine
% \cs{DTMdisplaydate} and \cs{DTMDisplaydate}, which may or may not
% use the separators \cs{dtm@yearmonthsep} and \cs{dtm@monthdaysep}.
%    \begin{macrocode}
\newcommand*{\DTMnewdatestyle}[2]{%
  \ifcsdef{@dtm@datestyle@#1}%
  {%
    \PackageError{datetime2}{Date style `#1' already exists}{}%
  }%
  {%
     \csdef{@dtm@datestyle@#1}{#2}%
  }%
}
%    \end{macrocode}
%\end{macro}
%
%\begin{macro}{\DTMrenewdatestyle}
% Redefine a date style. This should only redefine
% \cs{DTMdisplaydate} and \cs{DTMDisplaydate}, which may or may not
% use the separators \cs{dtm@yearmonthsep} and \cs{dtm@monthdaysep}.
% This may also be used to modify the date part of a full style.
%\changes{1.2}{2015-11-10}{new}
%    \begin{macrocode}
\newcommand*{\DTMrenewdatestyle}[2]{%
  \ifcsundef{@dtm@datestyle@#1}%
  {%
    \PackageError{datetime2}{Date style `#1' doesn't exist}{}%
  }%
  {%
     \csdef{@dtm@datestyle@#1}{#2}%
  }%
}
%    \end{macrocode}
%\end{macro}
%
%\begin{macro}{\DTMprovidedatestyle}
% Define a date style if it doesn't already exist.
%\changes{1.2}{2015-11-10}{new}
%    \begin{macrocode}
\newcommand*{\DTMprovidedatestyle}[2]{%
  \ifcsdef{@dtm@datestyle@#1}%
  {%
  }%
  {%
     \csdef{@dtm@datestyle@#1}{#2}%
  }%
}
%    \end{macrocode}
%\end{macro}
%
%\begin{macro}{\DTMnewtimestyle}
% Define a new time-only style. This should only redefine
% \cs{DTMdisplaytime}, which may or may not use the separators
% \cs{dtm@hourminsep} and \cs{dtm@minsecsep}.
%    \begin{macrocode}
\newcommand*{\DTMnewtimestyle}[2]{%
  \ifcsdef{@dtm@timestyle@#1}%
  {%
    \PackageError{datetime2}{Time style `#1' already exists}{}%
  }%
  {%
     \csdef{@dtm@timestyle@#1}{#2}%
  }%
}
%    \end{macrocode}
%\end{macro}
%
%\begin{macro}{\DTMrenewtimestyle}
% Redefine a time style. This should only redefine
% \cs{DTMdisplaytime}, which may or may not use the separators
% \cs{dtm@hourminsep} and \cs{dtm@minsecsep}.
% This may also be used to modify the time part of a full style.
%\changes{1.2}{2015-11-10}{new}
%    \begin{macrocode}
\newcommand*{\DTMrenewtimestyle}[2]{%
  \ifcsundef{@dtm@timestyle@#1}%
  {%
    \PackageError{datetime2}{Time style `#1' doesn't exist}{}%
  }%
  {%
     \csdef{@dtm@timestyle@#1}{#2}%
  }%
}
%    \end{macrocode}
%\end{macro}
%
%\begin{macro}{\DTMprovidetimestyle}
% Define a time style if it doesn't already exist.
%\changes{1.2}{2015-11-10}{new}
%    \begin{macrocode}
\newcommand*{\DTMprovidetimestyle}[2]{%
  \ifcsdef{@dtm@timestyle@#1}%
  {%
  }%
  {%
     \csdef{@dtm@timestyle@#1}{#2}%
  }%
}
%    \end{macrocode}
%\end{macro}
%
%\begin{macro}{\DTMnewzonestyle}
% Define a new zone-only style. This should only redefine
% \cs{DTMdisplayzone}, which may or may not use the separator
% \cs{dtm@hourminsep}.
%    \begin{macrocode}
\newcommand*{\DTMnewzonestyle}[2]{%
  \ifcsdef{@dtm@zonestyle@#1}%
  {%
    \PackageError{datetime2}{Zone style `#1' already exists}{}%
  }%
  {%
     \csdef{@dtm@zonestyle@#1}{#2}%
  }%
}
%    \end{macrocode}
%\end{macro}
%
%\begin{macro}{\DTMrenewzonestyle}
% Redefine a new zone style. This should only redefine
% \cs{DTMdisplayzone}, which may or may not use the separator
% \cs{dtm@hourminsep}.
% This may also be used to modify the zone part of a full style.
%\changes{1.2}{2015-11-10}{new}
%    \begin{macrocode}
\newcommand*{\DTMrenewzonestyle}[2]{%
  \ifcsundef{@dtm@zonestyle@#1}%
  {%
    \PackageError{datetime2}{Zone style `#1' doesn't exist}{}%
  }%
  {%
     \csdef{@dtm@zonestyle@#1}{#2}%
  }%
}
%    \end{macrocode}
%\end{macro}
%
%\begin{macro}{\DTMprovidezonestyle}
% Defines a new zone style if it doesn't already exist.
%\changes{1.2}{2015-11-10}{new}
%    \begin{macrocode}
\newcommand*{\DTMprovidezonestyle}[2]{%
  \ifcsdef{@dtm@zonestyle@#1}%
  {%
  }%
  {%
     \csdef{@dtm@zonestyle@#1}{#2}%
  }%
}
%    \end{macrocode}
%\end{macro}
%
% Zone styles may use mappings to use a~regional time zone (such as
% GMT or BST). It's up to the language modules to define these
% mappings. A~mapping for time zone \meta{TZh}:\meta{TZm} is stored in 
% \cs{@dtm@zonemap@}\meta{TZh}\texttt{:}\meta{TZm}.
%\begin{macro}{\DTMdefzonemap}
%\begin{definition}
%\cs{DTMdefzonemap}\marg{TZh}\marg{TZm}\marg{map}
%\end{definition}
% This will override any previous mapping for the given time zone.
%    \begin{macrocode}
\newcommand*{\DTMdefzonemap}[3]{%
  \csdef{@dtm@zonemap@\DTMtwodigits{#1}:\DTMtwodigits{#2}}{#3}%
}
%    \end{macrocode}
%\end{macro}
%
%\begin{macro}{\DTMusezonemapordefault}
% Expands to the mapping or the default if not defined.
%    \begin{macrocode}
\newcommand*{\DTMusezonemapordefault}[2]{%
  \ifcsundef{@dtm@zonemap@\DTMtwodigits{#1}:\DTMtwodigits{#2}}%
  {%
    \ifnum#1<0\else+\fi
    \DTMtwodigits{#1}%
    \ifDTMshowzoneminutes\DTMsep{hourmin}\DTMtwodigits{#2}\fi
  }%
  {\csname @dtm@zonemap@\DTMtwodigits{#1}:\DTMtwodigits{#2}\endcsname}%
}
%    \end{macrocode}
%\end{macro}
%
%\begin{macro}{\DTMusezonemap}
% Expands to the mapping. (No check if defined.)
%    \begin{macrocode}
\newcommand*{\DTMusezonemap}[2]{%
  \csname @dtm@zonemap@\DTMtwodigits{#1}:\DTMtwodigits{#2}\endcsname
}
%    \end{macrocode}
%\end{macro}
%
%\begin{macro}{\DTMhaszonemap}
%    \begin{macrocode}
\newcommand*{\DTMhaszonemap}[4]{%
  \ifcsundef{@dtm@zonemap@\DTMtwodigits{#1}:\DTMtwodigits{#2}}{#4}{#3}%
}
%    \end{macrocode}
%\end{macro}
%
%\begin{macro}{\DTMclearmap}
% Undefines the given zone mapping. No check is made to determine if
% the map exists.
%    \begin{macrocode}
\newcommand*{\DTMclearmap}[2]{%
 \csundef{@dtm@zonemap@\DTMtwodigits{#1}:\DTMtwodigits{#2}}%
}
%    \end{macrocode}
%\end{macro}
%
%\begin{macro}{\DTMshowmap}
%Debugging command.
%    \begin{macrocode}
\newcommand*{\DTMshowmap}[2]{%
 \csshow{@dtm@zonemap@\DTMtwodigits{#1}:\DTMtwodigits{#2}}%
}
%    \end{macrocode}
%\end{macro}
%
%\begin{macro}{\DTMresetzones}
% Regional modules should use this before setting their local zones,
% so that users can unset previously defined zones that are outside
% the region if they require. By default this does nothing, so no
% modifications are made.
%    \begin{macrocode}
\newcommand*{\DTMresetzones}{}
%    \end{macrocode}
%\end{macro}
%
%\begin{macro}{\DTMNatoZoneMaps}
% Provide a command to set the time zone abbreviations to the
% military\slash NATO style.
%    \begin{macrocode}
\newcommand*{\DTMNatoZoneMaps}{%
  \defzonemap{01}{00}{A}% Alpha time zone
  \defzonemap{02}{00}{B}% Bravo time zone
  \defzonemap{03}{00}{C}% Charlie time zone
  \defzonemap{04}{00}{D}% Delta time zone
  \defzonemap{05}{00}{E}% Echo time zone
  \defzonemap{06}{00}{F}% Foxtrot time zone
  \defzonemap{07}{00}{G}% Golf time zone
  \defzonemap{08}{00}{H}% Hotel time zone
  \defzonemap{09}{00}{I}% India time zone
  \defzonemap{10}{00}{K}% Kilo time zone
  \defzonemap{11}{00}{L}% Lima time zone
  \defzonemap{12}{00}{M}% Mike time zone
  \defzonemap{-01}{00}{N}% November time zone
  \defzonemap{-02}{00}{O}% Oscar time zone
  \defzonemap{-03}{00}{P}% Papa time zone
  \defzonemap{-04}{00}{Q}% Quebec time zone
  \defzonemap{-05}{00}{R}% Romeo time zone
  \defzonemap{-06}{00}{S}% Sierra time zone
  \defzonemap{-07}{00}{T}% Tango time zone
  \defzonemap{-08}{00}{U}% Uniform time zone
  \defzonemap{-09}{00}{V}% Victor time zone
  \defzonemap{-10}{00}{W}% Whiskey time zone
  \defzonemap{-11}{00}{X}% X-ray time zone
  \defzonemap{-12}{00}{Y}% Yankee time zone
  \defzonemap{00}{00}{Z}% Zulu time zone
}
%    \end{macrocode}
%\end{macro}
%
%\begin{macro}{\DTMnewstyle}
%\begin{definition}
%\cs{DTMnewstyle}\marg{label}\marg{date style definition}\marg{time
%style definition}\marg{zone style definition}\marg{full format
%definition}
%\end{definition}
% Define a new style. The full format redefines \cs{DTMdisplay}
% and \cs{DTMDisplay}.
%    \begin{macrocode}
\newcommand*{\DTMnewstyle}[5]{%
  \ifcsdef{@dtm@style@#1}%
  {%
    \PackageError{datetime2}{Style `#1' already exists}{}%
  }%
  {%
    \DTMnewdatestyle{#1}{#2}%
    \DTMnewtimestyle{#1}{#3}%
    \DTMnewzonestyle{#1}{#4}%
     \csdef{@dtm@style@#1}{%
        \csuse{@dtm@datestyle@#1}%
        \csuse{@dtm@timestyle@#1}%
        \csuse{@dtm@zonestyle@#1}%
        #5%
     }%
  }%
}
%    \end{macrocode}
%\end{macro}
%
%\begin{macro}{\DTMrenewstyle}
%\begin{definition}
%\cs{DTMrenewstyle}\marg{label}\marg{date style definition}\marg{time
%style definition}\marg{zone style definition}\marg{full format
%definition}
%\end{definition}
% Redefine a style. The full format redefines \cs{DTMdisplay}
% and \cs{DTMDisplay}.
%\changes{1.2}{2015-11-10}{new}
%    \begin{macrocode}
\newcommand*{\DTMrenewstyle}[5]{%
  \ifcsundef{@dtm@style@#1}%
  {%
    \PackageError{datetime2}{Style `#1' doesn't exist}{}%
  }%
  {%
    \DTMrenewdatestyle{#1}{#2}%
    \DTMrenewtimestyle{#1}{#3}%
    \DTMrenewzonestyle{#1}{#4}%
     \csdef{@dtm@style@#1}{%
        \csuse{@dtm@datestyle@#1}%
        \csuse{@dtm@timestyle@#1}%
        \csuse{@dtm@zonestyle@#1}%
        #5%
     }%
  }%
}
%    \end{macrocode}
%\end{macro}
%
%\begin{macro}{\DTMprovidestyle}
%\begin{definition}
%\cs{DTMprovidestyle}\marg{label}\marg{date style definition}\marg{time
%style definition}\marg{zone style definition}\marg{full format
%definition}
%\end{definition}
% Defines a full style if it doesn't already exist.
%\changes{1.2}{2015-11-10}{new}
%    \begin{macrocode}
\newcommand*{\DTMprovidestyle}[5]{%
  \ifcsdef{@dtm@style@#1}%
  {%
  }%
  {%
    \DTMprovidedatestyle{#1}{#2}%
    \DTMprovidetimestyle{#1}{#3}%
    \DTMprovidezonestyle{#1}{#4}%
     \csdef{@dtm@style@#1}{%
        \csuse{@dtm@datestyle@#1}%
        \csuse{@dtm@timestyle@#1}%
        \csuse{@dtm@zonestyle@#1}%
        #5%
     }%
  }%
}
%    \end{macrocode}
%\end{macro}
%
%\begin{macro}{\DTMsetdatestyle}
%    \begin{macrocode}
\newrobustcmd*{\DTMsetdatestyle}[1]{%
  \ifcsdef{@dtm@datestyle@#1}%
  {\csuse{@dtm@datestyle@#1}}%
  {%
    \PackageError{datetime2}{Date style `#1' not defined}{}%
  }%
}
%    \end{macrocode}
%\end{macro}
%
%\begin{macro}{\DTMsettimestyle}
%    \begin{macrocode}
\newrobustcmd*{\DTMsettimestyle}[1]{%
  \ifcsdef{@dtm@timestyle@#1}%
  {\csuse{@dtm@timestyle@#1}}%
  {%
    \PackageError{datetime2}{Time style `#1' not defined}{}%
  }%
}
%    \end{macrocode}
%\end{macro}
%
%\begin{macro}{\DTMsetzonestyle}
%    \begin{macrocode}
\newrobustcmd*{\DTMsetzonestyle}[1]{%
  \ifcsdef{@dtm@zonestyle@#1}%
  {\csuse{@dtm@zonestyle@#1}}%
  {%
    \PackageError{datetime2}{Zone style `#1' not defined}{}%
  }%
}
%    \end{macrocode}
%\end{macro}
%
%\begin{macro}{\DTMsetstyle}
%    \begin{macrocode}
\newrobustcmd*{\DTMsetstyle}[1]{%
  \ifcsdef{@dtm@style@#1}%
  {\csuse{@dtm@style@#1}}%
  {%
     \let\dtm@unknownstyle\@dtm@unknownstyle
     \ifcsdef{@dtm@datestyle#1}%
       {\csuse{@dtm@datestyle@#1}\let\dtm@unknownstyle\@dtm@unknown@style}%
       {\@dtm@warning{No date style `#1' defined}}%
     \ifcsdef{@dtm@timestyle#1}%
       {\csuse{@dtm@timestyle@#1}\let\dtm@unknownstyle\@dtm@unknown@style}%
       {\@dtm@warning{No time style `#1' defined}}%
     \ifcsdef{@dtm@zonestyle#1}%
       {\csuse{@dtm@zonestyle@#1}\let\dtm@unknownstyle\@dtm@unknown@style}%
       {\@dtm@warning{No zone style `#1' defined}}%
     \dtm@unknownstyle{#1}%
  }%
}
%    \end{macrocode}
%\end{macro}
%
%\begin{macro}{\@dtm@unknownstyle}
%    \begin{macrocode}
\newcommand*{\@dtm@unknownstyle}[1]{%
  \PackageError{datetime2}{Unknown style `#1'}{}%
}
%    \end{macrocode}
%\end{macro}
%
%\begin{macro}{\@dtm@unknown@style}
%    \begin{macrocode}
\newcommand*{\@dtm@unknown@style}[1]{%
  \@dtm@warning{No full style `#1' defined}{}%
}
%    \end{macrocode}
%\end{macro}
%
% Define \style{default} style:
%    \begin{macrocode}
\DTMnewstyle
 {default}%label
 {% date style
    \renewcommand*\DTMdisplaydate[4]{%
      \number##1\DTMsep{yearmonth}\DTMtwodigits{##2}%
      \DTMsep{monthday}\DTMtwodigits{##3}%
    }%
    \renewcommand*{\DTMDisplaydate}{\DTMdisplaydate}%
 }%
 {% time style
    \renewcommand*\DTMdisplaytime[3]{%
      \DTMtwodigits{##1}\DTMsep{hourmin}\DTMtwodigits{##2}%
      \ifDTMshowseconds\DTMsep{minsec}\DTMtwodigits{##3}\fi
    }%
 }%
 {% zone style
   \renewcommand*{\DTMdisplayzone}[2]{%
     \ifboolexpe
     { bool{DTMshowisoZ}
       and test{\ifnumequal{##1}{0}}
       and test{\ifnumequal{##2}{0}}
     }%
     {%
       Z%
     }%
     {%
      \ifnum##1<0\else+\fi\DTMtwodigits{##1}%
      \ifDTMshowzoneminutes\DTMsep{hourmin}\DTMtwodigits{##2}\fi
     }%
   }%
 }%
 {% full style
   \renewcommand*{\DTMdisplay}[9]{%
    \ifDTMshowdate
     \DTMdisplaydate{##1}{##2}{##3}{##4}%
     \DTMsep{datetime}%
    \fi
    \DTMdisplaytime
     {##5}%
     {##6}%
     {##7}%
    \ifDTMshowzone
     \DTMsep{timezone}%
     \DTMdisplayzone
      {##8}%
      {##9}%
    \fi
   }%
   \renewcommand*{\DTMDisplay}{\DTMdisplay}%
 }
%    \end{macrocode}
%
% Define \style{iso} style which ignores the separator settings:
%    \begin{macrocode}
\DTMnewstyle
 {iso}%label
 {% date style
    \renewcommand*\DTMdisplaydate[4]{%
      \number##1-\DTMtwodigits{##2}-\DTMtwodigits{##3}%
    }%
    \renewcommand*{\DTMDisplaydate}{\DTMdisplaydate}%
 }%
 {% time style
    \renewcommand*\DTMdisplaytime[3]{%
      \DTMtwodigits{##1}:\DTMtwodigits{##2}%
      \ifDTMshowseconds:\DTMtwodigits{##3}\fi
    }%
 }%
 {% zone style
   \renewcommand*{\DTMdisplayzone}[2]{%
     \ifboolexpe
     { bool{DTMshowisoZ}
       and test{\ifnumequal{##1}{0}}
       and test{\ifnumequal{##2}{0}}
     }%
     {%
       Z%
     }%
     {%
      \ifnum##1<0\else+\fi\DTMtwodigits{##1}%
      \ifDTMshowzoneminutes:\DTMtwodigits{##2}\fi
     }%
   }%
 }%
 {% full style
   \renewcommand*{\DTMdisplay}[9]{%
    \ifDTMshowdate
     \DTMdisplaydate{##1}{##2}{##3}{##4}%
     T%
    \fi
    \DTMdisplaytime
     {##5}%
     {##6}%
     {##7}%
    \ifDTMshowzone
     \DTMdisplayzone
      {##8}%
      {##9}%
    \fi
   }%
   \renewcommand*{\DTMDisplay}{\DTMdisplay}%
 }
%    \end{macrocode}
%
% Define \style{pdf} style which converts into a format that can be
% used in \cs{pdfinfo}:
%    \begin{macrocode}
\DTMnewstyle
 {pdf}%label
 {% date style
    \renewcommand*\DTMdisplaydate[4]{%
      D:\number##1 % space intended
      \DTMtwodigits{##2}\DTMtwodigits{##3}%
    }%
    \renewcommand*{\DTMDisplaydate}{\DTMdisplaydate}%
 }%
 {% time style
    \renewcommand*\DTMdisplaytime[3]{%
      \DTMtwodigits{##1}\DTMtwodigits{##2}\DTMtwodigits{##3}%
    }%
 }%
 {% zone style
   \renewcommand*{\DTMdisplayzone}[2]{%
     \ifboolexpe
     { bool{DTMshowisoZ}
       and test{\ifnumequal{##1}{0}}
       and test{\ifnumequal{##2}{0}}
     }%
     {%
       Z%
     }%
     {%
      \ifnum##1<0\else+\fi\DTMtwodigits{##1}'\DTMtwodigits{##2}'%
     }%
   }%
 }%
 {% full style
   \renewcommand*{\DTMdisplay}[9]{%
    \DTMdisplaydate{##1}{##2}{##3}{##4}%
    \DTMdisplaytime{##5}{##6}{##7}%
    \DTMdisplayzone{##8}{##9}%
   }%
   \renewcommand*{\DTMDisplay}{\DTMdisplay}%
 }
%    \end{macrocode}
%
% Define \style{yyyymd} style:
%    \begin{macrocode}
\DTMnewstyle
 {yyyymd}%label
 {% date style
    \renewcommand*\DTMdisplaydate[4]{%
      \number##1
      \DTMsep{yearmonth}%
      \number##2
      \DTMsep{monthday}%
      \number##3
    }%
    \renewcommand*{\DTMDisplaydate}{\DTMdisplaydate}%
 }%
 {% time style
    \renewcommand*\DTMdisplaytime[3]{%
      \DTMtwodigits{##1}\DTMsep{hourmin}\DTMtwodigits{##2}%
      \ifDTMshowseconds\DTMsep{minsec}\DTMtwodigits{##3}\fi
    }%
 }%
 {% zone style
   \renewcommand*{\DTMdisplayzone}[2]{%
     \ifboolexpe
     { bool{DTMshowisoZ}
       and test{\ifnumequal{##1}{0}}
       and test{\ifnumequal{##2}{0}}
     }%
     {%
       Z%
     }%
     {%
      \ifnum##1<0\else+\fi\DTMtwodigits{##1}%
      \ifDTMshowzoneminutes\DTMsep{hourmin}\DTMtwodigits{##2}\fi
     }%
   }%
 }%
 {% full style
   \renewcommand*{\DTMdisplay}[9]{%
    \ifDTMshowdate
     \DTMdisplaydate{##1}{##2}{##3}{##4}%
     \DTMsep{datetime}%
    \fi
    \DTMdisplaytime
     {##5}%
     {##6}%
     {##7}%
    \ifDTMshowzone
     \DTMsep{timezone}%
     \DTMdisplayzone
      {##8}%
      {##9}%
    \fi
   }%
   \renewcommand*{\DTMDisplay}{\DTMdisplay}%
 }
%    \end{macrocode}
%
% Define \style{ddmmyyyy} style:
%    \begin{macrocode}
\DTMnewstyle
 {ddmmyyyy}%label
 {% date style
    \renewcommand*\DTMdisplaydate[4]{%
      \DTMtwodigits{##3}\DTMsep{monthday}%
      \DTMtwodigits{##2}\DTMsep{yearmonth}%
      \number##1
    }%
    \renewcommand*{\DTMDisplaydate}{\DTMdisplaydate}%
 }%
 {% time style
    \renewcommand*\DTMdisplaytime[3]{%
      \DTMtwodigits{##1}\DTMsep{hourmin}\DTMtwodigits{##2}%
      \ifDTMshowseconds\DTMsep{minsec}\DTMtwodigits{##3}\fi
    }%
 }%
 {% zone style
   \renewcommand*{\DTMdisplayzone}[2]{%
     \ifboolexpe
     { bool{DTMshowisoZ}
       and test{\ifnumequal{##1}{0}}
       and test{\ifnumequal{##2}{0}}
     }%
     {%
       Z%
     }%
     {%
      \ifnum##1<0\else+\fi\DTMtwodigits{##1}%
      \ifDTMshowzoneminutes\DTMsep{hourmin}\DTMtwodigits{##2}\fi
     }%
   }%
 }%
 {% full style
   \renewcommand*{\DTMdisplay}[9]{%
    \ifDTMshowdate
     \DTMdisplaydate{##1}{##2}{##3}{##4}%
     \DTMsep{datetime}%
    \fi
    \DTMdisplaytime
     {##5}%
     {##6}%
     {##7}%
    \ifDTMshowzone
     \DTMsep{timezone}%
     \DTMdisplayzone
      {##8}%
      {##9}%
    \fi
   }%
   \renewcommand*{\DTMDisplay}{\DTMdisplay}%
 }
%    \end{macrocode}
%
% Define \style{dmyyyy} style:
%    \begin{macrocode}
\DTMnewstyle
 {dmyyyy}%label
 {% date style
    \renewcommand*\DTMdisplaydate[4]{%
      \number##3
      \DTMsep{monthday}%
      \number##2
      \DTMsep{yearmonth}%
      \number##1
    }%
    \renewcommand*{\DTMDisplaydate}{\DTMdisplaydate}%
 }%
 {% time style
    \renewcommand*\DTMdisplaytime[3]{%
      \DTMtwodigits{##1}\DTMsep{hourmin}\DTMtwodigits{##2}%
      \ifDTMshowseconds\DTMsep{minsec}\DTMtwodigits{##3}\fi
    }%
 }%
 {% zone style
   \renewcommand*{\DTMdisplayzone}[2]{%
     \ifboolexpe
     { bool{DTMshowisoZ}
       and test{\ifnumequal{##1}{0}}
       and test{\ifnumequal{##2}{0}}
     }%
     {%
       Z%
     }%
     {%
      \ifnum##1<0\else+\fi\DTMtwodigits{##1}%
      \ifDTMshowzoneminutes\DTMsep{hourmin}\DTMtwodigits{##2}\fi
     }%
   }%
 }%
 {% full style
   \renewcommand*{\DTMdisplay}[9]{%
    \ifDTMshowdate
     \DTMdisplaydate{##1}{##2}{##3}{##4}%
     \DTMsep{datetime}%
    \fi
    \DTMdisplaytime
     {##5}%
     {##6}%
     {##7}%
    \ifDTMshowzone
     \DTMsep{timezone}%
     \DTMdisplayzone
      {##8}%
      {##9}%
    \fi
   }%
   \renewcommand*{\DTMDisplay}{\DTMdisplay}%
 }
%    \end{macrocode}
%
% Define \style{dmyy} style:
%    \begin{macrocode}
\DTMnewstyle
 {dmyy}%label
 {% date style
    \renewcommand*\DTMdisplaydate[4]{%
      \number##3 % space intended
      \DTMsep{monthday}%
      \number##2 % space intended
      \DTMsep{yearmonth}%
      \DTMtwodigits{##1}%
    }%
    \renewcommand*{\DTMDisplaydate}{\DTMdisplaydate}%
 }%
 {% time style
    \renewcommand*\DTMdisplaytime[3]{%
      \DTMtwodigits{##1}\DTMsep{hourmin}\DTMtwodigits{##2}%
      \ifDTMshowseconds\DTMsep{minsec}\DTMtwodigits{##3}\fi
    }%
 }%
 {% zone style
   \renewcommand*{\DTMdisplayzone}[2]{%
     \ifboolexpe
     { bool{DTMshowisoZ}
       and test{\ifnumequal{##1}{0}}
       and test{\ifnumequal{##2}{0}}
     }%
     {%
       Z%
     }%
     {%
      \ifnum##1<0\else+\fi\DTMtwodigits{##1}%
      \ifDTMshowzoneminutes\DTMsep{hourmin}\DTMtwodigits{##2}\fi
     }%
   }%
 }%
 {% full style
   \renewcommand*{\DTMdisplay}[9]{%
    \ifDTMshowdate
     \DTMdisplaydate{##1}{##2}{##3}{##4}%
     \DTMsep{datetime}%
    \fi
    \DTMdisplaytime
     {##5}%
     {##6}%
     {##7}%
    \ifDTMshowzone
     \DTMsep{timezone}%
     \DTMdisplayzone
      {##8}%
      {##9}%
    \fi
   }%
   \renewcommand*{\DTMDisplay}{\DTMdisplay}%
 }
%    \end{macrocode}
%
% Define \style{ddmmyy} style:
%\changes{1.3}{2016-01-22}{added ddmmyy style}
%    \begin{macrocode}
\DTMnewstyle
 {ddmmyy}%label
 {% date style
    \renewcommand*\DTMdisplaydate[4]{%
      \DTMtwodigits{##3}\DTMsep{monthday}%
      \DTMtwodigits{##2}\DTMsep{yearmonth}%
      \DTMtwodigits{##1}%
    }%
    \renewcommand*{\DTMDisplaydate}{\DTMdisplaydate}%
 }%
 {% time style
    \renewcommand*\DTMdisplaytime[3]{%
      \DTMtwodigits{##1}\DTMsep{hourmin}\DTMtwodigits{##2}%
      \ifDTMshowseconds\DTMsep{minsec}\DTMtwodigits{##3}\fi
    }%
 }%
 {% zone style
   \renewcommand*{\DTMdisplayzone}[2]{%
     \ifboolexpe
     { bool{DTMshowisoZ}
       and test{\ifnumequal{##1}{0}}
       and test{\ifnumequal{##2}{0}}
     }%
     {%
       Z%
     }%
     {%
      \ifnum##1<0\else+\fi\DTMtwodigits{##1}%
      \ifDTMshowzoneminutes\DTMsep{hourmin}\DTMtwodigits{##2}\fi
     }%
   }%
 }%
 {% full style
   \renewcommand*{\DTMdisplay}[9]{%
    \ifDTMshowdate
     \DTMdisplaydate{##1}{##2}{##3}{##4}%
     \DTMsep{datetime}%
    \fi
    \DTMdisplaytime
     {##5}%
     {##6}%
     {##7}%
    \ifDTMshowzone
     \DTMsep{timezone}%
     \DTMdisplayzone
      {##8}%
      {##9}%
    \fi
   }%
   \renewcommand*{\DTMDisplay}{\DTMdisplay}%
 }
%    \end{macrocode}
%
% Define \style{mmddyyyy} style:
%    \begin{macrocode}
\DTMnewstyle
 {mmddyyyy}%label
 {% date style
    \renewcommand*\DTMdisplaydate[4]{%
      \DTMtwodigits{##2}\DTMsep{monthday}%
      \DTMtwodigits{##3}\DTMsep{dayyear}%
      \number##1
    }%
    \renewcommand*{\DTMDisplaydate}{\DTMdisplaydate}%
 }%
 {% time style
    \renewcommand*\DTMdisplaytime[3]{%
      \DTMtwodigits{##1}\DTMsep{hourmin}\DTMtwodigits{##2}%
      \ifDTMshowseconds\DTMsep{minsec}\DTMtwodigits{##3}\fi
    }%
 }%
 {% zone style
   \renewcommand*{\DTMdisplayzone}[2]{%
     \ifboolexpe
     { bool{DTMshowisoZ}
       and test{\ifnumequal{##1}{0}}
       and test{\ifnumequal{##2}{0}}
     }%
     {%
       Z%
     }%
     {%
      \ifnum##1<0\else+\fi\DTMtwodigits{##1}%
      \ifDTMshowzoneminutes\DTMsep{hourmin}\DTMtwodigits{##2}\fi
     }%
   }%
 }%
 {% full style
   \renewcommand*{\DTMdisplay}[9]{%
    \ifDTMshowdate
     \DTMdisplaydate{##1}{##2}{##3}{##4}%
     \DTMsep{datetime}%
    \fi
    \DTMdisplaytime
     {##5}%
     {##6}%
     {##7}%
    \ifDTMshowzone
     \DTMsep{timezone}%
     \DTMdisplayzone
      {##8}%
      {##9}%
    \fi
   }%
   \renewcommand*{\DTMDisplay}{\DTMdisplay}%
 }
%    \end{macrocode}
%
% Define \style{mdyyyy} style:
%    \begin{macrocode}
\DTMnewstyle
 {mdyyyy}%label
 {% date style
    \renewcommand*\DTMdisplaydate[4]{%
      \number##2 % space intended
      \DTMsep{monthday}%
      \number##3 % space intended
      \DTMsep{dayyear}%
      \number##1 % space intended
    }%
    \renewcommand*{\DTMDisplaydate}{\DTMdisplaydate}%
 }%
 {% time style
    \renewcommand*\DTMdisplaytime[3]{%
      \DTMtwodigits{##1}\DTMsep{hourmin}\DTMtwodigits{##2}%
      \ifDTMshowseconds\DTMsep{minsec}\DTMtwodigits{##3}\fi
    }%
 }%
 {% zone style
   \renewcommand*{\DTMdisplayzone}[2]{%
     \ifboolexpe
     { bool{DTMshowisoZ}
       and test{\ifnumequal{##1}{0}}
       and test{\ifnumequal{##2}{0}}
     }%
     {%
       Z%
     }%
     {%
      \ifnum##1<0\else+\fi\DTMtwodigits{##1}%
      \ifDTMshowzoneminutes\DTMsep{hourmin}\DTMtwodigits{##2}\fi
     }%
   }%
 }%
 {% full style
   \renewcommand*{\DTMdisplay}[9]{%
    \ifDTMshowdate
     \DTMdisplaydate{##1}{##2}{##3}{##4}%
     \DTMsep{datetime}%
    \fi
    \DTMdisplaytime
     {##5}%
     {##6}%
     {##7}%
    \ifDTMshowzone
     \DTMsep{timezone}%
     \DTMdisplayzone
      {##8}%
      {##9}%
    \fi
   }%
   \renewcommand*{\DTMDisplay}{\DTMdisplay}%
 }
%    \end{macrocode}
%
% Define \style{mdyy} style:
%    \begin{macrocode}
\DTMnewstyle
 {mdyy}%label
 {% date style
    \renewcommand*\DTMdisplaydate[4]{%
      \number##2 % space intended
      \DTMsep{monthday}%
      \number##3 % space intended
      \DTMsep{dayyear}%
      \DTMtwodigits{##1}%
    }%
    \renewcommand*{\DTMDisplaydate}{\DTMdisplaydate}%
 }%
 {% time style
    \renewcommand*\DTMdisplaytime[3]{%
      \DTMtwodigits{##1}\DTMsep{hourmin}\DTMtwodigits{##2}%
      \ifDTMshowseconds\DTMsep{minsec}\DTMtwodigits{##3}\fi
    }%
 }%
 {% zone style
   \renewcommand*{\DTMdisplayzone}[2]{%
     \ifboolexpe
     { bool{DTMshowisoZ}
       and test{\ifnumequal{##1}{0}}
       and test{\ifnumequal{##2}{0}}
     }%
     {%
       Z%
     }%
     {%
      \ifnum##1<0\else+\fi\DTMtwodigits{##1}%
      \ifDTMshowzoneminutes\DTMsep{hourmin}\DTMtwodigits{##2}\fi
     }%
   }%
 }%
 {% full style
   \renewcommand*{\DTMdisplay}[9]{%
    \ifDTMshowdate
     \DTMdisplaydate{##1}{##2}{##3}{##4}%
     \DTMsep{datetime}%
    \fi
    \DTMdisplaytime
     {##5}%
     {##6}%
     {##7}%
    \ifDTMshowzone
     \DTMsep{timezone}%
     \DTMdisplayzone
      {##8}%
      {##9}%
    \fi
   }%
   \renewcommand*{\DTMDisplay}{\DTMdisplay}%
 }
%    \end{macrocode}
%
% Define \style{mmddyy} style:
%\changes{1.3}{2016-01-22}{added mmddyy style}
%    \begin{macrocode}
\DTMnewstyle
 {mmddyy}%label
 {% date style
    \renewcommand*\DTMdisplaydate[4]{%
      \DTMtwodigits{##2}\DTMsep{monthday}%
      \DTMtwodigits{##3}\DTMsep{dayyear}%
      \DTMtwodigits{##1}%
    }%
    \renewcommand*{\DTMDisplaydate}{\DTMdisplaydate}%
 }%
 {% time style
    \renewcommand*\DTMdisplaytime[3]{%
      \DTMtwodigits{##1}\DTMsep{hourmin}\DTMtwodigits{##2}%
      \ifDTMshowseconds\DTMsep{minsec}\DTMtwodigits{##3}\fi
    }%
 }%
 {% zone style
   \renewcommand*{\DTMdisplayzone}[2]{%
     \ifboolexpe
     { bool{DTMshowisoZ}
       and test{\ifnumequal{##1}{0}}
       and test{\ifnumequal{##2}{0}}
     }%
     {%
       Z%
     }%
     {%
      \ifnum##1<0\else+\fi\DTMtwodigits{##1}%
      \ifDTMshowzoneminutes\DTMsep{hourmin}\DTMtwodigits{##2}\fi
     }%
   }%
 }%
 {% full style
   \renewcommand*{\DTMdisplay}[9]{%
    \ifDTMshowdate
     \DTMdisplaydate{##1}{##2}{##3}{##4}%
     \DTMsep{datetime}%
    \fi
    \DTMdisplaytime
     {##5}%
     {##6}%
     {##7}%
    \ifDTMshowzone
     \DTMsep{timezone}%
     \DTMdisplayzone
      {##8}%
      {##9}%
    \fi
   }%
   \renewcommand*{\DTMDisplay}{\DTMdisplay}%
 }
%    \end{macrocode}
%
% Define \style{hmmss} time style
%    \begin{macrocode}
\DTMnewtimestyle
 {hmmss}% label
 {% 
    \renewcommand*\DTMdisplaytime[3]{%
      \number##1
      \DTMsep{hourmin}\DTMtwodigits{##2}%
      \ifDTMshowseconds\DTMsep{minsec}\DTMtwodigits{##3}\fi
    }%
 }%
%    \end{macrocode}
%
% Define \style{map} zone style
%    \begin{macrocode}
\DTMnewzonestyle
 {map}% label
 {% 
    \renewcommand*\DTMdisplaytime[3]{%
      \DTMusezonemapordefault{##1}{##2}%
    }%
 }%
%    \end{macrocode}
%
% Define \style{hhmm} zone style
%    \begin{macrocode}
\DTMnewzonestyle
 {hhmm}% label
 {%
    \renewcommand*\DTMdisplaytime[3]{%
      \ifnum##1<0\else+\fi\DTMtwodigits{##1}%
      \ifDTMshowzoneminutes\DTMsep{hourmin}\DTMtwodigits{##2}\fi
    }%
 }
%    \end{macrocode}
%
%\subsection{Saving and Using Dates}
% Date and time information is stored in control sequences in the
% form \cs{@dtm@}\meta{label}\texttt{@}\meta{tag}, where
% \meta{label} is the label uniquely identifying the information and
% \meta{tag} is the element (\texttt{year}, \texttt{month},
% \texttt{day}, \texttt{dow}, \texttt{hour}, \texttt{minute},
% \texttt{second}, \texttt{TZhour} and \texttt{TZminute}). Missing
% information is stored as \texttt{0} (except for the day of week, which is
% stored as \texttt{-1}).
%
%\begin{macro}{\DTMsavedate}
% Save the date specified in the format
% \meta{yyyy}-\meta{mm}-\meta{dd}. \cs{expandafter} is used in case
% the argument is a control sequence storing the date. This will
% redefine an existing saved date with the same label. The first
% argument is the label.
%    \begin{macrocode}
\newrobustcmd*{\DTMsavedate}[2]{%
  \expandafter\@dtm@parsedate#2\@dtm@endparsedate
  \cslet{@dtm@#1@year}{\@dtm@year}%
  \cslet{@dtm@#1@month}{\@dtm@month}%
  \cslet{@dtm@#1@day}{\@dtm@day}%
  \cslet{@dtm@#1@dow}{\@dtm@dow}%
  \ifcsundef{@dtm@#1@hour}{\csdef{@dtm@#1@hour}{0}}{}%
  \ifcsundef{@dtm@#1@minute}{\csdef{@dtm@#1@minute}{0}}{}%
  \ifcsundef{@dtm@#1@second}{\csdef{@dtm@#1@second}{0}}{}%
  \ifcsundef{@dtm@#1@TZhour}{\csdef{@dtm@#1@TZhour}{0}}{}%
  \ifcsundef{@dtm@#1@TZminute}{\csdef{@dtm@#1@TZminute}{0}}{}%
}
%    \end{macrocode}
%\end{macro}
%
%\begin{macro}{\DTMsavenoparsedate}
% Save the date without parsing the
% \meta{YYYY}-\meta{MM}-\meta{DD} format.
%    \begin{macrocode}
\newrobustcmd*{\DTMsavenoparsedate}[5]{%
  \csedef{@dtm@#1@year}{\number#2}%
  \csedef{@dtm@#1@month}{\number#3}%
  \csedef{@dtm@#1@day}{\number#4}%
  \csedef{@dtm@#1@dow}{\number#5}%
  \ifcsundef{@dtm@#1@hour}{\csdef{@dtm@#1@hour}{0}}{}%
  \ifcsundef{@dtm@#1@minute}{\csdef{@dtm@#1@minute}{0}}{}%
  \ifcsundef{@dtm@#1@second}{\csdef{@dtm@#1@second}{0}}{}%
  \ifcsundef{@dtm@#1@TZhour}{\csdef{@dtm@#1@TZhour}{0}}{}%
  \ifcsundef{@dtm@#1@TZminute}{\csdef{@dtm@#1@TZminute}{0}}{}%
}
%    \end{macrocode}
%\end{macro}
%\begin{macro}{\DTMsavetime}
% Save the time specified in the format
% \meta{hh}:\meta{mm}:\meta{ss}. \cs{expandafter} is used in case
% the argument is a control sequence storing the date. This will
% redefine an existing saved date with the same label. The first
% argument is the label.
%    \begin{macrocode}
\newrobustcmd*{\DTMsavetime}[2]{%
  \expandafter\@dtm@parsetime#2\@dtm@endparsetime
  \cslet{@dtm@#1@hour}{\@dtm@hour}%
  \cslet{@dtm@#1@minute}{\@dtm@minute}%
  \cslet{@dtm@#1@second}{\@dtm@second}%
  \ifcsundef{@dtm@#1@year}{\csdef{@dtm@#1@year}{0}}{}%
  \ifcsundef{@dtm@#1@month}{\csdef{@dtm@#1@month}{0}}{}%
  \ifcsundef{@dtm@#1@day}{\csdef{@dtm@#1@day}{0}}{}%
  \ifcsundef{@dtm@#1@dow}{\csdef{@dtm@#1@dow}{-1}}{}%
  \ifcsundef{@dtm@#1@TZhour}{\csdef{@dtm@#1@TZhour}{0}}{}%
  \ifcsundef{@dtm@#1@TZminute}{\csdef{@dtm@#1@TZminute}{0}}{}%
}
%    \end{macrocode}
%\end{macro}
%
%\begin{macro}{\DTMsavetimezn}
% Save the time (including zone) specified in the format
% \meta{hh}:\meta{mm}:\meta{ss} \meta{tzh}:\meta{tzm}.
% \cs{expandafter} is used in case
% the argument is a control sequence storing the date. This will
% redefine an existing saved date with the same label. The first
% argument is the label.
%    \begin{macrocode}
\newrobustcmd*{\DTMsavetimezn}[2]{%
  \expandafter\@dtm@parsetimezn#2\@dtm@endparsetimezn
  \cslet{@dtm@#1@hour}{\@dtm@hour}%
  \cslet{@dtm@#1@minute}{\@dtm@minute}%
  \cslet{@dtm@#1@second}{\@dtm@second}%
  \cslet{@dtm@#1@TZhour}{\@dtm@timezonehour}%
  \cslet{@dtm@#1@TZminute}{\@dtm@timezoneminute}%
  \ifcsundef{@dtm@#1@year}{\csdef{@dtm@#1@year}{0}}{}%
  \ifcsundef{@dtm@#1@month}{\csdef{@dtm@#1@month}{0}}{}%
  \ifcsundef{@dtm@#1@day}{\csdef{@dtm@#1@day}{0}}{}%
  \ifcsundef{@dtm@#1@dow}{\csdef{@dtm@#1@dow}{-1}}{}%
}
%    \end{macrocode}
%\end{macro}
%
%\begin{macro}{\DTMsavetimestamp}
% Save the time (including zone) specified in the format
%\meta{YYYY}-\meta{MM}-\meta{DD}T\meta{hh}:\meta{mm}:\meta{ss}\meta{time
%zone}
%
%The first argument is the label.
%    \begin{macrocode}
\newrobustcmd*{\DTMsavetimestamp}[2]{%
  \expandafter\@dtm@parsetimestamp#2\@dtm@endparsetimestamp
  \cslet{@dtm@#1@year}{\@dtm@year}%
  \cslet{@dtm@#1@month}{\@dtm@month}%
  \cslet{@dtm@#1@day}{\@dtm@day}%
  \cslet{@dtm@#1@dow}{\@dtm@dow}%
  \cslet{@dtm@#1@hour}{\@dtm@hour}%
  \cslet{@dtm@#1@minute}{\@dtm@minute}%
  \cslet{@dtm@#1@second}{\@dtm@second}%
  \cslet{@dtm@#1@TZhour}{\@dtm@timezonehour}%
  \cslet{@dtm@#1@TZminute}{\@dtm@timezoneminute}%
}
%    \end{macrocode}
%\end{macro}
%
%\begin{macro}{\DTMsavenow}
%Save the current time.
%    \begin{macrocode}
\newrobustcmd{\DTMsavenow}[1]{%
  \cslet{@dtm@#1@year}{\@dtm@currentyear}%
  \cslet{@dtm@#1@month}{\@dtm@currentmonth}%
  \cslet{@dtm@#1@day}{\@dtm@currentday}%
  \cslet{@dtm@#1@dow}{\@dtm@currentdow}%
  \cslet{@dtm@#1@hour}{\@dtm@currenthour}%
  \cslet{@dtm@#1@minute}{\@dtm@currentminute}%
  \cslet{@dtm@#1@second}{\@dtm@currentsecond}%
  \cslet{@dtm@#1@TZhour}{\@dtm@currenttimezonehour}%
  \cslet{@dtm@#1@TZminute}{\@dtm@currenttimezoneminute}%
}
%    \end{macrocode}
%\end{macro}
%
%\begin{macro}{\DTMmakeglobal}
%Globally set the stored information.
%    \begin{macrocode}
\newrobustcmd{\DTMmakeglobal}[1]{%
  \global\csletcs{@dtm@#1@year}{@dtm@#1@year}%
  \global\csletcs{@dtm@#1@month}{@dtm@#1@month}%
  \global\csletcs{@dtm@#1@day}{@dtm@#1@day}%
  \global\csletcs{@dtm@#1@dow}{@dtm@#1@dow}%
  \global\csletcs{@dtm@#1@hour}{@dtm@#1@hour}%
  \global\csletcs{@dtm@#1@minute}{@dtm@#1@minute}%
  \global\csletcs{@dtm@#1@second}{@dtm@#1@second}%
  \global\csletcs{@dtm@#1@TZhour}{@dtm@#1@TZhour}%
  \global\csletcs{@dtm@#1@TZminute}{@dtm@#1@TZminute}%
}
%    \end{macrocode}
%\end{macro}
%
% Expandable ways of fetching saved data. (No check for existence
% performed.) The argument is the label.
%\begin{macro}{\DTMfetchyear}
%    \begin{macrocode}
\newcommand*{\DTMfetchyear}[1]{\csname @dtm@#1@year\endcsname}
%    \end{macrocode}
%\end{macro}
%\begin{macro}{\DTMfetchmonth}
%    \begin{macrocode}
\newcommand*{\DTMfetchmonth}[1]{\csname @dtm@#1@month\endcsname}
%    \end{macrocode}
%\end{macro}
%\begin{macro}{\DTMfetchday}
%    \begin{macrocode}
\newcommand*{\DTMfetchday}[1]{\csname @dtm@#1@day\endcsname}
%    \end{macrocode}
%\end{macro}
%\begin{macro}{\DTMfetchdow}
%    \begin{macrocode}
\newcommand*{\DTMfetchdow}[1]{\csname @dtm@#1@dow\endcsname}
%    \end{macrocode}
%\end{macro}
%\begin{macro}{\DTMfetchhour}
%    \begin{macrocode}
\newcommand*{\DTMfetchhour}[1]{\csname @dtm@#1@hour\endcsname}
%    \end{macrocode}
%\end{macro}
%\begin{macro}{\DTMfetchminute}
%    \begin{macrocode}
\newcommand*{\DTMfetchminute}[1]{\csname @dtm@#1@minute\endcsname}
%    \end{macrocode}
%\end{macro}
%\begin{macro}{\DTMfetchsecond}
%    \begin{macrocode}
\newcommand*{\DTMfetchsecond}[1]{\csname @dtm@#1@second\endcsname}
%    \end{macrocode}
%\end{macro}
%\begin{macro}{\DTMfetchTZhour}
%    \begin{macrocode}
\newcommand*{\DTMfetchTZhour}[1]{\csname @dtm@#1@TZhour\endcsname}
%    \end{macrocode}
%\end{macro}
%\begin{macro}{\DTMfetchTZminute}
%    \begin{macrocode}
\newcommand*{\DTMfetchTZminute}[1]{\csname @dtm@#1@TZminute\endcsname}
%    \end{macrocode}
%\end{macro}
%
%\begin{macro}{\DTMusedate}
%\begin{definition}
%\cs{DTMusedate}\marg{label}
%\end{definition}
% Displays the previously saved date using \cs{DTMdisplaydate}.
%    \begin{macrocode}
\newcommand*\DTMusedate[1]{%
  \ifcsundef{@dtm@#1@year}%
  {%
     \PackageError{datetime2}{Undefined date `#1'}{}%
  }%
  {%
     \DTMdisplaydate
      {\csname @dtm@#1@year\endcsname}%
      {\csname @dtm@#1@month\endcsname}%
      {\csname @dtm@#1@day\endcsname}%
      {\csname @dtm@#1@dow\endcsname}%
  }%
}%
%    \end{macrocode}
%\end{macro}
%
%\begin{macro}{\DTMUsedate}
%\begin{definition}
%\cs{DTMUsedate}\marg{label}
%\end{definition}
% Displays the previously saved date using \cs{DTMDisplaydate}.
%    \begin{macrocode}
\newcommand*\DTMUsedate[1]{%
  \ifcsundef{@dtm@#1@year}%
  {%
     \PackageError{datetime2}{Undefined date `#1'}{}%
  }%
  {%
     \DTMDisplaydate
      {\csname @dtm@#1@year\endcsname}%
      {\csname @dtm@#1@month\endcsname}%
      {\csname @dtm@#1@day\endcsname}%
      {\csname @dtm@#1@dow\endcsname}%
  }%
}%
%    \end{macrocode}
%\end{macro}
%
%\begin{macro}{\DTMusetime}
%\begin{definition}
%\cs{DTMusetime}\marg{label}
%\end{definition}
% Displays the previously saved time using \cs{DTMdisplaytime}.
%    \begin{macrocode}
\newcommand*\DTMusetime[1]{%
  \ifcsundef{@dtm@#1@hour}%
  {%
     \PackageError{datetime2}{Undefined time `#1'}{}%
  }%
  {%
     \DTMdisplaytime
      {\csname @dtm@#1@hour\endcsname}%
      {\csname @dtm@#1@minute\endcsname}%
      {\csname @dtm@#1@second\endcsname}%
  }%
}%
%    \end{macrocode}
%\end{macro}
%
%\begin{macro}{\DTMusezone}
%\begin{definition}
%\cs{DTMusezone}\marg{label}
%\end{definition}
% Displays the previously saved date using \cs{DTMdisplayzone}.
%    \begin{macrocode}
\newcommand*\DTMusezone[1]{%
  \ifcsundef{@dtm@#1@TZhour}%
  {%
     \PackageError{datetime2}{Undefined time `#1'}{}%
  }%
  {%
     \DTMdisplayzone
      {\csname @dtm@#1@TZhour\endcsname}%
      {\csname @dtm@#1@TZminute\endcsname}%
  }%
}%
%    \end{macrocode}
%\end{macro}
%
%\begin{macro}{\DTMuse}
%\begin{definition}
%\cs{DTMuse}\marg{label}
%\end{definition}
% Displays the previously saved date and time.
%    \begin{macrocode}
\newcommand*\DTMuse[1]{%
  \ifcsundef{@dtm@#1@year}%
  {%
     \PackageError{datetime2}{Undefined date-time `#1'}{}%
  }%
  {%
     \DTMdisplay
      {\csname @dtm@#1@year\endcsname}%
      {\csname @dtm@#1@month\endcsname}%
      {\csname @dtm@#1@day\endcsname}%
      {\csname @dtm@#1@dow\endcsname}%
      {\csname @dtm@#1@hour\endcsname}%
      {\csname @dtm@#1@minute\endcsname}%
      {\csname @dtm@#1@second\endcsname}%
      {\csname @dtm@#1@TZhour\endcsname}%
      {\csname @dtm@#1@TZminute\endcsname}%
  }%
}%
%    \end{macrocode}
%\end{macro}
%
%\begin{macro}{\DTMUse}
%\begin{definition}
%\cs{DTMUse}\marg{label}
%\end{definition}
% Displays the previously saved date and time.
%    \begin{macrocode}
\newcommand*\DTMUse[1]{%
  \ifcsundef{@dtm@#1@year}%
  {%
     \PackageError{datetime2}{Undefined date-time `#1'}{}%
  }%
  {%
     \DTMDisplay
      {\csname @dtm@#1@year\endcsname}%
      {\csname @dtm@#1@month\endcsname}%
      {\csname @dtm@#1@day\endcsname}%
      {\csname @dtm@#1@dow\endcsname}%
      {\csname @dtm@#1@hour\endcsname}%
      {\csname @dtm@#1@minute\endcsname}%
      {\csname @dtm@#1@second\endcsname}%
      {\csname @dtm@#1@TZhour\endcsname}%
      {\csname @dtm@#1@TZminute\endcsname}%
  }%
}%
%    \end{macrocode}
%\end{macro}
%
%\begin{macro}{\DTMifsaveddate}
% Determine if the given label has been assigned to a date, time and
% zone.
%    \begin{macrocode}
\newcommand{\DTMifsaveddate}[3]{%
  \ifcsundef{@dtm@#1@year}{#3}{#2}%
}
%    \end{macrocode}
%\end{macro}
%
%\subsection{Language Module Loading}
%
% Define commands to load regional settings.
%
%\begin{macro}{\@dtm@requiremodule}
% Use \sty{tracklang} interface to find the associated file for the
% given dialect.
%    \begin{macrocode}
\newcommand*{\@dtm@requiremodule}[1]{%
  \IfTrackedLanguageFileExists{#1}%
  {datetime2-}% prefix
  {.ldf}% suffix
  {%
    \RequireDateTimeModule{\CurrentTrackedTag}%
  }%
  {%
    \@dtm@warning{Date-Time Language Module `#1' not installed}%
  }%
}
%    \end{macrocode}
%\end{macro}
%
%\begin{macro}{\@dtm@loadedregions}
% List of loaded \styfmt{datetime2} language modules.
%    \begin{macrocode}
\newcommand*{\@dtm@loadedregions}{}
%    \end{macrocode}
%\end{macro}
%
%\begin{macro}{\RequireDateTimeModule}
% Input the language file, if not already loaded.
% Should only be used with \cs{@dtm@requiremodule}
% which sets commands like \cs{CurrentTrackedDialect}.
% Since the language modules are loaded within
% \cs{@dtm@requiremodule} they may use this command to load
% dependent modules.
%    \begin{macrocode}
\newcommand*{\RequireDateTimeModule}[1]{%
 \ifundef\CurrentTrackedDialect
 {%
   \PackageError{datetime2}%
   {\string\RequireDateTimeModule\space not permitted here}%
   {This command is only permitted inside datetime2 language
    modules.}%
 }%
 {%
   \ifcsundef{ver@datetime2-#1.ldf}%
   {%
     \input{datetime2-#1.ldf}%
     \ifdefempty\@dtm@loadedregions
     {%
       \edef\@dtm@loadedregions{#1}%
     }%
     {%
       \edef\@dtm@loadedregions{\@dtm@loadedregions,#1}%
     }%
%    \end{macrocode}
% In case a synonym is also used, add a mapping from the module name
% to the current tracked dialect.
%\changes{1.3}{2016-01-22}{added module to dialect map}
%    \begin{macrocode}
     \csedef{@dtm@moddialectmap@#1}{\CurrentTrackedDialect}%
   }%
   {%
%    \end{macrocode}
% The module has already been loaded, but the current tracked
% dialect might be a synonym for a different language label that
% might've already loaded the module. If \cs{date\meta{dialect}}
% exists, this needs to be set.
%    \begin{macrocode}
     \ifcsdef{date\CurrentTrackedDialect}
     {%
       \letcs{\@dtm@otherdialect}{@dtm@moddialectmap@#1}%
       \edef\@dtm@thisdialect{\CurrentTrackedDialect}%
       \ifdefequal\@dtm@thisdialect\@dtm@otherdialect
       {}%
       {%
         \ifcsdef{date\@dtm@otherdialect}%
         {%
           \csletcs{date\@dtm@thisdialect}{date\@dtm@otherdialect}%
         }%
         {}%
       }%
     }%
     {}%
   }%
%    \end{macrocode}
% In case it's needed, create a mapping between the dialect name and
% the module name.
%\changes{1.3}{2016-01-22}{added dialect to module map}
%    \begin{macrocode}
   \csedef{@dtm@dialectmodmap@\CurrentTrackedDialect}{#1}%
 }%
}
%    \end{macrocode}
%\end{macro}
%
%\begin{macro}{\DTMdialecttomodulemap}
%\changes{1.3}{2016-01-22}{new}
%\begin{definition}
%\cs{DTMdialecttomodulemap}\marg{dialect}
%\end{definition}
% Expands to name of the module loaded with the given dialect name
% or \cs{relax} if no module has been loaded for the given dialect.
%    \begin{macrocode}
\newcommand*{\DTMdialecttomodulemap}[1]{%
 \ifcsdef{ver@datetime2-#1.ldf}%
 {#1}%
 {\csname @dtm@dialectmodmap@#1\endcsname}%
}
%    \end{macrocode}
%\end{macro}
%
%\begin{macro}{\ProvidesDateTimeModule}
% For use in language module to identify itself.
%    \begin{macrocode}
\newcommand*{\ProvidesDateTimeModule}[1]{%
  \ProvidesFile{datetime2-#1.ldf}%
}
%    \end{macrocode}
%\end{macro}
%
%\begin{macro}{\DTMusemodule}
%\changes{1.03}{2016-01-20}{new}
% Provided for packages or documents that need to load a module.
% This shouldn't be used inside the \texttt{.ldf} files.
%    \begin{macrocode}
\newcommand*{\DTMusemodule}[2]{%
 \ifcsdef{@tracklang@add@#1}%
 {%
   \TrackPredefinedDialect{#1}%
 }%
 {}%
 \let\@dtm@org@dialect\CurrentTrackedDialect
 \def\CurrentTrackedDialect{#1}%
 \RequireDateTimeModule{#2}%
 \let\CurrentTrackedDialect\@dtm@org@dialect
}
%    \end{macrocode}
%\end{macro}
%
%\begin{macro}{\DTMdefkey}
%\begin{definition}
%\cs{DTMdefkey}\marg{region}\marg{key}\oarg{default}\marg{func}
%\end{definition}
% Used by language modules to define a key.
%    \begin{macrocode}
\newcommand*{\DTMdefkey}[1]{\define@key[dtm]{#1}}
%    \end{macrocode}
%\end{macro}
%
%\begin{macro}{\DTMdefchoicekey}
%\begin{definition}
%\cs{DTMdefchoicekey}\marg{region}\marg{key}\oarg{bin}\marg{choice
%list}\marg{default}\marg{func}
%\end{definition}
% Used by language modules to define a choice key.
%    \begin{macrocode}
\newcommand*{\DTMdefchoicekey}[1]{\define@choicekey[dtm]{#1}}
%    \end{macrocode}
%\end{macro}
%
%\begin{macro}{\DTMdefboolkey}
%\begin{definition}
%\cs{DTMdefboolkey}\marg{region}\oarg{mp}\marg{key}\oarg{default}\marg{func}
%\end{definition}
% Used by language modules to define a boolean key.
%    \begin{macrocode}
\newcommand*{\DTMdefboolkey}[1]{\define@boolkey[dtm]{#1}}
%    \end{macrocode}
%\end{macro}
%
%\begin{macro}{\DTMifbool}
%\begin{definition}
%\cs{DTMifbool}\marg{region}\marg{key}\marg{true}\marg{false}
%\end{definition}
% Test boolean key that was defined using \cs{DTMdefboolkey}
%    \begin{macrocode}
\newcommand*{\DTMifbool}[4]{\ifbool{dtm@#1@#2}{#3}{#4}}
%    \end{macrocode}
%\end{macro}
%
%\begin{macro}{\DTMsetbool}
%\begin{definition}
%\cs{DTMsetbool}\marg{region}\marg{key}\marg{value}
%\end{definition}
% Set boolean key that was defined using \cs{DTMdefboolkey}
%    \begin{macrocode}
\newcommand*{\DTMsetbool}[3]{\setbool{dtm@#1@#2}{#3}}
%    \end{macrocode}
%\end{macro}
%
%\begin{macro}{\DTMlangsetup}
% Set up options for language modules. The optional argument is a
% list of language/regions. If omitted all loaded regions are
% iterated over. (I'm not sure why \cs{setkeys} doesn't work
% if the same key is present in multiple families, so this iterates
% over the families instead.)
% The starred version doesn't warn on unknown keys.
%\changes{1.3}{2016-01-22}{added starred version}
%    \begin{macrocode}
\newcommand*{\DTMlangsetup}{%
  \@ifstar\s@DTMlangsetup\@DTMlangsetup}
%    \end{macrocode}
% Unstarred version:
%    \begin{macrocode}
\newcommand*{\@DTMlangsetup}[2][\@dtm@loadedregions]{%
 \@for\@dtm@region:=#1\do{%
   \setkeys*+[dtm]{\@dtm@region}{#2}%
   \ifdefempty\XKV@rm{}%
   {%
     \@dtm@warning{Region `\@dtm@region' has ignored
      \MessageBreak the following settings:\MessageBreak
      \XKV@rm
      ^^J}%
   }%
 }%
}
%    \end{macrocode}
%
% Starred version:
%    \begin{macrocode}
\newcommand*{\s@DTMlangsetup}[2][\@dtm@loadedregions]{%
 \@for\@dtm@region:=#1\do{%
   \setkeys*+[dtm]{\@dtm@region}{#2}%
 }%
}
%    \end{macrocode}
%\end{macro}
%
% Now load all the required modules (if installed) using the
% \sty{tracklang} interface. (Language packages, such as \sty{babel}
% or \sty{polyglossia} must be loaded before this.)
%    \begin{macrocode}
\AnyTrackedLanguages
{%
  \ForEachTrackedDialect{\this@dialect}%
  {%
    \@dtm@requiremodule\this@dialect
  }%
}
{%
%    \end{macrocode}
% No tracked languages. The default is already set up, so nothing to
% do here.
%    \begin{macrocode}
}
%    \end{macrocode}
%
% Load \sty{datetime2-calc} if required.
%    \begin{macrocode}
\@dtm@usecalc
%    \end{macrocode}
%
% Use the \pkgopt{style} package option, if set.
%    \begin{macrocode}
\ifdefempty\@dtm@initialstyle{}{\DTMsetstyle{\@dtm@initialstyle}}
%    \end{macrocode}
%\iffalse
%    \begin{macrocode}
%</datetime2.sty>
%    \end{macrocode}
%\fi
%\iffalse
%    \begin{macrocode}
%<*datetime2-calc.sty>
%    \end{macrocode}
%\fi
%\section{datetime2-calc.sty code}
%    \begin{macrocode}
\NeedsTeXFormat{LaTeX2e}
\ProvidesPackage{datetime2-calc}[2016/06/05 v1.5.1 (NLCT)]
%    \end{macrocode}
% Load other required packages
%    \begin{macrocode}
\RequirePackage{pgfkeys}
\RequirePackage{pgfcalendar}
%    \end{macrocode}
%
%\subsection{Conversions and Calculations}
%\begin{macro}{\@dtm@julianday}
% Register for storing Julian day number.
%    \begin{macrocode}
\newcount\@dtm@julianday
%    \end{macrocode}
%\end{macro}
%
%\begin{macro}{\@dtm@parsedate}
% Redefine \cs{@dtm@parsedate} so that it uses \sty{pgfcalendar} to
% compute the required information. This allows for offsets, the use
% of \texttt{last} and also determine the day of week.
%    \begin{macrocode}
\def\@dtm@parsedate#1-#2-#3\@dtm@endparsedate{%
  \pgfcalendardatetojulian{#1-#2-#3}{\@dtm@julianday}%
  \pgfcalendarjuliantodate{\@dtm@julianday}{\@dtm@year}{\@dtm@month}{\@dtm@day}%
  \pgfcalendarjuliantoweekday{\@dtm@julianday}{\count@}%
  \edef\@dtm@dow{\number\count@}%
}
%    \end{macrocode}
%\end{macro}
%
% Set the current day of week
%\begin{macro}{\@dtm@currentdow}
%    \begin{macrocode}
\pgfcalendardatetojulian
 {\@dtm@currentyear-\@dtm@currentmonth-\@dtm@currentday}%
 {\@dtm@julianday}%
\pgfcalendarjuliantoweekday{\@dtm@julianday}{\count@}%
\edef\@dtm@currentdow{\number\count@}%
%    \end{macrocode}
%\end{macro}
%
%\begin{macro}{\DTMsavejulianday}
% Save the date obtained from the Julian day number.
%    \begin{macrocode}
\newrobustcmd*{\DTMsavejulianday}[2]{%
  \pgfcalendarjuliantodate{#2}{\@dtm@year}{\@dtm@month}{\@dtm@day}%
  \pgfcalendarjuliantoweekday{#2}{\count@}%
  \csedef{@dtm@#1@dow}{\number\count@}%
  \cslet{@dtm@#1@year}{\@dtm@year}%
  \cslet{@dtm@#1@month}{\@dtm@month}%
  \cslet{@dtm@#1@day}{\@dtm@day}%
  \ifcsundef{@dtm@#1@hour}{\csdef{@dtm@#1@hour}{0}}{}%
  \ifcsundef{@dtm@#1@minute}{\csdef{@dtm@#1@minute}{0}}{}%
  \ifcsundef{@dtm@#1@second}{\csdef{@dtm@#1@second}{0}}{}%
  \ifcsundef{@dtm@#1@TZhour}{\csdef{@dtm@#1@TZhour}{0}}{}%
  \ifcsundef{@dtm@#1@TZminute}{\csdef{@dtm@#1@TZminute}{0}}{}%
}
%    \end{macrocode}
%\end{macro}
%
%\begin{macro}{\DTMsaveddatetojulianday}
% Converts a saved date to a Julian day number. The first argument
% is the name referencing the saved date, the second is a count
% register in which to store the result.
%    \begin{macrocode}
\newrobustcmd*{\DTMsaveddatetojulianday}[2]{%
  \ifcsundef{@dtm@#1@year}%
  {%
     \PackageError{datetime2-calc}{Unknown date `#1'}{}%
  }%
  {%
    \pgfcalendardatetojulian
     {\csname @dtm@#1@year\endcsname
     -\csname @dtm@#1@month\endcsname
     -\csname @dtm@#1@day\endcsname}
     {#2}%
  }%
}
%    \end{macrocode}
%\end{macro}
%
%\begin{macro}{\DTMsaveddateoffsettojulianday}
% Converts an offset from the saved date to a Julian day number.
% The first argument is the name referencing the saved date,
% the second is the offset increment and the third is a count register
% in which to store the result.
%    \begin{macrocode}
\newrobustcmd*{\DTMsaveddateoffsettojulianday}[3]{%
  \ifcsundef{@dtm@#1@year}%
  {%
     \PackageError{datetime2-calc}{Unknown date `#1'}{}%
  }%
  {%
    \pgfcalendardatetojulian
     {\csname @dtm@#1@year\endcsname
     -\csname @dtm@#1@month\endcsname
     -\csname @dtm@#1@day\endcsname
     +#2}
     {#3}%
  }%
}
%    \end{macrocode}
%\end{macro}
%
%\begin{macro}{\DTMifdate}
% Test a saved date using \cs{pgfcalendarifdate}
%    \begin{macrocode}
\newrobustcmd*{\DTMifdate}[4]{%
  \ifcsundef{@dtm@#1@year}%
  {%
     \PackageError{datetime2-calc}{Unknown date `#1'}{}%
  }%
  {%
    \pgfcalendarifdate
     {\csname @dtm@#1@year\endcsname
     -\csname @dtm@#1@month\endcsname
     -\csname @dtm@#1@day\endcsname}
     {#2}{#3}{#4}%
  }%
}
%    \end{macrocode}
%\end{macro}
%
%\begin{macro}{\DTMsaveddatediff}
% Computes the difference between two saved dates. The result is
% stored in the third argument, which should be a count register.
%    \begin{macrocode}
\newrobustcmd*{\DTMsaveddatediff}[3]{%
  \ifcsundef{@dtm@#1@year}%
  {%
     \PackageError{datetime2-calc}{Unknown date `#1'}{}%
  }%
  {%
     \ifcsundef{@dtm@#2@year}%
     {%
        \PackageError{datetime2-calc}{Unknown date `#1'}{}%
     }%
     {%
       \pgfcalendardatetojulian
        {\csname @dtm@#1@year\endcsname
        -\csname @dtm@#1@month\endcsname
        -\csname @dtm@#1@day\endcsname}
        {#3}%
       \pgfcalendardatetojulian
        {\csname @dtm@#2@year\endcsname
        -\csname @dtm@#2@month\endcsname
        -\csname @dtm@#2@day\endcsname}
        {\@dtm@julianday}%
       \advance#3 by -\@dtm@julianday\relax
    }
  }%
}
%    \end{macrocode}
%\end{macro}
%
%\begin{macro}{\DTMtozulu}
% Converts the datetime data referenced by the first argument into
% Zulu time and saves it to data referenced by the second argument.
%    \begin{macrocode}
\newrobustcmd*{\DTMtozulu}[2]{%
  \ifcsundef{@dtm@#1@year}%
  {%
     \PackageError{datetime2-calc}{Unknown date `#1'}{}%
  }%
  {%
    \DTMsaveaszulutime{#2}%
    {\DTMfetchyear{#1}}%
    {\DTMfetchmonth{#1}}%
    {\DTMfetchday{#1}}%
    {\DTMfetchhour{#1}}%
    {\DTMfetchminute{#1}}%
    {\DTMfetchsecond{#1}}%
    {\DTMfetchTZhour{#1}}%
    {\DTMfetchTZminute{#1}}%
  }%
}
%    \end{macrocode}
%\end{macro}
%
%\begin{macro}{\DTMsaveaszulutime}
% Converts the given datetime into Zulu (+00:00) and saves
% the result.
%\changes{1.1}{2015-09-15}{fixed bug in incorrect conversion}
%\begin{definition}
%\cs{DTMsavetozulutime}\marg{name}\marg{year}\marg{month}\marg{day}\marg{hour}\newline\marg{minute}\marg{second}\marg{tzh}\marg{tzm}
%\end{definition}
%    \begin{macrocode}
\newrobustcmd*{\DTMsaveaszulutime}[9]{%
  \edef\@dtm@year{\number#2}%
  \edef\@dtm@month{\number#3}%
  \edef\@dtm@day{\number#4}%
  \edef\@dtm@hour{\number#5}%
  \edef\@dtm@minute{\number#6}%
  \edef\@dtm@second{\number#7}%
  \edef\@dtm@TZhour{\number#8}%
  \edef\@dtm@TZminute{\number#9}%
  \pgfcalendardatetojulian{\@dtm@year-\@dtm@month-\@dtm@day}{\@dtm@julianday}%
%    \end{macrocode}
% First adjust the minute offset if non-zero
%    \begin{macrocode}
  \ifnum\@dtm@TZminute=0\relax
  \else
    \count@=\@dtm@minute\relax
%    \end{macrocode}
% Add or subtract the offset minute
%    \begin{macrocode}
    \ifnum\@dtm@TZhour<0\relax
      \advance\count@ by \@dtm@TZminute\relax
    \else
      \advance\count@ by -\@dtm@TZminute\relax
    \fi
    \edef\@dtm@minute{\number\count@}%
%    \end{macrocode}
% Does the hour need adjusting?
%    \begin{macrocode}
    \ifnum\count@<0\relax
      \advance\count@ by 60\relax
      \edef\@dtm@minute{\number\count@}%
%    \end{macrocode}
% Need to subtract 1 from the hour
% but does the day need adjusting?
%    \begin{macrocode}
      \ifnum\@dtm@hour=0\relax
       \def\@dtm@hour{23}%
%    \end{macrocode}
% Day needs adjusting.
%    \begin{macrocode}
       \advance\@dtm@julianday by -1\relax
      \else
%    \end{macrocode}
% Subtract 1 from the hour
%    \begin{macrocode}
        \count@ = \@dtm@hour\relax
        \advance\count@ by -1\relax
        \edef\@dtm@hour{\number\count@}%
      \fi
    \else
%    \end{macrocode}
% Minute isn't negative. Is it $\ge 60$?
%    \begin{macrocode}
      \ifnum\count@>59\relax
        \advance\count@ by -60\relax
        \edef\@dtm@minute{\number\count@}%
%    \end{macrocode}
% Add 1 to the hour
%    \begin{macrocode}
        \count@ = \@dtm@hour\relax
        \advance\count@ by 1\relax
        \edef\@dtm@hour{\number\count@}%
%    \end{macrocode}
% Does the day need adjusting?
%\changes{1.1}{2015-09-15}{fixed bug in misnamed \cs{@dtm@hour}}
%    \begin{macrocode}
        \ifnum\@dtm@hour=24\relax
          \def\@dtm@hour{00}%
          \advance\@dtm@julianday by 1\relax
        \fi
      \fi
    \fi
  \fi
%    \end{macrocode}
% Now adjust the hour offset if non-zero
%    \begin{macrocode}
  \ifnum\@dtm@TZhour=0\relax
  \else
    \count@=\@dtm@hour\relax
    \advance\count@ by -\@dtm@TZhour\relax
%    \end{macrocode}
% Does the day need adjusting?
%    \begin{macrocode}
    \ifnum\count@<0\relax
      \advance\count@ by 24\relax
      \edef\@dtm@hour{\number\count@}%
      \advance\@dtm@julianday by -1\relax
    \else
      \ifnum\count@>23\relax
      \advance\count@ by -24\relax
      \edef\@dtm@hour{\number\count@}%
      \advance\@dtm@julianday by 1\relax
      \else
        \edef\@dtm@hour{\number\count@}%
      \fi
    \fi
  \fi
  \pgfcalendarjuliantodate{\@dtm@julianday}{\@dtm@year}{\@dtm@month}{\@dtm@day}%
  \pgfcalendarjuliantoweekday{\@dtm@julianday}{\count@}%
%    \end{macrocode}
% Save the results.
%    \begin{macrocode}
  \csedef{@dtm@#1@dow}{\number\count@}%
  \cslet{@dtm@#1@year}{\@dtm@year}%
  \cslet{@dtm@#1@month}{\@dtm@month}%
  \cslet{@dtm@#1@day}{\@dtm@day}%
  \cslet{@dtm@#1@hour}{\@dtm@hour}%
  \cslet{@dtm@#1@minute}{\@dtm@minute}%
  \cslet{@dtm@#1@second}{\@dtm@second}%
  \csdef{@dtm@#1@TZhour}{0}%
  \csdef{@dtm@#1@TZminute}{0}%
}
%    \end{macrocode}
%\end{macro}
%
%\subsection{Month and Weekday Names}
%These commands \emph{should not} be used in date styles. One of the
%reasons for replacing \sty{datetime} with \sty{datetime2} was
%caused by styles using language-variable names in language-specific
%syntax resulting in a mismatch with the syntax of one language (or
%region) with a translation of the month (and possibly weekday) name.
%These commands are provided for standalone use outside of styles.
%(Additionally, they're not expandable, which also makes them
%inappropriate for the styles that are expected to provide
%expandable dates.)
%
%\begin{macro}{\dtm@ifdianameexists}
%\changes{1.3}{2016-01-22}{new}
%    \begin{macrocode}
\newcommand*{\dtm@ifdianameexists}[3]{%
 \IfTrackedDialect{\languagename}%
 {%
   \ifcsdef{DTM\TrackedLanguageFromDialect{\languagename}#1}%
   {#2}%
   {#3}%
 }%
 {#3}%
}
%    \end{macrocode}
%\end{macro}
%
%\begin{macro}{\DTMmonthname}
%\changes{1.3}{2016-01-22}{new}
%    \begin{macrocode}
\newrobustcmd{\DTMmonthname}[1]{%
%    \end{macrocode}
% First check if \cs{DTM\meta{language}monthname} exists.
%    \begin{macrocode}
  \ifcsdef{DTM\languagename monthname}%
  {%
%    \end{macrocode}
% It exists, so use it.
%    \begin{macrocode}
    \csuse{DTM\languagename monthname}{#1}%
  }%
  {%
%    \end{macrocode}
% Try obtaining the language name from the dialect using
% \sty{tracklang}'s interface.
%    \begin{macrocode}
    \dtm@ifdianameexists{monthname}%
    {%
%    \end{macrocode}
% It exists, so use it.
%    \begin{macrocode}
       \csuse{DTM\TrackedLanguageFromDialect{\languagename}monthname}{#1}%
    }%
    {%
%    \end{macrocode}
% Can't determine the language name macro. This may be because the
% actual name can't be determined or it could be because the
% relevant language module can't be loaded so use \sty{pgf}'s
% command instead, which also has limited language support.
%    \begin{macrocode}
      \dtmnamewarning{\DTMmonthname}%
      \pgfcalendarmonthname{#1}%
    }%
  }%
}
%    \end{macrocode}
%\end{macro}
%
%\begin{macro}{\DTMMonthname}
%\changes{1.3}{2016-01-22}{new}
%    \begin{macrocode}
\newrobustcmd{\DTMMonthname}[1]{%
%    \end{macrocode}
% First check if \cs{DTM\meta{language}Monthname} exists.
%    \begin{macrocode}
  \ifcsdef{DTM\languagename Monthname}%
  {%
%    \end{macrocode}
% It exists, so use it.
%    \begin{macrocode}
    \csuse{DTM\languagename Monthname}{#1}%
  }%
  {%
%    \end{macrocode}
% Try obtaining the language name from the dialect using
% \sty{tracklang}'s interface.
%    \begin{macrocode}
    \dtm@ifdianameexists{Monthname}%
    {%
%    \end{macrocode}
% It exists, so use it.
%    \begin{macrocode}
       \csuse{DTM\TrackedLanguageFromDialect{\languagename}Monthname}{#1}%
    }%
    {%
%    \end{macrocode}
% Can't determine the language name macro. This could be because
% there's no upper case macro as the names always start with a
% capital (like English).
%    \begin{macrocode}
      \ifcsdef{DTM\languagename monthname}%
      {%
        \csuse{DTM\languagename monthname}{#1}%
      }%
      {%
        \dtm@ifdianameexists{monthname}%
        {%
          \csuse{DTM\TrackedLanguageFromDialect{\languagename}monthname}{#1}%
        }%
        {%
%    \end{macrocode}
% Can't find no-case change version either, so use \sty{pgfcalendar}
% command instead (which will need a case-change applied).
%    \begin{macrocode}
          \dtmnamewarning{\DTMMonthname}%
%    \end{macrocode}
% If \sty{mfirstuc} has been loaded, use it.
%    \begin{macrocode}
          \ifdef\emakefirstuc
          {%
           \emakefirstuc{\pgfcalendarmonthname{#1}}%
          }%
          {%
%    \end{macrocode}
% Hasn't been loaded, so just expand and apply \cs{MakeUppercase}:
%    \begin{macrocode}
            \protected@edef\dtm@tmp@monthname{\pgfcalendarmonthname{#1}}%
            \expandafter\MakeUppercase\dtm@tmp@monthname
          }%
        }%
      }%
    }%
  }%
}
%    \end{macrocode}
%\end{macro}
%
%\begin{macro}{\DTMshortmonthname}
%\changes{1.3}{2016-01-22}{new}
%    \begin{macrocode}
\newrobustcmd{\DTMshortmonthname}[1]{%
%    \end{macrocode}
% First check if \cs{DTM\meta{language}shortmonthname} exists.
%    \begin{macrocode}
  \ifcsdef{DTM\languagename shortmonthname}%
  {%
%    \end{macrocode}
% It exists, so use it.
%    \begin{macrocode}
    \csuse{DTM\languagename shortmonthname}{#1}%
  }%
  {%
%    \end{macrocode}
% Try obtaining the language name from the dialect using
% \sty{tracklang}'s interface.
%    \begin{macrocode}
    \dtm@ifdianameexists{shortmonthname}%
    {%
%    \end{macrocode}
% It exists, so use it.
%    \begin{macrocode}
       \csuse{DTM\TrackedLanguageFromDialect{\languagename}shortmonthname}{#1}%
    }%
    {%
%    \end{macrocode}
% Can't determine the language name macro. This may be because the
% actual name can't be determined or it could be because the
% relevant language module can't be loaded so use \sty{pgf}'s
% command instead, which also has limited language support.
%    \begin{macrocode}
      \dtmnamewarning{\DTMshortmonthname}%
      \pgfcalendarmonthshortname{#1}%
    }%
  }%
}
%    \end{macrocode}
%\end{macro}
%
%\begin{macro}{\DTMshortMonthname}
%\changes{1.3}{2016-01-22}{new}
%    \begin{macrocode}
\newrobustcmd{\DTMshortMonthname}[1]{%
%    \end{macrocode}
% First check if \cs{DTM\meta{language}shortMonthname} exists.
%    \begin{macrocode}
  \ifcsdef{DTM\languagename shortMonthname}%
  {%
%    \end{macrocode}
% It exists, so use it.
%    \begin{macrocode}
    \csuse{DTM\languagename shortMonthname}{#1}%
  }%
  {%
%    \end{macrocode}
% Try obtaining the language name from the dialect using
% \sty{tracklang}'s interface.
%    \begin{macrocode}
    \dtm@ifdianameexists{shortMonthname}%
    {%
%    \end{macrocode}
% It exists, so use it.
%    \begin{macrocode}
       \csuse{DTM\TrackedLanguageFromDialect{\languagename}shortMonthname}{#1}%
    }%
    {%
%    \end{macrocode}
% Can't determine the language name macro. This could be because
% there's no upper case macro as the names always start with a
% capital (like English).
%    \begin{macrocode}
      \ifcsdef{DTM\languagename shortmonthname}%
      {%
        \csuse{DTM\languagename shortmonthname}{#1}%
      }%
      {%
        \dtm@ifdianameexists{shortmonthname}%
        {%
           \csuse{DTM\TrackedLanguageFromDialect{\languagename}shortmonthname}%
           {#1}%
        }%
        {%
%    \end{macrocode}
% Can't find no-case change version either, so use \sty{pgfcalendar}
% command instead (which will need a case-change applied).
%    \begin{macrocode}
          \dtmnamewarning{\DTMshortMonthname}%
%    \end{macrocode}
% If \sty{mfirstuc} has been loaded, use it.
%    \begin{macrocode}
          \ifdef\emakefirstuc
          {%
           \emakefirstuc{\pgfcalendarmonthshortname{#1}}%
          }%
          {%
%    \end{macrocode}
% Hasn't been loaded, so just expand and apply \cs{MakeUppercase}:
%    \begin{macrocode}
            \protected@edef\dtm@tmp@monthname{\pgfcalendarmonthshortname{#1}}%
            \expandafter\MakeUppercase\dtm@tmp@monthname
          }%
        }%
      }%
    }%
  }%
}
%    \end{macrocode}
%\end{macro}
%
%\begin{macro}{\DTMcomputedayofweekindex}
%\begin{definition}
%\cs{DTMcomputedayofweekindex}\marg{date}\marg{cs}
%\end{definition}
% This is for standalone use and shouldn't be used in any date
% styles (since the day of week index is already supplied).
% The result is stored in the supplied control sequence.
%\changes{2016-02-11}{1.4}{new}
%    \begin{macrocode}
\newrobustcmd*{\DTMcomputedayofweekindex}[2]{%
  \pgfcalendardatetojulian{#1}{\@dtm@julianday}%
  \pgfcalendarjuliantoweekday{\@dtm@julianday}{\count@}%
  \edef#2{\number\count@}%
}
%    \end{macrocode}
%\end{macro}
%
%\begin{macro}{\DTMweekdayname}
%\changes{1.3}{2016-01-22}{new}
%    \begin{macrocode}
\newrobustcmd{\DTMweekdayname}[1]{%
%    \end{macrocode}
% First check if \cs{DTM\meta{language}weekdayname} exists.
%    \begin{macrocode}
  \ifcsdef{DTM\languagename weekdayname}%
  {%
%    \end{macrocode}
% It exists, so use it.
%    \begin{macrocode}
    \csuse{DTM\languagename weekdayname}{#1}%
  }%
  {%
%    \end{macrocode}
% Try obtaining the language name from the dialect using
% \sty{tracklang}'s interface.
%    \begin{macrocode}
    \dtm@ifdianameexists{weekdayname}%
    {%
%    \end{macrocode}
% It exists, so use it.
%    \begin{macrocode}
       \csuse{DTM\TrackedLanguageFromDialect{\languagename}weekdayname}{#1}%
    }%
    {%
%    \end{macrocode}
% Can't determine the language name macro. This may be because the
% actual name can't be determined or it could be because the
% relevant language module can't be loaded so use \sty{pgf}'s
% command instead, which also has limited language support.
%    \begin{macrocode}
      \dtmnamewarning{\DTMweekdayname}%
      \pgfcalendarweekdayname{#1}%
    }%
  }%
}
%    \end{macrocode}
%\end{macro}
%
%\begin{macro}{\DTMWeekdayname}
%\changes{1.3}{2016-01-22}{new}
%    \begin{macrocode}
\newrobustcmd{\DTMWeekdayname}[1]{%
%    \end{macrocode}
% First check if \cs{DTM\meta{language}Weekdayname} exists.
%    \begin{macrocode}
  \ifcsdef{DTM\languagename Weekdayname}%
  {%
%    \end{macrocode}
% It exists, so use it.
%    \begin{macrocode}
    \csuse{DTM\languagename Weekdayname}{#1}%
  }%
  {%
%    \end{macrocode}
% Try obtaining the language name from the dialect using
% \sty{tracklang}'s interface.
%    \begin{macrocode}
    \dtm@ifdianameexists{Weekdayname}%
    {%
%    \end{macrocode}
% It exists, so use it.
%    \begin{macrocode}
       \csuse{DTM\TrackedLanguageFromDialect{\languagename}Weekdayname}{#1}%
    }%
    {%
%    \end{macrocode}
% Can't determine the language name macro. This could be because
% there's no upper case macro as the names always start with a
% capital (like English).
%    \begin{macrocode}
      \ifcsdef{DTM\languagename weekdayname}%
      {%
        \csuse{DTM\languagename weekdayname}{#1}%
      }%
      {%
        \dtm@ifdianameexists{weekdayname}%
        {%
           \csuse{DTM\TrackedLanguageFromDialect{\languagename}weekdayname}{#1}%
        }%
        {%
%    \end{macrocode}
% Can't find no-case change version either, so use \sty{pgfcalendar}
% command instead (which will need a case-change applied).
%    \begin{macrocode}
          \dtmnamewarning{\DTMWeekdayname}%
%    \end{macrocode}
% If \sty{mfirstuc} has been loaded, use it.
%    \begin{macrocode}
          \ifdef\emakefirstuc
          {%
           \emakefirstuc{\pgfcalendarweekdayname{#1}}%
          }%
          {%
%    \end{macrocode}
% Hasn't been loaded, so just expand and apply \cs{MakeUppercase}:
%    \begin{macrocode}
            \protected@edef\dtm@tmp@weekdayname{\pgfcalendarweekdayname{#1}}%
            \expandafter\MakeUppercase\dtm@tmp@weekdayname
          }%
        }%
      }%
    }%
  }%
}
%    \end{macrocode}
%\end{macro}
%
%\begin{macro}{\DTMshortweekdayname}
%\changes{1.3}{2016-01-22}{new}
%    \begin{macrocode}
\newrobustcmd{\DTMshortweekdayname}[1]{%
%    \end{macrocode}
% First check if \cs{DTM\meta{language}shortweekdayname} exists.
%    \begin{macrocode}
  \ifcsdef{DTM\languagename shortweekdayname}%
  {%
%    \end{macrocode}
% It exists, so use it.
%    \begin{macrocode}
    \csuse{DTM\languagename shortweekdayname}{#1}%
  }%
  {%
%    \end{macrocode}
% Try obtaining the language name from the dialect using
% \sty{tracklang}'s interface.
%    \begin{macrocode}
    \dtm@ifdianameexists{shortweekdayname}%
    {%
%    \end{macrocode}
% It exists, so use it.
%    \begin{macrocode}
       \csuse{DTM\TrackedLanguageFromDialect{\languagename}shortweekdayname}{#1}%
    }%
    {%
%    \end{macrocode}
% Can't determine the language name macro. This may be because the
% actual name can't be determined or it could be because the
% relevant language module can't be loaded so use \sty{pgf}'s
% command instead, which also has limited language support.
%    \begin{macrocode}
      \dtmnamewarning{\DTMshortweekdayname}%
      \pgfcalendarweekdayshortname{#1}%
    }%
  }%
}
%    \end{macrocode}
%\end{macro}
%
%\begin{macro}{\DTMshortWeekdayname}
%\changes{1.3}{2016-01-22}{new}
%    \begin{macrocode}
\newrobustcmd{\DTMshortWeekdayname}[1]{%
%    \end{macrocode}
% First check if \cs{DTM\meta{language}shortWeekdayname} exists.
%    \begin{macrocode}
  \ifcsdef{DTM\languagename shortWeekdayname}%
  {%
%    \end{macrocode}
% It exists, so use it.
%    \begin{macrocode}
    \csuse{DTM\languagename shortWeekdayname}{#1}%
  }%
  {%
%    \end{macrocode}
% Try obtaining the language name from the dialect using
% \sty{tracklang}'s interface.
%    \begin{macrocode}
    \dtm@ifdianameexists{shortWeekdayname}%
    {%
%    \end{macrocode}
% It exists, so use it.
%    \begin{macrocode}
       \csuse{DTM\TrackedLanguageFromDialect{\languagename}shortWeekdayname}{#1}%
    }%
    {%
%    \end{macrocode}
% Can't determine the language name macro. This could be because
% there's no upper case macro as the names always start with a
% capital (like English).
%    \begin{macrocode}
      \ifcsdef{DTM\languagename shortweekdayname}%
      {%
        \csuse{DTM\languagename shortweekdayname}{#1}%
      }%
      {%
        \dtm@ifdianameexists{shortweekdayname}%
        {%
           \csuse{DTM\TrackedLanguageFromDialect{\languagename}shortweekdayname}%
           {#1}%
        }%
        {%
%    \end{macrocode}
% Can't find no-case change version either, so use \sty{pgfcalendar}
% command instead (which will need a case-change applied).
%    \begin{macrocode}
          \dtmnamewarning{\DTMshortWeekdayname}%
%    \end{macrocode}
% If \sty{mfirstuc} has been loaded, use it.
%    \begin{macrocode}
          \ifdef\emakefirstuc
          {%
           \emakefirstuc{\pgfcalendarweekdayshortname{#1}}%
          }%
          {%
%    \end{macrocode}
% Hasn't been loaded, so just expand and apply \cs{MakeUppercase}:
%    \begin{macrocode}
            \protected@edef\dtm@tmp@weekdayname{%
              \pgfcalendarweekdayshortname{#1}}%
            \expandafter\MakeUppercase\dtm@tmp@weekdayname
          }%
        }%
      }%
    }%
  }%
}
%    \end{macrocode}
%\end{macro}
%
%\begin{macro}{\DTMordinal}
%\changes{1.3}{2016-01-22}{new}
%    \begin{macrocode}
\newrobustcmd{\DTMordinal}[1]{%
%    \end{macrocode}
% First check if \cs{DTM\meta{language}ordinal} exists.
%    \begin{macrocode}
  \ifcsdef{DTM\languagename ordinal}%
  {%
%    \end{macrocode}
% It exists, so use it.
%    \begin{macrocode}
    \csuse{DTM\languagename ordinal}{#1}%
  }%
  {%
%    \end{macrocode}
% Try obtaining the language name from the dialect using
% \sty{tracklang}'s interface.
%    \begin{macrocode}
    \dtm@ifdianameexists{ordinal}%
    {%
%    \end{macrocode}
% It exists, so use it.
%    \begin{macrocode}
       \csuse{DTM\TrackedLanguageFromDialect{\languagename}ordinal}{#1}%
    }%
    {%
%    \end{macrocode}
% Can't determine the language name macro. This may be because the
% actual name can't be determined or it could be because the
% relevant language module can't be loaded so just display the
% number.
%    \begin{macrocode}
      \number#1
    }%
  }%
}
%    \end{macrocode}
%\end{macro}
%
%\begin{macro}{\dtmnamewarning}
%\changes{1.3}{2016-01-22}{new}
% Issue warning unless warnings have been suppressed.
%    \begin{macrocode}
\newcommand*{\dtmnamewarning}[1]{%
 \if@dtm@warn
  \PackageWarning{datetime2-calc}%
  {Can't find underlying language macro for \MessageBreak
   \string#1\space(language: \languagename); \MessageBreak
    using pgfcalendar macro instead}%
 \fi
}
%    \end{macrocode}
%\end{macro}
%\iffalse
%    \begin{macrocode}
%</datetime2-calc.sty>
%    \end{macrocode}
%\fi
%\Finale
\endinput
