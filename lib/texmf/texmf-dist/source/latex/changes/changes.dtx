% \CheckSum{1086}
%
% \iffalse meta-comment
%
%  Copyright (C) 2007-2015
%  Ekkart Kleinod (ekleinod@edgesoft.de)
% --------------------------------------------------------------------------
%
%  This work may be distributed and/or modified under the
%  conditions of the \LaTeX\ Project Public License, either version~1.3
%  of this license or any later version.
%  The latest version of this license is in\\
%   \url{http://www.latex-project.org/lppl.txt}\\
%  and version~1.3 or later is part of all distributions of \LaTeX\
%  version 2005/12/01 or later.
%
%  This work has the LPPL maintenance status "maintained".
%  The current maintainer of this work is Ekkart Kleinod.
%
%  Some code for providing multilanguage documentation was
%  used from the pst-pdf package by Rolf Niepraschk and Hubert Gaesslein.
% \fi
%
% \CharacterTable
%  {Upper-case    \A\B\C\D\E\F\G\H\I\J\K\L\M\N\O\P\Q\R\S\T\U\V\W\X\Y\Z
%   Lower-case    \a\b\c\d\e\f\g\h\i\j\k\l\m\n\o\p\q\r\s\t\u\v\w\x\y\z
%   Digits        \0\1\2\3\4\5\6\7\8\9
%   Exclamation   \!     Double quote  \"     Hash (number) \#
%   Dollar        \$     Percent       \%     Ampersand     \&
%   Acute accent  \'     Left paren    \(     Right paren   \)
%   Asterisk      \*     Plus          \+     Comma         \,
%   Minus         \-     Point         \.     Solidus       \/
%   Colon         \:     Semicolon     \;     Less than     \<
%   Equals        \=     Greater than  \>     Question mark \?
%   Commercial at \@     Left bracket  \[     Backslash     \\
%   Right bracket \]     Circumflex    \^     Underscore    \_
%   Grave accent  \`     Left brace    \{     Vertical bar  \|
%   Right brace   \}     Tilde         \~}
%
% \changes{v0.1}{2007/01/16}{Initial version.}
% \changes{v0.2}{2007/01/17}{new convenience commands, LPPL, bugfixes: missing babel package, ifthen-placement, loc, author markup}
% \changes{v0.3}{2007/01/22}{english documentation, replaced command changed with command replaced}
% \changes{v0.4}{2007/01/24}{pdfcolmk for improved markup, introduced author-ids, first CTAN version}
% \changes{v0.5}{2007/08/26}{reimplementation without array package, UTF-8, grayed text, change pf command arguments}
% \changes{v0.5.1}{2007/08/27}{deleted text is striked out again using package ulem, greying didn't work}
% \changes{v0.5.2}{2007/10/10}{package options for ulem and xcolor}
% \changes{v0.5.3}{2010/11/22}{use class options (final, draft) as well}
% \changes{v0.5.4}{2011/04/25}{extract user documentation; default language changed to English; script for removal of commands}
% \changes{v0.6.0}{2012/01/11}{redefined user interface for setting options, markup, authors; newly structured documentation}
% \changes{v1.0.0}{2012/04/25}{key-value-interface for change commands; special characters in list of changes}
% \changes{v2.0.0}{2013/06/30}{fixed problem with special characters in tabbing environment (loc), real list of changes, authormarkup}
% \changes{v2.0.1}{2013/08/10}{no changes in behavior or code; fixed problems with CTAN upload}
% \changes{v2.0.2}{2013/08/13}{again no changes in behavior or code; fixed CTAN upload - pdf files were corrupt; improved documentation}
% \changes{v2.0.3}{2014/10/15}{bugfix when using with amsart}
% \changes{v2.0.4}{2015/04/27}{unknown language does not lead to error: fallback English}
% \GetFileInfo{changes.dtx}
% \RecordChanges
%
%^^A --------------------------------------------------------------------------
%
% \maketitle
%
% \tableofcontents
% \cleardoublepage
%
% \ifENGLISH
% 	% \def\filename{changes.tex}
% \def\fileversion{1.0}
% \def\filedate{2004/06/30}
%
% American Mathematical Society
% Technical Support
% Publications Technical Group
% 201 Charles Street
% Providence, RI 02904
% USA
% tel: (401) 455-4080
%      (800) 321-4267 (USA and Canada only)
% fax: (401) 331-3842
% email: tech-support@ams.org
%
% Copyright 2004, 2007, 2010 American Mathematical Society.
%
% This work may be distributed and/or modified under the
% conditions of the LaTeX Project Public License, either version 1.3c
% of this license or (at your option) any later version.
% The latest version of this license is in
%   http://www.latex-project.org/lppl.txt
% and version 1.3c or later is part of all distributions of LaTeX
% version 2005/12/01 or later.
% 
% This work has the LPPL maintenance status `maintained'.
% 
% The Current Maintainer of this work is the American Mathematical
% Society.
%
% ====================================================================

\documentclass[11pt]{amsdtx}

%\usepackage[T1]{fontenc}

\usepackage[latin1]{inputenc}

\usepackage{fullpage}

\usepackage{amsrefs}

\MakeShortVerb{\|}

\makeatletter

\def\bysame{%
    \leavevmode\hbox to3em{\hrulefill}\thinspace
    \kern\z@
}

\DeclareRobustCommand{\fld}{\category@index{field}}
\DeclareRobustCommand{\btype}{\category@index{entry type}}

\long\def\@ifempty#1{\@xifempty#1@@..\@nil}

\long\def\@xifempty#1#2@#3#4#5\@nil{%
    \ifx#3#4\@xp\@firstoftwo\else\@xp\@secondoftwo\fi
}

\long\def\@ifnotempty#1{\@ifempty{#1}{}}

\newcounter{changecounter}

\newcommand\change[1][]{%
    \refstepcounter{changecounter}%
    \medskip
    \par
    \noindent
    \textbf{Change~\thechangecounter\@ifnotempty{#1}{ [#1]}.}\enskip
    \ignorespaces
}

%%  \newcommand\bugref[1]{%
%%      [#1]%
%%  }

\newcounter{tableline}[table]

\newcommand{\tbl}{%
    \refstepcounter{tableline}
    \thetableline.%
}

\makeatother

\title{User-Visible Changes to the \pkg{amsrefs} Package, version~2.0}

\author{David M. Jones}

\date{June 30, 2004}

\begin{document}

\maketitle

This document describes some of the user-visible changes made to the
\pkg{amsrefs} package between version~1.23, the last public release,
and version~2.0.  This list is not exhaustive.  Instead, the focus is
on correct (or at least reasonable) user documents that would produce
different output when processed under the two versions.  In general,
the changes discussed here fall between the following poles:
\begin{enumerate}

\item
Version~1.23 produces results that are subtly wrong.  These range from
things that would go unnoticed by any but the most attentive reader to
things that, even if noticed, would likely be shrugged off as
unimportant.

\item
Version~1.23 produces results that are obviously wrong, i.e., they
would be obvious under even the most cursory examination of the
output.  In most cases, this means the resulting document would be
unacceptable.

\end{enumerate}

We don't include the even more severe category of examples where
version~1.23 generates a fatal error since no document would be
produced in that case.  We also don't include enhancements or new
features that depend upon new syntax, since such constructs would not
be expected to be found in version~1.23 documents.

Reference numbers in square brackets refer to bugs reported by users.

\section{The \opt{initials} option}

See table~\ref{tbl:initials_bugs} for illustrations of these changes.

\begin{table}\small
\caption{Bugs in the \opt{initials} option}
\label{tbl:initials_bugs}
\begin{tabular*}{\textwidth}{@{}ll@{\extracolsep{\fill}}l@{\extracolsep{\fill}}l@{}}
& \textbf{Input} & \textbf{v1.23 output} & \textbf{v2.0 output} \\

\tbl &
|Doe, J.|
& J. . Doe
& J. Doe
\\

\tbl &
|Doe, John R.|
& J. R. . Doe
& J. R. Doe
\\

\tbl &
|Doe, Yu.|
& Y. . Doe
& Yu. Doe
\\

\tbl &
|Doe, J.~R.|
& J.~ R. . Doe
& J. R. Doe
\\

\tbl &
|Doe, John~Henry|
& J. Doe
& J. H. Doe
\\

\tbl &
|Bing, R H| & R H. Bing & R H Bing\\

\tbl &
|Katzenbach, N. {deB}elleville|
& @B@@@@@@@@ @@ N. Katzenbach
& N. deB. Katzenbach
\\

\tbl &
|Katzenbach, N. deB.|
& @ @ @@ N. Katzenbach
& N. deB. Katzenbach
\\

\tbl &
\texttt{Gr�mov, M�rtin}
& M�. Gr�mov
& M. Gr�mov
\\

\tbl &
|Doe, {Yu}ri|
& Y. Doe
& Yu. Doe

\end{tabular*}
\end{table}

\change[B-2005]\label{chng:final_dots}
When a given name ends in a period, an extra space and period would be
appended to the name.  Lines 1 and~2 in table~\ref{tbl:initials_bugs}
illustrate the most common cases.  Line~3 illustrates a related case
that is both a bug fix and an enhancement (cf.\ lines 7 and~8):
initials consisting of multiple characters were not recognized (cf.\
change~\ref{chng:multi_char_inits}).

\change[B-2005a]
Ties (|~|) between given names resulted in incorrect output, either
extra space (line~4) or a dropped initial (line~5).  Since \bibtex/
automatically generates ties, this problem occurs frequently when
using \bst{amsxport}.

\change
A name consisting of a single letter would receive an erroneous period
(line~6).

\change\label{chng:initial_lowercase}
Names beginning with lowercase letters were handled badly (lines 7
and~8) (cf.\ change~\ref{chng:multi_char_inits}).

\change
Extended character sets were not handled well.  For example, when
using T1 fonts, lowercase 8-bit characters were not processed
correctly (line~9).

\change\label{chng:multi_char_inits}
There was no way to indicate a multi-character initial
(line~10). (Cf.\ changes \ref{chng:final_dots}
and~\ref{chng:initial_lowercase}).

\section{The \fld{xref} field}

\change[B-2009]
Repeatable fields (such as \fld{editor}) were not inherited via the
\fld{xref} field. Consider the following example:
\begin{verbatim}
    \bib*{icalp77}{proceedings}{
        editor={Arto Salomaa},
        editor={Magnus Steinby},
        title={Automata, Languages and Programming, Fourth Colloquium}
    }

    \bib{AhoS1977}{article}{
        title={How Hard is Compiler Code Generation?},
        author={Alfred V. Aho},
        author={Ravi Sethi},
        xref={icalp77},
        pages={1\ndash 15}
    }
\end{verbatim}
Version~1.23 would produce
\begin{quote}
Alfred V. Aho and Ravi Sethi, \emph{How Hard is Compiler Code
  Generation?}, Automata, Languages and Programming, Fourth
  Colloquium, pp.~1--15.
\end{quote}
(without the editors) instead of
\begin{quote}
Alfred V. Aho and Ravi Sethi, \emph{How Hard is Compiler Code
  Generation?}, Automata, Languages and Programming, Fourth Colloquium
  (Arto Salomaa and Magnus Steinby, eds.), pp.~1--15.
\end{quote}
Other additive keys like \fld{author}, \fld{isbn} and \fld{review}
would similarly be dropped.

\section{The \opt{alphabetic} option}

%%  \change[B-2013]\label{chng:b2013}
%%  A compound citation such as
%%  \begin{verbatim}
%%      \citelist{ \cite{SS75}*{Theorem 1} \cite{TM86}*{Prop. 2} }
%%  \end{verbatim}
%%  used to produce
%%  \begin{quote}
%%      [SS75, Theorem 1, TM86, Prop. 2]
%%  \end{quote}
%%  whereas version~2.0 correctly separates the citations with a
%%  semi-colon instead of a comma:
%%  \begin{quote}
%%      [SS75, Theorem 1; TM86, Prop. 2]
%%  \end{quote}

\change[B-2009]
In version~1.23, generation of alphabetic labels was done by \bibtex/
rather than by \fn{amsrefs.sty}.  This means that if you did not use
\bibtex/ to generate your \cs{bib} entries from a \fn{.bib} file,
\pkg{amsrefs} would use the author's citation key as the alphabetic
label\footnote{It was also possible for the user to supply a different
alphabetic label using the \fld{label} field, but this feature was
undocumented.}.  This means that an entry such as
\begin{verbatim}
    \bib{D1999}{article}{
        author={Doe, John~R.},
        year={1999}
    }
\end{verbatim}
would appear as
\begin{quote}
[D1999] John R. Doe (1999).
\end{quote}
while if the same entry was generated from a \fn{.bib} file using
\bst{amsxport}, the entry would appear as
\begin{quote}
[Doe99] John R. Doe (1999).
\end{quote}
This approach had a number of problems, such as lack of uniformity, so
in version~2.0, it is \pkg{amsrefs.sty} that generates the alphabetic
label, not \bibtex/, and the entry above would appear as [Doe99] even
if \bibtex/ is not used.

\section{Changes to bibliography style}

\change
A book with both an author and an editor, for example,
\begin{verbatim}
    \bib{X}{book}{
        author={Doe, John},
        editor={Crow, Jack},
        title={Book with one author and one editor}
    }
\end{verbatim}
would be formatted as
\begin{quote}
    John DoeJack Crow (ed.), \emph{Book with one author and one editor}.
\end{quote}
rather than
\begin{quote}
    John Doe, \emph{Book with one author and one editor}.  Edited by
    Jack Crow.
\end{quote}

\change
Editors did not participate in the automatic \cs{bysame} processing,
so if you had two books by the same editor, e.g.
\begin{verbatim}
    \bib{A}{book}{
        editor={Crow, Jack},
        title={First book}
    }

    \bib{B}{book}{
        editor={Crow, Jack},
        title={Second book}
    }
\end{verbatim}
the result would be
\begin{quote}
[1] Jack Crow (ed.), \emph{First book}.

[2] Jack Crow (ed.), \emph{Second book}.
\end{quote}
instead of
\begin{quote}
[1] Jack Crow (ed.), \emph{First book}.

[2] \bysame (ed.), \emph{Second book}.
\end{quote}

Consider also what happens if we change the |editor| in the first
entry to |author|:
\begin{quote}
[1] Jack Crow, \emph{First book}.

[2] Jack Crow (ed.), \emph{Second book}.
\end{quote}
versus
\begin{quote}
[1] Jack Crow, \emph{First book}.

[2] \bysame (ed.), \emph{Second book}.
\end{quote}

\change
Some changes have been made to punctuation and order of fields to
bring the \pkg{amsrefs} format into line with the AMS house style.

\section{\cs{cite}-like commands}

\change
The \cs{ycites}, \cs{ocites}, \cs{fullcite} and \cs{fullocite}
commands for author-year citations were not implemented in
version~1.23.  Instead, they produced a warning on the terminal and
then behaved identically to the \cs{cites}, \cs{cite} or \cs{ocite}
commands.  They will be implemented in version~2.0 and thus produce
different output.  Table~\ref{tbl:cites} illustrates the changes.

\begin{table}[hb]\small
\caption{The \cs{cite}-like commands}\label{tbl:cites}
\begin{center}
\begin{minipage}{.9\textwidth}\small
Consider the following entries:
\begin{quote}
    Crow, Jack, John Doe, and John Bull. 1999.\,\dots

    Doe, John. 2000.\,\dots
\end{quote}
If the citation keys for the items are  ``A'' and~``B'', respectively,
the various \cs{cite} commands would produce the following:

\begin{tabular*}{\textwidth}{@{}ll@{\extracolsep{\fill}}l@{\extracolsep{\fill}}l@{}}
& \textbf{Input} & \textbf{v1.23 output} & \textbf{v2.0 output} \\

\tbl &
|\cite{A}|
& (Crow et al., 1999)
& (Crow et al., 1999)
\\

\tbl &
|\fullcite{A}|
& (Crow et al., 1999)
& (Crow, Doe, and Bull, 1999)
\\

\tbl &
|\ocite{A}|
& Crow et al. (1999)
& Crow et al. (1999)
\\

\tbl &
|\fullocite{A}|
& Crow et al. (1999)
& Crow, Doe, and Bull (1999)
\\

\tbl &
|\cites{A,B}|
& (Crow et al., 1999; Doe, 2000)
& (Crow et al., 1999; Doe, 2000)
\\

\tbl &
|\ycites{A,B}|
& (Crow et al., 1999; Doe, 2000)
& (1999, 2000)
\\

\tbl &
|\ocites{A,B}|
& (Crow et al., 1999; Doe, 2000)
& Crow et al. (1999) and Doe (2000)
\\

\end{tabular*}
\end{minipage}
\end{center}
\end{table}

\change
Citation-sorting did not always work correctly when there were
optional arguments involved.  So,
\begin{verbatim}
    \citelist{\cite{2} \cite{1} \cite{3}}
\end{verbatim}
would correctly produce
\begin{quote}
    [1--3]
\end{quote}
but
\begin{verbatim}
    \citelist{\cite{2} \cite{1} \cite{3}*{Theorem 1}}
\end{verbatim}
would produce
\begin{quote}
    [1, 3, Theorem 1; 2]
\end{quote}
instead of
\begin{quote}
    [1; 2; 3, Theorem 1]
\end{quote}
Note also that version~2.0 correctly switches to using semi-colons
while version~1.23 erroneously used a comma between the first and
second items.% (cf.\ change~\ref{chng:b2013}).

\section{The \pkg{inicap} package and \cs{EnglishInitialCaps}}

\change

As of version~2.0, \pkg{amsrefs} no longer attempts to adjust the
capitalization of book titles.  Version~1.23 shipped with a package
called \pkg{inicap} that provided a function for converting English
titles from normal upper/lower case to \qq{initial caps} form.  For
example, if an author typed
\begin{verbatim}
    booktitle={Group rings, crossed products and Galois theory}
\end{verbatim}
\pkg{amsrefs} would automatically adjust it to
\begin{quote}
    Group Rings, Crossed Products and Galois Theory
\end{quote}
However, \pkg{inicap} is incomplete and produces incorrect results in
cases like the following:
\begin{quote}
\verb+Term rewriting with sharing and memo{\"\i}zation+\hfill\break
\i Term Rewriting with Sharing and memo\"zation

\medskip

\verb+The \ldq low $M^*$-estimate\rdq\ for covering numbers+\hfill\break
The \ldq low $M^*$-estimate\rdq\ for Covering Numbers
\hfill\break(where ``low'' and ``estimate'' should be capitalized)

\end{quote}
and an example like the following, which uses the \verb+\qq+ command
that we recommend, would result in a fatal error:
\begin{quote}
\verb+The \qq{low $M^*$-estimate} for covering numbers+
\end{quote}

In addition, this feature was of limited usefulness since it was only
applied to book titles.  Authors should already be accustomed to
taking full responsibility for the capitalization of book titles since
\bibtex/ does not alter the capitalization of book titles.

The low utility of the feature does not seem to warrant the effort
that would be required to make the \pkg{inicap} package robust enough
for use in a production environment, so we have decided instead to
deprecate it and remove this functionality from the \pkg{amsrefs}
package.

\end{document}

% \fi
% \ifGERMAN
% 	% \def\filename{changes.tex}
% \def\fileversion{1.0}
% \def\filedate{2004/06/30}
%
% American Mathematical Society
% Technical Support
% Publications Technical Group
% 201 Charles Street
% Providence, RI 02904
% USA
% tel: (401) 455-4080
%      (800) 321-4267 (USA and Canada only)
% fax: (401) 331-3842
% email: tech-support@ams.org
%
% Copyright 2004, 2007, 2010 American Mathematical Society.
%
% This work may be distributed and/or modified under the
% conditions of the LaTeX Project Public License, either version 1.3c
% of this license or (at your option) any later version.
% The latest version of this license is in
%   http://www.latex-project.org/lppl.txt
% and version 1.3c or later is part of all distributions of LaTeX
% version 2005/12/01 or later.
% 
% This work has the LPPL maintenance status `maintained'.
% 
% The Current Maintainer of this work is the American Mathematical
% Society.
%
% ====================================================================

\documentclass[11pt]{amsdtx}

%\usepackage[T1]{fontenc}

\usepackage[latin1]{inputenc}

\usepackage{fullpage}

\usepackage{amsrefs}

\MakeShortVerb{\|}

\makeatletter

\def\bysame{%
    \leavevmode\hbox to3em{\hrulefill}\thinspace
    \kern\z@
}

\DeclareRobustCommand{\fld}{\category@index{field}}
\DeclareRobustCommand{\btype}{\category@index{entry type}}

\long\def\@ifempty#1{\@xifempty#1@@..\@nil}

\long\def\@xifempty#1#2@#3#4#5\@nil{%
    \ifx#3#4\@xp\@firstoftwo\else\@xp\@secondoftwo\fi
}

\long\def\@ifnotempty#1{\@ifempty{#1}{}}

\newcounter{changecounter}

\newcommand\change[1][]{%
    \refstepcounter{changecounter}%
    \medskip
    \par
    \noindent
    \textbf{Change~\thechangecounter\@ifnotempty{#1}{ [#1]}.}\enskip
    \ignorespaces
}

%%  \newcommand\bugref[1]{%
%%      [#1]%
%%  }

\newcounter{tableline}[table]

\newcommand{\tbl}{%
    \refstepcounter{tableline}
    \thetableline.%
}

\makeatother

\title{User-Visible Changes to the \pkg{amsrefs} Package, version~2.0}

\author{David M. Jones}

\date{June 30, 2004}

\begin{document}

\maketitle

This document describes some of the user-visible changes made to the
\pkg{amsrefs} package between version~1.23, the last public release,
and version~2.0.  This list is not exhaustive.  Instead, the focus is
on correct (or at least reasonable) user documents that would produce
different output when processed under the two versions.  In general,
the changes discussed here fall between the following poles:
\begin{enumerate}

\item
Version~1.23 produces results that are subtly wrong.  These range from
things that would go unnoticed by any but the most attentive reader to
things that, even if noticed, would likely be shrugged off as
unimportant.

\item
Version~1.23 produces results that are obviously wrong, i.e., they
would be obvious under even the most cursory examination of the
output.  In most cases, this means the resulting document would be
unacceptable.

\end{enumerate}

We don't include the even more severe category of examples where
version~1.23 generates a fatal error since no document would be
produced in that case.  We also don't include enhancements or new
features that depend upon new syntax, since such constructs would not
be expected to be found in version~1.23 documents.

Reference numbers in square brackets refer to bugs reported by users.

\section{The \opt{initials} option}

See table~\ref{tbl:initials_bugs} for illustrations of these changes.

\begin{table}\small
\caption{Bugs in the \opt{initials} option}
\label{tbl:initials_bugs}
\begin{tabular*}{\textwidth}{@{}ll@{\extracolsep{\fill}}l@{\extracolsep{\fill}}l@{}}
& \textbf{Input} & \textbf{v1.23 output} & \textbf{v2.0 output} \\

\tbl &
|Doe, J.|
& J. . Doe
& J. Doe
\\

\tbl &
|Doe, John R.|
& J. R. . Doe
& J. R. Doe
\\

\tbl &
|Doe, Yu.|
& Y. . Doe
& Yu. Doe
\\

\tbl &
|Doe, J.~R.|
& J.~ R. . Doe
& J. R. Doe
\\

\tbl &
|Doe, John~Henry|
& J. Doe
& J. H. Doe
\\

\tbl &
|Bing, R H| & R H. Bing & R H Bing\\

\tbl &
|Katzenbach, N. {deB}elleville|
& @B@@@@@@@@ @@ N. Katzenbach
& N. deB. Katzenbach
\\

\tbl &
|Katzenbach, N. deB.|
& @ @ @@ N. Katzenbach
& N. deB. Katzenbach
\\

\tbl &
\texttt{Gr�mov, M�rtin}
& M�. Gr�mov
& M. Gr�mov
\\

\tbl &
|Doe, {Yu}ri|
& Y. Doe
& Yu. Doe

\end{tabular*}
\end{table}

\change[B-2005]\label{chng:final_dots}
When a given name ends in a period, an extra space and period would be
appended to the name.  Lines 1 and~2 in table~\ref{tbl:initials_bugs}
illustrate the most common cases.  Line~3 illustrates a related case
that is both a bug fix and an enhancement (cf.\ lines 7 and~8):
initials consisting of multiple characters were not recognized (cf.\
change~\ref{chng:multi_char_inits}).

\change[B-2005a]
Ties (|~|) between given names resulted in incorrect output, either
extra space (line~4) or a dropped initial (line~5).  Since \bibtex/
automatically generates ties, this problem occurs frequently when
using \bst{amsxport}.

\change
A name consisting of a single letter would receive an erroneous period
(line~6).

\change\label{chng:initial_lowercase}
Names beginning with lowercase letters were handled badly (lines 7
and~8) (cf.\ change~\ref{chng:multi_char_inits}).

\change
Extended character sets were not handled well.  For example, when
using T1 fonts, lowercase 8-bit characters were not processed
correctly (line~9).

\change\label{chng:multi_char_inits}
There was no way to indicate a multi-character initial
(line~10). (Cf.\ changes \ref{chng:final_dots}
and~\ref{chng:initial_lowercase}).

\section{The \fld{xref} field}

\change[B-2009]
Repeatable fields (such as \fld{editor}) were not inherited via the
\fld{xref} field. Consider the following example:
\begin{verbatim}
    \bib*{icalp77}{proceedings}{
        editor={Arto Salomaa},
        editor={Magnus Steinby},
        title={Automata, Languages and Programming, Fourth Colloquium}
    }

    \bib{AhoS1977}{article}{
        title={How Hard is Compiler Code Generation?},
        author={Alfred V. Aho},
        author={Ravi Sethi},
        xref={icalp77},
        pages={1\ndash 15}
    }
\end{verbatim}
Version~1.23 would produce
\begin{quote}
Alfred V. Aho and Ravi Sethi, \emph{How Hard is Compiler Code
  Generation?}, Automata, Languages and Programming, Fourth
  Colloquium, pp.~1--15.
\end{quote}
(without the editors) instead of
\begin{quote}
Alfred V. Aho and Ravi Sethi, \emph{How Hard is Compiler Code
  Generation?}, Automata, Languages and Programming, Fourth Colloquium
  (Arto Salomaa and Magnus Steinby, eds.), pp.~1--15.
\end{quote}
Other additive keys like \fld{author}, \fld{isbn} and \fld{review}
would similarly be dropped.

\section{The \opt{alphabetic} option}

%%  \change[B-2013]\label{chng:b2013}
%%  A compound citation such as
%%  \begin{verbatim}
%%      \citelist{ \cite{SS75}*{Theorem 1} \cite{TM86}*{Prop. 2} }
%%  \end{verbatim}
%%  used to produce
%%  \begin{quote}
%%      [SS75, Theorem 1, TM86, Prop. 2]
%%  \end{quote}
%%  whereas version~2.0 correctly separates the citations with a
%%  semi-colon instead of a comma:
%%  \begin{quote}
%%      [SS75, Theorem 1; TM86, Prop. 2]
%%  \end{quote}

\change[B-2009]
In version~1.23, generation of alphabetic labels was done by \bibtex/
rather than by \fn{amsrefs.sty}.  This means that if you did not use
\bibtex/ to generate your \cs{bib} entries from a \fn{.bib} file,
\pkg{amsrefs} would use the author's citation key as the alphabetic
label\footnote{It was also possible for the user to supply a different
alphabetic label using the \fld{label} field, but this feature was
undocumented.}.  This means that an entry such as
\begin{verbatim}
    \bib{D1999}{article}{
        author={Doe, John~R.},
        year={1999}
    }
\end{verbatim}
would appear as
\begin{quote}
[D1999] John R. Doe (1999).
\end{quote}
while if the same entry was generated from a \fn{.bib} file using
\bst{amsxport}, the entry would appear as
\begin{quote}
[Doe99] John R. Doe (1999).
\end{quote}
This approach had a number of problems, such as lack of uniformity, so
in version~2.0, it is \pkg{amsrefs.sty} that generates the alphabetic
label, not \bibtex/, and the entry above would appear as [Doe99] even
if \bibtex/ is not used.

\section{Changes to bibliography style}

\change
A book with both an author and an editor, for example,
\begin{verbatim}
    \bib{X}{book}{
        author={Doe, John},
        editor={Crow, Jack},
        title={Book with one author and one editor}
    }
\end{verbatim}
would be formatted as
\begin{quote}
    John DoeJack Crow (ed.), \emph{Book with one author and one editor}.
\end{quote}
rather than
\begin{quote}
    John Doe, \emph{Book with one author and one editor}.  Edited by
    Jack Crow.
\end{quote}

\change
Editors did not participate in the automatic \cs{bysame} processing,
so if you had two books by the same editor, e.g.
\begin{verbatim}
    \bib{A}{book}{
        editor={Crow, Jack},
        title={First book}
    }

    \bib{B}{book}{
        editor={Crow, Jack},
        title={Second book}
    }
\end{verbatim}
the result would be
\begin{quote}
[1] Jack Crow (ed.), \emph{First book}.

[2] Jack Crow (ed.), \emph{Second book}.
\end{quote}
instead of
\begin{quote}
[1] Jack Crow (ed.), \emph{First book}.

[2] \bysame (ed.), \emph{Second book}.
\end{quote}

Consider also what happens if we change the |editor| in the first
entry to |author|:
\begin{quote}
[1] Jack Crow, \emph{First book}.

[2] Jack Crow (ed.), \emph{Second book}.
\end{quote}
versus
\begin{quote}
[1] Jack Crow, \emph{First book}.

[2] \bysame (ed.), \emph{Second book}.
\end{quote}

\change
Some changes have been made to punctuation and order of fields to
bring the \pkg{amsrefs} format into line with the AMS house style.

\section{\cs{cite}-like commands}

\change
The \cs{ycites}, \cs{ocites}, \cs{fullcite} and \cs{fullocite}
commands for author-year citations were not implemented in
version~1.23.  Instead, they produced a warning on the terminal and
then behaved identically to the \cs{cites}, \cs{cite} or \cs{ocite}
commands.  They will be implemented in version~2.0 and thus produce
different output.  Table~\ref{tbl:cites} illustrates the changes.

\begin{table}[hb]\small
\caption{The \cs{cite}-like commands}\label{tbl:cites}
\begin{center}
\begin{minipage}{.9\textwidth}\small
Consider the following entries:
\begin{quote}
    Crow, Jack, John Doe, and John Bull. 1999.\,\dots

    Doe, John. 2000.\,\dots
\end{quote}
If the citation keys for the items are  ``A'' and~``B'', respectively,
the various \cs{cite} commands would produce the following:

\begin{tabular*}{\textwidth}{@{}ll@{\extracolsep{\fill}}l@{\extracolsep{\fill}}l@{}}
& \textbf{Input} & \textbf{v1.23 output} & \textbf{v2.0 output} \\

\tbl &
|\cite{A}|
& (Crow et al., 1999)
& (Crow et al., 1999)
\\

\tbl &
|\fullcite{A}|
& (Crow et al., 1999)
& (Crow, Doe, and Bull, 1999)
\\

\tbl &
|\ocite{A}|
& Crow et al. (1999)
& Crow et al. (1999)
\\

\tbl &
|\fullocite{A}|
& Crow et al. (1999)
& Crow, Doe, and Bull (1999)
\\

\tbl &
|\cites{A,B}|
& (Crow et al., 1999; Doe, 2000)
& (Crow et al., 1999; Doe, 2000)
\\

\tbl &
|\ycites{A,B}|
& (Crow et al., 1999; Doe, 2000)
& (1999, 2000)
\\

\tbl &
|\ocites{A,B}|
& (Crow et al., 1999; Doe, 2000)
& Crow et al. (1999) and Doe (2000)
\\

\end{tabular*}
\end{minipage}
\end{center}
\end{table}

\change
Citation-sorting did not always work correctly when there were
optional arguments involved.  So,
\begin{verbatim}
    \citelist{\cite{2} \cite{1} \cite{3}}
\end{verbatim}
would correctly produce
\begin{quote}
    [1--3]
\end{quote}
but
\begin{verbatim}
    \citelist{\cite{2} \cite{1} \cite{3}*{Theorem 1}}
\end{verbatim}
would produce
\begin{quote}
    [1, 3, Theorem 1; 2]
\end{quote}
instead of
\begin{quote}
    [1; 2; 3, Theorem 1]
\end{quote}
Note also that version~2.0 correctly switches to using semi-colons
while version~1.23 erroneously used a comma between the first and
second items.% (cf.\ change~\ref{chng:b2013}).

\section{The \pkg{inicap} package and \cs{EnglishInitialCaps}}

\change

As of version~2.0, \pkg{amsrefs} no longer attempts to adjust the
capitalization of book titles.  Version~1.23 shipped with a package
called \pkg{inicap} that provided a function for converting English
titles from normal upper/lower case to \qq{initial caps} form.  For
example, if an author typed
\begin{verbatim}
    booktitle={Group rings, crossed products and Galois theory}
\end{verbatim}
\pkg{amsrefs} would automatically adjust it to
\begin{quote}
    Group Rings, Crossed Products and Galois Theory
\end{quote}
However, \pkg{inicap} is incomplete and produces incorrect results in
cases like the following:
\begin{quote}
\verb+Term rewriting with sharing and memo{\"\i}zation+\hfill\break
\i Term Rewriting with Sharing and memo\"zation

\medskip

\verb+The \ldq low $M^*$-estimate\rdq\ for covering numbers+\hfill\break
The \ldq low $M^*$-estimate\rdq\ for Covering Numbers
\hfill\break(where ``low'' and ``estimate'' should be capitalized)

\end{quote}
and an example like the following, which uses the \verb+\qq+ command
that we recommend, would result in a fatal error:
\begin{quote}
\verb+The \qq{low $M^*$-estimate} for covering numbers+
\end{quote}

In addition, this feature was of limited usefulness since it was only
applied to book titles.  Authors should already be accustomed to
taking full responsibility for the capitalization of book titles since
\bibtex/ does not alter the capitalization of book titles.

The low utility of the feature does not seem to warrant the effort
that would be required to make the \pkg{inicap} package robust enough
for use in a production environment, so we have decided instead to
deprecate it and remove this functionality from the \pkg{amsrefs}
package.

\end{document}

% \fi
%
%^^A -- source code
%
% \StopEventually
%
% \selectlanguage{english}
%
% \section{The documented sourcecode}
%
% The sourcecode is documented in English only.
% This is intended, please do not provide translations for the text below, just corrections or improvements.
%
%    \begin{macrocode}
%<*changes>
%    \end{macrocode}
%
% \subsection{Package information and options}
%
% Set needed \LaTeX-format to \LaTeXe{}, provide name, date, version.
% Type some information to the console.
%    \begin{macrocode}
\NeedsTeXFormat{LaTeX2e}
\ProvidesPackage{changes}
[2015/04/27 v2.0.4 changes package]
\typeout{*** changes package 2015/04/27 v2.0.4 ***}
%    \end{macrocode}
%
% Package \chpackage{xkeyval} provides options with key-value-pairs.
%    \begin{macrocode}
\RequirePackage{xkeyval}
%    \end{macrocode}
%
% Package \chpackage{xifthen} provides improved \texttt{if} as well as a \texttt{while}-loop.
%    \begin{macrocode}
\RequirePackage{xifthen}
%    \end{macrocode}
%
% \subsubsection{Package options}
%
% Option \choption{draft}, \emph{default} is \texttt{true}.
%    \begin{macrocode}
\newboolean{Changes@optiondraft}
\setboolean{Changes@optiondraft}{true}
\DeclareOptionX{draft}{
	\setboolean{Changes@optiondraft}{true}
	\typeout{changes-option '\CurrentOption'}
}
%    \end{macrocode}
%
% Option \choption{final}, sets \choption{draft} to \texttt{false}.
%    \begin{macrocode}
\DeclareOptionX{final}{
	\setboolean{Changes@optiondraft}{false}
	\typeout{changes-option '\CurrentOption'}
}
%    \end{macrocode}
%
% Declare storage for markup options, they are set by the markup option but can be changed with the more special options, therefore they have to be declared at this place.
%    \begin{macrocode}
\newcommand{\Changes@optionaddedmarkup}{none}
\newcommand{\Changes@optiondeletedmarkup}{sout}
%    \end{macrocode}
%
% Option \choption{markup}, sets markup options accordingly.
%    \begin{macrocode}
\newcommand{\Changes@optionmarkup}{default}
\DeclareOptionX{markup}{
	\ifthenelse{\equal{\@empty}{#1}}
		{}
		{
			\ifthenelse{
				\equal{#1}{default}\or
				\equal{#1}{underlined}\or
				\equal{#1}{bfit}\or
				\equal{#1}{nocolor}
			}
				{\renewcommand{\Changes@optionmarkup}{#1}}
				{\PackageWarning{changes}{markup '#1' unknown, using '\Changes@optionmarkup'}}
		}
	\ifthenelse{\equal{\Changes@optionmarkup}{default}}
		{
			\renewcommand{\Changes@optionaddedmarkup}{none}
			\renewcommand{\Changes@optiondeletedmarkup}{sout}
		}
		{}
	\ifthenelse{\equal{\Changes@optionmarkup}{underlined}}
		{
			\renewcommand{\Changes@optionaddedmarkup}{uline}
			\renewcommand{\Changes@optiondeletedmarkup}{sout}
		}
		{}
	\ifthenelse{\equal{\Changes@optionmarkup}{bfit}}
		{
			\renewcommand{\Changes@optionaddedmarkup}{bf}
			\renewcommand{\Changes@optiondeletedmarkup}{it}
		}
		{}
	\ifthenelse{\equal{\Changes@optionmarkup}{nocolor}}
		{
			\renewcommand{\Changes@optionaddedmarkup}{uline}
			\renewcommand{\Changes@optiondeletedmarkup}{sout}
		}
		{}
	\typeout{changes-option 'markup=\Changes@optionmarkup'}
}
%    \end{macrocode}
%
% Option \choption{addedmarkup}, stored or set to default value ``\texttt{none}''.
%    \begin{macrocode}
\DeclareOptionX{addedmarkup}{
	\ifthenelse{\equal{\@empty}{#1}}
		{}
		{
			\ifthenelse{
				\equal{#1}{none}\or
				\equal{#1}{uline}\or
				\equal{#1}{uuline}\or
				\equal{#1}{uwave}\or
				\equal{#1}{dashuline}\or
				\equal{#1}{dotuline}\or
				\equal{#1}{sout}\or
				\equal{#1}{xout}\or
				\equal{#1}{bf}\or
				\equal{#1}{it}\or
				\equal{#1}{sl}\or
				\equal{#1}{em}
			}
				{\renewcommand{\Changes@optionaddedmarkup}{#1}}
				{\PackageWarning{changes}{addedmarkup '#1' unknown, using '\Changes@optionaddedmarkup'}}
		}
	\typeout{changes-option 'addedmarkup=\Changes@optionaddedmarkup'}
}
%    \end{macrocode}
%
% Option \choption{deletedmarkup}, stored or set to default value ``\texttt{striked}''.
%    \begin{macrocode}
\DeclareOptionX{deletedmarkup}{
	\ifthenelse{\equal{\@empty}{#1}}
		{}
		{
			\ifthenelse{
				\equal{#1}{none}\or
				\equal{#1}{uline}\or
				\equal{#1}{uuline}\or
				\equal{#1}{uwave}\or
				\equal{#1}{dashuline}\or
				\equal{#1}{dotuline}\or
				\equal{#1}{sout}\or
				\equal{#1}{xout}\or
				\equal{#1}{bf}\or
				\equal{#1}{it}\or
				\equal{#1}{sl}\or
				\equal{#1}{em}
			}
				{\renewcommand{\Changes@optiondeletedmarkup}{#1}}
				{\PackageWarning{changes}{deletedmarkup '#1' unknown, using '\Changes@optiondeletedmarkup'}}
		}
	\typeout{changes-option 'deletedmarkup=\Changes@optiondeletedmarkup'}
}
%    \end{macrocode}
%
% Declare storage for authormarkup option and store option value or set to default value ``\texttt{superscript}''.
%    \begin{macrocode}
\newcommand{\Changes@optionauthormarkup}{superscript}
\DeclareOptionX{authormarkup}{
	\ifthenelse{\equal{\@empty}{#1}}
		{}
		{
			\ifthenelse{
				\equal{#1}{superscript}\or
				\equal{#1}{subscript}\or
				\equal{#1}{brackets}\or
				\equal{#1}{footnote}\or
				\equal{#1}{none}
			}
				{\renewcommand{\Changes@optionauthormarkup}{#1}}
				{\PackageWarning{changes}{authormarkup '#1' unknown, using '\Changes@optionauthormarkup'}}
		}
	\typeout{changes-option 'authormarkup=\Changes@optionauthormarkup'}
}
%    \end{macrocode}
%
% Declare storage for authormarkupposition option and store option value or set to default value ``\texttt{right}''.
%    \begin{macrocode}
\newcommand{\Changes@optionauthormarkupposition}{right}
\DeclareOptionX{authormarkupposition}{
	\ifthenelse{\equal{\@empty}{#1}}
		{}
		{
			\ifthenelse{
				\equal{#1}{right}\or
				\equal{#1}{left}
			}
				{\renewcommand{\Changes@optionauthormarkupposition}{#1}}
				{\PackageWarning{changes}{authormarkupposition '#1' unknown, using '\Changes@optionauthormarkupposition'}}
		}
	\typeout{changes-option 'authormarkupposition=\Changes@optionauthormarkupposition'}
}
%    \end{macrocode}
%
% Declare storage for authormarkuptext option and store option value or set to default value ``\texttt{id}''.
%    \begin{macrocode}
\newcommand{\Changes@optionauthormarkuptext}{id}
\DeclareOptionX{authormarkuptext}{
	\ifthenelse{\equal{\@empty}{#1}}
		{}
		{
			\ifthenelse{
				\equal{#1}{id}\or
				\equal{#1}{name}
			}
				{\renewcommand{\Changes@optionauthormarkuptext}{#1}}
				{\PackageWarning{changes}{authormarkuptext '#1' unknown, using '\Changes@optionauthormarkuptext'}}
		}
	\typeout{changes-option 'authormarkuptext=\Changes@optionauthormarkuptext'}
}
%    \end{macrocode}
%
%
%
% Options for package \chpackage{ulem} are directly passed to the package.
%    \begin{macrocode}
\DeclareOptionX{ulem}{
	\typeout{ulem-option '#1', passed to package ulem}
	\PassOptionsToPackage{#1}{ulem}
}
%    \end{macrocode}
%
% Options for package \chpackage{xcolor} are directly passed to the package.
%    \begin{macrocode}
\DeclareOptionX{xcolor}{
	\typeout{xcolor-option '#1', passed to package xcolor}
	\PassOptionsToPackage{#1}{xcolor}
}
%    \end{macrocode}
%
% Unknown options generate a package warning.
%    \begin{macrocode}
\DeclareOptionX*{
	\PackageWarning{changes}{Unknown option '\CurrentOption'}
}
%    \end{macrocode}
%
% \subsubsection{Command options}
%
% All options for commands (e.g. \chcommand{definechangesauthor}) have to be declared before option processing.
%
% \minisec{\chcommand{definechangesauthor}}
%
% Declare available options of the command, define value storage.
%    \begin{macrocode}
\DeclareOptionX<Changes@definechangesauthor>{name}{\def\Changes@definechangesauthor@name{#1}}
\DeclareOptionX<Changes@definechangesauthor>{color}{\def\Changes@definechangesauthor@color{#1}}
%    \end{macrocode}
%
% Set the default values of the options.
%    \begin{macrocode}
\presetkeys{Changes@definechangesauthor}{
	name=\@empty,
	color=black
}{}
%    \end{macrocode}
%
% \minisec{\chcommand{added}}
%
% Declare available options of the command, define value storage.
%    \begin{macrocode}
\DeclareOptionX<Changes@added>{id}{\def\Changes@added@id{#1}}
\DeclareOptionX<Changes@added>{remark}{\def\Changes@added@remark{#1}}
\DeclareOptionX<Changes@added>{decision}{\def\Changes@added@dec{#1}}
\DeclareOptionX<Changes@added>{decisionid}{\def\Changes@added@decid{#1}}
\DeclareOptionX<Changes@added>{decisionremark}{\def\Changes@added@decrem{#1}}
%    \end{macrocode}
%
% Set the default values of the options.
%    \begin{macrocode}
\presetkeys{Changes@added}{
	id=\@empty,
	remark=\@empty,
	decision=\@empty,
	decisionid=\@empty,
	decisionremark=\@empty
}{}
%    \end{macrocode}
%
% \minisec{\chcommand{deleted}}
%
% Declare available options of the command, define value storage.
%    \begin{macrocode}
\DeclareOptionX<Changes@deleted>{id}{\def\Changes@deleted@id{#1}}
\DeclareOptionX<Changes@deleted>{remark}{\def\Changes@deleted@remark{#1}}
\DeclareOptionX<Changes@deleted>{decision}{\def\Changes@deleted@dec{#1}}
\DeclareOptionX<Changes@deleted>{decisionid}{\def\Changes@deleted@decid{#1}}
\DeclareOptionX<Changes@deleted>{decisionremark}{\def\Changes@deleted@decrem{#1}}
%    \end{macrocode}
%
% Set the default values of the options.
%    \begin{macrocode}
\presetkeys{Changes@deleted}{
	id=\@empty,
	remark=\@empty,
	decision=\@empty,
	decisionid=\@empty,
	decisionremark=\@empty
}{}
%    \end{macrocode}
%
% \minisec{\chcommand{replaced}}
%
% Declare available options of the command, define value storage.
%    \begin{macrocode}
\DeclareOptionX<Changes@replaced>{id}{\def\Changes@replaced@id{#1}}
\DeclareOptionX<Changes@replaced>{remark}{\def\Changes@replaced@remark{#1}}
\DeclareOptionX<Changes@replaced>{decision}{\def\Changes@replaced@dec{#1}}
\DeclareOptionX<Changes@replaced>{decisionid}{\def\Changes@replaced@decid{#1}}
\DeclareOptionX<Changes@replaced>{decisionremark}{\def\Changes@replaced@decrem{#1}}
%    \end{macrocode}
%
% Set the default values of the options.
%    \begin{macrocode}
\presetkeys{Changes@replaced}{
	id=\@empty,
	remark=\@empty,
	decision=\@empty,
	decisionid=\@empty,
	decisionremark=\@empty
}{}
%    \end{macrocode}
%
% \minisec{\chcommand{listofchanges}}
%
% Declare available options of the command, define value storage.
%    \begin{macrocode}
\DeclareOptionX<Changes@loc>{style}{\def\Changes@loc@style{#1}}
%    \end{macrocode}
%
% Set the default values of the options.
%    \begin{macrocode}
\presetkeys{Changes@loc}{
	style=list
}{}
%    \end{macrocode}
%
% \subsubsection{Option processing}
%
% Process the options.
%    \begin{macrocode}
\ProcessOptionsX*\relax
%    \end{macrocode}
%
% \subsection{Packages}
%
% Package \chpackage{xcolor} provides colored text.
% Package \chpackage{pdfcolmk} solves the problem of colored text and page breaks (has to be loaded after \chpackage{xcolor}).
%    \begin{macrocode}
\newboolean{Changes@colored}
\setboolean{Changes@colored}{true}
\ifthenelse{\equal{\Changes@optionmarkup}{nocolor}}
	{\setboolean{Changes@colored}{false}}
	{}
\ifthenelse{\boolean{Changes@colored}}
	{
		\RequirePackage{xcolor}
		\RequirePackage{pdfcolmk}
	}
	{}
%    \end{macrocode}
%
% Package \chpackage{ulem} provides commands for striking out text.
%    \begin{macrocode}
\ifthenelse{
	\equal{\Changes@optionaddedmarkup}{uline}\or
	\equal{\Changes@optionaddedmarkup}{uuline}\or
	\equal{\Changes@optionaddedmarkup}{uwave}\or
	\equal{\Changes@optionaddedmarkup}{dashuline}\or
	\equal{\Changes@optionaddedmarkup}{dotuline}\or
	\equal{\Changes@optionaddedmarkup}{sout}\or
	\equal{\Changes@optionaddedmarkup}{xout}\or
	\equal{\Changes@optiondeletedmarkup}{uline}\or
	\equal{\Changes@optiondeletedmarkup}{uuline}\or
	\equal{\Changes@optiondeletedmarkup}{uwave}\or
	\equal{\Changes@optiondeletedmarkup}{dashuline}\or
	\equal{\Changes@optiondeletedmarkup}{dotuline}\or
	\equal{\Changes@optiondeletedmarkup}{sout}\or
	\equal{\Changes@optiondeletedmarkup}{xout}
}
	{\RequirePackage[normalem,normalbf]{ulem}}
	{}
%    \end{macrocode}
%
% \subsection{Language dependent texts}
%
% If the \chpackage{babel} package is not loaded, the default language is English, in order to use another language, the user has to redefine the variables.
% If the \chpackage{babel} or the \chpackage{polyglossia} package is loaded, the default language is English too for undefined languages.
%    \begin{macrocode}
\newcommand*\listofchangesname{List of changes}
\newcommand*\summaryofchangesname{Changes}
\newcommand*\changesaddname{Added}
\newcommand*\changesdeletename{Deleted}
\newcommand*\changesreplacename{Replaced}
\newcommand*\changesauthorname{Author}
\newcommand*\changesanonymousname{anonymous}
\newcommand*\changesnoloc{List of changes is available after the next \LaTeX\ run.}
\newcommand*\changesnosoc{Summary of changes is available after the next \LaTeX\ run.}
%    \end{macrocode}
%
% The check for \chpackage{babel} or \chpackage{polyglossia}, define language dependent texts afterwards.
%    \begin{macrocode}
\newboolean{Changes@langpackage}
\setboolean{Changes@langpackage}{false}
\@ifpackageloaded{babel}
	{\setboolean{Changes@langpackage}{true}}
	{}
\@ifpackageloaded{polyglossia}
	{\setboolean{Changes@langpackage}{true}}
	{}
\ifthenelse{\boolean{Changes@langpackage}}
	{
		\addto\captionsngerman{\def\listofchangesname{Liste der \"Anderungen}}
		\addto\captionsngerman{\def\summaryofchangesname{\"Anderungen}}
		\addto\captionsngerman{\def\changesaddname{Eingef\"ugt}}
		\addto\captionsngerman{\def\changesdeletename{Gel\"oscht}}
		\addto\captionsngerman{\def\changesreplacename{Ersetzt}}
		\addto\captionsngerman{\def\changesauthorname{Autor}}
		\addto\captionsngerman{\def\changesanonymousname{Anonym}}
		\addto\captionsngerman{\def\changesnoloc{Liste der \"Anderungen nach dem n\"achsten \LaTeX-Lauf verf\"ugbar.}}
		\addto\captionsngerman{\def\changesnosoc{\"Anderungen nach dem n\"achsten \LaTeX-Lauf verf\"ugbar.}}

		\addto\captionsgerman{\def\listofchangesname{Liste der \"Anderungen}}
		\addto\captionsgerman{\def\summaryofchangesname{\"Anderungen}}
		\addto\captionsgerman{\def\changesaddname{Eingef\"ugt}}
		\addto\captionsgerman{\def\changesdeletename{Gel\"oscht}}
		\addto\captionsgerman{\def\changesreplacename{Ersetzt}}
		\addto\captionsgerman{\def\changesauthorname{Autor}}
		\addto\captionsgerman{\def\changesanonymousname{Anonym}}
		\addto\captionsgerman{\def\changesnoloc{Liste der \"Anderungen nach dem n\"achsten \LaTeX-Lauf verf\"ugbar.}}
		\addto\captionsgerman{\def\changesnosoc{\"Anderungen nach dem n\"achsten \LaTeX-Lauf verf\"ugbar.}}

		\addto\captionsenglish{\def\listofchangesname{List of changes}}
		\addto\captionsenglish{\def\summaryofchangesname{Changes}}
		\addto\captionsenglish{\def\changesaddname{Added}}
		\addto\captionsenglish{\def\changesdeletename{Deleted}}
		\addto\captionsenglish{\def\changesreplacename{Replaced}}
		\addto\captionsenglish{\def\changesauthorname{Author}}
		\addto\captionsenglish{\def\changesanonymousname{anonymous}}
		\addto\captionsenglish{\def\changesnoloc{List of changes is available after the next \LaTeX\ run.}}
		\addto\captionsenglish{\def\changesnosoc{Summary of changes is available after the next \LaTeX\ run.}}

		\addto\captionsbritish{\def\listofchangesname{List of changes}}
		\addto\captionsbritish{\def\summaryofchangesname{Changes}}
		\addto\captionsbritish{\def\changesaddname{Added}}
		\addto\captionsbritish{\def\changesdeletename{Deleted}}
		\addto\captionsbritish{\def\changesreplacename{Replaced}}
		\addto\captionsbritish{\def\changesauthorname{Author}}
		\addto\captionsbritish{\def\changesanonymousname{anonymous}}
		\addto\captionsbritish{\def\changesnoloc{List of changes is available after the next \LaTeX\ run.}}
		\addto\captionsbritish{\def\changesnosoc{Summary of changes is available after the next \LaTeX\ run.}}

		\addto\captionsitalian{\def\listofchangesname{Lista delle modifiche}}
		\addto\captionsitalian{\def\summaryofchangesname{Modifiche}}
		\addto\captionsitalian{\def\changesaddname{Aggiunte}}
		\addto\captionsitalian{\def\changesdeletename{Cancellazioni}}
		\addto\captionsitalian{\def\changesreplacename{Sostituzioni}}
		\addto\captionsitalian{\def\changesauthorname{Autore}}
		\addto\captionsitalian{\def\changesanonymousname{anonimo}}
		\addto\captionsitalian{\def\changesnoloc{La lista delle modifiche sar\`a disponibile alla prossima esecuzione di \LaTeX.}}
		\addto\captionsitalian{\def\changesnosoc{Le modifiche sar\`a disponibile alla prossima esecuzione di \LaTeX.}}
	}
	{}
%    \end{macrocode}
%
% \subsection{File extension}
%
% \begin{macro}{\Changes@extension}
% Store file extension in variable, set default to \texttt{soc} (summary of changes).
%    \begin{macrocode}
\newcommand{\Changes@extension}{soc}
%    \end{macrocode}
% \end{macro}
%
% \begin{macro}{\setsocextension}
%  Set a new file extension.
%  Argument: new extension.
%    \begin{macrocode}
\newcommand{\setsocextension}[1]{
	\renewcommand{\Changes@extension}{#1}
}
%    \end{macrocode}
% \end{macro}
%
%
% \subsection{Authors}
%
% \subsubsection{Author management}
%
% Author counter.
%    \begin{macrocode}
\newcounter{Changes@AuthorCount}
\setcounter{Changes@AuthorCount}{0}
\newcounter{Changes@Author}
%    \end{macrocode}
%
% \begin{macro}{\definechangesauthor}
%  Define a new author.
%  Mandatory argument: author's id.
%  Optional arguments (key-value): author's name (default: empty) and author's color (default: black).
%
%  Store id, name and color using named variables.
%  Define counter and color per author.
%    \begin{macrocode}
\newcommand*\definechangesauthor[2][]{%
%    \end{macrocode}
%
% Call \emph{setkeys} in order to evaluate the key-value-options and fill the value storage.
%    \begin{macrocode}
	\setkeys{Changes@definechangesauthor}{#1}
%    \end{macrocode}
%
% Increment author counter, later needed for \emph{while} loop of authors.
%    \begin{macrocode}
	\stepcounter{Changes@AuthorCount}
%    \end{macrocode}
%
% Store the id in a name with the given counter/index.
% All other storage refers to the id.
%    \begin{macrocode}
	\@namedef{Changes@AuthorID\theChanges@AuthorCount}{#2}
%    \end{macrocode}
%
% Store the author's definition in according variables/colors, create change counters.
%    \begin{macrocode}
	\expandafter
	\let\csname Changes@AuthorName#2\endcsname=\Changes@definechangesauthor@name
	\newcounter{Changes@AddCount#2}
	\newcounter{Changes@DeleteCount#2}
	\newcounter{Changes@ReplaceCount#2}
	\ifthenelse{\boolean{Changes@colored}}
		{
			\expandafter
			\let\csname Changes@AuthorColor#2\endcsname=\Changes@definechangesauthor@color
			\colorlet{Changes@Color#2}{\@nameuse{Changes@AuthorColor#2}}
		}
		{}
}
%    \end{macrocode}
% \end{macro}
%
% Define default-author (anonymous) with empty id and blue color.
%    \begin{macrocode}
\definechangesauthor[color=blue]{\@empty}
%    \end{macrocode}
%
%
% \subsubsection{Author markup}
%
% \begin{macro}{\Changes@Markup@Author}
% Store markup for authors.
%    \begin{macrocode}
\newcommand{\Changes@Markup@Author}[1]{%
	\ifthenelse{\equal{\Changes@optionauthormarkup}{superscript}}{\textsuperscript{#1}}{}%
	\ifthenelse{\equal{\Changes@optionauthormarkup}{subscript}}{\textsubscript{#1}}{}%
	\ifthenelse{\equal{\Changes@optionauthormarkup}{brackets}}{(#1)}{}%
	\ifthenelse{\equal{\Changes@optionauthormarkup}{footnote}}{\footnote{#1}}{}%
	\ifthenelse{\equal{\Changes@optionauthormarkup}{none}}{}{}%
}
%    \end{macrocode}
% \end{macro}
%
% \begin{macro}{\setauthormarkup}
% Set markup for authors.
%    \begin{macrocode}
\newcommand{\setauthormarkup}[1]{
	\renewcommand{\Changes@Markup@Author}[1]{#1}
}
%    \end{macrocode}
% \end{macro}
%
% \begin{macro}{\textsubscript}
% Define the command \chcommand{textsubscript} in case the author markup \choption{subscript} is used and the command \chcommand{textsubscript} is not defined yet.
% \chcommand{textsubscript} is the antagonist of \chcommand{textsuperscript} which is predefined in \LaTeX.
% The code is taken from \LaTeX-FAQ 8.5.17.
%    \begin{macrocode}
\ifthenelse{\isundefined{\textsubscript}}
	{
		\DeclareRobustCommand*\textsubscript[1]{\@textsubscript{\selectfont#1}}
		\newcommand{\@textsubscript}[1]{{\m@th\ensuremath{_{\mbox{\fontsize\sf@size\z@#1}}}}}
	}{}
%    \end{macrocode}
% \end{macro}
%
% \begin{macro}{\setauthormarkupposition}
% Set position for author markup text.
%    \begin{macrocode}
\newcommand{\setauthormarkupposition}[1]{
	\renewcommand{\Changes@optionauthormarkupposition}{#1}
}
%    \end{macrocode}
% \end{macro}
%
% \begin{macro}{\setauthormarkuptext}
% Set author markup text to be displayed.
%    \begin{macrocode}
\newcommand{\setauthormarkuptext}[1]{
	\renewcommand{\Changes@optionauthormarkuptext}{#1}
}
%    \end{macrocode}
% \end{macro}
%
% \begin{macro}{\Changes@Remark}
%  Markup of remarks.
%  Default: in a footnote.
%    \begin{macrocode}
\newcommand{\Changes@Remark}[2]{%
	\footnote{%
		\ifthenelse{\not\equal{#1}{\@empty}}%
			{#1: }%
			{}%
		#2%
	}%
}
%    \end{macrocode}
% \end{macro}
%
% \begin{macro}{\setremarkmarkup}
%  Redefining the remark markup.
%  Mandatory argument: markup definition.
%    \begin{macrocode}
\newcommand{\setremarkmarkup}[1]{%
	\renewcommand{\Changes@Remark}[2]{#1}%
}
%    \end{macrocode}
% \end{macro}
%
% \subsection{Change management commands}
%
% \subsubsection{Text markup definition}
%
% Replaced text is always typeset as follows: \meta{added text}\meta{deleted text}.
% Therefore no extra command for markup of replaced text is given.
%
% \begin{macro}{\Changes@Markup@Added}
% Store markup for added text.
%    \begin{macrocode}
\newcommand{\Changes@Markup@Added}[1]{%
	\ifthenelse{\equal{\Changes@optionaddedmarkup}{none}}{#1}{}%
	\ifthenelse{\equal{\Changes@optionaddedmarkup}{uline}}{\uline{#1}}{}%
	\ifthenelse{\equal{\Changes@optionaddedmarkup}{uuline}}{\uuline{#1}}{}%
	\ifthenelse{\equal{\Changes@optionaddedmarkup}{uwave}}{\uwave{#1}}{}%
	\ifthenelse{\equal{\Changes@optionaddedmarkup}{dashuline}}{\dashuline{#1}}{}%
	\ifthenelse{\equal{\Changes@optionaddedmarkup}{dotuline}}{\dotuline{#1}}{}%
	\ifthenelse{\equal{\Changes@optionaddedmarkup}{sout}}{\sout{#1}}{}%
	\ifthenelse{\equal{\Changes@optionaddedmarkup}{xout}}{\xout{#1}}{}%
	\ifthenelse{\equal{\Changes@optionaddedmarkup}{bf}}{\textbf{#1}}{}%
	\ifthenelse{\equal{\Changes@optionaddedmarkup}{it}}{\textit{#1}}{}%
	\ifthenelse{\equal{\Changes@optionaddedmarkup}{sl}}{\textsl{#1}}{}%
	\ifthenelse{\equal{\Changes@optionaddedmarkup}{em}}{\emph{#1}}{}%
}
%    \end{macrocode}
% \end{macro}
%
% \begin{macro}{\setaddedmarkup}
% Set markup for added text.
%    \begin{macrocode}
\newcommand{\setaddedmarkup}[1]{
	\renewcommand{\Changes@Markup@Added}[1]{#1}
}
%    \end{macrocode}
% \end{macro}
%
% \begin{macro}{\Changes@Markup@Deleted}
% Store markup for deleted text.
%    \begin{macrocode}
\newcommand{\Changes@Markup@Deleted}[1]{%
	\ifthenelse{\equal{\Changes@optiondeletedmarkup}{none}}{#1}{}%
	\ifthenelse{\equal{\Changes@optiondeletedmarkup}{uline}}{\uline{#1}}{}%
	\ifthenelse{\equal{\Changes@optiondeletedmarkup}{uuline}}{\uuline{#1}}{}%
	\ifthenelse{\equal{\Changes@optiondeletedmarkup}{uwave}}{\uwave{#1}}{}%
	\ifthenelse{\equal{\Changes@optiondeletedmarkup}{dashuline}}{\dashuline{#1}}{}%
	\ifthenelse{\equal{\Changes@optiondeletedmarkup}{dotuline}}{\dotuline{#1}}{}%
	\ifthenelse{\equal{\Changes@optiondeletedmarkup}{sout}}{\sout{#1}}{}%
	\ifthenelse{\equal{\Changes@optiondeletedmarkup}{xout}}{\xout{#1}}{}%
	\ifthenelse{\equal{\Changes@optiondeletedmarkup}{bf}}{\textbf{#1}}{}%
	\ifthenelse{\equal{\Changes@optiondeletedmarkup}{it}}{\textit{#1}}{}%
	\ifthenelse{\equal{\Changes@optiondeletedmarkup}{sl}}{\textsl{#1}}{}%
	\ifthenelse{\equal{\Changes@optiondeletedmarkup}{em}}{\emph{#1}}{}%
}
%    \end{macrocode}
% \end{macro}
%
% \begin{macro}{\setdeletedmarkup}
% Set markup for deleted text.
%    \begin{macrocode}
\newcommand{\setdeletedmarkup}[1]{
	\renewcommand{\Changes@Markup@Deleted}[1]{#1}
}
%    \end{macrocode}
% \end{macro}
%
%
% \subsubsection{Change management command definition}
%
% \begin{macro}{\Changes@output}
% Output command for the changed text.
% This command has the following arguments:
% \begin{enumerate}
%		\item changed text (including markup)
%		\item unchanged text
%		\item author's id
%		\item remark
%		\item text for list of changes
%		\item change type for list of changes
%		\item decision (accept or reject)
%		\item decision author's id
%		\item decision remark
% \end{enumerate}
% Define boolean for author test.
%    \begin{macrocode}
\newboolean{Changes@WrongID}
\newcommand{\Changes@output}[9]{%
%    \end{macrocode}
%	Output changed text if option \choption{draft} is set, otherwise output unchanged text.
%    \begin{macrocode}
	\ifthenelse{\boolean{Changes@optiondraft}}%
		{%
%    \end{macrocode}
%	Check if the author exists, error message otherwise.
% I have the feeling that this code is optimizable.
%    \begin{macrocode}
			\setboolean{Changes@WrongID}{true}%
			\setcounter{Changes@Author}{0}%
			\whiledo{\value{Changes@Author} < \value{Changes@AuthorCount}}{%
				\stepcounter{Changes@Author}%
				\ifthenelse{\equal{#3}{\@nameuse{Changes@AuthorID\theChanges@Author}}}%
					{\setboolean{Changes@WrongID}{false}}%
					{}%
			}%
			\ifthenelse{\boolean{Changes@WrongID}}%
				{\PackageError{changes}%
					{Undefined changes author: #3}%
					{You have to define the author #3 with e.g.: \definechangesauthor{#3}}}%
				{}%
%    \end{macrocode}
%	Check if the decision's author exists, error message otherwise.
%    \begin{macrocode}
			\ifthenelse{\not\equal{#8}{\@empty}}%
				{%
					\setboolean{Changes@WrongID}{true}%
					\setcounter{Changes@Author}{0}%
					\whiledo{\value{Changes@Author} < \value{Changes@AuthorCount}}{%
						\stepcounter{Changes@Author}%
						\ifthenelse{\equal{#8}{\@nameuse{Changes@AuthorID\theChanges@Author}}}%
							{\setboolean{Changes@WrongID}{false}}%
							{}%
					}%
				\ifthenelse{\boolean{Changes@WrongID}}%
					{\PackageError{changes}%
						{Undefined changes author: #8}%
						{You have to define the author #8 with e.g.: \definechangesauthor{#8}}}%
					{}%
				}%
				{}%
%    \end{macrocode}
%	Save author text for output.
%    \begin{macrocode}
			\ifthenelse{\equal{\Changes@optionauthormarkuptext}{id}}%
				{%
					\@namedef{Changes@AuthorText}{#3}%
					\ifthenelse{\not\equal{#8}{\@empty}}%
						{\@namedef{Changes@DecAuthorText}{#8}}%
						{}%
				}%
				{}%
			\ifthenelse{\equal{\Changes@optionauthormarkuptext}{name}}%
				{%
					\@namedef{Changes@AuthorText}{\@nameuse{Changes@AuthorName#3}}%
					\ifthenelse{\not\equal{#8}{\@empty}}%
						{\@namedef{Changes@DecAuthorText}{\@nameuse{Changes@AuthorName#8}}}%
						{}%
				}%
				{}%
%    \end{macrocode}
%	Begin output
%    \begin{macrocode}
			{%
%    \end{macrocode}
%	Change color, if colored text is used.
%    \begin{macrocode}
				\ifthenelse{\boolean{Changes@colored}}%
					{\color{Changes@Color#3}}%
					{}%
%    \end{macrocode}
%	Output author text if author's id is given and text should appear left of changes.
%    \begin{macrocode}
				\ifthenelse{\equal{\Changes@optionauthormarkupposition}{left} \and \not\equal{#3}{\@empty}}%
					{\Changes@Markup@Author{\@nameuse{Changes@AuthorText}}}%
					{}%
%    \end{macrocode}
%	Output changed text.
%    \begin{macrocode}
				{#1}%
%    \end{macrocode}
%	Output author text if author's id is given and text should appear right of changes.
%    \begin{macrocode}
				\ifthenelse{\equal{\Changes@optionauthormarkupposition}{right} \and \not\equal{#3}{\@empty}}%
					{\Changes@Markup@Author{\@nameuse{Changes@AuthorText}}}%
					{}%
%    \end{macrocode}
%	Output remark if a remark is given.
%    \begin{macrocode}
				\ifthenelse{\not\equal{#4}{\@empty}}%
					{\Changes@Remark{#3}{#4}}%
					{}%
			}%
%    \end{macrocode}
%	Store line for list of changes.
%    \begin{macrocode}
			\ifthenelse{\equal{\@empty}{#3}}%
				{\def\Changes@locid{}}%
				{\def\Changes@locid{~(#3)}}%
			\addcontentsline{loc}{subsection}{#6\Changes@locid: \truncate{.3\textwidth}{#5}}%
		}%
%    \end{macrocode}
%	Output unchanged text (option \choption{final} was set).
%    \begin{macrocode}
		{#2}%
}
%    \end{macrocode}
% \end{macro}
%
% \begin{macro}{\added}
%  The command formats text as new text.
%
%  Mandatory argument: added text.
%  Optional argument (key-value): author's id, remark, decision, decision author's id, decision remark
%    \begin{macrocode}
\newcommand{\added}[2][\@empty]{%
%    \end{macrocode}
% Call \emph{setkeys} in order to evaluate the key-value-options and fill the value storage.
%    \begin{macrocode}
	\setkeys{Changes@added}{#1}%
	\Changes@output%
		{\Changes@Markup@Added{#2}}%
		{#2}%
		{\Changes@added@id}%
		{\Changes@added@remark}%
		{#2}%
		{\changesaddname}%
		{\Changes@added@dec}%
		{\Changes@added@decid}%
		{\Changes@added@decremark}%
	\stepcounter{Changes@AddCount\Changes@added@id}%
}
%    \end{macrocode}
% \end{macro}
%
% \begin{macro}{\deleted}
%  The command formats text as deleted text.
%
%  The definition of the empty text for unchanged text is taken from a tip from \texttt{de.comp.text.tex}.
%  It solves the problem of additional space caused by an empty command.
%
%  Mandatory argument: deleted text.
%  Optional argument (key-value): author's id, remark, decision, decision author's id, decision remark
%    \begin{macrocode}
\newcommand{\deleted}[2][\@empty]{%
%    \end{macrocode}
% Call \emph{setkeys} in order to evaluate the key-value-options and fill the value storage.
%    \begin{macrocode}
	\setkeys{Changes@deleted}{#1}%
	\Changes@output%
		{\Changes@Markup@Deleted{#2}}%
		{\@bsphack \expandafter \@esphack}%
		{\Changes@deleted@id}%
		{\Changes@deleted@remark}%
		{#2}%
		{\changesdeletename}%
		{\Changes@deleted@dec}%
		{\Changes@deleted@decid}%
		{\Changes@deleted@decremark}%
	\stepcounter{Changes@DeleteCount\Changes@deleted@id}%
}
%    \end{macrocode}
% \end{macro}
%
% \begin{macro}{\replaced}
%  The command formats text as replaced text.
%
%  Mandatory arguments: new text and old text.
%  Optional argument (key-value): author's id, remark, decision, decision author's id, decision remark
%    \begin{macrocode}
\newcommand{\replaced}[3][\@empty]{%
%    \end{macrocode}
% Call \emph{setkeys} in order to evaluate the key-value-options and fill the value storage.
%    \begin{macrocode}
	\setkeys{Changes@replaced}{#1}%
	\Changes@output
		{{\Changes@Markup@Added{#2}}{\Changes@Markup@Deleted{#3}}}
		{#2}
		{\Changes@replaced@id}
		{\Changes@replaced@remark}%
		{#2}%
		{\changesreplacename}%
		{\Changes@replaced@dec}%
		{\Changes@replaced@decid}%
		{\Changes@replaced@decremark}%
	\stepcounter{Changes@ReplaceCount\Changes@replaced@id}%
}
%    \end{macrocode}
% \end{macro}
%
% \subsection{List of changes}
%
% The list of changes truncates text, therefore the \chpackage{truncate} package is used.
%    \begin{macrocode}
\RequirePackage[breakall]{truncate}
%    \end{macrocode}
%
% \begin{macro}{\changes@chopline}
%  Auxiliary command for reading the content of the loc-files.
%  The content is read line by line.
%  One line is parsed with this macro, the order of entries is: id, color, name, added, deleted, replaced.
%  The contents have to be separated by a semicolon.
%    \begin{macrocode}
\def\changes@chopline#1;#2;#3;#4;#5;#6 \\{
	\def\Changes@InID{#1}
	\def\Changes@InColor{#2}
	\def\Changes@InName{#3}
	\def\Changes@InAdded{#4}
	\def\Changes@InDeleted{#5}
	\def\Changes@InReplaced{#6}
}
%    \end{macrocode}
% \end{macro}
%
% \begin{macro}{\listofchanges}
%		This command outputs the list of changes.
%
% 	Two styles are available: \choption{list} (default) and \choption{summary}.
% 	\choption{list} prints the list of changes lika a list of figures.
% 	\choption{summary} prints a summary of changes separated by authors.
%
% 	For the list, the values are read from the auxiliary file.
%
%		For the summary, the values are read from the loc-file, if it exists.
%		If no loc-file exists, an according message is generated.
%
% Some definitions that have to reside outside the command in order to use the command multiple times.
%	In further versions: compute length of bounding box für summary.
%    \begin{macrocode}
\newlength{\Changes@Len@summ}
\setlength{\Changes@Len@summ}{.2\textwidth}
\newboolean{Changes@MoreLines}
%    \end{macrocode}
% The definition of \chcommand{listofchanges}.
%    \begin{macrocode}
\newcommand{\listofchanges}[1][style=list]{%
	\setkeys{Changes@loc}{#1}%
	\ifthenelse{\equal{\@empty}{\Changes@loc@style}}
		{\def\Changes@loc@style{list}}
		{
			\ifthenelse{
				\equal{\Changes@loc@style}{list}\or
				\equal{\Changes@loc@style}{summary}
			}
				{}
				{\def\Changes@loc@style{list}}
		}
%    \end{macrocode}
%	Print list.
%    \begin{macrocode}
	\ifthenelse{\equal{\Changes@loc@style}{list}}
		{
			\section*{\listofchangesname}
			\IfFileExists{\jobname.loc}
				{}{
					\emph{\changesnoloc}
					\PackageWarning{changes}{LaTeX rerun needed for list of changes}
				}
			\@starttoc{loc}{}
		}
%    \end{macrocode}
%	Print summary, but only in draft mode.
%    \begin{macrocode}
		{
			\ifthenelse{\boolean{Changes@optiondraft}}
				{
					\section*{\summaryofchangesname}
					\IfFileExists{\jobname.\Changes@extension}
						{
							\setboolean{Changes@MoreLines}{true}
							\newread\Changes@InFile
							\openin\Changes@InFile = \jobname.\Changes@extension
							\whiledo{\boolean{Changes@MoreLines}}{
								\read\Changes@InFile to \Changes@Line
								\ifeof\Changes@InFile
									\setboolean{Changes@MoreLines}{false}
								\else
									\expandafter\changes@chopline\Changes@Line\\
									\textbf{%
										\ifthenelse{\boolean{Changes@colored}}
											{\color{\Changes@InColor}}
											{}
										\ifthenelse{\equal{\Changes@InID}{\@empty}}
											{\changesauthorname: \changesanonymousname}
											{%
												\changesauthorname: \Changes@InID
												\ifthenelse{\equal{\Changes@InName}{\@empty}}
													{}
													{ (\Changes@InName)}
											}
									}\\
									\parbox{\Changes@Len@summ}{\changesaddname~\dotfill~\Changes@InAdded}\\
									\parbox{\Changes@Len@summ}{\changesdeletename~\dotfill~\Changes@InDeleted}\\
									\parbox{\Changes@Len@summ}{\changesreplacename~\dotfill~\Changes@InReplaced}\\[1ex]
								\fi
							}
							\closein\Changes@InFile
						}{%
							\emph{\changesnosoc}
							\PackageWarning{changes}{LaTeX rerun needed for summary of changes}
						}
				}{}
		}
}
%    \end{macrocode}
% \end{macro}
%
%  At the end of the document: write the list of changes in the loc-file, therefore open file, write values, close file.
%  Changes are written as \LaTeX-formatted text, so they can simply be read via \chcommand{input}.
%
%  The order of entries is: id, color, name, added, deleted, replaced.
%  The contents have to be separated by a semicolon.
%    \begin{macrocode}
\AtEndDocument{
%    \end{macrocode}
% Open output file.
%    \begin{macrocode}
	\newwrite\Changes@OutFile
	\immediate\openout\Changes@OutFile = \jobname.\Changes@extension
%    \end{macrocode}
% Redefine expandable of \chcommand{protect} in order to write correct special characters.
% Store original definition for resetting \chcommand{protect}.
%    \begin{macrocode}
	\let\Changes@protect\protect
	\let\protect\@unexpandable@protect
%    \end{macrocode}
% Output data for list of changes.
%    \begin{macrocode}
	\setcounter{Changes@Author}{0}
	\whiledo{\value{Changes@Author} < \value{Changes@AuthorCount}}{
		\stepcounter{Changes@Author}
		\def\Changes@ID{\@nameuse{Changes@AuthorID\theChanges@Author}}
		\immediate\write\Changes@OutFile{\Changes@ID;%
			\@nameuse{Changes@AuthorColor\Changes@ID};%
			\@nameuse{Changes@AuthorName\Changes@ID};%
			\the\value{Changes@AddCount\Changes@ID};%
			\the\value{Changes@DeleteCount\Changes@ID};%
			\the\value{Changes@ReplaceCount\Changes@ID}}
	}
%    \end{macrocode}
% Close output file.
%    \begin{macrocode}
	\immediate\closeout\Changes@OutFile
%    \end{macrocode}
% Restore original definition of \chcommand{protect}.
%    \begin{macrocode}
	\let\protect\Changes@protect
}
%    \end{macrocode}
%
%    \begin{macrocode}
%</changes>
%    \end{macrocode}
%
% \PrintChanges
% \PrintIndex
%
%\Finale
\endinput

