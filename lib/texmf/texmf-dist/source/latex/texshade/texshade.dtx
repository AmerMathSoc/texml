%    \iffalse
%
% texshade.dtx
% Docstrip archive,  to extract documentation  run twice through 
% LaTeX.
% To  successfully  extract the documentation it is necessary to
% first run the file `texshade.ins' through LaTeX. This produces 
% the needed style  file  `texshade.sty'  and the parameter file
% `texshade.def' as well as several example files.  See the file
% `README.txt' for further information!
% 
%
% Copyright (C) 1999-2011 Eric Beitz
% See the file texshade.txt
% 
%    \fi
%
%    \changes{1.0}{1999-5-12}{First release}
%    \changes{1.1}{1999-5-26}{%
%      Corrections:  `emphregion'  is not extending to the next
%                                  alignment any more;
%                    `namecolor' & `numbercolor' are now correctly
%                                                reordered;
%                    sequence gaps at the beginning or the end are
%                    now treated correctly, i.e. no symbols are shown.
%      Introduction: `seqlength'
%                    `gapcolors'
%                    `rulersteps'
%                    `hideresidues'
%                    `showresidues'
%                    `fingerprint'.}
%    \changes{1.2}{1999-6-12}{%
%      Corrections:  functional shading error in funcgroup no. 8.
%      Introduction: `includeDSSP'
%                    `includeSTRIDE'
%                    `includePHDsec'
%                    `includePHDtopo'
%                    `appearance'
%                    `numcount'
%                    `alphacount'
%                    `Alphacount'
%                    `showonDSSP'
%                    `hideonDSSP'
%                    `showonSTRIDE'
%                    `hideonSTRIDE'
%                    `showonPHDsec'
%                    `hideonPHDsec'
%                    `showonPHDtopo'
%                    `hideonPHDtopo'.}
%    \changes{1.2a}{1999-6-24}{%
%      Corrections:  `namecolor' & `numbercolor' are now really
%                                                correctly reordered;
%                    in sequence names ( and ) are now allowed;
%                    option `case' in `funcshadingstyle' works now.}
%    \changes{1.3}{2000-3-3}{%
%      Corrections:  features in the ttop row do not produce line
%                                                scrambling any more;
%                    `language' replaced by `germanlanguage'
%                                       and `englishlanguage' due to
%                                       incompatibilities with babel;
%                    incompatibility with amsmath's text command fixed.
%      Introduction: new feature option `translate'
%                    `codon'
%                    `geneticcode'
%                    `backtranslabel'
%                    `backtranstext'
%                    `romancount'
%                    `Romancount'
%                    TeXtopo compatibility.}
%    \changes{1.3a}{2000-7-28}{%
%      Introduction: `showleadinggaps'
%                    `hideleadinggaps' to hide or show gap symbols 
%                                      before the actual seq start.}
%    \changes{1.3b}{2000-7-30}{%
%      Corrections:  `showleadinggaps'
%                    `hideleadinggaps' were extended to `ending' gaps;
%                    sequence names input routine now accepts special 
%                    characters.}
%                    
%    \changes{1.4}{2000-9-12}{%
%      Introduction: `movelegend'  allows one to move the legend;
%                    series of sequence numbers, e.g. in `orderseqs',
%                    can now be written with a dash, e.g. {1-3,6-4,7} instead
%                    of {1,2,3,6,5,4,7}.}
%
%    \changes{1.4a}{2000-10-3}{Documentation and FAQ additions}
%    
%    \changes{1.5}{2001-02-22}{%
%      Corrections:  `X's in the alignment lead to a run-time error;
%      Introduction: `ttopspace'
%                    `topspace'
%                    `bottomspace'
%                    `bbottomspace' for controlling vertical space 
%                                   between feature lines.
%                    `showcaption'  for adding a caption to the alignment.
%                    the sequence lengths are now stored in the .aux 
%                    in order to have correct gap breaks after the seqs.}
%                    
%    \changes{1.5a}{2001-03-08}{%
%      Corrections:  Eckhart Guth\"orlein noticed a sorting problem 
%                    when in addition some sequences where `killed' and
%                    the consensus was set to a particular sequence.
%                    This update fixes this problem.}
%                    
%    \changes{1.6}{2002-03-26}{%
%      Corrections:  There is no restriction to `dvips' anymore. One
%                    can add an option to the \usepackage{texshade}
%                    call which is passed to color.sty, e.g. `dvipdf';
%                    `noblockskip' led to over-printing of lines;
%                    `namecolor' and `numbercolor' did not support 
%                    sequence lists - fixed (thanks to Denys Bashtovyy).
%      Introduction: The FASTA file format is now supported;
%                    references to sequences can be made by name in
%                    addition to number;
%                    sequences can be refered to by their name in 
%                    addition to their number in the input file
%                    (suggested by Christoph Gille);
%                    `flexblockspace'   optimizes the space between 
%                                       sequence blocks to be minimal (is
%                                       default as before);
%                    `fixblockspace'    leads to an equal separation of
%                                       sequence blocks independent of
%                                       feature lines;
%                    `firstcolumnDSSP'  lets you choose the first numbering
%                                       column in DSSP input files and
%                    `secondcolumnDSSP' the second column.}
%                    
%    \changes{1.7}{2004-01-05}{%
%      Corrections:  Several bugs were fixed.
%                    In gaps the wrong character was plotted in `donotshade'
%                    mode. Gaps were colored incorrectly when a single 
%                    sequence was set as consensus. (thanks to Jeferson J. 
%                    Arenzon). Another `donotshade' problem was solved 
%                    which led to a halt of the LaTeX run (thanks go to
%                    Naomi Siew). The gap and match labels in diverse
%                    mode were switched (`-' in gaps; `.' at matching
%                    positions) in order to follow convention.
%      Introduction: Spanish labels (contributed by Mikel Egana Aranguren);
%                    New feature label `helix'.}
%   
%    \changes{1.8}{2004-08-26}{%
%      Corrections:  Minor bugs were fixed.
%      Introduction: Definition of "light" versions of all colors;
%                    Definition of three color ramps: 
%                        Red-Blue, Green-Red and Cold-Hot;
%                    New feature labels `bar' and `color'.}
%   
%    \changes{1.9}{2005-02-08}{%
%      Corrections:  Fixed TeXtopo incompatibility introduced with v1.8.
%                    (Thanks to Meike Schmedt)
%      Introduction: Implementation of HMMTOP topology prediction.                   
%                    `includeHMMTOP'
%                    `showonHMMTOP'
%                    `hideonHMMTOP'
%                    new `appearance' option {HMMTOP} with {internal}
%                                                          {external}
%                                                          {TM};
%                    new arrow look with scalable line thickness;
%                    new arrow option `ball';
%                    `frameblock'   colored frame around sequence block;
%                    `shortcaption' allows one to define short caption
%                                   versions for the List of Figures.}
%
%    \changes{1.10}{2005-03-29}{%
%      Corrections:  Sped up drawing of color scales and bar graphs by
%                    by more than 10fold!
%                    (Thanks, Christoph Gille, for asking for it)
%      Introduction: Definition of even lighter versions of all colors;
%                    implementation of a new labeling mode 'tint':                   
%                    `tintregion'
%                    `tintblock'
%                    `tintdefault';
%                    new `feature' option {restriction} for putting a
%                    triangle label pointing between two residues;
%                    data files for color scales and bar graphs can
%                    now contain 'NaN' (not a number) values
%                    (Also requested by Christoph Gille.)}
%
%    \changes{1.11}{2005-04-13}{%
%      Corrections:  Frames were drawn with the wrong height when
%                    separation lines were used. Fixed.
%                    Spacing between bar graph feature line and 
%                    sequence block was wrong after `bargraphstretch'.
%      Introduction: Additional optional parameter for feature rule
%                    thickness;                   
%                    additional optional parameters for feature box
%                    frame color and frame thickness;
%                    definition of three more color scales:
%                    {RedBlue}, {RedGreen}, and {HotCold};
%                    plotting of amino acid features as bar graphs
%                    or color scales:
%                    `hydrophobicity'
%                    `molweight'
%                    `charge';
%                    plotting of protein sequence conservation as
%                    bar graph or color scale:
%                    `conservation';
%                    separate command for stretching color scales:
%                    `colorscalestretch';
%                    color scales on consensus sequence according
%                    to sequence conservation.
%                    }
%
%    \changes{1.12}{2005-09-20}{%
%      Corrections:  Combination of 'setends' with regional labeling
%                    using 'shaderegion', 'frameblock', 'emphregion'
%                    or 'tintregion' produced incorrect output
%                    (thanks to Chris Page). Fixed.
%      Introduction: Optional colors for `showconsensus' foreground
%                    and background.
%                    }
%
%    \changes{1.13}{2006-02-23}{%
%      Corrections:  Helix symbols in feature lines were not drawn
%                    correctly if the standard Computer Modern Font
%                    was changed to another one, e.g. Palatino (thanks
%                    to Markus Heller). Fixed.
%                    Unintended gaps occurred due to numbers at the
%                    end of lines in Clustal W alignment files. Fixed.
%                    Frames were too tall when sequences were hidden
%                    or killed. Fixed.
%                    The limitations in the number of sequences per
%                    alignment have finally been overcome by a more
%                    restrictive use of counter variables.
%      Introduction: The numbering can now be displayed on both sides
%                    of the alignment with the optional parameter
%                    {leftright};
%                    TeXshade tries to guess the sequence type (protein 
%                    or nucleotide) if not defined by the user;
%                    Implementation of sequence logos:
%                    `showsequencelogo', `hidesequencelogo', 
%                    `namesequencelogo', `logostretch',
%                    `logocolor', `clearlogocolors',
%                    `showlogoscale', `hidelogoscale'
%                    `dofrequencycorrection', `undofrequencycorrection';
%                    The ruler numbering can now be rotated with
%                    `rotateruler' and back with `unrotateruler',
%                    this way every position can be numbered which is
%                    often wanted when e.g. sequence logos are plotted;
%                    the font family (sf, rm, tt) can be set for the
%                    ruler, e.g. `rulertt' or `setfamily{ruler}{tt}';
%                    `hideseqs' and `showseqs' in order to hide/show
%                    all sequences, esp. useful with sequence logos;
%                    `allowzero' and `disallowzero' - use (or do not 
%                    use) the number `0' in the sequence numbering as
%                    sometimes wanted in sequence logos;
%                    Implementation of a new way to visualize residues  
%                    which are characteristic for protein subfamilies, 
%                    i.e. subfamily logos:
%                    `showsubfamilylogo', `hidesubfamilylogo', 
%                    `namesubfamilylogo', 
%                    `setsubfamily', 
%                    `shownegatives', `hidenegatives'.
%                    }
%
%    \changes{1.14}{2006-05-11}{%
%      Introduction: `showrelevance', `hiderelevance',
%                    `relevance': commands to set a bit-value above
%                    which subfamily deviations are considered relevant
%                    and to label such positions in the subfamily logo.
%                    }
%
%    \changes{1.15}{2006-06-27}{%
%      Correction:   Logos can now be plotted with pdflatex; pstricks is
%                    not needed anymore.
%                    }
%
%    \changes{1.16}{2007-02-18}{%
%      Corrections:  TeXshade crashed when doing conservation
%                    calculations with sequences containing untypical
%                    residues symbols, such as X. Fixed.
%                    Shading of the reference sequence in diverse mode
%                    is now achieved with `conservedresidues' and
%                    `allmatchresidues' instead of `nomatchresidues'.
%      Introduction: `exportconsensus' produces a pymol script for
%                    coloring according to the TeXshade conservation
%                    calculation;
%                    `namerulerpos' allows one to change labels of
%                    the ruler individually;
%                    `hideblock' allows one to hide parts of the
%                    alignment (still in experimental stage!).
%      New home:     TeXshade, TeXtopo, and BioTeX have a new home:
%                    `www.pharmazie.uni-kiel.de/chem/Prof_Beitz/biotex.html'
%                    }
%
%    \changes{1.17}{2007-06-19}{%
%      Corrections:  .
%      Introduction: `allmatchspecial' now accepts an optional threshold
%                    percentage allowing one to set two levels off 
%                    conservation;
%                    the same effect can be achieved by using a
%                    number as an optional parameter in
%                    `shadingmode',
%                    or by setting an additional parameter in 
%                    `threshold';
%                    names can be displayed left or right of feature
%                    lines using:
%                    `showfeaturename', `showfeaturestylename';
%                    `hidefeaturename', `hidefeaturestylename';
%                    `hidefeaturenames', `hidefeaturestylenames';
%                    the color of such names can be changed with
%                    `featurenamescolor'
%                    `featurestylenamescolor'
%                    font styles can be set as usual, e.g.
%                    `setsize{featurenames}{large}' or
%                    `featurestylenamesrm' etc.
%                    }
%
%
%    \changes{1.18}{2008-04-15}{%
%      Corrections:  bug fixes (featurenames, ordering, numbering).
%      Introduction: T-Coffee shading in alignment, consensus and
%                    feature color scales and bar graphs
%                    `shadingmode[ASCII-file]{T-Coffee}'
%                    `includeTCoffee';
%                    two more feature lines on top and at bottom:
%                    `ttttop', `tttop', `bbbottom', and `bbbbottom';
%                    fusion of the `startnumber' and `setends' commands
%                    via optional parameters.
%                    }
%
%    \changes{1.19}{2009-03-09}{%
%      Corrections:  horizontal scaling of logo characters with `charstretch';
%                    further bug fixes.
%      Introduction: selection of residues depending on PDB file coordinates;
%                    works with `feature', `shaderegion', `shadeblock',
%                    `tintregion', `tintblock', `emphregion',
%                    `emphblock', `frameblock';
%                    3D selection possible around a point, along a line, or 
%                    within a plane;
%                    list of selected residues printable with `printPDBlist',
%                    or viewable during the TeX run with `messagePDBlist';
%                    show only selected residues in alignment with `setdomain',
%                    `domaingaprule' sets thickness of a domain separator rule,
%                    `domaingapcolors' sets fg/bg colors for the separator rule,
%                    these commands replace the unstable `hideblock' command.
%                    }
%
%    \changes{1.19a}{2009-06-11}{%
%      Corrections:  gaps are now treated correctly when setting domains.
%      Introduction: a special character can be displayed for stop-positions
%                    in protein sequences indicated by `*' in the input file;
%                    the stop character can be set using `stopchar'.
%                    }
%
%    \changes{1.20}{2009-10-05}{%
%      Corrections:  landscape mode works now correctly;
%                    annoying `Overfull \hbox ...' signals appearing after
%                    `residuesperline*' commands are ignored;
%                    domains work now with frequency corrected sequence logos.
%      Introduction: new commands for setting residue weight tables: 
%                    `weighttable' (with parameters: identity, structural,
%                                   PAM250, PAM100, and BLOSUM62),
%                    `gappenalty', and `setweight';
%                    similarity/identity percentage tables can be printed with
%                    `similaritytable' and values for specific sequence pairs 
%                    can be utilized with `percentsimilarity{seq1}{seq2}' and 
%                    `percentidentity{seq1}{seq2}'.
%                    }
%
%    \changes{1.20a}{2009-11-30}{%
%      Corrections:  identity/similarity tables work now without the need to
%                    set a conservation plot in the feature line.
%                    }
%
%    \changes{1.21}{2010-03-01}{%
%      Corrections:  `setdomain' ignored a changed `startnumber', corrected;
%                    further bug fixes. 
%      Introduction: in several commands sequence motifs can be given instead
%                    of numeral sequence stretch definitions;
%                    new type of sequence emphasis by lowercase characters:
%                    `lowerregion' and `lowerblock'.
%                    }
%
%    \changes{1.22}{2010-10-11}{%
%      Corrections:  minor bug fixes. 
%      Introduction: `defshadingcolors' defines and names sets of shading  
%                    colors for later re-use within the alignment;
%                    `changeshadingcolors' allows one to use different sets
%                    of shading colors in different sequence blocks.
%                    }
%
%    \changes{1.23}{2011-05-13}{%
%      Introduction: `hideallmatchpositions' removes all positions in the
%                    alignment where all residues match, i.e. a handy feature for
%                    diverse mode to condense the output to the relevant sites.
%                    }
%
%    \changes{1.24}{2011-12-01}{%
%      Introduction: `rulerspace' allows one to adjust the space between the
%                    ruler and the top or bottom sequence row;
%                    the textsize of the ruler numbering can now be set;
%                    Postscript color samples are shown in the manual.
%                    }
%
%
% \CharacterTable
%  {Upper-case    \A\B\C\D\E\F\G\H\I\J\K\L\M\N\O\P\Q\R\S\T\U\V\W\X\Y\Z
%   Lower-case    \a\b\c\d\e\f\g\h\i\j\k\l\m\n\o\p\q\r\s\t\u\v\w\x\y\z
%   Digits        \0\1\2\3\4\5\6\7\8\9
%   Exclamation   \!     Double quote  \"     Hash (number) \#
%   Dollar        \$     Percent       \%     Ampersand     \&
%   Acute accent  \'     Left paren    \(     Right paren   \)
%   Asterisk      \*     Plus          \+     Comma         \,
%   Minus         \-     Point         \.     Solidus       \/
%   Colon         \:     Semicolon     \;     Less than     \<
%   Equals        \=     Greater than  \>     Question mark \?
%   Commercial at \@     Left bracket  \[     Backslash     \\
%   Right bracket \]     Circumflex    \^     Underscore    \_
%   Grave accent  \`     Left brace    \{     Vertical bar  \|
%   Right brace   \}     Tilde         \~}
%   
%
%    \newsavebox{\mybox}
%    \newenvironment{fmpage}[1][0.975\textwidth]{%
%       \begin{lrbox}{\mybox}\begin{minipage}{#1}}
%      {\end{minipage}\end{lrbox}\fbox{\usebox{\mybox}}}
%        
%    \parindent0mm
%            
%     
%    \title{The \TeXshade{} package\footnote{Please cite: Eric Beitz (2000), 
%           \TeX{}shade: 
%           shading and labeling multiple sequence alignments using \LaTeXe. 
%           \textit{Bioinformatics}: \textbf{16}, 135--139.}\\[2mm] \large
%           Typesetting \\ nucleotide and peptide alignments}
%    \author{Eric Beitz\footnote{University of Kiel, 
%            Pharmaceutical Chemistry, Gutenbergstrasse 8, 
%            D-24118 Kiel, Germany; 
%            send electronic mail to \texttt{ebeitz@pharmazie.uni-kiel.de};
%            for further information, updates and on-line documentation 
%            see my homepage at
%            \texttt{www.pharmazie.uni-kiel.de/chem/Prof\_Beitz/biotex.html} }}
%    \date{\small v1.24; 2011/12/01\\[2pt]}
%    \maketitle
%    \begin{abstract}
%      Setting alignments of nucleotides and peptides for publication 
%      or presentation purposes is usually a time consuming two-step process. 
%      First, a scientific software is used for the calculation of the
%      alignment. This
%      is done in a few minutes. Then, in order to highlight special sequence
%      relationships and to label positions and regions of interest a 
%      second software with high quality output capability is needed.
%      Manipulating sequence alignments with standard word processing
%      or graphics programs takes its time---often several hours---and 
%      simple layout changes such as
%      re-breaking lines, say from 50 to 40 residues per line, 
%      elongate the working time considerably.
%      
%      \TeXshade{} is an alignment shading software  
%      written in \TeX/\LaTeX{} which can process
%      multiple sequence alignments in the MSF, ALN 
%      and FASTA file format.
%      It provides in addition to common shading algorithms special
%      shading modes featuring functional aspects, e.\,g.\ charge or
%      hydropathy, and a plenitude of commands for handling
%      shading colors, text styles, labels, legends and even allows 
%      the user to define completely new shading modes. \TeXshade{} 
%      combines highest flexibility and the habitual \TeX{} output 
%      quality---with reasonable time expenditure.
%      
%    \end{abstract}
%        
%    \thispagestyle{empty}
%    
%    \tableofcontents
%    \newpage
%    
%    \section{Package Overview}
%    
%    \label{over}
%   
%    After |texshade.ins| is run through \TeX{} the following files
%    should appear in the directory:
%    
%    \begin{tabbing}
%    \quad|texshade.sty|\quad\= the style file with all \TeXshade{} 
%                                      commands\\
%    \quad|texshade.def|\>      an example parameter file with the 
%                                      standard \\
%                              \>      parameter settings\\
%    \quad|AQPDNA.MSF|  \>      an example nucleotide alignment
%                                      (MSF-format)\\
%    \quad|AQPpro.MSF|  \>      an example protein alignment
%                                      (MSF-format)\\
%    \quad|AQP_TC.asc|  \>      an example T-Coffee shading file
%                                      (|score_ascii|-format)\\
%    \quad|AQP2spec.ALN|\>      a further protein alignment
%                                      (minimal ALN-file)\\
%    \quad|AQP1.phd|\>          secondary structure information
%                                      (PHD-format)\\
%    \quad|AQP1.top|\>          topology data extracted 
%                                      from |AQP1.phd|\\
%    \quad|AQP1_HMM.sgl|\>      topology information (single line,
%                                      HMMTOP-format)\\
%    \quad|AQP1_HMM.ext|\>      topology information (extended,
%                                      HMMTOP-format)\\
%    \quad|standard.cod|\>      standard genetic code definitions\\
%    \quad|ciliate.cod|\>       ciliate macronuclear genetic code\\
%    \end{tabbing}
%    The alignment file examples as well as the topology data file are 
%    needed for \TeX{}ing this documentation
%    and can serve as illustrations for the MSF and ALN 
%    file format.
%    
%    The following subsections give an overview on the capabilities of
%    the \TeXshade{} package. All commands are described in detail 
%    later on.
%    
%    
%    \subsection{Version History}
%
%    \textbf{v1.24  2011/12/01}
%    \medskip
%
%    \emph{Introductions:} 
%    The space between the ruler and the top or bottom sequence row
%    can now be adjusted using |\rulerspace| and the textsize of the 
%    ruler numbering can be set with |\setsize| or |\rulerLarge| etc. 
%    Postscript color samples are shown in the manual next to the 
%    color names list.
%    \medskip
%
%    \textbf{v1.23  2011/05/13}
%    \medskip
%
%    \emph{Introductions:} 
%    In diverse mode sequence positions where all residues match do not
%    contain much information. A new command, 
%    |\hideallmatchpositions|, \footnote{Requested by Matt Russell.}
%    will remove all such positions from the alignment and hence condense
%    the output considerably.
%    \medskip
%
%    \textbf{v1.22  2010/10/11}
%    \medskip
%
%    \emph{Introductions:} 
%    Sets of shading colors can be defined and named using
%    |\defshadingcolors| (p.\ \pageref{Ldefshadingcolors}) and different
%    sets of colors can be selected for different sequence blocks by 
%    |\changeshadingcolors| (\ref{Lchangeshadingcolors}).
%    \medskip
%
%
%    \textbf{v1.21 2010/03/01}
%    \medskip
%
%    \emph{Corrections:} |\setdomain| now recognizes a changed |\startnumber|. 
%
%    \emph{Introductions:} 
%    (a) Sequence motifs (\ref{Lshaderegion}) can be given as a selection pattern
%    in sequence stretch labeling commands, such as |\shaderegion|,
%    |\emphregion| or |\frameblock|, instead of numeral stretch definitions.
%    
%    (b) A new type of sequence emphasis by lowercase letters was implemented: 
%    |\lowerregion| and |\lowerblock| (\ref{Llowerregion}).
%    \medskip
%
%    \textbf{v1.20a 2009/11/30}
%    \medskip
%
%    \emph{Corrections:}  Identity/similarity tables
%    work now without having to set a conservation plot in the feature
%    line. 
%    \medskip
%
%
%    \textbf{v1.20 2009/10/05}
%    \medskip
%
%    \emph{Corrections:}  Landscape mode works now correctly.
%    Annoying |Overfull \hbox ...| signals appearing after
%    |\residuesperline*| commands are ignored;
%    domains work now with frequency corrected sequence logos.
%
%    \emph{Introductions:} 
%    (a) New commands for setting residue weight tables were introduced 
%    |\weighttable| (with parameters: |identity|, |structural|,
%    |PAM250|, |PAM100|, and |BLOSUM62|),
%    |\gappenalty|, and |\setweight|.
%    (b) Similarity/identity percentage tables can be printed with
%    |\similaritytable|\footnote{Asked for by Giovanni M.\ Lesa.} and  
%    values for specific sequence pairs can be utilized with 
%    |\percentsimilarity{|\meta{seqref1}|}{|\meta{seqref2}|}| and 
%    |\percentidentity{|\meta{seqref1}|}{|\meta{seqref2}|}|.
%    \medskip  
%
%    \textbf{v1.19a 2009/06/11}
%    \medskip
%
%    \emph{Correction:} Gaps are now treated correctly when setting 
%    domains.
%    
%    \emph{Introduction:}
%    Stop-positions in protein sequences (|*| in the input file) can be
%    indicated by a special character (|\stopchar|) in the alignment 
%    output.\footnote{Suggestion by Yun He.}
%    \medskip  
%
%    \newpage
%
%    \textbf{v1.19 2009/03/09}
%    \medskip
%
%    \emph{Correction:} Logo characters are now horizontally 
%    scalable using |\charstretch|. Minor bugs were fixed.
%    
%    \emph{Introduction:}
%    (a) Selection of residue positions was enhanced (\ref{Lshaderegion}) 
%    \TeXshade{} can
%    now select residues based on their 3D coordinates provided in a 
%    PDB structure file. 3D selection can be due to a certain distance 
%    around a point, along a line, or above and below a plane. It
%    works with |\feature|, |\shaderegion|, |\shadeblock|, |\tintregion|,
%    |\tintblock|, |\emphregion|, |\emphblock|, |\frameblock|.
%    (b) The list of selected residues is printable with |\printPDBlist| or 
%    viewable during the \TeX{} run with |\messagePDBlist|.
%    (c) |\hideblock| and related commands were replaced by |\setdomain|
%    (\ref{Lsetdomain}). 
%    This command will display only selected residues in the alignment.
%    Thickness and colors of a domain separator rule can be set using
%    |\domaingaprule| and |\domaingapcolors|.
%    \bigskip
%
%    \textbf{v1.18 2008/04/15}
%    \medskip
%
%    \emph{Correction:} several bug fixed concerning featurename
%    display, sequence ordering and numbering.
%    
%    \emph{Introduction:}
%    (a) T-Coffee shading information can be loaded and put on the
%    alignment.\footnote{Suggestion by Florian Mertes.} 
%    The conservation data can also be displayed in the 
%    consensus as well as feature color scales and bar plots.
%    (b) Two more feature lines were added on the top and at the
%    bottom (|ttttop|, |tttop|, |bbbottom|, |bbbbottom|).
%    (c) The |startnumber| and |setends| commands have been fused;
%    either command can set both, a new start number as well as
%    end definitions of the sequence section to be displayed.
%    \bigskip
%
%    \textbf{v1.17 2007/06/19}
%    \medskip
%    
%    \emph{Introduction:}\footnote{Asked for by Marat Kazanov.}
%    (a) A second threshold percentage was introduced in order to label
%    two levels of conservation in `identical' and `similar' mode. This is 
%    achieved by setting an optional parameter in |\threshold| or in 
%    |\allmatchspecial|, or by using a number as an optional parameter in 
%    |\shadingmode|. (b) The feature lines can be additionally labeled with
%    a name left or right of the feature. This is handled using
%    |\showfeaturename|, |\showfeaturestylename|,
%    |\hidefeaturename|, |\hidefeaturestylename|,
%    |\hidefeaturenames|, |\hidefeaturestylenames|.
%    The color of such names can be changed with
%    |\featurenamecolor|, |\featurestylenamecolor|,
%    |\featurenamescolor|, |\featurestylenamescolor|,
%    Font styles in feature names can be set as usual, e.g.
%    |\setsize{featurenames}{large}| or
%    |\featurestylenamesrm|.
%    \bigskip
%
%
%    \textbf{v1.16 2007/02/18}
%    \medskip
%    
%    \emph{Correction:} \TeXshade{} crashed when calculating conservation
%    using sequences with untypical residue characters, such as "X".
%    Fixed. The reference sequence in diverse mode can now be shaded with
%    |\conservedresidues| and, if active, |\allmatchresidues|.\footnote{For this
%    and suggesting |namerulerpos| credit to Marco Pasi.}
%
%    \emph{Introduction:}\footnote{Both extensions were suggested by Phillip Hahn.}
%    (a) A command was introduced, i.e. |\exportconsensus| 
%    which produces a pymol script file for coloring a 3D model according to 
%    \TeXshade's conservation calculation. (b) With |namerulerpos| labels of the
%    ruler can be exchanged by a string. (c) Various parts of the alignment 
%    can now be hidden by |\hideblock|. 
%
%    \emph{New home:} \TeXshade, \TeXtopo, and \BioTeX{} have a new home:
%    |www.pharmazie.uni-kiel.de/chem/Prof_Beitz/biotex.html|.
%    \bigskip
%
%
%    \textbf{v1.15 2006/06/27}
%    \medskip
%    
%    \emph{Correction:} Sequence and subfamily logos can now be plotted
%    with pdflatex; pstricks is not needed anymore.
%    \bigskip
%
%
%    \textbf{v1.14 2006/05/11}
%    \medskip
%    
%    \emph{Introduction:} In order to better recognize relevant positions
%    in a subfamily logo [14], a bit-value can now be set by |\relevance|
%    above which a deviation is considered relevant. Such positions
%    can be labeled with a symbol by |\showrelevance| and hidden
%    by |\hiderelevance|.
%    \bigskip
%
%
%    \textbf{v1.13 2006/02/23}
%    \medskip
%    
%    \emph{Corrections:} Helix symbols in feature lines were not drawn
%    correctly if the standard Computer Modern Font was changed to 
%    another one, e.g. Palatino.\footnote{Thanks to Markus Heller} 
%    Fixed. Unintended gaps occurred due to numbers at the
%    end of lines in Clustal W alignment files. Fixed. The limitations
%    in the number of sequences per alignment have finally been overcome
%    by a more restrictive use of counter variables.
%
%    \emph{Introductions:} (a) The numbering can now be displayed---in 
%    addition to left or right---on both sides of the alignment with 
%    the optional parameter |{leftright}| in the |\shownumbering| 
%    command (p.\pageref{Lshownumbering}). (b) TeXshade tries to guess 
%    the sequence type, i.\,e.\ protein or nucleotide, if not defined 
%    by the user. (c) Plotting of sequence logos has been implemented
%    (p.\pageref{Lshowsequencelogo}).
%    Logos can be shown in addition to or together with the consensus,
%    or alone without any alignment sequences. (d) The ruler numbering
%    can be rotated in order to make labeling of every position possible.
%    (e) A new way to visualize subfamily characteristics has been
%    implemented, i.e. subfamily logos (p.\pageref{Lshowsubfamilylogo}) [14].
%    \bigskip
%
%
%    \textbf{v1.12 2005/09/20}
%    \medskip
%    
%    \emph{Corrections:} When regional labeling with |\shaderegion|,
%    |\emphregion|, |\tintregion|, or |\frameblock| was combined with
%    |\setends| incorrect output was produced lacking the
%    labeling.\footnote{Discovered by Chris Page.} Other minor fixes.
%
%    \emph{Introductions:} An additional optional parameter for setting 
%    consensus colors was implemented in the |\showconsensus| command
%    (p.\pageref{Lshowconsensus}). This even allows one to use color 
%    scales illustrating sequence conservation in the consensus line.
%    \bigskip
%
%    \textbf{v1.11 2005/04/13}
%    \medskip
%    
%    \emph{Corrections:} Bounding boxes with |\frameblock| had a wrong
%    height when |\separationline|s were used. Other minor fixes.
%
%    \emph{Introductions:} (a) An additional parameter for setting 
%    individual bar and arrow thicknesses in feature lines has been 
%    introduced. (b) Additional parameters for setting the frame color
%    and thickness of boxes in feature lines have been implemented. (c)
%    Three more color scales have been defined: |RedBlue|, |RedGreen|,
%    and |HotCold|. (d) Plotting of amino acid features (|hydrophobicity|,
%    |molweight|, |charge|) as bar graphs or color scales. (e) Plotting
%    of protein sequence |conservation| as bar graph or color
%    scale\footnote{Ahmad Mirza asked for (e) and (f), great suggestion!}.
%    (f) Color scales can be used for shading the consensus sequence
%    according to protein sequence conservation.
%    (g) Separate command for stretching color scales |\colorscalestretch|.
%    \bigskip
%    
%    \textbf{v1.10 2005/03/29}
%    \medskip
%    
%    \emph{Corrections:} Plotting of color scales and bar graphs has
%    been sped up by more than a factor of 10.\footnote{This and (d)
%    I owe again to Christoph Gille.}
%
%    \emph{Introductions:} (a) More colors have been introduced, i.e. 
%    even lighter versions of the existing PostScript colors 
%    `LightLight' plus color name and `LightLightLight' plus color 
%    name. (b) Sequence stretches and blocks can be tinted for
%    labeling purposes |\tintreqion|, |\tintblock| and |\tintdefault|. 
%    (c) A new feature label style |{restriction}| has been introduced. 
%    (d) Java-typical `NaN' values are now allowed in data files for
%    bar graphs and color scales.
%    \bigskip
%    
%    
%
%    \textbf{v1.9 2005/02/08}
%    \medskip
%    
%    \emph{Corrections:} \TeXshade{} version 1.8 introduced an
%    incompatibility with \TeXtopo{}. This problem was identified
%    by Meike Schmedt and has been fixed.
%
%    \emph{Introductions:} (a) A short version of the figure caption
%    can now be defined for display in the list of figures\footnote{%
%    Meike, here you go \dots} |\shortcaption{|\meta{text}|}|. (b) A
%    colored frame can be drawn around a sequence block for labeling
%    purposes with the command |\frameblock|.\footnote{Alan Robinson, 
%    this is for you.} (c) A new look for feature arrows has been 
%    implemented with scalable line thickness and a new end style
%    `ball'. (d) HMMTOP topology predictions can
%    now be included for plotting feature lines with information on
%    the location of the transmembrane domains.\footnote{Implemented 
%    after a request by Steffen Moeller.} 
%    \bigskip
%
%    \textbf{v1.8 2004/08/26}
%    \medskip
%    
%    \emph{Corrections:} Only minor bugs were fixed.
%
%    \emph{Introductions:} (a) More colors have been designed, i.e. 
%    `light' versions of the existing PostScript colors. (b) 
%    Three color ramps in 5\% steps have been introduced: 
%    i) Blue-Red, ii) Green-Red and iii) Cold-Hot.
%    (c) Two new feature label styles |bar| and |color| have been 
%    introduced which allow one to display number
%    values as bar graphs or color scales along the
%    alignment\footnote{Inspired by Christoph Gille's {\tt STRAP}}. 
%    \bigskip
%
%
%    \textbf{v1.7 2004/01/05}
%    \medskip
%    
%    \emph{Corrections:} Several bugs were fixed.
%    In gaps the wrong character was plotted in `donotshade'
%    mode. Gaps were colored incorrectly when a single 
%    sequence was set as consensus. Another `donotshade' problem was 
%    solved which led to a halt of the LaTeX 
%    run\footnote{Thanks to Jeferson J.\ Arenzon and Naomi Siew}. 
%    Due to several requests, the gap and match labels in |diverse|
%    mode were switched (`|-|' in gaps; `|.|' at matching
%    positions) in order to follow convention.
%
%    \emph{Introduction:} \TeXshade{} speaks spanish (|\spanishlanguage|).
%    Necessary translations were contributed by Mikel Ega\~na Aranguren.
%    A new feature label style |helix| has been introduced.
%    \bigskip
%
%
%    \textbf{v1.6 2002/03/26}
%    \medskip
%    
%    \emph{Corrections:} The unnecessary restriction to the DVIPS
%    driver for |color.sty| has been removed\footnote{As suggested by
%    Eckhart Guth\"ohrlein.}. Any color.sty compatible
%    driver option can be given with the |\usepackage{texshade}| call
%    and is then passed to the |color| package. The `|\namecolor|' and 
%    `|\numbercolor|' commands do now support sequence 
%    lists.\footnote{Thanks to Denys Bashtovyy.}
%        
%    \emph{Introductions:} (a) The FASTA file format is supported by
%    \TeXshade{} as alignment inputs. (b) Two commands set the space
%    between sequence blocks either to be flexible (as so far) 
%    `|\flexblockspace|' or the be fixed `|\fixblockspace|'. (c) One 
%    can now refer to sequences by their name in addition to the number
%    in the input file. (d) Using
%    `|\firstcolumnDSSP|' and `|\secondcolumnDSSP|' one can choose 
%    which of the first to columns should refer to the sequence numbering 
%    (the second column remains default setting)\footnote{c and d were
%    suggested by Christoph Gille.}.
%    \bigskip
%    
%    \textbf{v1.5a 2001/03/08}
%    \medskip
%    
%    \emph{Corrections:} `X's in the alignment file caused a run-time
%    error. Fixed.
%        
%    \emph{Introductions:} (a) The vertical space between feature
%    lines can be controlled by four new commands: |\ttopspace|,
%    |\topspace|, |\bottomspace| and 
%    |\bbottomspace|\footnote{Suggested by Ulrike Folkers.}. (b) It is 
%    now easily possible to add a caption to the alignment with 
%    the |\showcaption| command. (c) \TeXshade{} stores the 
%    sequence lengths in the |.aux| file in order to have correct
%    breaks of the gaps after the sequences.
%    \bigskip
%
%
%    \textbf{v1.4\&4a 2000/9/12 \& 2000/10/3}
%    \medskip
%    
%    \emph{Introductions:} (a) The alignment legend can now be moved 
%    by the command `|\movelegend|'. (b) In commands with parameters 
%    that contain series of sequence numbers, e.\,g. |\orderseqs|, a
%    dash can be used, e.\,g. |{1-3,6-4,7}| instead of 
%    |{1,2,3,6,5,4,7}|.
%    \bigskip
%    
%    \newpage
%
%    \textbf{v1.3a\&b 2000/7/28 \& 2000/7/30}
%    \medskip
%    
%    \emph{Introductions:} (a) It is now possible to force \TeXshade{} to 
%    display gap symbols before and after the actual sequence 
%    by the commands `|\showleadinggaps|' and `|\hideleadinggaps|'
%    (\ref{Lshowleadinggaps}).
%    (b) The sequence names input routine is now more tolerant concerning
%    special characters.
%    \bigskip
%    
%    \textbf{v1.3 2000/3/3}
%    \medskip
%    
%    \emph{Corrections:} Line scrambling occured when features where
%    set in the |ttop| row without a feature in the |top| row. Fixed.
%    The incompatible command `|\language|' with the |babel| package has been
%    replaced by `|\germanlanguage|' and `|\englishlanguage|'\footnote%
%    {Thanks to Eckhart Guth\"ohrlein.}.
%    
%    \emph{Introductions:} (a) Now, translations of sequence stretches
%    are possible. Either nucleotide or amino acid sources can be 
%    translated. This is done by the new |{translate}| option for the 
%    feature command. (b) The codons are defined by the new command
%    `|\codon|'. Complete codon sets can be loaded by `|\geneticcode|'.
%    (c) Further, the size and style of the nucleotide triplets of
%    backtranslations can be set by `|\backtranslabel|' and 
%    `|\backtranstext|'. (d) Two more feature counter styles were introduced:
%    `|\Romancount|' and `|\romancount|'. (e) \TeXshade{} is now 
%    compatible with \TeXtopo, a new \TeX{} software
%    for drawing and shading topology plots of membrane proteins.
%    \bigskip
%    
%    \textbf{v1.2a 1999/6/24 (not released)}
%    \medskip
%    
%    \emph{Minor corrections:} `|\namecolor|' and `|\numbercolor|' are 
%    now really correctly reordered. Brackets ( and ) are now allowed
%    in sequence names. The option |{case}| in `|\funcshadingstyle|' 
%    works now.
%    \bigskip
%    
%
%    \textbf{v1.2 1999/6/12}
%    \medskip
%    
%    \emph{Corrections:} (a) Functional group definitions of more than
%    seven groups produced an error when displaying group number
%    eight. These residues where skipped in the alignment. Fixed.
%    
%    \emph{Introductions:} (a) Protein secondary structure files in the DSSP,
%    STRIDE and PHD format can be included and displayed auto\-matically
%    within the alignment by `|\includeDSSP|' (and similar commands for 
%    STRIDE, PHDsec and PHDtopo, \ref{structure}).
%    (b) Which types of secondary structures are to be included or
%    skipped in the alignment is chosen by `|\showonDSSP|' and 
%    `|\hideonDSSP|' (and respective commands for STRIDE, PHDsec and PHDtopo).
%    (c) The appearance of the labels is defined by `|\appearance|'.
%    (d) Internal counters for repeatedly occuring structure types
%    can be activated by `|\numcount|', `|\alphacount|' and
%    `|\Alphacount|'. All commands are described in \ref{structure}.
%    \bigskip
%        
%
%    \textbf{v1.1 1999/5/26}
%    \medskip
%    
%    \emph{Corrections:} (a) The activation of `|emphregion|' lead to 
%    an em\-pha\-sized following alignment. This has been 
%    corrected. (b) `|\namecolor|' and `|\numbercolor|' were not 
%    reordered with the command `|orderseqs|'. Fixed. (c) Sequence 
%    gaps at the beginning or the end of a sequence, i.\,e. before
%    the first and after the last residue where labeled with the
%    gap symbol. Now these positions are left blank.
%    
%    \emph{Introductions:} (a) In order to treat the preceeding and 
%    sequence following gaps correctly, \TeXshade{} needs to know the
%    length of the sequences. Therefore, the command `|\seqlength|' was
%    introduced (\ref{seqlines}). (b) With `|\gapcolors|' (also 
%    \ref{seqlines}) the
%    color selection for gap symbols is independent from non conserved
%    residues. (c) The divisions of the ruler where so far fixed to
%    10. Now, this value is changeable by `|\rulersteps|' (again 
%    \ref{seqlines}). (d) `|\hideresidues|' and `|\showresidues|' turn
%    off or on the residue names, i.\,e. one can choose between a
%    display of shaded boxes only or with letters in the boxes 
%    (\ref{kill}). (e) The changes (c) through (d) were necessary 
%    for the introduction of `|\fingerprint|'. This command allows one to
%    display the complete sequence in one line for an easy survey of
%    the alignment (\ref{fingerprint}).
%    \bigskip
%    
%    \textbf{v1.0 1999/5/12}
%    \medskip
%    
%    First release.
%    \bigskip
%    
%    
%    \subsection{\LaTeX{} basics}
%    
%    \subsubsection{Typesetting documents using \LaTeX}
%    
%    In order to use any of the macros provided by the 
%    \BioTeX-project 
%    (\TeXshade/\TeXtopo) efficiently a basic understanding of the \TeX{} 
%    typesetting system and its usage is required. Several books are
%    available on this topic, but a rather quick and easy introduction
%    is the \emph{Not so short introduction to \LaTeX}. This document
%    is available from all Comprehensive \TeX{} Archive Network 
%    (CTAN) servers,
%    e.\,g. from |ftp://ftp.dante.de/pub/tex/documentation/lshort/|,
%    in many different languages and formats besides \LaTeX{}, such 
%    as \textsc{PostScript} and on-line viewable PDF.
%    I also put a link from the \BioTeX{} (\TeXshade/\TeXtopo) homepage 
%    to the document collection
%    (|http://pharmazie.uni-kiel.de/chem/Prof_Beitz/BioTeX|).
%    
%    
%    \subsubsection{Memory shortness when using \TeX{}shade}
%    
%    If you are using \TeXshade{} to align several large sequences (about 1000
%    residues/sequence), LaTeX will probably stop compiling and quit with one
%    of the following messages: 
%    
%    |!\ TeX capacity exceeded, sorry [main memory size=384000]| 
%    
%    or 
%    
%    |!\ TeX capacity exceeded, sorry [stack size=300]|.
%    
%    \TeX{} allocates space for different kinds of internal variables. 
%    Setting alignments needs lots of memory,
%    usually more than for typesetting plain text.
%    Thus, the parameter settings of a standard \TeX{} installation might not
%    be sufficient for certain projects. This manifests
%    in \TeX{} error messages about insufficient memory  
%    and the setting process is interrupted. There is no reason to be 
%    concerned. The parameters can be set by hand. Unfortunately, 
%    each \TeX{} system hides its default parameter file in a different
%    place in the system. 
%    
%    In the following, an excerpt from the FAQ-list to \TeXshade{} is added. 
%    This explains how 
%    to increase the settings in Oz\TeX{} for the Macintosh, Mik\TeX{} 
%    for Windows and te\TeX{} for *NIX \TeX{} distributions. Please contribute 
%    to this list!
%    
%    \begin{enumerate}
%    
%    \item
%    
%    \textbf{Oz\TeX{} 4.0 for the Macintosh:}
%    
%    Find the file `OzTeX:TeX:Configs:Default'. This file contains
%    all memory settings. Look for the section 
%    `\% TeX parameters' and increase the values that \TeX{} complains
%    about during the run. You will have to restart Oz\TeX{} before the
%    changes are active.
%    
%    For older versions of Oz\TeX{} the configuration file has the 
%    same name but the path is somewhat different.
%    
%    
%    \item
%    
%    \textbf{te\TeX{} for *NIX:} (contributed by Joerg Daehn)
%    
%    Find the file: `/usr/share/texmf/web2c/texmf.cnf' or use 
%    
%    |locate texmf.cnf| at the command prompt to find it.
%
%    Login as super user. Backup `texmf.cnf' in case you destroy something and
%    then open the `texmf.cnf' file in your favorite text editor and use its
%    search function to locate |main_memory|. This variable is set to 384000.
%    Change this to some higher value, i.e. 4000000 (works fine for me!). The
%    total amount of memory should not exceed 8000000, so check the other
%    values in that section. 
%
%    Next, you want to change the stack size. Search for |stack_size|. This
%    will be set to 300. I changed it to 4000 and it works fine.
%
%    There might be complains by \TeX{} about further specific parameters such
%    as |stack_size|. You find all those in the same file.
%
%    After this you have to run `texconfig init'.
%
%    Logout as root.
%
%    After this all should be set for large alignments. Happy \TeX{}ing!
%
%    The information on how to achieve this was derived from a mail in the
%    te\TeX{} mail archive. The original question was posted by Pascal Francq and
%    answered by Rolf Nieprasch.
%
%    
%    \item
%    
%    \textbf{MiK\TeX{} for Windows:}
%    
%    The MiK\TeX{} documentation describes very detailed how the memory
%    settings can be changed. In brief, you must locate the 
%    configuration file `miktex/config/miktex.ini'. In the [MiKTeX] 
%    section of this file you find all the parameters you need, e.\,g.\ 
%    |mem_min|, |mem_max|, |buf_size|, |stack_size| etc.
%    
%    It appears, that the standard settings of MiK\TeX{} are bigger 
%    than that of other \TeX{} installations, so it may not always be necessary 
%    to increase the values.
%    
%    
%    \end{enumerate}
%    
%    
%
%    \subsection{System requirements} \label{require}
%    
%    \TeXshade{} requires \LaTeXe{} with |color.sty| and |graphics.sty|
%    for shading. For arrows in the feature line (p.\pageref{Lfeature})  
%    the AMS Math style is needed.
%    David Carlisle's |color.sty| is part of the Standard \LaTeX{}
%    `Graphics Bundle' [1]. This and the other packages can be downloaded 
%    from any \TeX{} archive, e.g.\ |ftp.dante.de|; usually they are 
%    included in a comprehensive \TeX{} installation. 
%
%    The |color| style allows one to use several |[|\meta{options}|]|, e.\,g.
%    |dvips|, |pdftex| or |dviwin|. These provide the commands which
%    different devices/programs need to display colored output. It is 
%    advisable to make yourself familiar with the |color.sty| manual. 
%    You should define a default driver in the file |color.cfg|.
%    Since there is no direct call of |color.sty| by the user, the 
%    option can be stated when \TeXshade{} is loaded, see next 
%    subsection. If no option is stated the |DVIPS| driver will be 
%    loaded. 
%     
%    With the |[dvips]| option the output DVI-file 
%    can be converted to \textsc{PostScript} using the |DVIPS| program
%    and can later be viewed or printed with the public domain 
%    {\sc GhostView} program which is
%    available for almost all computer platforms. Further, more and more
%    standard \TeX{} viewers are to a certain extent \textsc{PostScript} 
%    compatible. 
%    \bigskip
%    
%    \subsection{The \texttt{texshade} environment}
%    
%    \label{tsenvironment}
%    
%    The commands provided by the \TeXshade{} package are enabled by
%    the following command in the document header section:
%    \medskip
%    
%    \quad |\usepackage[|\meta{option}|]{texshade}|
%    
%    \medskip
%    Make sure that the file `|texshade.sty|' is present in a directory
%    searched by \TeX{} (see the installation notes in the file
%    `|texshade.txt|').
%    
%    The \meta{option} given here is passed to |color.sty| which 
%    handles the color commands for a particular output device, see 
%    previous subsection and the |color.sty| manual.
%    
%    The \TeXshade{} package provides only one single new environment:
%    |texshade|. This environment has one mandatory and
%    one optional argument, both of them designating file names which
%    must be present in a directory searched by \TeX. The
%    required file \meta{alignmentfile} contains the aligned nucleotide 
%    or peptide sequences
%    (see section~\ref{alignfilestruc}). This file is needed, because 
%    \TeXshade{} does no alignment by 
%    itself, it has to take a preprocessed alignment as input.
%    The optional file is a parameter file (section~\ref{paramfilestruc})
%    with definitions for the
%    customized calculation of the consensus, special sequence features
%    or labels etc. In this parameter file all \TeXshade{} commands 
%    which are allowed in the |texshade| environment can be used and are 
%    fully functional.
%    Within the environment further \TeXshade{} commands can be given
%    to replace or complete settings from the parameter file.
%    
%    Thus, setting an alignment with \TeXshade{} is as simple as
%    this:
%    
%    \begin{quote}
%    |\begin{texshade}[|\meta{parameterfile}|]|
%         |{|\meta{alignmentfile}|}|
%         
%    \quad\emph{further \emph{\TeXshade} commands, if needed}
%    
%    |\end{texshade}|
%    \end{quote}
%    
%    \subsection{Shading modes predefined in this package}
%    
%    \subsubsection{Identity mode}
%    
%    \label{ident}
%     
%    This basic type of shading is provided by almost any alignment
%    program. All identical residues at a position are shaded if the
%    number of matching residues is higher than a given threshold
%    (default is 50\%).\medskip
%    
%    \begin{texshade}{AQPpro.MSF}
%       \setends{1}{80..112}
%       \hideconsensus
%    \end{texshade}
%
%    Code:\medskip
%
%    \vbox{%
%    \begin{verbatim}    
%    \begin{texshade}{AQPpro.MSF}
%       \setends{1}{80..112}
%       \hideconsensus
%    \end{texshade}
%    \end{verbatim}  
%    }  
%    
%    Quite uncommon for an alignment shading program is the possibility
%    to display only selected sequence domains, e.\,g.\ to eliminate uninteresting
%    positions from the output:
%    \medskip
%    
%    \begin{texshade}{AQPpro.MSF}
%       \setdomain{1}{80..90,100..110,120..130}
%       \showruler{1}{top}
%       \hidenumbering
%       \hideconsensus
%    \end{texshade}
%
%    Code:\medskip
%
%    \vbox{%
%    \begin{verbatim}    
%    \begin{texshade}{AQPpro.MSF}
%       \setdomain{1}{80..90,100..110,120..130}
%       \showruler{1}{top}
%       \hidenumbering
%       \hideconsensus
%    \end{texshade}
%    \end{verbatim}  
%    }  
%    
%    This goes even furher. You can have \TeXshade{} select positions
%    based on the 3D coordinates provided by a PDB file, e.\,g.\ show
%    all residues that are within an 8 \AA{} radius around the
%    $\alpha$-carbon of the residue at position 81:
%    \bigskip
%    
%    \begin{texshade}{AQPpro.MSF}
%       \setdomain{1}{75..86,103..103,148..148,151..153,155..156,193..195,218..219,222..222}
%       \showruler{top}{1} \rulersteps{1}
%       \hidenumbering
%       \hideconsensus
%    \end{texshade}
%
%    Code:\medskip
%
%    \vbox{%
%    \begin{verbatim}    
%    \begin{texshade}{AQPpro.MSF}
%       \setdomain{1}{point[8]:1J4N.pdb,81[CA]}
%       \showruler{top}{1} \rulersteps{1}
%       \hidenumbering
%       \hideconsensus
%    \end{texshade}
%    \end{verbatim}  
%    }  
%    
%    If you like, positions where conservation is very high (here $\ge$ 80\%) can
%    be shaded in a special color and the consensus can be shown with
%    or without shading according to the degree of conservation:
%    \medskip\label{shadecons}
%    
%    \begin{texshade}{AQPpro.MSF}
%       \threshold[80]{50}
%       \setends{1}{80..112}
%       \showconsensus[ColdHot]{bottom}
%       \defconsensus{.}{lower}{upper}
%       \showlegend
%    \end{texshade}
%    
%    Code:\medskip
%
%    \vbox{%
%    \begin{verbatim}    
%    \begin{texshade}{AQPpro.MSF}
%       \threshold[80]{50}
%       \setends{1}{80..112}
%       \showconsensus[ColdHot]{bottom}
%       \defconsensus{.}{lower}{upper}
%       \showlegend
%    \end{texshade}
%    \end{verbatim}}
%    
%    \subsubsection{Similarity mode}
%    
%    \label{similar}
%    
%    In many cases it is expedient---mostly when comparing protein
%    sequences---to shade also residues
%    which are not identical but similar to the consensus sequence.
%    Consider a position where three out of five residues are basic
%    arginines and two more residues are also basic but lysines.
%    In similarity mode \TeXshade{} shades similar residues in a different 
%    color to distinguish them from the consensus residue. Even when 
%    none of the residues alone reaches the 
%    threshold but a group of similar residues does these are shaded
%    in the `similarity' color. This case is given for instance
%    when at a position in a five sequence alignment two aliphatic 
%    valines and two also aliphatic isoleucins are present and the 
%    threshold is set to 50\%. Neither residue exceeds this percentage 
%    but as a group of similars they do. 
%    
%    In grayscale printouts some colors of the following alignment may appear 
%    undistinguishable. Don't worry if you usually use grayscale---all 
%    colors/grays can be selected freely (see \ref{colors}).
%    \medskip 
%    
%    \begin{texshade}{AQPpro.MSF}
%       \shadingmode{similar}
%       \threshold[80]{50}
%       \setends{1}{80..112}
%       \hideconsensus
%       \feature{top}{1}{93..93}{fill:$\downarrow$}{first case (see text)}
%       \feature{bottom}{1}{98..98}{fill:$\uparrow$}{second case (see text)}
%    \end{texshade}    
%    
%    Code:\medskip
%
%    \vbox{%
%    \begin{verbatim}    
%    \begin{texshade}{AQPpro.MSF}
%       \shadingmode{similar}
%       \threshold[80]{50}
%       \setends{1}{80..112}
%       \hideconsensus
%       \feature{top}{1}{93..93}{fill:$\downarrow$}{first case (see text)}
%       \feature{bottom}{1}{98..98}{fill:$\uparrow$}{second case (see text)}
%    \end{texshade}    
%    \end{verbatim}}
%    
%    Probably you know 
%    this kind of shading from the public domain program 
%    |BoxShade| 
%    by \textsc{Kay Hofmann} or from the Macintosh version 
%    |MacBoxShade| by \textsc{Michael D. Barron}. \TeXshade{}
%    provides the same functionality---and goes truly beyond---for the 
%    \TeX{} community.
%
%    
%    \subsubsection{T-Coffee shading}
%    
%    \label{TCoffee}
%    
%    \TeXshade{}'s capabilities of calculating alignment shadings are 
%    limited. |T-Coffee| (|www.tcoffee.org|) is a sophisticated alignment/shading 
%    software. You can apply shading from |T-Coffee| in \TeXshade{}
%    by loading the shading information file (|score_ascii|) generated by
%    |T-Coffee|.
%    \medskip 
%    
%
%    \begin{texshade}{AQPpro.MSF}
%       \shadingmode[AQP_TC.asc]{T-Coffee}
%       \setends{1}{30..63}
%       \feature{top}{1}{30..63}{color:conservation[T-Coffee]}{}
%       \showfeaturestylename{top}{feat-cons}
%       \showconsensus{bottom}
%    \end{texshade}    
%    
%    Code:\medskip
%
%    \vbox{%
%    \begin{verbatim}    
%    \begin{texshade}{AQPpro.MSF}
%       \shadingmode[AQP_TC.asc]{T-Coffee}
%       \setends{1}{30..63}
%       \feature{top}{1}{30..63}{color:conservation[T-Coffee]}{}
%       \showfeaturestylename{top}{feat-cons}
%       \showconsensus{bottom}
%    \end{texshade}    
%    \end{verbatim}}
%    
%    
%    
%    
%    
%    
%    
%    \subsubsection{Diversity mode}
%    
%    \label{diverse}
%    
%    Contrary to the above described modes this shading style displays
%    sequence differences. Thus, it is most suitable for comparing very
%    similar sequences, e.\,g.\ species variants of a protein.
%    
%    One sequence is used as consensus. 
%    Matching residues in other sequences are blanked out, 
%    mismatches are shown in lowercase.
%    \medskip
%        
%    \begin{texshade}{AQP2spec.ALN}
%       \shadingmode{diverse}
%       \setends{1}{77..109} \residuesperline*{33}
%       \featureslarge
%       \feature{top}{1}{77..109}{}{AQP2 species variants}
%       \namesrm\namessl
%       \hidenumbering
%       \showruler{top}{1}
%       \shownames{left}
%       \nameseq{1}{Bos taurus}
%       \nameseq{2}{Canis familiaris}
%       \nameseq{3}{Dugong dugong}
%       \nameseq{4}{Equus caballus}
%       \nameseq{5}{Elephas maximus}
%       \frameblock{1}{82..82,106..106}{Red[1pt]}
%    \end{texshade}\label{frame}
%    
%    Code:\medskip
%
%    \vbox{%
%    \begin{verbatim}    
%    \begin{texshade}{AQP2spec.ALN}
%       \shadingmode{diverse}
%       \setends{1}{77..109}
%       \featureslarge
%       \feature{top}{1}{77..109}{}{AQP2 species variants}
%       \namesrm\namessl
%       \hidenumbering\showruler{top}{1}
%       \shownames{left}
%       \nameseq{1}{Bos taurus}
%       \nameseq{2}{Canis familiaris}
%       \nameseq{3}{Dugong dugong}
%       \nameseq{4}{Equus caballus}
%       \nameseq{5}{Elephas maximus}
%       \frameblock{1}{82..82,106..106}{Red[1pt]}
%    \end{texshade}\label{frame}
%    \end{verbatim}}   
%    
%    
%    \subsubsection{Functionality modes}
%    
%    \label{func}
%    
%    Displaying functional peptide similarities is one of \TeXshade's
%    strong capabilities. Six functional shading modes are predefined;
%    further user specific modes can easily be created. The examples 
%    may not look very impressive when printed in grayscale. Enjoy 
%    them on your screen or use color printouts. As mentioned before,
%    all colors can be changed to others or to grays without restrictions
%    (see chapter \ref{colors}).
%    
%    \begin{itemize}
%    \item [\textbf{charge}:] residues which are charged at physiological pH
%                    (7.4) are shaded if their number at a position
%                    is higher than the threshold \label{charge}
%                    
%    \begin{texshade}{AQPpro.MSF}
%       \residuesperline*{34}
%       \shadingmode[charge]{functional}
%       \setends{1}{138..170}
%       \feature{top}{3}{153..165}{bar[-50,50]:-50,-45,%
%          -40,-30,-20,-10,0,10,20,30,40,45,50}{}
%       \feature{top}{3}{167..186}{color:5,10,15,20,25,30,35,%
%          40,45,50,55,60,65,70,75,80,85,90,95,100[ColdHot]}{}
%       \showlegend
%    \end{texshade}
%    
%    Code:\medskip
%
%    \vbox{%
%    \begin{verbatim}    
%    \begin{texshade}{AQPpro.MSF}
%       \shadingmode[charge]{functional}
%       \setends{1}{138..170}
%       \feature{top}{3}{153..165}{bar[-50,50]:-50,-45,%
%          -40,-30,-20,-10,0,10,20,30,40,45,50}{}
%       \feature{top}{3}{167..186}{color:5,10,15,20,25,30,35,%
%          40,45,50,55,60,65,70,75,80,85,90,95,100[ColdHot]}{}
%       \showlegend
%    \end{texshade}
%    \end{verbatim}}
%    
%    \item [\textbf{hydropathy}:] discrimination between acidic and
%                        basic, polar uncharged and hydrophobic nonpolar residues
%                        \label{hydro}
%    \bigskip    
%    
%    \begin{texshade}{AQPpro.MSF}
%       \residuesperline*{34}
%       \shadingmode[hydropathy]{functional}
%       \feature{top}{1}{158..163}{brace}{tinted}
%       \tintblock{1}{158..163}
%       \feature{top}{1}{QLVLC}{brace}{lowercased}
%       \lowerblock{1}{QLVLC}
%       \setends{1}{138..170}
%       \showlegend
%    \end{texshade}
%    
%    Code:\medskip
%
%    \vbox{%
%    \begin{verbatim}    
%    \begin{texshade}{AQPpro.MSF}
%       \shadingmode[hydropathy]{functional}
%       \feature{top}{1}{158..163}{brace}{tinted}
%       \tintblock{1}{158..163}
%       \feature{top}{1}{QLVLC}{brace}{lowercased}
%       \lowerblock{1}{QLVLC}
%       \setends{1}{138..170}
%       \showlegend
%    \end{texshade}
%    \end{verbatim}}    
%    
%    
%    \item [\textbf{structure}:] displays the potential
%                        localization within the tertiary structure of
%                        the protein \label{struc}
%    \bigskip    
%    
%    \begin{texshade}{AQPpro.MSF}
%       \residuesperline*{34}
%       \shadingmode[structure]{functional}
%       \setends{1}{138..170}
%       \feature{top}{1}{138..157}{box[Blue,Red][0.5pt]: %
%           $\alpha$-helix[Yellow]}{transmembrane domain 4}
%       \feature{top}{1}{158..163}{translate[Blue]}{}
%       \backtranslabel{oblique}
%       \feature{bottom}{1}{[DE]RXXR[DE]}{brace[Blue]}{loop D [Blue]}
%       \feature{top}{1}{164..170}{o->[Red]}{trans. dom. 5}
%       \showlegend
%    \end{texshade}
%          
%    Code:\medskip
%
%    \vbox{%
%    \begin{verbatim}    
%    \begin{texshade}{AQPpro.MSF}
%       \shadingmode[structure]{functional}
%       \setends{1}{138..170}
%       \feature{top}{1}{138..157}{box[Blue,Red][0.5pt]: %
%           $\alpha$-helix[Yellow]}{transmembrane domain 4}
%       \feature{top}{1}{158..163}{translate[Blue]}{}
%       \backtranslabel{oblique}
%       \feature{bottom}{1}{[DE]RXXR[DE]}{brace[Blue]}{loop D [Blue]}
%       \feature{top}{1}{164..170}{o->[Red]}{trans. dom. 5}
%       \showlegend
%    \end{texshade}
%    \end{verbatim}}   
%          
%    
%    \item [\textbf{chemical}:] residues are shaded due to chemical 
%                      properties of
%                      their functional groups \label{chem}
%    \bigskip    
%                      
%    \begin{texshade}{AQPpro.MSF}
%       \residuesperline*{34}
%       \shadingmode[chemical]{functional}
%       \setends{1}{138..170}
%       \showlegend
%    \end{texshade}
%          
%    Code:\medskip
%
%    \vbox{%
%    \begin{verbatim}    
%    \begin{texshade}{AQPpro.MSF}
%       \shadingmode[chemical]{functional}
%       \setends{1}{138..170}
%       \showlegend
%    \end{texshade}
%    \end{verbatim}}    
%          
%          With |\shadeallresidues| the threshold is ignored and
%          all residues are shaded due to their group assignment.
%          This is \emph{not} identical to a threshold of 0\%
%          where only the majority group would be shaded. See the
%          difference:
%    \bigskip    
%          
%    \begin{texshade}{AQPpro.MSF}
%       \residuesperline*{34}
%       \shadingmode[chemical]{functional}
%       \setends{1}{138..170}
%       \shadeallresidues
%    \end{texshade}
%          
%    Code:\medskip
%
%    \vbox{%
%    \begin{verbatim}    
%    \begin{texshade}{AQPpro.MSF}
%       \shadingmode[chemical]{functional}
%       \setends{1}{138..170}
%       \shadeallresidues
%    \end{texshade}
%    \end{verbatim}}    
%          
%    
%    \item [\textbf{rasmol}:] similar to |[chemical]| but with
%                      shading following the rasmol
%                      color scheme \label{ras}
%    \bigskip
%    \bigskip
%    \bigskip
%                      
%    \begin{texshade}{AQPpro.MSF}
%       \residuesperline*{34}
%       \shadingmode[rasmol]{functional}
%       \setends{1}{138..170}
%       \showruler{top}{1}
%       \rulersteps{1}
%       \namerulerpos{150}{site A[Red]}
%       \namerulerpos{155}{site B[Green]}
%       \shadeallresidues
%       \showlegend
%    \end{texshade}
%          
%    Code:\medskip
%
%    \vbox{%
%    \begin{verbatim}    
%    \begin{texshade}{AQPpro.MSF}
%       \shadingmode[rasmol]{functional}
%       \setends{1}{138..170}
%       \showruler{bottom}{1}
%       \rulersteps{1}
%       \namerulerpos{150}{site A[Red]}
%       \namerulerpos{155}{site B[Green]}
%       \shadeallresidues
%       \showlegend
%    \end{texshade}
%    \end{verbatim}}    
%          
%          
%          
%    \item [\textbf{standard area}:] this shading displays the 
%                      differences in the surface 
%                      area \label{starea}
%                      of the different amino acid's sidechains
%    \bigskip    
%                      
%    \begin{texshade}{AQPpro.MSF}
%       \residuesperline*{34}
%       \shadingmode[standard area]{functional}
%       \setends{1}{138..170}
%       \showlegend
%       \shadeallresidues
%    \end{texshade}
%          
%    Code:\medskip
%
%    \vbox{%
%    \begin{verbatim}    
%    \begin{texshade}{AQPpro.MSF}
%       \shadingmode[standard area]{functional}
%       \setends{1}{138..170}
%       \showlegend
%       \shadeallresidues
%    \end{texshade}
%    \end{verbatim}}    
%          
%    \item [\textbf{accessible area}:] \label{accarea} 
%                     here, the surface area which can
%                     be accessed by solvent molecules is used as a 
%                     basis for shading; low accessibility means 
%                     hydrophobic (i.\,e.\ strongly buried 
%                     residues), whereas highly accessible 
%                     sidechains are hydrophilic (compare to
%                     \textbf{hydropathy} and \textbf{structure})
%    \bigskip    
%                   
%    \begin{texshade}{AQPpro.MSF}
%       \residuesperline*{34}
%       \shadingmode[accessible area]{functional}
%       \setends{1}{138..170}
%       \showlegend
%       \feature{top}{1}{138..157,164..170}{helix}{membr.}
%       \feature{top}{1}{158..163}{---}{loop}
%       \featurerule{1mm}
%       \shadeallresidues
%    \end{texshade}
%
%    Code:\medskip
%
%    \vbox{%
%    \begin{verbatim}    
%    \begin{texshade}{AQPpro.MSF}
%       \shadingmode[accessible area]{functional}
%       \setends{1}{138..170}
%       \showlegend
%       \feature{top}{1}{138..157,164..170}{helix}{membr.}
%       \feature{top}{1}{158..163}{---}{loop}
%       \featurerule{1mm}
%       \shadeallresidues
%    \end{texshade}
%    \end{verbatim}}    
%
%    \end{itemize}
%    
%    
%    
%    \subsection{Bar graphs and color scales}
%    
%   \label{graphs}
%    
%    Amino acid properties, such as hydrophobicity, molecular weight,
%    or charge can be shown as bar graphs or color scales along the
%    alignment. Further, the degree of protein sequence conservation 
%    can be indicated. As an example, in the following  
%    aquaporin alignment plots of residue conservation (bars, top), 
%    are shown as well as properties of the AQP1 sequence: charge (scale, top), 
%    molecular weight are shown (scale, bottom), and hydrophobicity (bars, bottom).
%    
%    
%    \begin{texshade}{AQPpro.MSF}
%       \residuesperline*{34}
%       \setends{1}{138..170}
%       \feature{ttop}{1}{138..170}{bar:conservation}{}
%       \showfeaturestylename{ttop}{conserv.}
%       \ttopspace{-\baselineskip}
%       \feature{top}{1}{138..170}{color:charge}{}
%       \showfeaturestylename{top}{charge}
%       \feature{bottom}{1}{138..170}{color:molweight[ColdHot]}{}
%       \showfeaturestylename{bottom}{weight}
%       \bbottomspace{-\baselineskip}
%       \feature{bbottom}{1}{138..170}{bar:hydrophobicity[Red,Gray10]}{}
%       \showfeaturestylename{bbottom}{hydrophob.}
%       \bargraphstretch{3}
%       \featurestylenamescolor{Red}
%       \featurestylenamesrm  \featurestylenamesit
%       \hideconsensus
%    \end{texshade}
%          
%    Code:\medskip
%
%    \vbox{%
%    \begin{verbatim}    
%    \begin{texshade}{AQPpro.MSF}
%       \setends{1}{138..170}
%       \feature{ttop}{1}{138..170}{bar:conservation}{}
%       \showfeaturestylename{ttop}{conserv.}
%       \ttopspace{-\baselineskip}
%       \feature{top}{1}{138..170}{color:charge}{}
%       \showfeaturestylename{top}{charge}
%       \feature{bottom}{1}{138..170}{color:molweight[ColdHot]}{}
%       \showfeaturestylename{bottom}{weight}
%       \bbottomspace{-\baselineskip}
%       \feature{bbottom}{1}{138..170}{bar:hydrophobicity[Red,Gray10]}{}
%       \showfeaturestylename{bbottom}{hydrophob.}
%       \bargraphstretch{3}
%       \featurestylenamescolor{Red}
%       \featurestylenamesrm  \featurestylenamesit
%       \hideconsensus
%    \end{texshade}
%    \end{verbatim}}    
%    \medskip      
%    
%    The degree of similarity and identity between all sequences in the alignment 
%    can be shown in a table using |\similaritytable| (\ref{simtable}) outside the |texshade|
%    environment: \label{simtableEx}
%    
%    \DeleteShortVerb{\|}
%    \begin{center}\similaritytable\end{center}
%    \MakeShortVerb{\|}
%    
%    
%    \subsection{Secondary structures}
%    
%   \label{sec}
%    
%    Predicted protein secondary structures in the DSSP, STRIDE
%    PHD or HMMTOP file format can be included and displayed in the
%    alignment. As an example, the following few commands show an 
%    aquaporin alignment with the PHD topology data for aquaporin 
%    type 1 (top sequence).
%    
%    \vbox{%
%          \begin{verbatim}
%          \begin{texshade}{AQPpro.MSF}
%             \shadingmode[allmatchspecial]{similar}
%             \includePHDtopo{1}{AQP1.phd}
%          \end{texshade}
%          \end{verbatim}
%    }         
%    
%    Abbr.: \emph{int.} -- internal; \emph{ext.} -- external; \emph{TM} --
%    transmembrane domain
%    
%          \begin{texshade}{AQPpro.MSF}
%             \shadingmode[allmatchspecial]{similar}
%             \includePHDtopo{1}{AQP1.phd}
%          \end{texshade}
%    
%    \subsection{Sequence fingerprints}
%    
%    \label{finger}
%    
%    To gain a quick overview of sequence similarities or properties 
%    the |\fingerprint| command has been implemented. It can depict the 
%    complete sequence in one single line. The residues are presented 
%    as colored vertical lines. The implementation of this kind of output
%    was inspired by a publication by \textsc{Kai-Uwe Fr\"ohlich} [6].
%    \medskip
%    
%    \begin{texshade}{AQPpro.MSF}
%       \shadingmode[allmatchspecial]{similar}
%       \shadingcolors{grays}
%       \fingerprint{360}
%       \showlegend
%       \feature{top}{1}{13..36,51..68,94..112,138..156,%
%                        165..185,211..232}{,-,}{TM}
%    \end{texshade}
%    
%    Code:\medskip
%
%    \vbox{%
%    \begin{verbatim}    
%    \begin{texshade}{AQPpro.MSF}
%       \shadingmode[allmatchspecial]{similar}
%       \shadingcolors{grays}
%       \fingerprint{360}
%       \showlegend
%       \feature{top}{1}{13..36,51..68,94..112,138..156,%
%                        165..185,211..232}{,-,}{TM}
%    \end{texshade}
%    \end{verbatim}}    
%    
%    The higher the similarity the darker the vertical lines. In this 
%    overview it becomes obvious that the transmembrane regions of the 
%    aquaporin isoforms are most conserved.
%    \medskip
%    
%    A fingerprint of charge distribution on different aquaporins is shown. 
%    below. Sequence gaps can be left blank (example above) or drawn as lines
%    between the sequence blocks.
%    
%    \begin{texshade}{AQPpro.MSF}
%       \shadingmode[charge]{functional}
%       \shadeallresidues
%       \fingerprint{360}
%       \gapchar{rule}
%       \showlegend
%    \end{texshade}
%
%
%    Code:\medskip
%
%    \vbox{%
%    \begin{verbatim}    
%    \begin{texshade}{AQPpro.MSF}
%       \shadingmode[charge]{functional}
%       \shadeallresidues
%       \fingerprint{360}
%       \gapchar{rule}
%       \showlegend
%    \end{texshade}
%    \end{verbatim}}    
%
%    
%    \subsection{Sequence logos}
%    
%    \label{logo}
%    
%    Sequence logos represent the information content of the aligned 
%    sequences at a position in bit (max.\ 2 bit for DNA, i.\,e.
%    log$_2$4, and 4.322 bit for proteins, i.\,e. log$_2$20) and the 
%    relative frequency of a base or amino acid at this 
%    position [7]. Thus, more information is contained in logos than in
%    a standard consensus sequence.
%    The example below shows a DNA sequence alignment with the logo on the  
%    top.
%
%    It must be remarked that a logo from only five sequences does not
%    produce meaningful results - it rather illustrates the technique.
%
%    \medskip
%    
%    \begin{texshade}{AQPDNA.MSF}
%       \setends{1}{414..443}
%       \showsequencelogo{top}
%    \end{texshade}
%    
%    Code:\medskip
%
%    \vbox{%
%    \begin{verbatim}    
%    \begin{texshade}{AQPDNA.MSF}
%       \setends{1}{414..443}
%       \showsequencelogo{top}
%    \end{texshade}
%    \end{verbatim}}    
%    
%    
%    Next, only the logo of a protein alignment is displayed plus the 
%    degree of sequence conservation as a color scale in the consensus 
%    line. Note, that the full functionality of the feature lines remains.
%    \medskip
%    
%    \begin{texshade}{AQPpro.MSF}
%       \setends{AQP3.PRO}{203..235}
%       \showsequencelogo{top} \showlogoscale{leftright}
%       \hideseqs
%       \residuesperline*{33}
%       \defconsensus{{$\bullet$}}{{$\bullet$}}{{$\bullet$}}
%       \showconsensus[ColdHot]{bottom}
%       \nameconsensus{conservation} \namessf\namessl
%       \showruler{bottom}{AQP3.PRO} \rulersteps{1}
%       \feature{top}{AQP3.PRO}{208..210}{---}{NPA}
%       \feature{top}{AQP3.PRO}{211..219}{helix}{}
%       \feature{top}{AQP3.PRO}{220..232}{brace}{loop E}
%       \feature{top}{AQP3.PRO}{233..235}{helix}{TM6}
%       \feature{bottom}{AQP3.PRO}{203..235}{brace}{1-step numbering}
%    \end{texshade}
%    
%    Code:\medskip
%
%    \vbox{%
%    \begin{verbatim}    
%    \begin{texshade}{AQPpro.MSF}
%       \setends{AQP3.PRO}{203..235}
%       \showsequencelogo{top} \showlogoscale{leftright}
%       \hideseqs
%       \residuesperline*{33}
%       \defconsensus{{$\bullet$}}{{$\bullet$}}{{$\bullet$}}
%       \showconsensus[ColdHot]{bottom}
%       \nameconsensus{conservation} \namessf\namessl
%       \showruler{bottom}{AQP3.PRO} \rulersteps{1}
%       \feature{top}{AQP3.PRO}{208..210}{---}{NPA}
%       \feature{top}{AQP3.PRO}{211..219}{helix}{}
%       \feature{top}{AQP3.PRO}{220..232}{brace}{loop E}
%       \feature{top}{AQP3.PRO}{233..235}{helix}{TM6}
%       \feature{bottom}{AQP3.PRO}{203..235}{brace}{1-step numbering}
%    \end{texshade}
%    \end{verbatim}} 
%    
%    The same logo is shown below but with frequency correction turned
%    on (|\dofrequencycorrection|), see p.\pageref{Lshowsequencelogo}.
%    This takes into account the difference between the amino acid 
%    distribution in the alignment and the equal distribution of
%    5\% for each residue.
%    \medskip
%
%    \begin{texshade}{AQPpro.MSF}
%       \setends{AQP3.PRO}{203..235}
%       \showsequencelogo{top} \showlogoscale{leftright}
%       \hideseqs
%       \residuesperline*{33}
%       \defconsensus{{$\bullet$}}{{$\bullet$}}{{$\bullet$}}
%       \showconsensus[ColdHot]{bottom}
%       \nameconsensus{conservation} \namessf\namessl
%       \showruler{bottom}{AQP3.PRO} \rulersteps{1}
%       \feature{top}{AQP3.PRO}{208..210}{---}{NPA}
%       \feature{top}{AQP3.PRO}{211..219}{helix}{}
%       \feature{top}{AQP3.PRO}{220..232}{brace}{loop E}
%       \feature{top}{AQP3.PRO}{233..235}{helix}{TM6}
%       \feature{bottom}{AQP3.PRO}{203..235}{brace}{1-step numbering}
%       \dofrequencycorrection
%    \end{texshade}
%    
%
%    \subsection{Subfamily logos}
%    
%    \label{sublogo}
%    
%    The following output is derived from the calculation of a
%    subfamily logo [14]. Such logos display relevant deviations of a
%    subfamily compared to the remaining set of sequences. Here,
%    typical residues of AQP3 are shown (upright) which deviate 
%    from the remaining  four aquaporins of this alignment (upside-down).
%    The output can be directly compared to the sequence logo above,
%    which displays the same section of the alignment. 
%    Note, that five sequences are far too few to obtain meaningful
%    results with this method. This is just to illustrate the 
%    approach.
%    \medskip
%
%    \begin{texshade}{AQPpro.MSF}
%       \setends{AQP3.PRO}{203..235}
%       \residuesperline*{33}
%       \setsubfamily{3} 
%       \showsubfamilylogo{top} \showlogoscale{leftright}
%       \namesubfamilylogo[others]{AQP3}
%       \namessf \namessl
%       \showruler{bottom}{AQP3.PRO} \rulersteps{1}
%       \hideseqs
%       \hideconsensus
%       \dofrequencycorrection
%    \end{texshade}
%
%    Code:\medskip
%
%    \vbox{%
%    \begin{verbatim}    
%    \begin{texshade}{AQPpro.MSF}
%       \setends{AQP3.PRO}{203..235}
%       \residuesperline*{33}
%       \setsubfamily{3} 
%       \showsubfamilylogo{top} \showlogoscale{leftright}
%       \namesubfamilylogo[others]{AQP3}
%       \namessf \namessl
%       \showruler{bottom}{AQP3.PRO} \rulersteps{1}
%       \hideseqs
%       \hideconsensus
%       \dofrequencycorrection
%    \end{texshade}
%    \end{verbatim}} 
%    
%
%
%
%    \subsection{Customization of the alignment output}
%    
%    Extensive possibilities are given to the user to customize 
%    the final output of an alignment. Thus, all parameters defining the
%    appearance of letters can be changed individually for sequence 
%    residues, names and numbering or the describing feature texts. 
%    Additional manual shading can be applied to any region or
%    block of residues. Sequences are easily re-ordered, separated, hidden 
%    or blanked out without recalculation of the entire alignment; 
%    sections of the alignment can also be shown. 
%    Numbering and rulers can be displayed and set to any value. 
%    A powerful tool is the |\feature|
%    command which allows one to label stretches of residues with bars, 
%    arrows, braces or any fill character and describing text. 
%    Legends are set automatically if desired, but user commands
%    are also provided to build individual legends.
%    
%    
%    \newpage
%    \section{Format of alignment input files}
%    
%    \label{alignfilestruc}
%    
%    \TeXshade{} can handle two common alignment input formats, i.\,e.\
%    the MSF format (\underline{m}ultiple \underline{s}equence 
%    \underline{f}ormat) and the ALN format 
%    (\underline{al}ig\underline{n}ment format). The MSF 
%    format is used by |PILEUP| of the Unix GCG sequence
%    analysis package\footnote{For a description see
%    |http://gene.md.huji.ac.il/Computer/GCG9doc|}. Files in the
%    ALN format are produced by |CLUSTAL| which is 
%    available for free for Unix, DOS and Macintosh. Further, upon 
%    request, the FASTA format is supported since version 1.6.
%    In addition to the mentioned software many alignment programs have 
%    export filters for the MSF, ALN or FASTA
%    format, e.\,g.\ |MACAW| produces ALN files. If
%    you are not sure whether your favorite sequence aligner 
%    produces one of the required formats compare its output to
%    the following examples. \TeXshade{} determines the format from
%    the internal file structure, thus extensions like MSF, ALN
%    or FASTA
%    are not required. If you can choose the alignment format 
%    MSF is recommended, because this format gives information
%    about the sequence type, i.\,e.\ peptide or nucleotide sequences, 
%    and length (for the correct setting of gaps at the sequence end).
%    
%    \subsection{The MSF file format}
%    Files of this type are divided into a header section and the 
%    multiple sequence alignment. The header may contain the
%    following components:
%    
%    
%    \begin{itemize}
%    \item[\textbf{File Type}:] (optional) The first header line 
%              reads for nucleic acids alignments
%              |!!NA_MULTIPLE_ALIGNMENT 1.0| and for amino acid sequences
%              |!!AA_MULTIPLE_ALIGNMENT 1.0| (all uppercase).
%    \item[\textbf{Description}:] (optional) Informative text
%              describing what is in the file.
%    \item[\textbf{Dividing line}:] (required!) Must include the 
%              following attributes:
%       \begin{itemize}
%          \item[|MSF|:] Displays the number of bases or residues in 
%                        the multiple sequence alignment.
%          \item[|Type|:] Displays the sequence type, `P' for a peptide
%                         and `N' for a nucleotide alignment.
%          \item[|Checksum|:] Displays an integer value that 
%                        characterizes the contents of the file.
%          \item[|..|] The two periods act as a divider between the 
%                        descriptive information and the following
%                        sequence information.
%       \end{itemize}
%    \item[\textbf{Name/Weight}:] (required!) Must include the name of
%             each sequence included in the alignment, as well as its
%             length, checksum and weight.
%    \item[\textbf{Two slashes} (|//|):] (required!) This separating 
%             line divides the name/weight information from the 
%             sequence alignment
%    \end{itemize}
%    
%    The alignment section consists of sequence blocks divided by an
%    empty line. Each sequence line starts out with the sequence name.
%    An example file is shown here:
%    \medskip
%    
%    \parindent-1mm
%    \begin{fmpage}
%    \begin{verbatim}
%    
%    AQP.MSF  MSF: 87  Type: P  May 1st, 1998 Check: 2586 ..
%    Name: AQP1.PRO  Len: 66  Check:  1367    Weight: 1.00
%    Name: AQP2.PRO  Len: 58  Check:  2176    Weight: 1.00
%    Name: AQP3.PRO  Len: 83  Check:  1893    Weight: 1.00
%    Name: AQP4.PRO  Len: 63  Check:  3737    Weight: 1.00
%    Name: AQP5.PRO  Len: 59  Check:  3413    Weight: 1.00
%    //
%              1                                           45
%    AQP1.PRO  MAS........................EIKKKLFWRAVVAEFLAM
%    AQP2.PRO  MW.........................ELRSIAFSRAVLAEFLAT
%    AQP3.PRO  M.........NRCG.....EMLHIRYR......LLRQALAECLGT
%    AQP4.PRO  MSDGAAARRWGKCGPPCSRESIMVAFKGVWTQAFWKAVTAEFLAM
%    AQP5.PRO  MK........................KEVCSLAFFKAVFAEFLAT
%    
%              45                                       87
%    AQP1.PRO  TLFVFISIGSALGFNYPLERNQTLVQDNVKVSLAFGLSIATL
%    AQP2.PRO  LLFVFFGLGSALQWA...SS....PPSVLQIAVAFGLGIGIL
%    AQP3.PRO  LILVMFGCGSVAQVVLSRGTHGGF....LTINLAFGFAVTLA
%    AQP4.PRO  LIFVLLSVGSTINWG...GSENPLPVDMVLISLCFGLSIATM
%    AQP5.PRO  LIFVFFGLGSALKWP...SA....LPTILQISIAFGLAIGTL
%    \end{verbatim}
%    \end{fmpage}
%    \bigskip
%    
%    \parindent0mm
%    \TeXshade{} extracts only the information from the file it
%    really needs. So, do not mind all the checksums listed
%    in the file---\TeXshade{} does not either. The same is true 
%    for |Weight|. Required are the string |MSF:| 
%    for the identification of the file format and |Type:| for the 
%    determination of the sequence type (both in the dividing line), 
%    further all |Name:| definitions and finally |//|. The MSF format 
%    allows one to comment out sequences. This is done 
%    by putting an exclamation point directly infront of the respective 
%    |Name|. These sequences are neither displayed nor used for the 
%    calculation of the consensus. This works for \TeXshade, too.
%    To comment out sequences without changing
%    the input file use the \TeXshade{} command 
%    |\killseq{|\meta{seqref}|}| (\ref{kill}). 
%    \medskip
%    
%    \parindent-1mm
%    \begin{fmpage}\label{commout}
%    \begin{verbatim}
%    
%    AQP.MSF  MSF: 87  Type: P  May 1st, 1998 Check: 2586 ..
%    Name: AQP1.PRO  Len: 66  Check:  1367    Weight: 1.00
%    !Name: AQP2.PRO  Len: 58  Check:  2176    Weight: 1.00
%    !Name: AQP3.PRO  Len: 83  Check:  1893    Weight: 1.00
%    Name: AQP4.PRO  Len: 63  Check:  3737    Weight: 1.00
%    Name: AQP5.PRO  Len: 59  Check:  3413    Weight: 1.00
%    //    
%              1                                           45
%    AQP1.PRO  MAS........................EIKKKLFWRAVVAEFLAM
%    AQP2.PRO  MW.........................ELRSIAFSRAVLAEFLAT
%    AQP3.PRO  M.........NRCG.....EMLHIRYR......LLRQALAECLGT
%    AQP4.PRO  MSDGAAARRWGKCGPPCSRESIMVAFKGVWTQAFWKAVTAEFLAM
%    AQP5.PRO  MK........................KEVCSLAFFKAVFAEFLAT
%    
%              45                                       87
%    AQP1.PRO  TLFVFISIGSALGFNYPLERNQTLVQDNVKVSLAFGLSIATL
%    AQP2.PRO  LLFVFFGLGSALQWA...SS....PPSVLQIAVAFGLGIGIL
%    AQP3.PRO  LILVMFGCGSVAQVVLSRGTHGGF....LTINLAFGFAVTLA
%    AQP4.PRO  LIFVLLSVGSTINWG...GSENPLPVDMVLISLCFGLSIATM
%    AQP5.PRO  LIFVFFGLGSALKWP...SA....LPTILQISIAFGLAIGTL
%    \end{verbatim}
%    \end{fmpage}
%    \parindent0mm
%    \bigskip
%    
%    The sequence lengths given after |Len:| are not used by 
%    \TeXshade. Due to the fact that most alignment programms calculate the 
%    sequence length by summing up residues and additionally gaps which 
%    is not really correct. In order to have the sequence break right 
%    after the last residue without printing further gap symbols 
%    \TeXshade{} counts the number of residues by itself. You can  
%    also use the command |\seqlength| in the \TeXshade{}
%    environment to set the values manually if you do not trust a machine.
%    
%    \subsection{The ALN file format}
%    ALN files are quite similar to the above described MSF files.
%    They simply lack a defined header section. Nevertheless, 
%    describing text is allowed before the alignment part. \TeXshade{}
%    determines the number of sequences and their names from the last
%    sequence block---so, no further text lines are allowed after this block!
%    Due to a lacking declaration in the file the sequence type has 
%    to be set in the |texshade| environment by |\seqtype{|\meta{type}|}|
%    \label{Lseqtype} with `P' for peptide and `N' for nucleotide sequences;
%    for the example below: |\seqtype{P}|. If no |\seqtype| command
%    is used \TeXshade{} assumes a nucleotide sequence.
%    \bigskip
%    
%    \parindent-1mm
%    \begin{fmpage}
%    \begin{verbatim}
%    
%                  profalign  May 1st, 1998, 16:58
%                  
%    of AQPpro.MSF{}
%    
%    Muliple alignment parameter:
%    
%    Gap Penalty (fixed):           10.00
%    Gap Penalty (varying):         .05
%    Gap separation penalty range:  8
%    Percent. identity for delay:   0%
%    List of hydrophilic residue:   GPSNDQEKRH
%    Protein Weight Matrix:         blosom
%    
%                      10        20        30        40
%                       .         .         .         .
%    AQP1.PRO  MAS........................EIKKKLFWRAVVAEFLAM
%    AQP2.PRO  MW.........................ELRSIAFSRAVLAEFLAT
%    AQP3.PRO  M.........NRCG.....EMLHIRYR......LLRQALAECLGT
%    AQP4.PRO  MSDGAAARRWGKCGPPCSRESIMVAFKGVWTQAFWKAVTAEFLAM
%    AQP5.PRO  MK........................KEVCSLAFFKAVFAEFLAT
%              *                                  .   ** *.
%    
%    AQP1.PRO  TLFVFISIGSALGFNYPLERNQTLVQDNVKVSLAFGLSIATL
%    AQP2.PRO  LLFVFFGLGSALQWA...SS....PPSVLQIAVAFGLGIGIL
%    AQP3.PRO  LILVMFGCGSVAQVVLSRGTHGGF....LTINLAFGFAVTLA
%    AQP4.PRO  LIFVLLSVGSTINWG...GSENPLPVDMVLISLCFGLSIATM
%    AQP5.PRO  LIFVFFGLGSALKWP...SA....LPTILQISIAFGLAIGTL
%              .. *   .**                    .   **  .
%    \end{verbatim}
%    \end{fmpage}
%    \bigskip
%    
%    The minimal contents of an ALN file are shown below; this 
%    is fully sufficient. Many sequence alignment programs can
%    produce such an output. Have a look at |seqpup| by 
%    \textsc{Don Gilbert} if you need a comprehensive conversion
%    program\footnote{Sorry, |seqpup| is much more!}.
%    \bigskip 
%   
%    \parindent-1mm
%    \begin{fmpage}
%    \begin{verbatim}
%    
%    AQP1.PRO  MAS........................EIKKKLFWRAVVAEFLAM
%    AQP2.PRO  MW.........................ELRSIAFSRAVLAEFLAT
%    AQP3.PRO  M.........NRCG.....EMLHIRYR......LLRQALAECLGT
%    AQP4.PRO  MSDGAAARRWGKCGPPCSRESIMVAFKGVWTQAFWKAVTAEFLAM
%    AQP5.PRO  MK........................KEVCSLAFFKAVFAEFLAT
%    
%    AQP1.PRO  TLFVFISIGSALGFNYPLERNQTLVQDNVKVSLAFGLSIATL
%    AQP2.PRO  LLFVFFGLGSALQWA...SS....PPSVLQIAVAFGLGIGIL
%    AQP3.PRO  LILVMFGCGSVAQVVLSRGTHGGF....LTINLAFGFAVTLA
%    AQP4.PRO  LIFVLLSVGSTINWG...GSENPLPVDMVLISLCFGLSIATM
%    AQP5.PRO  LIFVFFGLGSALKWP...SA....LPTILQISIAFGLAIGTL
%    \end{verbatim}
%    \end{fmpage}
%    \bigskip
%    
%    \subsection{The FASTA file format}
%    In FASTA files each sequence is led
%    by a single description line starting with a `|>|'. \TeXshade{} uses
%    the first word delimited by the leading `|>|' and a space as 
%    the sequence name. If no descriptive text is present \TeXshade{}
%    generates a sequence name consisting of `|seq|' plus a consecutive
%    number. The lines following the description line 
%    contain the sequence.
%    \bigskip
%    
%    \begin{fmpage}
%    \begin{verbatim}
%    
%    >AQP1.PRO  
%    MAS........................EIKKKLFWRAVVAEFLAM
%    TLFVFISIGSALGFNYPLERNQTLVQDNVKVSLAFGLSIATL
%    
%    >AQP2.PRO  
%    MW.........................ELRSIAFSRAVLAEFLAT
%    LLFVFFGLGSALQWA...SS....PPSVLQIAVAFGLGIGIL
%    
%    >AQP3.PRO  
%    M.........NRCG.....EMLHIRYR......LLRQALAECLGT
%    LILVMFGCGSVAQVVLSRGTHGGF....LTINLAFGFAVTLA
%    
%    >AQP4.PRO  
%    MSDGAAARRWGKCGPPCSRESIMVAFKGVWTQAFWKAVTAEFLAM
%    LIFVLLSVGSTINWG...GSENPLPVDMVLISLCFGLSIATM
%    
%    >AQP5.PRO  
%    MK........................KEVCSLAFFKAVFAEFLAT
%    LIFVFFGLGSALKWP...SA....LPTILQISIAFGLAIGTL
%    \end{verbatim}
%    \end{fmpage}
%    \bigskip
%    
%    
%    \parindent0mm
%    \newpage
%    \section{Use of a \TeX{}shade parameter file}
%    
%    \label{paramfilestruc}
%    
%    Using predefined parameter files for repeatedly occuring situations
%    can save a lot of typing and makes the output throughout the 
%    publication or presentation more consistent. Further, such
%    files are an easy way to exchange self-defined shading 
%    modes or new color schemes (i.\,e.\ for a satisfying grayscale output) 
%    with other users. If you have created a
%    parameter file, which you think is of interest for others, please 
%    submit it to me\footnote{|ebeitz@pharmazie.uni-kiel.de|} as an e-mail 
%    attachment together with a short 
%    description. I will take care of those files and post them---with 
%    a reference to the author---together with the next \TeXshade{} 
%    distribution to make them available for all interested users.
%    
%    No special file format is required for parameter 
%    files. \TeXshade{} simply calls the file using the |\input|
%    command right after resetting all parameters to default. An
%    example parameter file is present containing the standard
%    parameters of \TeXshade{} called |texshade.def|. This file can be
%    changed freely and can be used as a template for the creation of
%    personal parameter files. 
%    
%    Five steps are executed by \TeXshade{} when 
%    processing the |texshade| environment:
%    
%    \bigskip
%    \begin{minipage}{12cm}
%    |\begin{texshade}[|\meta{parameterfile}|]{|\meta{alignmentfile}|}|
%    
%    \begin{enumerate}
%    \item Analysis of the \meta{alignmentfile}; determination of 
%    the number of sequences and sequence names
%    
%    \item Setting parameters to default
%    
%    \item Setting parameters to the definitions of the 
%    \meta{parameterfile}, if existent
%    
%    \item Execution of further \TeXshade{} commands within the
%    evironment, if existent
%    
%    \parindent-1cm
%    \medskip
%    |\end{texshade}|
%    
%    \parindent0cm
%    \item Loading and setting the alignment on a line by line basis
%    \end{enumerate}
%    \end{minipage}
%    
%    \newpage
%    \section{\texttt{texshade} user commands}
%    
%    The \TeXshade{} package must be loaded by the |\usepackage|
%    command in the document header section.
%    \medskip
%    
%    \quad|\usepackage[|\meta{option}|]{texshade}|
%    \medskip
%    
%    Then, the |texshade| environment is ready to use as described
%    in \ref{tsenvironment}. See also section \ref{paramfilestruc} for
%    a description of the optional parameter file. All other 
%    commands provided by \TeXshade{} (except |\shadebox| [\ref{Lshadebox}], |\molweight| and
%    |\charge| [\ref{molcharge}], and |\percentsimilarity|, |\percentidentity| and
%    |\similaritytable| [\ref{Lpercentidentity}]) must 
%    be used within the |texshade| environment.
%    
%    
%    
%    \subsection{Using predefined shading modes}
%    
%    \label{predef}
%    
%    \label{Lshadingmode}
%    If no |\shadingmode| command is given in the |texshade| 
%    environment the default shading mode (\emph{identical}, see 
%    \ref{ident}) is active. For the selection of one of the other
%    predefined shading modes the following command is provided.
%    \bigskip
%    
%    \quad |\shadingmode[|\meta{option}|]{|\meta{mode}|}|
%    \bigskip
%    
%    You can choose from four shading modes and declare one option 
%    which depends on the selected mode. 
%    
%    \begin{enumerate}
%    
%    \item |\shadingmode[|\meta{allmatchspecial/number}|]{identical}|
%    
%       See \ref{ident} for examples. Use the
%       option |allmatchspecial| to shade positions with a special color
%       where all residues are identical. Or use a percentage number
%       (0--100) as an option to set an additional threshold for highly
%       conserved residues, e.\,g.\ |\shadingmode[90]{identical}|. 
%       \label{Lallmatchspecial}|\allmatchspecial| can also be 
%       used as a command with or without an optional parameter for
%       setting the high conservation threshold. As both, option or command, 
%       |allmatchspecial| is only active in the \emph{identical} and
%       \emph{similar} shading modes.
%       
%       \label{Lshadingcolors}
%       One can choose from five predefined shading color schemes with
%       the command 
%       |\shadingcolors{|\meta{scheme}|}|. The sets are named `blues' 
%       (used in the example, \ref{ident}), `reds', `greens', 
%       `grays' and `black'. Default is |\shadingcolors{blues}|. Further, the colors 
%       for the non matching, the 
%       conserved and all matching (or highly conserved) residues can be set individually 
%       plus the letter case (lower or upper) or any character
%       can be chosen: \label{Lnomatchresidues}
%       \label{Lconservedresidues}
%       \label{Lallmatchresidues} 
%       \bigskip
%       
%       |\nomatchresidues{|\meta{res.col.}|}{|\meta{shad.col.}|}{|\meta{case}|}{|\meta{style}|}|
%       
%       |\conservedresidues{|\meta{res.col.}|}{|\meta{shad.col.}|}{|\meta{case}|}{|\meta{style}|}|
%       
%       |\allmatchresidues{|\meta{res.col.}|}{|\meta{shad.col.}|}{|\meta{case}|}{|\meta{style}|}|
%       \bigskip
%       
%       For how to handle colors for the foreground \meta{res.col.} and 
%       the background \meta{shad.col.} see section \ref{colors}. 
%       The third parameter \meta{case} tells \TeXshade{} to print the
%       corresponding residue as a lowercase or an uppercase letter or
%       even to print any other character. Finally, the \meta{style} 
%       parameter tells \TeXshade{} which shape to use for the letters.
%       Use one of the following styles 
%       for \meta{style}.
%    
%       \begin{center}
%       \begin{tabular}{cl}
%       \meta{style} & \emph{effect} \\ \hline
%       |bf| & bold face series\\
%       |md| & normal series \\
%       |up| & upright shape (normal shape)\\
%       |it| & italics shape \\
%       |sl| & slanted shape \\
%       |rm| & modern roman family \\
%       |sf| & sans serif family \\
%       |tt| & typewriter family \\
%       \end{tabular}
%       \end{center}
%       \medskip    
%    
%       In order to change only some
%       of the parameters it is sufficient to declare these
%       and use empty braces for the others. Examples:
%       \bigskip
%       
%       \quad|\conservedresidues{White}{Blue}{upper}{bf}|: the conserved
%       residues are printed as bold face white uppercase letters on blue.
%       \bigskip
%       
%       \quad|\nomatchresidues{}{}{{$\bullet$}}{}|: instead of the non 
%       matching residues a `$\bullet$' is printed. The colors and style 
%       are not changed.
%       Note the double curly braces which make \TeXshade{} 
%       interpret this complex symbol description as one single 
%       character.
%
%       Once a set of shading colors is redefined, it can be saved using
%       |\defshadingcolors{|\meta{name}|}|\label{Ldefshadingcolors} for
%       later use in the document (see \ref{shaderegion}).
%
%       \bigskip
%       
%       
%    \item |\shadingmode[|\meta{allmatchspecial/number}|]{similar}|
%    
%       \label{Lsimilarresidues}
%       See \ref{similar} for an example output and an explanation
%       of the shading. In addition to the described commands
%       for changing shading colors this shading mode provides 
%       the command |\similarresidues|.
%       Use it in analogy to the commands above.
%       
%       \label{Lpepsims}\label{Lpepgroups}
%       \label{LDNAsims}\label{LDNAgroups}
%       How does \TeXshade{} know which residues are
%       considered to be similar? These definitions are set by two command
%       couples, i.\,e.\ 
%       |\pepsims|,|\pepgroups| for peptides and
%       |\DNAsims|,|\DNAgroups| for nucleotides. With |\pepsims| and
%       |\DNAsims| residues are defined which are similar to the 
%       consensus residue. Examples:
%       
%       \quad |\pepsims{S}{TA}|\quad If a serine is the consensus
%             residue then all threonins and alanines at this
%             position are shaded in the color for similars. This
%             definition does \emph{not} imply that threonine and
%             alanine are similar to each other! This becomes 
%             obvious when you inspect the next definition:
%             
%       \quad |\pepsims{T}{S}|\quad  Serine but not alanine is declared
%             to be similar to threonine.
%             
%       What happens if there is no consensus residue? How does
%       \TeXshade{} decide if a group of similars is greater than
%       the threshold? For this groups are pre-defined:
%       
%       \quad |\pepgroups{FYW,ILVM,RK,DE,GA,ST,NQ}| This command allows
%             one to set up to nine groups of similars, separated by commas. 
%             Each residue can belong to only one group. If one residue
%             is assigned to several groups only the last assignment is 
%             carried out.
%             
%       \quad |\DNAgroups{GAR,CTY}| This command is used in analogy to
%             the amino acid groups. Here, two ambiguity codes (`R' for
%             pu\underline{r}ine base and `Y' for p\underline{y}rimidine
%             base) are assigned in addition.
%             
%       Residues which do not appear in any of the four commands are 
%       considered not to belong to a group. The default 
%       settings for similars are listed below:
%       \bigskip
%       
%       \begin{verbatim}
% \pepgroups{FYW,ILVM,RK,DE,GA,ST,NQ}
%       
% \pepsims{F}{YW}  % Y and W are similar to F
% \pepsims{Y}{WF}  % W and F are similar to Y
% \pepsims{W}{YF}  % Y and F are similar to W
%    
% \pepsims{I}{LVM} % L, V and M are similar to I
% \pepsims{L}{VMI} % V, M and I are similar to L
% \pepsims{V}{MIL} % M, I and L are similar to V
%    
% \pepsims{R}{KH}  % K and H are similar to R
% \pepsims{K}{HR}  % H and R are similar to K
% \pepsims{H}{RK}  % R and K are similar to H
%       
% \pepsims{A}{GS}  % G and S are similar to A
% \pepsims{G}{A}   % A (but not S) is similar to G
%       
% \pepsims{S}{TA}  % T and A are similar to S
% \pepsims{T}{S}   % S (but not A) is similar to T
%    
% \pepsims{D}{EN}  % E and N (but not Q) are similar to D
% \pepsims{E}{DQ}  % D and Q (but not N) are similar to E
% \pepsims{N}{QD}  % Q and D (but not E) are similar to N
% \pepsims{Q}{NE}  % N and E (but not D) are similar to Q
%    
% \DNAgroups{GAR,CTY}
%       
% \DNAsims{A}{GR}  % G and R are similar to A
% \DNAsims{G}{AR}  % A and R are similar to G
% \DNAsims{R}{AG}  % A and G are similar to R
%       
% \DNAsims{C}{TY}  % T and Y are similar to C
% \DNAsims{T}{CY}  % C and Y are similar to T
% \DNAsims{Y}{CT}  % C and T are similar to Y
%       \end{verbatim}
%
%       
%    \item |\shadingmode[|\meta{filename}|]{T-Coffee}|
%       
%       Enter a \meta{filename} to load the shading
%       information from a |T-Coffee| |score_ascii| file (|www.tcoffee.org|); 
%       see example in \ref{TCoffee}. Make sure
%       that the alignment file specified in the |\texshade| command
%       and this shading file correspond to each other.
%    
%       If you do not enter a \meta{filename} here, a separate
%       command |\includeTCoffee{|\meta{filename}|}| must be used.
%    
%       |T-Coffee| shading can also be used in the consensus
%       p.\,\pageref{Lshowconsensus} and in the feature lines,  
%       in particular color scales and bar plots p.\,\pageref{Lgraphs}, 
%       for the display of shading information.
%    
%       
%    \item |\shadingmode[|\meta{seqref}|]{diverse}|
%    
%       \ref{diverse} depicts an example alignment. Choose the 
%       number or the name of the sequence \meta{seqref} which will be treated
%       as the consensus and to which the other sequences are compared. 
%       If no \meta{seqref} is declared the first sequence is set as 
%       consensus (\meta{seqref} = 1).
%       
%       Standard definitions for |diverse| 
%       mode are:
%
%       \begin{verbatim}
%          \nomatchresidues{Black}{White}{lower}{up}
%          \similarresidues{Black}{White}{lower}{up}
%          \conservedresidues{Black}{White}{{.}}{up}
%          \allmatchresidues{Black}{White}{{.}}{up}
%          \gapchar{-}
%       \end{verbatim}
%   
%       After calling |\shadingmode{diverse}| these commands can be 
%       used to redefine the |diverse| mode settings (mind the double 
%       curly braces around the dot-symbol!). 
%    
%       \label{Lhideallmatchpositions}
%       Since alignment positions where all residues match do not contain
%       much information, those sites can be blanked out using
%      �|\hideallmatchpositions|. The resulting break in the alignment is
%       indicated by a gap and a vertical line. See the |\setdomain| 
%       command (\ref{Lsetdomain}) 
%       for further information on how to change the gap and ruler colors.  
%       A single-stepped ruler is also recommended (\ref{Lshowruler}).
%      �|\hideallmatchpositions| can be combined with |\setends| 
%       (\ref{Lsetends}).
%   
%    
%    \item |\shadingmode[|\meta{type}|]{functional}|\label{funcdef}
%       There are seven different functional shading modes available for
%       peptide sequences; nucleotide sequences can not be shaded due
%       to functional aspects. Five of \TeXshade's functional modes
%       correspond to the four `alphabets' employed by \textsc{Karlin} 
%       and \textsc{Ghandour} for peptide alignments [2] or by the
%       rasmol software. Additional 
%       `alphabets' to the standard 20-letter array of amino acids
%       can highlight peptide similarities which were otherwise not visible.
%       For the `alphabet' definitions see below:
%       
%       \begin{itemize}
%       \item \meta{type} = |charge|\quad Acidic (D, E) and basic (H, 
%                           K, R).
%    
%       \item \meta{type} = |hydropathy|\quad Acidic and basic (as 
%                           above), polar uncharged (C, G, N, Q, S, 
%                           T, Y) and hydrophobic nonpolar (A, F, I, L, M, 
%                           P, V, W), see also \textsc{Kyte} \& 
%                           \textsc{Doolittle} [3].
%    
%       \item \meta{type} = |structure|\quad External (D, E, H, K, N, Q, R),
%                           internal (F, I, L, M, V) and ambivalent (A, C,
%                           G, P, S, T, W, Y).
%       
%       \item \meta{type} = |chemical|\quad Acidic (D, E), aliphatic
%                           (I, L, V), aliphatic (small) (A, G), 
%                           amide (N, Q), aromatic
%                           (F, W, Y), basic (H, K, R), hydroxyl
%                           (S, T), imino (P) and sulfur (C, M).
%                           
%       \item \meta{type} = |rasmol|\quad (D, E), (K, R, H), (F, Y, W),
%                           (A, G), (C, M), (S, T), (N, Q), (I, L, V), 
%                           (P).
%                           
%       \end{itemize}
%                           
%       The two modes described below highlight sidechain sizes and 
%       hydrophobicity, respectively, according to \textsc{Rose} 
%       \emph{et al.}\ [4,5]. Standard area stands for the surface area
%       of the residue in \AA$^2$, i.\,e. it is a measure for the size
%       of a residue's sidechain. The accessible area value (also in 
%       \AA$^2$) gives information about the size of the surface area
%       which is accessible by solvent molecules within the folded 
%       protein. A very small area means that the residue is
%       strongly buried and is thus very hydrophobic. Hydrophilic
%       residues in turn possess large accessible areas due
%       to their prefered location at the protein surface. Therefore,
%       this kind of shading provides another method, in addition
%       to |hydropathy| and |structure|, for the
%       visualization of structural protein properties.
%       
%       \begin{itemize}
%       
%       \item \meta{type} = |standard area|\quad for the area values
%                           see legend of the alignment in \ref{starea}
%                           
%       \item \meta{type} = |accessible area|\quad for values see
%                           \ref{accarea}
%       
%       \end{itemize}
%       
%       \label{Lclearfuncgroups}
%       If no \meta{type} or an unknown \meta{type} is designated as option
%       all functional groups and shading colors are cleared. This is
%       also achieved by the command 
%       |\clearfuncgroups|. With all groups cleared one can start to
%       build new shading modes from scratch. How to do this is explained
%       in the next section.
%       
%    \label{Lfuncshadingstyle} 
%    In order to exchange the colors but to keep the group definitions
%    and descriptions the command 
%    |\funcshadingstyle| can be
%    employed. Usage:
%    \medskip
%    
%    \quad|\funcshadingstyle{|\meta{residue}|}{|\meta{res.col.}|}{|\meta{shad.col.}|}|
%    
%    \hfill|{|\meta{case}|}{|\meta{style}|}|
%    \medskip
%    
%    \meta{residue} is one representative of the whole amino acid group. The 
%    colors which are declared by the next four parameters are used 
%    for all residues in this group. \meta{case} and \meta{style} are 
%    as described for example in |\nomatchresidues|.
%    \end{enumerate}
%    
%       With |\shadeallresidues| \label{Lshadeallresidues} the 
%       threshold is ignored and
%       all residues are shaded due to their group assignment.
%
%    \subsection{Creating new functional shading modes}
%    
%    The grouping of amino acids due to other properties can make sense as
%    suggested by \textsc{Karlin} and \textsc{Ghandour} [2], e.\,g.\ 
%    physical properties (molecular weight, shape), kinetic properties
%    (reaction velocity, Michaelis-Menton constant), or structure
%    ($\alpha$-helices, $\beta$-sheets, turns).
%    
%    \label{Lfuncgroup}
%    New amino acid groups are defined with the
%    |\funcgroup| command. This command needs six parameters:
%    \medskip
%    
%    \quad|\funcgroup{|\meta{descr}|}{|\meta{residues}|}{|\meta{res.col.}|}{|\meta{shad.col.}|}|
%    
%    \hfill|{|\meta{case}|}{|\meta{style}|}|
%    \medskip
%    
%    \meta{descr} contains descriptive text which is displayed in the legend.
%    The second parameter \meta{residues} holds the amino acids to be
%    grouped. The colors for the foreground and background are set 
%    with the following two parameters, the case and style is declared by the 
%    last parameters. The example below defines a 
%    funcional group named `acidic ($-$)' containing the amino acids
%    aspartic and glutamic acid with white letters on a red background:
%    \bigskip
%    
%    \quad|\funcgroup{acidic ($-$)}{DE}{White}{Red}{upper}{up}|
%    \bigskip
%    
%    For the usage of colors see section \ref{colors}. Up to nine
%    individual groups can be defined. New groups are simply added to the
%    already existing groups, i.\,e.\ if an extension of the group
%    definitions of an existing shading mode is desired there is
%    no need to clear these groups und re-define them again. Just 
%    add the new groups with the |\funcgroup| command. To create 
%    completely new modes use the command
%    |\shadingmode{functional}| without an option 
%    \emph{before} setting the new groups. The new definitions are active
%    only in the functional shading mode---so be sure to
%    have it switched on before setting the new groups.
%    Remember, |\shadingmode{functional}| without an optional parameter
%    clears all groups defined before, see above. The following example
%    shows the definitions needed to produce an output which is identical 
%    to the functional mode `charge':
%    \bigskip
%    
%    \quad|\begin{texshade}{|\meta{alignmentfile}|}|
%    \medskip
%    
%    \quad\quad |\shadingmode{functional}|
%    
%    \quad\quad |\funcgroup{acidic ($-$)}{DE}{White}{Red}{upper}{up}|
%    
%    \quad\quad |\funcgroup{basic ($+$)}{HKR}{White}{Blue}{upper}{up}|
%    \medskip
%    
%    \quad|\end{texshade}|
%    
%    
%    \subsection{Appearance of the consensus line}
%    
%    \label{Lthreshold}
%    An important parameter for the calculation of the consensus is the
%    threshold percentage. Default setting is 50\%, i.\,e.\ to become
%    the consensus residue more than half of the residues at this
%    position must be identical or similar, depending on the shading
%    mode. Any percentage between 0 and 100 is allowed and can be 
%    set with 
%    |\threshold{|\meta{percentage}|}|, e.\,g.\ |\threshold{50}|.
%
%    Additionally, an optional parameter can be set, e.\,g.\,
%    |\threshold[90]{50}|, to label residues that are highly conserved
%    in a special color (see example on page \pageref{ident}).
%    
%    \label{Lconstosingleseq}
%    Another possibility is to set one sequence of the alignment
%    as consensus and 
%    compare the other sequences to this one. Therefore, the
%    command 
%    |\constosingleseq{|\meta{seqref}|}| is provided. The
%    \meta{seqref} selects the sequence to be used as consensus
%    (numbering according to the appearance in the alignment file;
%    top sequence is number~1, or use the sequence name). 
%    Nevertheless, the threshold percentage is also taken into 
%    account, i.\,e.\ with a threshold of 50\% half
%    of the sequences must be identical or similar compared to the 
%    specified consensus sequence in order to be shaded.
%    \label{Lconstoallseqs} With |\constoallseqs| the 
%    consensus is calculated considering all sequences (the case 
%    described in the paragraph above).
%    
%    \label{Lshowconsensus}\label{Lhideconsensus}
%    \label{Lnameconsensus}
%    Consensus lines are displayed either on the top or at the bottom
%    of the alignment by calling 
%    \medskip
%
%    |\showconsensus[|\meta{color/scale}|[,|\meta{color/scale}|]]{|\meta{position}|}| 
%    \medskip
%
%    with 
%    \meta{scale} |Gray|, |BlueRed|, |RedBlue|, |GreenRed|, |RedGreen|,
%    |ColdHot| (recommended), |HotCold|, or |T-Coffee| \ref{TCoffee}, and \meta{position} |top| 
%    or |bottom|. 
%
%    The first color defines the foreground, i.e. the letters, the
%    second color---if specified---defines the background. 
%    If a color scale is named the consensus will be shaded according
%    to the level of sequence conservation (see section \ref{resweight} on residue 
%    weight tables below). For an example output see page
%    \pageref{shadecons}. You can find more information on color scales 
%    on page \pageref{Lgraphs}. The calculated consensus colors can be exported
%    as a Pymol [8] \label{Lexportconsensus}
%    script by |\exportconsensus[|\meta{filename}|]{|\meta{seqref}|}|.
%    If no \meta{filename} is specified |export.txt| will be used. The 
%    generated file can be opened in Pymol in order to shade a 3D model
%    of the sequence \meta{seqref}.
%
%    To hide the consensus use 
%    |\hideconsensus|. The consensus
%    line is named `consensus' in english texts, `consenso' in spanish
%    or `Konsensus' if the |german.sty| is used. With 
%    |\nameconsensus{|\meta{name}|}| any name can be set.
%    
%    \label{Ldefconsensus} 
%    You can tell \TeXshade{} which symbols or letters to use in
%    the consensus line for different matching qualities by 
%    \bigskip
%    
%    \quad|\defconsensus{|\meta{symbol1}|}{|\meta{symbol2}|}{|\meta{symbol3}|}|.
%    \bigskip
%    
%    The following parameters are allowed for symobols 1--3:
%    
%    \begin{enumerate}
%    
%    \item \meta{symbol1} = no match symbol (if below threshold)
%    
%          \begin{itemize}
%             \item any character or letter
%             \item |{}| (empty braces) for blank space
%          \end{itemize}
%          
%    \item \meta{symbol2} = conserved symbol (if threshold is exceeded)
%    
%          \begin{itemize}
%             \item |upper| (prints the consensus residue in uppercase)
%             \item |lower| (prints the consensus residue in lowercase)
%             \item any character or letter
%             \item |{}| (empty braces) for blank space
%          \end{itemize}
%          
%    \item \meta{symbol3} = highly conserved symbol (if
%
%                                 \hfill|\allmatchspecial| is active)
%    
%          \begin{itemize}
%             \item see \meta{symbol2}
%          \end{itemize}
%    
%    \end{enumerate}
%    
%    Example: |\defconsensus{{}}{*}{upper}| does not show non matching
%    residues in the consensus line, marks conserved residues 
%    with `|*|', and displays the uppercase letter of the consensus
%    residue at positions with high conservation.
%    
%    
%    \label{Lconsensuscolors}
%    Finally, the colors of the above defined symbols are adjustable
%    by the command: 
%    
%    \begin{tabbing}
%    \quad|\consensuscolors|\=|{|\meta{res.col.1}|}{|\meta{shad.col.1}|}|\\
%    
%    \>|{|\meta{res.col.2}|}{|\meta{shad.col.2}|}|\\
%    
%    \>|{|\meta{res.col.3}|}{|\meta{shad.col.3}|}|\\
%    \end{tabbing}
%    
%    The color definitions are in the same order as in the
%    |\defconsensus| command:
%    
%    \begin{enumerate}
%    
%    \item \meta{res.col.1} = no match residue color (if below threshold)
%    
%          \meta{shad.col.1} = no match background color 
%          
%    \item \meta{res.col.2} = conserved residue color (if threshold is exceeded)
%    
%          \meta{shad.col.2} = conserved background color 
%          
%    \item \meta{res.col.3} = highly conserved residue color (if 
%
%                                 \hfill|\allmatchspecial| is active)
%    
%          \meta{shad.col.3} = highly conserved background color 
%    
%    \end{enumerate}
%    
%    For colors which are not to be changed empty braces can be used.
%    
%    Example:\medskip
%    
%    \quad|\consensuscolors{}{}{Blue}{White}{Red}{Green}|
%    \medskip
%    
%    Non matching symbol colors are not changed, 
%    conserved residues are displayed blue on white and highly conserved residues
%    appear as red symbols on a green background in the 
%    consensus line.
%    
%    
%    \subsubsection{Residue weight tables} \label{resweight}
%    
%    The degree of similarity between two amino acid residues is defined using
%    so-called \emph{residue weight tables}. The values usually range roughly
%    from $-10$ to 10, with positive values denoting similarity and negative
%    values dissimilarity. The most simple table sets pairs
%    of identical residues to a value of 10 and all others to 0, i.e. the
%    |identity| matrix. Several more matrices based on extensive protein alignments
%    exist and can be used, e.g. |PAM250| (|Point Accepted Mutations|), |PAM100|,
%    or |BLOSUM62| (|BLOcks of amino acid SUbstitution Matrix|); for details see
%    respective sources and section \ref{weightmatrix}. \TeXshade{} further contains a 
%    |structural| matrix where 
%    similarity is defined on simple comparisons of the sidechain properties with
%    respect to volume and hydropathy.
%    
%    For calculation of the consensus color shading or for bar graphs or color scales
%    in the |\feature| lines (\ref{Lfeature}), a residue weight table can be selected
%    by \label{Lweighttable} |\weighttable{|\meta{table}|}| with \meta{table}
%    being |identity|, |structural|, |PAM250|, |PAM100|, or |BLOSUM62| (default is
%    |identity|). Which matrix suits the analysis best needs to be decided case by case.
%    Due to the all-positive values of the |structural| matrix (section \ref{weightmatrix}) the similarity level
%    appears usually very high; the |identity| matrix simply represents the number
%    of identical residues at each position. The |PAM| and |BLOSUM| matrices provide
%    more differentiated results.
%    One can change individual values or even define his own weight table
%    using the command \label{Lsetweight} 
%    |\setweight{|\meta{res.1}|}{|\meta{res.2}|}{|\meta{value}|}|, e.g.
%    |\setweight{E}{Q}{2}| or |\setweight{K}{C}{-5}|. A full table, thus, needs 200
%    entries ($20 * 20 / 2$). \label{Lgappenalty}
%    A value for the gap penalty is set with |\gappenalty{|\meta{value}|}|.
%    
%    \subsection{Display of logos}
%    
%    \subsubsection{Sequence logos}
%    
%    \label{Lshowsequencelogo}\label{Lhidesequencelogo}
%    In a sequence logo [7], the information content $I(P_i)$ of 
%    each alignment position $i$ is defined as 
%
%    \[ 
%    I(P_i) = \log_2 \vert\Sigma\vert + \sum P_{ij} \cdot \log_2 P_{ij}
%    \]
%
%    \noindent
%    with $\vert\Sigma\vert$ being the cardinality of the used alphabet, 
%    i.\,e. 4 for DNA and 20 for protein sequences, and $P_{ij}$ 
%    being the frequency of residue $j$ at this position. Each position 
%    is displayed as a stack of residue symbols whose heights 
%    represent their proportion of the information content (example on
%    p.\pageref{logo}).
%
%    The display of sequence logos can be either on the top or at the bottom
%    of a nucleotide or protein alignment. Logos will be shown after the
%    command: |\showsequencelogo[|\meta{colorset}|]{|\meta{top/bottom}|}|. If no optional
%    \meta{colorset} is selected the residues will be shaded as follows:\medskip
%
%    \begin{itemize}
%    \item Nucleotide sequences
%
%    \begin{itemize}
%      \item[G]: Black
%      \item[A]: Green
%      \item[T,U]: Red
%      \item[C]: Blue
%    \end{itemize}
%    
%
%    \item Protein sequences (similar to rasmol)
%
%    \begin{itemize}
%      \item[D,E]: Red
%      \item[C,M]: Yellow
%      \item[K,R]: Blue
%      \item[S,T]: Orange
%      \item[F,Y]: MidnightBlue
%      \item[N,Q]: Cyan
%      \item[G]:   LightGray
%      \item[L,V,I]: Green
%      \item[A]: DarkGray
%      \item[W]: CarnationPink
%      \item[H]: CornflowerBlue
%      \item[P]: Apricot
%      \item[B,Z]: LightMagenta
%    \end{itemize}
%    \end{itemize}
%    
%    Optional color sets correspond to the functional shading modes
%    |chemical|, |rasmol|, |hydropathy|, |structure|, |standard area|, 
%    |accessible area| (see p.\pageref{funcdef}). The |\showsequencelogo| 
%    command can be reversed by |\hidesequencelogo|.
%
%    \label{Llogocolor}\label{Lclearlogocolors}
%    Logo colors can be turned to `Black' with the command
%    |\clearlogocolors[|\meta{color}|}| with the optional parameter
%    not set. The optional parameter can be used to set all 
%    residue colors to \meta{color}, e.g.\ |\clearlogocolors[Blue]|.
%    User specific logo color sets are defined by using
%    |\logocolor{|\meta{residues}|}{|\meta{color}|}|, e.g.\ 
%    |\logocolor{DE}{Red} \logocolor{CM}{Yellow}| etc.
%
%    \label{Ldofrequencycorrection}\label{Lundofrequencycorrection}
%    It is common practice for protein sequence logos to correct
%    amino acid frequencies to the background frequency in the
%    alignment, which usually differs from the equal distribution
%    of 5\% for each residue. Frequency correction can be turned on
%    by |\dofrequencycorrection| and off by |\undofrequencycorrection|.
%
%    \label{Llogostretch}The vertical extent of the logo can be changed by 
%    |\logostretch{|\meta{factor}|}|, e.g.\ |\logostretch{1.5}|.
%    The width of the logo characters is dependent on the character
%    width set for the alignment, see |\charstretch| on p.\pageref{Lcharstretch}.
%
%    \label{Lshowlogoscale}\label{Lhidelogoscale}Finally, the bit-scale 
%    can be turned off and on using |\hidelogoscale| and
%    |\showlogoscale[|\meta{color}|]{|\meta{position}|}|, respectively, with
%    \meta{position} |left|, |right|, or |leftright| and an optional
%    \meta{color}. 
%    \label{Lnamesequencelogo}
%    A name for the sequence logo can be set, which is displayed
%    next to the scale by |\namesequencelogo{|\meta{name}|}|.
%
%
%    \subsubsection{Subfamily logos}
%
%    Subfamily logos provide a novel tool to visualize
%    subfamily-specific sequence deviations at alignment positions with
%    a high information content in an intuitive way [14].
%
%    This is achieved by subtracting from the frequency of a residue within 
%    a pre-defined subset of sequences, i.\,e. a subfamily, the frequency of 
%    this residue in the remaining set of sequences. The difference is then 
%    weighted by the information content, see above section on sequence logos.
%    An example is shown on p.\pageref{sublogo}.
%
%    Subtraction of frequencies produces values from $-1$ to $1$. Positive 
%    values correspond to residues which are characteristic for the subfamily 
%    (shown upright in the output), negative values to those that are typical 
%    for the remaining sequences (shown upside-down). Positions with an equal 
%    distribution of the residue result in a zero value. 
%
%    \label{Lshowsubfamilylogo}\label{Lhidesubfamilylogo}\label{Lsetsubfamily}
%    Subfamily logos are displayed analogous to sequence logos by the command 
%    |\showsubfamilylogo[|\meta{colorset}|]{|\meta{top/bottom}|}| and hidden by
%    |\hidesubfamilylogo|. To calculate a subfamily logo, it is further required
%    to define a subfamily within the alignment by 
%    |\setsubfamily{|\meta{seqrefs}|}|, e.g. |\setsubfamily{1-10,20,AQP3}|. 
%
%    For coloring residues, display/stretching of the scales, and frequency 
%    correction the same commands as for sequence logos apply with two exceptions.
%    \label{Lshownegatives}\label{Lhidenegatives}
%    First, subfamily logos contain negative values, which can be displayed
%    |\shownegatives[|\meta{weak, medium, strong}|]| or hidden 
%    |\hidenegatives|. Without the optional parameter negative residues will
%    be tinted by 50\%, i.e. |medium|. This greatly improves readability. 
%    \label{Lnamesubfamilylogo}
%    Second, a name for the subfamily logo is set by 
%    |\namesubfamilylogo[|\meta{neg.name}|]{|\meta{pos.name}|}| with a required
%    name for the positive part of the logo and an optional name for the negative
%    part.
%
%    \label{Lrelevance}\label{Lshowrelevance}\label{Lhiderelevance}
%    In order to better recognize relevant positions in the subfamily logo, a 
%    bit-value can be set above which the deviation is considered relevant
%    by the command |\relevance{|\meta{bit-value}|}|. If this command is
%    not given 2.321\,bit is assumed for proteins, i.\,e.
%    $\log_2 5$, and 1\,bit for DNA, i.\,e. $\log_2 2$. Such positions will 
%    be labeled by 
%    |\showrelevance[|\meta{color}|]{|\meta{symbol}|}|, e.\,g.
%    |\showrelevance[Blue]{$\nabla$}|. The symbol will be hidden with
%    |\hiderelevance|. 
%
%    \subsection{Appearance of the sequence lines}
%    
%    \label{seqlines}
%    
%    \subsubsection{Names, numbers and gaps}
%     \label{Lshownames}\label{Lshownumbering}
%    Many parameters that influence the appearance of the actual sequence
%    lines can be changed for customization.
%    Thus, the sequence names can be shown colored via \meta{color}
%    either left or right by
%    \medskip
%    
%    \quad|\shownames[|\meta{color}|]{|\meta{position}|}|
%    \medskip
%
%    with \meta{position} set to |left| or |right|. The numbering can be
%    displayed either left or right and even on both sides by
%    \medskip
%    
%    \quad|\shownumbering[|\meta{color}|]{|\meta{position}|}|
%    \medskip
%    
%    with \meta{position} |left|, |right| or |leftright|. Both,  
%    names and numbering can be displayed on the same side.     
%    \label{Lnamescolor}\label{Lnumberingcolor}    
%    The colors can also be set with |\namescolor{|\meta{color}|}| and 
%    |\numberingcolor{|\meta{color}|}|, respectively. 
%
%    \label{Lnameseq}
%    \TeXshade{} uses the sequence names from the 
%    alignment input file. This can cause some
%    problems during the \TeX-run when special characters are present
%    in those names! \TeXshade{} does not accept the following characters
%    in sequence names: |\ { } @| spaces and the tilde. Those have to be replaced in 
%    the input file. The characters |#| and |%| can only be used with a
%    leading backslash, e.\,g. |\#|. This must also be changed in the
%    input file. All other special characters should be displayed 
%    properly.    
%    
%    Sequence names that are accepted by \TeXshade{} can further be
%    changed in the |texshade| environment: 
%    \medskip
%    
%    \quad|\nameseq{|\meta{seqref}|}{|\meta{name}|}| 
%    \medskip
%    
%    \meta{seqref} selects the sequence whose name is to be changed. 
%    The basis for the \meta{seqref} is the appearance in
%    the alignment input file with the top sequence = 1, or the old 
%    name. 
%    \label{Lnamecolor}\label{Lnumbercolor}    
%    In order to change the colors only of some sequence names or numbers 
%    the commands 
%    |\namecolor{|\meta{seq1}|, ... ,|\meta{seq n}|}{|\meta{color}|}| and 
%    |\numbercolor{|\meta{seq1}|, ... ,|\meta{seq n}|}{|\meta{color}|}| 
%    are provided.
%    
%    \label{Lhidenames}\label{Lhidename}
%    \label{Lhidenumbering}\label{Lhidenumber}
%     In order to hide all names or the numbering use the command
%     |\hidenames| or |\hidenumbering|. If only the names or numbers of
%     some sequences should be hidden apply 
%     
%     |\hidename{|\meta{seq1}|, ... ,|\meta{seq n}|}| or
%     
%     |\hidenumber{|\meta{seq1}|, ... ,|\meta{seq n}|}|, respectively.
%    
%    \label{Lstartnumber} \label{Lallowzero} \label{Ldisallowzero}
%    In some situations, e.\,g.\ when only sections of sequences are 
%    displayed, one
%    may not want to have the residue numbering start out with number~1.
%    The command 
%    |\startnumber[|\meta{start..stop}|]{|\meta{seqref}|}{|\meta{startnumber}|}|
%    allows one to set the starting number of any sequence to any value
%    incl.\ negative values but except `0' which is not usually used in
%    sequence numbering (the transition from negative to positive
%    values is like this: \ldots\ $-2$, $-1$, 1, 2 \ldots). If, however,
%    the use of the number `0' is wanted as sometimes in sequence logos
%    this can be turned on by |\allowzero| and off with |\disallowzero|.
%    The optional parameter can be used to truncate the sequence display
%    to a certain section (see also |\setends| below). 
%    
%    \label{Lseqlength}
%    \TeXshade{} needs to know the correct length of the sequences
%    to be able to break them right after the last residue. If
%    MSF files are used as an input the length is already given
%    but the calculation is usually wrong because the gaps are
%    also counted. Thus, \TeXshade{} counts the number of residues
%    during each run by itself and stores the values in the |.aux| file. That
%    means that it needs two runs to get the numbers right. Again,
%    this is only important if the gap symbol after the sequence end
%    should be suppressed, see below (|\hideleadinggaps|).
%    
%    If you know the correct length of the sequences you can use the
%    command 
%    \medskip
%    
%    \quad|\seqlength{|\meta{seqref}|}{|\meta{length}|}| 
%    \medskip
%    
%    in order to set the values by hand and have the gaps break 
%    properly already in the first \TeX{} run.
%    \medskip
%    
%    Example: |\seqlength{1}{346}| means that sequence no.~1 is 346 
%    residues long.
%    
%    
%     \label{Lshowruler}\label{Lhideruler}
%     \label{Lrulersteps}\label{Lrulercolor}
%     \label{Lrotateruler}\label{Lunrotateruler}
%     \label{Lnamerulerpos}\label{Lrulerspace}
%     Another possibility to label sequence positions is to switch
%    on a ruler on the top or at the bottom of the sequence block
%    using \label{ruler}
%    |\showruler[|\meta{color}|]{|\meta{position}|}{|\meta{seqref}|}|. 
%    The residue ruler of one sequence \meta{seqref} or the consensus 
%    (declare `|consensus|' as \meta{seqref}) can be 
%    displayed at \meta{position} |top| or |bottom|.
%    The ruler is hidden with  |\hideruler|. The steps between two
%    numbers are set by |\rulersteps{|\meta{number}|}|. If the steps
%    are set to be very close ($< 4$) or when every position is numbered, the
%    numbering is automatically rotated by 90$^\circ$. Using |\rotateruler|
%    and |\unrotateruler| this can be done and undone manually. 
%    In order to change the
%    ruler color use the optional parameter or the command 
%    |\rulercolor{|\meta{color}|}|. Also, the label and its color at individual 
%    ruler positions can be changed by the user to a string using 
%    |\namerulerpos{|\meta{number}|}{|\meta{text}|[|\meta{color}|]}| 
%    (see example on p.\ \pageref{ras}). Finally, to adjust the distance
%    between the ruler and the top or bottom sequence row use
%    |\rulerspace{|\meta{length}|}|, e.\,g. |\rulerspace{1mm}|.
%    
%    \label{Lgapchar}\label{Lgaprule}
%    \label{Lgapcolors}\label{gapchar}
%    Further, the symbol which is displayed in sequence gaps is freely 
%    selectable with  
%    |\gapchar{|\meta{symbol}|}|. \meta{symbol} can be any character
%    or symbol. If math symbols are to be used math mode must be
%    activated by |$| characters, i.\,e. |\gapchar{{$\triangle$}}|.
%    Note the double curly braces in the last command. Everytime a
%    `complex' character is used, i.\,e. a character definition consisting 
%    of more than one letter, it must be braced in order to be interpreted as one 
%    character. One exception is |\gapchar{rule}|; with this 
%    parameter lines are drawn in the sequence gaps with a certain 
%    thickness defined by |\gaprule{|\meta{thickness}|}|, e.\,g.
%    |\gaprule{1.5pt}|. The colors of the gaps and gap symbols are set by 
%    |\gapcolors{|\meta{symbol color}|}{|\meta{background color}|}|.
%    
%    There are some discussions whether or not to display gap symbols before
%    and after the actual sequence. Since v1.3a one can control the
%    appearance of those gap symbols by the commands
%    \label{Lshowleadinggaps} \label{Lhideleadinggaps}
%    |\showleadinggaps| and |\hideleadinggaps|. By default, leading 
%    gaps are indicated by symbols despite my personal
%    thinking that it could suggest that 
%    there are some not displayed residues upstream resp.\ downstream of the 
%    gap.
%    
%    At certain instances a protein alignment input file may contain stop 
%    positions, e.\,g. due to frame shifts in the underlying DNA sequence.
%    If such positions are labeled in the input with an |*| this will be 
%    shown in the output as well as an asterisk, i.\,e. distinguishable from
%    a normal gap symbol. The character shown at stop positions can be
%    changed by |\stopchar{|\meta{symbol}|}|. \label{Lstopchar}
%    
%    \subsubsection{Displaying selected residues in the alignment}
%    
%    \label{Lsetends}
%    \TeXshade{} can display a section of the complete alignment
%    without the need to edit the alignment input file or even 
%    to re-calculate
%    the entire alignment. This allows one to use one single 
%    alignment of the full length proteins or open reading frames for
%    multiple visualizations of different sections in a document as
%    done in this manual. Thus, the file |AQPpro.MSF| contains
%    the full-length multiple protein alignment of five aquaporins but
%    only sections are displayed as examples in 
%    \ref{ident} through \ref{accarea}. The definition of a section
%    is done by  
%    \medskip
%    
%    \quad|\setends[|\meta{startnumber}|]{|\meta{seqref}|}{|\meta{start..stop}|}|.
%    \medskip
%    
%    Again, \meta{seqref} is the sequence number based on the 
%    appearance in the alignment file, or the name; further, in order to use
%    the consensus as a measure for the sequence section the
%    string `|consensus|' as \meta{seqref} is accepted. The 
%    specified sequence is truncated at
%    positions \meta{start} and \meta{stop}. All other 
%    sequences are cut accordingly. If the number of the first
%    residue in the sequence is set to a new value with the 
%    |\startnumber| command (s.\,a.) this is taken into account. The 
%    \meta{startnumber} can be set as an optional parameter directly
%    in the |\setends| command as well.
%    \medskip
%
%    Some examples:
%    \medskip
%    
%    \quad a) |\setends{1}{21..100}|
%    \medskip
%    
%    \quad b) |\startnumber{1}{101} \setends{1}{121..200}|
%    \medskip
%    
%    Both commands select the same sequence section from the alignment but
%    numbering for sequence 1 starts at position~21 in the first example and at 
%    position~121 in the latter.
%    \medskip
%    
%    \quad c) |\setends[101]{1}{121..200}| equals example b.
%    \medskip
%    
%    \medskip
%    
%    \quad d) |\startnumber[121..200]{1}{101}| also equals example b.
%    \medskip
%    
%    \medskip
%    
%    \quad e) |\setends{consensus}{21..100}|
%    \medskip
%    
%    This may describe a very different section of the multiple
%    sequence alignment because the consensus counts every position
%    including gaps.
%    
%    \label{Lsetdomain}
%    The output can be even further restricted to
%    individually selected residues, e.g.\ to eliminate uninteresting alignment stretches or
%    to condense the output, by:
%    \medskip
%    
%    \quad|\setdomain{|\meta{seqref}|}{|\meta{selection}|}|
%    \medskip
%    
%    Here, \meta{seqref} denotes the reference sequence by its number or name.
%    This sequence is used to define the alignment positions \meta{selection} 
%    to be shown. 
%    \meta{selection} can have two different formats
%    depending on whether (a) the user wants to select the positions manually
%    or (b) \TeXshade{} is supposed to select the residues based on 3D
%    coordinates provided by a PDB file.
%    
%    To select the residues manually, the user provides a position list of
%    the following format:
%    \medskip
%    
%    \quad|{|\meta{start1}..\meta{stop1}|,|\meta{start2}..\meta{stop2}|,|\ldots|,|\meta{start n}..\meta{stop n}|}|
%    \medskip
%
%    For how to select positions by 3D coordinates
%    provided by a PDB file, see \ref{Lshaderegion}. 
%    \medskip
%    
%    Examples (see also p.\,\pageref{ident}ff):
%
%    \quad|\setdomain{1}{20..80}|
%    
%    \quad|\setdomain{consensus}{20..80,100..150,200..220}|
%    
%    \quad|\setdomain{AQP1}{point[6]:1FX8.pdb,173[side]}|
%    
%    \quad|\setdomain{3}{plane[0.5]:1JN4.pdb,66[CA],73[side],199[CA]}|
%    \medskip
%
%    
%    It is helpful to show a ruler (probably single-stepped, see p.\,\pageref{Lshowruler}) to 
%    label the residue positions. 
%    
%    The resulting gaps between sequence stretches are marked by a vertical rule, which
%    can be changed in thickness by 
%    \medskip
%
%    \quad|\domaingaprule{|\meta{thickness}|}|, e.\,g.\
%    |\domaingaprule{1pt}|. \label{Ldomaingaprule}
%    \medskip
%
%    Also, the colors can be set by 
%    \medskip
%
%    \quad|\domaingapcolors{|\meta{foreground}|}{|\meta{background}|}| \label{Ldomaingapcolors}
%    \medskip
%
%     e.\,g.\ |\domaingapcolors{Blue}{Yellow}|.
%    
%    
%    \subsubsection{Hiding, killing, separating and ordering}
%    
%    \label{kill}
%    
%    \label{Lhideseq}\label{Lhideseqs}\label{Lshowseqs}\label{Lkillseq}
%    If one or more sequences from the alignment input file should be used for
%    the calculation of the consensus but it is desired not to
%    display these sequences in the final output use the command 
%    |\hideseq{|\meta{seq1}|,|\meta{seq2}|,|\ldots|,|\meta{seq n}|}|. 
%    For consecutive sequence numbers a dash can be used, e.\,g. 
%    |\hideseq{1-3}| instead of |\hideseq{1,2,3}|. Decending series 
%    are also permitted, e.\,g. |\hideseq{3-1}|.
%    This command allows one for example to hide 
%    the sequence which has been defined as the consensus sequence
%    with |\constosingleseq|. When all sequences should be hidden, e.g. to
%    show a sequence logo alone, one can simply say |\hideseqs|. This
%    command is reversed by |\showseqs|.
%
%    In order to completely exclude sequences the command 
%    |\killseq{|\meta{seq1}|,|\meta{seq2}|,|\ldots|,|\meta{seq n}|}| is 
%    provided. Again, for number series the dash can be used (s.\,a.). The 
%    designated sequences are neither displayed nor
%    considered for the calculation of the consensus. This is
%    another possibility to comment out sequences in addition
%    to the use of an exclamation point infront of the |Name:|
%    definition in an MSF-file (see figure on page \pageref{commout}).
%    
%    \label{Ldonotshade}
%    The command 
%    |\donotshade{|\meta{seq1}|,|\meta{seq2}\ldots|,|\meta{seq n}|}| 
%    makes 
%    one or more sequences (remember the dash, s.\,a.) appear unshaded 
%    in black letters on white background.
%    This does not influence any other sequences or the consensus
%    calculation.
%    
%    \label{Lhideresidues}\label{Lshowresidues}
%    If a very graphical output of the sequences is desired, the 
%    residue symbols or letters can be blanked out by
%    |\hideresidues|. Now, only the shaded boxes are printed. 
%    In combination with |\gapchar{rule}| one obtains alignments
%    in a style \`a la Mondrian.
%    The residues reappear with |\showresidues|.
%    
%    \label{Lseparationline}\label{Lsmallsep}
%    \label{Lmedsep}\label{Lbigsep}
%    \label{Lvsepspace}
%    If an alignment contains members of several subgroups of a
%    protein or a gene family it may be rather helpful to visualize the group
%    divisions by a separation line. Therefore, the command
%    |\separationline{|\meta{seqref}|}| is applicable. This 
%    command inserts vertical space after the sequence which is
%    refered to by \meta{seqref}. How much space is inserted
%    is defined by one of the following commands: 
%    |\smallsep|, |\medsep| (default) or |\bigsep|. These lengths
%    correspond to the known |\small|-, |\med|- and |\bigskip| commands. 
%    With |\vsepspace{|\meta{length}|}| any length with any
%    \TeX{} unit can be assigned, e.\,g. |\vsepspace{2mm}|.
%    
%    \label{Lorderseqs}
%    The sequence order given by the alignment input file is easily
%    reorganized by 
%    |\orderseqs{|\meta{seq1}|,|\meta{seq2}|,|\ldots|,|\meta{seq n}|}|
%    without the need for editing the alignment input file (which 
%    would be a big copy'n'paste job).
%    Make sure that all sequences are assigned in this
%    command. If there are more sequences present than numbers or names in the
%    command an error message will occur. Here also, the dash can be 
%    used for sequence number series. Example: |\orderseqs{1-3,6-4,7}| 
%    is equivalent to |\orderseqs{1,2,3,6,5,4,7}|.
%    Reordering of sequences only changes the output; all commands using
%    the parameter \meta{seqref} are not influenced, because \meta{seqref} 
%    always corresponds to the appearance in the alignment file. Thus, 
%    to completely reverse the order of a five sequence alignment simply type 
%    |\orderseqs{5-1}|.
%    
%    
%    \subsubsection{Residues per line and further settings}
%    
%    \label{Lresiduesperline}\label{Lresiduesperline*}
%    By default \TeXshade{} puts the highest possible by five 
%    divisible number of residues in one line depending on the 
%    |\linewidth|. With |\residuesperline{|\meta{number}|}| a new 
%    value can be set. If this value exceeds the highest possible 
%    number of residues per line it is ignored; lower values are
%    accepted of course. But also in the latter case the number
%    of residues printed per line is rounded such to be divisible by five. 
%    To force \TeXshade{}
%    to set lines with exactly the desired number of residues use
%    the asterisk-extended command |\residuesperline*{|\meta{number}|}|. 
%    You have to take care yourself of the alignment width after this command, 
%    because in this mode \TeXshade{} does not check the length of the 
%    lines any more.
%    
%    \label{Lcharstretch}\label{Llinestretch}
%    \TeXshade{} calculates the dimensions of a shaded box from
%    the width and height of the uppercase letter `M' and the depth of
%    the lowercase `g'. Depending on the font used for the
%    sequence residues the box dimensions might not be fully
%    satisfactory. With |\charstretch{|\meta{factor}|}| and
%    |\linestretch{|\meta{factor}|}| the width and height/depth,
%    respectively, of the boxes can be multiplied individually by a 
%    \meta{factor} to stretch ($>1$) or shrink ($<1$) the dimensions.
%    
%    \label{Lnumberingwidth}
%    The reserved space for the sequence numbering is set by the
%    command |\numberingwidth{|\meta{n digits}|}|. Here, the default setting
%    is four-digit numbering, i.\,e.\ $-999$ through 9999. If this range
%    is to be changed assign the desired number as parameter
%    \meta{n digits}, e.\,g.\ |\numberingwidth{111111}| reserves
%    space for 6 digit numbering.
%    
%    The vertical space between the sequence blocks can be controlled
%    by the commands |\smallblockskip|, |\medblockskip| (default 
%    setting), 
%    \label{Lsmallblockskip}\label{Lmedblockskip}
%    \label{Lbigblockskip}\label{Lnoblockskip}
%    \label{Lvblockspace}
%    |\bigblockskip| or |\noblockskip|. Further, the command 
%    |\vblockspace{|\meta{length}|}| allows one to set a defined space
%    length using any \TeX{} unit, e.\,g.\ |\vblockspace{0.4in}|.
%    
%    Two more commands set the space between the sequence blocks to be
%    \label{Lflexblockspace}\label{Lfixblockspace}
%    flexible (|\flexblockspace|) (default) or fixed (|\fixblockspace|).
%    Flexible means, that only the vertical white space between the
%    blocks is kept to the settings by
%    e.\,g. |\medblockskip|. This results in flexible space between 
%    the actual blocks depending on the presence of feature lines. When 
%    switching to fixed space the distance of the blocks is kept constant
%    by using more white space between blocks without feature lines.
%    Thus, a difference between flexible and fixed space will only be 
%    noticeable when features are used.
%    
%    \label{Lalignment}
%    The position of the output can be aligned left, right
%    or centered on the page by |\alignment{|\meta{position}|}| 
%    with the \meta{position} parameter |left|, |center| or 
%    |right|.
%    
%    
%    
%    \subsubsection{Fingerprinting}
%    
%    \label{fingerprint}
%    
%    \label{Lfingerprint}
%    An easy way to gain an overview on complete alignments is 
%    provided by displaying a so called alignment `fingerprint'.
%    In this style the whole sequence can be shown in one line. Due to
%    the lacking space the residue names are hidden and the shaded
%    boxes are reduced to thin vertical colored lines. The command
%    |\fingerprint{|\meta{res. per line}|}| takes one argument stating
%    the desired number of residues per line, e.\,g. |\fingerprint{1000}|.
%    All \TeXshade{} commands are compatible with |\fingerprint|, 
%    i.\,e. all shading modes are applicable for displaying overviews
%    on similarity or every functional aspect. Also, all kinds of
%    labeling---as described in the following---work with this
%    command.
%    
%    
%    \subsection{Individual shading and labeling of sequence stretches}
%    
%    Computer calculated conservation shading is informative---but
%    even more information can be visualized by additional labeling
%    of positions and regions of interest with different colors,
%    text styles or graphical marks and descriptive text. All this 
%    is provided by easy to handle \TeXshade{} commands.
%    
%    
%    \subsubsection{Shading of regions and blocks}
%    \label{shaderegion}
%    
%    \label{Lshaderegion}
%    Besides the shading calculated by \TeXshade{} any region can be
%    additionally shaded with a color specified by the user. This is very
%    useful to highlight secondary protein modification
%    sites such as phosphorylation or glycosylation sites, or longer
%    motifs for example protein/protein interaction sites or protein domains.
%    This is done with the following command:
%    \medskip
%    
%    \quad|\shaderegion{|\meta{seqref}|}{|\meta{selection}|}{|\meta{res.col.}|}{|\meta{shad.col.}|}|
%    \medskip
%    
%    Here, \meta{seqref} refers to the sequence by its name or number
%    within the alignment. The 
%    foreground and background colors can be set with the last two 
%    parameters. 
%    \meta{selection} can have three different formats
%    depending on whether (a) the user wants to select the positions manually,
%    (b) the selection should be based on a sequence motif to be found by \TeXshade{}, 
%    or (c) \TeXshade{} is supposed to select the residues based on 3D
%    coordinates provided by a PDB file.
%    \medskip
%    
%    (a) To select the residues manually, the user provides a position list of
%    the following format:
%    \medskip
%    
%    \quad|{|\meta{start1}..\meta{stop1}|,|\meta{start2}..\meta{stop2}|,|\ldots|,|\meta{start n}..\meta{stop n}|}|
%    \medskip
%    
%    In order to shade residue number 13 and the region
%    20--30 of sequence number 1 in red letters on a green background 
%    type the following command:
%    \medskip
%    
%    \quad|\shaderegion{1}{13..13,20..30}{Red}{Green}|
%    \medskip
%    
%    If the consensus is to be shaded use |consensus| as 
%    \meta{seqref}.
%    \medskip
%    
%    (b) A sequence motif can be given, which will be found and labeled. A simple 
%    example selecting all sequence motifs `NNAD' in sequence `1' would be:
%    \medskip
%
%    \quad|\shaderegion{1}{NNAD}{Red}{Green}|
%    \medskip
%
%    The definition can further include `X' for any residue and groups of residues in brackets
%    at uncertain positions. The example below will find all motifs in sequence `1'
%    that start with an asparagine, followed by any two residues, an acidic, a
%    basic residue, again any two residues and finally a glutamine.
%    \medskip
%    
%    \quad|\shaderegion{1}{NXX[DE][KR]XXQ}{Red}{Green}|
%    \medskip
%    
%    (c) In order to select positions based on the 3D structure a PDB
%    structure file is required. \TeXshade{} can select residues within
%    a given distance in \AA{} around a point, along a line, or above and
%    below a plane, which are defined by one to three residues. The points 
%    can be further specified to be the $\alpha$-carbon atom, i.\,e.\ the
%    protein backbone, or the most distant atom of the sidechain. Accordingly,
%    \meta{selection} has one of the formats:
%    \medskip
%
%    \quad|{point[|\meta{dist}|]:|\meta{file}|,|\meta{num}|[CA/side]}|
%    \medskip
%    
%    \quad|{line[|\meta{dist}|]:|\meta{file}|,|\meta{num1}|[CA/side],|\meta{num2}|[CA/side]}|
%    \medskip
%
%    \vbox{%    
%    \quad|{plane[|\meta{dist}|]:|\meta{file}|,|\meta{num1}|[CA/side],|\meta{num2}|[CA/side],|
%    \medskip
%
%    \hfill\meta{num3}|[CA/side]}|
%    }
%    \medskip
%    
%
%    Example: in order to select and shade as above all residues that are within an 8\ \AA{} 
%    sphere around the $\alpha$-carbon of residue 81 and the data are
%    provided in the PDB file |1J4N.pdb|, type:
%    \medskip
%    
%    \quad|\shaderegion{1}{point[8]:1J4N.pdb,81[CA]}{Red}{Green}|
%    \medskip
%    
%    Example: two points denote a line, hence, give two residues to select 
%    everything within 1\ \AA{} along the line between the $\alpha$-carbon of
%    residue 81 and the sidechain of residue 168 with:
%    \medskip
%    
%    \quad|\shaderegion{1}{line[1]:1J4N.pdb,81[CA],168[side]}|
%    \medskip
%
%    \hfill|{Red}{Green}|
%    \medskip
%    
%    Definition of a plane follows the same format but requires three points. 
%    If the optional parameters |[\meta{dist}]| and |[CA/side]| are not given,
%    \TeXshade{} assumes |[1]| and |[side]|, respectively.
%
%    In case one needs the position numbers of the selected residues for usage
%    in other applications, those can be either printed in the \TeX{} document with
%    |\printPDBlist{|\meta{selection}|}| or shown during the \TeX{} run with
%    |\messagePDBlist{|\meta{selection}|}|. The commands can be used outside of the 
%    \TeXshade{} environment. \label{LprintPDBlist}\label{LmessagePDBlist}
%    
%    Both selection formats, i.\,e. a manually given list and the 3D selection,
%    can be used with |\shadeblock| (see below), 
%    |\tintregion|, |\tintblock|, |\emphregion|, |\emphblock|, |\lowerregion|, |\lowerblock|,
%    |\frameblock| (all in \ref{Lframeblock}), and |\feature| (\ref{Lfeature}).
%
%
%    \label{Lshadeblock}
%    In analogy to |\shaderegion| which is restricted to a single
%    sequence, |\shadeblock| shades the corresponding region in all 
%    other sequences as well
%    except the consensus. If also the consensus is to be shaded 
%    define the region using |consensus| as \meta{seqref}.
%    \medskip
%    
%    \quad|\shadeblock{|\meta{seqref}|}{|\meta{selection}|}{|\meta{res.col.}|}{|\meta{shad.col.}|}|
%    \medskip
%    
%    
%    Another option is to change the whole set of shading colors for certain
%    sequence blocks, e.\,g.\ from |blues| to |reds| or self-defined color sets (see \ref{Lshadingmode}).
%    Therefore, the following command was implemented: 
%    \medskip
%    
%    \quad |\changeshadingcolors{|\meta{seqref}|}{|\meta{selection}|}{|\meta{name}|}|
%    \medskip \label{Lchangeshadingcolors}
%
%    Examples:
%    \medskip
%
%    \quad |\changeshadingcolors{1}{10..50}{reds}|
%    \medskip
%
%    \quad |\changeshadingcolors{AQP1}{[AS]NKD}{my_set}|
%    \medskip
%
%    etc.
%    
%    \subsubsection{Emphasizing, tinting, lowercasing, and framing}
%    
%    \label{Lemphregion}\label{Lemphblock}
%    If it is preferred to keep the calculated shading colors 
%    but distinct regions or blocks are yet to be emphasized one
%    can use the following commands to change the font style of 
%    such regions:
%    \medskip
%    
%    \quad|\emphregion{|\meta{seqref}|}{|\meta{selection}|}|
%    \medskip   
%    
%    and
%    \medskip
%    
%    \quad|\emphblock{|\meta{seqref}|}{|\meta{selection}|}|
%    \medskip   
%    
%    For the format possibilities of the \meta{selection} parameter please see \ref{Lshaderegion}.
%
%    \label{Lemphdefault}
%    Which style \TeXshade{} uses for emphasizing regions is defined by 
%    |\emphdefault{|\meta{style}|}|. Default setting is the
%    \emph{italics} font shape (set by |\emphdefault{it}|). In order to change
%    this setting choose one of the styles |bf, md, up, it, sl, rm, sf, tt|.
%    
%    Example: |\emphdefault{bf}|
%    \medskip
%    
%    \label{Ltintregion}\label{Ltintblock}
%    Further, it is possible to tint the region or block in question
%    or to switch the characters to lowercase
%    by using the commands (for example see hydropathy-figure on page 
%    \pageref{hydro}):
%    \medskip
%    
%    \quad|\tintregion{|\meta{seqref}|}{|\meta{selection}|}|
%    \medskip   
%    
%    and
%    \medskip
%    
%    \quad|\tintblock{|\meta{seqref}|}{|\meta{selection}|}|
%    \medskip   
%    
%    \label{Ltintdefault}
%    The level of tinting in the region in question can be set by 
%    |\tintdefault{|\meta{level}|}| with |weak|, |normal|, and
%    |strong| as possible \meta{level}s.
%    \medskip
%    
%    \label{Llowerregion}\label{Llowerblock}
%    \quad|\lowerregion{|\meta{seqref}|}{|\meta{selection}|}|
%    \medskip   
%    
%    and
%    \medskip
%    
%    \quad|\lowerblock{|\meta{seqref}|}{|\meta{selection}|}|
%    \medskip   
%    
%    
%    Another option is to draw a bounding box around the sequence block
%    in question (for an example see diversity mode-figure on page 
%    \pageref{frame}) with the 
%    command:\footnote{Thanks to Alan Robinson for inspiration.}
%    \medskip\label{Lframeblock}
%
%    \quad|\frameblock{|\meta{seqref}|}{|\meta{selection}|}{|\meta{color}|[|\meta{length}|]}|
%    \medskip
%
%    With the optional parameter the default line thickness of the frame can
%    be changed, example: |\frameblock{1}{10..20,50..70}{Red[2pt]}|
%
%
%
%
%    \subsubsection{Graphical labeling of sequence features}
%    
%    \label{feature}
%    
%    \label{Lfeature}
%    The |\feature| command is designed to fulfill most needs for the
%    graphical labeling of sequence stretches and the setting of descriptive 
%    text. It needs five parameters:
%    \medskip
%    
%    \quad|\feature{|\meta{position}|}{|\meta{seqref}|}{|\meta{selection}|}{|\meta{labelstyle}|}{|\meta{text}|}|
%    \medskip
%    
%    In the following paragraphs all possible parameter settings of 
%    this rather complex but mighty command are discussed in detail.
%    The parameter \meta{position} tells \TeXshade{} where to display
%    the feature label, i.\,e. on the top of the alignment (|top|),
%    or at the bottom (|bottom|). Further, there can be three more feature lines
%    ontop of the top feature line (|ttop|, |tttop|, and |ttttop|) or below the bottom
%    feature line (|bbottom|, |bbbottom|, |bbbbottom|). Thus, up to eight features
%    overlapping in eight different lines may be displayed. 
%    Depending on the content of the feature lines the gaps between 
%    them might be not satisfactory. 
%    \label{Ltopspace}\label{Lttopspace}
%    \label{Ltttopspace}\label{Lttttopspace}
%    \label{Lbottomspace}\label{Lbbottomspace}
%    \label{Lbbbottomspace}\label{Lbbbbottomspace}
%    Therefore, eight separate commands can be employed to change the 
%    space below |ttttop|, |tttop|, |ttop|, or |top| 
%    (|\topspace{|\meta{length}|}| etc.), and above 
%    |bottom|, |bbottom|, |bbbottom|, or |bbbbottom| 
%    (|\bottomspace{|\meta{length}|}| etc.). Use positive
%    values to further separate the lines, e.\,g. 
%    |\ttopspace{3mm}| or negative values to reduce the space, e.\,g.
%    |\bottomspace{-0.1in}|.
%    
%    The argument \meta{seqref} and the third 
%    parameter containing the \meta{selection} of the specified residues
%    are identical to the ones described before in several commands, e.\,g.
%    |\shaderegion| (\ref{shaderegion}). 
%    
%    New is the fourth parameter for the definition of the label style.
%    There are many possibilities like braces, helices, boxes, arrows, bars, any
%    fill character, bar graphs, color scales or even translations of the 
%    specified regions. 
%    \medskip
%
%    \textbf{Braces:}\\
%    In order to display an over- or underbrace as
%    a label use the parameter |{brace}|. Depending on the 
%    \meta{position} (|ttttop|, |tttop|, |ttop|, |top|, |bottom|,
%    |bbottom|, |bbbottom|, or |bbbbottom|) the respective brace is
%    displayed. The standard color of braces is 
%    black. It can be changed by an optional parameter directly after 
%    the definition of the symbol, e.\,g. |{brace[Red]}|.
%    \medskip
%
%    \textbf{Protein $\alpha$-Helices:}\\
%    The parameter |{helix}| will plot a symbolized $\alpha$-helix
%    as a label. The standard color of the helix spiral is 
%    black. It can be changed by an optional parameter directly after 
%    the definition of the symbol, e.\,g. |{helix[Red]}|.
%    \medskip
%
%    \textbf{Filling a stretch with a symbol:}\\
%    A region can be filled with any character for
%    labeling purposes using the parameter |{fill:|\meta{symbol}|}|.
%    The \meta{symbol} is freely selectable; the usage is like
%    in |\gapchar| (\ref{gapchar}). Do not use spaces before or after 
%    the expression \meta{symbol}; this will shift the symbols to the
%    respective direction. The standard color of the fill symbol is 
%    black. It can be changed by an optional parameter directly after 
%    the definition of the symbol, e.\,g. |{fill:$\bullet$[Red]}|.
%    
%    The |\feature| command does not like special characters in
%    text mode, e.\,g. |\dag|. One has to use the math version of
%    those symbols between |$|-signs. The following quite common
%    text symbols have also a math equivalent\footnote{Thanks to 
%    Darrell Conklin for reporting this problem}:
%    
%    \begin{center}
%    \begin{tabular}{cll}
%    \emph{symbol} & \emph{command} & \emph{description} \\ \hline
%    $\dagger$ & |$\dagger$| & dagger\\
%    $\ddagger$ & |$\ddagger$| & double dagger\\
%    $\mathparagraph$ & |$\mathparagraph$| & paragraph mark\\
%    $\mathsection$ & |$\mathsection$| & section mark\\
%    $\mathdollar$ & |$\mathdollar$| & dollar\\
%    $\lbrace$ & |$\lbrace$| & left brace\\
%    $\rbrace$ & |$\rbrace$| & right brace\\
%    \end{tabular}
%    \end{center}
%    \medskip
%
%    \textbf{Labeling restriction or protease cutting sites:}\\
%    If a label is needed that points between two residues, e.\,g.
%    for showing restriction sites, simply use the feature style
%    |{restriction[|\meta{color}|]}|. This will show a filled 
%    triangle with the tip right between the residues to be labeled,
%    e.\,g. |\feature{top}{1}{25..26}{restriction[Blue]}{EcoR I}|.
%
%    \medskip
%    
%    \textbf{Boxes:}\\
%    Boxed text is printed using the parameter |{box:|\meta{text}|}|.
%    By default black letters in a white framed box are displayed. In 
%    order to change these colors optional parameters can be included
%    in the argument: 
%    \medskip
%
%    \quad|{box[|\meta{framecolor,boxcolor}|][|\meta{length}|]:|\meta{text}|[|\meta{textcolor}|]}|.
%    \medskip
%
%    If the box frame and fill colors are the same it is sufficient to
%    use only this one color as an argument in the command. The optional
%    parameter \meta{length} defines the thickness of the box frame. If
%    this parameter is not set in the command the value from the 
%    |\featurerule{|\meta{length}|}| command (see below) is used.
%    \medskip
%    
%    Examples: 
%    \medskip
%
%    \quad|{box[Blue]:$\alpha$-helix[Yellow]}|
%    \smallskip
%
%    \quad|{box[Blue,Red]:$\alpha$-helix[Yellow]}|
%    \smallskip
%
%    \quad|{box[Blue,Red][2pt]:$\alpha$-helix[Yellow]}|
%    \medskip
%
%    \medskip
%    
%    \textbf{Horizontal bars and arrows:}\\
%    For displaying bars and arrows a simple selection scheme 
%    consisting of three consecutive characters is
%    used as the \meta{labelstyle} parameter. Each bar or arrow is
%    defined by its left end, the middle part, and the right end.
%    The following table gives some examples for the construction
%    of arrows and bars.
%    
%    \begin{center}
%    \begin{tabular}{cl}
%    middle & \\
%    \hbox to 1.6cm{\hss left end} \raisebox{1mm}{$\downarrow$} \hbox to 1.6cm{right end} & \\ \hline
%    |---|& plain bar \\
%    |===|& double bar \\
%    |-->|& right arrow \\
%    |'->|& right arrow with up hook \\
%    |<-|$\vert$ & left \emph{maps to} arrow \\
%    |<-o| & left arrow with ball at right end\\
%    |<=>|& double arrow, two heads \\
%    |,-,|& plain bar with down hooks\\
%    $\vert$|=|$\vert$ & double bar with vertical ends\\
%    \end{tabular}
%    \end{center}
%    
%    All combinations of the left-end-characters 
%    (|-=<',|$\vert$o), the middle-characters (|-=|),
%    and the right-end-characters (|-=>',|$\vert$o) are
%    allowed and produce the desired arrow or bar.
%    The color is changed as described above.
%    \label{Lfeaturerule} The thickness can be generally 
%    set by the separate command |\featurerule{|\meta{length}|}|
%    with any \TeX{} measure as \meta{length}, e.\,g.\ |\featurerule{3pt}|. 
%    This value is then used for all arrows, bars, and boxes (see above)
%    throughout the alignment. If an individual thickness for a 
%    particular arrow should be set one can add an optional
%    parameter to the \meta{labelstyle} parameter, e.g.
%    |{o->[Red][1mm]}|. Similar to the boxes described above, a text can be put on
%    the arrow or bar, e.\,g.\ |{<->[Red][1mm]:$\beta$-sheet[Blue]}|.
%
%
%    In \TeXshade{} versions before v1.9, the original \LaTeX{}-arrows 
%    were used. These have now been replaced by more modern looking
%    arrows with scalable line thickness. If the classical look is
%    requested, use |v| instead of |<| or |>| in the arrow definition,
%    e.\,g.\ |{--v}|, to get them back. The new arrow style makes use of
%    of the AMS math symbol font (amssymb.sty). Thus, in order to 
%    display the arrow heads correctly make sure that this style is    
%    present on your system (usually it is in a common \LaTeX{} installation).
%    \medskip
%
%    \textbf{Sequence translations:}\\
%    With the option |{translate}|, sequence stretches can be 
%    translated from nucleotide to peptide sequences as well as 
%    backtranslations from peptide to nucleotide sequences are
%    possible. Default setting for the translations is the standard
%    genetic code. Of course, the codons can be re-defined by the
%    user. The command \label{Lcodon}
%    |\codon{|\meta{amino acid}|}{|\meta{triplet1, \ldots, triplet n}|}|
%    has been implemented for this issue. The usage is simple. Replace
%    \meta{amino acid} by the single letter code of the amino acid
%    to be defined and add a list of triplets for this residue.    
%    Example definition for the amino acid \emph{alanine}:
%    \medskip
%    
%    \quad |\codon{A}{GCA,GCG,GCC,GCT,GCU,GCN}|
%    \medskip
%    
%    Note the last triplet in the list. It contains an ambiguity code
%    |N| which stands for \emph{any} nucleotide. This triplet has been 
%    added at the last position because the last triplet is used
%    for the generation of the backtranslated nucleotide sequence from
%    a peptide. Two files are included in the \TeXshade{}
%    distribution as examples (|standard.cod, ciliate.cod|). If you 
%    want to define a new genetic code store your commands in a file 
%    like the examples. Such files with the suffix |.cod| can be 
%    loaded in the \TeXshade{} environment by \label{Lgeneticcode}
%    |\geneticcode{|\meta{filename}|}|, e.\,g. |\geneticcode{ciliate}|.
%    Do not designate the suffix |.cod| in \meta{filename}. Please 
%    note, when inspecting the example files, that only the exchanges
%    compared to the standard code need to be defined in a new genetic code file. 
%    
%    When DNA sequences are translated to protein the resulting amino
%    acids are aligned to the second nucleotide of each triplet. 
%    It is more difficult to produce a satisfactory display of
%    backtranslated nucleotide sequences due to the lack of space. 
%    You need thrice as much space than the original peptide sequence,
%    because single letter amino acid code is translated to a triplet
%    code. Therefore, the user can choose from five display styles
%    for backtranslations depending on personal preferences:
%    \medskip\label{Lbacktranslabel}
%    
%    \quad |\backtranslabel[|\meta{size}|]{|\meta{style}|}|, with
%    \medskip
%    
%    \begin{tabbing}
%    \qquad\qquad|{|\meta{style}|}|\ \= = |{horizontal}|\\
%       \> = |{alternating}|\\
%       \> = |{zigzag}|\\
%       \> = |{oblique}|\\
%       \> = |{vertical}|
%    \end{tabbing}
%    
%    \meta{size} can be any \TeX{} size from |tiny| up to |Huge|, but 
%    |tiny| is recommended (and default setting). Translations
%    can be colored as all other labels, see above.
%    \medskip
%
%    \textbf{Bar graphs and color scales:}\label{Lgraphs}\\
%    Sequence related numeral data, such as hydropathy or solvent
%    accessibility data etc., can be shown in a feature line as bar graphs
%    or color scales. The data are (a) pre-defined or calculated by 
%    \TeXshade{} due to amino acid properties or conservation, (b) are 
%    provided in a separate file or (c) may be entered by hand in the 
%    |\feature| command. 
%
%    (a) Currently, three different
%    properties can be plotted, i.e. |hydrophobicity|, |molweight|, and 
%    |charge|. Further, the level of sequence conservation at the given
%    protein sequence stretch can be shown (|conservation|). See 
%    \ref{resweight} for selecting an appropriate \emph{residue weight table}.
%
%    (b) The format of a data file is simple: every value must 
%    appear in a separate line. Numbers and the Java-typical `NaN' for
%    `Not a Number' are permitted. Comments are allowed, because \TeXshade{}
%    ignores all lines starting with a letter except `NaN' lines (avoid 
%    `|-|' as the first
%    character of a comment line as this is interpreted as a negative number).
%    Make sure that there are as many values as positions defined as the 
%    sequence stretch in the feature command.
%    \TeXshade{} will read this file and determine the minimal and maximal 
%    values. These data are then normalized for plotting.
%    Due to \TeX's limited calculation capabilities no values above 10\,737
%    are allowed and the difference between minimum and maximum must not
%    exceed this very number. Values below 0.001 may be susceptible to major
%    rounding errors. Thus, try to provide your data already normalized to
%    moderate scales, e.g. 0.0\,--\,1.0 or -100\,--\,100. 

%    (c) Data which is  
%    directly entered in the |\feature| command must be normalized to integer
%    values with a maximal difference of 100 between the highest and lowest
%    value, e.g. -50\,--\,50 or 0\,--\,100.
%
%    For (b) and (c), the range to be plotted can be set by hand as an optional parameter
%    in the |\feature| command. This can be necessary when the data file
%    contains values between e.g. $-0.44$ and $0.87$. Without help \TeXshade{} 
%    will assume $-0.44$ as minimum and $0.87$ as maximum. But if the actual 
%    range to be plotted should be $-1.0$\,--\,$1.0$ this needs to be set
%    manually, see examples below. Be aware of the fact, that if you 
%    define a scale by hand, which is more narrow than the values of the
%    input, this will stretch the bars accordingly. It is NOT recommended
%    to use this method for stretching bars vertically. Instead another
%    command has been introduced. 
%    \label{Lbargraphstretch}\label{Lcolorscalestretch}
%    The plotted bars can be stretched by a factor if the appearance is
%    not as desired: |\bargraphstretch{|\meta{factor}|}|. Here, the factor 
%    is multiplied with the bar length, e.g |\bargraphstretch{2}| will double
%    the bar height, |\bargraphstretch{0.5}| will make them half as high.
%    Similarly, color scales can be stretched vertically with 
%    |\colorscalestretch{|\meta{factor}|}|.
%
%    The default color of bar graphs is gray and can be changed by an 
%    optional parameter at the end of the |label| definition. Further, an optional 
%    background color can be chosen for the bars. Doing so will visualize
%    the maximal bar extension.
%
%    Default for
%    color scales is a 5\% gray scale from very light gray to black (|Gray|).
%    More colorful scales have been implemented, i.e. |BlueRed|, |RedBlue|, |GreenRed|,
%    |RedGreen|, |ColdHot| and |HotCold|, the latter two being particularly 
%    useful for ranges from negative to positive values. Further, a scale called |T-Coffee|
%    is available if |T-Coffee| shading information has been imported as the 
%    |\shadingmode| \ref{Lshadingmode}.
%
%
%
%    The general format of this feature label definition for bar graphs is:
%    \medskip
%
%    \quad |{bar[|\meta{min}|,|\meta{max}|]:|\meta{properties/file/data}|[|\meta{color(,bgcolor)}|]}|
%    \medskip
%
%    and for color scales:
%    \medskip
%
%    \quad|{color[|\meta{min}|,|\meta{max}|]:|\meta{properties/file/data}|[|\meta{scale}|]}|
%    \medskip
%
%    Some examples:
%    \medskip
%
%    \qquad |{bar:conservation}|
%    \medskip
%
%    \qquad |{bar:conservation[T-Coffee]}|
%    \medskip
%
%    \qquad |{bar:hydrophobicity}|
%    \medskip
%
%    \qquad |{bar:charge[Red]}|
%    \medskip
%
%    \qquad |{bar:molweight[Red,Gray10]}|
%    \medskip
%
%    \qquad |{bar:10,20,30,40,50[Red]}|
%    \medskip
%
%    \qquad |{bar[-20,40]:-10,0,10,20,30[Red,Gray10]}|
%    \medskip
%
%    \qquad |{bar:data.txt}|
%    \medskip
%
%    \qquad |{bar[-10,10]:data.txt[Red,Gray10]}|
%    \medskip
%
%    \qquad |{color:conservation[BlueRed]}|
%    \medskip
%
%    \qquad |{color:conservation[T-Coffee]}|
%    \medskip
%
%    \qquad |{color:hydrophobicity[GreenRed]}|
%    \medskip
%
%    \qquad |{color:charge}|
%    \medskip
%
%    \qquad |{color:molweight}|
%    \medskip
%
%    \qquad |{color[-10,10]:data.txt[ColdHot]}|
%    \medskip
%
%    \qquad |{color[-0.1,0.1]:otherdata.txt[ColdHot]}|
%    \medskip
%
%    See also the example output in section \ref{graphs} on page
%    \pageref{graphs}.
%
%    \medskip
%
%    \textbf{No graphical label, only text:}\\
%    If no graphical label is
%    wanted the fourth parameter of |\feature| can be empty
%    braces.
%    \medskip
%    
%    Finally, the fifth parameter of the |\feature| command contains 
%    the descriptive text
%    for the labeled region. Type whatever you want incl. symbols and
%    math chars. The text field can also contain sequence translations.
%    In this case just set \meta{text} = |{translate}|. There is a 
%    command for setting the size and style of backtranslated sequences
%    in the feature \meta{text} which corresponds to the one
%    described above: \label{Lbacktranstext}
%    \medskip
%    
%    \quad |\backtranstext[|\meta{size}|]{|\meta{style}|}|
%    \medskip    
%    
%    Again, the color can be set by an
%    optional parameter appended to the text. For how to change the 
%    font size of text or symbols in the feature style line 
%    (|featurestyles|) or the in descriptive text line (|features|)
%    see section \ref{Lsetsize}, page \pageref{Lsetsize}.
%    
%    Another set of commands can be used to set a name for a feature
%    line, which is printed together with the sequence names at the
%    left or right side of the alignment, i.\,e.\ 
%    \medskip
%
%                    \label{Lshowfeaturename} \label{Lshowfeaturestylename}
%                    \label{Lhidefeaturename} \label{Lhidefeaturestylename}
%                    \label{Lhidefeaturenames} \label{Lhidefeaturestylenames}
%    \quad |\showfeaturename{|\meta{ttttop...bbbbottom}|}{|\meta{name}|}|,
%    \medskip
%
%    \quad |\showfeaturestylename{|\meta{ttttop...bbbbottom}|}{|\meta{name}|}|
%    \medskip
%
%    \quad |\hidefeaturename{|\meta{ttttop...bbbbottom}|}|
%    \medskip
%
%    \quad |\hidefeaturestylename{|\meta{ttttop...bbbbottom}|}|
%    \medskip
%
%    \quad |\hidefeaturenames|, and |\hidefeaturestylenames|.
%    \medskip
%
%    Using |\showfeaturename| will print the name in the same line as
%    the descriptive text of the feature whereas | \showfeaturestylename|
%    will put the name in the same line as the feature symbols.
%
%    The color of such names can be generally changed with \label{Lfeaturenamescolor}
%    \medskip
%
%    \quad |\featurenamescolor{|\meta{color}|}| and \label{Lfeaturestylenamescolor}
%    \medskip
%
%    \quad |\featurestylenamescolor{|\meta{color}|}| 
%    \medskip
%
%    or individually with \label{Lfeaturenamecolor}
%    \medskip
%
%    \quad |\featurenamecolor{|\meta{ttttop...bbbbottom}|}{|\meta{color}|}| and \label{Lfeaturestylenamecolor}
%    \medskip
%
%    \quad |\featurestylenamecolor{|\meta{ttttop...bbbbottom}|}{|\meta{color}|}| 
%    \medskip
%
%    See section \ref{colors} for how to select colors in \TeX{}shade.
%
%    Font styles can be set as usual (see section \ref{Lsetfamily}), e.\,g.\ 
%    \medskip
%    
%    |\setsize{featurenames}{large}| or |\featurestylenamesrm| etc. 
%    \medskip
%
%
%    
%    Examples for the appearance of features are given in the 
%    overview section (\ref{over}), see:
%    \medskip
%    
%    \emph{similarity mode} (\ref{similar}): fill-character; here, only 
%         one position is labeled. It is also possible to label a longer 
%         stretch, then, the character is printed several times to
%         fill the specified region.
%         
%    \quad|\feature{top}{1}{93..93}{fill:$\downarrow$}{first...}|   
%     
%    \quad|\feature{bottom}{1}{98..98}{fill:$\uparrow$}{second...}|    
%    \medskip
%    
%    \emph{T-Coffee mode} (\ref{TCoffee}): |T-Coffee| color scale
%    
%    \quad|\feature{top}{1}{30..63}{color:conservation[T-Coffee]}{}|
%
%    \quad|\showfeaturestylename{bottom}{cons}|
%    \medskip
%    
%    \emph{diversity mode} (\ref{diverse}): frames, text only
%    
%    \quad|\feature{top}{1}{77..109}{}{AQP2 species variants}|
%    
%    \quad|\frameblock{1}{82..82,106..106}{Red[1pt]}|
%    \medskip
%    
%    \emph{functional mode} (\ref{func}): bar graph, color scale, tinting, box, arrow, 
%                                         translation, brace, helix
%    
%    \quad|\feature{top}{3}{153..165}|
%
%    \quad\quad\quad\quad\quad|{bar[-50,50]:-50,-45,-40,...,40,45,50}{}|
%    \medskip
%
%    \quad|\feature{top}{3}{167..186}|
%
%    \quad\quad\quad\quad\quad|{color:5,10,15,...,90,95,100[ColdHot]}{}|
%    \medskip
%
%    \quad |\feature{top}{1}{158..163}{brace}{tinted}|
%
%    \quad|\tintblock{1}{158..163}|
%    \medskip
%
%    \quad|\feature{top}{1}{138..157}|
%    
%    \quad\quad\quad\quad\quad|{box[Blue,Red][0.5pt]:$\alpha$-helix[Yellow]}|
%    
%    \quad\quad\quad\quad\quad|{transmembrane domain 4}|
%    
%    \quad|\feature{top}{1}{164..170}{o->[Red]}{trans. dom. 5}|
%    
%    \quad|\feature{top}{1}{158..163}{translate[Blue]}{}|
%    
%    \quad|\backtranslabel{oblique}|
%    
%    \quad|\feature{bottom}{1}{158..163}|
%    
%    \quad\quad\quad\quad\quad|{brace[Blue]}{loop D[Blue]}|
%    \medskip
%
%    \quad|\feature{top}{1}{138..157,164..170}{helix}{membr.}|
%
%    \quad|\feature{top}{1}{158..163}{---}{loop}|
%
%    \quad|\featurerule{1mm}|
%    \medskip
%    
%    \emph{bar graphs and color scales} (\ref{graphs}): sequence conservation, 
%                                         charge, molecular weight, hydrophobicity
%    
%    \quad|\feature{ttop}{1}{138..170}{bar:conservation}{}|
%
%    \quad|\feature{top}{1}{138..170}{color:charge}{}|
%
%    \quad|\feature{bottom}{1}{138..170}|
%
%    \quad\quad\quad\quad\quad|{color:molweight[ColdHot]}{}|
%
%    \quad|\feature{bbottom}{1}{138..170}|
%
%    \quad\quad\quad\quad\quad|{bar:hydrophobicity[Red,Gray10]}{}|
%    \medskip 
%    
%    \subsubsection{Including secondary structure information}
%    
%    \label{structure}
%    
%    \label{LincludeDSSP}
%    \label{LincludeSTRIDE}
%    \label{LincludePHDsec}
%    \label{LincludePHDtopo}
%    \label{LincludeHMMTOP}
%    The DSSP [9], STRIDE [10], PHD [11] and HMMTOP [12] algorithms produce 
%    secondary protein structure predictions. PHD files contain both,
%    secondary structure information and topology data. This information can be
%    displayed in an alignment by one of the commands:
%    \bigskip
%    
%    \begin{tabular}{ll}
%    |\includeDSSP| & sec. structure calculated by DSSP\\
%    
%    |\includeSTRIDE| & sec. structure calculated by STRIDE \\
%    
%    |\includePHDsec| & sec. structure calculated by PHD \\
%    
%    |\includePHDtopo| & topology data calculated by PHD \\
%    
%    |\includeHMMTOP| & topology data calculated by HMMTOP \\
%    \end{tabular}
%    \bigskip
%    
%    The syntax is |\includeDSSP{|\meta{seqref}|}{|\meta{filename}|}|,
%    with |seqref| indicating the number or name of the sequence for which 
%    the secondary structure data is calculated and |filename| designating the 
%    corresponding structure file to be included.
%    
%    Several types of secondary structures are predicted by these 
%    programs; in order to designate them in \TeXshade{} use the names
%    from the right column:
%    
%    \begin{center}
%    \begin{tabular}{ll}
%    secondary structure & designation\\[3mm]
%    \emph{DSSP and STRIDE} & \\[2mm]
%    4-helix ($\alpha$-helix) & |alpha| \\
%    isolated $\beta$-bridge  & |bridge| \\
%    extended strand ($\beta$-strand) & |beta| \\
%    3-helix (3$_{10}$-helix) & |3-10| \\
%    5-helix ($\pi$-helix)    & |pi| \\
%    H-bonded turn            & |turn| \\[3mm]
%    \emph{PHDsec} & \\[2mm]
%    helix                    & |alpha| \\
%    sheet                    & |beta| \\[3mm]
%    \emph{PHDtopo and HMMTOP} & \\[2mm]
%    internal region          & |internal| \\
%    external region          & |external| \\
%    transmembrane domain     & |TM| \\
%    \end{tabular}
%    \end{center}
%    
%    \label{LshowonDSSP}
%    \label{LshowonSTRIDE}
%    \label{LshowonPHDsec}
%    \label{LshowonPHDtopo}
%    \label{LshowonHMMTOP}
%    \label{LhideonDSSP}
%    \label{LhideonSTRIDE}
%    \label{LhideonPHDsec}
%    \label{LhideonPHDtopo}
%    \label{LhideonHMMTOP}
%    By default all three types of helices and the strands are 
%    displayed whereas turns and bridges are skipped. If it is 
%    desired to shown them as well, call for example |\shownonDSSP{bridge,turn}|.
%    In analogy to this example all structure features can be activated
%    in DSSP, STRIDE, PHDsec, PHDtopo and HMMTOP. In order to hide
%    certain structure types use for example |\hideonDSSP{3-10,pi}|.
%    
%    The DSSP format has two columns of sequence numberings. The first 
%    column is consecutive, whereas the second column contains the 
%    actual sequence numbering. This can be different from the first
%    column when sequence parts are missing in the DSSP file. One can
%    choose which column will be read by \TeXshade{} by 
%    \label{LfirstcolumnDSSP} \label{LsecondcolumnDSSP}
%    `|\firstcolumnDSSP|' and |\secondcolumnDSSP|'. The second column 
%    is still default.
%    
%    The HMMTOP algorithm can present its results as plain text or 
%    as HTML---plain text needs to be selected here. Further, the 
%    output can be formatted in a single line or in an extended form 
%    (see the HMMTOP documentation). Both can be read and interpreted 
%    by \TeXshade{}. Importantly, HMMTOP files can contain topology
%    predictions of multiple sequences. \TeXshade{} tries to find
%    the correct data based on the respective sequence name. If the
%    sequence name is not found in the file, the first topology data
%    is used. Using an optional parameter (number of the prediction
%    in the file or name) one can define which data from the file is
%    to be used:
%    \medskip
%    
%    |\includeHMMTOP{|\meta{seqref in texshade}|[|\meta{seqref in file}|]}{|\meta{filename}|}|
%    \medskip
%    
%    PHD predictions: when starting the PHD software do not
%    restrict the prediction to secondary structure or topology alone.
%    This leads to changes in the PHD output file which are not
%    correctly interpretable by \TeXshade{} due to ambiguities. There
%    is no way around it---thus, run the full prediction.
%
%    Now, some information on how \TeXshade{} extracts and displays
%    secondary structure features. In short, it is a two step process.
%    First, \TeXshade{} analyzes the secondary structure file and 
%    extracts all necessary data. This data is converted into a 
%    format which is readable and processable by \TeXshade{} using the 
%    |feature| command (see \ref{feature}). This command allows one to 
%    label sequence stretches graphically. For a detailed explanation
%    see the indicated reference. A list of feature commands is saved
%    in a file with the ending `|.sec|' for DSSP, STRIDE and PHDsec
%    or `|.top|' for PHDtopo. Then, in a second step, this file is loaded 
%    again and executed. When \TeXshade{} encouters this file a
%    second time, i.\,e. in a second \TeX{} run, it uses the already
%    existing file for the output. The great advantage of this method 
%    is its flexibility. Due to the simple reason that the feature 
%    file can be edited in the meantime. Thus, the user has the 
%    ability to change the computer generated file according to his
%    personal needs. On the other hand, one can force \TeXshade{} to
%    write a new file every time by the optional argument |[make new]| in the 
%    include command, e.\,g. |\includePHDsec[make new]{1}{AQP.phd}|.
%    
%    \label{Lappearance}
%    Finally, the appearance of the feature labels can be assigned by 
%    the command 
%    
%    |\appearance{|\meta{filetype}|}{|\meta{type}|}{|\meta{position}|}{|\meta{labelstyle}|}{|\meta{text}|}|.
%    
%    Here, \meta{filetype} stands for one of the following secondary structure 
%    file types: |DSSP|, |STRIDE|, |PHDsec|, |PHDtopo| or |HMMTOP| and 
%    \meta{type} designates the secondary structure type as shown in 
%    the right column of the table above. The other
%    arguments \meta{position}, \meta{labelstyle} and \meta{text} 
%    are almost as described in \ref{feature}. 
%    \label{Lnumcount}
%    \label{Lalphacount}
%    \label{LAlphacount}
%    \label{Lromancount}
%    \label{LRomancount}
%    One further possibility
%    is to include internal counters for each secondary structure type.
%    Just add one of the following commands 
%    to the text in the feature description. 
%    
%    \begin{center}
%    \begin{tabular}{ll}
%    \emph{counter} & \emph{display} \\[2mm]
%    |\numcount|     & 1, 2, 3 \ldots \\
%    |\alphacount|   & a, b, c \ldots \\
%    |\Alphacount|   & A, B, C \ldots \\
%    |\romancount|   & i, ii, iii \ldots \\
%    |\Romancount|   & I, II, III \ldots \\
%    \end{tabular}
%    \end{center}
%    
%    Examples: 
%    
%    \quad|\appearance{DSSP}{alpha}{ttop}|
%    
%    \quad\quad\quad\quad\quad\quad\quad|{-->}{$\alpha$-helix~\Alphacount}|
%    
%    \quad|\appearance{PHDtopo}{TM}{bottom}|
%    
%    \quad\quad\quad\quad\quad\quad\quad|{box[Blue]:TM\numcount[Yellow]}{}|
%    
%    
%    \subsection{Displaying and building legends}
%    
%    \label{Lshowlegend}\label{Lhidelegend}\label{Lmovelegend}
%    \label{Lgermanlanguage}\label{Lenglishlanguage}\label{Llegendcolor}
%    \label{Lspanishlanguage}
%    For each predefined shading mode \TeXshade{} can print an appropriate
%    legend to explain the used
%    shading colors. The commands |\showlegend| and |\hidelegend|
%    display or clear the legend at the end of the alignment. 
%    The legend is displayed by default beneath the first residue 
%    of the last alignment line. The location can be changed by 
%    |\movelegend{|\meta{x-offset}|}{|\meta{y-offset}|}|. Both 
%    parameters require a \TeX{} length, e.\,g. |\movelegend{5cm}{-2cm}|
%    moves the legend 5\,cm to the right and 2\,cm up.
%    
%    The language for the descriptions is english by default;
%    if the |\german.sty| package is active legend texts are in 
%    german. So far, german, spanish and english are implemented. With the 
%    commands |\germanlanguage|, |\spanishlanguage| and |\englishlanguage| 
%    switching between the languages 
%    is made possible. For the addition of other languages contact me.
%    Finally, the color of the describing legend texts can be set 
%    with the command |\legendcolor{|\meta{color}|}|.
%    
%    User defined legends are easily built with the following command
%    \label{Lshadebox}|\shadebox{|\meta{color}|}|. Use this command outside 
%    the \TeXshade{} environment, e.\,g. in the text or in the caption. As 
%    \meta{color} any color can be designated (see section \ref{colors}) or 
%    one of the following parameters:
%    
%    \begin{itemize}
%    \item |nomatch| = the color used for nonmatching residues
%    
%    \item |similar| = the color used for similar residues
%    
%    \item |conserved| = the color used for conserved residues
%    
%    \item |allmatch| = the color used for highly conserved residues 
%                       (if |\allmatchspecial| is active)
%    
%    \end{itemize}
%    
%    The command simply prints a shaded box in the specified color 
%    then a describing text can be appended. Examples:
%    \medskip
%    
%    \quad|\shadebox{nomatch}---nonmatching residues| 
%    
%    \quad|\shadebox{similar}: similar residues| 
%    
%    \quad|\shadebox{conserved}~conserved residues| 
%    
%    \quad|\shadebox{Yellow}\quad PKA phosphorylation sites|
%    
%    
%    
%    \subsection{Adding captions to the alignment}
%    
%    Since \TeXshade{} v1.5 captions can be added to the alignment. 
%    So far, captions were difficult to use when the alignment was
%    bigger than one page and therefore did not fit into a 
%    figure environment. The \TeXshade{} captions behave exactly as 
%    normal figure captions. They
%    adopt their style, use the figure counter number and appear in
%    the list of figures as any other figure.
%    
%    The usage is slightly different from normal captions but 
%    intuitive: \label{Lshowcaption}
%    \medskip
%    
%    \quad |\showcaption[|\meta{position}|]{|\meta{text}|}|
%    \medskip
%    
%    The optional \meta{position} tells \TeXshade{} to put the caption on
%    |top| or at the |bottom| of the alignment. If nothing is stated here
%    the caption will appear at the bottom. The parameter
%    \meta{text} just holds the caption text as in the normal |\caption|.
%    The command can be used at any position within the |texshade|
%    environment. A simple example would be: 
%    \medskip
%    
%    \quad |\showcaption{A beautiful \TeXshade{} alignment.}|
%    \medskip
%
%    \label{Lshortcaption}
%    In order to show a short version of the caption in the 
%    "List of Figures" the |\shortcaption{|\meta{short caption text}|}| 
%    command can be used.
%    
%    \subsection{Font handling}
%    
%    \subsubsection{Changing font styles}
%    
%    \label{Lsetfamily}\label{Lsetseries}
%    \label{Lsetshape}\label{Lsetsize}
%    The font styles for the numbering, the sequence names,
%    the sequence residues, the descriptive feature texts
%    and the legends can be changed by several commands.
%    \medskip
%    
%    \quad|\setfamily{|\meta{text}|}{|\meta{family}|}|
%    
%    \quad|\setseries{|\meta{text}|}{|\meta{series}|}|
%    
%    \quad|\setshape{|\meta{text}|}{|\meta{shape}|}|
%    
%    \quad|\setsize{|\meta{text}|}{|\meta{size}|}|
%    \medskip
%    
%    The first parameter selects the text whose style is to be
%    changed. Possible first parameters are
%    |numbering|, |names|, |residues|, |features|, |featurestyles|,
%    |hideblock|, |ruler|, and |legend|.
%    \medskip
%    
%    The style is set by the second parameter:
%    
%    \begin{center}
%    \begin{tabular}{lll}
%    command & \meta{2. parameter} & \\
%    \hline
%    |\setfamily| & |rm| & modern roman font family \\
%           & |sf| & sans serif font family \\
%           & |tt| & typewriter font family \\ \hline
%    |\setseries| & |bf| & bold face series \\
%           & |md| & normal series \\ \hline
%    |\setshape|  & |it| & italics shape \\
%           & |sl| & slanted shape \\
%           & |sc| & small capitals shape \\
%           & |up| & upright shape \\ \hline
%    |\setsize|   & |tiny| & the known \TeX{} sizes \\
%           & |scriptsize| & \\
%           & |footnotesize| & \\
%           & |small| & \\
%           & |normalsize| & \\
%           & |large| & \\
%           & |Large| & \\
%           & |LARGE| & \\
%           & |huge| & \\
%           & |Huge| & \\ \hline
%    \end{tabular}
%    \end{center}
%    
%    Example: |\setfamily{features}{it} \setseries{features}{bf}|
%    \medskip
%    
%    \label{Lsetfont}
%    With the command
%    \medskip 
%    
%    \quad|\setfont{|\meta{text}|}{|\meta{family}|}{|\meta{series}|}{|\meta{shape}|}{|\meta{size}|}|
%    \medskip
%    
%    all four font attributes of one \meta{text} can be changed 
%    simultaneously. The order of the parameters is as indicated.
%    \medskip
%    
%    Example: |\setfont{features}{rm}{it}{bf}{normalsize}|
%    \medskip
%    
%    Further, short commands are provided to change single font
%    attributes quickly. The following commands set attributes
%    of feature texts. 
%    \medskip
%    \enlargethispage{\baselineskip}
%    
%    \quad |\featuresrm| \quad |\featurestiny| \label{Lfeaturesrm}
%    
%    \quad |\featuressf| \quad |\featuresscriptsize|
%    
%    \quad |\featurestt| \quad |\featuresfootnotesize|
%    
%    \quad |\featuresbf| \quad |\featuressmall|
%    
%    \quad |\featuresmd| \quad |\featuresnormalsize|
%    
%    \quad |\featuresit| \quad |\featureslarge|
%    
%    \quad |\featuressl| \quad |\featuresLarge|
%    
%    \quad |\featuressc| \quad |\featuresLARGE|
%    
%    \quad |\featuresup| \quad |\featureshuge|
%    
%    \quad |           | \quad |\featuresHuge|
%    \medskip
%    
%    Corresponding sets are provided for the
%    numbering (|\numberingrm| etc.), 
%    featurestyles (|featurestylesrm| etc.), names (|\namesrm| etc.), 
%    featurenames (|\featurenamesrm| etc.),
%    featurestylenames (|\featurestylenames| etc.),
%    residues (|\residuesrm| etc.),
%    hideblock labels (|hideblockrm| etc.),
%    rulers (|\rulerrm| etc.), and
%    legend texts (|legendrm| etc.).
%     
%    
%    \subsubsection{Using PostScript fonts}
%    
%    As already mentioned \TeXshade{} makes intensive use of
%    \textsc{PostScript} for shading. Now, that 
%    \textsc{PostScript} output is active anyway, including \textsc{PostScript}
%    fonts is very easy. Just declare in the document header
%    \medskip
%    
%    \quad |\usepackage{|\meta{PS-font}|}|.
%    \medskip
%    
%    
%    The typewriter font of \TeX{} is always a topic of discussions.
%    By including the package |\usepackage{courier}| \TeX's 
%    typewriter font is replaced by the widely accepted \textsc{Courier}.
%    Have a look into the directory |..texinputs:latex:psnfss|; there, 
%    some styles are located which exchange the common \TeX{} fonts by 
%    \textsc{PostScript} fonts, e.\,g.\ |avant.sty|, |bookman.sty|,
%    |chancery.sty|, |courier.sty|, |helvet.sty| or |utopia.sty|.
%    Depending on the style used the |\rmdefault|-, |\sfdefault|-,
%    and |\ttdefault| fonts are substituted partly or completely.
%    Thus, |courier.sty| for instance exchanges only the typewriter font, 
%    whereas |bookman.sty| sets \textsc{Bookman} as |\rmdefault|,
%    \textsc{Avantgarde} as |\sfdefault| and \textsc{Courier} as
%    |\ttdefault|.
%    
%    For further information see \textsc{Tomas Rokicki}'s 
%    |dvips| manual [13].
%    
%    
%    
%    
%    \subsection{Goodies}
%    
%    The following commands give information on sequence properties,
%    such as molecular weight, charge or similarity data. They can
%    be used outside the |texshade| environment directly in the 
%    document or in a caption text.
%
%    \subsubsection{Molweight and charge}
%    
%    \label{molcharge}
%    
%    \label{Lmolweight}\label{Lcharge}
%    During the process of sequence setting \TeXshade{}
%    sums up the molecular weight and charge of the
%    aligned proteins. This data can be accessed by the 
%    following commands.
%    \medskip
%    
%    \quad|\molweight{|\meta{seqref}|}{|\meta{Da/kDa}|}|
%    
%    \quad|\charge{|\meta{seqref}|}{|\meta{i/o/N/C}|}|
%    \medskip
%    
%    The first parameter \meta{seqref} selects the sequence. The
%    second parameter in the |\molweight| command allows one to
%    switch the units between Dalton (|Da|) and kilo-Dalton 
%    (|kDa|). The |\charge| command needs the second parameter
%    for the correct consideration of the charged protein termini.
%    Thus, `|i|' refers to internal sequences, `|o|' to the 
%    overall charge, `|N|' to N-terminal sequence parts, and
%    `|C|' to the C-terminal end of a protein.
%    \medskip
%    
%    Example: \quad Charge: |\charge{1}{o}|; Weight: |\molweight{1}{Da}|
%    
%    \subsubsection{Similarity/identity data and tables}
%    
%    \label{simtable}
%    The degree of similarity and identity in percent for any two
%    sequences in the displayed alignment section can be read out 
%    with the commands
%    \label{Lpercentsimilarity} \label{Lpercentidentity} 
%    |\percentsimilarity{|\meta{seqref1}|}{|\meta{seqref2}|}| and
%    |\percentidentity{|\meta{seqref1}|}{|\meta{seqref2}|}|.
%    
%    Using the example alignment on page \pageref{simtableEx} and
%    typing outside the |texshade| environment in the document text the
%    following phrase:
%    \bigskip
% 
%    \quad\quad |AQP1 and AQP2 share a sequence similarity|
%
%    \quad\quad |of \percentsimilarity{AQP1.pro}{AQP2.pro}\%| 
%    \bigskip
% 
%    will result in the text:
%    \bigskip
%
%    \quad\quad AQP1 and AQP2 share a sequence similarity of 69.6\%
%    \bigskip
%    
%    Likewise, |\percentidentity{1}{2}| will give the value |48.4|; note that 
%    sequences can be referred to by their number or their assigned name. The
%    percent value is calculated by dividing the number of identical or similar
%    positions, respectively, by the total of non-gap positions shared by both 
%    sequences. Here, only the part of the alignment is taken into account that
%    is actually displayed. Two residues are considered similar when this is 
%    defined by the command |\pepsims| (see page \pageref{Lpepsims}).
%
%    A full similarity/identity table showing values for all sequences of the
%    alignment can be set using |\similaritytable|.\label{Lsimilaritytable} 
%    The labels and number format will be adjusted according to the language 
%    settings (\ref{Lgermanlanguage}).
%    \medskip
% 
%    Example (see section \ref{simtableEx}):
%    \medskip
%
%    \vbox{%
%    \quad |\begin{center}|
%    \medskip
%
%    \quad\quad|\similaritytable|
%    \medskip
%
%    \quad |\end{center}|
%    }
%    \bigskip
%
%    The command generates a valid \LaTeX{} |tabular|
%    environment, which can be embedded into a |table| environment, e.g.
%    \medskip
%    
%    \vbox{%
%    \quad |\begin{table}[htdp]|
%
%    \quad |\caption{Text ...}|
%
%    \quad |\begin{center}|
%    \medskip
%
%    \quad\quad|\similaritytable|
%    \medskip
%
%    \quad |\end{center}|
%
%    \quad |\end{table}|
%    }
%    
%    \newpage
%    \section{The PostScript color selection scheme}
%    
%    \label{colors}
%    
%    \textsc{PostScript} provides 64 standard colors. All these 
%    colors are predefined in the |color.sty|. Each color
%    has a pictorial name such as |Bittersweet| and a distinct 
%    composition, e.\,g.\ 0\% cyan + 75\% magenta + 100\% yellow + 
%    24\% black---the so-called CMYK scheme. \TeXshade{} enhances this
%    color scheme by gray scales in 5\% steps. 
%    The following colors and grays can be used in \TeXshade{} by 
%    simply declaring the name of the color in the respective 
%    command, e.\,g.\ |\consensuscolors|:
%    
%    
%    \begin{footnotesize}
%    \begin{tabbing}
%    \emph{name}\hspace{2.7cm}\= \emph{CMYK}\hspace{1.6cm}
%    \=\emph{name}\hspace{2.5cm}\= \emph{CMYK}\\
%    
%     \textcolor{GreenYellow}{$\bullet$}GreenYellow   \>{0.15,0,0.69,0}\>\textcolor{Yellow}{$\bullet$}Yellow        \>{0,0,1,0}\\
%     \textcolor{Goldenrod}{$\bullet$}Goldenrod     \>{0,0.10,0.84,0}\>\textcolor{Dandelion}{$\bullet$}Dandelion     \>{0,0.29,0.84,0}\\
%     \textcolor{Apricot}{$\bullet$}Apricot       \>{0,0.32,0.52,0}\>\textcolor{Peach}{$\bullet$}Peach         \>{0,0.50,0.70,0}\\
%     \textcolor{Melon}{$\bullet$}Melon         \>{0,0.46,0.50,0}\>\textcolor{YellowOrange}{$\bullet$}YellowOrange  \>{0,0.42,1,0}\\
%     \textcolor{Orange}{$\bullet$}Orange        \>{0,0.61,0.87,0}\>\textcolor{BurntOrange}{$\bullet$}BurntOrange   \>{0,0.51,1,0}\\
%     \textcolor{Bittersweet}{$\bullet$}Bittersweet   \>{0,0.75,1,0.24}\>\textcolor{RedOrange}{$\bullet$}RedOrange     \>{0,0.77,0.87,0}\\
%     \textcolor{Mahagony}{$\bullet$}Mahagony      \>{0,0.85,0.87,0.35}\>\textcolor{Maroon}{$\bullet$}Maroon        \>{0,0.87,0.68,0.32}\\
%     \textcolor{BrickRed}{$\bullet$}BrickRed      \>{0,0.89,0.94,0.28}\>\textcolor{Red}{$\bullet$}Red           \>{0,1,1,0}\\
%     \textcolor{OrangeRed}{$\bullet$}OrangeRed     \>{0,1,0.50,0}\>\textcolor{RubineRed}{$\bullet$}RubineRed     \>{0,1,0.13,0}\\
%     \textcolor{WildStrawberry}{$\bullet$}WildStrawberry\>{0,0.96,0.39,0}\>\textcolor{Salmon}{$\bullet$}Salmon        \>{0,0.53,0.38,0}\\
%     \textcolor{CarnationPink}{$\bullet$}CarnationPink \>{0,0.63,0,0}\>\textcolor{Magenta}{$\bullet$}Magenta       \>{0,1,0,0}\\
%     \textcolor{VioletRed}{$\bullet$}VioletRed     \>{0,0.81,0,0}\>\textcolor{Rhodamine}{$\bullet$}Rhodamine     \>{0,0.82,0,0}\\
%     \textcolor{Mulberry}{$\bullet$}Mulberry      \>{0.34,0.90,0,0.02}\>\textcolor{RedViolet}{$\bullet$}RedViolet     \>{0.07,0.90,0,0.34}\\
%     \textcolor{Fuchsia}{$\bullet$}Fuchsia       \>{0.47,0.91,0,0.08}\>\textcolor{Lavender}{$\bullet$}Lavender      \>{0,0.48,0,0}\\
%     \textcolor{Thistle}{$\bullet$}Thistle       \>{0.12,0.59,0,0}\>\textcolor{Orchid}{$\bullet$}Orchid        \>{0.32,0.64,0,0}\\
%     \textcolor{DarkOrchid}{$\bullet$}DarkOrchid    \>{0.40,0.80,0.20,0}\>\textcolor{Purple}{$\bullet$}Purple        \>{0.45,0.86,0,0}\\
%     \textcolor{Plum}{$\bullet$}Plum          \>{0.50,1,0,0}\>\textcolor{Violet}{$\bullet$}Violet        \>{0.79,0.88,0,0}\\
%     \textcolor{RoyalPurple}{$\bullet$}RoyalPurple   \>{0.75,0.90,0,0}\>\textcolor{BlueViolet}{$\bullet$}BlueViolet    \>{0.86,0.91,0,0.04}\\
%     \textcolor{Periwinkle}{$\bullet$}Periwinkle    \>{0.57,0.55,0,0}\>\textcolor{CadetBlue}{$\bullet$}CadetBlue     \>{0.62,0.57,0.23,0}\\
%     \textcolor{CornflowerBlue}{$\bullet$}CornflowerBlue\>{0.65,0.13,0,0}\>\textcolor{MidnightBlue}{$\bullet$}MidnightBlue  \>{0.98,0.13,0,0.43}\\
%     \textcolor{NavyBlue}{$\bullet$}NavyBlue      \>{0.94,0.54,0,0}\>\textcolor{RoyalBlue}{$\bullet$}RoyalBlue     \>{1,0.50,0,0}\\
%     \textcolor{Blue}{$\bullet$}Blue          \>{1,1,0,0}\>\textcolor{Cerulean}{$\bullet$}Cerulean      \>{0.94,0.11,0,0}\\
%     \textcolor{Cyan}{$\bullet$}Cyan          \>{1,0,0,0}\>\textcolor{ProcessBlue}{$\bullet$}ProcessBlue   \>{0.96,0,0,0}\\
%     \textcolor{SkyBlue}{$\bullet$}SkyBlue       \>{0.62,0,0.12,0}\>\textcolor{Turquoise}{$\bullet$}Turquoise     \>{0.85,0,0.20,0}\\
%     \textcolor{TealBlue}{$\bullet$}TealBlue      \>{0.86,0,0.34,0.02}\>\textcolor{Aquamarine}{$\bullet$}Aquamarine    \>{0.82,0,0.30,0}\\
%     \textcolor{BlueGreen}{$\bullet$}BlueGreen     \>{0.85,0,0.33,0}\>\textcolor{Emerald}{$\bullet$}Emerald       \>{1,0,0.50,0}\\
%     \textcolor{JungleGreen}{$\bullet$}JungleGreen   \>{0.99,0,0.52,0}\>\textcolor{SeaGreen}{$\bullet$}SeaGreen      \>{0.69,0,0.50,0}\\
%     \textcolor{Green}{$\bullet$}Green         \>{1,0,1,0}\>\textcolor{ForestGreen}{$\bullet$}ForestGreen   \>{0.91,0,0.88,0.12}\\
%     \textcolor{PineGreen}{$\bullet$}PineGreen     \>{0.92,0,0.59,0.25}\>\textcolor{LimeGreen}{$\bullet$}LimeGreen     \>{0.50,0,1,0}\\
%     \textcolor{YellowGreen}{$\bullet$}YellowGreen   \>{0.44,0,0.74,0}\>\textcolor{SpringGreen}{$\bullet$}SpringGreen   \>{0.26,0,0.76,0}\\
%     \textcolor{OliveGreen}{$\bullet$}OliveGreen    \>{0.64,0,0.95,0.40}\>\textcolor{RawSienna}{$\bullet$}RawSienna     \>{0,0.72,1,0.45}\\
%     \textcolor{Sepia}{$\bullet$}Sepia         \>{0,0.83,1,0.70}\>\textcolor{Brown}{$\bullet$}Brown         \>{0,0.81,1,0.60}\\
%     \textcolor{Tan}{$\bullet$}Tan           \>{0.14,0.42,0.56,0}\>\>\\
%     \textcolor{White}{$\bullet$}White (Gray0)  \>{0,0,0,0}\>\textcolor{Black}{$\bullet$}Black (Gray100)    \>{0,0,0,1}\\
%     \textcolor{Gray5}{$\bullet$}Gray5          \>{0,0,0,0.05}\>\textcolor{Gray10}{$\bullet$}Gray10          \>{0,0,0,0.10}\\
%     \textcolor{Gray15}{$\bullet$}Gray15          \>{0,0,0,0.15}\>\textcolor{Gray20}{$\bullet$}Gray20          \>{0,0,0,0.20}\\
%     \textcolor{Gray25}{$\bullet$}Gray25          \>{0,0,0,0.25}\>\textcolor{Gray30}{$\bullet$}Gray30          \>{0,0,0,0.30}\\
%     \textcolor{LightGray}{$\bullet$}LightGray       \>{0,0,0,0.33}\>\textcolor{Gray35}{$\bullet$}Gray35          \>{0,0,0,0.35}\\
%     \textcolor{Gray40}{$\bullet$}Gray40          \>{0,0,0,0.40}\>\textcolor{Gray45}{$\bullet$}Gray45          \>{0,0,0,0.45}\\
%     \textcolor{Gray50}{$\bullet$}Gray50          \>{0,0,0,0.50}\>\textcolor{Gray}{$\bullet$}Gray          \>{0,0,0,0.50}\\
%     \textcolor{Gray55}{$\bullet$}Gray55          \>{0,0,0,0.55}\>\textcolor{Gray60}{$\bullet$}Gray60          \>{0,0,0,0.60}\\
%     \textcolor{Gray65}{$\bullet$}Gray65          \>{0,0,0,0.65}\>\textcolor{DarkGray}{$\bullet$}DarkGray        \>{0,0,0,0.66}\\
%     \textcolor{Gray70}{$\bullet$}Gray70          \>{0,0,0,0.70}\>\textcolor{Gray75}{$\bullet$}Gray75          \>{0,0,0,0.75}\\
%     \textcolor{Gray80}{$\bullet$}Gray80          \>{0,0,0,0.80}\>\textcolor{Gray85}{$\bullet$}Gray85          \>{0,0,0,0.85}\\
%     \textcolor{Gray90}{$\bullet$}Gray90          \>{0,0,0,0.90}\>\textcolor{Gray95}{$\bullet$}Gray95          \>{0,0,0,0.95}\\
%     \textcolor{LightGreenYellow}{$\bullet$}LightGreenYellow\>{0.08,0,0.35,0}\>\textcolor{LightYellow}{$\bullet$}LightYellow \>{0,0,0.50,0}\\
%     \textcolor{LightGoldenrod}{$\bullet$}LightGoldenrod  \>{0,0.05,0.42,0}\>\textcolor{LightDandelion}{$\bullet$}LightDandelion\> {0,0.15,0.42,0}\\
%     \textcolor{LightApricot}{$\bullet$}LightApricot    \>{0,0.16,0.26,0}\>\textcolor{LightPeach}{$\bullet$}LightPeach   \>{0,0.25,0.35,0}\\
%     \textcolor{LightMelon}{$\bullet$}LightMelon      \>{0,0.23,0.25,0}\>\textcolor{LightYellowOrange}{$\bullet$}LightYellowOrange \>{0,0.21,0.50,0}\\
%     \textcolor{LightOrange}{$\bullet$}LightOrange     \>{0,0.31,0.44,0}\>\textcolor{LightBurntOrange}{$\bullet$}LightBurntOrange \>{0,0.26,0.50,0}\\
%     \textcolor{LightBittersweet}{$\bullet$}LightBittersweet\>{0,0.38,0.50,0.12}\>\textcolor{LightRedOrange}{$\bullet$}LightRedOrange\>{0,0.39,0.44,0}\\
%     \textcolor{LightMahagony}{$\bullet$}LightMahagony   \>{0,0.43,0.44,0.18}\>\textcolor{LightMaroon}{$\bullet$}LightMaroon \>{0,0.44,0.34,0.16}\\
%     \textcolor{LightBrickRed}{$\bullet$}LightBrickRed   \>{0,0.45,0.47,0.14}\>\textcolor{LightRed}{$\bullet$}LightRed    \>{0,0.50,0.50,0}\\
%     \textcolor{LightOrangeRed}{$\bullet$}LightOrangeRed  \>{0,0.50,0.25,0}\>\textcolor{LightRubineRed}{$\bullet$}LightRubineRed  \>{0,0.50,0.07,0}\\
%     \textcolor{LightWildStrawberry}{$\bullet$}LightWildStrawberry\>{0,0.48,0.20,0}\>\textcolor{LightSalmon}{$\bullet$}LightSalmon  \>{0,0.27,0.19,0}\\
%     \textcolor{LightCarnationPink}{$\bullet$}LightCarnationPink \>{0,0.32,0,0} \>\textcolor{LightMagenta}{$\bullet$}LightMagenta   \>{0,0.50,0,0}\\
%     \textcolor{LightVioletRed}{$\bullet$}LightVioletRed  \>{0,0.40,0,0} \>\textcolor{LightRhodamine}{$\bullet$}LightRhodamine   \>{0,0.41,0,0}\\
%     \textcolor{LightMulberry}{$\bullet$}LightMulberry   \>{0.17,0.45,0,0.01}\>\textcolor{LightRedViolet}{$\bullet$}LightRedViolet \>{0.04,0.45,0,0.17}\\
%     \textcolor{LightFuchsia}{$\bullet$}LightFuchsia    \>{0.24,0.46,0,0.04}\>\textcolor{LightLavender}{$\bullet$}LightLavender \> {0,0.24,0,0}\\
%     \textcolor{LightThistle}{$\bullet$}LightThistle    \>{0.06,0.30,0,0} \>\textcolor{LightOrchid}{$\bullet$}LightOrchid    \>{0.16,0.32,0,0}\\
%     \textcolor{LightDarkOrchid}{$\bullet$}LightDarkOrchid \>{0.20,0.40,0.10,0}\>\textcolor{LightPurple}{$\bullet$}LightPurple  \>{0.23,0.43,0,0}\\
%     \textcolor{LightPlum}{$\bullet$}LightPlum       \>{0.25,0.50,0,0} \>\textcolor{LightViolet}{$\bullet$}LightViolet \>{0.40,0.44,0,0}\\
%     \textcolor{LightRoyalPurple}{$\bullet$}LightRoyalPurple\>{0.38,0.45,0,0} \>\textcolor{LightBlueViolet}{$\bullet$}LightBlueViolet \>{0.43,0.46,0,0.02}\\
%     \textcolor{LightPeriwinkle}{$\bullet$}LightPeriwinkle \>{0.29,0.28,0,0} \>\textcolor{LightCadetBlue}{$\bullet$}LightCadetBlue \> {0.31,0.29,0.12,0}\\
%     \textcolor{LightCornflowerBlue}{$\bullet$}LightCornflowerBlue\>{0.33,0.07,0,0}\>\textcolor{LightMidnightBlue}{$\bullet$}LightMidnightBlue\>{0.49,0.07,0,0.22}\\
%     \textcolor{LightNavyBlue}{$\bullet$}LightNavyBlue   \>{0.47,0.27,0,0} \>\textcolor{LightRoyalBlue}{$\bullet$}LightRoyalBlue \> {0.50,0.25,0,0}\\
%     \textcolor{LightBlue}{$\bullet$}LightBlue       \>{0.50,0.50,0,0} \>\textcolor{LightCerulean}{$\bullet$}LightCerulean \> {0.47,0.06,0,0}\\
%     \textcolor{LightCyan}{$\bullet$}LightCyan       \>{0.50,0,0,0} \>\textcolor{LightProcessBlue}{$\bullet$}LightProcessBlue \> {0.48,0,0,0}\\
%     \textcolor{LightSkyBlue}{$\bullet$}LightSkyBlue    \>{0.31,0,0.06,0} \>\textcolor{LightTurquoise}{$\bullet$}LightTurquoise \>{0.43,0,0.10,0}\\
%     \textcolor{LightTealBlue}{$\bullet$}LightTealBlue   \>{0.43,0,0.17,0.01}\>\textcolor{LightAquamarine}{$\bullet$}LightAquamarine \>{0.41,0,0.15,0}\\
%     \textcolor{LightBlueGreen}{$\bullet$}LightBlueGreen  \>{0.43,0,0.17,0}\>\textcolor{LightEmerald}{$\bullet$}LightEmerald   \>{0.50,0,0.25,0}\\
%     \textcolor{LightJungleGreen}{$\bullet$}LightJungleGreen\>{0.50,0,0.26,0} \>\textcolor{LightSeaGreen}{$\bullet$}LightSeaGreen  \>{0.35,0,0.25,0}\\
%     \textcolor{LightGreen}{$\bullet$}LightGreen      \>{0.50,0,0.50,0} \>\textcolor{LightForestGreen}{$\bullet$}LightForestGreen\>{0.46,0,0.44,0.06}\\
%     \textcolor{LightPineGreen}{$\bullet$}LightPineGreen  \>{0.46,0,0.30,0.13}\>\textcolor{LightLimeGreen}{$\bullet$}LightLimeGreen\>{0.25,0,0.50,0}\\
%     \textcolor{LightYellowGreen}{$\bullet$}LightYellowGreen\>{0.22,0,0.37,0} \>\textcolor{LightSpringGreen}{$\bullet$}LightSpringGreen \>{0.13,0,0.38,0}\\
%     \textcolor{LightOliveGreen}{$\bullet$}LightOliveGreen \>{0.32,0,0.48,0.20} \>\textcolor{LightRawSienna}{$\bullet$}LightRawSienna\>{0,0.36,0.50,0.23}\\
%     \textcolor{LightSepia}{$\bullet$}LightSepia      \>{0,0.44,0.50,0.35} \>\textcolor{LightBrown}{$\bullet$}LightBrown    \>{0,0.41,0.50,0.30}\\
%     \textcolor{LightTan}{$\bullet$}LightTan        \>{0.07,0.21,0.28,0}\\ 
%     LightLight- and LightLightLight-versions were derived by dividing all values\\ 
%     from Light-color definitions by 2 and 4, respectively. 
%    \end{tabbing}
%
%    \begin{tabbing}
%    \emph{name}\hspace{2.5cm}\= \emph{RGB\quad}\hspace{1.8cm}
%    \=\emph{name}\hspace{2.5cm}\= \emph{RGB\quad}\\
%    
%     \textcolor{BlueRed5}{$\bullet$}BlueRed5        \>{0.15,0.17,0.55} \>\textcolor{BlueRed10}{$\bullet$}BlueRed10 \> {0.20,0.23,0.57}\\
%     \textcolor{BlueRed15}{$\bullet$}BlueRed15       \> {0.24,0.29,0.60} \>\textcolor{BlueRed20}{$\bullet$}BlueRed20 \> {0.33,0.35,0.64}\\
%     \textcolor{BlueRed25}{$\bullet$}BlueRed25       \> {0.43,0.43,0.68} \>\textcolor{BlueRed30}{$\bullet$}BlueRed30 \> {0.52,0.52,0.73}\\
%     \textcolor{BlueRed35}{$\bullet$}BlueRed35       \> {0.60,0.60,0.78} \>\textcolor{BlueRed40}{$\bullet$}BlueRed40 \> {0.70,0.70,0.84}\\
%     \textcolor{BlueRed45}{$\bullet$}BlueRed45       \> {0.80,0.80,0.85} \>\textcolor{BlueRed50}{$\bullet$}BlueRed50 \> {0.86,0.82,0.82}\\
%     \textcolor{BlueRed55}{$\bullet$}BlueRed55       \> {0.87,0.73,0.73} \>\textcolor{BlueRed60}{$\bullet$}BlueRed60 \> {0.89,0.64,0.64}\\
%     \textcolor{BlueRed65}{$\bullet$}BlueRed65       \> {0.90,0.55,0.55} \>\textcolor{BlueRed70}{$\bullet$}BlueRed70 \> {0.91,0.47,0.46}\\
%     \textcolor{BlueRed75}{$\bullet$}BlueRed75       \> {0.91,0.39,0.37} \>\textcolor{BlueRed80}{$\bullet$}BlueRed80 \> {0.90,0.33,0.28}\\
%     \textcolor{BlueRed85}{$\bullet$}BlueRed85       \> {0.89,0.25,0.20} \>\textcolor{BlueRed90}{$\bullet$}BlueRed90 \> {0.88,0.23,0.14}\\
%     \textcolor{BlueRed95}{$\bullet$}BlueRed95       \> {0.87,0.21,0.09} \>\textcolor{BlueRed100}{$\bullet$}BlueRed100\> {0.87,0.16,0.04}\\
%     \textcolor{GreenRed5}{$\bullet$}GreenRed5       \> {0,1,0}          \>\textcolor{GreenRed10}{$\bullet$}GreenRed10\> {0.05,0.95,0}\\
%     \textcolor{GreenRed15}{$\bullet$}GreenRed15      \> {0.10,0.90,0}    \>\textcolor{GreenRed20}{$\bullet$}GreenRed20\> {0.15,0.85,0}\\
%     \textcolor{GreenRed25}{$\bullet$}GreenRed25      \> {0.20,0.80,0}    \>\textcolor{GreenRed30}{$\bullet$}GreenRed30\> {0.25,0.75,0}\\
%     \textcolor{GreenRed35}{$\bullet$}GreenRed35      \> {0.30,0.70,0}    \>\textcolor{GreenRed40}{$\bullet$}GreenRed40\> {0.35,0.65,0}\\
%     \textcolor{GreenRed45}{$\bullet$}GreenRed45      \> {0.40,0.60,0}    \>\textcolor{GreenRed50}{$\bullet$}GreenRed50\> {0.45,0.55,0}\\
%     \textcolor{GreenRed55}{$\bullet$}GreenRed55      \> {0.50,0.50,0}    \>\textcolor{GreenRed60}{$\bullet$}GreenRed60\> {0.55,0.45,0}\\
%     \textcolor{GreenRed65}{$\bullet$}GreenRed65      \> {0.60,0.40,0}    \>\textcolor{GreenRed70}{$\bullet$}GreenRed70\> {0.65,0.35,0}\\
%     \textcolor{GreenRed75}{$\bullet$}GreenRed75      \> {0.70,0.30,0}    \>\textcolor{GreenRed80}{$\bullet$}GreenRed80\> {0.75,0.25,0}\\
%     \textcolor{GreenRed85}{$\bullet$}GreenRed85      \> {0.80,0.20,0}    \>\textcolor{GreenRed90}{$\bullet$}GreenRed90\> {0.85,0.15,0}\\
%     \textcolor{GreenRed95}{$\bullet$}GreenRed95      \> {0.90,0.10,0}    \>\textcolor{GreenRed100}{$\bullet$}GreenRed100\> {0.95,0.05,0}\\
%     \textcolor{ColdHot5}{$\bullet$}ColdHot5        \> {0,0.08,1}        \>\textcolor{ColdHot10}{$\bullet$}ColdHot10 \> {0,0.29,1}\\
%     \textcolor{ColdHot15}{$\bullet$}ColdHot15       \> {0,0.49,1}       \>\textcolor{ColdHot20}{$\bullet$}ColdHot20 \> {0,0.70,1}\\
%     \textcolor{ColdHot25}{$\bullet$}ColdHot25       \> {0,0.90,1}       \>\textcolor{ColdHot30}{$\bullet$}ColdHot30 \> {0,1,0.87}\\
%     \textcolor{ColdHot35}{$\bullet$}ColdHot35       \> {0,1,0.68}       \>\textcolor{ColdHot40}{$\bullet$}ColdHot40 \> {0,1,0.46}\\
%     \textcolor{ColdHot45}{$\bullet$}ColdHot45       \> {0,1,0.25}       \>\textcolor{ColdHot50}{$\bullet$}ColdHot50 \> {0,1,0.04}\\
%     \textcolor{ColdHot55}{$\bullet$}ColdHot55       \> {0.16,1,0}       \>\textcolor{ColdHot60}{$\bullet$}ColdHot60 \> {0.35,1,0}\\
%     \textcolor{ColdHot65}{$\bullet$}ColdHot65       \> {0.56,1,0}       \>\textcolor{ColdHot70}{$\bullet$}ColdHot70 \> {0.79,1,0}\\
%     \textcolor{ColdHot75}{$\bullet$}ColdHot75       \> {0.98,1,0}       \>\textcolor{ColdHot80}{$\bullet$}ColdHot80 \> {1,0.82,0}\\
%     \textcolor{ColdHot85}{$\bullet$}ColdHot85       \> {1,0.60,0}       \>\textcolor{ColdHot90}{$\bullet$}ColdHot90 \> {1,0.40,0}\\
%     \textcolor{ColdHot95}{$\bullet$}ColdHot95       \> {1,0.20,0}       \>\textcolor{ColdHot100}{$\bullet$}ColdHot100\> {0.91,0,0}\\
%     and reverse definitions: |RedBlue|, |RedGreen|, |HotCold|.\\
%    \end{tabbing}
%    \end{footnotesize}
%    
%    Type the color names with the upper case letters exactly as described above.
%    For the definition of new colors use one of the |color.sty| commands:
%    \medskip
%    
%    \quad|\definecolor{|\meta{name}|}{cmyk}{|\meta{C,M,Y,K}|}|
%    \medskip
%    
%    \quad|\definecolor{|\meta{name}|}{rgb}{|\meta{R,G,B}|}|
%    \medskip
%    
%    The \meta{name} can be chosen freely, the values for the color
%    composition must be in the range 0--1, i\,e.\ 0--100\% of the 
%    respective component (`C' -- cyan, `M' -- magenta, `Y' -- yellow,
%    `K' -- black; or `R' -- red, `G' -- green, `Blue' -- blue) separated by 
%    commas.
%    \medskip
%    
%    Examples: 
%    \medskip
%    
%    |\definecolor{Salmon}{cmyk}{0,0.53,0.38,0}|
%    \medskip
%
%    |\definecolor{ColdHot15}{rgb}{0,0.49,1}|
%    \medskip
%
%    \newpage
%    \section{Listing of the \texttt{texshade} default settings}
%    
%    \subsection{Standard definitions}
%    
%    The file |texshade.def| mirrors all commands which are
%    carried out at the beginning of the |texshade| environment.
%    Short comments are also included, thus, it is refered to
%    this file for further information. 
%    
%    \subsection{Colors used in the different shading modes}
%    
%    \vspace{5mm}
%    
%    Color scheme \emph{blues}:
%    \medskip
%    
%    \begin{tabbing}
%    \hspace{1cm}\=\emph{res.color}\hspace{1.5cm}\=\emph{shad.color}\hspace{1.5cm} \= \emph{residues} \\
%    \>Black \> White   \> no match \\
%    \>Black \> Magenta \> similar  \\
%    \>White \> RoyalBlue \> identical \\
%    \>Goldenrod \> RoyalPurple \> all match\\
%    \end{tabbing}
%    \medskip
%    
%    Color scheme \emph{greens}:
%    \medskip
%    
%    \begin{tabbing}
%    \hspace{1cm}\=\emph{res.color}\hspace{1.5cm}\=\emph{shad.color}\hspace{1.5cm} \= \emph{residues} \\
%    \>Black \> White   \> no match \\
%    \>Black \> GreenYellow \> similar  \\
%    \>White \> PineGreen \> identical \\
%    \>YellowOrange \> OliveGreen \> all match\\
%    \end{tabbing}
%    \medskip
%    
%    Color scheme \emph{reds}:
%    \medskip
%    
%    \begin{tabbing}
%    \hspace{1cm}\=\emph{res.color}\hspace{1.5cm}\=\emph{shad.color}\hspace{1.5cm} \= \emph{residues} \\
%    \>Black \> White   \> no match \\
%    \>Black \> YellowOrange \> similar  \\
%    \>White \> BrickRed \> identical \\
%    \>YellowGreen \> Mahagony \> all match\\
%    \end{tabbing}
%    \medskip
%    
%    \newpage
%    Color scheme \emph{grays}:
%    \medskip
%    
%    \begin{tabbing}
%    \hspace{1cm}\=\emph{res.color}\hspace{1.5cm}\=\emph{shad.color}\hspace{1.5cm} \= \emph{residues} \\
%    \>Black \> White   \> no match \\
%    \>Black \> LightGray \> similar  \\
%    \>White \> DarkGray \> identical \\
%    \>White \> Black \> all match\\
%    \end{tabbing}
%    \medskip
%    
%    Color scheme \emph{black}:
%    \medskip
%    
%    \begin{tabbing}
%    \hspace{1cm}\=\emph{res.color}\hspace{1.5cm}\=\emph{shad.color}\hspace{1.5cm} \= \emph{residues} \\
%    \>Black \> White   \> no match \\
%    \>Black \> White \> similar  \\
%    \>White \> Black \> identical \\
%    \>White \> Black \> all match\\
%    \end{tabbing}
%    \medskip
%    
%    Functional mode \emph{charge}:
%    \medskip
%    
%    \begin{tabbing}
%    \hspace{1cm}\=\emph{res.color}\hspace{1.5cm}\=\emph{shad.color}\hspace{1.5cm} \= \emph{residues} \\
%    \>Black \> White   \> no match \\
%    \>White \> Red \> acidic  \\
%    \>White \> Blue \> basic \\
%    \end{tabbing}
%    \medskip
%    
%    Functional mode \emph{hydropathy}:
%    \medskip
%    
%    \begin{tabbing}
%    \hspace{1cm}\=\emph{res.color}\hspace{1.5cm}\=\emph{shad.color}\hspace{1.5cm} \= \emph{residues} \\
%    \>Black \> White   \> no match \\
%    \>White \> Red \> acidic  \\
%    \>White \> Blue \> basic \\
%    \>Black \> Yellow   \> polar uncharged \\
%    \>White \> Green   \> hydrophobic nonpolar \\
%    \end{tabbing}
%    \medskip
%    
%    \newpage
%    Functional mode \emph{chemical}:
%    \medskip
%    
%    \begin{tabbing}
%    \hspace{1cm}\=\emph{res.color}\hspace{1.5cm}\=\emph{shad.color}\hspace{1.5cm} \= \emph{residues} \\
%    \>Black \> White   \> no match \\
%    \>White \> Red \> acidic  \\
%    \>White \> Black \> aliphatic  \\
%    \>White \> Gray \> aliphatic (small) \\
%    \>White \> Green \> amide  \\
%    \>White \> Brown \> aromatic  \\
%    \>White \> Blue \> basic \\
%    \>Black \> Magenta   \> hydroxyl \\
%    \>Black \> Orange   \> imino \\
%    \>Black \> Yellow   \> sulfur \\
%    \end{tabbing}
%    \medskip
%    
%    Functional mode \emph{rasmol}:
%    \medskip
%    
%    \begin{tabbing}
%    \hspace{1cm}\=\emph{res.color}\hspace{1.5cm}\=\emph{shad.color}\hspace{1.5cm} \= \emph{residues} \\
%    \>Black \> White   \> no match \\
%    \>Red \> White   \> Asp, Glu \\
%    \>Blue \> White \> Arg, Lys, His  \\
%    \>MidnightBlue \> White \> Phe, Tyr, Trp  \\
%    \>Gray \> White \> Ala, Gly  \\
%    \>Yellow \> White \> Cys, Met  \\
%    \>Orange \> White \> Ser, Thr \\
%    \>Cyan \> White   \> Asn, Gln \\
%    \>Gree \> White   \> Leu, Val, Ile \\
%    \>Apricot \> White   \> Pro \\
%    \end{tabbing}
%    \medskip
%    
%    Functional mode \emph{structure}:
%    \medskip
%    
%    \begin{tabbing}
%    \hspace{1cm}\=\emph{res.color}\hspace{1.5cm}\=\emph{shad.color}\hspace{1.5cm} \= \emph{residues} \\
%    \>Black \> White   \> no match \\
%    \>Black \> Orange \> external  \\
%    \>Black \> Yellow   \> ambivalent \\
%    \>White \> Green \> internal  \\
%    \end{tabbing}
%    \medskip
%    
%    \newpage
%    Functional mode \emph{standard area}:
%    \medskip
%    
%    \begin{tabbing}
%    \hspace{1cm}\=\emph{res.color}\hspace{1.5cm}\=\emph{shad.color}\hspace{1.5cm} \= \emph{residues} \\
%    \>Black \> White   \> no match \\
%    \>Black \> BrickRed \> G\\
%    \>Black \> Orange  \> A, S\\
%    \>Black \> Yellow   \> C, P  \\
%    \>Black \> YellowGreen   \> T, D, V, N \\
%    \>White \> PineGreen   \> I, E \\
%    \>Black \> SkyBlue   \> L, Q, H, M \\
%    \>White \> RoyalPurple \> F, K \\
%    \>White \> RedViolet   \> Y \\
%    \>White \> Black   \> R, W \\
%    \end{tabbing}
%    \medskip
%    
%    Functional mode \emph{accessible area}:
%    \medskip
%    
%    \begin{tabbing}
%    \hspace{1cm}\=\emph{res.color}\hspace{1.5cm}\=\emph{shad.color}\hspace{1.5cm} \= \emph{residues} \\
%    \>Black \> White   \> no match \\
%    \>Black \> BrickRed \> C \\
%    \>Black \> Orange  \> I, V, G \\
%    \>Black \> Yellow   \> F, L, M, A \\
%    \>Black \> YellowGreen   \> W, S, T, H \\
%    \>White \> PineGreen   \> P \\
%    \>Black \> SkyBlue   \> Y, D, N \\
%    \>White \> RoyalPurple \> E, Q  \\
%    \>White \> RedViolet   \> R \\
%    \>White \> Black   \> K \\
%    \end{tabbing}
%    \medskip
%    
%    
%    \newpage
%    \subsection{Residue weight tables}\label{weightmatrix}
%    \bigskip
%
%    |identity|
%    \bigskip
%    \bigskip
%    
%{\footnotesize\tt
%
%\weighttable{identity}
%\hspace*{-2cm}
%\begin{tabular}{lrrrrrrrrrrrrrrrrrrrr}
%
%C & \consCC &&&&&&&&&&&&&&&&&&& \\
%
%S & \consSC & \consSS &&&&&&&&&&&&&&&&&& \\
%
%T & \consTC & \consTS & \consTT &&&&&&&&&&&&&&&&& \\
%
%P & \consPC & \consPS & \consPT & \consPP &&&&&&&&&&&&&&&& \\
%
%A & \consAC & \consAS & \consAT & \consAP & \consAA
%&&&&&&&&&&&&&&& \\
%
%G & \consGC & \consGS & \consGT & \consGP & \consGA
%& \consGG &&&&&&&&&&&&&& \\
%
%N & \consNC & \consNS & \consNT & \consNP & \consNA
%& \consNG & \consNN &&&&&&&&&&&&& \\
%
%D & \consDC & \consDS & \consDT & \consDP & \consDA
%& \consDG & \consDN & \consDD &&&&&&&&&&&& \\
%
%E & \consEC & \consES & \consET & \consEP & \consEA
%& \consEG & \consEN & \consED & \consEE &&&&&&&&&&& \\
%
%Q & \consQC & \consQS & \consQT & \consQP & \consQA
%& \consQG & \consQN & \consQD & \consQE & \consQQ
%&&&&&&&&&& \\
%
%H & \consHC & \consHS & \consHT & \consHP & \consHA
%& \consHG & \consHN & \consHD & \consHE & \consHQ
%& \consHH &&&&&&&&& \\
%
%R & \consRC & \consRS & \consRT & \consRP & \consRA
%& \consRG & \consRN & \consRD & \consRE & \consRQ
%& \consRH & \consRR &&&&&&&& \\
%
%K & \consKC & \consKS & \consKT & \consKP & \consKA
%& \consKG & \consKN & \consKD & \consKE & \consKQ
%& \consKH & \consKR & \consKK &&&&&&& \\
%
%M & \consMC & \consMS & \consMT & \consMP & \consMA
%& \consMG & \consMN & \consMD & \consME & \consMQ
%& \consMH & \consMR & \consMK & \consMM &&&&&& \\
%
%I & \consIC & \consIS & \consIT & \consIP & \consIA
%& \consIG & \consIN & \consID & \consIE & \consIQ
%& \consIH & \consIR & \consIK & \consIM & \consII
%&&&&& \\
%
%L & \consLC & \consLS & \consLT & \consLP & \consLA
%& \consLG & \consLN & \consLD & \consLE & \consLQ
%& \consLH & \consLR & \consLK & \consLM & \consLI
%& \consLL &&&& \\
%
%V & \consVC & \consVS & \consVT & \consVP & \consVA
%& \consVG & \consVN & \consVD & \consVE & \consVQ
%& \consVH & \consVR & \consVK & \consVM & \consVI
%& \consVL & \consVV &&& \\
%
%F & \consFC & \consFS & \consFT & \consFP & \consFA
%& \consFG & \consFN & \consFD & \consFE & \consFQ
%& \consFH & \consFR & \consFK & \consFM & \consFI
%& \consFL & \consFV & \consFF && \\
%
%Y & \consYC & \consYS & \consYT & \consYP & \consYA
%& \consYG & \consYN & \consYD & \consYE & \consYQ
%& \consYH & \consYR & \consYK & \consYM & \consYI
%& \consYL & \consYV & \consYF & \consYY & \\
%
%W & \consWC & \consWS & \consWT & \consWP & \consWA
%& \consWG & \consWN & \consWD & \consWE & \consWQ
%& \consWH & \consWR & \consWK & \consWM & \consWI
%& \consWL & \consWV & \consWF & \consWY & \consWW \\[1.5ex]
%
%& C & S & T & P & A & G & N & D & E & Q & H & R & K & M & I & L & V & F & Y & W \\
%
%\end{tabular}
%}
%    \newpage
%
%    |structural|
%    \bigskip
%    \bigskip
%    
%{\footnotesize\tt
%
%\weighttable{structural}
%\hspace*{-2cm}
%\begin{tabular}{lrrrrrrrrrrrrrrrrrrrr}
%
%C & \consCC &&&&&&&&&&&&&&&&&&& \\
%
%S & \consSC & \consSS &&&&&&&&&&&&&&&&&& \\
%
%T & \consTC & \consTS & \consTT &&&&&&&&&&&&&&&&& \\
%
%P & \consPC & \consPS & \consPT & \consPP &&&&&&&&&&&&&&&& \\
%
%A & \consAC & \consAS & \consAT & \consAP & \consAA
%&&&&&&&&&&&&&&& \\
%
%G & \consGC & \consGS & \consGT & \consGP & \consGA
%& \consGG &&&&&&&&&&&&&& \\
%
%N & \consNC & \consNS & \consNT & \consNP & \consNA
%& \consNG & \consNN &&&&&&&&&&&&& \\
%
%D & \consDC & \consDS & \consDT & \consDP & \consDA
%& \consDG & \consDN & \consDD &&&&&&&&&&&& \\
%
%E & \consEC & \consES & \consET & \consEP & \consEA
%& \consEG & \consEN & \consED & \consEE &&&&&&&&&&& \\
%
%Q & \consQC & \consQS & \consQT & \consQP & \consQA
%& \consQG & \consQN & \consQD & \consQE & \consQQ
%&&&&&&&&&& \\
%
%H & \consHC & \consHS & \consHT & \consHP & \consHA
%& \consHG & \consHN & \consHD & \consHE & \consHQ
%& \consHH &&&&&&&&& \\
%
%R & \consRC & \consRS & \consRT & \consRP & \consRA
%& \consRG & \consRN & \consRD & \consRE & \consRQ
%& \consRH & \consRR &&&&&&&& \\
%
%K & \consKC & \consKS & \consKT & \consKP & \consKA
%& \consKG & \consKN & \consKD & \consKE & \consKQ
%& \consKH & \consKR & \consKK &&&&&&& \\
%
%M & \consMC & \consMS & \consMT & \consMP & \consMA
%& \consMG & \consMN & \consMD & \consME & \consMQ
%& \consMH & \consMR & \consMK & \consMM &&&&&& \\
%
%I & \consIC & \consIS & \consIT & \consIP & \consIA
%& \consIG & \consIN & \consID & \consIE & \consIQ
%& \consIH & \consIR & \consIK & \consIM & \consII
%&&&&& \\
%
%L & \consLC & \consLS & \consLT & \consLP & \consLA
%& \consLG & \consLN & \consLD & \consLE & \consLQ
%& \consLH & \consLR & \consLK & \consLM & \consLI
%& \consLL &&&& \\
%
%V & \consVC & \consVS & \consVT & \consVP & \consVA
%& \consVG & \consVN & \consVD & \consVE & \consVQ
%& \consVH & \consVR & \consVK & \consVM & \consVI
%& \consVL & \consVV &&& \\
%
%F & \consFC & \consFS & \consFT & \consFP & \consFA
%& \consFG & \consFN & \consFD & \consFE & \consFQ
%& \consFH & \consFR & \consFK & \consFM & \consFI
%& \consFL & \consFV & \consFF && \\
%
%Y & \consYC & \consYS & \consYT & \consYP & \consYA
%& \consYG & \consYN & \consYD & \consYE & \consYQ
%& \consYH & \consYR & \consYK & \consYM & \consYI
%& \consYL & \consYV & \consYF & \consYY & \\
%
%W & \consWC & \consWS & \consWT & \consWP & \consWA
%& \consWG & \consWN & \consWD & \consWE & \consWQ
%& \consWH & \consWR & \consWK & \consWM & \consWI
%& \consWL & \consWV & \consWF & \consWY & \consWW \\[1.5ex]
%
%& C & S & T & P & A & G & N & D & E & Q & H & R & K & M & I & L & V & F & Y & W \\
%
%\end{tabular}
%}
%    \newpage
%
%    |PAM250|
%    \bigskip
%    \bigskip
%
%{\footnotesize\tt
%
%\weighttable{PAM250}
%\hspace*{-2cm}
%\begin{tabular}{lrrrrrrrrrrrrrrrrrrrr}
%
%C & \consCC &&&&&&&&&&&&&&&&&&& \\
%
%S & \consSC & \consSS &&&&&&&&&&&&&&&&&& \\
%
%T & \consTC & \consTS & \consTT &&&&&&&&&&&&&&&&& \\
%
%P & \consPC & \consPS & \consPT & \consPP &&&&&&&&&&&&&&&& \\
%
%A & \consAC & \consAS & \consAT & \consAP & \consAA
%&&&&&&&&&&&&&&& \\
%
%G & \consGC & \consGS & \consGT & \consGP & \consGA
%& \consGG &&&&&&&&&&&&&& \\
%
%N & \consNC & \consNS & \consNT & \consNP & \consNA
%& \consNG & \consNN &&&&&&&&&&&&& \\
%
%D & \consDC & \consDS & \consDT & \consDP & \consDA
%& \consDG & \consDN & \consDD &&&&&&&&&&&& \\
%
%E & \consEC & \consES & \consET & \consEP & \consEA
%& \consEG & \consEN & \consED & \consEE &&&&&&&&&&& \\
%
%Q & \consQC & \consQS & \consQT & \consQP & \consQA
%& \consQG & \consQN & \consQD & \consQE & \consQQ
%&&&&&&&&&& \\
%
%H & \consHC & \consHS & \consHT & \consHP & \consHA
%& \consHG & \consHN & \consHD & \consHE & \consHQ
%& \consHH &&&&&&&&& \\
%
%R & \consRC & \consRS & \consRT & \consRP & \consRA
%& \consRG & \consRN & \consRD & \consRE & \consRQ
%& \consRH & \consRR &&&&&&&& \\
%
%K & \consKC & \consKS & \consKT & \consKP & \consKA
%& \consKG & \consKN & \consKD & \consKE & \consKQ
%& \consKH & \consKR & \consKK &&&&&&& \\
%
%M & \consMC & \consMS & \consMT & \consMP & \consMA
%& \consMG & \consMN & \consMD & \consME & \consMQ
%& \consMH & \consMR & \consMK & \consMM &&&&&& \\
%
%I & \consIC & \consIS & \consIT & \consIP & \consIA
%& \consIG & \consIN & \consID & \consIE & \consIQ
%& \consIH & \consIR & \consIK & \consIM & \consII
%&&&&& \\
%
%L & \consLC & \consLS & \consLT & \consLP & \consLA
%& \consLG & \consLN & \consLD & \consLE & \consLQ
%& \consLH & \consLR & \consLK & \consLM & \consLI
%& \consLL &&&& \\
%
%V & \consVC & \consVS & \consVT & \consVP & \consVA
%& \consVG & \consVN & \consVD & \consVE & \consVQ
%& \consVH & \consVR & \consVK & \consVM & \consVI
%& \consVL & \consVV &&& \\
%
%F & \consFC & \consFS & \consFT & \consFP & \consFA
%& \consFG & \consFN & \consFD & \consFE & \consFQ
%& \consFH & \consFR & \consFK & \consFM & \consFI
%& \consFL & \consFV & \consFF && \\
%
%Y & \consYC & \consYS & \consYT & \consYP & \consYA
%& \consYG & \consYN & \consYD & \consYE & \consYQ
%& \consYH & \consYR & \consYK & \consYM & \consYI
%& \consYL & \consYV & \consYF & \consYY & \\
%
%W & \consWC & \consWS & \consWT & \consWP & \consWA
%& \consWG & \consWN & \consWD & \consWE & \consWQ
%& \consWH & \consWR & \consWK & \consWM & \consWI
%& \consWL & \consWV & \consWF & \consWY & \consWW \\[1.5ex]
%
%& C & S & T & P & A & G & N & D & E & Q & H & R & K & M & I & L & V & F & Y & W \\
%
%\end{tabular}
%}
%    \newpage
%
%    |PAM100|
%    \bigskip
%    \bigskip
%    
%{\footnotesize\tt
%
%\weighttable{PAM100}
%\hspace*{-3cm}
%\begin{tabular}{lrrrrrrrrrrrrrrrrrrrr}
%
%C & \consCC &&&&&&&&&&&&&&&&&&& \\
%
%S & \consSC & \consSS &&&&&&&&&&&&&&&&&& \\
%
%T & \consTC & \consTS & \consTT &&&&&&&&&&&&&&&&& \\
%
%P & \consPC & \consPS & \consPT & \consPP &&&&&&&&&&&&&&&& \\
%
%A & \consAC & \consAS & \consAT & \consAP & \consAA
%&&&&&&&&&&&&&&& \\
%
%G & \consGC & \consGS & \consGT & \consGP & \consGA
%& \consGG &&&&&&&&&&&&&& \\
%
%N & \consNC & \consNS & \consNT & \consNP & \consNA
%& \consNG & \consNN &&&&&&&&&&&&& \\
%
%D & \consDC & \consDS & \consDT & \consDP & \consDA
%& \consDG & \consDN & \consDD &&&&&&&&&&&& \\
%
%E & \consEC & \consES & \consET & \consEP & \consEA
%& \consEG & \consEN & \consED & \consEE &&&&&&&&&&& \\
%
%Q & \consQC & \consQS & \consQT & \consQP & \consQA
%& \consQG & \consQN & \consQD & \consQE & \consQQ
%&&&&&&&&&& \\
%
%H & \consHC & \consHS & \consHT & \consHP & \consHA
%& \consHG & \consHN & \consHD & \consHE & \consHQ
%& \consHH &&&&&&&&& \\
%
%R & \consRC & \consRS & \consRT & \consRP & \consRA
%& \consRG & \consRN & \consRD & \consRE & \consRQ
%& \consRH & \consRR &&&&&&&& \\
%
%K & \consKC & \consKS & \consKT & \consKP & \consKA
%& \consKG & \consKN & \consKD & \consKE & \consKQ
%& \consKH & \consKR & \consKK &&&&&&& \\
%
%M & \consMC & \consMS & \consMT & \consMP & \consMA
%& \consMG & \consMN & \consMD & \consME & \consMQ
%& \consMH & \consMR & \consMK & \consMM &&&&&& \\
%
%I & \consIC & \consIS & \consIT & \consIP & \consIA
%& \consIG & \consIN & \consID & \consIE & \consIQ
%& \consIH & \consIR & \consIK & \consIM & \consII
%&&&&& \\
%
%L & \consLC & \consLS & \consLT & \consLP & \consLA
%& \consLG & \consLN & \consLD & \consLE & \consLQ
%& \consLH & \consLR & \consLK & \consLM & \consLI
%& \consLL &&&& \\
%
%V & \consVC & \consVS & \consVT & \consVP & \consVA
%& \consVG & \consVN & \consVD & \consVE & \consVQ
%& \consVH & \consVR & \consVK & \consVM & \consVI
%& \consVL & \consVV &&& \\
%
%F & \consFC & \consFS & \consFT & \consFP & \consFA
%& \consFG & \consFN & \consFD & \consFE & \consFQ
%& \consFH & \consFR & \consFK & \consFM & \consFI
%& \consFL & \consFV & \consFF && \\
%
%Y & \consYC & \consYS & \consYT & \consYP & \consYA
%& \consYG & \consYN & \consYD & \consYE & \consYQ
%& \consYH & \consYR & \consYK & \consYM & \consYI
%& \consYL & \consYV & \consYF & \consYY & \\
%
%W & \consWC & \consWS & \consWT & \consWP & \consWA
%& \consWG & \consWN & \consWD & \consWE & \consWQ
%& \consWH & \consWR & \consWK & \consWM & \consWI
%& \consWL & \consWV & \consWF & \consWY & \consWW \\[1.5ex]
%
%& C & S & T & P & A & G & N & D & E & Q & H & R & K & M & I & L & V & F & Y & W \\
%
%\end{tabular}
%}
%    \newpage
%
%    |BLOSUM62|
%\bigskip
%\bigskip
%    
%{\footnotesize\tt
%
%\weighttable{BLOSUM62}
%\hspace*{-2cm}
%\begin{tabular}{lrrrrrrrrrrrrrrrrrrrr}
%
%C & \consCC &&&&&&&&&&&&&&&&&&& \\
%
%S & \consSC & \consSS &&&&&&&&&&&&&&&&&& \\
%
%T & \consTC & \consTS & \consTT &&&&&&&&&&&&&&&&& \\
%
%P & \consPC & \consPS & \consPT & \consPP &&&&&&&&&&&&&&&& \\
%
%A & \consAC & \consAS & \consAT & \consAP & \consAA
%&&&&&&&&&&&&&&& \\
%
%G & \consGC & \consGS & \consGT & \consGP & \consGA
%& \consGG &&&&&&&&&&&&&& \\
%
%N & \consNC & \consNS & \consNT & \consNP & \consNA
%& \consNG & \consNN &&&&&&&&&&&&& \\
%
%D & \consDC & \consDS & \consDT & \consDP & \consDA
%& \consDG & \consDN & \consDD &&&&&&&&&&&& \\
%
%E & \consEC & \consES & \consET & \consEP & \consEA
%& \consEG & \consEN & \consED & \consEE &&&&&&&&&&& \\
%
%Q & \consQC & \consQS & \consQT & \consQP & \consQA
%& \consQG & \consQN & \consQD & \consQE & \consQQ
%&&&&&&&&&& \\
%
%H & \consHC & \consHS & \consHT & \consHP & \consHA
%& \consHG & \consHN & \consHD & \consHE & \consHQ
%& \consHH &&&&&&&&& \\
%
%R & \consRC & \consRS & \consRT & \consRP & \consRA
%& \consRG & \consRN & \consRD & \consRE & \consRQ
%& \consRH & \consRR &&&&&&&& \\
%
%K & \consKC & \consKS & \consKT & \consKP & \consKA
%& \consKG & \consKN & \consKD & \consKE & \consKQ
%& \consKH & \consKR & \consKK &&&&&&& \\
%
%M & \consMC & \consMS & \consMT & \consMP & \consMA
%& \consMG & \consMN & \consMD & \consME & \consMQ
%& \consMH & \consMR & \consMK & \consMM &&&&&& \\
%
%I & \consIC & \consIS & \consIT & \consIP & \consIA
%& \consIG & \consIN & \consID & \consIE & \consIQ
%& \consIH & \consIR & \consIK & \consIM & \consII
%&&&&& \\
%
%L & \consLC & \consLS & \consLT & \consLP & \consLA
%& \consLG & \consLN & \consLD & \consLE & \consLQ
%& \consLH & \consLR & \consLK & \consLM & \consLI
%& \consLL &&&& \\
%
%V & \consVC & \consVS & \consVT & \consVP & \consVA
%& \consVG & \consVN & \consVD & \consVE & \consVQ
%& \consVH & \consVR & \consVK & \consVM & \consVI
%& \consVL & \consVV &&& \\
%
%F & \consFC & \consFS & \consFT & \consFP & \consFA
%& \consFG & \consFN & \consFD & \consFE & \consFQ
%& \consFH & \consFR & \consFK & \consFM & \consFI
%& \consFL & \consFV & \consFF && \\
%
%Y & \consYC & \consYS & \consYT & \consYP & \consYA
%& \consYG & \consYN & \consYD & \consYE & \consYQ
%& \consYH & \consYR & \consYK & \consYM & \consYI
%& \consYL & \consYV & \consYF & \consYY & \\
%
%W & \consWC & \consWS & \consWT & \consWP & \consWA
%& \consWG & \consWN & \consWD & \consWE & \consWQ
%& \consWH & \consWR & \consWK & \consWM & \consWI
%& \consWL & \consWV & \consWF & \consWY & \consWW \\[1.5ex]
%
%& C & S & T & P & A & G & N & D & E & Q & H & R & K & M & I & L & V & F & Y & W \\
%
%\end{tabular}
%}
%    \bigskip
%
%
%    \newpage
%    \section{Quick Reference}
%    
%    \textbf{The \TeXshade{} logo}
%    \medskip
%    
%    \quad |\TeXshade|
%    
%    \vspace{1.5\baselineskip}
%    
%    \textbf{The \TeXshade{} environment} (\pageref{tsenvironment}\,ff.)
%    \medskip
%    
%    \begin{quote}
%    |\begin{texshade}[|\meta{parameterfile}|]|
%         |{|\meta{alignmentfile}|}|
%         
%    \quad\emph{further \emph{\TeXshade} commands, if needed}
%    
%    |\end{texshade}|
%    \end{quote}
%    \bigskip
%    
%    \textbf{Predefined shading modes}
%    \medskip
%    
%    \quad|\seqtype{|\meta{type}|}| 
%        \hfill(|P| -- peptide, |N| -- nucleotide) \hfill[\pageref{Lseqtype}]
%        
%    \medskip
%    
%    \quad|\shadingmode[|\meta{option}|]{|\meta{mode}|}|
%        \hfill[\pageref{Lshadingmode}]
%        
%    \medskip
%        
%    \begin{center}    
%    \begin{tabular}{lll}
%    \meta{mode} & \meta{option} &\\ \hline
%    |identical| & |allmatchspecial/number| &\\
%    |similar|   & |allmatchspecial/number| &\\
%    |T-Coffee|  & \meta{filename} &\\
%    |diverse|   & \meta{seqref} &\\
%    |functional|& \meta{type} & |charge| \\
%                &             & |hydropathy| \\
%                &             & |structure| \\
%                &             & |chemical| \\
%                &             & |rasmol| \\
%                &             & |standard area| \\
%                &             & |accessible area| \\ \hline
%    \end{tabular}
%    \end{center}
%    \medskip
%    
%    \quad|\allmatchspecial[|\meta{percentage}|]|
%        \hfill[\pageref{Lallmatchspecial}]
%    
%    \quad|\hideallmatchpositions|
%        \hfill[\pageref{Lhideallmatchpositions}]
%    
%    \quad|\shadeallresidues|
%        \hfill[\pageref{Lshadeallresidues}]
%        
%    \vspace{1.5\baselineskip}
%                
%    \textbf{Shading colors} (\pageref{Lshadingcolors}\,ff.)
%    \medskip
%    
%    \quad|\shadingcolors{|\meta{scheme}|}| \,\, (|blues|, |reds|, 
%                                       |greens|, |grays|, |black|)
%    
%    \quad|\nomatchresidues{|\meta{res.col.}|}{|\meta{shad.col.}|}{|\meta{case}|}{|\meta{style}|}|
%       
%    \quad|\similarresidues{|\meta{res.col.}|}{|\meta{shad.col.}|}{|\meta{case}|}{|\meta{style}|}|
%    
%    \quad|\conservedresidues{|\meta{res.col.}|}{|\meta{shad.col.}|}{|\meta{case}|}{|\meta{style}|}|
%       
%    \quad|\allmatchresidues{|\meta{res.col.}|}{|\meta{shad.col.}|}{|\meta{case}|}{|\meta{style}|}|
%       
%    \quad|\defshadingcolors{|\meta{name}|}|
%    \newpage
%    
%    \quad|\funcshadingstyle{|\meta{residue}|}{|\meta{res.col.}|}{|\meta{shad.color}|}|
%    
%    \hfill|{|\meta{case}|}{|\meta{style}|}| [\pageref{Lfuncshadingstyle}]
%    
%    \vspace{1.5\baselineskip}            
%    
%    \textbf{Residue grouping}
%    \medskip
%    
%    \quad|\pepsims{|\meta{residue}|}{|\meta{similars}|}|
%       \hfill[\pageref{Lpepsims}]
%    
%    \quad|\pepgroups{|\meta{group1}|,|\meta{group2}|, ... , |\meta{groupn}|}|
%       \hfill[\pageref{Lpepgroups}]
%    
%    \quad|\DNAsims{|\meta{residue}|}{|\meta{similars}|}|
%       \hfill[\pageref{LDNAsims}]
%    
%    \quad|\DNAgroups{|\meta{group1}|,|\meta{group2}|, ... , |\meta{groupn}|}|
%       \hfill[\pageref{LDNAgroups}]
%       
%    \vspace{1.5\baselineskip}
%    
%    \textbf{Definition of new functional shading modes}
%    \medskip
%    
%    \quad|\clearfuncgroups| \hfill [\pageref{Lclearfuncgroups}]
%    
%    \quad|\funcgroup{|\meta{descr}|}{|\meta{residues}|}{|\meta{res.col.}|}{|\meta{shad.col.}|}|
%
%    \hfill|{|\meta{case}|}{|\meta{style}|}|
%       \hfill[\pageref{Lfuncgroup}]
%       
%    \vspace{1.5\baselineskip}
%    
%    \textbf{Appearance of the consensus line}
%    \medskip
%    
%    \quad|\threshold[|\meta{percentage}|]{|\meta{percentage}|}|
%       \hfill[\pageref{Lthreshold}]
%    
%    \quad|\constosingleseq{|\meta{seqref}|}| 
%       \hfill[\pageref{Lconstosingleseq}]
%    
%    \quad|\showconsensus[|\meta{color/scale}|[,|\meta{color/scale}|]]{|\meta{top/bot.}|}|
%       \hfill[\pageref{Lshowconsensus}]
%
%    \quad|\exportconsensus[|\meta{filename}|]{|\meta{seqref}|}|
%       \hfill[\pageref{Lexportconsensus}]
%
%    \quad|\hideconsensus|
%       \hfill[\pageref{Lhideconsensus}]
%    
%    \quad|\nameconsensus{|\meta{name}|}|
%       \hfill[\pageref{Lnameconsensus}]
%    
%    \quad|\defconsensus{|\meta{symbol1}|}{|\meta{symbol2}|}{|\meta{symbol3}|}|
%       \hfill[\pageref{Ldefconsensus}]
%       
%    \vspace*{-0.5\baselineskip}
%    
%    \begin{tabbing}
%    \quad|\consensuscolors|\=|{|\meta{res.col.1}|}{|\meta{shad.col.1}|}|\\
%    
%    \>|{|\meta{res.col.2}|}{|\meta{shad.col.2}|}|\\
%    
%    \>|{|\meta{res.col.3}|}{|\meta{shad.col.3}|}|
%        \hspace{1.2in}[\pageref{Lconsensuscolors}]\\
%    \end{tabbing}
%
%    \quad|\weighttable{|\meta{table}|}| 
%       \hfill[\pageref{Lweighttable}]
%    
%    \quad|\setweight{|\meta{res.1}|}{|\meta{res.2}|}{|\meta{value}|}|
%       \hfill[\pageref{Lsetweight}]
%    
%    \quad|\gappenalty{|\meta{value}|}|
%       \hfill[\pageref{Lgappenalty}]
%    
%    \vspace{1.5\baselineskip}
%    
%    \textbf{Sequence logos}
%    \medskip
%    
%    \quad|\showsequencelogo[|\meta{colorset}|]{|\meta{top/bottom}|}|
%       \hfill[\pageref{Lshowsequencelogo}]
%    
%    \quad|\hidesequencelogo|
%       \hfill[\pageref{Lhidesequencelogo}]
%    
%    \quad|\clearlogocolors[|\meta{color}|]|
%       \hfill[\pageref{Lclearlogocolors}]
%    
%    \quad|\logocolor{|\meta{residues}|}{|\meta{color}|}|
%       \hfill[\pageref{Llogocolor}]
%    
%    \quad|\showlogoscale[|\meta{color}|]{|\meta{left/right/leftright}|}|
%       \hfill[\pageref{Lshowlogoscale}]
%    
%    \quad|\hidelogoscale|
%       \hfill[\pageref{Lhidelogoscale}]
%    
%    \quad|\logostretch{|\meta{factor}|}|
%       \hfill[\pageref{Llogostretch}]
%    
%    \quad|\namesequencelogo{|\meta{name}|}|
%       \hfill[\pageref{Lnamesequencelogo}]
%    
%    \quad|\dofrequencycorrection|
%       \hfill[\pageref{Ldofrequencycorrection}]
%    
%    \quad|\undofrequencycorrection|
%       \hfill[\pageref{Lundofrequencycorrection}]
%    
%    \vspace{1.5\baselineskip}
%    
%    \textbf{Subfamily logos}
%    \medskip
%    
%    \quad|\showsubfamilylogo[|\meta{colorset}|]{|\meta{top/bottom}|}|
%       \hfill[\pageref{Lshowsubfamilylogo}]
%    
%    \quad|\hidesubfamilylogo|
%       \hfill[\pageref{Lhidesubfamilylogo}]
%    
%    \quad|\setsubfamily{|\meta{seqrefs}|}|
%       \hfill[\pageref{Lsetsubfamily}]
%    
%    \quad|\shownegatives[|\meta{weak, medium, strong}|]|
%       \hfill[\pageref{Lshownegatives}]
%    
%    \quad|\hidenegatives|
%       \hfill[\pageref{Lhidenegatives}]
%    
%    \quad|\namesubfamilylogo[|\meta{neg.name}|]{|\meta{name}|}|
%       \hfill[\pageref{Lnamesubfamilylogo}]
%    
%    \quad|\relevance{|\meta{bit-value}|}|
%       \hfill[\pageref{Lrelevance}]
%    
%    \quad|\showrelevance[|\meta{color}|]{|\meta{symbol}|}|
%       \hfill[\pageref{Lshowrelevance}]
%    
%    \quad|\hiderelevance|
%       \hfill[\pageref{Lhiderelevance}]
%    
%    
%    \vspace{1.5\baselineskip}
%    
%    \textbf{Appearance of the sequence lines}
%    \medskip
%    
%    \quad|\shownames[|\meta{color}|]{|\meta{left/right}|}|
%       \hfill[\pageref{Lshownames}]
%    
%    \quad|\shownumbering[|\meta{color}|]{|\meta{left/right/leftright}|}|
%       \hfill[\pageref{Lshownumbering}]
%    
%    \quad|\nameseq{|\meta{seqref}|}{|\meta{name}|}| 
%       \hfill[\pageref{Lnameseq}]
%    
%    \quad|\namescolor{|\meta{color}|}|
%       \hfill[\pageref{Lnamescolor}]
%    
%    \quad|\namecolor{|\meta{seq1}|, ... ,|\meta{seq n}|}{|\meta{color}|}|
%       \hfill[\pageref{Lnamecolor}]
%    
%    \quad|\hidenames|
%       \hfill[\pageref{Lhidenames}]
%    
%    \quad|\hidename{|\meta{seq1}|, ... ,|\meta{seq n}|}|
%       \hfill[\pageref{Lhidename}]
%    
%    \quad|\numberingcolor{|\meta{color}|}|
%       \hfill[\pageref{Lnumberingcolor}]
%    
%    \quad|\numbercolor{|\meta{seq1}|, ... ,|\meta{seq n}|}{|\meta{color}|}|
%       \hfill[\pageref{Lnumbercolor}]
%    
%    \quad|\hidenumbering|
%       \hfill[\pageref{Lhidenumbering}]
%    
%    \quad|\hidenumber{|\meta{seq1}|, ... ,|\meta{seq n}|}|
%       \hfill[\pageref{Lhidenumber}]
%    
%    \quad|\hideresidues|
%       \hfill[\pageref{Lhideresidues}]
%    
%    \quad|\showresidues|
%       \hfill[\pageref{Lshowresidues}]
%    
%    \quad|\startnumber[|\meta{start..stop}|]{|\meta{seqref}|}{|\meta{startnumber}|}|
%       \hfill[\pageref{Lstartnumber}]
%    
%    \quad|\allowzero|
%       \hfill[\pageref{Lallowzero}]
%    
%    \quad|\disallowzero|
%       \hfill[\pageref{Lallowzero}]
%    
%    \quad|\seqlength{|\meta{seqref}|}{|\meta{length}|}|
%       \hfill[\pageref{Lseqlength}]
%    
%    \quad|\showruler[|\meta{color}|]{|\meta{top/bottom}|}{|\meta{seqref}|}|
%       \hfill[\pageref{Lshowruler}]
%    
%    \quad|\rulersteps{|\meta{number}|}|
%       \hfill[\pageref{Lrulersteps}]
%    
%    \quad|\rulercolor{|\meta{color}|}|
%       \hfill[\pageref{Lrulercolor}]
%    
%    \quad|\hideruler|
%       \hfill[\pageref{Lhideruler}]
%    
%    \quad|\rotateruler|
%       \hfill[\pageref{Lrotateruler}]
%    
%    \quad|\unrotateruler|
%       \hfill[\pageref{Lunrotateruler}]
%
%    \quad|\namerulerpos{|\meta{number}|}{|\meta{text}|[|\meta{color}|]}|
%       \hfill[\pageref{Lnamerulerpos}]
%    
%    \quad|\rulerspace{|\meta{length}|}|
%       \hfill[\pageref{Lrulerspace}]
%    
%    \quad|\gapchar{|\meta{symbol}|}| 
%       \qquad (incl. |rule|) \hfill [\pageref{Lgapchar}]
%    
%    \quad|\gapcolors{|\meta{symbol color}|}{|\meta{background color}|}|
%       \hfill[\pageref{Lgapcolors}]
%    
%    \quad|\showleadinggaps|
%       \hfill[\pageref{Lshowleadinggaps}]
%    
%    \quad|\hideleadinggaps|
%       \hfill[\pageref{Lhideleadinggaps}]
%    
%    \quad|\stopchar{|\meta{symbol}|}| 
%       \hfill [\pageref{Lstopchar}]
%    
%    \quad|\fingerprint{|\meta{res. per line}|}|
%       \hfill[\pageref{Lfingerprint}]
%       
%    \vspace{1.5\baselineskip}
%    
%    \textbf{Displaying selected residues in the alignment}
%    \medskip
%    
%    \quad|\setends[|\meta{startnumber}|]{|\meta{seqref}|}{|\meta{start..stop}|}|
%       \hfill[\pageref{Lsetends}]
%    
%    \quad|\setdomain{|\meta{seqref}|}{|\meta{selection}|}| (see |\shaderegion| p.\,\pageref{Lshaderegion})
%       \hfill[\pageref{Lsetdomain}]
%
%    \quad|\domaingaprule{|\meta{thickness}|}|
%       \hfill[\pageref{Lsetdomain}]
%
%    \quad|\domaingapcolors{|\meta{foreground}|}{|\meta{background}|}|
%       \hfill[\pageref{Lsetdomain}]
%
%    \vspace{1.5\baselineskip}
%    
%    \textbf{Hiding, killing, separating and ordering}
%    \medskip
%    
%    \quad|\hideseq{|\meta{seq1}|,|\meta{seq2}|,|\ldots|,|\meta{seq n}|}|
%       \hfill[\pageref{Lhideseq}]
%    
%    \quad|\hideseqs|
%       \hfill[\pageref{Lhideseqs}]
%    
%    \quad|\showseqs|
%       \hfill[\pageref{Lshowseqs}]
%    
%    \quad|\killseq{|\meta{seq1}|,|\meta{seq2}|,|\ldots|,|\meta{seq n}|}|
%       \hfill[\pageref{Lkillseq}]
%    
%    \quad|\donotshade{|\meta{seq1}|,|\meta{seq2},\ldots|,|\meta{seq n}|}|
%       \hfill[\pageref{Ldonotshade}]
%    
%    \quad|\separationline{|\meta{seqref}|}|
%       \hfill[\pageref{Lseparationline}]
%    
%    \quad|\smallsep|
%       \hfill[\pageref{Lsmallsep}]
%    
%    \quad|\medsep|
%       \hfill[\pageref{Lmedsep}]
%    
%    \quad|\bigsep|
%       \hfill[\pageref{Lbigsep}]
%    
%    \quad|\vsepspace{|\meta{length}|}|
%       \hfill[\pageref{Lvsepspace}]
%    
%    \quad|\orderseqs{|\meta{seq1}|,|\meta{seq2}|,|\ldots|,|\meta{seq n}|}|
%       \hfill[\pageref{Lorderseqs}]
%
%    \vspace{1.5\baselineskip}
%    
%    \textbf{Residues per line and further settings}
%    \medskip
%    
%    \quad|\residuesperline{|\meta{number}|}|
%       \hfill[\pageref{Lresiduesperline}]
%    
%    \quad|\residuesperline*{|\meta{number}|}|
%       \hfill[\pageref{Lresiduesperline*}]
%    
%    \quad|\charstretch{|\meta{factor}|}|
%       \hfill[\pageref{Lcharstretch}]
%    
%    \quad|\linestretch{|\meta{factor}|}|
%       \hfill[\pageref{Llinestretch}]
%    
%    \quad|\numberingwidth{|\meta{n digits}|}|
%       \hfill[\pageref{Lnumberingwidth}]
%    
%    \quad|\smallblockskip|
%       \hfill[\pageref{Lsmallblockskip}]
%    
%    \quad|\medblockskip|
%       \hfill[\pageref{Lmedblockskip}]
%    
%    \quad|\bigblockskip|
%       \hfill[\pageref{Lbigblockskip}]
%    
%    \quad|\noblockskip|
%       \hfill[\pageref{Lnoblockskip}]
%    
%    \quad|\vblockspace{|\meta{length}|}|
%       \hfill[\pageref{Lvblockspace}]
%    
%    \quad|\flexblockspace|
%       \hfill[\pageref{Lflexblockspace}]
%    
%    \quad|\fixblockspace|
%       \hfill[\pageref{Lfixblockspace}]
%    
%    \quad|\alignment{|\meta{left/center/right}|}|
%       \hfill[\pageref{Lalignment}]
%       
%    \vspace{1.5\baselineskip}
%    
%    
%
%    
%    \textbf{Individual shading and labeling of sequence stretches}
%    \medskip
%    
%    \quad|\shaderegion{|\meta{seqref}|}{|\meta{selection}|}{|\meta{res.col.}|}{|\meta{shad.col.}|}|
%       \hfill[\pageref{Lshaderegion}]
%
%    \medskip
%    
%    \quad |{|\meta{selection}|}| =
%    \medskip%    
%    
%    \quad\quad|{|\meta{start1}..\meta{stop1}|,|\meta{start2}..\meta{stop2}|,|\ldots|,|\meta{start n}..\meta{stop n}|}|
%    \medskip
%    
%    \quad\quad|{point[|\meta{dist}|]:|\meta{file}|,|\meta{num}|[CA/side]}|
%    \medskip
%    
%    \quad\quad|{line[|\meta{dist}|]:|\meta{file}|,|\meta{num1}|[CA/side],|\meta{num2}|[CA/side]}|
%    \medskip
%    
%    \quad\quad|{plane[|\meta{dist}|]:|\meta{file}|,|\meta{num1}|[CA/side],|\meta{num2}|[CA/side],|
%    \medskip
%
%    \hfill\meta{num3}|[CA/side]}|
%    \medskip
%    
%    
%    \quad|\printPDBlist{|\meta{selection}|}| \quad|\messagePDBlist{|\meta{selection}|}|
%       \hfill[\pageref{LprintPDBlist}]
%    
%    \quad|\shadeblock{|\meta{seqref}|}{|\meta{selection}|}{|\meta{res.col.}|}{|\meta{shad.col.}|}|
%       \hfill[\pageref{Lshadeblock}]
%    
%    \quad|\changeshadingcolors{|\meta{seqref}|}{|\meta{selection}|}{|\meta{name}|}|
%       \hfill[\pageref{Lchangeshadingcolors}]
%    
%    \quad|\emphregion{|\meta{seqref}|}{|\meta{selection}|}|
%       \hfill[\pageref{Lemphregion}]
%    
%    \quad|\emphblock{|\meta{seqref}|}{|\meta{selection}|}|
%       \hfill[\pageref{Lemphblock}]
%    
%    \quad|\emphdefault{|\meta{style}|}|
%       \hfill[\pageref{Lemphdefault}]
%    
%    \quad|\tintregion{|\meta{seqref}|}{|\meta{selection}|}|
%       \hfill[\pageref{Ltintregion}]
%    
%    \quad|\tintblock{|\meta{seqref}|}{|\meta{selection}|}|
%       \hfill[\pageref{Ltintblock}]
%    
%    \quad|\tintdefault{|\meta{effect}|}| \qquad\qquad|weak, normal, strong|
%       \hfill[\pageref{Ltintdefault}]
%    
%    \quad|\lowerregion{|\meta{seqref}|}{|\meta{selection}|}|
%       \hfill[\pageref{Llowerregion}]
%    
%    \quad|\lowerblock{|\meta{seqref}|}{|\meta{selection}|}|
%       \hfill[\pageref{Llowerblock}]
%    
%    \quad|\frameblock{|\meta{seqref}|}{|\meta{selection}|}{|\meta{color}|[|\meta{length}|]}|
%       \hfill[\pageref{Lframeblock}]
%    
%    \quad|\feature{|\meta{position}|}{|\meta{seqref}|}{|\meta{selection}|}|
%
%       \hfill |{|\meta{labelstyle}|}{|\meta{text}|}| [\pageref{Lfeature}]
%    
%    \begin{tabbing}
%    \quad\quad\quad|{|\meta{labelstyle}|}|\ \= = |{brace[|\meta{color}|]}|\\
%       \> = |{fill:|\meta{symbol}|[|\meta{textcolor}|]}|\\
%       \> = |{restriction[|\meta{color}|]}|\\
%       \> = |{helix[|\meta{helixcolor}|]}|\\
%       \> = |{box[|\meta{framecolor,boxcolor}|][|\meta{length}|]:|\\
%               \hspace{8.7cm}\meta{text}|[|\meta{textcolor}|]}|\\
%       \> = arrows and bars (|-=<',|$\vert$|o|)(|-=|)(|-=>',|$\vert$|o|)\\
%       \> = |{translate[|\meta{color}|]}|\\
%       \> = |{bar[|\meta{min}|,|\meta{max}|]:|\\
%               \hspace{5cm}\meta{properties/file/data}|[|\meta{color(,bgcolor)}|]}|\\
%       \> = |{color[|\meta{min}|,|\meta{max}|]:|\\
%               \hspace{5cm}\meta{properties/file/data}|[|\meta{scale}|]}|\\
%               \hspace{5cm}\meta{properties}: |hydrophobicity|, |charge|,\\
%               \hspace{7.4cm}|molweight|, |conservation|\\
%               \hspace{5cm}\meta{scale}: |Gray|, |BlueRed|, |RedBlue|,\\ 
%               \hspace{6.5cm}|GreenRed|, |RedGreen|, |ColdHot|,\\
%               \hspace{6.5cm}|HotCold|, |T-Coffee|\\
%    \end{tabbing}
%    
%    \quad|\ttopspace{|\meta{length}|}|
%       \hfill[\pageref{Lttopspace}]
%    
%    \quad|\topspace{|\meta{length}|}|
%       \hfill[\pageref{Ltopspace}]
%    
%    \quad|\bottomspace{|\meta{length}|}|
%       \hfill[\pageref{Lbottomspace}]
%    
%    \quad|\bbottomspace{|\meta{length}|}|
%       \hfill[\pageref{Lbottomspace}]
%    
%    \quad|\featurerule{|\meta{length}|}|
%       \hfill[\pageref{Lfeaturerule}]
%       
%    \quad|\bargraphstretch{|\meta{factor}|}|
%       \hfill[\pageref{Lbargraphstretch}]
%       
%    \quad|\colorscalestretch{|\meta{factor}|}|
%       \hfill[\pageref{Lcolorscalestretch}]
%       
%    \quad|\codon{|\meta{amino acid}|}{|\meta{triplet1,\ldots, triplet n}|}|
%       \hfill[\pageref{Lcodon}]
%    
%    \quad|\geneticcode{|\meta{filename}|}|
%       \hfill[\pageref{Lgeneticcode}]
%    
%    \quad|\backtranslabel[|\meta{size}|]{|\meta{style}|}|
%       \hfill[\pageref{Lbacktranslabel}]
%       
%    \quad|\backtranstext[|\meta{size}|]{|\meta{style}|}|
%       \hfill[\pageref{Lbacktranstext}]
%       
%    \begin{tabbing}
%    \quad\quad\quad|{|\meta{style}|}|\ \= = |{horizontal}|\\
%       \> = |{alternating}|\\
%       \> = |{zigzag}|\\
%       \> = |{oblique}|\\
%       \> = |{vertical}|
%    \end{tabbing}
%       
%    \quad |\showfeaturename{|\meta{ttttop...bbbbottom}|}{|\meta{name}|}|
%       \hfill[\pageref{Lshowfeaturename}]
%
%    \quad |\showfeaturestylename{|\meta{ttttop...bbbbottom}|}{|\meta{name}|}|
%       \hfill[\pageref{Lshowfeaturestylename}]
%
%    \quad |\hidefeaturename{|\meta{ttttop...bbbbottom}|}|
%       \hfill[\pageref{Lhidefeaturename}]
%
%    \quad |\hidefeaturestylename{|\meta{ttttop...bbbbottom}|}|
%       \hfill[\pageref{Lhidefeaturestylename}]
%
%    \quad |\hidefeaturenames|
%       \hfill[\pageref{Lhidefeaturenames}]
%
%    \quad|\hidefeaturestylenames|
%       \hfill[\pageref{Lhidefeaturestylenames}]
%
%    \quad |\featurenamescolor{|\meta{color}|}|
%       \hfill[\pageref{Lfeaturenamescolor}]
%
%    \quad |\featurestylenamescolor{|\meta{color}|}| 
%       \hfill[\pageref{Lfeaturestylenamescolor}]
%
%    \quad |\featurenamecolor{|\meta{ttttop...bbbbottom}|}{|\meta{color}|}|
%       \hfill[\pageref{Lfeaturenamecolor}]
%
%    \quad |\featurestylenamecolor{|\meta{ttttop...bbbbottom}|}{|\meta{color}|}| 
%       \hfill[\pageref{Lfeaturestylenamecolor}]
%
%    \vspace{1.5\baselineskip}
%    
%    \newpage
%
%    \textbf{Including secondary structure information}
%    \medskip
%    
%    \quad|\includeDSSP[make new]{|\meta{seqref}|}{|\meta{filename}|}|
%       \hfill[\pageref{LincludeDSSP}]
%    
%    \quad|\includeSTRIDE[make new]{|\meta{seqref}|}{|\meta{filename}|}|
%       \hfill[\pageref{LincludeSTRIDE}]
%    
%    \quad|\includePHDsec[make new]{|\meta{seqref}|}{|\meta{filename}|}|
%       \hfill[\pageref{LincludePHDsec}]
%    
%    \quad|\includePHDtopo[make new]{|\meta{seqref}|}{|\meta{filename}|}|
%       \hfill[\pageref{LincludePHDtopo}]
%    
%    \quad|\includeHMMTOP[make new]{|\meta{seqref}|[|\meta{seqref}|]}{|\meta{filename}|}|
%       \hfill[\pageref{LincludeHMMTOP}]
%    
%    \quad|\showonDSSP{|\meta{structures}|}|
%       \hfill[\pageref{LshowonDSSP}]
%    
%    \quad|\showonSTRIDE{|\meta{structures}|}|
%       \hfill[\pageref{LshowonSTRIDE}]
%    
%    \quad|\showonPHDsec{|\meta{structures}|}|
%       \hfill[\pageref{LshowonPHDsec}]
%    
%    \quad|\showonPHDtopo{|\meta{structures}|}|
%       \hfill[\pageref{LshowonPHDtopo}]
%    
%    \quad|\showonHMMTOP{|\meta{structures}|}|
%       \hfill[\pageref{LshowonHMMTOP}]
%    
%    \quad|\hideonDSSP{|\meta{structures}|}|
%       \hfill[\pageref{LhideonDSSP}]
%    
%    \quad|\hideonSTRIDE{|\meta{structures}|}|
%       \hfill[\pageref{LhideonSTRIDE}]
%    
%    \quad|\hideonPHDsec{|\meta{structures}|}|
%       \hfill[\pageref{LhideonPHDsec}]
%    
%    \quad|\hideonPHDtopo{|\meta{structures}|}|
%       \hfill[\pageref{LhideonPHDtopo}]
%    
%    \quad|\hideonHMMTOP{|\meta{structures}|}|
%       \hfill[\pageref{LhideonHMMTOP}]
%    
%    \quad|\appearance{|\meta{type}|}{|\meta{position}|}{|\meta{labelstyle}|}{|\meta{text}|}|
%       \hfill[\pageref{Lappearance}]
%    
%    \quad|\numcount|
%       \hfill[\pageref{Lnumcount}]
%    
%    \quad|\alphacount|
%       \hfill[\pageref{Lalphacount}]
%    
%    \quad|\Alphacount|
%       \hfill[\pageref{LAlphacount}]
%       
%    \quad|\romancount|
%       \hfill[\pageref{Lromancount}]
%    
%    \quad|\Romancount|
%       \hfill[\pageref{LRomancount}]
%    
%    \quad|\firstcolumnDSSP|
%       \hfill[\pageref{LfirstcolumnDSSP}]   
%             
%    \quad|\secondcolumnDSSP|
%       \hfill[\pageref{LsecondcolumnDSSP}]   
%             
%    \vspace{1.5\baselineskip}
%    
%
%    \textbf{Displaying and building legends}
%    \medskip
%    
%    \quad|\showlegend|
%       \hfill[\pageref{Lshowlegend}]
%    
%    \quad|\hidelegend|
%       \hfill[\pageref{Lhidelegend}]
%    
%    \quad|\movelegend{|\meta{x-offset}|}{|\meta{y-offset}|}|
%       \hfill[\pageref{Lmovelegend}]
%    
%    \quad|\germanlanguage|, |\spanishlanguage|, |\englishlanguage|
%       \hfill[\pageref{Lgermanlanguage}]
%    
%    \quad|\legendcolor{|\meta{color}|}|
%       \hfill[\pageref{Llegendcolor}]
%    
%    \quad|\shadebox{|\meta{color}|}|
%       \hfill[\pageref{Lshadebox}]
%       
%    \vspace{1.5\baselineskip}
%    
%    
%    \textbf{Adding captions to the alignment}
%    \medskip
%    
%    \quad|\showcaption[|\meta{top/bottom}|]{|\meta{text}|}|
%       \hfill[\pageref{Lshowcaption}]
%    
%    \quad|\shortcaption{|\meta{text}|}|
%       \hfill[\pageref{Lshortcaption}]
%    
%    \vspace{1.5\baselineskip}
%    
%    \newpage    
%    
%    \textbf{Font handling}
%    \medskip
%    
%    \quad|\setfamily{|\meta{text}|}{|\meta{family}|}|
%       \hfill[\pageref{Lsetfamily}]
%    
%    \quad|\setseries{|\meta{text}|}{|\meta{series}|}|
%       \hfill[\pageref{Lsetseries}]
%    
%    \quad|\setshape{|\meta{text}|}{|\meta{shape}|}|
%       \hfill[\pageref{Lsetshape}]
%    
%    \quad|\setsize{|\meta{text}|}{|\meta{size}|}|
%       \hfill[\pageref{Lsetsize}]
%    
%    \quad|\setfont{|\meta{text}|}{|\meta{family}|}{|\meta{series}|}{|\meta{shape}|}{|\meta{size}|}|
%       \hfill[\pageref{Lsetfont}]
%       
%    \medskip
%    
%    \quad |\featuresrm| \quad |\featurestiny| \hfill[\pageref{Lfeaturesrm}]
%    
%    \quad |\featuressf| \quad |\featuresscriptsize|
%    
%    \quad |\featurestt| \quad |\featuresfootnotesize|
%    
%    \quad |\featuresbf| \quad |\featuressmall|
%    
%    \quad |\featuresmd| \quad |\featuresnormalsize|
%    
%    \quad |\featuresit| \quad |\featureslarge|
%    
%    \quad |\featuressl| \quad |\featuresLarge|
%    
%    \quad |\featuressc| \quad |\featuresLARGE|
%    
%    \quad |\featuresup| \quad |\featureshuge|
%    
%    \quad |           | \quad |\featuresHuge|
%    \medskip
%    
%    Corresponding sets are provided for the
%    numbering (|\numberingrm| etc.), 
%    featurestyles (|featurestylesrm| etc.), names (|\namesrm| etc.), 
%    featurenames (|\featurenamesrm| etc.),
%    featurestylenames (|\featurestylenames| etc.),
%    residues (|\residuesrm| etc.),
%    hideblock labels (|hideblockrm| etc.), 
%    rulers (|\rulerrm| etc.), and
%    legend texts (|legendrm| etc.).
%    \bigskip
%    
%   
%    \textbf{Goodies---molweight and charge}
%    \medskip
%    
%    \quad|\molweight{|\meta{seqref}|}{|\meta{Da/kDa}|}|
%       \hfill[\pageref{Lmolweight}]
%    
%    \quad|\charge{|\meta{seqref}|}{|\meta{i/o/N/C}|}|    
%       \hfill[\pageref{Lcharge}]
%    
%    \quad|\percentsimilarity{|\meta{seqref1}|}{|\meta{seqref2}|}| 
%       \hfill[\pageref{Lpercentsimilarity}]
%
%    \quad|\percentidentity{|\meta{seqref1}|}{|\meta{seqref2}|}|
%       \hfill[\pageref{Lpercentidentity}]
%
%    \quad|\similaritytable|
%       \hfill[\pageref{Lsimilaritytable}]
%    
%    \StopEventually{%
%    \newpage
%    \section{References}
%      [1] \textsc{Carlisle, D.} The Standard \LaTeX{} `Graphics 
%          Bundle', |color.sty|.
%    
%      [2] \textsc{Karlin, S.; Ghandour, G.} (1985) Multiple-alphabet
%          amino acid sequence comparisons of the immunoglobulin
%          $\kappa$-chain constant domain.
%          \newblock \textit{Proc. Natl. Acad. Sci. USA}: \textbf{82},
%          8597--8601.
%          
%      [3] \textsc{Kyte, J.; Doolittle, R. F.} (1982) A simple
%          method for displaying the hydropathic character of a
%          protein.
%          \newblock \textit{J. Mol. Biol.}: \textbf{157}, 105--132.
%          
%      [4] \textsc{Rose, G. D.; Geselowitz, A. R.; Lesser, G. J.; 
%          Lee, R. H.; Zehfus, M. H.} (1985) Hydrophobicity of amino 
%          acid residues in globular proteins.
%          \newblock \textit{Science}: \textbf{229}, 835--838.
%          
%      [5] \textsc{Lesser, G. J.; Rose, G. D.} (1990) Hydrophobicity
%          of amino acid subgroups in proteins.
%          \newblock \textit{Proteins: structure, function and 
%          genetics}: \textbf{8}, 6--13.
%        
%      [6] \textsc{Fr\"ohlich, K.-U.} (1994) Sequence similarity 
%          presenter: a tool for the graphic display of similarities
%          of long sequences for use in presentations.
%          \newblock \textit{Comput. Applic. Biosci.}:
%          \textbf{10}, 179--183.
%          
%      [7] \textsc{Schneider, T.D.; Stephens, R.M.} (1990) Sequence logos:
%          a new way to display consensus
%          \newblock \textit{Nucleic Acid Res.}: \textbf{18}, 6097--6100.
% 
%      [8] DeLano Scientific LLC `www.pymol.org'
%         
%      [9] \textsc{Kabsch, W.; Sander, C.} (1983) Dictionary of
%          protein secondary structure: pattern recognition of
%          hydrogen-bonded and geometrical features.
%          \newblock \textit{Biopolymers}: \textbf{22}, 2577--2637.
%          
%      [10] \textsc{Frishman, D.; Argos, P.} (1995) Knowledge-based
%          protein secondary structure assignment.
%          \newblock \textit{Proteins: structure, function and 
%          genetics}: \textbf{23}, 566--579.    
%          
%      [11] \textsc{Rost, B.; Sander, C.} (1994)
%          Combining evolutionary information and neural networks to predict
%          protein secondary structure.
%          \newblock \textit{Proteins: structure, function and 
%          genetics}: \textbf{19}, 55--72.
%
%      [12] \textsc{Tusnady, G.E.; Simon, I.} (2001) 
%           The HMMTOP transmembrane topology prediction server.
%           \newblock \textit{Bioinformatics}: \textbf{17}, 849-850.
%              
%      [13] \textsc{Rokicki, T.} DVIPS: A \TeX{} driver.
%         
%      [14] \textsc{Beitz, E.} (2006) Subfamily logos: visualization of sequence 
%           deviations at alignment positions with high information content.
%           \newblock \textit{BMC Bioinformatics}: \textbf{7}:313.
%         
%    }
%    \section{Implementation}
%    \subsection{Documentation Driver}
%    \begin{macrocode}
%<*driver>
\documentclass[12pt,a4paper]{ltxdoc} 
\usepackage[dvips]{texshade}
\openin\structurefile = hyperref.sty
\ifeof\structurefile
 \else
     \usepackage[dvips,colorlinks]{hyperref}
 \fi
\closein\structurefile
\DisableCrossrefs
\sloppy
\def\BioTeX{\textsc{Bio}\kern-0.5ex\TeX}
\def\TeXtopo{\mbox{\TeX\textsf{topo}}}
\begin{document}
  \OnlyDescription
  \DocInput{texshade.dtx}
\end{document}
%</driver>
%    \end{macrocode}
%    \subsection{\texttt{texshade.sty}---no comments}
%    \begin{macrocode}
%<*texshade>
\NeedsTeXFormat{LaTeX2e}
\ProvidesPackage{texshade}[2011/05/13 LaTeX TeXshade (v1.24)]
\message{Package `texshade', Version 1.24 of 2011/12/01.}

\PassOptionsToPackage{dvips}{color}
\PassOptionsToPackage{dvips}{graphicx}
\DeclareOption*{%
   \PassOptionsToPackage{\CurrentOption}{color}%
   \PassOptionsToPackage{\CurrentOption}{graphicx}%
}
\ProcessOptions
\RequirePackage{color,graphics}

\expandafter\ifx\csname TeXshade\endcsname\relax \else \endinput \fi

\expandafter\ifx\csname TeXtopo\endcsname\relax \else
 \PackageError{TeXtopo}
  {TeXtopo loaded before TeXshade}
  {\MessageBreak
   For the proper function of the TeXtopo/TeXshade combo the \MessageBreak
   TeXshade package must be loaded before the TeXtopo package.\MessageBreak
   Please change the order of the \noexpand\usepackage commands in your
   \MessageBreak
   document header section or use the `biotex.sty'.\MessageBreak\MessageBreak
   Quit here by typing \space X <return>. \MessageBreak
}
\fi

\catcode`\@11

\def\rotopo#1{%
  \Grot@setangle{#1}%
  \setbox\z@\hbox\bgroup\ignorespaces}
\def\endrotopo{%
  \unskip\egroup
  \Grot@x\z@
  \Grot@y\z@
  \wd0\z@\dp0\z@\ht0\z@
  \Grot@box
}

\newread\structurefile  \newwrite\featurefile
\newread\alignfile      \newread\sublogofile
\newwrite\exp@rtfile

\expandafter\ifx\csname blacktriangleright\endcsname\relax
 \openin\structurefile = amssymb.sty
 \ifeof\structurefile
  \message{<AMS symbol style `amssymb.sty' not installed - using round heads>}
  \def\blacktriangleright{%
    \rule[\width@tmp]{0.65ex}{\temp@@length}\kern-0.55ex\ensuremath{\bullet}%
  }
  \def\blacktriangleleft{%
    \ensuremath{\bullet}\kern-0.55ex\rule[\width@tmp]{0.65ex}{\temp@@length}%
  }
 \else \RequirePackage[]{amssymb} \fi
 \closein\structurefile
\fi

\DeclareSymbolFont{alphahelix}{OML}{cmm}{m}{it}
\DeclareMathSymbol{\helixhook}{\mathrel}{alphahelix}{"5E}

\newcount\loopcount     \newcount\innerloopcount   \newcount\outerloopcount
\newcount\seq@count     \newcount\killseq@count
\newcount\seq@percent   \newcount\res@count
\newcount\seq@pointer   \newcount\pos@count
\newcount\res@perline   \newcount\end@count
\newcount\cons@count    \newcount\total@count
\newcount\temp@count    \newcount\triple@count
\newcount\temp@@count   \newcount\pos@sum

\newlength\box@width    \newlength\name@width
\newlength\box@depth    \newlength\width@tmp
\newlength\box@height   \newlength\number@width
\newlength\line@stretch
\newlength\center@fill  \newlength\arrow@width
\newlength\arrow@height \newlength\rule@thick
\newlength\arrow@thick  \newlength\logo@height
\newlength\equal@width  \newlength\equal@tmp
\newlength\equal@height \newlength\temp@@length
\newlength\vspace@legend
\newlength\hspace@legend

\newif\ifletter       \newif\ifnumber
\newif\ifnewres       \newif\ifall@shade
\newif\ifnames@right
\newif\ifnumbers@left \newif\ifnumbers@right
\newif\ifhide@cons    \newif\ifshow@cons
\newif\iffuncmode     \newif\iflegend@
\newif\ifT@coffee
\newif\ifnumbers@     \newif\ifnames@
\newif\ifgerm@n       \newif\ifsp@nish
\newif\ifrpl@fix
\newif\ifnosh@de      \newif\ifregionalshade
\newif\ifstart@       \newif\ifstop@
\newif\iftopfeature   \newif\ifbottomfeature
\newif\ifttopfeature  \newif\ifbbottomfeature
\newif\iftttopfeature \newif\ifbbbottomfeature
\newif\ifttttopfeature\newif\ifbbbbottomfeature
\newif\ifall@fshade   \newif\ifregionalemph 
\newif\ifregionallower    
\newif\ifframe@       \newif\ifregionaltint
\newif\ifshading@
\newif\ifshow@logo    \newif\ifshow@sublogo
\newif\ifhidechar     \newif\ifsh@wg@ps
\newif\ifsimmode
\newif\ifregionaltintnow
\newif\ifregionalemphnow
\newif\ifregionallowernow
\newif\ifregionalshadenow
\newif\iftopfeaturenow
\newif\ifttopfeaturenow
\newif\iftttopfeaturenow
\newif\ifttttopfeaturenow
\newif\ifbottomfeaturenow
\newif\ifbbottomfeaturenow
\newif\ifbbbottomfeaturenow
\newif\ifbbbbottomfeaturenow
\newif\ifframenow
\newif\ifshadingnow
\newif\iffix@

\expandafter\ifx\csname mdqon\endcsname\relax 
     \germ@nfalse \sp@nishfalse \def\cons@name{consensus}
\else \germ@ntrue \sp@nishfalse \def\cons@name{Konsensus} \fi

\def\n@me{Name:} \def\@msf{MSF:} \def\he@derend{//} \def\ampers@nd{&}
\def\comm@{,} \def\@loc{LOC} \def\@asg{ASG} \def\@t{@} \def\@HP{>HP:}
\def\gre@ter{>} \def\sm@ller{<} \def\N@{N} \def\equ@l{=} \def\H@{H}
\def\gap@char{.} \def\dom@char{{\dom@rule}} \def\yes{yes} \def\y@{y} \def\n@{n} 
\def\o@{o} \def\d@t{.} \def\questi@n{?} \def\st@p@char{*} \def\semic@n{;}
\def\br@cket{[} \def\p@r@gr@ph{�}
\def\@TOM{ATOM} \def\C@lpha{CA} \def\@point@{point} \def\@line@{line} \def\@plane@{plane}
\def\gap@rule{\rule[0.3\box@height]{\box@width}{\gap@rulethick}}
\def\dom@rule{\vrule depth\box@depth height\box@height width\domgap@rulethick}
\def\fgroup@num{0} \def\max@seqnumber{0}  \def\@lign@count{0}
\def\resn@m@tch{upper} \def\ressimm@tch{upper}
\def\resm@tch{upper}   \def\res@llm@tch{upper}
\def\tr@ns{translate}  \def\gr@ydef@ult{GrayDefault}
\xdef\par@{\expandafter\string\par}
\expandafter\def\csname fg@textcolor/\endcsname{White}
\expandafter\def\csname fg@color/\endcsname{White}
\expandafter\def\csname func@style/\endcsname{\csname textup\endcsname}
\expandafter\def\csname func@style*\endcsname{\csname textup\endcsname}
\expandafter\def\csname funcm@tch/\endcsname{upper}
\expandafter\def\csname funcm@tch*\endcsname{upper}

\setlength\hspace@legend{0pt} \setlength\vspace@legend{0pt}

\expandafter\xdef\csname log2@1\endcsname{0}
\expandafter\xdef\csname log2@2\endcsname{1000}
\expandafter\xdef\csname log2@3\endcsname{1585}
\expandafter\xdef\csname log2@4\endcsname{2000}
\expandafter\xdef\csname log2@5\endcsname{2322}
\expandafter\xdef\csname log2@6\endcsname{2585}
\expandafter\xdef\csname log2@7\endcsname{2807}
\expandafter\xdef\csname log2@8\endcsname{3000}
\expandafter\xdef\csname log2@9\endcsname{3170}
\expandafter\xdef\csname log2@10\endcsname{3322}
\expandafter\xdef\csname log2@11\endcsname{3459}
\expandafter\xdef\csname log2@12\endcsname{3585}
\expandafter\xdef\csname log2@13\endcsname{3700}
\expandafter\xdef\csname log2@14\endcsname{3807}
\expandafter\xdef\csname log2@15\endcsname{3907}
\expandafter\xdef\csname log2@16\endcsname{4000}
\expandafter\xdef\csname log2@17\endcsname{4087}
\expandafter\xdef\csname log2@18\endcsname{4170}
\expandafter\xdef\csname log2@19\endcsname{4248}
\expandafter\xdef\csname log2@20\endcsname{4322}
\expandafter\xdef\csname log2@21\endcsname{4392}
\expandafter\xdef\csname log2@22\endcsname{4459}
\expandafter\xdef\csname log2@23\endcsname{4524}
\expandafter\xdef\csname log2@24\endcsname{4585}
\expandafter\xdef\csname log2@25\endcsname{4644}
\expandafter\xdef\csname log2@26\endcsname{4700}
\expandafter\xdef\csname log2@27\endcsname{4755}
\expandafter\xdef\csname log2@28\endcsname{4807}
\expandafter\xdef\csname log2@29\endcsname{4858}
\expandafter\xdef\csname log2@30\endcsname{4907}
\expandafter\xdef\csname log2@31\endcsname{4954}
\expandafter\xdef\csname log2@32\endcsname{5000}
\expandafter\xdef\csname log2@33\endcsname{5044}
\expandafter\xdef\csname log2@34\endcsname{5087}
\expandafter\xdef\csname log2@35\endcsname{5129}
\expandafter\xdef\csname log2@36\endcsname{5170}
\expandafter\xdef\csname log2@37\endcsname{5209}
\expandafter\xdef\csname log2@38\endcsname{5248}
\expandafter\xdef\csname log2@39\endcsname{5285}
\expandafter\xdef\csname log2@40\endcsname{5322}
\expandafter\xdef\csname log2@41\endcsname{5358}
\expandafter\xdef\csname log2@42\endcsname{5392}
\expandafter\xdef\csname log2@43\endcsname{5426}
\expandafter\xdef\csname log2@44\endcsname{5459}
\expandafter\xdef\csname log2@45\endcsname{5492}
\expandafter\xdef\csname log2@46\endcsname{5524}
\expandafter\xdef\csname log2@47\endcsname{5555}
\expandafter\xdef\csname log2@48\endcsname{5585}
\expandafter\xdef\csname log2@49\endcsname{5615}
\expandafter\xdef\csname log2@50\endcsname{5644}
\expandafter\xdef\csname log2@51\endcsname{5672}
\expandafter\xdef\csname log2@52\endcsname{5700}
\expandafter\xdef\csname log2@53\endcsname{5728}
\expandafter\xdef\csname log2@54\endcsname{5755}
\expandafter\xdef\csname log2@55\endcsname{5781}
\expandafter\xdef\csname log2@56\endcsname{5807}
\expandafter\xdef\csname log2@57\endcsname{5833}
\expandafter\xdef\csname log2@58\endcsname{5858}
\expandafter\xdef\csname log2@59\endcsname{5883}
\expandafter\xdef\csname log2@60\endcsname{5907}
\expandafter\xdef\csname log2@61\endcsname{5931}
\expandafter\xdef\csname log2@62\endcsname{5954}
\expandafter\xdef\csname log2@63\endcsname{5977}
\expandafter\xdef\csname log2@64\endcsname{6000}
\expandafter\xdef\csname log2@65\endcsname{6022}
\expandafter\xdef\csname log2@66\endcsname{6044}
\expandafter\xdef\csname log2@67\endcsname{6066}
\expandafter\xdef\csname log2@68\endcsname{6087}
\expandafter\xdef\csname log2@69\endcsname{6109}
\expandafter\xdef\csname log2@70\endcsname{6129}
\expandafter\xdef\csname log2@71\endcsname{6149}
\expandafter\xdef\csname log2@72\endcsname{6170}
\expandafter\xdef\csname log2@73\endcsname{6190}
\expandafter\xdef\csname log2@74\endcsname{6209}
\expandafter\xdef\csname log2@75\endcsname{6229}
\expandafter\xdef\csname log2@76\endcsname{6248}
\expandafter\xdef\csname log2@77\endcsname{6267}
\expandafter\xdef\csname log2@78\endcsname{6285}
\expandafter\xdef\csname log2@79\endcsname{6304}
\expandafter\xdef\csname log2@80\endcsname{6322}
\expandafter\xdef\csname log2@81\endcsname{6340}
\expandafter\xdef\csname log2@82\endcsname{6358}
\expandafter\xdef\csname log2@83\endcsname{6375}
\expandafter\xdef\csname log2@84\endcsname{6392}
\expandafter\xdef\csname log2@85\endcsname{6409}
\expandafter\xdef\csname log2@86\endcsname{6426}
\expandafter\xdef\csname log2@87\endcsname{6443}
\expandafter\xdef\csname log2@88\endcsname{6459}
\expandafter\xdef\csname log2@89\endcsname{6476}
\expandafter\xdef\csname log2@90\endcsname{6492}
\expandafter\xdef\csname log2@91\endcsname{6508}
\expandafter\xdef\csname log2@92\endcsname{6524}
\expandafter\xdef\csname log2@93\endcsname{6539}
\expandafter\xdef\csname log2@94\endcsname{6555}
\expandafter\xdef\csname log2@95\endcsname{6570}
\expandafter\xdef\csname log2@96\endcsname{6585}
\expandafter\xdef\csname log2@97\endcsname{6600}
\expandafter\xdef\csname log2@98\endcsname{6615}
\expandafter\xdef\csname log2@99\endcsname{6629}
\expandafter\xdef\csname log2@100\endcsname{6644}
\expandafter\xdef\csname log2@101\endcsname{6658}
\expandafter\xdef\csname log2@102\endcsname{6672}
\expandafter\xdef\csname log2@103\endcsname{6687}
\expandafter\xdef\csname log2@104\endcsname{6700}
\expandafter\xdef\csname log2@105\endcsname{6714}
\expandafter\xdef\csname log2@106\endcsname{6728}
\expandafter\xdef\csname log2@107\endcsname{6741}
\expandafter\xdef\csname log2@108\endcsname{6755}
\expandafter\xdef\csname log2@109\endcsname{6768}
\expandafter\xdef\csname log2@110\endcsname{6781}
\expandafter\xdef\csname log2@111\endcsname{6794}
\expandafter\xdef\csname log2@112\endcsname{6807}
\expandafter\xdef\csname log2@113\endcsname{6820}
\expandafter\xdef\csname log2@114\endcsname{6833}
\expandafter\xdef\csname log2@115\endcsname{6845}
\expandafter\xdef\csname log2@116\endcsname{6858}
\expandafter\xdef\csname log2@117\endcsname{6870}
\expandafter\xdef\csname log2@118\endcsname{6883}
\expandafter\xdef\csname log2@119\endcsname{6895}
\expandafter\xdef\csname log2@120\endcsname{6907}
\expandafter\xdef\csname log2@121\endcsname{6919}
\expandafter\xdef\csname log2@122\endcsname{6931}
\expandafter\xdef\csname log2@123\endcsname{6943}
\expandafter\xdef\csname log2@124\endcsname{6954}
\expandafter\xdef\csname log2@125\endcsname{6966}
\expandafter\xdef\csname log2@126\endcsname{6977}
\expandafter\xdef\csname log2@127\endcsname{6989}
\expandafter\xdef\csname log2@128\endcsname{7000}
\expandafter\xdef\csname log2@129\endcsname{7011}
\expandafter\xdef\csname log2@130\endcsname{7022}
\expandafter\xdef\csname log2@131\endcsname{7033}
\expandafter\xdef\csname log2@132\endcsname{7044}
\expandafter\xdef\csname log2@133\endcsname{7055}
\expandafter\xdef\csname log2@134\endcsname{7066}
\expandafter\xdef\csname log2@135\endcsname{7077}
\expandafter\xdef\csname log2@136\endcsname{7088}
\expandafter\xdef\csname log2@137\endcsname{7098}
\expandafter\xdef\csname log2@138\endcsname{7108}
\expandafter\xdef\csname log2@139\endcsname{7119}
\expandafter\xdef\csname log2@140\endcsname{7129}
\expandafter\xdef\csname log2@141\endcsname{7140}
\expandafter\xdef\csname log2@142\endcsname{7150}
\expandafter\xdef\csname log2@143\endcsname{7160}
\expandafter\xdef\csname log2@144\endcsname{7170}
\expandafter\xdef\csname log2@145\endcsname{7180}
\expandafter\xdef\csname log2@146\endcsname{7190}
\expandafter\xdef\csname log2@147\endcsname{7200}
\expandafter\xdef\csname log2@148\endcsname{7209}
\expandafter\xdef\csname log2@149\endcsname{7219}
\expandafter\xdef\csname log2@150\endcsname{7229}
\expandafter\xdef\csname log2@151\endcsname{7238}
\expandafter\xdef\csname log2@152\endcsname{7248}
\expandafter\xdef\csname log2@153\endcsname{7257}
\expandafter\xdef\csname log2@154\endcsname{7267}
\expandafter\xdef\csname log2@155\endcsname{7276}
\expandafter\xdef\csname log2@156\endcsname{7285}
\expandafter\xdef\csname log2@157\endcsname{7295}
\expandafter\xdef\csname log2@158\endcsname{7304}
\expandafter\xdef\csname log2@159\endcsname{7313}
\expandafter\xdef\csname log2@160\endcsname{7323}
\expandafter\xdef\csname log2@161\endcsname{7331}
\expandafter\xdef\csname log2@162\endcsname{7340}
\expandafter\xdef\csname log2@163\endcsname{7349}
\expandafter\xdef\csname log2@164\endcsname{7358}
\expandafter\xdef\csname log2@165\endcsname{7366}
\expandafter\xdef\csname log2@166\endcsname{7375}
\expandafter\xdef\csname log2@167\endcsname{7374}
\expandafter\xdef\csname log2@168\endcsname{7392}
\expandafter\xdef\csname log2@169\endcsname{7401}
\expandafter\xdef\csname log2@170\endcsname{7409}
\expandafter\xdef\csname log2@171\endcsname{7418}
\expandafter\xdef\csname log2@172\endcsname{7463}
\expandafter\xdef\csname log2@173\endcsname{7435}
\expandafter\xdef\csname log2@174\endcsname{7443}
\expandafter\xdef\csname log2@175\endcsname{7451}
\expandafter\xdef\csname log2@176\endcsname{7460}
\expandafter\xdef\csname log2@177\endcsname{7468}
\expandafter\xdef\csname log2@178\endcsname{7476}
\expandafter\xdef\csname log2@179\endcsname{7484}
\expandafter\xdef\csname log2@180\endcsname{7492}
\expandafter\xdef\csname log2@181\endcsname{7500}
\expandafter\xdef\csname log2@182\endcsname{7508}
\expandafter\xdef\csname log2@183\endcsname{7516}
\expandafter\xdef\csname log2@184\endcsname{7524}
\expandafter\xdef\csname log2@185\endcsname{7531}
\expandafter\xdef\csname log2@186\endcsname{7539}
\expandafter\xdef\csname log2@187\endcsname{7547}
\expandafter\xdef\csname log2@188\endcsname{7555}
\expandafter\xdef\csname log2@189\endcsname{7562}
\expandafter\xdef\csname log2@190\endcsname{7570}
\expandafter\xdef\csname log2@191\endcsname{7577}
\expandafter\xdef\csname log2@192\endcsname{7585}
\expandafter\xdef\csname log2@193\endcsname{7592}
\expandafter\xdef\csname log2@194\endcsname{7600}
\expandafter\xdef\csname log2@195\endcsname{7607}
\expandafter\xdef\csname log2@196\endcsname{7615}
\expandafter\xdef\csname log2@197\endcsname{7622}
\expandafter\xdef\csname log2@198\endcsname{7629}
\expandafter\xdef\csname log2@199\endcsname{7637}
\expandafter\xdef\csname log2@200\endcsname{7644}
\expandafter\xdef\csname log2@201\endcsname{7651}
\expandafter\xdef\csname log2@202\endcsname{7658}
\expandafter\xdef\csname log2@203\endcsname{7665}
\expandafter\xdef\csname log2@204\endcsname{7672}
\expandafter\xdef\csname log2@205\endcsname{7679}
\expandafter\xdef\csname log2@206\endcsname{7687}
\expandafter\xdef\csname log2@207\endcsname{7693}
\expandafter\xdef\csname log2@208\endcsname{7700}
\expandafter\xdef\csname log2@209\endcsname{7707}
\expandafter\xdef\csname log2@210\endcsname{7714}
\expandafter\xdef\csname log2@211\endcsname{7721}
\expandafter\xdef\csname log2@212\endcsname{7728}
\expandafter\xdef\csname log2@213\endcsname{7735}
\expandafter\xdef\csname log2@214\endcsname{7741}
\expandafter\xdef\csname log2@215\endcsname{7748}
\expandafter\xdef\csname log2@216\endcsname{7755}
\expandafter\xdef\csname log2@217\endcsname{7761}
\expandafter\xdef\csname log2@218\endcsname{7768}
\expandafter\xdef\csname log2@219\endcsname{7775}
\expandafter\xdef\csname log2@220\endcsname{7781}


\expandafter\xdef\csname ch@r@65\endcsname{A}
\expandafter\xdef\csname ch@r@66\endcsname{B}
\expandafter\xdef\csname ch@r@67\endcsname{C}
\expandafter\xdef\csname ch@r@68\endcsname{D}
\expandafter\xdef\csname ch@r@69\endcsname{E}
\expandafter\xdef\csname ch@r@70\endcsname{F}
\expandafter\xdef\csname ch@r@71\endcsname{G}
\expandafter\xdef\csname ch@r@72\endcsname{H}
\expandafter\xdef\csname ch@r@73\endcsname{I}
\expandafter\xdef\csname ch@r@74\endcsname{J}
\expandafter\xdef\csname ch@r@75\endcsname{K}
\expandafter\xdef\csname ch@r@76\endcsname{L}
\expandafter\xdef\csname ch@r@77\endcsname{M}
\expandafter\xdef\csname ch@r@78\endcsname{N}
\expandafter\xdef\csname ch@r@79\endcsname{O}
\expandafter\xdef\csname ch@r@80\endcsname{P}
\expandafter\xdef\csname ch@r@81\endcsname{Q}
\expandafter\xdef\csname ch@r@82\endcsname{R}
\expandafter\xdef\csname ch@r@83\endcsname{S}
\expandafter\xdef\csname ch@r@84\endcsname{T}
\expandafter\xdef\csname ch@r@85\endcsname{U}
\expandafter\xdef\csname ch@r@86\endcsname{V}
\expandafter\xdef\csname ch@r@87\endcsname{W}
\expandafter\xdef\csname ch@r@88\endcsname{X}
\expandafter\xdef\csname ch@r@89\endcsname{Y}
\expandafter\xdef\csname ch@r@90\endcsname{Z}

\def\clear@sims{%
    \expandafter\xdef\csname \prfx simA\endcsname{(1)A}
    \expandafter\xdef\csname \prfx simB\endcsname{(1)B}
    \expandafter\xdef\csname \prfx simC\endcsname{(1)C}
    \expandafter\xdef\csname \prfx simD\endcsname{(1)D}
    \expandafter\xdef\csname \prfx simE\endcsname{(1)E}
    \expandafter\xdef\csname \prfx simF\endcsname{(1)F}
    \expandafter\xdef\csname \prfx simG\endcsname{(1)G}
    \expandafter\xdef\csname \prfx simH\endcsname{(1)H}
    \expandafter\xdef\csname \prfx simI\endcsname{(1)I}
    \expandafter\xdef\csname \prfx simJ\endcsname{(1)J}
    \expandafter\xdef\csname \prfx simK\endcsname{(1)K}
    \expandafter\xdef\csname \prfx simL\endcsname{(1)L}
    \expandafter\xdef\csname \prfx simM\endcsname{(1)M}
    \expandafter\xdef\csname \prfx simN\endcsname{(1)N}
    \expandafter\xdef\csname \prfx simO\endcsname{(1)O}
    \expandafter\xdef\csname \prfx simP\endcsname{(1)P}
    \expandafter\xdef\csname \prfx simQ\endcsname{(1)Q}
    \expandafter\xdef\csname \prfx simR\endcsname{(1)R}
    \expandafter\xdef\csname \prfx simS\endcsname{(1)S}
    \expandafter\xdef\csname \prfx simT\endcsname{(1)T}
    \expandafter\xdef\csname \prfx simU\endcsname{(1)U}
    \expandafter\xdef\csname \prfx simV\endcsname{(1)V}
    \expandafter\xdef\csname \prfx simW\endcsname{(1)W}
    \expandafter\xdef\csname \prfx simX\endcsname{(1)X}
    \expandafter\xdef\csname \prfx simY\endcsname{(1)Y}
    \expandafter\xdef\csname \prfx simZ\endcsname{(1)Z}
}

\def\clear@simpairs{%
\xdef\simpairCC{1} \xdef\simpairCS{0} \xdef\simpairCT{0} \xdef\simpairCP{0} \xdef\simpairCA{0}
\xdef\simpairCG{0} \xdef\simpairCN{0} \xdef\simpairCD{0} \xdef\simpairCE{0} \xdef\simpairCQ{0}
\xdef\simpairCH{0} \xdef\simpairCR{0} \xdef\simpairCK{0} \xdef\simpairCM{0} \xdef\simpairCI{0}
\xdef\simpairCL{0} \xdef\simpairCV{0} \xdef\simpairCF{0} \xdef\simpairCY{0} \xdef\simpairCW{0}
\xdef\simpairCB{0}  \xdef\simpairCJ{0}  \xdef\simpairCO{0}  \xdef\simpairCU{0}  \xdef\simpairCX{0}  \xdef\simpairCZ{0}

\xdef\simpairSC{0} \xdef\simpairSS{1} \xdef\simpairST{0} \xdef\simpairSP{0} \xdef\simpairSA{0}
\xdef\simpairSG{0} \xdef\simpairSN{0} \xdef\simpairSD{0} \xdef\simpairSE{0} \xdef\simpairSQ{0}
\xdef\simpairSH{0} \xdef\simpairSR{0} \xdef\simpairSK{0} \xdef\simpairSM{0} \xdef\simpairSI{0}
\xdef\simpairSL{0} \xdef\simpairSV{0} \xdef\simpairSF{0} \xdef\simpairSY{0} \xdef\simpairSW{0}
\xdef\simpairSB{0}  \xdef\simpairSJ{0}  \xdef\simpairSO{0}  \xdef\simpairSU{0}  \xdef\simpairSX{0}  \xdef\simpairSZ{0}

\xdef\simpairTC{0} \xdef\simpairTS{0} \xdef\simpairTT{1} \xdef\simpairTP{0} \xdef\simpairTA{0}
\xdef\simpairTG{0} \xdef\simpairTN{0} \xdef\simpairTD{0} \xdef\simpairTE{0} \xdef\simpairTQ{0}
\xdef\simpairTH{0} \xdef\simpairTR{0} \xdef\simpairTK{0} \xdef\simpairTM{0} \xdef\simpairTI{0}
\xdef\simpairTL{0} \xdef\simpairTV{0} \xdef\simpairTF{0} \xdef\simpairTY{0} \xdef\simpairTW{0}
\xdef\simpairTB{0}  \xdef\simpairTJ{0}  \xdef\simpairTO{0}  \xdef\simpairTU{0}  \xdef\simpairTX{0}  \xdef\simpairTZ{0}

\xdef\simpairPC{0} \xdef\simpairPS{0} \xdef\simpairPT{0} \xdef\simpairPP{1} \xdef\simpairPA{0}
\xdef\simpairPG{0} \xdef\simpairPN{0} \xdef\simpairPD{0} \xdef\simpairPE{0} \xdef\simpairPQ{0}
\xdef\simpairPH{0} \xdef\simpairPR{0} \xdef\simpairPK{0} \xdef\simpairPM{0} \xdef\simpairPI{0}
\xdef\simpairPL{0} \xdef\simpairPV{0} \xdef\simpairPF{0} \xdef\simpairPY{0} \xdef\simpairPW{0}
\xdef\simpairPB{0}  \xdef\simpairPJ{0}  \xdef\simpairPO{0}  \xdef\simpairPU{0}  \xdef\simpairPX{0}  \xdef\simpairPZ{0}

\xdef\simpairAC{0} \xdef\simpairAS{0} \xdef\simpairAT{0} \xdef\simpairAP{0} \xdef\simpairAA{1}
\xdef\simpairAG{0} \xdef\simpairAN{0} \xdef\simpairAD{0} \xdef\simpairAE{0} \xdef\simpairAQ{0}
\xdef\simpairAH{0} \xdef\simpairAR{0} \xdef\simpairAK{0} \xdef\simpairAM{0} \xdef\simpairAI{0}
\xdef\simpairAL{0} \xdef\simpairAV{0} \xdef\simpairAF{0} \xdef\simpairAY{0} \xdef\simpairAW{0} 
\xdef\simpairAB{0}  \xdef\simpairAJ{0}  \xdef\simpairAO{0}  \xdef\simpairAU{0}  \xdef\simpairAX{0}  \xdef\simpairAZ{0}

\xdef\simpairGC{0} \xdef\simpairGS{0} \xdef\simpairGT{0} \xdef\simpairGP{0} \xdef\simpairGA{0}
\xdef\simpairGG{1} \xdef\simpairGN{0} \xdef\simpairGD{0} \xdef\simpairGE{0} \xdef\simpairGQ{0}
\xdef\simpairGH{0} \xdef\simpairGR{0} \xdef\simpairGK{0} \xdef\simpairGM{0} \xdef\simpairGI{0}
\xdef\simpairGL{0} \xdef\simpairGV{0} \xdef\simpairGF{0} \xdef\simpairGY{0} \xdef\simpairGW{0}
\xdef\simpairGB{0}  \xdef\simpairGJ{0}  \xdef\simpairGO{0}  \xdef\simpairGU{0}  \xdef\simpairGX{0}  \xdef\simpairGZ{0}

\xdef\simpairNC{0} \xdef\simpairNS{0} \xdef\simpairNT{0} \xdef\simpairNP{0} \xdef\simpairNA{0}
\xdef\simpairNG{0} \xdef\simpairNN{1} \xdef\simpairND{0} \xdef\simpairNE{0} \xdef\simpairNQ{0}
\xdef\simpairNH{0} \xdef\simpairNR{0} \xdef\simpairNK{0} \xdef\simpairNM{0} \xdef\simpairNI{0}
\xdef\simpairNL{0} \xdef\simpairNV{0} \xdef\simpairNF{0} \xdef\simpairNY{0} \xdef\simpairNW{0}
\xdef\simpairNB{0}  \xdef\simpairNJ{0}  \xdef\simpairNO{0}  \xdef\simpairNU{0}  \xdef\simpairNX{0}  \xdef\simpairNZ{0}

\xdef\simpairDC{0} \xdef\simpairDS{0} \xdef\simpairDT{0} \xdef\simpairDP{0} \xdef\simpairDA{0}
\xdef\simpairDG{0} \xdef\simpairDN{0} \xdef\simpairDD{1} \xdef\simpairDE{0} \xdef\simpairDQ{0}
\xdef\simpairDH{0} \xdef\simpairDR{0} \xdef\simpairDK{0} \xdef\simpairDM{0} \xdef\simpairDI{0}
\xdef\simpairDL{0} \xdef\simpairDV{0} \xdef\simpairDF{0} \xdef\simpairDY{0} \xdef\simpairDW{0}
\xdef\simpairDB{0}  \xdef\simpairDJ{0}  \xdef\simpairDO{0}  \xdef\simpairDU{0}  \xdef\simpairDX{0}  \xdef\simpairDZ{0}

\xdef\simpairEC{0} \xdef\simpairES{0} \xdef\simpairET{0} \xdef\simpairEP{0} \xdef\simpairEA{0}
\xdef\simpairEG{0} \xdef\simpairEN{0} \xdef\simpairED{0} \xdef\simpairEE{1} \xdef\simpairEQ{0}
\xdef\simpairEH{0} \xdef\simpairER{0} \xdef\simpairEK{0} \xdef\simpairEM{0} \xdef\simpairEI{0}
\xdef\simpairEL{0} \xdef\simpairEV{0} \xdef\simpairEF{0} \xdef\simpairEY{0} \xdef\simpairEW{0}
\xdef\simpairEB{0}  \xdef\simpairEJ{0}  \xdef\simpairEO{0}  \xdef\simpairEU{0}  \xdef\simpairEX{0}  \xdef\simpairEZ{0}

\xdef\simpairQC{0} \xdef\simpairQS{0} \xdef\simpairQT{0} \xdef\simpairQP{0} \xdef\simpairQA{0}
\xdef\simpairQG{0} \xdef\simpairQN{0} \xdef\simpairQD{0} \xdef\simpairQE{0} \xdef\simpairQQ{1}
\xdef\simpairQH{0} \xdef\simpairQR{0} \xdef\simpairQK{0} \xdef\simpairQM{0} \xdef\simpairQI{0}
\xdef\simpairQL{0} \xdef\simpairQV{0} \xdef\simpairQF{0} \xdef\simpairQY{0} \xdef\simpairQW{0}
\xdef\simpairQB{0}  \xdef\simpairQJ{0}  \xdef\simpairQO{0}  \xdef\simpairQU{0}  \xdef\simpairQX{0}  \xdef\simpairQZ{0}

\xdef\simpairHC{0} \xdef\simpairHS{0} \xdef\simpairHT{0} \xdef\simpairHP{0} \xdef\simpairHA{0}
\xdef\simpairHG{0} \xdef\simpairHN{0} \xdef\simpairHD{0} \xdef\simpairHE{0} \xdef\simpairHQ{0}
\xdef\simpairHH{1} \xdef\simpairHR{0} \xdef\simpairHK{0} \xdef\simpairHM{0} \xdef\simpairHI{0}
\xdef\simpairHL{0} \xdef\simpairHV{0} \xdef\simpairHF{0} \xdef\simpairHY{0} \xdef\simpairHW{0}
\xdef\simpairHB{0}  \xdef\simpairHJ{0}  \xdef\simpairHO{0}  \xdef\simpairHU{0}  \xdef\simpairHX{0}  \xdef\simpairHZ{0}

\xdef\simpairRC{0} \xdef\simpairRS{0} \xdef\simpairRT{0} \xdef\simpairRP{0} \xdef\simpairRA{0}
\xdef\simpairRG{0} \xdef\simpairRN{0} \xdef\simpairRD{0} \xdef\simpairRE{0} \xdef\simpairRQ{0}
\xdef\simpairRH{0} \xdef\simpairRR{1} \xdef\simpairRK{0} \xdef\simpairRM{0} \xdef\simpairRI{0}
\xdef\simpairRL{0} \xdef\simpairRV{0} \xdef\simpairRF{0} \xdef\simpairRY{0} \xdef\simpairRW{0}
\xdef\simpairRB{0}  \xdef\simpairRJ{0}  \xdef\simpairRO{0}  \xdef\simpairRU{0}  \xdef\simpairRX{0}  \xdef\simpairRZ{0}

\xdef\simpairKC{0} \xdef\simpairKS{0} \xdef\simpairKT{0} \xdef\simpairKP{0} \xdef\simpairKA{0}
\xdef\simpairKG{0} \xdef\simpairKN{0} \xdef\simpairKD{0} \xdef\simpairKE{0} \xdef\simpairKQ{0}
\xdef\simpairKH{0} \xdef\simpairKR{0} \xdef\simpairKK{1} \xdef\simpairKM{0} \xdef\simpairKI{0}
\xdef\simpairKL{0} \xdef\simpairKV{0} \xdef\simpairKF{0} \xdef\simpairKY{0} \xdef\simpairKW{0}
\xdef\simpairKB{0}  \xdef\simpairKJ{0}  \xdef\simpairKO{0}  \xdef\simpairKU{0}  \xdef\simpairKX{0}  \xdef\simpairKZ{0}

\xdef\simpairMC{0} \xdef\simpairMS{0} \xdef\simpairMT{0} \xdef\simpairMP{0} \xdef\simpairMA{0}
\xdef\simpairMG{0} \xdef\simpairMN{0} \xdef\simpairMD{0} \xdef\simpairME{0} \xdef\simpairMQ{0}
\xdef\simpairMH{0} \xdef\simpairMR{0} \xdef\simpairMK{0} \xdef\simpairMM{1} \xdef\simpairMI{0}
\xdef\simpairML{0} \xdef\simpairMV{0} \xdef\simpairMF{0} \xdef\simpairMY{0} \xdef\simpairMW{0}
\xdef\simpairMB{0}  \xdef\simpairMJ{0}  \xdef\simpairMO{0}  \xdef\simpairMU{0}  \xdef\simpairMX{0}  \xdef\simpairMZ{0}

\xdef\simpairIC{0} \xdef\simpairIS{0} \xdef\simpairIT{0} \xdef\simpairIP{0} \xdef\simpairIA{0}
\xdef\simpairIG{0} \xdef\simpairIN{0} \xdef\simpairID{0} \xdef\simpairIE{0} \xdef\simpairIQ{0}
\xdef\simpairIH{0} \xdef\simpairIR{0} \xdef\simpairIK{0} \xdef\simpairIM{0} \xdef\simpairII{1}
\xdef\simpairIL{0} \xdef\simpairIV{0} \xdef\simpairIF{0} \xdef\simpairIY{0} \xdef\simpairIW{0}
\xdef\simpairIB{0}  \xdef\simpairIJ{0}  \xdef\simpairIO{0}  \xdef\simpairIU{0}  \xdef\simpairIX{0}  \xdef\simpairIZ{0}

\xdef\simpairLC{0} \xdef\simpairLS{0} \xdef\simpairLT{0} \xdef\simpairLP{0} \xdef\simpairLA{0}
\xdef\simpairLG{0} \xdef\simpairLN{0} \xdef\simpairLD{0} \xdef\simpairLE{0} \xdef\simpairLQ{0}
\xdef\simpairLH{0} \xdef\simpairLR{0} \xdef\simpairLK{0} \xdef\simpairLM{0} \xdef\simpairLI{0}
\xdef\simpairLL{1} \xdef\simpairLV{0} \xdef\simpairLF{0} \xdef\simpairLY{0} \xdef\simpairLW{0}
\xdef\simpairLB{0}  \xdef\simpairLJ{0}  \xdef\simpairLO{0}  \xdef\simpairLU{0}  \xdef\simpairLX{0}  \xdef\simpairLZ{0}

\xdef\simpairVC{0} \xdef\simpairVS{0} \xdef\simpairVT{0} \xdef\simpairVP{0} \xdef\simpairVA{0}
\xdef\simpairVG{0} \xdef\simpairVN{0} \xdef\simpairVD{0} \xdef\simpairVE{0} \xdef\simpairVQ{0}
\xdef\simpairVH{0} \xdef\simpairVR{0} \xdef\simpairVK{0} \xdef\simpairVM{0} \xdef\simpairVI{0}
\xdef\simpairVL{0} \xdef\simpairVV{1} \xdef\simpairVF{0} \xdef\simpairVY{0} \xdef\simpairVW{0}
\xdef\simpairVB{0}  \xdef\simpairVJ{0}  \xdef\simpairVO{0}  \xdef\simpairVU{0}  \xdef\simpairVX{0}  \xdef\simpairVZ{0}

\xdef\simpairFC{0} \xdef\simpairFS{0} \xdef\simpairFT{0} \xdef\simpairFP{0} \xdef\simpairFA{0}
\xdef\simpairFG{0} \xdef\simpairFN{0} \xdef\simpairFD{0} \xdef\simpairFE{0} \xdef\simpairFQ{0}
\xdef\simpairFH{0} \xdef\simpairFR{0} \xdef\simpairFK{0} \xdef\simpairFM{0} \xdef\simpairFI{0}
\xdef\simpairFL{0} \xdef\simpairFV{0} \xdef\simpairFF{1} \xdef\simpairFY{0} \xdef\simpairFW{0}
\xdef\simpairFB{0}  \xdef\simpairFJ{0}  \xdef\simpairFO{0}  \xdef\simpairFU{0}  \xdef\simpairFX{0}  \xdef\simpairFZ{0}

\xdef\simpairYC{0} \xdef\simpairYS{0} \xdef\simpairYT{0} \xdef\simpairYP{0} \xdef\simpairYA{0}
\xdef\simpairYG{0} \xdef\simpairYN{0} \xdef\simpairYD{0} \xdef\simpairYE{0} \xdef\simpairYQ{0}
\xdef\simpairYH{0} \xdef\simpairYR{0} \xdef\simpairYK{0} \xdef\simpairYM{0} \xdef\simpairYI{0}
\xdef\simpairYL{0} \xdef\simpairYV{0} \xdef\simpairYF{0} \xdef\simpairYY{1} \xdef\simpairYW{0}
\xdef\simpairYB{0}  \xdef\simpairYJ{0}  \xdef\simpairYO{0}  \xdef\simpairYU{0}  \xdef\simpairYX{0}  \xdef\simpairYZ{0}

\xdef\simpairWC{0} \xdef\simpairWS{0} \xdef\simpairWT{0} \xdef\simpairWP{0} \xdef\simpairWA{0}
\xdef\simpairWG{0} \xdef\simpairWN{0} \xdef\simpairWD{0} \xdef\simpairWE{0} \xdef\simpairWQ{0}
\xdef\simpairWH{0} \xdef\simpairWR{0} \xdef\simpairWK{0} \xdef\simpairWM{0} \xdef\simpairWI{0}
\xdef\simpairWL{0} \xdef\simpairWV{0} \xdef\simpairWF{0} \xdef\simpairWY{0} \xdef\simpairWW{1}
\xdef\simpairWB{0}  \xdef\simpairWJ{0}  \xdef\simpairWO{0}  \xdef\simpairWU{0}  \xdef\simpairWX{0}  \xdef\simpairWZ{0}

\xdef\simpairBC{0} \xdef\simpairBS{0} \xdef\simpairBT{0} \xdef\simpairBP{0} \xdef\simpairBA{0}
\xdef\simpairBG{0} \xdef\simpairBN{0} \xdef\simpairBD{0} \xdef\simpairBE{0} \xdef\simpairBQ{0}
\xdef\simpairBH{0} \xdef\simpairBR{0} \xdef\simpairBK{0} \xdef\simpairBM{0} \xdef\simpairBI{0}
\xdef\simpairBL{0} \xdef\simpairBV{0} \xdef\simpairBF{0} \xdef\simpairBY{0} \xdef\simpairBW{0}
\xdef\simpairBB{1} \xdef\simpairBJ{0} \xdef\simpairBO{0} \xdef\simpairBU{0}  \xdef\simpairBX{0}  \xdef\simpairBZ{0}

\xdef\simpairJC{0} \xdef\simpairJS{0} \xdef\simpairJT{0} \xdef\simpairJP{0} \xdef\simpairJA{0}
\xdef\simpairJG{0} \xdef\simpairJN{0} \xdef\simpairJD{0} \xdef\simpairJE{0} \xdef\simpairJQ{0}
\xdef\simpairJH{0} \xdef\simpairJR{0} \xdef\simpairJK{0} \xdef\simpairJM{0} \xdef\simpairJI{0}
\xdef\simpairJL{0} \xdef\simpairJV{0} \xdef\simpairJF{0} \xdef\simpairJY{0} \xdef\simpairJW{0}
\xdef\simpairJB{0} \xdef\simpairJJ{1} \xdef\simpairJO{0} \xdef\simpairJU{0}  \xdef\simpairJX{0}  \xdef\simpairJZ{0}

\xdef\simpairOC{0} \xdef\simpairOS{0} \xdef\simpairOT{0} \xdef\simpairOP{0} \xdef\simpairOA{0}
\xdef\simpairOG{0} \xdef\simpairON{0} \xdef\simpairOD{0} \xdef\simpairOE{0} \xdef\simpairOQ{0}
\xdef\simpairOH{0} \xdef\simpairOR{0} \xdef\simpairOK{0} \xdef\simpairOM{0} \xdef\simpairOI{0}
\xdef\simpairOL{0} \xdef\simpairOV{0} \xdef\simpairOF{0} \xdef\simpairOY{0} \xdef\simpairOW{0}
\xdef\simpairOB{0} \xdef\simpairOJ{0} \xdef\simpairOO{1} \xdef\simpairOU{0}  \xdef\simpairOX{0}  \xdef\simpairOZ{0}

\xdef\simpairUC{0} \xdef\simpairUS{0} \xdef\simpairUT{0} \xdef\simpairUP{0} \xdef\simpairUA{0}
\xdef\simpairUG{0} \xdef\simpairUN{0} \xdef\simpairUD{0} \xdef\simpairUE{0} \xdef\simpairUQ{0}
\xdef\simpairUH{0} \xdef\simpairUR{0} \xdef\simpairUK{0} \xdef\simpairUM{0} \xdef\simpairUI{0}
\xdef\simpairUL{0} \xdef\simpairUV{0} \xdef\simpairUF{0} \xdef\simpairUY{0} \xdef\simpairUW{0}
\xdef\simpairUB{0} \xdef\simpairUJ{0} \xdef\simpairUO{0} \xdef\simpairUU{1}  \xdef\simpairUX{0}  \xdef\simpairUZ{0}

\xdef\simpairXC{0} \xdef\simpairXS{0} \xdef\simpairXT{0} \xdef\simpairXP{0} \xdef\simpairXA{0}
\xdef\simpairXG{0} \xdef\simpairXN{0} \xdef\simpairXD{0} \xdef\simpairXE{0} \xdef\simpairXQ{0}
\xdef\simpairXH{0} \xdef\simpairXR{0} \xdef\simpairXK{0} \xdef\simpairXM{0} \xdef\simpairXI{0}
\xdef\simpairXL{0} \xdef\simpairXV{0} \xdef\simpairXF{0} \xdef\simpairXY{0} \xdef\simpairXW{0}
\xdef\simpairXB{0} \xdef\simpairXJ{0} \xdef\simpairXO{0} \xdef\simpairXU{0}  \xdef\simpairXX{1}  \xdef\simpairXZ{0}

\xdef\simpairZC{0} \xdef\simpairZS{0} \xdef\simpairZT{0} \xdef\simpairZP{0} \xdef\simpairZA{0}
\xdef\simpairZG{0} \xdef\simpairZN{0} \xdef\simpairZD{0} \xdef\simpairZE{0} \xdef\simpairZQ{0}
\xdef\simpairZH{0} \xdef\simpairZR{0} \xdef\simpairZK{0} \xdef\simpairZM{0} \xdef\simpairZI{0}
\xdef\simpairZL{0} \xdef\simpairZV{0} \xdef\simpairZF{0} \xdef\simpairZY{0} \xdef\simpairZW{0}
\xdef\simpairZB{0} \xdef\simpairZJ{0} \xdef\simpairZO{0} \xdef\simpairZU{0}  \xdef\simpairZX{0}  \xdef\simpairZZ{1}

}

\def\clear@sim@count{
   \def\clear@sim@nums{%
      \advance\innerloopcount by 1
      \ifnum\innerloopcount>\seq@count
      \else
         \expandafter\xdef\csname poscount\the\outerloopcount @\the\innerloopcount\endcsname{0}
         \expandafter\xdef\csname identcount\the\outerloopcount @\the\innerloopcount\endcsname{0}
         \expandafter\xdef\csname simcount\the\outerloopcount @\the\innerloopcount\endcsname{0}
         \clear@sim@nums
      \fi
   }

   \outerloopcount=1
   \loop
      \innerloopcount=\outerloopcount
      \clear@sim@nums
      \advance\outerloopcount by 1
   \ifnum\outerloopcount<\seq@count \repeat
}

\xdef\pepmwA{711}  \xdef\pepmwB{1146} \xdef\pepmwC{1032}
\xdef\pepmwD{1151} \xdef\pepmwE{1291} \xdef\pepmwF{1472}
\xdef\pepmwG{571}  \xdef\pepmwH{1372} \xdef\pepmwI{1132}
\xdef\pepmwJ{0}    \xdef\pepmwK{1282} \xdef\pepmwL{1132}
\xdef\pepmwM{1312} \xdef\pepmwN{1141} \xdef\pepmwO{0}
\xdef\pepmwP{971}  \xdef\pepmwQ{1281} \xdef\pepmwR{1562}
\xdef\pepmwS{871}  \xdef\pepmwT{1011} \xdef\pepmwU{0}
\xdef\pepmwV{991}  \xdef\pepmwW{1862} \xdef\pepmwX{1282}
\xdef\pepmwY{1632} \xdef\pepmwZ{1286}

\xdef\DNAmwA{3462}  \xdef\DNAmwB{0}     \xdef\DNAmwC{3222}
\xdef\DNAmwD{0}     \xdef\DNAmwE{0}     \xdef\DNAmwF{0}
\xdef\DNAmwG{3622}  \xdef\DNAmwH{0}     \xdef\DNAmwI{0}
\xdef\DNAmwJ{0}     \xdef\DNAmwK{0}     \xdef\DNAmwL{0}
\xdef\DNAmwM{0}     \xdef\DNAmwN{0}     \xdef\DNAmwO{0}
\xdef\DNAmwP{0}     \xdef\DNAmwQ{0}     \xdef\DNAmwR{0}
\xdef\DNAmwS{0}     \xdef\DNAmwT{3372}  \xdef\DNAmwU{3232}
\xdef\DNAmwV{0}     \xdef\DNAmwW{0}     \xdef\DNAmwX{0}
\xdef\DNAmwY{0}     \xdef\DNAmwZ{0}

\xdef\pepchargeA{0}     \xdef\pepchargeB{0}     \xdef\pepchargeC{-30}
\xdef\pepchargeD{-1000} \xdef\pepchargeE{-1000} \xdef\pepchargeF{0}
\xdef\pepchargeG{0}     \xdef\pepchargeH{165}   \xdef\pepchargeI{0}
\xdef\pepchargeJ{0}     \xdef\pepchargeK{1000}  \xdef\pepchargeL{0}
\xdef\pepchargeM{0}     \xdef\pepchargeN{0}     \xdef\pepchargeO{0}
\xdef\pepchargeP{0}     \xdef\pepchargeQ{0}     \xdef\pepchargeR{1000}
\xdef\pepchargeS{0}     \xdef\pepchargeT{0}     \xdef\pepchargeU{0}
\xdef\pepchargeV{0}     \xdef\pepchargeW{0}     \xdef\pepchargeX{0}
\xdef\pepchargeY{0}     \xdef\pepchargeZ{0}
\xdef\chargeNterm{910}  \xdef\chargeCterm{-1000}

\xdef\chargeA{0}     \xdef\chargeB{0}     \xdef\chargeC{0}
\xdef\chargeD{-50}   \xdef\chargeE{-50}   \xdef\chargeF{0}
\xdef\chargeG{0}     \xdef\chargeH{30}    \xdef\chargeI{0}
\xdef\chargeJ{0}     \xdef\chargeK{50}    \xdef\chargeL{0}
\xdef\chargeM{0}     \xdef\chargeN{0}     \xdef\chargeO{0}
\xdef\chargeP{0}     \xdef\chargeQ{0}     \xdef\chargeR{50}
\xdef\chargeS{0}     \xdef\chargeT{0}     \xdef\chargeU{0}
\xdef\chargeV{0}     \xdef\chargeW{0}     \xdef\chargeX{0}
\xdef\chargeY{0}     \xdef\chargeZ{0}

\xdef\molwA{11}   \xdef\molwB{45}   \xdef\molwC{36}
\xdef\molwD{45}   \xdef\molwE{66}   \xdef\molwF{70}
\xdef\molwG{1}    \xdef\molwH{62}   \xdef\molwI{44}
\xdef\molwJ{N}    \xdef\molwK{55}   \xdef\molwL{44}
\xdef\molwM{58}   \xdef\molwN{44}   \xdef\molwO{N}
\xdef\molwP{31}   \xdef\molwQ{55}   \xdef\molwR{77}
\xdef\molwS{19}   \xdef\molwT{34}   \xdef\molwU{N}
\xdef\molwV{33}   \xdef\molwW{100}  \xdef\molwX{55}
\xdef\molwY{82}   \xdef\molwZ{66}

\xdef\HydroA{21}  \xdef\HydroB{N}   \xdef\HydroC{10}
\xdef\HydroD{-31} \xdef\HydroE{-25} \xdef\HydroF{41}
\xdef\HydroG{16}  \xdef\HydroH{-14} \xdef\HydroI{47}
\xdef\HydroJ{N}   \xdef\HydroK{-52} \xdef\HydroL{36}
\xdef\HydroM{22}  \xdef\HydroN{-27} \xdef\HydroO{N}
\xdef\HydroP{4}   \xdef\HydroQ{-29} \xdef\HydroR{-53}
\xdef\HydroS{-6}  \xdef\HydroT{-2}  \xdef\HydroU{N}
\xdef\HydroV{37}  \xdef\HydroW{28}  \xdef\HydroX{N}
\xdef\HydroY{9}   \xdef\HydroZ{N}



\xdef\consCB{0}  \xdef\consCJ{0}  \xdef\consCO{0}  \xdef\consCU{0}  \xdef\consCX{0}  \xdef\consCZ{0}
\expandafter\xdef\csname consC?\endcsname{0}

\xdef\consSB{0}  \xdef\consSJ{0}  \xdef\consSO{0}  \xdef\consSU{0}  \xdef\consSX{0}  \xdef\consSZ{0}
\expandafter\xdef\csname consS?\endcsname{0}

\xdef\consTB{0}  \xdef\consTJ{0}  \xdef\consTO{0}  \xdef\consTU{0}  \xdef\consTX{0}  \xdef\consTZ{0}
\expandafter\xdef\csname consT?\endcsname{0}

\xdef\consPB{0}  \xdef\consPJ{0}  \xdef\consPO{0}  \xdef\consPU{0}  \xdef\consPX{0}  \xdef\consPZ{0}
\expandafter\xdef\csname consP?\endcsname{0}

\xdef\consAB{0}  \xdef\consAJ{0}  \xdef\consAO{0}  \xdef\consAU{0}  \xdef\consAX{0}  \xdef\consAZ{0}
\expandafter\xdef\csname consA?\endcsname{0}

\xdef\consGB{0}  \xdef\consGJ{0}  \xdef\consGO{0}  \xdef\consGU{0}  \xdef\consGX{0}  \xdef\consGZ{0}
\expandafter\xdef\csname consG?\endcsname{0}

\xdef\consNB{0}  \xdef\consNJ{0}  \xdef\consNO{0}  \xdef\consNU{0}  \xdef\consNX{0}  \xdef\consNZ{0}
\expandafter\xdef\csname consN?\endcsname{0}

\xdef\consDB{0}  \xdef\consDJ{0}  \xdef\consDO{0}  \xdef\consDU{0}  \xdef\consDX{0}  \xdef\consDZ{0}
\expandafter\xdef\csname consD?\endcsname{0}

\xdef\consEB{0}  \xdef\consEJ{0}  \xdef\consEO{0}  \xdef\consEU{0}  \xdef\consEX{0}  \xdef\consEZ{0}
\expandafter\xdef\csname consE?\endcsname{0}

\xdef\consQB{0}  \xdef\consQJ{0}  \xdef\consQO{0}  \xdef\consQU{0}  \xdef\consQX{0}  \xdef\consQZ{0}
\expandafter\xdef\csname consQ?\endcsname{0}

\xdef\consHB{0}  \xdef\consHJ{0}  \xdef\consHO{0}  \xdef\consHU{0}  \xdef\consHX{0}  \xdef\consHZ{0}
\expandafter\xdef\csname consH?\endcsname{0}

\xdef\consRB{0}  \xdef\consRJ{0}  \xdef\consRO{0}  \xdef\consRU{0}  \xdef\consRX{0}  \xdef\consRZ{0}
\expandafter\xdef\csname consR?\endcsname{0}

\xdef\consKB{0}  \xdef\consKJ{0}  \xdef\consKO{0}  \xdef\consKU{0}  \xdef\consKX{0}  \xdef\consKZ{0}
\expandafter\xdef\csname consK?\endcsname{0}

\xdef\consMB{0}  \xdef\consMJ{0}  \xdef\consMO{0}  \xdef\consMU{0}  \xdef\consMX{0}  \xdef\consMZ{0}
\expandafter\xdef\csname consM?\endcsname{0}

\xdef\consIB{0}  \xdef\consIJ{0}  \xdef\consIO{0}  \xdef\consIU{0}  \xdef\consIX{0}  \xdef\consIZ{0}
\expandafter\xdef\csname consI?\endcsname{0}

\xdef\consLB{0}  \xdef\consLJ{0}  \xdef\consLO{0}  \xdef\consLU{0}  \xdef\consLX{0}  \xdef\consLZ{0}
\expandafter\xdef\csname consL?\endcsname{0}

\xdef\consVB{0}  \xdef\consVJ{0}  \xdef\consVO{0}  \xdef\consVU{0}  \xdef\consVX{0}  \xdef\consVZ{0}
\expandafter\xdef\csname consV?\endcsname{0}

\xdef\consFB{0}  \xdef\consFJ{0}  \xdef\consFO{0}  \xdef\consFU{0}  \xdef\consFX{0}  \xdef\consFZ{0}
\expandafter\xdef\csname consF?\endcsname{0}

\xdef\consYB{0}  \xdef\consYJ{0}  \xdef\consYO{0}  \xdef\consYU{0}  \xdef\consYX{0}  \xdef\consYZ{0}
\expandafter\xdef\csname consY?\endcsname{0}

\xdef\consWB{0}  \xdef\consWJ{0}  \xdef\consWO{0}  \xdef\consWU{0}  \xdef\consWX{0}  \xdef\consWZ{0}
\expandafter\xdef\csname consW?\endcsname{0}

\xdef\consBC{0} \xdef\consBS{0} \xdef\consBT{0} \xdef\consBP{0} \xdef\consBA{0}
\xdef\consBG{0} \xdef\consBN{0} \xdef\consBD{0} \xdef\consBE{0} \xdef\consBQ{0}
\xdef\consBH{0} \xdef\consBR{0} \xdef\consBK{0} \xdef\consBM{0} \xdef\consBI{0}
\xdef\consBL{0} \xdef\consBV{0} \xdef\consBF{0} \xdef\consBY{0} \xdef\consBW{0}
\xdef\consBB{10} \xdef\consBJ{0} \xdef\consBO{0} \xdef\consBU{0}  \xdef\consBX{0}  \xdef\consBZ{0}
\expandafter\xdef\csname consB.\endcsname{0}
\expandafter\xdef\csname consB?\endcsname{0}

\xdef\consJC{0} \xdef\consJS{0} \xdef\consJT{0} \xdef\consJP{0} \xdef\consJA{0}
\xdef\consJG{0} \xdef\consJN{0} \xdef\consJD{0} \xdef\consJE{0} \xdef\consJQ{0}
\xdef\consJH{0} \xdef\consJR{0} \xdef\consJK{0} \xdef\consJM{0} \xdef\consJI{0}
\xdef\consJL{0} \xdef\consJV{0} \xdef\consJF{0} \xdef\consJY{0} \xdef\consJW{0}
\xdef\consJB{0} \xdef\consJJ{10} \xdef\consJO{0} \xdef\consJU{0}  \xdef\consJX{0}  \xdef\consJZ{0}
\expandafter\xdef\csname consJ.\endcsname{0}
\expandafter\xdef\csname consJ?\endcsname{0}

\xdef\consOC{0} \xdef\consOS{0} \xdef\consOT{0} \xdef\consOP{0} \xdef\consOA{0}
\xdef\consOG{0} \xdef\consON{0} \xdef\consOD{0} \xdef\consOE{0} \xdef\consOQ{0}
\xdef\consOH{0} \xdef\consOR{0} \xdef\consOK{0} \xdef\consOM{0} \xdef\consOI{0}
\xdef\consOL{0} \xdef\consOV{0} \xdef\consOF{0} \xdef\consOY{0} \xdef\consOW{0}
\xdef\consOB{0} \xdef\consOJ{0} \xdef\consOO{10} \xdef\consOU{0}  \xdef\consOX{0}  \xdef\consOZ{0}
\expandafter\xdef\csname consO.\endcsname{0}
\expandafter\xdef\csname consO?\endcsname{0}

\xdef\consUC{0} \xdef\consUS{0} \xdef\consUT{0} \xdef\consUP{0} \xdef\consUA{0}
\xdef\consUG{0} \xdef\consUN{0} \xdef\consUD{0} \xdef\consUE{0} \xdef\consUQ{0}
\xdef\consUH{0} \xdef\consUR{0} \xdef\consUK{0} \xdef\consUM{0} \xdef\consUI{0}
\xdef\consUL{0} \xdef\consUV{0} \xdef\consUF{0} \xdef\consUY{0} \xdef\consUW{0}
\xdef\consUB{0} \xdef\consUJ{0} \xdef\consUO{0} \xdef\consUU{10}  \xdef\consUX{0}  \xdef\consUZ{0}
\expandafter\xdef\csname consU.\endcsname{0}
\expandafter\xdef\csname consU?\endcsname{0}

\xdef\consXC{0} \xdef\consXS{0} \xdef\consXT{0} \xdef\consXP{0} \xdef\consXA{0}
\xdef\consXG{0} \xdef\consXN{0} \xdef\consXD{0} \xdef\consXE{0} \xdef\consXQ{0}
\xdef\consXH{0} \xdef\consXR{0} \xdef\consXK{0} \xdef\consXM{0} \xdef\consXI{0}
\xdef\consXL{0} \xdef\consXV{0} \xdef\consXF{0} \xdef\consXY{0} \xdef\consXW{0}
\xdef\consXB{0} \xdef\consXJ{0} \xdef\consXO{0} \xdef\consXU{0}  \xdef\consXX{10}  \xdef\consXZ{0}
\expandafter\xdef\csname consX.\endcsname{0}
\expandafter\xdef\csname consX?\endcsname{0}

\xdef\consZC{0} \xdef\consZS{0} \xdef\consZT{0} \xdef\consZP{0} \xdef\consZA{0}
\xdef\consZG{0} \xdef\consZN{0} \xdef\consZD{0} \xdef\consZE{0} \xdef\consZQ{0}
\xdef\consZH{0} \xdef\consZR{0} \xdef\consZK{0} \xdef\consZM{0} \xdef\consZI{0}
\xdef\consZL{0} \xdef\consZV{0} \xdef\consZF{0} \xdef\consZY{0} \xdef\consZW{0}
\xdef\consZB{0} \xdef\consZJ{0} \xdef\consZO{0} \xdef\consZU{0}  \xdef\consZX{0}  \xdef\consZZ{10}
\expandafter\xdef\csname consZ.\endcsname{0}
\expandafter\xdef\csname consZ?\endcsname{0}

\expandafter\xdef\csname cons..\endcsname{0}
\expandafter\xdef\csname cons.?\endcsname{0}

\expandafter\xdef\csname cons?C\endcsname{0}
\expandafter\xdef\csname cons?S\endcsname{0}
\expandafter\xdef\csname cons?T\endcsname{0}
\expandafter\xdef\csname cons?P\endcsname{0}
\expandafter\xdef\csname cons?A\endcsname{0}
\expandafter\xdef\csname cons?G\endcsname{0}
\expandafter\xdef\csname cons?N\endcsname{0}
\expandafter\xdef\csname cons?D\endcsname{0}
\expandafter\xdef\csname cons?E\endcsname{0}
\expandafter\xdef\csname cons?Q\endcsname{0}
\expandafter\xdef\csname cons?H\endcsname{0}
\expandafter\xdef\csname cons?R\endcsname{0}
\expandafter\xdef\csname cons?K\endcsname{0}
\expandafter\xdef\csname cons?M\endcsname{0}
\expandafter\xdef\csname cons?I\endcsname{0}
\expandafter\xdef\csname cons?L\endcsname{0}
\expandafter\xdef\csname cons?V\endcsname{0}
\expandafter\xdef\csname cons?F\endcsname{0}
\expandafter\xdef\csname cons?Y\endcsname{0}
\expandafter\xdef\csname cons?W\endcsname{0}
\expandafter\xdef\csname cons?B\endcsname{0}
\expandafter\xdef\csname cons?J\endcsname{0}
\expandafter\xdef\csname cons?O\endcsname{0}
\expandafter\xdef\csname cons?U\endcsname{0}
\expandafter\xdef\csname cons?X\endcsname{0}
\expandafter\xdef\csname cons?Z\endcsname{0}
\expandafter\xdef\csname cons?.\endcsname{0}
\expandafter\xdef\csname cons??\endcsname{0}

\expandafter\xdef\csname cons==\endcsname{0}


\def\gappenalty#1{%
\expandafter\xdef\csname cons.C\endcsname{#1} \expandafter\xdef\csname consC.\endcsname{#1}
\expandafter\xdef\csname cons.S\endcsname{#1} \expandafter\xdef\csname consS.\endcsname{#1}
\expandafter\xdef\csname cons.T\endcsname{#1} \expandafter\xdef\csname consT.\endcsname{#1}
\expandafter\xdef\csname cons.P\endcsname{#1} \expandafter\xdef\csname consP.\endcsname{#1}
\expandafter\xdef\csname cons.A\endcsname{#1} \expandafter\xdef\csname consA.\endcsname{#1}
\expandafter\xdef\csname cons.G\endcsname{#1} \expandafter\xdef\csname consG.\endcsname{#1}
\expandafter\xdef\csname cons.N\endcsname{#1} \expandafter\xdef\csname consN.\endcsname{#1}
\expandafter\xdef\csname cons.D\endcsname{#1} \expandafter\xdef\csname consD.\endcsname{#1}
\expandafter\xdef\csname cons.E\endcsname{#1} \expandafter\xdef\csname consE.\endcsname{#1}
\expandafter\xdef\csname cons.Q\endcsname{#1} \expandafter\xdef\csname consQ.\endcsname{#1}
\expandafter\xdef\csname cons.H\endcsname{#1} \expandafter\xdef\csname consH.\endcsname{#1}
\expandafter\xdef\csname cons.R\endcsname{#1} \expandafter\xdef\csname consR.\endcsname{#1}
\expandafter\xdef\csname cons.K\endcsname{#1} \expandafter\xdef\csname consK.\endcsname{#1}
\expandafter\xdef\csname cons.M\endcsname{#1} \expandafter\xdef\csname consM.\endcsname{#1}
\expandafter\xdef\csname cons.I\endcsname{#1} \expandafter\xdef\csname consI.\endcsname{#1}
\expandafter\xdef\csname cons.L\endcsname{#1} \expandafter\xdef\csname consL.\endcsname{#1}
\expandafter\xdef\csname cons.V\endcsname{#1} \expandafter\xdef\csname consV.\endcsname{#1}
\expandafter\xdef\csname cons.F\endcsname{#1} \expandafter\xdef\csname consF.\endcsname{#1}
\expandafter\xdef\csname cons.Y\endcsname{#1} \expandafter\xdef\csname consY.\endcsname{#1}
\expandafter\xdef\csname cons.W\endcsname{#1} \expandafter\xdef\csname consW.\endcsname{#1}
\expandafter\xdef\csname cons.B\endcsname{#1} \expandafter\xdef\csname consB.\endcsname{#1}
\expandafter\xdef\csname cons.J\endcsname{#1} \expandafter\xdef\csname consJ.\endcsname{#1}
\expandafter\xdef\csname cons.O\endcsname{#1} \expandafter\xdef\csname consO.\endcsname{#1}
\expandafter\xdef\csname cons.U\endcsname{#1} \expandafter\xdef\csname consU.\endcsname{#1}
\expandafter\xdef\csname cons.X\endcsname{#1} \expandafter\xdef\csname consX.\endcsname{#1}
\expandafter\xdef\csname cons.Z\endcsname{#1} \expandafter\xdef\csname consZ.\endcsname{#1}
}

\def\weighttable#1{%
\xdef\first@{#1}

\xdef\temp@{identity}
\ifx\first@\temp@


%%%% identity matrix

\xdef\consCC{10} \xdef\consCS{0} \xdef\consCT{0} \xdef\consCP{0} \xdef\consCA{0}
\xdef\consCG{0} \xdef\consCN{0} \xdef\consCD{0} \xdef\consCE{0} \xdef\consCQ{0}
\xdef\consCH{0} \xdef\consCR{0} \xdef\consCK{0} \xdef\consCM{0} \xdef\consCI{0}
\xdef\consCL{0} \xdef\consCV{0} \xdef\consCF{0} \xdef\consCY{0} \xdef\consCW{0}

\xdef\consSC{0} \xdef\consSS{10} \xdef\consST{0} \xdef\consSP{0} \xdef\consSA{0}
\xdef\consSG{0} \xdef\consSN{0} \xdef\consSD{0} \xdef\consSE{0} \xdef\consSQ{0}
\xdef\consSH{0} \xdef\consSR{0} \xdef\consSK{0} \xdef\consSM{0} \xdef\consSI{0}
\xdef\consSL{0} \xdef\consSV{0} \xdef\consSF{0} \xdef\consSY{0} \xdef\consSW{0}

\xdef\consTC{0} \xdef\consTS{0} \xdef\consTT{10} \xdef\consTP{0} \xdef\consTA{0}
\xdef\consTG{0} \xdef\consTN{0} \xdef\consTD{0} \xdef\consTE{0} \xdef\consTQ{0}
\xdef\consTH{0} \xdef\consTR{0} \xdef\consTK{0} \xdef\consTM{0} \xdef\consTI{0}
\xdef\consTL{0} \xdef\consTV{0} \xdef\consTF{0} \xdef\consTY{0} \xdef\consTW{0}

\xdef\consPC{0} \xdef\consPS{0} \xdef\consPT{0} \xdef\consPP{10} \xdef\consPA{0}
\xdef\consPG{0} \xdef\consPN{0} \xdef\consPD{0} \xdef\consPE{0} \xdef\consPQ{0}
\xdef\consPH{0} \xdef\consPR{0} \xdef\consPK{0} \xdef\consPM{0} \xdef\consPI{0}
\xdef\consPL{0} \xdef\consPV{0} \xdef\consPF{0} \xdef\consPY{0} \xdef\consPW{0}

\xdef\consAC{0} \xdef\consAS{0} \xdef\consAT{0} \xdef\consAP{0} \xdef\consAA{10}
\xdef\consAG{0} \xdef\consAN{0} \xdef\consAD{0} \xdef\consAE{0} \xdef\consAQ{0}
\xdef\consAH{0} \xdef\consAR{0} \xdef\consAK{0} \xdef\consAM{0} \xdef\consAI{0}
\xdef\consAL{0} \xdef\consAV{0} \xdef\consAF{0} \xdef\consAY{0} \xdef\consAW{0} 

\xdef\consGC{0} \xdef\consGS{0} \xdef\consGT{0} \xdef\consGP{0} \xdef\consGA{0}
\xdef\consGG{10} \xdef\consGN{0} \xdef\consGD{0} \xdef\consGE{0} \xdef\consGQ{0}
\xdef\consGH{0} \xdef\consGR{0} \xdef\consGK{0} \xdef\consGM{0} \xdef\consGI{0}
\xdef\consGL{0} \xdef\consGV{0} \xdef\consGF{0} \xdef\consGY{0} \xdef\consGW{0}

\xdef\consNC{0} \xdef\consNS{0} \xdef\consNT{0} \xdef\consNP{0} \xdef\consNA{0}
\xdef\consNG{0} \xdef\consNN{10} \xdef\consND{0} \xdef\consNE{0} \xdef\consNQ{0}
\xdef\consNH{0} \xdef\consNR{0} \xdef\consNK{0} \xdef\consNM{0} \xdef\consNI{0}
\xdef\consNL{0} \xdef\consNV{0} \xdef\consNF{0} \xdef\consNY{0} \xdef\consNW{0}

\xdef\consDC{0} \xdef\consDS{0} \xdef\consDT{0} \xdef\consDP{0} \xdef\consDA{0}
\xdef\consDG{0} \xdef\consDN{0} \xdef\consDD{10} \xdef\consDE{0} \xdef\consDQ{0}
\xdef\consDH{0} \xdef\consDR{0} \xdef\consDK{0} \xdef\consDM{0} \xdef\consDI{0}
\xdef\consDL{0} \xdef\consDV{0} \xdef\consDF{0} \xdef\consDY{0} \xdef\consDW{0}

\xdef\consEC{0} \xdef\consES{0} \xdef\consET{0} \xdef\consEP{0} \xdef\consEA{0}
\xdef\consEG{0} \xdef\consEN{0} \xdef\consED{0} \xdef\consEE{10} \xdef\consEQ{0}
\xdef\consEH{0} \xdef\consER{0} \xdef\consEK{0} \xdef\consEM{0} \xdef\consEI{0}
\xdef\consEL{0} \xdef\consEV{0} \xdef\consEF{0} \xdef\consEY{0} \xdef\consEW{0}

\xdef\consQC{0} \xdef\consQS{0} \xdef\consQT{0} \xdef\consQP{0} \xdef\consQA{0}
\xdef\consQG{0} \xdef\consQN{0} \xdef\consQD{0} \xdef\consQE{0} \xdef\consQQ{10}
\xdef\consQH{0} \xdef\consQR{0} \xdef\consQK{0} \xdef\consQM{0} \xdef\consQI{0}
\xdef\consQL{0} \xdef\consQV{0} \xdef\consQF{0} \xdef\consQY{0} \xdef\consQW{0}

\xdef\consHC{0} \xdef\consHS{0} \xdef\consHT{0} \xdef\consHP{0} \xdef\consHA{0}
\xdef\consHG{0} \xdef\consHN{0} \xdef\consHD{0} \xdef\consHE{0} \xdef\consHQ{0}
\xdef\consHH{10} \xdef\consHR{0} \xdef\consHK{0} \xdef\consHM{0} \xdef\consHI{0}
\xdef\consHL{0} \xdef\consHV{0} \xdef\consHF{0} \xdef\consHY{0} \xdef\consHW{0}

\xdef\consRC{0} \xdef\consRS{0} \xdef\consRT{0} \xdef\consRP{0} \xdef\consRA{0}
\xdef\consRG{0} \xdef\consRN{0} \xdef\consRD{0} \xdef\consRE{0} \xdef\consRQ{0}
\xdef\consRH{0} \xdef\consRR{10} \xdef\consRK{0} \xdef\consRM{0} \xdef\consRI{0}
\xdef\consRL{0} \xdef\consRV{0} \xdef\consRF{0} \xdef\consRY{0} \xdef\consRW{0}

\xdef\consKC{0} \xdef\consKS{0} \xdef\consKT{0} \xdef\consKP{0} \xdef\consKA{0}
\xdef\consKG{0} \xdef\consKN{0} \xdef\consKD{0} \xdef\consKE{0} \xdef\consKQ{0}
\xdef\consKH{0} \xdef\consKR{0} \xdef\consKK{10} \xdef\consKM{0} \xdef\consKI{0}
\xdef\consKL{0} \xdef\consKV{0} \xdef\consKF{0} \xdef\consKY{0} \xdef\consKW{0}

\xdef\consMC{0} \xdef\consMS{0} \xdef\consMT{0} \xdef\consMP{0} \xdef\consMA{0}
\xdef\consMG{0} \xdef\consMN{0} \xdef\consMD{0} \xdef\consME{0} \xdef\consMQ{0}
\xdef\consMH{0} \xdef\consMR{0} \xdef\consMK{0} \xdef\consMM{10} \xdef\consMI{0}
\xdef\consML{0} \xdef\consMV{0} \xdef\consMF{0} \xdef\consMY{0} \xdef\consMW{0}

\xdef\consIC{0} \xdef\consIS{0} \xdef\consIT{0} \xdef\consIP{0} \xdef\consIA{0}
\xdef\consIG{0} \xdef\consIN{0} \xdef\consID{0} \xdef\consIE{0} \xdef\consIQ{0}
\xdef\consIH{0} \xdef\consIR{0} \xdef\consIK{0} \xdef\consIM{0} \xdef\consII{10}
\xdef\consIL{0} \xdef\consIV{0} \xdef\consIF{0} \xdef\consIY{0} \xdef\consIW{0}

\xdef\consLC{0} \xdef\consLS{0} \xdef\consLT{0} \xdef\consLP{0} \xdef\consLA{0}
\xdef\consLG{0} \xdef\consLN{0} \xdef\consLD{0} \xdef\consLE{0} \xdef\consLQ{0}
\xdef\consLH{0} \xdef\consLR{0} \xdef\consLK{0} \xdef\consLM{0} \xdef\consLI{0}
\xdef\consLL{10} \xdef\consLV{0} \xdef\consLF{0} \xdef\consLY{0} \xdef\consLW{0}

\xdef\consVC{0} \xdef\consVS{0} \xdef\consVT{0} \xdef\consVP{0} \xdef\consVA{0}
\xdef\consVG{0} \xdef\consVN{0} \xdef\consVD{0} \xdef\consVE{0} \xdef\consVQ{0}
\xdef\consVH{0} \xdef\consVR{0} \xdef\consVK{0} \xdef\consVM{0} \xdef\consVI{0}
\xdef\consVL{0} \xdef\consVV{10} \xdef\consVF{0} \xdef\consVY{0} \xdef\consVW{0}

\xdef\consFC{0} \xdef\consFS{0} \xdef\consFT{0} \xdef\consFP{0} \xdef\consFA{0}
\xdef\consFG{0} \xdef\consFN{0} \xdef\consFD{0} \xdef\consFE{0} \xdef\consFQ{0}
\xdef\consFH{0} \xdef\consFR{0} \xdef\consFK{0} \xdef\consFM{0} \xdef\consFI{0}
\xdef\consFL{0} \xdef\consFV{0} \xdef\consFF{10} \xdef\consFY{0} \xdef\consFW{0}

\xdef\consYC{0} \xdef\consYS{0} \xdef\consYT{0} \xdef\consYP{0} \xdef\consYA{0}
\xdef\consYG{0} \xdef\consYN{0} \xdef\consYD{0} \xdef\consYE{0} \xdef\consYQ{0}
\xdef\consYH{0} \xdef\consYR{0} \xdef\consYK{0} \xdef\consYM{0} \xdef\consYI{0}
\xdef\consYL{0} \xdef\consYV{0} \xdef\consYF{0} \xdef\consYY{10} \xdef\consYW{0}

\xdef\consWC{0} \xdef\consWS{0} \xdef\consWT{0} \xdef\consWP{0} \xdef\consWA{0}
\xdef\consWG{0} \xdef\consWN{0} \xdef\consWD{0} \xdef\consWE{0} \xdef\consWQ{0}
\xdef\consWH{0} \xdef\consWR{0} \xdef\consWK{0} \xdef\consWM{0} \xdef\consWI{0}
\xdef\consWL{0} \xdef\consWV{0} \xdef\consWF{0} \xdef\consWY{0} \xdef\consWW{10}

\gappenalty{0} \xdef\m@trixf@ctor{10}

\else

\xdef\temp@{structural}
\ifx\first@\temp@


%%%% structural

\xdef\consCC{10} \xdef\consCS{6} \xdef\consCT{3} \xdef\consCP{3} \xdef\consCA{3}
\xdef\consCG{5} \xdef\consCN{3} \xdef\consCD{1} \xdef\consCE{0} \xdef\consCQ{1}
\xdef\consCH{3} \xdef\consCR{3} \xdef\consCK{0} \xdef\consCM{3} \xdef\consCI{3}
\xdef\consCL{3} \xdef\consCV{3} \xdef\consCF{5} \xdef\consCY{5} \xdef\consCW{5}

\xdef\consSC{6} \xdef\consSS{10} \xdef\consST{8} \xdef\consSP{6} \xdef\consSA{8}
\xdef\consSG{8} \xdef\consSN{8} \xdef\consSD{6} \xdef\consSE{5} \xdef\consSQ{5}
\xdef\consSH{5} \xdef\consSR{5} \xdef\consSK{5} \xdef\consSM{3} \xdef\consSI{3}
\xdef\consSL{3} \xdef\consSV{6} \xdef\consSF{5} \xdef\consSY{5} \xdef\consSW{3}

\xdef\consTC{3} \xdef\consTS{8} \xdef\consTT{10} \xdef\consTP{6} \xdef\consTA{8}
\xdef\consTG{6} \xdef\consTN{6} \xdef\consTD{5} \xdef\consTE{5} \xdef\consTQ{5}
\xdef\consTH{3} \xdef\consTR{5} \xdef\consTK{6} \xdef\consTM{5} \xdef\consTI{5}
\xdef\consTL{3} \xdef\consTV{6} \xdef\consTF{3} \xdef\consTY{3} \xdef\consTW{1}

\xdef\consPC{3} \xdef\consPS{6} \xdef\consPT{6} \xdef\consPP{10} \xdef\consPA{8}
\xdef\consPG{6} \xdef\consPN{3} \xdef\consPD{5} \xdef\consPE{5} \xdef\consPQ{5}
\xdef\consPH{5} \xdef\consPR{5} \xdef\consPK{3} \xdef\consPM{3} \xdef\consPI{3}
\xdef\consPL{5} \xdef\consPV{6} \xdef\consPF{5} \xdef\consPY{3} \xdef\consPW{3}

\xdef\consAC{3} \xdef\consAS{8} \xdef\consAT{8} \xdef\consAP{8} \xdef\consAA{10}
\xdef\consAG{8} \xdef\consAN{5} \xdef\consAD{6} \xdef\consAE{6} \xdef\consAQ{5}
\xdef\consAH{3} \xdef\consAR{3} \xdef\consAK{5} \xdef\consAM{5} \xdef\consAI{3}
\xdef\consAL{3} \xdef\consAV{8} \xdef\consAF{5} \xdef\consAY{3} \xdef\consAW{3} 

\xdef\consGC{5} \xdef\consGS{8} \xdef\consGT{6} \xdef\consGP{6} \xdef\consGA{8}
\xdef\consGG{10} \xdef\consGN{5} \xdef\consGD{6} \xdef\consGE{6} \xdef\consGQ{3}
\xdef\consGH{1} \xdef\consGR{5} \xdef\consGK{3} \xdef\consGM{1} \xdef\consGI{3}
\xdef\consGL{3} \xdef\consGV{6} \xdef\consGF{3} \xdef\consGY{3} \xdef\consGW{5}

\xdef\consNC{3} \xdef\consNS{8} \xdef\consNT{6} \xdef\consNP{3} \xdef\consNA{5}
\xdef\consNG{5} \xdef\consNN{10} \xdef\consND{8} \xdef\consNE{6} \xdef\consNQ{5}
\xdef\consNH{6} \xdef\consNR{5} \xdef\consNK{6} \xdef\consNM{1} \xdef\consNI{3}
\xdef\consNL{1} \xdef\consNV{3} \xdef\consNF{3} \xdef\consNY{5} \xdef\consNW{0}

\xdef\consDC{1} \xdef\consDS{6} \xdef\consDT{5} \xdef\consDP{5} \xdef\consDA{6}
\xdef\consDG{6} \xdef\consDN{8} \xdef\consDD{10} \xdef\consDE{8} \xdef\consDQ{6}
\xdef\consDH{5} \xdef\consDR{3} \xdef\consDK{5} \xdef\consDM{3} \xdef\consDI{1}
\xdef\consDL{1} \xdef\consDV{5} \xdef\consDF{1} \xdef\consDY{3} \xdef\consDW{0}

\xdef\consEC{0} \xdef\consES{5} \xdef\consET{5} \xdef\consEP{5} \xdef\consEA{6}
\xdef\consEG{6} \xdef\consEN{6} \xdef\consED{8} \xdef\consEE{10} \xdef\consEQ{6}
\xdef\consEH{3} \xdef\consER{5} \xdef\consEK{6} \xdef\consEM{3} \xdef\consEI{1}
\xdef\consEL{1} \xdef\consEV{6} \xdef\consEF{3} \xdef\consEY{1} \xdef\consEW{1}

\xdef\consQC{1} \xdef\consQS{5} \xdef\consQT{5} \xdef\consQP{5} \xdef\consQA{5}
\xdef\consQG{3} \xdef\consQN{5} \xdef\consQD{6} \xdef\consQE{6} \xdef\consQQ{10}
\xdef\consQH{6} \xdef\consQR{5} \xdef\consQK{6} \xdef\consQM{3} \xdef\consQI{1}
\xdef\consQL{3} \xdef\consQV{3} \xdef\consQF{1} \xdef\consQY{3} \xdef\consQW{1}

\xdef\consHC{3} \xdef\consHS{5} \xdef\consHT{3} \xdef\consHP{5} \xdef\consHA{3}
\xdef\consHG{1} \xdef\consHN{6} \xdef\consHD{5} \xdef\consHE{3} \xdef\consHQ{6}
\xdef\consHH{10} \xdef\consHR{6} \xdef\consHK{5} \xdef\consHM{3} \xdef\consHI{3}
\xdef\consHL{5} \xdef\consHV{1} \xdef\consHF{3} \xdef\consHY{5} \xdef\consHW{1}

\xdef\consRC{3} \xdef\consRS{5} \xdef\consRT{5} \xdef\consRP{5} \xdef\consRA{3}
\xdef\consRG{5} \xdef\consRN{5} \xdef\consRD{3} \xdef\consRE{5} \xdef\consRQ{5}
\xdef\consRH{6} \xdef\consRR{10} \xdef\consRK{8} \xdef\consRM{3} \xdef\consRI{3}
\xdef\consRL{3} \xdef\consRV{3} \xdef\consRF{1} \xdef\consRY{1} \xdef\consRW{3}

\xdef\consKC{0} \xdef\consKS{5} \xdef\consKT{6} \xdef\consKP{3} \xdef\consKA{5}
\xdef\consKG{3} \xdef\consKN{6} \xdef\consKD{5} \xdef\consKE{6} \xdef\consKQ{6}
\xdef\consKH{5} \xdef\consKR{8} \xdef\consKK{10} \xdef\consKM{3} \xdef\consKI{3}
\xdef\consKL{3} \xdef\consKV{5} \xdef\consKF{1} \xdef\consKY{1} \xdef\consKW{1}

\xdef\consMC{3} \xdef\consMS{5} \xdef\consMT{5} \xdef\consMP{3} \xdef\consMA{5}
\xdef\consMG{1} \xdef\consMN{1} \xdef\consMD{3} \xdef\consME{3} \xdef\consMQ{3}
\xdef\consMH{3} \xdef\consMR{3} \xdef\consMK{3} \xdef\consMM{10} \xdef\consMI{6}
\xdef\consML{8} \xdef\consMV{6} \xdef\consMF{5} \xdef\consMY{3} \xdef\consMW{5}

\xdef\consIC{3} \xdef\consIS{3} \xdef\consIT{5} \xdef\consIP{3} \xdef\consIA{3}
\xdef\consIG{3} \xdef\consIN{3} \xdef\consID{1} \xdef\consIE{1} \xdef\consIQ{1}
\xdef\consIH{3} \xdef\consIR{3} \xdef\consIK{3} \xdef\consIM{6} \xdef\consII{10}
\xdef\consIL{8} \xdef\consIV{3} \xdef\consIF{6} \xdef\consIY{5} \xdef\consIW{5}

\xdef\consLC{3} \xdef\consLS{3} \xdef\consLT{3} \xdef\consLP{5} \xdef\consLA{3}
\xdef\consLG{3} \xdef\consLN{1} \xdef\consLD{1} \xdef\consLE{1} \xdef\consLQ{3}
\xdef\consLH{5} \xdef\consLR{3} \xdef\consLK{3} \xdef\consLM{8} \xdef\consLI{8}
\xdef\consLL{10} \xdef\consLV{3} \xdef\consLF{6} \xdef\consLY{5} \xdef\consLW{6}

\xdef\consVC{3} \xdef\consVS{6} \xdef\consVT{6} \xdef\consVP{6} \xdef\consVA{8}
\xdef\consVG{6} \xdef\consVN{3} \xdef\consVD{5} \xdef\consVE{6} \xdef\consVQ{3}
\xdef\consVH{1} \xdef\consVR{3} \xdef\consVK{5} \xdef\consVM{6} \xdef\consVI{3}
\xdef\consVL{3} \xdef\consVV{10} \xdef\consVF{6} \xdef\consVY{5} \xdef\consVW{5}

\xdef\consFC{5} \xdef\consFS{5} \xdef\consFT{3} \xdef\consFP{5} \xdef\consFA{5}
\xdef\consFG{3} \xdef\consFN{3} \xdef\consFD{1} \xdef\consFE{3} \xdef\consFQ{1}
\xdef\consFH{3} \xdef\consFR{1} \xdef\consFK{1} \xdef\consFM{5} \xdef\consFI{6}
\xdef\consFL{6} \xdef\consFV{6} \xdef\consFF{10} \xdef\consFY{8} \xdef\consFW{5}

\xdef\consYC{5} \xdef\consYS{5} \xdef\consYT{3} \xdef\consYP{3} \xdef\consYA{3}
\xdef\consYG{3} \xdef\consYN{5} \xdef\consYD{3} \xdef\consYE{1} \xdef\consYQ{3}
\xdef\consYH{5} \xdef\consYR{1} \xdef\consYK{1} \xdef\consYM{3} \xdef\consYI{5}
\xdef\consYL{5} \xdef\consYV{5} \xdef\consYF{8} \xdef\consYY{10} \xdef\consYW{5}

\xdef\consWC{5} \xdef\consWS{3} \xdef\consWT{1} \xdef\consWP{3} \xdef\consWA{3}
\xdef\consWG{5} \xdef\consWN{0} \xdef\consWD{0} \xdef\consWE{1} \xdef\consWQ{1}
\xdef\consWH{1} \xdef\consWR{3} \xdef\consWK{1} \xdef\consWM{5} \xdef\consWI{5}
\xdef\consWL{6} \xdef\consWV{5} \xdef\consWF{5} \xdef\consWY{5} \xdef\consWW{10}

\gappenalty{0} \xdef\m@trixf@ctor{10}

\else

\xdef\temp@{PAM250}
\ifx\first@\temp@

%%%% PAM250

\xdef\consCC{4} \xdef\consCS{0} \xdef\consCT{-2} \xdef\consCP{-3} \xdef\consCA{-2}
\xdef\consCG{-3} \xdef\consCN{-4} \xdef\consCD{2} \xdef\consCE{-5} \xdef\consCQ{-5}
\xdef\consCH{-3} \xdef\consCR{-4} \xdef\consCK{-5} \xdef\consCM{-5} \xdef\consCI{-2}
\xdef\consCL{-6} \xdef\consCV{-2} \xdef\consCF{-4} \xdef\consCY{0} \xdef\consCW{-8}
\xdef\consCB{-4} \xdef\consCZ{-5} \xdef\consCX{-3}

\xdef\consSC{0} \xdef\consSS{3} \xdef\consST{1} \xdef\consSP{1} \xdef\consSA{1}
\xdef\consSG{1} \xdef\consSN{1} \xdef\consSD{0} \xdef\consSE{0} \xdef\consSQ{-1}
\xdef\consSH{-1} \xdef\consSR{0} \xdef\consSK{0} \xdef\consSM{-2} \xdef\consSI{-1}
\xdef\consSL{-3} \xdef\consSV{-1} \xdef\consSF{-3} \xdef\consSY{-3} \xdef\consSW{-2}
\xdef\consSB{0} \xdef\consSZ{0} \xdef\consSX{0}

\xdef\consTC{-2} \xdef\consTS{1} \xdef\consTT{3} \xdef\consTP{0} \xdef\consTA{1}
\xdef\consTG{0} \xdef\consTN{0} \xdef\consTD{0} \xdef\consTE{0} \xdef\consTQ{-1}
\xdef\consTH{-1} \xdef\consTR{-1} \xdef\consTK{0} \xdef\consTM{-1} \xdef\consTI{0}
\xdef\consTL{-2} \xdef\consTV{0} \xdef\consTF{-2} \xdef\consTY{-3} \xdef\consTW{-5}
\xdef\consTB{0} \xdef\consTZ{-1} \xdef\consTX{0}

\xdef\consPC{-3} \xdef\consPS{1} \xdef\consPT{0} \xdef\consPP{6} \xdef\consPA{1}
\xdef\consPG{-1} \xdef\consPN{-1} \xdef\consPD{-1} \xdef\consPE{-1} \xdef\consPQ{0}
\xdef\consPH{0} \xdef\consPR{0} \xdef\consPK{-1} \xdef\consPM{-2} \xdef\consPI{-2}
\xdef\consPL{-3} \xdef\consPV{-1} \xdef\consPF{-5} \xdef\consPY{-5} \xdef\consPW{-6}
\xdef\consPB{-1} \xdef\consPZ{0} \xdef\consPX{-1}

\xdef\consAC{-2} \xdef\consAS{1} \xdef\consAT{1} \xdef\consAP{1} \xdef\consAA{2}
\xdef\consAG{1} \xdef\consAN{0} \xdef\consAD{0} \xdef\consAE{0} \xdef\consAQ{0}
\xdef\consAH{-1} \xdef\consAR{-2} \xdef\consAK{-1} \xdef\consAM{-1} \xdef\consAI{-1}
\xdef\consAL{-2} \xdef\consAV{0} \xdef\consAF{-4} \xdef\consAY{-3} \xdef\consAW{-6} 
\xdef\consAB{0} \xdef\consAZ{0} \xdef\consAX{0}

\xdef\consGC{-3} \xdef\consGS{1} \xdef\consGT{0} \xdef\consGP{-1} \xdef\consGA{1}
\xdef\consGG{5} \xdef\consGN{0} \xdef\consGD{1} \xdef\consGE{0} \xdef\consGQ{-1}
\xdef\consGH{-2} \xdef\consGR{-3} \xdef\consGK{-2} \xdef\consGM{-3} \xdef\consGI{-3}
\xdef\consGL{-4} \xdef\consGV{-1} \xdef\consGF{-5} \xdef\consGY{-5} \xdef\consGW{-7}
\xdef\consGB{0} \xdef\consGZ{0} \xdef\consGX{-1}

\xdef\consNC{-4} \xdef\consNS{1} \xdef\consNT{0} \xdef\consNP{-1} \xdef\consNA{0}
\xdef\consNG{0} \xdef\consNN{2} \xdef\consND{2} \xdef\consNE{1} \xdef\consNQ{1}
\xdef\consNH{2} \xdef\consNR{0} \xdef\consNK{1} \xdef\consNM{-2} \xdef\consNI{-2}
\xdef\consNL{-3} \xdef\consNV{-2} \xdef\consNF{-4} \xdef\consNY{-2} \xdef\consNW{-4}
\xdef\consNB{2} \xdef\consNZ{1} \xdef\consNX{0}

\xdef\consDC{-5} \xdef\consDS{0} \xdef\consDT{0} \xdef\consDP{-1} \xdef\consDA{0}
\xdef\consDG{1} \xdef\consDN{2} \xdef\consDD{4} \xdef\consDE{3} \xdef\consDQ{2}
\xdef\consDH{1} \xdef\consDR{-1} \xdef\consDK{0} \xdef\consDM{-3} \xdef\consDI{-2}
\xdef\consDL{-4} \xdef\consDV{-2} \xdef\consDF{-6} \xdef\consDY{-4} \xdef\consDW{-7}
\xdef\consDB{3} \xdef\consDZ{3} \xdef\consDX{-1}

\xdef\consEC{-5} \xdef\consES{0} \xdef\consET{0} \xdef\consEP{-1} \xdef\consEA{0}
\xdef\consEG{0} \xdef\consEN{1} \xdef\consED{3} \xdef\consEE{4} \xdef\consEQ{2}
\xdef\consEH{1} \xdef\consER{-1} \xdef\consEK{0} \xdef\consEM{-2} \xdef\consEI{-2}
\xdef\consEL{-3} \xdef\consEV{-2} \xdef\consEF{-5} \xdef\consEY{-4} \xdef\consEW{-7}
\xdef\consEB{3} \xdef\consEZ{3} \xdef\consEX{-1}

\xdef\consQC{-5} \xdef\consQS{-1} \xdef\consQT{-1} \xdef\consQP{0} \xdef\consQA{0}
\xdef\consQG{-1} \xdef\consQN{1} \xdef\consQD{2} \xdef\consQE{2} \xdef\consQQ{4}
\xdef\consQH{3} \xdef\consQR{1} \xdef\consQK{1} \xdef\consQM{-1} \xdef\consQI{-2}
\xdef\consQL{-2} \xdef\consQV{-2} \xdef\consQF{-5} \xdef\consQY{-4} \xdef\consQW{-5}
\xdef\consQB{1} \xdef\consQZ{3} \xdef\consQX{-1}

\xdef\consHC{-3} \xdef\consHS{-1} \xdef\consHT{-1} \xdef\consHP{0} \xdef\consHA{-1}
\xdef\consHG{-2} \xdef\consHN{2} \xdef\consHD{1} \xdef\consHE{1} \xdef\consHQ{3}
\xdef\consHH{6} \xdef\consHR{2} \xdef\consHK{0} \xdef\consHM{-2} \xdef\consHI{-2}
\xdef\consHL{-2} \xdef\consHV{-2} \xdef\consHF{-2} \xdef\consHY{-5} \xdef\consHW{-7}
\xdef\consHB{1} \xdef\consHZ{2} \xdef\consHX{-1}

\xdef\consRC{-4} \xdef\consRS{0} \xdef\consRT{-1} \xdef\consRP{0} \xdef\consRA{-2}
\xdef\consRG{-3} \xdef\consRN{0} \xdef\consRD{-1} \xdef\consRE{-1} \xdef\consRQ{1}
\xdef\consRH{2} \xdef\consRR{6} \xdef\consRK{3} \xdef\consRM{0} \xdef\consRI{-2}
\xdef\consRL{-3} \xdef\consRV{-2} \xdef\consRF{-4} \xdef\consRY{-4} \xdef\consRW{2}
\xdef\consRB{-1} \xdef\consRZ{0} \xdef\consRX{-1}

\xdef\consKC{-5} \xdef\consKS{0} \xdef\consKT{0} \xdef\consKP{-1} \xdef\consKA{-1}
\xdef\consKG{-2} \xdef\consKN{1} \xdef\consKD{0} \xdef\consKE{0} \xdef\consKQ{1}
\xdef\consKH{0} \xdef\consKR{3} \xdef\consKK{5} \xdef\consKM{0} \xdef\consKI{-2}
\xdef\consKL{-3} \xdef\consKV{-2} \xdef\consKF{-5} \xdef\consKY{-4} \xdef\consKW{-3}
\xdef\consKB{1} \xdef\consKZ{-2} \xdef\consKX{-1}

\xdef\consMC{-5} \xdef\consMS{-2} \xdef\consMT{-1} \xdef\consMP{-2} \xdef\consMA{-1}
\xdef\consMG{-3} \xdef\consMN{-2} \xdef\consMD{-3} \xdef\consME{-2} \xdef\consMQ{-1}
\xdef\consMH{-2} \xdef\consMR{0} \xdef\consMK{0} \xdef\consMM{6} \xdef\consMI{2}
\xdef\consML{4} \xdef\consMV{2} \xdef\consMF{0} \xdef\consMY{-2} \xdef\consMW{-4}
\xdef\consMB{-2} \xdef\consMZ{-2} \xdef\consMX{-1}

\xdef\consIC{-2} \xdef\consIS{-1} \xdef\consIT{0} \xdef\consIP{-2} \xdef\consIA{-1}
\xdef\consIG{-3} \xdef\consIN{-2} \xdef\consID{-2} \xdef\consIE{-2} \xdef\consIQ{-2}
\xdef\consIH{-2} \xdef\consIR{-2} \xdef\consIK{-2} \xdef\consIM{2} \xdef\consII{5}
\xdef\consIL{2} \xdef\consIV{4} \xdef\consIF{1} \xdef\consIY{-1} \xdef\consIW{-5}
\xdef\consIB{-2} \xdef\consIZ{-2} \xdef\consIX{-1}

\xdef\consLC{-6} \xdef\consLS{-3} \xdef\consLT{-2} \xdef\consLP{-3} \xdef\consLA{-2}
\xdef\consLG{-4} \xdef\consLN{-3} \xdef\consLD{-4} \xdef\consLE{-3} \xdef\consLQ{-2}
\xdef\consLH{-2} \xdef\consLR{-3} \xdef\consLK{-3} \xdef\consLM{4} \xdef\consLI{2}
\xdef\consLL{6} \xdef\consLV{2} \xdef\consLF{2} \xdef\consLY{-1} \xdef\consLW{-2}
\xdef\consLB{-3} \xdef\consLZ{-3} \xdef\consLX{-1}

\xdef\consVC{-2} \xdef\consVS{-1} \xdef\consVT{0} \xdef\consVP{-1} \xdef\consVA{0}
\xdef\consVG{-1} \xdef\consVN{-2} \xdef\consVD{-2} \xdef\consVE{-2} \xdef\consVQ{-2}
\xdef\consVH{-2} \xdef\consVR{-2} \xdef\consVK{-2} \xdef\consVM{2} \xdef\consVI{4}
\xdef\consVL{2} \xdef\consVV{4} \xdef\consVF{-1} \xdef\consVY{-2} \xdef\consVW{-6}
\xdef\consVB{-2} \xdef\consVZ{-2} \xdef\consVX{-1}

\xdef\consFC{-4} \xdef\consFS{-3} \xdef\consFT{-2} \xdef\consFP{-5} \xdef\consFA{-4}
\xdef\consFG{-5} \xdef\consFN{-4} \xdef\consFD{-6} \xdef\consFE{-5} \xdef\consFQ{-5}
\xdef\consFH{-2} \xdef\consFR{-4} \xdef\consFK{-5} \xdef\consFM{0} \xdef\consFI{1}
\xdef\consFL{2} \xdef\consFV{-1} \xdef\consFF{9} \xdef\consFY{7} \xdef\consFW{0}
\xdef\consFB{-4} \xdef\consFZ{-5} \xdef\consFX{-2}

\xdef\consYC{0} \xdef\consYS{-3} \xdef\consYT{-3} \xdef\consYP{-5} \xdef\consYA{-3}
\xdef\consYG{-5} \xdef\consYN{-2} \xdef\consYD{-4} \xdef\consYE{-4} \xdef\consYQ{-4}
\xdef\consYH{0} \xdef\consYR{-4} \xdef\consYK{-4} \xdef\consYM{-2} \xdef\consYI{-1}
\xdef\consYL{-1} \xdef\consYV{-2} \xdef\consYF{7} \xdef\consYY{10} \xdef\consYW{0}
\xdef\consYB{-3} \xdef\consYZ{-4} \xdef\consYX{-2}

\xdef\consWC{-8} \xdef\consWS{-2} \xdef\consWT{-5} \xdef\consWP{-6} \xdef\consWA{-6}
\xdef\consWG{-7} \xdef\consWN{-4} \xdef\consWD{-7} \xdef\consWE{-7} \xdef\consWQ{-5}
\xdef\consWH{-3} \xdef\consWR{2} \xdef\consWK{-3} \xdef\consWM{-4} \xdef\consWI{-5}
\xdef\consWL{-2} \xdef\consWV{-6} \xdef\consWF{0} \xdef\consWY{0} \xdef\consWW{17}
\xdef\consWB{-5} \xdef\consWZ{-6} \xdef\consWX{-4}

\xdef\consBC{-4} \xdef\consBS{0} \xdef\consBT{0} \xdef\consBP{-1} \xdef\consBA{0}
\xdef\consBG{0} \xdef\consBN{2} \xdef\consBD{3} \xdef\consBE{3} \xdef\consBQ{1}
\xdef\consBH{1} \xdef\consBR{-1} \xdef\consBK{1} \xdef\consBM{-2} \xdef\consBI{-2}
\xdef\consBL{-3} \xdef\consBV{-2} \xdef\consBF{-4} \xdef\consBY{-3} \xdef\consBW{-5}
\xdef\consBB{3} \xdef\consBZ{2}  \xdef\consBX{-1}

\xdef\consZC{-5} \xdef\consZS{0} \xdef\consZT{-1} \xdef\consZP{0} \xdef\consZA{0}
\xdef\consZG{0} \xdef\consZN{1} \xdef\consZD{3} \xdef\consZE{3} \xdef\consZQ{3}
\xdef\consZH{2} \xdef\consZR{0} \xdef\consZK{0} \xdef\consZM{-2} \xdef\consZI{-2}
\xdef\consZL{-3} \xdef\consZV{-2} \xdef\consZF{-5} \xdef\consZY{-4} \xdef\consZW{-6}
\xdef\consZB{2} \xdef\consZZ{3}  \xdef\consZX{-1}

\xdef\consXC{-3} \xdef\consXS{0} \xdef\consXT{0} \xdef\consXP{-1} \xdef\consXA{0}
\xdef\consXG{-1} \xdef\consXN{0} \xdef\consXD{-1} \xdef\consXE{-1} \xdef\consXQ{-1}
\xdef\consXH{-1} \xdef\consXR{-1} \xdef\consXK{-1} \xdef\consXM{-1} \xdef\consXI{-1}
\xdef\consXL{-1} \xdef\consXV{-1} \xdef\consXF{-2} \xdef\consXY{-2} \xdef\consXW{-4}
\xdef\consXB{-1} \xdef\consXZ{-1}  \xdef\consXX{-1}

\expandafter\xdef\csname cons..\endcsname{1}
\gappenalty{-8} \xdef\m@trixf@ctor{10}

\else
\xdef\temp@{PAM100} \xdef\m@trixf@ctor{10}
\ifx\first@\temp@

%%%% PAM100

\xdef\consCC{14} \xdef\consCS{-1} \xdef\consCT{-5} \xdef\consCP{-6} \xdef\consCA{-5}
\xdef\consCG{-8} \xdef\consCN{-8} \xdef\consCD{-11} \xdef\consCE{-11} \xdef\consCQ{-11}
\xdef\consCH{-6} \xdef\consCR{-6} \xdef\consCK{-11} \xdef\consCM{-11} \xdef\consCI{-5}
\xdef\consCL{-12} \xdef\consCV{-4} \xdef\consCF{-10} \xdef\consCY{-2} \xdef\consCW{-13}
\xdef\consCB{-6} \xdef\consCZ{-8}  \xdef\consCX{-5}

\xdef\consSC{-1} \xdef\consSS{6} \xdef\consST{2} \xdef\consSP{1} \xdef\consSA{2}
\xdef\consSG{1} \xdef\consSN{2} \xdef\consSD{-1} \xdef\consSE{-2} \xdef\consSQ{-3}
\xdef\consSH{-4} \xdef\consSR{-1} \xdef\consSK{-2} \xdef\consSM{-4} \xdef\consSI{-4}
\xdef\consSL{-7} \xdef\consSV{-4} \xdef\consSF{-5} \xdef\consSY{-6} \xdef\consSW{-4}
\xdef\consSB{0} \xdef\consSZ{-2}  \xdef\consSX{1}

\xdef\consTC{-5} \xdef\consTS{2} \xdef\consTT{7} \xdef\consTP{-1} \xdef\consTA{2}
\xdef\consTG{-3} \xdef\consTN{0} \xdef\consTD{-2} \xdef\consTE{-3} \xdef\consTQ{-3}
\xdef\consTH{-5} \xdef\consTR{-4} \xdef\consTK{-1} \xdef\consTM{-2} \xdef\consTI{-1}
\xdef\consTL{-5} \xdef\consTV{-1} \xdef\consTF{-6} \xdef\consTY{-6} \xdef\consTW{-10}
\xdef\consTB{-1} \xdef\consTZ{-2}  \xdef\consTX{-1}

\xdef\consPC{-6} \xdef\consPS{1} \xdef\consPT{-1} \xdef\consPP{10} \xdef\consPA{1}
\xdef\consPG{-3} \xdef\consPN{-3} \xdef\consPD{-4} \xdef\consPE{-3} \xdef\consPQ{-1}
\xdef\consPH{-2} \xdef\consPR{-2} \xdef\consPK{-4} \xdef\consPM{-6} \xdef\consPI{-6}
\xdef\consPL{-5} \xdef\consPV{-4} \xdef\consPF{-9} \xdef\consPY{-11} \xdef\consPW{-11}
\xdef\consPB{-3} \xdef\consPZ{-1}  \xdef\consPX{-2}

\xdef\consAC{-5} \xdef\consAS{2} \xdef\consAT{2} \xdef\consAP{1} \xdef\consAA{6}
\xdef\consAG{1} \xdef\consAN{-1} \xdef\consAD{-1} \xdef\consAE{0} \xdef\consAQ{-2}
\xdef\consAH{-5} \xdef\consAR{-5} \xdef\consAK{-4} \xdef\consAM{-3} \xdef\consAI{-3}
\xdef\consAL{-5} \xdef\consAV{0} \xdef\consAF{-7} \xdef\consAY{-6} \xdef\consAW{-11} 
\xdef\consAB{-1} \xdef\consAZ{-1}  \xdef\consAX{-1}

\xdef\consGC{-8} \xdef\consGS{1} \xdef\consGT{-3} \xdef\consGP{-3} \xdef\consGA{1}
\xdef\consGG{8} \xdef\consGN{-1} \xdef\consGD{-1} \xdef\consGE{-2} \xdef\consGQ{-5}
\xdef\consGH{-7} \xdef\consGR{-8} \xdef\consGK{-5} \xdef\consGM{-8} \xdef\consGI{-7}
\xdef\consGL{-8} \xdef\consGV{-4} \xdef\consGF{-8} \xdef\consGY{-11} \xdef\consGW{-13}
\xdef\consGB{-1} \xdef\consGZ{-2}  \xdef\consGX{-2}

\xdef\consNC{-8} \xdef\consNS{2} \xdef\consNT{0} \xdef\consNP{-3} \xdef\consNA{-1}
\xdef\consNG{-1} \xdef\consNN{7} \xdef\consND{4} \xdef\consNE{1} \xdef\consNQ{-1}
\xdef\consNH{2} \xdef\consNR{-3} \xdef\consNK{1} \xdef\consNM{-5} \xdef\consNI{-4}
\xdef\consNL{-6} \xdef\consNV{-5} \xdef\consNF{-6} \xdef\consNY{-3} \xdef\consNW{-8}
\xdef\consNB{4} \xdef\consNZ{0}  \xdef\consNX{-1}

\xdef\consDC{-11} \xdef\consDS{-1} \xdef\consDT{-2} \xdef\consDP{-4} \xdef\consDA{-1}
\xdef\consDG{-1} \xdef\consDN{4} \xdef\consDD{8} \xdef\consDE{5} \xdef\consDQ{1}
\xdef\consDH{-1} \xdef\consDR{-6} \xdef\consDK{-2} \xdef\consDM{-8} \xdef\consDI{-6}
\xdef\consDL{-9} \xdef\consDV{-6} \xdef\consDF{-11} \xdef\consDY{-9} \xdef\consDW{-13}
\xdef\consDB{4} \xdef\consDZ{3}  \xdef\consDX{-2}

\xdef\consEC{-11} \xdef\consES{-2} \xdef\consET{-3} \xdef\consEP{-3} \xdef\consEA{0}
\xdef\consEG{-2} \xdef\consEN{1} \xdef\consED{5} \xdef\consEE{8} \xdef\consEQ{4}
\xdef\consEH{-2} \xdef\consER{-5} \xdef\consEK{-2} \xdef\consEM{-6} \xdef\consEI{-5}
\xdef\consEL{-7} \xdef\consEV{-5} \xdef\consEF{-11} \xdef\consEY{-7} \xdef\consEW{-14}
\xdef\consEB{3} \xdef\consEZ{4}  \xdef\consEX{-2}

\xdef\consQC{-11} \xdef\consQS{-3} \xdef\consQT{-3} \xdef\consQP{-1} \xdef\consQA{-2}
\xdef\consQG{-5} \xdef\consQN{-1} \xdef\consQD{1} \xdef\consQE{4} \xdef\consQQ{9}
\xdef\consQH{4} \xdef\consQR{1} \xdef\consQK{-1} \xdef\consQM{-2} \xdef\consQI{-5}
\xdef\consQL{-3} \xdef\consQV{-5} \xdef\consQF{-10} \xdef\consQY{-9} \xdef\consQW{-11}
\xdef\consQB{0} \xdef\consQZ{5}  \xdef\consQX{-2}

\xdef\consHC{-6} \xdef\consHS{-4} \xdef\consHT{-5} \xdef\consHP{-2} \xdef\consHA{-5}
\xdef\consHG{-7} \xdef\consHN{2} \xdef\consHD{-1} \xdef\consHE{-2} \xdef\consHQ{4}
\xdef\consHH{11} \xdef\consHR{1} \xdef\consHK{-3} \xdef\consHM{-7} \xdef\consHI{-7}
\xdef\consHL{-5} \xdef\consHV{-6} \xdef\consHF{-4} \xdef\consHY{-1} \xdef\consHW{-7}
\xdef\consHB{1} \xdef\consHZ{1}  \xdef\consHX{-2}

\xdef\consRC{-6} \xdef\consRS{-1} \xdef\consRT{-4} \xdef\consRP{-2} \xdef\consRA{-5}
\xdef\consRG{-8} \xdef\consRN{-3} \xdef\consRD{-6} \xdef\consRE{-5} \xdef\consRQ{1}
\xdef\consRH{1} \xdef\consRR{10} \xdef\consRK{3} \xdef\consRM{-2} \xdef\consRI{-4}
\xdef\consRL{-7} \xdef\consRV{-6} \xdef\consRF{-7} \xdef\consRY{-10} \xdef\consRW{1}
\xdef\consRB{-3} \xdef\consRZ{-1}  \xdef\consRX{-2}

\xdef\consKC{-11} \xdef\consKS{-2} \xdef\consKT{-1} \xdef\consKP{-4} \xdef\consKA{-4}
\xdef\consKG{-5} \xdef\consKN{1} \xdef\consKD{-2} \xdef\consKE{-2} \xdef\consKQ{-1}
\xdef\consKH{-3} \xdef\consKR{3} \xdef\consKK{8} \xdef\consKM{1} \xdef\consKI{-4}
\xdef\consKL{-6} \xdef\consKV{-6} \xdef\consKF{-11} \xdef\consKY{-10} \xdef\consKW{-9}
\xdef\consKB{0} \xdef\consKZ{-1}  \xdef\consKX{-2}

\xdef\consMC{-11} \xdef\consMS{-4} \xdef\consMT{-2} \xdef\consMP{-6} \xdef\consMA{-3}
\xdef\consMG{-8} \xdef\consMN{-5} \xdef\consMD{-8} \xdef\consME{-6} \xdef\consMQ{-2}
\xdef\consMH{-7} \xdef\consMR{-2} \xdef\consMK{1} \xdef\consMM{13} \xdef\consMI{2}
\xdef\consML{4} \xdef\consMV{1} \xdef\consMF{-2} \xdef\consMY{-8} \xdef\consMW{-11}
\xdef\consMB{-4} \xdef\consMZ{-2}  \xdef\consMX{-2}

\xdef\consIC{-5} \xdef\consIS{-4} \xdef\consIT{-1} \xdef\consIP{-6} \xdef\consIA{-3}
\xdef\consIG{-7} \xdef\consIN{-4} \xdef\consID{-6} \xdef\consIE{-5} \xdef\consIQ{-5}
\xdef\consIH{-7} \xdef\consIR{-4} \xdef\consIK{-4} \xdef\consIM{2} \xdef\consII{9}
\xdef\consIL{2} \xdef\consIV{5} \xdef\consIF{0} \xdef\consIY{-4} \xdef\consIW{-12}
\xdef\consIB{-3} \xdef\consIZ{-3}  \xdef\consIX{-2}

\xdef\consLC{-12} \xdef\consLS{-7} \xdef\consLT{-5} \xdef\consLP{-5} \xdef\consLA{-5}
\xdef\consLG{-8} \xdef\consLN{-6} \xdef\consLD{-9} \xdef\consLE{-7} \xdef\consLQ{-3}
\xdef\consLH{-5} \xdef\consLR{-7} \xdef\consLK{-6} \xdef\consLM{4} \xdef\consLI{2}
\xdef\consLL{9} \xdef\consLV{1} \xdef\consLF{0} \xdef\consLY{-5} \xdef\consLW{-7}
\xdef\consLB{-5} \xdef\consLZ{-4}  \xdef\consLX{-3}

\xdef\consVC{-4} \xdef\consVS{-4} \xdef\consVT{-1} \xdef\consVP{-4} \xdef\consVA{0}
\xdef\consVG{-4} \xdef\consVN{-5} \xdef\consVD{-6} \xdef\consVE{-5} \xdef\consVQ{-5}
\xdef\consVH{-6} \xdef\consVR{-6} \xdef\consVK{-6} \xdef\consVM{1} \xdef\consVI{5}
\xdef\consVL{1} \xdef\consVV{8} \xdef\consVF{-5} \xdef\consVY{-6} \xdef\consVW{-14}
\xdef\consVB{-4} \xdef\consVZ{-3}  \xdef\consVX{-2}

\xdef\consFC{-10} \xdef\consFS{-5} \xdef\consFT{-6} \xdef\consFP{-9} \xdef\consFA{-7}
\xdef\consFG{-8} \xdef\consFN{-6} \xdef\consFD{-11} \xdef\consFE{-11} \xdef\consFQ{-10}
\xdef\consFH{-4} \xdef\consFR{-7} \xdef\consFK{-11} \xdef\consFM{-2} \xdef\consFI{0}
\xdef\consFL{0} \xdef\consFV{-5} \xdef\consFF{12} \xdef\consFY{6} \xdef\consFW{-2}
\xdef\consFB{-6} \xdef\consFZ{-7}  \xdef\consFX{-4}

\xdef\consYC{-2} \xdef\consYS{-6} \xdef\consYT{-6} \xdef\consYP{-11} \xdef\consYA{-6}
\xdef\consYG{-11} \xdef\consYN{-3} \xdef\consYD{-9} \xdef\consYE{-7} \xdef\consYQ{-9}
\xdef\consYH{-1} \xdef\consYR{-10} \xdef\consYK{-10} \xdef\consYM{-8} \xdef\consYI{-4}
\xdef\consYL{-5} \xdef\consYV{-6} \xdef\consYF{6} \xdef\consYY{13} \xdef\consYW{-2}
\xdef\consYB{-4} \xdef\consYZ{-6}  \xdef\consYX{-4}

\xdef\consWC{-13} \xdef\consWS{-4} \xdef\consWT{-10} \xdef\consWP{-11} \xdef\consWA{-11}
\xdef\consWG{-13} \xdef\consWN{-8} \xdef\consWD{-13} \xdef\consWE{-14} \xdef\consWQ{-11}
\xdef\consWH{-7} \xdef\consWR{1} \xdef\consWK{-9} \xdef\consWM{-11} \xdef\consWI{-12}
\xdef\consWL{-7} \xdef\consWV{-14} \xdef\consWF{-2} \xdef\consWY{-2} \xdef\consWW{19}
\xdef\consWB{-6} \xdef\consWZ{-8}  \xdef\consWX{-6}

\xdef\consBC{-6} \xdef\consBS{0} \xdef\consBT{-1} \xdef\consBP{-3} \xdef\consBA{-1}
\xdef\consBG{-1} \xdef\consBN{4} \xdef\consBD{4} \xdef\consBE{3} \xdef\consBQ{0}
\xdef\consBH{1} \xdef\consBR{-3} \xdef\consBK{0} \xdef\consBM{-4} \xdef\consBI{-3}
\xdef\consBL{-5} \xdef\consBV{-4} \xdef\consBF{-6} \xdef\consBY{-4} \xdef\consBW{-6}
\xdef\consBB{4} \xdef\consBZ{2}  \xdef\consBX{-2}

\xdef\consZC{-8} \xdef\consZS{-2} \xdef\consZT{-2} \xdef\consZP{-1} \xdef\consZA{-1}
\xdef\consZG{-2} \xdef\consZN{0} \xdef\consZD{3} \xdef\consZE{4} \xdef\consZQ{5}
\xdef\consZH{1} \xdef\consZR{-1} \xdef\consZK{-1} \xdef\consZM{-2} \xdef\consZI{-3}
\xdef\consZL{-4} \xdef\consZV{-3} \xdef\consZF{-7} \xdef\consZY{-6} \xdef\consZW{-8}
\xdef\consZB{2} \xdef\consZZ{5}  \xdef\consZX{-2}

\xdef\consXC{-5} \xdef\consXS{-1} \xdef\consXT{-1} \xdef\consXP{-2} \xdef\consXA{-1}
\xdef\consXG{-2} \xdef\consXN{-1} \xdef\consXD{-2} \xdef\consXE{-2} \xdef\consXQ{-2}
\xdef\consXH{-2} \xdef\consXR{-2} \xdef\consXK{-2} \xdef\consXM{-2} \xdef\consXI{-2}
\xdef\consXL{-3} \xdef\consXV{-2} \xdef\consXF{-4} \xdef\consXY{-4} \xdef\consXW{-6}
\xdef\consXB{-2} \xdef\consXZ{-2}  \xdef\consXX{-2}

\expandafter\xdef\csname cons..\endcsname{1}
\gappenalty{-9} \xdef\m@trixf@ctor{10}

\else
\xdef\temp@{BLOSUM62}
\ifx\first@\temp@

%%%% BLOSUM62

\xdef\consCC{9} \xdef\consCS{-1} \xdef\consCT{-1} \xdef\consCP{-3} \xdef\consCA{0}
\xdef\consCG{-3} \xdef\consCN{-3} \xdef\consCD{-3} \xdef\consCE{-4} \xdef\consCQ{-3}
\xdef\consCH{-3} \xdef\consCR{-3} \xdef\consCK{-3} \xdef\consCM{-1} \xdef\consCI{-1}
\xdef\consCL{-1} \xdef\consCV{-1} \xdef\consCF{-2} \xdef\consCY{-2} \xdef\consCW{-2}

\xdef\consSC{-1} \xdef\consSS{4} \xdef\consST{1} \xdef\consSP{-1} \xdef\consSA{1}
\xdef\consSG{0} \xdef\consSN{1} \xdef\consSD{0} \xdef\consSE{0} \xdef\consSQ{0}
\xdef\consSH{-1} \xdef\consSR{-1} \xdef\consSK{0} \xdef\consSM{-1} \xdef\consSI{-2}
\xdef\consSL{-2} \xdef\consSV{-2} \xdef\consSF{-2} \xdef\consSY{-2} \xdef\consSW{-3}

\xdef\consTC{-1} \xdef\consTS{1} \xdef\consTT{4} \xdef\consTP{1} \xdef\consTA{-1}
\xdef\consTG{1} \xdef\consTN{0} \xdef\consTD{1} \xdef\consTE{0} \xdef\consTQ{0}
\xdef\consTH{0} \xdef\consTR{-1} \xdef\consTK{0} \xdef\consTM{-1} \xdef\consTI{-2}
\xdef\consTL{-2} \xdef\consTV{-2} \xdef\consTF{-2} \xdef\consTY{-2} \xdef\consTW{-3}

\xdef\consPC{-3} \xdef\consPS{-1} \xdef\consPT{1} \xdef\consPP{7} \xdef\consPA{-1}
\xdef\consPG{-2} \xdef\consPN{-1} \xdef\consPD{-1} \xdef\consPE{-1} \xdef\consPQ{-1}
\xdef\consPH{-2} \xdef\consPR{-2} \xdef\consPK{-1} \xdef\consPM{-2} \xdef\consPI{-3}
\xdef\consPL{-3} \xdef\consPV{-2} \xdef\consPF{-4} \xdef\consPY{-3} \xdef\consPW{-4}

\xdef\consAC{0} \xdef\consAS{1} \xdef\consAT{-1} \xdef\consAP{-1} \xdef\consAA{4}
\xdef\consAG{0} \xdef\consAN{-1} \xdef\consAD{-2} \xdef\consAE{-1} \xdef\consAQ{-1}
\xdef\consAH{-2} \xdef\consAR{-1} \xdef\consAK{-1} \xdef\consAM{-1} \xdef\consAI{-1}
\xdef\consAL{-1} \xdef\consAV{-2} \xdef\consAF{-2} \xdef\consAY{-2} \xdef\consAW{-3} 

\xdef\consGC{-3} \xdef\consGS{0} \xdef\consGT{1} \xdef\consGP{-2} \xdef\consGA{0}
\xdef\consGG{6} \xdef\consGN{-2} \xdef\consGD{-1} \xdef\consGE{-2} \xdef\consGQ{-2}
\xdef\consGH{-2} \xdef\consGR{-2} \xdef\consGK{-2} \xdef\consGM{-3} \xdef\consGI{-4}
\xdef\consGL{-4} \xdef\consGV{0} \xdef\consGF{-3} \xdef\consGY{-3} \xdef\consGW{-2}

\xdef\consNC{-3} \xdef\consNS{1} \xdef\consNT{0} \xdef\consNP{-2} \xdef\consNA{-2}
\xdef\consNG{0} \xdef\consNN{6} \xdef\consND{1} \xdef\consNE{0} \xdef\consNQ{0}
\xdef\consNH{-1} \xdef\consNR{0} \xdef\consNK{0} \xdef\consNM{-2} \xdef\consNI{-3}
\xdef\consNL{-3} \xdef\consNV{-3} \xdef\consNF{-3} \xdef\consNY{-2} \xdef\consNW{-4}

\xdef\consDC{-3} \xdef\consDS{0} \xdef\consDT{1} \xdef\consDP{-1} \xdef\consDA{-2}
\xdef\consDG{-1} \xdef\consDN{1} \xdef\consDD{6} \xdef\consDE{2} \xdef\consDQ{0}
\xdef\consDH{-1} \xdef\consDR{-2} \xdef\consDK{-1} \xdef\consDM{-3} \xdef\consDI{-3}
\xdef\consDL{-4} \xdef\consDV{-3} \xdef\consDF{-3} \xdef\consDY{-3} \xdef\consDW{-4}

\xdef\consEC{-4} \xdef\consES{0} \xdef\consET{0} \xdef\consEP{-1} \xdef\consEA{-1}
\xdef\consEG{-2} \xdef\consEN{0} \xdef\consED{2} \xdef\consEE{5} \xdef\consEQ{2}
\xdef\consEH{0} \xdef\consER{0} \xdef\consEK{1} \xdef\consEM{-2} \xdef\consEI{-3}
\xdef\consEL{-3} \xdef\consEV{-3} \xdef\consEF{-3} \xdef\consEY{-2} \xdef\consEW{-3}

\xdef\consQC{-3} \xdef\consQS{0} \xdef\consQT{0} \xdef\consQP{-1} \xdef\consQA{-1}
\xdef\consQG{-2} \xdef\consQN{0} \xdef\consQD{0} \xdef\consQE{2} \xdef\consQQ{5}
\xdef\consQH{0} \xdef\consQR{1} \xdef\consQK{1} \xdef\consQM{0} \xdef\consQI{-3}
\xdef\consQL{-2} \xdef\consQV{-2} \xdef\consQF{-3} \xdef\consQY{-1} \xdef\consQW{-2}

\xdef\consHC{-3} \xdef\consHS{-1} \xdef\consHT{0} \xdef\consHP{-2} \xdef\consHA{-2}
\xdef\consHG{-2} \xdef\consHN{1} \xdef\consHD{1} \xdef\consHE{0} \xdef\consHQ{0}
\xdef\consHH{8} \xdef\consHR{0} \xdef\consHK{-1} \xdef\consHM{-2} \xdef\consHI{-3}
\xdef\consHL{-3} \xdef\consHV{-2} \xdef\consHF{-1} \xdef\consHY{2} \xdef\consHW{-2}

\xdef\consRC{-3} \xdef\consRS{-1} \xdef\consRT{-1} \xdef\consRP{-2} \xdef\consRA{-1}
\xdef\consRG{-2} \xdef\consRN{0} \xdef\consRD{-2} \xdef\consRE{0} \xdef\consRQ{1}
\xdef\consRH{0} \xdef\consRR{5} \xdef\consRK{2} \xdef\consRM{-1} \xdef\consRI{-3}
\xdef\consRL{-2} \xdef\consRV{-3} \xdef\consRF{-3} \xdef\consRY{-2} \xdef\consRW{-3}

\xdef\consKC{-3} \xdef\consKS{0} \xdef\consKT{0} \xdef\consKP{-1} \xdef\consKA{-1}
\xdef\consKG{-2} \xdef\consKN{0} \xdef\consKD{-1} \xdef\consKE{1} \xdef\consKQ{1}
\xdef\consKH{-1} \xdef\consKR{2} \xdef\consKK{5} \xdef\consKM{-1} \xdef\consKI{-3}
\xdef\consKL{-2} \xdef\consKV{-3} \xdef\consKF{-3} \xdef\consKY{-2} \xdef\consKW{-3}

\xdef\consMC{-1} \xdef\consMS{-1} \xdef\consMT{-1} \xdef\consMP{-2} \xdef\consMA{-1}
\xdef\consMG{-3} \xdef\consMN{-2} \xdef\consMD{-3} \xdef\consME{-2} \xdef\consMQ{0}
\xdef\consMH{-2} \xdef\consMR{-1} \xdef\consMK{-1} \xdef\consMM{5} \xdef\consMI{1}
\xdef\consML{2} \xdef\consMV{-2} \xdef\consMF{0} \xdef\consMY{-1} \xdef\consMW{-1}

\xdef\consIC{-1} \xdef\consIS{-2} \xdef\consIT{-2} \xdef\consIP{-3} \xdef\consIA{-1}
\xdef\consIG{-4} \xdef\consIN{-3} \xdef\consID{-3} \xdef\consIE{-3} \xdef\consIQ{-3}
\xdef\consIH{-3} \xdef\consIR{-3} \xdef\consIK{-3} \xdef\consIM{1} \xdef\consII{4}
\xdef\consIL{2} \xdef\consIV{1} \xdef\consIF{0} \xdef\consIY{-1} \xdef\consIW{-3}

\xdef\consLC{-1} \xdef\consLS{-2} \xdef\consLT{-2} \xdef\consLP{-3} \xdef\consLA{-1}
\xdef\consLG{-4} \xdef\consLN{-3} \xdef\consLD{-4} \xdef\consLE{-3} \xdef\consLQ{-2}
\xdef\consLH{-3} \xdef\consLR{-2} \xdef\consLK{-2} \xdef\consLM{2} \xdef\consLI{2}
\xdef\consLL{4} \xdef\consLV{3} \xdef\consLF{0} \xdef\consLY{-1} \xdef\consLW{-2}

\xdef\consVC{-1} \xdef\consVS{-2} \xdef\consVT{-2} \xdef\consVP{-2} \xdef\consVA{0}
\xdef\consVG{-3} \xdef\consVN{-3} \xdef\consVD{-3} \xdef\consVE{-2} \xdef\consVQ{-2}
\xdef\consVH{-3} \xdef\consVR{-3} \xdef\consVK{-2} \xdef\consVM{1} \xdef\consVI{3}
\xdef\consVL{1} \xdef\consVV{4} \xdef\consVF{-1} \xdef\consVY{-1} \xdef\consVW{-3}

\xdef\consFC{-2} \xdef\consFS{-2} \xdef\consFT{-2} \xdef\consFP{-4} \xdef\consFA{-2}
\xdef\consFG{-3} \xdef\consFN{-3} \xdef\consFD{-3} \xdef\consFE{-3} \xdef\consFQ{-3}
\xdef\consFH{-1} \xdef\consFR{-3} \xdef\consFK{-3} \xdef\consFM{0} \xdef\consFI{0}
\xdef\consFL{0} \xdef\consFV{-1} \xdef\consFF{6} \xdef\consFY{3} \xdef\consFW{1}

\xdef\consYC{-2} \xdef\consYS{-2} \xdef\consYT{-2} \xdef\consYP{-3} \xdef\consYA{-2}
\xdef\consYG{-3} \xdef\consYN{-2} \xdef\consYD{-3} \xdef\consYE{-2} \xdef\consYQ{-1}
\xdef\consYH{2} \xdef\consYR{-2} \xdef\consYK{-2} \xdef\consYM{-1} \xdef\consYI{-1}
\xdef\consYL{-1} \xdef\consYV{-1} \xdef\consYF{3} \xdef\consYY{7} \xdef\consYW{2}

\xdef\consWC{-2} \xdef\consWS{-3} \xdef\consWT{-3} \xdef\consWP{-4} \xdef\consWA{-3}
\xdef\consWG{-2} \xdef\consWN{-4} \xdef\consWD{-4} \xdef\consWE{-3} \xdef\consWQ{-2}
\xdef\consWH{-2} \xdef\consWR{-3} \xdef\consWK{-3} \xdef\consWM{-1} \xdef\consWI{-3}
\xdef\consWL{-2} \xdef\consWV{-3} \xdef\consWF{1} \xdef\consWY{2} \xdef\consWW{11}

\gappenalty{0} \xdef\m@trixf@ctor{10}

\else

\message{<Unknown residue weight matrix - using `identity'>} \weighttable{identity}

\fi
\fi
\fi
\fi
\fi
}


\def\setweight#1#2#3{%
   \expandafter\xdef\csname cons#1#2\endcsname{#3}
   \expandafter\xdef\csname cons#2#1\endcsname{#3}
}

\weighttable{identity}
\gappenalty{0}


\def\c@d@ns{%
\codon{A}{GCA,GCG,GCC,GCT,GCU,GCN}
\codon{B}{---}
\codon{C}{TGC,TGT,UGC,UGU,TGY}
\codon{D}{GAC,GAT,GAU,GAY}
\codon{E}{GAA,GAG,GAR}
\codon{F}{TTC,TTT,UUC,UUU,TTY}
\codon{G}{GGA,GGG,GGC,GGT,GGU,GGN}
\codon{H}{CAC,CAT,CAY}
\codon{I}{ATA,ATC,ATT,AUA,AUC,AUU,ATH}
\codon{J}{---}
\codon{K}{AAA,AAG,AAG,AAR}
\codon{L}{CTA,CTG,CTC,CTT,TTA,TTG,CUG,CUG,CUC,CUU,UUA,UUG,YTN}
\codon{M}{ATG,AUG,ATG}
\codon{N}{AAC,AAT,AAU,AAY}
\codon{O}{---}
\codon{P}{CCA,CCG,CCC,CCT,CCU,CCN}
\codon{Q}{CAA,CAG,CAR}
\codon{R}{AGA,AGG,CGA,CGG,CGC,CGT,CGU,MGN}
\codon{S}{TCT,TCC,TCG,TCA,AGT,AGC,UCU,UCC,UCG,UCA,AGU,WSN}
\codon{T}{ACT,ACC,ACG,ACA,ACU,ACN}
\codon{U}{---}
\codon{V}{GTA,GTG,GTC,GTT,GUA,GUG,GUC,GUU,GTN}
\codon{W}{TGG,UGG,TGG}
\codon{X}{---}
\codon{Y}{TAC,TAT,UAC,UAU,TAY}
\codon{Z}{---}
\codon{.}{TAA,TAG,TGA,UAA,UAG,UGA,TRR}
}

\definecolor{GreenYellow}   {cmyk}{0.15,0,0.69,0}
\definecolor{Yellow}        {cmyk}{0,0,1,0}
\definecolor{Goldenrod}     {cmyk}{0,0.10,0.84,0}
\definecolor{Dandelion}     {cmyk}{0,0.29,0.84,0}
\definecolor{Apricot}       {cmyk}{0,0.32,0.52,0}
\definecolor{Peach}         {cmyk}{0,0.50,0.70,0}
\definecolor{Melon}         {cmyk}{0,0.46,0.50,0}
\definecolor{YellowOrange}  {cmyk}{0,0.42,1,0}
\definecolor{Orange}        {cmyk}{0,0.61,0.87,0}
\definecolor{BurntOrange}   {cmyk}{0,0.51,1,0}
\definecolor{Bittersweet}   {cmyk}{0,0.75,1,0.24}
\definecolor{RedOrange}     {cmyk}{0,0.77,0.87,0}
\definecolor{Mahagony}      {cmyk}{0,0.85,0.87,0.35}
\definecolor{Maroon}        {cmyk}{0,0.87,0.68,0.32}
\definecolor{BrickRed}      {cmyk}{0,0.89,0.94,0.28}
\definecolor{Red}           {cmyk}{0,1,1,0}
\definecolor{OrangeRed}     {cmyk}{0,1,0.50,0}
\definecolor{RubineRed}     {cmyk}{0,1,0.13,0}
\definecolor{WildStrawberry}{cmyk}{0,0.96,0.39,0}
\definecolor{Salmon}        {cmyk}{0,0.53,0.38,0}
\definecolor{CarnationPink} {cmyk}{0,0.63,0,0}
\definecolor{Magenta}       {cmyk}{0,1,0,0}
\definecolor{VioletRed}     {cmyk}{0,0.81,0,0}
\definecolor{Rhodamine}     {cmyk}{0,0.82,0,0}
\definecolor{Mulberry}      {cmyk}{0.34,0.90,0,0.02}
\definecolor{RedViolet}     {cmyk}{0.07,0.90,0,0.34}
\definecolor{Fuchsia}       {cmyk}{0.47,0.91,0,0.08}
\definecolor{Lavender}      {cmyk}{0,0.48,0,0}
\definecolor{Thistle}       {cmyk}{0.12,0.59,0,0}
\definecolor{Orchid}        {cmyk}{0.32,0.64,0,0}
\definecolor{DarkOrchid}    {cmyk}{0.40,0.80,0.20,0}
\definecolor{Purple}        {cmyk}{0.45,0.86,0,0}
\definecolor{Plum}          {cmyk}{0.50,1,0,0}
\definecolor{Violet}        {cmyk}{0.79,0.88,0,0}
\definecolor{RoyalPurple}   {cmyk}{0.75,0.90,0,0}
\definecolor{BlueViolet}    {cmyk}{0.86,0.91,0,0.04}
\definecolor{Periwinkle}    {cmyk}{0.57,0.55,0,0}
\definecolor{CadetBlue}     {cmyk}{0.62,0.57,0.23,0}
\definecolor{CornflowerBlue}{cmyk}{0.65,0.13,0,0}
\definecolor{MidnightBlue}  {cmyk}{0.98,0.13,0,0.43}
\definecolor{NavyBlue}      {cmyk}{0.94,0.54,0,0}
\definecolor{RoyalBlue}     {cmyk}{1,0.50,0,0}
\definecolor{Blue}          {cmyk}{1,1,0,0}
\definecolor{Cerulean}      {cmyk}{0.94,0.11,0,0}
\definecolor{Cyan}          {cmyk}{1,0,0,0}
\definecolor{ProcessBlue}   {cmyk}{0.96,0,0,0}
\definecolor{SkyBlue}       {cmyk}{0.62,0,0.12,0}
\definecolor{Turquoise}     {cmyk}{0.85,0,0.20,0}
\definecolor{TealBlue}      {cmyk}{0.86,0,0.34,0.02}
\definecolor{Aquamarine}    {cmyk}{0.82,0,0.30,0}
\definecolor{BlueGreen}     {cmyk}{0.85,0,0.33,0}
\definecolor{Emerald}       {cmyk}{1,0,0.50,0}
\definecolor{JungleGreen}   {cmyk}{0.99,0,0.52,0}
\definecolor{SeaGreen}      {cmyk}{0.69,0,0.50,0}
\definecolor{Green}         {cmyk}{1,0,1,0}
\definecolor{ForestGreen}   {cmyk}{0.91,0,0.88,0.12}
\definecolor{PineGreen}     {cmyk}{0.92,0,0.59,0.25}
\definecolor{LimeGreen}     {cmyk}{0.50,0,1,0}
\definecolor{YellowGreen}   {cmyk}{0.44,0,0.74,0}
\definecolor{SpringGreen}   {cmyk}{0.26,0,0.76,0}
\definecolor{OliveGreen}    {cmyk}{0.64,0,0.95,0.40}
\definecolor{RawSienna}     {cmyk}{0,0.72,1,0.45}
\definecolor{Sepia}         {cmyk}{0,0.83,1,0.70}
\definecolor{Brown}         {cmyk}{0,0.81,1,0.60}
\definecolor{Tan}           {cmyk}{0.14,0.42,0.56,0}
\definecolor{White}         {cmyk}{0,0,0,0}
\definecolor{Gray0}         {cmyk}{0,0,0,0}
\definecolor{Gray5}         {cmyk}{0,0,0,0.05}
\definecolor{Gray10}        {cmyk}{0,0,0,0.10}
\definecolor{Gray15}        {cmyk}{0,0,0,0.15}
\definecolor{Gray20}        {cmyk}{0,0,0,0.20}
\definecolor{Gray25}        {cmyk}{0,0,0,0.25}
\definecolor{Gray30}        {cmyk}{0,0,0,0.30}
\definecolor{LightGray}     {cmyk}{0,0,0,0.33}
\definecolor{Gray35}        {cmyk}{0,0,0,0.35}
\definecolor{Gray40}        {cmyk}{0,0,0,0.40}
\definecolor{Gray45}        {cmyk}{0,0,0,0.45}
\definecolor{Gray50}        {cmyk}{0,0,0,0.50}
\definecolor{Gray}          {cmyk}{0,0,0,0.50}
\definecolor{GrayDefault}   {cmyk}{0,0,0,0.50}
\definecolor{Gray55}        {cmyk}{0,0,0,0.55}
\definecolor{Gray60}        {cmyk}{0,0,0,0.60}
\definecolor{Gray65}        {cmyk}{0,0,0,0.65}
\definecolor{DarkGray}      {cmyk}{0,0,0,0.66}
\definecolor{Gray70}        {cmyk}{0,0,0,0.70}
\definecolor{Gray75}        {cmyk}{0,0,0,0.75}
\definecolor{Gray80}        {cmyk}{0,0,0,0.80}
\definecolor{Gray85}        {cmyk}{0,0,0,0.85}
\definecolor{Gray90}        {cmyk}{0,0,0,0.90}
\definecolor{Gray95}        {cmyk}{0,0,0,0.95}
\definecolor{Black}         {cmyk}{0,0,0,1}
\definecolor{Gray100}       {cmyk}{0,0,0,1}
\definecolor{TC0}           {cmyk}{0.4,0.4,0,0}
\definecolor{TC1}           {cmyk}{0.6,0,0.7,0}
\definecolor{TC2}           {cmyk}{0.4,0,0.7,0}
\definecolor{TC3}           {cmyk}{0.2,0,1,0}
\definecolor{TC4}           {cmyk}{0,0,1,0}
\definecolor{TC5}           {cmyk}{0,0.2,1,0}
\definecolor{TC6}           {cmyk}{0,0.4,1,0}
\definecolor{TC7}           {cmyk}{0,0.6,1,0}
\definecolor{TC8}           {cmyk}{0,0.8,1,0}
\definecolor{TC9}           {cmyk}{0,0.875,1,0}
\definecolor{TC99}          {cmyk}{0,0,0,0}
\definecolor{LightGreenYellow}   {cmyk}{0.08,0,0.35,0}
\definecolor{LightYellow}        {cmyk}{0,0,0.50,0}
\definecolor{LightGoldenrod}     {cmyk}{0,0.05,0.42,0}
\definecolor{LightDandelion}     {cmyk}{0,0.15,0.42,0}
\definecolor{LightApricot}       {cmyk}{0,0.16,0.26,0}
\definecolor{LightPeach}         {cmyk}{0,0.25,0.35,0}
\definecolor{LightMelon}         {cmyk}{0,0.23,0.25,0}
\definecolor{LightYellowOrange}  {cmyk}{0,0.21,0.50,0}
\definecolor{LightOrange}        {cmyk}{0,0.31,0.44,0}
\definecolor{LightBurntOrange}   {cmyk}{0,0.26,0.50,0}
\definecolor{LightBittersweet}   {cmyk}{0,0.38,0.50,0.12}
\definecolor{LightRedOrange}     {cmyk}{0,0.39,0.44,0}
\definecolor{LightMahagony}      {cmyk}{0,0.43,0.44,0.18}
\definecolor{LightMaroon}        {cmyk}{0,0.44,0.34,0.16}
\definecolor{LightBrickRed}      {cmyk}{0,0.45,0.47,0.14}
\definecolor{LightRed}           {cmyk}{0,0.50,0.50,0}
\definecolor{LightOrangeRed}     {cmyk}{0,0.50,0.25,0}
\definecolor{LightRubineRed}     {cmyk}{0,0.50,0.07,0}
\definecolor{LightWildStrawberry}{cmyk}{0,0.48,0.20,0}
\definecolor{LightSalmon}        {cmyk}{0,0.27,0.19,0}
\definecolor{LightCarnationPink} {cmyk}{0,0.32,0,0}
\definecolor{LightMagenta}       {cmyk}{0,0.50,0,0}
\definecolor{LightVioletRed}     {cmyk}{0,0.40,0,0}
\definecolor{LightRhodamine}     {cmyk}{0,0.41,0,0}
\definecolor{LightMulberry}      {cmyk}{0.17,0.45,0,0.01}
\definecolor{LightRedViolet}     {cmyk}{0.04,0.45,0,0.17}
\definecolor{LightFuchsia}       {cmyk}{0.24,0.46,0,0.04}
\definecolor{LightLavender}      {cmyk}{0,0.24,0,0}
\definecolor{LightThistle}       {cmyk}{0.06,0.30,0,0}
\definecolor{LightOrchid}        {cmyk}{0.16,0.32,0,0}
\definecolor{LightDarkOrchid}    {cmyk}{0.20,0.40,0.10,0}
\definecolor{LightPurple}        {cmyk}{0.23,0.43,0,0}
\definecolor{LightPlum}          {cmyk}{0.25,0.50,0,0}
\definecolor{LightViolet}        {cmyk}{0.40,0.44,0,0}
\definecolor{LightRoyalPurple}   {cmyk}{0.38,0.45,0,0}
\definecolor{LightBlueViolet}    {cmyk}{0.43,0.46,0,0.02}
\definecolor{LightPeriwinkle}    {cmyk}{0.29,0.28,0,0}
\definecolor{LightCadetBlue}     {cmyk}{0.31,0.29,0.12,0}
\definecolor{LightCornflowerBlue}{cmyk}{0.33,0.07,0,0}
\definecolor{LightMidnightBlue}  {cmyk}{0.49,0.07,0,0.22}
\definecolor{LightNavyBlue}      {cmyk}{0.47,0.27,0,0}
\definecolor{LightRoyalBlue}     {cmyk}{0.50,0.25,0,0}
\definecolor{LightBlue}          {cmyk}{0.50,0.50,0,0}
\definecolor{LightCerulean}      {cmyk}{0.47,0.06,0,0}
\definecolor{LightCyan}          {cmyk}{0.50,0,0,0}
\definecolor{LightProcessBlue}   {cmyk}{0.48,0,0,0}
\definecolor{LightSkyBlue}       {cmyk}{0.31,0,0.06,0}
\definecolor{LightTurquoise}     {cmyk}{0.43,0,0.10,0}
\definecolor{LightTealBlue}      {cmyk}{0.43,0,0.17,0.01}
\definecolor{LightAquamarine}    {cmyk}{0.41,0,0.15,0}
\definecolor{LightBlueGreen}     {cmyk}{0.43,0,0.17,0}
\definecolor{LightEmerald}       {cmyk}{0.50,0,0.25,0}
\definecolor{LightJungleGreen}   {cmyk}{0.50,0,0.26,0}
\definecolor{LightSeaGreen}      {cmyk}{0.35,0,0.25,0}
\definecolor{LightGreen}         {cmyk}{0.50,0,0.50,0}
\definecolor{LightForestGreen}   {cmyk}{0.46,0,0.44,0.06}
\definecolor{LightPineGreen}     {cmyk}{0.46,0,0.30,0.13}
\definecolor{LightLimeGreen}     {cmyk}{0.25,0,0.50,0}
\definecolor{LightYellowGreen}   {cmyk}{0.22,0,0.37,0}
\definecolor{LightSpringGreen}   {cmyk}{0.13,0,0.38,0}
\definecolor{LightOliveGreen}    {cmyk}{0.32,0,0.48,0.20}
\definecolor{LightRawSienna}     {cmyk}{0,0.36,0.50,0.23}
\definecolor{LightSepia}         {cmyk}{0,0.44,0.50,0.35}
\definecolor{LightBrown}         {cmyk}{0,0.41,0.50,0.30}
\definecolor{LightTan}           {cmyk}{0.07,0.21,0.28,0}
\definecolor{LightWhite}         {cmyk}{0,0,0,0}
\definecolor{LightGray0}         {cmyk}{0,0,0,0}
\definecolor{LightGray5}         {cmyk}{0,0,0,0.02}
\definecolor{LightGray10}        {cmyk}{0,0,0,0.05}
\definecolor{LightGray15}        {cmyk}{0,0,0,0.07}
\definecolor{LightGray20}        {cmyk}{0,0,0,0.10}
\definecolor{LightGray25}        {cmyk}{0,0,0,0.12}
\definecolor{LightGray30}        {cmyk}{0,0,0,0.15}
\definecolor{LightLightGray}     {cmyk}{0,0,0,0.16}
\definecolor{LightGray35}        {cmyk}{0,0,0,0.17}
\definecolor{LightGray40}        {cmyk}{0,0,0,0.20}
\definecolor{LightGray45}        {cmyk}{0,0,0,0.22}
\definecolor{LightGray50}        {cmyk}{0,0,0,0.25}
\definecolor{LightGray}          {cmyk}{0,0,0,0.25}
\definecolor{LightGray55}        {cmyk}{0,0,0,0.27}
\definecolor{LightGray60}        {cmyk}{0,0,0,0.30}
\definecolor{LightGray65}        {cmyk}{0,0,0,0.32}
\definecolor{LightDarkGray}      {cmyk}{0,0,0,0.33}
\definecolor{LightGray70}        {cmyk}{0,0,0,0.35}
\definecolor{LightGray75}        {cmyk}{0,0,0,0.37}
\definecolor{LightGray80}        {cmyk}{0,0,0,0.40}
\definecolor{LightGray85}        {cmyk}{0,0,0,0.42}
\definecolor{LightGray90}        {cmyk}{0,0,0,0.45}
\definecolor{LightGray95}        {cmyk}{0,0,0,0.47}
\definecolor{LightBlack}         {cmyk}{0,0,0,0.50}
\definecolor{LightGray100}       {cmyk}{0,0,0,0.50}
\definecolor{LightTC0}           {cmyk}{0.2,0.2,0,0}
\definecolor{LightTC1}           {cmyk}{0.3,0,0.35,0}
\definecolor{LightTC2}           {cmyk}{0.2,0,0.35,0}
\definecolor{LightTC3}           {cmyk}{0.1,0,0.5,0}
\definecolor{LightTC4}           {cmyk}{0,0,0.5,0}
\definecolor{LightTC5}           {cmyk}{0,0.1,0.5,0}
\definecolor{LightTC6}           {cmyk}{0,0.2,0.5,0}
\definecolor{LightTC7}           {cmyk}{0,0.3,0.5,0}
\definecolor{LightTC8}           {cmyk}{0,0.4,0.5,0}
\definecolor{LightTC99}          {cmyk}{0,0,0,0}
\definecolor{LightLightGreenYellow}   {cmyk}{0.04,0,0.17,0}
\definecolor{LightLightYellow}        {cmyk}{0,0,0.25,0}
\definecolor{LightLightGoldenrod}     {cmyk}{0,0.02,0.21,0}
\definecolor{LightLightDandelion}     {cmyk}{0,0.07,0.21,0}
\definecolor{LightLightApricot}       {cmyk}{0,0.08,0.13,0}
\definecolor{LightLightPeach}         {cmyk}{0,0.12,0.17,0}
\definecolor{LightLightMelon}         {cmyk}{0,0.11,0.12,0}
\definecolor{LightLightYellowOrange}  {cmyk}{0,0.10,0.25,0}
\definecolor{LightLightOrange}        {cmyk}{0,0.15,0.22,0}
\definecolor{LightLightBurntOrange}   {cmyk}{0,0.13,0.25,0}
\definecolor{LightLightBittersweet}   {cmyk}{0,0.19,0.25,0.06}
\definecolor{LightLightRedOrange}     {cmyk}{0,0.14,0.22,0}
\definecolor{LightLightMahagony}      {cmyk}{0,0.21,0.22,0.09}
\definecolor{LightLightMaroon}        {cmyk}{0,0.22,0.17,0.08}
\definecolor{LightLightBrickRed}      {cmyk}{0,0.22,0.23,0.07}
\definecolor{LightLightRed}           {cmyk}{0,0.25,0.25,0}
\definecolor{LightLightOrangeRed}     {cmyk}{0,0.25,0.12,0}
\definecolor{LightLightRubineRed}     {cmyk}{0,0.25,0.03,0}
\definecolor{LightLightWildStrawberry}{cmyk}{0,0.24,0.10,0}
\definecolor{LightLightSalmon}        {cmyk}{0,0.13,0.09,0}
\definecolor{LightLightCarnationPink} {cmyk}{0,0.16,0,0}
\definecolor{LightLightMagenta}       {cmyk}{0,0.25,0,0}
\definecolor{LightLightVioletRed}     {cmyk}{0,0.20,0,0}
\definecolor{LightLightRhodamine}     {cmyk}{0,0.20,0,0}
\definecolor{LightLightMulberry}      {cmyk}{0.08,0.22,0,0.005}
\definecolor{LightLightRedViolet}     {cmyk}{0.02,0.22,0,0.08}
\definecolor{LightLightFuchsia}       {cmyk}{0.12,0.23,0,0.02}
\definecolor{LightLightLavender}      {cmyk}{0,0.12,0,0}
\definecolor{LightLightThistle}       {cmyk}{0.03,0.15,0,0}
\definecolor{LightLightOrchid}        {cmyk}{0.08,0.16,0,0}
\definecolor{LightLightDarkOrchid}    {cmyk}{0.10,0.20,0.05,0}
\definecolor{LightLightPurple}        {cmyk}{0.11,0.21,0,0}
\definecolor{LightLightPlum}          {cmyk}{0.12,0.25,0,0}
\definecolor{LightLightViolet}        {cmyk}{0.20,0.22,0,0}
\definecolor{LightLightRoyalPurple}   {cmyk}{0.19,0.22,0,0}
\definecolor{LightLightBlueViolet}    {cmyk}{0.21,0.23,0,0.01}
\definecolor{LightLightPeriwinkle}    {cmyk}{0.14,0.14,0,0}
\definecolor{LightLightCadetBlue}     {cmyk}{0.15,0.14,0.06,0}
\definecolor{LightLightCornflowerBlue}{cmyk}{0.16,0.03,0,0}
\definecolor{LightLightMidnightBlue}  {cmyk}{0.24,0.03,0,0.11}
\definecolor{LightLightNavyBlue}      {cmyk}{0.23,0.13,0,0}
\definecolor{LightLightRoyalBlue}     {cmyk}{0.25,0.12,0,0}
\definecolor{LightLightBlue}          {cmyk}{0.25,0.25,0,0}
\definecolor{LightLightCerulean}      {cmyk}{0.23,0.03,0,0}
\definecolor{LightLightCyan}          {cmyk}{0.25,0,0,0}
\definecolor{LightLightProcessBlue}   {cmyk}{0.24,0,0,0}
\definecolor{LightLightSkyBlue}       {cmyk}{0.15,0,0.03,0}
\definecolor{LightLightTurquoise}     {cmyk}{0.21,0,0.05,0}
\definecolor{LightLightTealBlue}      {cmyk}{0.21,0,0.08,0.005}
\definecolor{LightLightAquamarine}    {cmyk}{0.20,0,0.07,0}
\definecolor{LightLightBlueGreen}     {cmyk}{0.21,0,0.08,0}
\definecolor{LightLightEmerald}       {cmyk}{0.25,0,0.12,0}
\definecolor{LightLightJungleGreen}   {cmyk}{0.25,0,0.13,0}
\definecolor{LightLightSeaGreen}      {cmyk}{0.17,0,0.12,0}
\definecolor{LightLightGreen}         {cmyk}{0.25,0,0.25,0}
\definecolor{LightLightForestGreen}   {cmyk}{0.23,0,0.22,0.03}
\definecolor{LightLightPineGreen}     {cmyk}{0.23,0,0.15,0.06}
\definecolor{LightLightLimeGreen}     {cmyk}{0.12,0,0.25,0}
\definecolor{LightLightYellowGreen}   {cmyk}{0.11,0,0.18,0}
\definecolor{LightLightSpringGreen}   {cmyk}{0.06,0,0.19,0}
\definecolor{LightLightOliveGreen}    {cmyk}{0.16,0,0.24,0.10}
\definecolor{LightLightRawSienna}     {cmyk}{0,0.18,0.25,0.11}
\definecolor{LightLightSepia}         {cmyk}{0,0.22,0.25,0.17}
\definecolor{LightLightBrown}         {cmyk}{0,0.20,0.25,0.15}
\definecolor{LightLightTan}           {cmyk}{0.03,0.10,0.14,0}
\definecolor{LightLightWhite}         {cmyk}{0,0,0,0}
\definecolor{LightLightGray0}         {cmyk}{0,0,0,0}
\definecolor{LightLightGray5}         {cmyk}{0,0,0,0.01}
\definecolor{LightLightGray10}        {cmyk}{0,0,0,0.02}
\definecolor{LightLightGray15}        {cmyk}{0,0,0,0.03}
\definecolor{LightLightGray20}        {cmyk}{0,0,0,0.05}
\definecolor{LightLightGray25}        {cmyk}{0,0,0,0.06}
\definecolor{LightLightGray30}        {cmyk}{0,0,0,0.07}
\definecolor{LightLightLightGray}     {cmyk}{0,0,0,0.08}
\definecolor{LightLightGray35}        {cmyk}{0,0,0,0.09}
\definecolor{LightLightGray40}        {cmyk}{0,0,0,0.10}
\definecolor{LightLightGray45}        {cmyk}{0,0,0,0.11}
\definecolor{LightLightGray50}        {cmyk}{0,0,0,0.12}
\definecolor{LightLightGray}          {cmyk}{0,0,0,0.13}
\definecolor{LightLightGray55}        {cmyk}{0,0,0,0.14}
\definecolor{LightLightGray60}        {cmyk}{0,0,0,0.15}
\definecolor{LightLightGray65}        {cmyk}{0,0,0,0.16}
\definecolor{LightLightDarkGray}      {cmyk}{0,0,0,0.17}
\definecolor{LightLightGray70}        {cmyk}{0,0,0,0.18}
\definecolor{LightLightGray75}        {cmyk}{0,0,0,0.19}
\definecolor{LightLightGray80}        {cmyk}{0,0,0,0.20}
\definecolor{LightLightGray85}        {cmyk}{0,0,0,0.21}
\definecolor{LightLightGray90}        {cmyk}{0,0,0,0.22}
\definecolor{LightLightGray95}        {cmyk}{0,0,0,0.23}
\definecolor{LightLightBlack}         {cmyk}{0,0,0,0.25}
\definecolor{LightLightGray100}       {cmyk}{0,0,0,0.25}
\definecolor{LightLightTC0}           {cmyk}{0.1,0.1,0,0}
\definecolor{LightLightTC1}           {cmyk}{0.15,0,0.175,0}
\definecolor{LightLightTC2}           {cmyk}{0.1,0,0.175,0}
\definecolor{LightLightTC3}           {cmyk}{0.05,0,0.25,0}
\definecolor{LightLightTC4}           {cmyk}{0,0,0.25,0}
\definecolor{LightLightTC5}           {cmyk}{0,0.05,0.25,0}
\definecolor{LightLightTC6}           {cmyk}{0,0.1,0.25,0}
\definecolor{LightLightTC7}           {cmyk}{0,0.15,0.25,0}
\definecolor{LightLightTC8}           {cmyk}{0,0.2,0.25,0}
\definecolor{LightLightTC99}          {cmyk}{0,0,0,0}
\definecolor{LightLightLightGreenYellow}   {cmyk}{0.02,0,0.08,0}
\definecolor{LightLightLightYellow}        {cmyk}{0,0,0.12,0}
\definecolor{LightLightLightGoldenrod}     {cmyk}{0,0.01,0.10,0}
\definecolor{LightLightLightDandelion}     {cmyk}{0,0.03,0.10,0}
\definecolor{LightLightLightApricot}       {cmyk}{0,0.04,0.06,0}
\definecolor{LightLightLightPeach}         {cmyk}{0,0.06,0.08,0}
\definecolor{LightLightLightMelon}         {cmyk}{0,0.05,0.06,0}
\definecolor{LightLightLightYellowOrange}  {cmyk}{0,0.05,0.12,0}
\definecolor{LightLightLightOrange}        {cmyk}{0,0.07,0.11,0}
\definecolor{LightLightLightBurntOrange}   {cmyk}{0,0.06,0.12,0}
\definecolor{LightLightLightBittersweet}   {cmyk}{0,0.09,0.12,0.03}
\definecolor{LightLightLightRedOrange}     {cmyk}{0,0.07,0.11,0}
\definecolor{LightLightLightMahagony}      {cmyk}{0,0.10,0.11,0.04}
\definecolor{LightLightLightMaroon}        {cmyk}{0,0.11,0.08,0.04}
\definecolor{LightLightLightBrickRed}      {cmyk}{0,0.11,0.11,0.03}
\definecolor{LightLightLightRed}           {cmyk}{0,0.12,0.12,0}
\definecolor{LightLightLightOrangeRed}     {cmyk}{0,0.12,0.06,0}
\definecolor{LightLightLightRubineRed}     {cmyk}{0,0.12,0.01,0}
\definecolor{LightLightLightWildStrawberry}{cmyk}{0,0.12,0.05,0}
\definecolor{LightLightLightSalmon}        {cmyk}{0,0.06,0.04,0}
\definecolor{LightLightLightCarnationPink} {cmyk}{0,0.08,0,0}
\definecolor{LightLightLightMagenta}       {cmyk}{0,0.12,0,0}
\definecolor{LightLightLightLightMagenta}  {cmyk}{0,0.06,0,0}
\definecolor{LightLightLightVioletRed}     {cmyk}{0,0.10,0,0}
\definecolor{LightLightLightRhodamine}     {cmyk}{0,0.10,0,0}
\definecolor{LightLightLightMulberry}      {cmyk}{0.04,0.11,0,0.002}
\definecolor{LightLightLightRedViolet}     {cmyk}{0.01,0.11,0,0.04}
\definecolor{LightLightLightFuchsia}       {cmyk}{0.06,0.11,0,0.01}
\definecolor{LightLightLightLavender}      {cmyk}{0,0.06,0,0}
\definecolor{LightLightLightThistle}       {cmyk}{0.01,0.07,0,0}
\definecolor{LightLightLightOrchid}        {cmyk}{0.04,0.08,0,0}
\definecolor{LightLightLightDarkOrchid}    {cmyk}{0.05,0.10,0.02,0}
\definecolor{LightLightLightPurple}        {cmyk}{0.05,0.10,0,0}
\definecolor{LightLightLightPlum}          {cmyk}{0.06,0.12,0,0}
\definecolor{LightLightLightViolet}        {cmyk}{0.10,0.11,0,0}
\definecolor{LightLightLightRoyalPurple}   {cmyk}{0.09,0.11,0,0}
\definecolor{LightLightLightBlueViolet}    {cmyk}{0.10,0.11,0,0.005}
\definecolor{LightLightLightPeriwinkle}    {cmyk}{0.07,0.07,0,0}
\definecolor{LightLightLightCadetBlue}     {cmyk}{0.07,0.07,0.03,0}
\definecolor{LightLightLightCornflowerBlue}{cmyk}{0.08,0.01,0,0}
\definecolor{LightLightLightMidnightBlue}  {cmyk}{0.12,0.01,0,0.05}
\definecolor{LightLightLightNavyBlue}      {cmyk}{0.11,0.06,0,0}
\definecolor{LightLightLightRoyalBlue}     {cmyk}{0.12,0.06,0,0}
\definecolor{LightLightLightBlue}          {cmyk}{0.12,0.12,0,0}
\definecolor{LightLightLightCerulean}      {cmyk}{0.11,0.01,0,0}
\definecolor{LightLightLightCyan}          {cmyk}{0.12,0,0,0}
\definecolor{LightLightLightProcessBlue}   {cmyk}{0.12,0,0,0}
\definecolor{LightLightLightSkyBlue}       {cmyk}{0.07,0,0.01,0}
\definecolor{LightLightLightTurquoise}     {cmyk}{0.10,0,0.02,0}
\definecolor{LightLightLightTealBlue}      {cmyk}{0.10,0,0.04,0.002}
\definecolor{LightLightLightAquamarine}    {cmyk}{0.10,0,0.03,0}
\definecolor{LightLightLightBlueGreen}     {cmyk}{0.10,0,0.04,0}
\definecolor{LightLightLightEmerald}       {cmyk}{0.12,0,0.06,0}
\definecolor{LightLightLightJungleGreen}   {cmyk}{0.12,0,0.06,0}
\definecolor{LightLightLightSeaGreen}      {cmyk}{0.08,0,0.06,0}
\definecolor{LightLightLightGreen}         {cmyk}{0.12,0,0.12,0}
\definecolor{LightLightLightForestGreen}   {cmyk}{0.11,0,0.11,0.01}
\definecolor{LightLightLightPineGreen}     {cmyk}{0.11,0,0.07,0.03}
\definecolor{LightLightLightLimeGreen}     {cmyk}{0.06,0,0.12,0}
\definecolor{LightLightLightYellowGreen}   {cmyk}{0.05,0,0.09,0}
\definecolor{LightLightLightSpringGreen}   {cmyk}{0.03,0,0.09,0}
\definecolor{LightLightLightOliveGreen}    {cmyk}{0.08,0,0.12,0.05}
\definecolor{LightLightLightRawSienna}     {cmyk}{0,0.09,0.12,0.05}
\definecolor{LightLightLightSepia}         {cmyk}{0,0.11,0.12,0.06}
\definecolor{LightLightLightBrown}         {cmyk}{0,0.10,0.12,0.07}
\definecolor{LightLightLightTan}           {cmyk}{0.01,0.05,0.07,0}
\definecolor{LightLightLightWhite}         {cmyk}{0,0,0,0}
\definecolor{LightLightLightGray0}         {cmyk}{0,0,0,0}
\definecolor{LightLightLightGray5}         {cmyk}{0,0,0,0.005}
\definecolor{LightLightLightGray10}        {cmyk}{0,0,0,0.01}
\definecolor{LightLightLightGray15}        {cmyk}{0,0,0,0.015}
\definecolor{LightLightLightGray20}        {cmyk}{0,0,0,0.025}
\definecolor{LightLightLightGray25}        {cmyk}{0,0,0,0.03}
\definecolor{LightLightLightGray30}        {cmyk}{0,0,0,0.035}
\definecolor{LightLightLightLightGray}     {cmyk}{0,0,0,0.04}
\definecolor{LightLightLightGray35}        {cmyk}{0,0,0,0.045}
\definecolor{LightLightLightGray40}        {cmyk}{0,0,0,0.05}
\definecolor{LightLightLightGray45}        {cmyk}{0,0,0,0.055}
\definecolor{LightLightLightGray50}        {cmyk}{0,0,0,0.06}
\definecolor{LightLightLightGray}          {cmyk}{0,0,0,0.065}
\definecolor{LightLightLightGray55}        {cmyk}{0,0,0,0.07}
\definecolor{LightLightLightGray60}        {cmyk}{0,0,0,0.075}
\definecolor{LightLightLightGray65}        {cmyk}{0,0,0,0.08}
\definecolor{LightLightLightDarkGray}      {cmyk}{0,0,0,0.085}
\definecolor{LightLightLightGray70}        {cmyk}{0,0,0,0.09}
\definecolor{LightLightLightGray75}        {cmyk}{0,0,0,0.095}
\definecolor{LightLightLightGray80}        {cmyk}{0,0,0,0.10}
\definecolor{LightLightLightGray85}        {cmyk}{0,0,0,0.105}
\definecolor{LightLightLightGray90}        {cmyk}{0,0,0,0.11}
\definecolor{LightLightLightGray95}        {cmyk}{0,0,0,0.115}
\definecolor{LightLightLightBlack}         {cmyk}{0,0,0,0.12}
\definecolor{LightLightLightGray100}       {cmyk}{0,0,0,0.125}
\definecolor{LightLightLightTC0}           {cmyk}{0.05,0.05,0,0}
\definecolor{LightLightLightTC1}           {cmyk}{0.075,0,0.086,0}
\definecolor{LightLightLightTC2}           {cmyk}{0.05,0,0.086,0}
\definecolor{LightLightLightTC3}           {cmyk}{0.025,0,0.125,0}
\definecolor{LightLightLightTC4}           {cmyk}{0,0,0.125,0}
\definecolor{LightLightLightTC5}           {cmyk}{0,0.025,0.125,0}
\definecolor{LightLightLightTC6}           {cmyk}{0,0.05,0.125,0}
\definecolor{LightLightLightTC7}           {cmyk}{0,0.0175,0.125,0}
\definecolor{LightLightLightTC8}           {cmyk}{0,0.1,0.125,0}
\definecolor{LightLightLightTC99}          {cmyk}{0,0,0,0}
\definecolor{BlueRed5}           {rgb} {0.15,0.17,0.55}
\definecolor{BlueRed10}          {rgb} {0.20,0.23,0.57}
\definecolor{BlueRed15}          {rgb} {0.24,0.29,0.60}
\definecolor{BlueRed20}          {rgb} {0.33,0.35,0.64}
\definecolor{BlueRed25}          {rgb} {0.43,0.43,0.68}
\definecolor{BlueRed30}          {rgb} {0.52,0.52,0.73}
\definecolor{BlueRed35}          {rgb} {0.60,0.60,0.78}
\definecolor{BlueRed40}          {rgb} {0.70,0.70,0.84}
\definecolor{BlueRed45}          {rgb} {0.80,0.80,0.85}
\definecolor{BlueRed50}          {rgb} {0.86,0.82,0.82}
\definecolor{BlueRed55}          {rgb} {0.87,0.73,0.73}
\definecolor{BlueRed60}          {rgb} {0.89,0.64,0.64}
\definecolor{BlueRed65}          {rgb} {0.90,0.55,0.55}
\definecolor{BlueRed70}          {rgb} {0.91,0.47,0.46}
\definecolor{BlueRed75}          {rgb} {0.91,0.39,0.37}
\definecolor{BlueRed80}          {rgb} {0.90,0.33,0.28}
\definecolor{BlueRed85}          {rgb} {0.89,0.25,0.20}
\definecolor{BlueRed90}          {rgb} {0.88,0.23,0.14}
\definecolor{BlueRed95}          {rgb} {0.87,0.21,0.09}
\definecolor{BlueRed100}         {rgb} {0.87,0.16,0.04}
\definecolor{RedBlue100}         {rgb} {0.15,0.17,0.55}
\definecolor{RedBlue95}          {rgb} {0.15,0.17,0.55}
\definecolor{RedBlue90}          {rgb} {0.20,0.23,0.57}
\definecolor{RedBlue85}          {rgb} {0.24,0.29,0.60}
\definecolor{RedBlue80}          {rgb} {0.33,0.35,0.64}
\definecolor{RedBlue75}          {rgb} {0.43,0.43,0.68}
\definecolor{RedBlue70}          {rgb} {0.52,0.52,0.73}
\definecolor{RedBlue65}          {rgb} {0.60,0.60,0.78}
\definecolor{RedBlue60}          {rgb} {0.70,0.70,0.84}
\definecolor{RedBlue55}          {rgb} {0.80,0.80,0.85}
\definecolor{RedBlue50}          {rgb} {0.86,0.82,0.82}
\definecolor{RedBlue45}          {rgb} {0.87,0.73,0.73}
\definecolor{RedBlue40}          {rgb} {0.89,0.64,0.64}
\definecolor{RedBlue35}          {rgb} {0.90,0.55,0.55}
\definecolor{RedBlue30}          {rgb} {0.91,0.47,0.46}
\definecolor{RedBlue25}          {rgb} {0.91,0.39,0.37}
\definecolor{RedBlue20}          {rgb} {0.90,0.33,0.28}
\definecolor{RedBlue15}          {rgb} {0.89,0.25,0.20}
\definecolor{RedBlue10}          {rgb} {0.88,0.23,0.14}
\definecolor{RedBlue5}           {rgb} {0.87,0.21,0.09}
\definecolor{GreenRed5}          {rgb} {0,1,0}
\definecolor{GreenRed10}         {rgb} {0.05,0.95,0}
\definecolor{GreenRed15}         {rgb} {0.10,0.90,0}
\definecolor{GreenRed20}         {rgb} {0.15,0.85,0}
\definecolor{GreenRed25}         {rgb} {0.20,0.80,0}
\definecolor{GreenRed30}         {rgb} {0.25,0.75,0}
\definecolor{GreenRed35}         {rgb} {0.30,0.70,0}
\definecolor{GreenRed40}         {rgb} {0.35,0.65,0}
\definecolor{GreenRed45}         {rgb} {0.40,0.60,0}
\definecolor{GreenRed50}         {rgb} {0.45,0.55,0}
\definecolor{GreenRed55}         {rgb} {0.50,0.50,0}
\definecolor{GreenRed60}         {rgb} {0.55,0.45,0}
\definecolor{GreenRed65}         {rgb} {0.60,0.40,0}
\definecolor{GreenRed70}         {rgb} {0.65,0.35,0}
\definecolor{GreenRed75}         {rgb} {0.70,0.30,0}
\definecolor{GreenRed80}         {rgb} {0.75,0.25,0}
\definecolor{GreenRed85}         {rgb} {0.80,0.20,0}
\definecolor{GreenRed90}         {rgb} {0.85,0.15,0}
\definecolor{GreenRed95}         {rgb} {0.90,0.10,0}
\definecolor{GreenRed100}        {rgb} {0.95,0.05,0}
\definecolor{RedGreen100}        {rgb} {0.05,0.95,0}
\definecolor{RedGreen95}         {rgb} {0.10,0.90,0}
\definecolor{RedGreen90}         {rgb} {0.15,0.85,0}
\definecolor{RedGreen85}         {rgb} {0.20,0.80,0}
\definecolor{RedGreen80}         {rgb} {0.25,0.75,0}
\definecolor{RedGreen75}         {rgb} {0.30,0.70,0}
\definecolor{RedGreen70}         {rgb} {0.35,0.65,0}
\definecolor{RedGreen65}         {rgb} {0.40,0.60,0}
\definecolor{RedGreen60}         {rgb} {0.45,0.55,0}
\definecolor{RedGreen55}         {rgb} {0.50,0.50,0}
\definecolor{RedGreen50}         {rgb} {0.55,0.45,0}
\definecolor{RedGreen45}         {rgb} {0.60,0.40,0}
\definecolor{RedGreen40}         {rgb} {0.65,0.35,0}
\definecolor{RedGreen35}         {rgb} {0.70,0.30,0}
\definecolor{RedGreen30}         {rgb} {0.75,0.25,0}
\definecolor{RedGreen25}         {rgb} {0.80,0.20,0}
\definecolor{RedGreen20}         {rgb} {0.85,0.15,0}
\definecolor{RedGreen15}         {rgb} {0.90,0.10,0}
\definecolor{RedGreen10}         {rgb} {0.95,0.05,0}
\definecolor{RedGreen5}          {rgb} {1,0,0}
\definecolor{ColdHot5}           {rgb} {0,0.08,1}
\definecolor{ColdHot10}          {rgb} {0,0.29,1}
\definecolor{ColdHot15}          {rgb} {0,0.49,1}
\definecolor{ColdHot20}          {rgb} {0,0.70,1}
\definecolor{ColdHot25}          {rgb} {0,0.90,1}
\definecolor{ColdHot30}          {rgb} {0,1,0.87}
\definecolor{ColdHot35}          {rgb} {0,1,0.68}
\definecolor{ColdHot40}          {rgb} {0,1,0.46}
\definecolor{ColdHot45}          {rgb} {0,1,0.25}
\definecolor{ColdHot50}          {rgb} {0,1,0.04}
\definecolor{ColdHot55}          {rgb} {0.16,1,0}
\definecolor{ColdHot60}          {rgb} {0.35,1,0}
\definecolor{ColdHot65}          {rgb} {0.56,1,0}
\definecolor{ColdHot70}          {rgb} {0.79,1,0}
\definecolor{ColdHot75}          {rgb} {0.98,1,0}
\definecolor{ColdHot80}          {rgb} {1,0.82,0}
\definecolor{ColdHot85}          {rgb} {1,0.60,0}
\definecolor{ColdHot90}          {rgb} {1,0.40,0}
\definecolor{ColdHot95}          {rgb} {1,0.20,0}
\definecolor{ColdHot100}         {rgb} {0.91,0,0}
\definecolor{HotCold100}         {rgb} {0,0.08,1}
\definecolor{HotCold95}          {rgb} {0,0.29,1}
\definecolor{HotCold90}          {rgb} {0,0.49,1}
\definecolor{HotCold85}          {rgb} {0,0.70,1}
\definecolor{HotCold80}          {rgb} {0,0.90,1}
\definecolor{HotCold75}          {rgb} {0,1,0.87}
\definecolor{HotCold70}          {rgb} {0,1,0.68}
\definecolor{HotCold65}          {rgb} {0,1,0.46}
\definecolor{HotCold60}          {rgb} {0,1,0.25}
\definecolor{HotCold55}          {rgb} {0,1,0.04}
\definecolor{HotCold50}          {rgb} {0.16,1,0}
\definecolor{HotCold45}          {rgb} {0.35,1,0}
\definecolor{HotCold40}          {rgb} {0.56,1,0}
\definecolor{HotCold35}          {rgb} {0.79,1,0}
\definecolor{HotCold30}          {rgb} {0.98,1,0}
\definecolor{HotCold25}          {rgb} {1,0.82,0}
\definecolor{HotCold20}          {rgb} {1,0.60,0}
\definecolor{HotCold15}          {rgb} {1,0.40,0}
\definecolor{HotCold10}          {rgb} {1,0.20,0}
\definecolor{HotCold5}           {rgb} {0.91,0,0}
\expandafter\def\csname BlueRed5\endcsname{[0.15,0.17,0.55]}
\expandafter\def\csname BlueRed10\endcsname{[0.20,0.23,0.57]}
\expandafter\def\csname BlueRed15\endcsname{[0.24,0.29,0.60]}
\expandafter\def\csname BlueRed20\endcsname{[0.33,0.35,0.64]}
\expandafter\def\csname BlueRed25\endcsname{[0.43,0.43,0.68]}
\expandafter\def\csname BlueRed30\endcsname{[0.52,0.52,0.73]}
\expandafter\def\csname BlueRed35\endcsname{[0.60,0.60,0.78]}
\expandafter\def\csname BlueRed40\endcsname{[0.70,0.70,0.84]}
\expandafter\def\csname BlueRed45\endcsname{[0.80,0.80,0.85]}
\expandafter\def\csname BlueRed50\endcsname{[0.86,0.82,0.82]}
\expandafter\def\csname BlueRed55\endcsname{[0.87,0.73,0.73]}
\expandafter\def\csname BlueRed60\endcsname{[0.89,0.64,0.64]}
\expandafter\def\csname BlueRed65\endcsname{[0.90,0.55,0.55]}
\expandafter\def\csname BlueRed70\endcsname{[0.91,0.47,0.46]}
\expandafter\def\csname BlueRed75\endcsname{[0.91,0.39,0.37]}
\expandafter\def\csname BlueRed80\endcsname{[0.90,0.33,0.28]}
\expandafter\def\csname BlueRed85\endcsname{[0.89,0.25,0.20]}
\expandafter\def\csname BlueRed90\endcsname{[0.88,0.23,0.14]}
\expandafter\def\csname BlueRed95\endcsname{[0.87,0.21,0.09]}
\expandafter\def\csname BlueRed100\endcsname{[0.87,0.16,0.04]}
\expandafter\def\csname RedBlue100\endcsname{[0.15,0.17,0.55]}
\expandafter\def\csname RedBlue95\endcsname{[0.15,0.17,0.55]}
\expandafter\def\csname RedBlue90\endcsname{[0.20,0.23,0.57]}
\expandafter\def\csname RedBlue85\endcsname{[0.24,0.29,0.60]}
\expandafter\def\csname RedBlue80\endcsname{[0.33,0.35,0.64]}
\expandafter\def\csname RedBlue75\endcsname{[0.43,0.43,0.68]}
\expandafter\def\csname RedBlue70\endcsname{[0.52,0.52,0.73]}
\expandafter\def\csname RedBlue65\endcsname{[0.60,0.60,0.78]}
\expandafter\def\csname RedBlue60\endcsname{[0.70,0.70,0.84]}
\expandafter\def\csname RedBlue55\endcsname{[0.80,0.80,0.85]}
\expandafter\def\csname RedBlue50\endcsname{[0.86,0.82,0.82]}
\expandafter\def\csname RedBlue45\endcsname{[0.87,0.73,0.73]}
\expandafter\def\csname RedBlue40\endcsname{[0.89,0.64,0.64]}
\expandafter\def\csname RedBlue35\endcsname{[0.90,0.55,0.55]}
\expandafter\def\csname RedBlue30\endcsname{[0.91,0.47,0.46]}
\expandafter\def\csname RedBlue25\endcsname{[0.91,0.39,0.37]}
\expandafter\def\csname RedBlue20\endcsname{[0.90,0.33,0.28]}
\expandafter\def\csname RedBlue15\endcsname{[0.89,0.25,0.20]}
\expandafter\def\csname RedBlue10\endcsname{[0.88,0.23,0.14]}
\expandafter\def\csname RedBlue5\endcsname{[0.87,0.21,0.09]}
\expandafter\def\csname GreenRed5\endcsname{[0,1,0]}
\expandafter\def\csname GreenRed10\endcsname{[0.05,0.95,0]}
\expandafter\def\csname GreenRed15\endcsname{[0.10,0.90,0]}
\expandafter\def\csname GreenRed20\endcsname{[0.15,0.85,0]}
\expandafter\def\csname GreenRed25\endcsname{[0.20,0.80,0]}
\expandafter\def\csname GreenRed30\endcsname{[0.25,0.75,0]}
\expandafter\def\csname GreenRed35\endcsname{[0.30,0.70,0]}
\expandafter\def\csname GreenRed40\endcsname{[0.35,0.65,0]}
\expandafter\def\csname GreenRed45\endcsname{[0.40,0.60,0]}
\expandafter\def\csname GreenRed50\endcsname{[0.45,0.55,0]}
\expandafter\def\csname GreenRed55\endcsname{[0.50,0.50,0]}
\expandafter\def\csname GreenRed60\endcsname{[0.55,0.45,0]}
\expandafter\def\csname GreenRed65\endcsname{[0.60,0.40,0]}
\expandafter\def\csname GreenRed70\endcsname{[0.65,0.35,0]}
\expandafter\def\csname GreenRed75\endcsname{[0.70,0.30,0]}
\expandafter\def\csname GreenRed80\endcsname{[0.75,0.25,0]}
\expandafter\def\csname GreenRed85\endcsname{[0.80,0.20,0]}
\expandafter\def\csname GreenRed90\endcsname{[0.85,0.15,0]}
\expandafter\def\csname GreenRed95\endcsname{[0.90,0.10,0]}
\expandafter\def\csname GreenRed100\endcsname{[0.95,0.05,0]}
\expandafter\def\csname RedGreen100\endcsname{[0.05,0.95,0]}
\expandafter\def\csname RedGreen95\endcsname{[0.10,0.90,0]}
\expandafter\def\csname RedGreen90\endcsname{[0.15,0.85,0]}
\expandafter\def\csname RedGreen85\endcsname{[0.20,0.80,0]}
\expandafter\def\csname RedGreen80\endcsname{[0.25,0.75,0]}
\expandafter\def\csname RedGreen75\endcsname{[0.30,0.70,0]}
\expandafter\def\csname RedGreen70\endcsname{[0.35,0.65,0]}
\expandafter\def\csname RedGreen65\endcsname{[0.40,0.60,0]}
\expandafter\def\csname RedGreen60\endcsname{[0.45,0.55,0]}
\expandafter\def\csname RedGreen55\endcsname{[0.50,0.50,0]}
\expandafter\def\csname RedGreen50\endcsname{[0.55,0.45,0]}
\expandafter\def\csname RedGreen45\endcsname{[0.60,0.40,0]}
\expandafter\def\csname RedGreen40\endcsname{[0.65,0.35,0]}
\expandafter\def\csname RedGreen35\endcsname{[0.70,0.30,0]}
\expandafter\def\csname RedGreen30\endcsname{[0.75,0.25,0]}
\expandafter\def\csname RedGreen25\endcsname{[0.80,0.20,0]}
\expandafter\def\csname RedGreen20\endcsname{[0.85,0.15,0]}
\expandafter\def\csname RedGreen15\endcsname{[0.90,0.10,0]}
\expandafter\def\csname RedGreen10\endcsname{[0.95,0.05,0]}
\expandafter\def\csname RedGreen5\endcsname{[1,0,0]}
\expandafter\def\csname ColdHot5\endcsname{[0,0.08,1]}
\expandafter\def\csname ColdHot10\endcsname{[0,0.29,1]}
\expandafter\def\csname ColdHot15\endcsname{[0,0.49,1]}
\expandafter\def\csname ColdHot20\endcsname{[0,0.70,1]}
\expandafter\def\csname ColdHot25\endcsname{[0,0.90,1]}
\expandafter\def\csname ColdHot30\endcsname{[0,1,0.87]}
\expandafter\def\csname ColdHot35\endcsname{[0,1,0.68]}
\expandafter\def\csname ColdHot40\endcsname{[0,1,0.46]}
\expandafter\def\csname ColdHot45\endcsname{[0,1,0.25]}
\expandafter\def\csname ColdHot50\endcsname{[0,1,0.04]}
\expandafter\def\csname ColdHot55\endcsname{[0.16,1,0]}
\expandafter\def\csname ColdHot60\endcsname{[0.35,1,0]}
\expandafter\def\csname ColdHot65\endcsname{[0.56,1,0]}
\expandafter\def\csname ColdHot70\endcsname{[0.79,1,0]}
\expandafter\def\csname ColdHot75\endcsname{[0.98,1,0]}
\expandafter\def\csname ColdHot80\endcsname{[1,0.82,0]}
\expandafter\def\csname ColdHot85\endcsname{[1,0.60,0]}
\expandafter\def\csname ColdHot90\endcsname{[1,0.40,0]}
\expandafter\def\csname ColdHot95\endcsname{[1,0.20,0]}
\expandafter\def\csname ColdHot100\endcsname{[0.91,0,0]}
\expandafter\def\csname HotCold100\endcsname{[0,0.08,1]}
\expandafter\def\csname HotCold95\endcsname{[0,0.29,1]}
\expandafter\def\csname HotCold90\endcsname{[0,0.49,1]}
\expandafter\def\csname HotCold85\endcsname{[0,0.70,1]}
\expandafter\def\csname HotCold80\endcsname{[0,0.90,1]}
\expandafter\def\csname HotCold75\endcsname{[0,1,0.87]}
\expandafter\def\csname HotCold70\endcsname{[0,1,0.68]}
\expandafter\def\csname HotCold65\endcsname{[0,1,0.46]}
\expandafter\def\csname HotCold60\endcsname{[0,1,0.25]}
\expandafter\def\csname HotCold55\endcsname{[0,1,0.04]}
\expandafter\def\csname HotCold50\endcsname{[0.16,1,0]}
\expandafter\def\csname HotCold45\endcsname{[0.35,1,0]}
\expandafter\def\csname HotCold40\endcsname{[0.56,1,0]}
\expandafter\def\csname HotCold35\endcsname{[0.79,1,0]}
\expandafter\def\csname HotCold30\endcsname{[0.98,1,0]}
\expandafter\def\csname HotCold25\endcsname{[1,0.82,0]}
\expandafter\def\csname HotCold20\endcsname{[1,0.60,0]}
\expandafter\def\csname HotCold15\endcsname{[1,0.40,0]}
\expandafter\def\csname HotCold10\endcsname{[1,0.20,0]}
\expandafter\def\csname HotCold5\endcsname{[0.91,0,0]}


\def\make@lower{%
\if\first@ A\xdef\first@{a}\else \if\first@ B\xdef\first@{b}\else
\if\first@ C\xdef\first@{c}\else \if\first@ D\xdef\first@{d}\else
\if\first@ E\xdef\first@{e}\else \if\first@ F\xdef\first@{f}\else
\if\first@ G\xdef\first@{g}\else \if\first@ H\xdef\first@{h}\else
\if\first@ I\xdef\first@{i}\else \if\first@ J\xdef\first@{j}\else
\if\first@ K\xdef\first@{k}\else \if\first@ L\xdef\first@{l}\else
\if\first@ M\xdef\first@{m}\else \if\first@ N\xdef\first@{n}\else
\if\first@ O\xdef\first@{o}\else \if\first@ P\xdef\first@{p}\else
\if\first@ Q\xdef\first@{q}\else \if\first@ R\xdef\first@{r}\else
\if\first@ S\xdef\first@{s}\else \if\first@ T\xdef\first@{t}\else
\if\first@ U\xdef\first@{u}\else \if\first@ V\xdef\first@{v}\else
\if\first@ W\xdef\first@{w}\else \if\first@ X\xdef\first@{x}\else
\if\first@ Y\xdef\first@{y}\else \if\first@ Z\xdef\first@{z}\else
\fi\fi\fi\fi\fi\fi\fi\fi\fi\fi\fi\fi\fi\fi\fi\fi\fi\fi\fi\fi\fi\fi\fi\fi\fi\fi}

\def\make@upper{%
\if\first@ a\xdef\first@{A}\else \if\first@ b\xdef\first@{B}\else
\if\first@ c\xdef\first@{C}\else \if\first@ d\xdef\first@{D}\else
\if\first@ e\xdef\first@{E}\else \if\first@ f\xdef\first@{F}\else
\if\first@ g\xdef\first@{G}\else \if\first@ h\xdef\first@{H}\else
\if\first@ i\xdef\first@{I}\else \if\first@ j\xdef\first@{J}\else
\if\first@ k\xdef\first@{K}\else \if\first@ l\xdef\first@{L}\else
\if\first@ m\xdef\first@{M}\else \if\first@ n\xdef\first@{N}\else
\if\first@ o\xdef\first@{O}\else \if\first@ p\xdef\first@{P}\else
\if\first@ q\xdef\first@{Q}\else \if\first@ r\xdef\first@{R}\else
\if\first@ s\xdef\first@{S}\else \if\first@ t\xdef\first@{T}\else
\if\first@ u\xdef\first@{U}\else \if\first@ v\xdef\first@{V}\else
\if\first@ w\xdef\first@{W}\else \if\first@ x\xdef\first@{X}\else
\if\first@ y\xdef\first@{Y}\else \if\first@ z\xdef\first@{Z}\else
\fi\fi\fi\fi\fi\fi\fi\fi\fi\fi\fi\fi\fi\fi\fi\fi\fi\fi\fi\fi\fi\fi\fi\fi\fi\fi}

\def\type@get#1 MSF: #2 Type: #3 #4@{\seqtype{#3}}
\def\inf@@get#1 #2 #3 #4 #5 #6@{%
   \def\first@{#1}
   \def\second@{#2}
   \def\third@{#3}
   \xdef\fourth@{#4 @}
   \expandafter\check@letter\fourth@
   \ifnumber 
      \def\fourth@{#4} 
   \else 
      \xdef\fourth@{#5 @}
      \expandafter\check@letter\fourth@
      \ifnumber 
         \def\fourth@{#5} 
      \else
         \def\fourth@{99999999}
      \fi
   \fi
   \def\fifth@{#5}
   \def\last@{#6}}
\def\check@char#1{\letterfalse \ifnum\catcode`#1=11 \lettertrue \fi
                  \numberfalse \ifnum\catcode`#1=12 \numbertrue \fi
                  \xdef\code@num{\the\catcode`#1}
                  \xdef\char@num{`#1}}
\def\check@letter#1#2@{\letterfalse\ifnum\catcode`#1=11 \lettertrue\fi%
                       \numberfalse\ifnum\catcode`#1=12 \numbertrue\fi}
\def\seq@get#1 #2@{\def\first@{#1} \def\seq@line{#2}}
\def\res@get#1#2@{\def\first@{#1} \def\seq@line{#2&@}}
\def\tot@get#1#2@{%
   \xdef\first@{#1}
   \expandafter\xdef\csname seq\the\triple@count\endcsname{#2&@}}
\def\dis@get#1#2@{%
   \xdef\first@{#1}
   \expandafter\xdef\csname res\the\triple@count\endcsname{#1}
   \expandafter\xdef\csname seq\the\triple@count\endcsname{#2&@}}
\def\firstchar@get#1#2@{\def\first@{#1}\def\third@{#2@}}
\def\residue@get#1#2@{\xdef\first@{#1}
                      \xdef\char@num{`#1}
                      \expandafter\xdef\csname res\the\loopcount\endcsname{\first@}
                      \expandafter\xdef\csname sequence\the\loopcount\endcsname{#2@}}
\def\remove@fromseq#1@{\expandafter\xdef\csname sequence\the\loopcount\endcsname{#1}}
\def\get@temp#1@{\xdef\temp@{#1}}
\def\TC@get#1#2@{\xdef\char@num{`#1}
                 \ifnum\char@num=45 \xdef\TC@first@{*}
                 \else
                       \ifnum\char@num>57 \xdef\TC@first@{*}
                       \else
                          \xdef\TC@first@{#1}
                       \fi
                 \fi
                 \expandafter\xdef\csname TC@num\the\loopcount\endcsname{\TC@first@}
                 \expandafter\xdef\csname T@coffee\the\loopcount\endcsname{#2@}}
\def\sublogo@get#1 @{\xdef\first@{#1}}
\def\re@write#1,#2@{%
   \xdef\third@{#2,@}
   \xdef\first@{\csname hide@@@seq#1\endcsname}
   \ifx\first@\last@
   \else
      \xdef\first@{#1&}
      \ifx\first@\ampers@nd \else \advance\innerloopcount by 1 \fi
      \if\second@ e \xdef\second@{\csname @rd#1\endcsname}
      \else \xdef\second@{\second@,\csname @rd#1\endcsname}\fi
   \fi
   }
\def\order@set#1,#2@{%
   \xdef\second@{#2,@}
      \expandafter\xdef\csname @rd#1\endcsname{\the\loopcount}
      \expandafter\xdef\csname res@count\the\loopcount\endcsname{%
                    \csname pos#1\endcsname}
      \expandafter\xdef\csname hide@seq\the\loopcount\endcsname{%
                    \csname hide@@seq#1\endcsname}
      \expandafter\xdef\csname hide@name\the\loopcount\endcsname{%
                    \csname hide@@name#1\endcsname}
      \expandafter\xdef\csname hide@number\the\loopcount\endcsname{%
                    \csname hide@@number#1\endcsname}
      \expandafter\xdef\csname seq@start\the\loopcount\endcsname{%
                    \csname seq@@start#1\endcsname}
      \expandafter\xdef\csname seq@len\the\loopcount\endcsname{%
                    \csname seq@@len#1\endcsname}
      \expandafter\xdef\csname seqname\the\loopcount\endcsname{%
                    \csname seq@name#1\endcsname}
      \expandafter\xdef\csname newseqname\the\loopcount\endcsname{%
                    \csname newseq@name#1\endcsname}
      \expandafter\xdef\csname seq@gap\the\loopcount\endcsname{%
                    \csname seq@@gap#1\endcsname}
      \expandafter\xdef\csname name@col\the\loopcount\endcsname{%
                    \csname name@@col#1\endcsname}
      \expandafter\xdef\csname number@col\the\loopcount\endcsname{%
                    \csname number@@col#1\endcsname}
      \expandafter\xdef\csname stack@reg\the\loopcount\endcsname{%
                    \csname stack@@reg#1\endcsname}
      \expandafter\xdef\csname stack@tintreg\the\loopcount\endcsname{%
                    \csname stack@@tintreg#1\endcsname}
      \expandafter\xdef\csname stack@emphreg\the\loopcount\endcsname{%
                    \csname stack@@emphreg#1\endcsname}
      \expandafter\xdef\csname stack@lowerreg\the\loopcount\endcsname{%
                    \csname stack@@lowerreg#1\endcsname}
      \expandafter\xdef\csname stack@framereg\the\loopcount\endcsname{%
                    \csname stack@@framereg#1\endcsname}
      \expandafter\xdef\csname stack@shadingreg\the\loopcount\endcsname{%
                    \csname stack@@shadingreg#1\endcsname}
      \expandafter\xdef\csname stack@top\the\loopcount\endcsname{%
                    \csname stack@@top#1\endcsname}
      \expandafter\xdef\csname stack@ttop\the\loopcount\endcsname{%
                    \csname stack@@ttop#1\endcsname}
      \expandafter\xdef\csname stack@tttop\the\loopcount\endcsname{%
                    \csname stack@@tttop#1\endcsname}
      \expandafter\xdef\csname stack@ttttop\the\loopcount\endcsname{%
                    \csname stack@@ttttop#1\endcsname}
      \expandafter\xdef\csname stack@bottom\the\loopcount\endcsname{%
                    \csname stack@@bottom#1\endcsname}
      \expandafter\xdef\csname stack@bbottom\the\loopcount\endcsname{%
                    \csname stack@@bbottom#1\endcsname}
      \expandafter\xdef\csname stack@bbbottom\the\loopcount\endcsname{%
                    \csname stack@@bbbottom#1\endcsname}
      \expandafter\xdef\csname stack@bbbbottom\the\loopcount\endcsname{%
                    \csname stack@@bbbbottom#1\endcsname}
}
\def\reorder@seqs#1{%
   \loopcount=0
   \loop
      \advance\loopcount by 1
         \expandafter\xdef\csname pos\the\loopcount\endcsname{%
                 \csname res@count\the\loopcount\endcsname}
         \expandafter\xdef\csname hide@@seq\the\loopcount\endcsname{%
            \csname hide@seq\the\loopcount\endcsname}
         \expandafter\xdef\csname hide@@name\the\loopcount\endcsname{%
            \csname hide@name\the\loopcount\endcsname}
         \expandafter\xdef\csname hide@@number\the\loopcount\endcsname{%
            \csname hide@number\the\loopcount\endcsname}
         \expandafter\xdef\csname seq@@start\the\loopcount\endcsname{%
            \csname seq@start\the\loopcount\endcsname}
         \expandafter\xdef\csname seq@@len\the\loopcount\endcsname{%
            \csname seq@len\the\loopcount\endcsname}
         \expandafter\xdef\csname seq@name\the\loopcount\endcsname{%
            \csname seqname\the\loopcount\endcsname}
         \expandafter\xdef\csname newseq@name\the\loopcount\endcsname{%
            \csname newseqname\the\loopcount\endcsname}
         \expandafter\xdef\csname seq@@gap\the\loopcount\endcsname{%
            \csname seq@gap\the\loopcount\endcsname}
         \expandafter\xdef\csname name@@col\the\loopcount\endcsname{%
            \csname name@col\the\loopcount\endcsname}
         \expandafter\xdef\csname number@@col\the\loopcount\endcsname{%
            \csname number@col\the\loopcount\endcsname}
         \expandafter\xdef\csname stack@@reg\the\loopcount\endcsname{%
            \csname stack@reg\the\loopcount\endcsname}
         \expandafter\xdef\csname stack@@tintreg\the\loopcount\endcsname{%
            \csname stack@tintreg\the\loopcount\endcsname}
         \expandafter\xdef\csname stack@@emphreg\the\loopcount\endcsname{%
            \csname stack@emphreg\the\loopcount\endcsname}
         \expandafter\xdef\csname stack@@lowerreg\the\loopcount\endcsname{%
            \csname stack@lowerreg\the\loopcount\endcsname}
         \expandafter\xdef\csname stack@@framereg\the\loopcount\endcsname{%
            \csname stack@framereg\the\loopcount\endcsname}
         \expandafter\xdef\csname stack@@shadingreg\the\loopcount\endcsname{%
            \csname stack@shadingreg\the\loopcount\endcsname}
         \expandafter\xdef\csname stack@@top\the\loopcount\endcsname{%
            \csname stack@top\the\loopcount\endcsname}
         \expandafter\xdef\csname stack@@ttop\the\loopcount\endcsname{%
            \csname stack@ttop\the\loopcount\endcsname}
         \expandafter\xdef\csname stack@@tttop\the\loopcount\endcsname{%
            \csname stack@tttop\the\loopcount\endcsname}
         \expandafter\xdef\csname stack@@ttttop\the\loopcount\endcsname{%
            \csname stack@ttttop\the\loopcount\endcsname}
         \expandafter\xdef\csname stack@@bottom\the\loopcount\endcsname{%
            \csname stack@bottom\the\loopcount\endcsname}
         \expandafter\xdef\csname stack@@bbottom\the\loopcount\endcsname{%
            \csname stack@bbottom\the\loopcount\endcsname}
         \expandafter\xdef\csname stack@@bbbottom\the\loopcount\endcsname{%
            \csname stack@bbbottom\the\loopcount\endcsname}
         \expandafter\xdef\csname stack@@bbbbottom\the\loopcount\endcsname{%
            \csname stack@bbbbottom\the\loopcount\endcsname}
   \ifnum\loopcount<\killseq@count \repeat
   \xdef\third@{#1}
   \loopcount=0 \innerloopcount=0 \xdef\last@{kill} \xdef\second@{e}
   \loop
      \advance\loopcount by 1
      \expandafter\re@write\third@
   \ifnum\loopcount<\seq@count \repeat
   \ifnum\innerloopcount<\killseq@count
       \@latex@error{Not enough sequences specified in `orderseqs'
       (\the\innerloopcount/\the\killseq@count)}\@ehc
   \fi
   \loopcount=0 \xdef\second@{\second@,@}
   \loop
      \advance\loopcount by 1
      \expandafter\order@set\second@
   \ifnum\loopcount<\killseq@count \repeat
   \ifnum\cons@num>0 \xdef\cons@num{\csname @rd\cons@num\endcsname} \fi
   \ifnum\rule@num>0 \xdef\rule@num{\csname @rd\rule@num\endcsname} \fi
   \ifnum\divref@>0  \xdef\divref@{\csname @rd\divref@\endcsname} \fi
   \loopcount=0
   \loop
      \advance\loopcount by 1
      \expandafter\xdef\csname pos\the\loopcount\endcsname{0}
   \ifnum\loopcount<\killseq@count \repeat
}
\def\group@get#1,#2@{%
   \def\group@set{\expandafter\residue@get\second@
                 \ifnum\char@num>96 \make@upper \fi
                 \ifx\first@\ampers@nd
                 \else \expandafter\xdef\csname \prfx grp\first@\endcsname{\the\loopcount}
                 \xdef\second@{\csname sequence\the\loopcount\endcsname} \group@set
                 \fi}
   \xdef\second@{#1 &@} \xdef\third@{#2&,@} \group@set}
\def\get@item#1,#2@{\xdef\first@{#2@}\xdef\first@@{#2}\xdef\fourth@{#1}}
\def\get@first@@#1-#2@{\xdef\first@@{#1}}
\def\get@digit#1,#2@{%
   \def\check@series##1-##2##3@{%
                    \xdef\first@@{##2}\xdef\fourth@{##1}\xdef\fourth@@{##3}}
   \xdef\first@{#2@}
   \xdef\fourth@{#1-&@}
   \expandafter\check@series\fourth@
   \ifx\first@@\ampers@nd
   \else
      \xdef\first@@{\first@@\fourth@@-&@}
      \expandafter\get@first@@\first@@
      \loopcount=\fourth@ 
      \ifnum\first@@>\fourth@
         \advance\loopcount by 1
         \xdef\first@{\the\loopcount-\first@@,#2@}
      \else
         \ifnum\first@@<\fourth@
            \advance\loopcount by -1
            \xdef\first@{\the\loopcount-\first@@,#2@}
         \fi
      \fi
   \fi
}
\def\donot@shade{%
   \expandafter\check@letter\first@
   \ifletter
      \expandafter\get@item\first@
      \xdef\first@{\fourth@}
      \get@name@number
      \xdef\third@{\csname hide@seq\first@\endcsname} \xdef\second@{kill}
      \ifx\third@\second@
      \else
         \expandafter\xdef\csname hide@seq\first@\endcsname{noshade}
      \fi
      \xdef\first@{\first@@ @}
      \donot@shade
   \else
      \expandafter\get@digit\first@
      \ifx\fourth@\ampers@nd
      \else
         \xdef\third@{\csname hide@seq\fourth@\endcsname} \xdef\second@{kill}
         \ifx\third@\second@
         \else
            \expandafter\xdef\csname hide@seq\fourth@\endcsname{noshade}
         \fi
        \donot@shade
      \fi
   \fi}
\def\clear@res@nums#1{%
   \temp@count=65
   \loop
      \xdef\first@@{\csname ch@r@\the\temp@count\endcsname}
      \expandafter\xdef\csname res@num\first@@ #1\endcsname{0}
      \advance\temp@count by 1
   \ifnum\temp@count>90\else\repeat
   \expandafter\xdef\csname res@num\d@t #1\endcsname{0}   
   \expandafter\xdef\csname res@num\questi@n #1\endcsname{0}   
}
\def\setsubfamily@{%
   \expandafter\check@letter\first@
   \ifletter
      \expandafter\get@item\first@
      \xdef\first@{\fourth@}
      \get@name@number
      \advance\res@count by 1 
      \expandafter\xdef\csname subfamily@num\first@\endcsname{\subfamily@count}
      \xdef\first@{\first@@ @}
      \setsubfamily@
   \else
      \expandafter\get@digit\first@
      \ifx\fourth@\ampers@nd
      \else
         \advance\res@count by 1 
         \expandafter\xdef\csname subfamily@num\fourth@\endcsname{\subfamily@count}
         \setsubfamily@
      \fi
   \fi
}
\def\hideseq@{%
   \expandafter\check@letter\first@
   \ifletter
      \expandafter\get@item\first@
      \xdef\first@{\fourth@}
      \get@name@number
      \xdef\third@{\csname hide@seq\first@\endcsname} \xdef\second@{kill}
      \ifx\third@\second@
      \else
         \expandafter\xdef\csname hide@seq\first@\endcsname{true}
      \fi
      \xdef\first@{\first@@ @}
      \hideseq@
   \else
      \expandafter\get@digit\first@
      \ifx\fourth@\ampers@nd
      \else
         \xdef\third@{\csname hide@seq\fourth@\endcsname} \xdef\second@{kill}
         \ifx\third@\second@
         \else
            \expandafter\xdef\csname hide@seq\fourth@\endcsname{true}
         \fi
        \hideseq@
      \fi
   \fi}
\def\hidename@{%
   \expandafter\check@letter\first@
   \ifletter
      \expandafter\get@item\first@
      \xdef\first@{\fourth@}
      \get@name@number
      \expandafter\xdef\csname hide@name\first@\endcsname{yes}
      \xdef\first@{\first@@ @}
      \hidename@
   \else
      \expandafter\get@digit\first@
      \ifx\fourth@\ampers@nd
      \else
         \expandafter\xdef\csname hide@name\fourth@\endcsname{yes}
         \hidename@
      \fi
   \fi}
\def\hidenumber@{%
   \expandafter\check@letter\first@
   \ifletter
      \expandafter\get@item\first@
      \xdef\first@{\fourth@}
      \get@name@number
      \expandafter\xdef\csname hide@number\first@\endcsname{yes}
      \xdef\first@{\first@@ @}
      \hidenumber@
   \else
      \expandafter\get@digit\first@
      \ifx\fourth@\ampers@nd
      \else
         \expandafter\xdef\csname hide@number\fourth@\endcsname{yes}
         \hidenumber@
      \fi
   \fi}
\def\namecolor@{%
   \expandafter\check@letter\first@
   \ifletter
      \expandafter\get@item\first@
      \xdef\first@{\fourth@}
      \get@name@number
      \expandafter\xdef\csname name@col\first@\endcsname{\third@}
      \xdef\first@{\first@@ @}
      \namecolor@
   \else
      \expandafter\get@digit\first@
      \ifx\fourth@\ampers@nd
      \else
         \expandafter\xdef\csname name@col\fourth@\endcsname{\third@}
         \namecolor@
      \fi
   \fi}
\def\numbercolor@{%
   \expandafter\check@letter\first@
   \ifletter
      \expandafter\get@item\first@
      \xdef\first@{\fourth@}
      \get@name@number
      \expandafter\xdef\csname number@col\first@\endcsname{\third@}
      \xdef\first@{\first@@ @}
      \numbercolor@
   \else
      \expandafter\get@digit\first@
      \ifx\fourth@\ampers@nd
      \else
         \expandafter\xdef\csname number@col\fourth@\endcsname{\third@}
         \numbercolor@
      \fi
   \fi}
\def\killseq@{%
   \expandafter\check@letter\first@
   \ifletter
      \expandafter\get@item\first@
      \xdef\first@{\fourth@}
      \get@name@number
      \ifnum\first@>\seq@count
      \else
      \ifnum\killseq@count>1
         \xdef\third@{\csname hide@seq\first@\endcsname} \xdef\second@{kill}
         \ifx\third@\second@
         \else
            \expandafter\xdef\csname hide@seq\first@\endcsname{kill}
            \expandafter\xdef\csname hide@@@seq\first@\endcsname{kill}
            \ifnum\first@=\cons@num \xdef\cons@num{0} \fi
            \advance\killseq@count by -1
            \seq@percent=100 \divide\seq@percent by \killseq@count
         \fi
      \fi\fi
      \xdef\first@{\first@@ @}
      \killseq@
   \else
      \expandafter\get@digit\first@
      \ifx\fourth@\ampers@nd
      \else
      \ifnum\fourth@>\seq@count
      \else
      \ifnum\killseq@count>1
         \xdef\third@{\csname hide@seq\fourth@\endcsname} \xdef\second@{kill}
         \ifx\third@\second@
         \else
            \expandafter\xdef\csname hide@seq\fourth@\endcsname{kill}
            \expandafter\xdef\csname hide@@@seq\fourth@\endcsname{kill}
            \ifnum\fourth@=\cons@num \xdef\cons@num{0} \fi
            \advance\killseq@count by -1
            \seq@percent=100 \divide\seq@percent by \killseq@count
         \fi
      \fi\fi
      \killseq@
      \fi
   \fi}
\def\kill@loop{%
   \advance\innerloopcount by 1
   \xdef\first@{\csname hide@seq\the\innerloopcount\endcsname}
   \ifx\first@\second@ \kill@loop \fi}
\def\kill@seqnow{%
   \xdef\first@{\csname hide@seq\the\loopcount\endcsname} \xdef\second@{kill}
   \innerloopcount=\loopcount
   \ifx\first@\second@
      \kill@loop
      \expandafter\xdef\csname hide@seq\the\loopcount\endcsname{%
            \csname hide@seq\the\innerloopcount\endcsname}
      \expandafter\xdef\csname seqname\the\loopcount\endcsname{%
            \csname seqname\the\innerloopcount\endcsname}
      \expandafter\xdef\csname newseqname\the\loopcount\endcsname{%
            \csname newseqname\the\innerloopcount\endcsname}
      \expandafter\xdef\csname seq@gap\the\loopcount\endcsname{%
            \csname seq@gap\the\innerloopcount\endcsname}
      \expandafter\xdef\csname stack@reg\the\loopcount\endcsname{%
            \csname stack@reg\the\innerloopcount\endcsname}
      \expandafter\xdef\csname stack@tintreg\the\loopcount\endcsname{%
            \csname stack@tintreg\the\innerloopcount\endcsname}
      \expandafter\xdef\csname stack@emphreg\the\loopcount\endcsname{%
            \csname stack@emphreg\the\innerloopcount\endcsname}
      \expandafter\xdef\csname stack@lowerreg\the\loopcount\endcsname{%
            \csname stack@lowerreg\the\innerloopcount\endcsname}
      \expandafter\xdef\csname stack@framereg\the\loopcount\endcsname{%
            \csname stack@framereg\the\innerloopcount\endcsname}
      \expandafter\xdef\csname stack@shadingreg\the\loopcount\endcsname{%
            \csname stack@shadingreg\the\innerloopcount\endcsname}
      \expandafter\xdef\csname stack@top\the\loopcount\endcsname{%
            \csname stack@top\the\innerloopcount\endcsname}
      \expandafter\xdef\csname stack@ttop\the\loopcount\endcsname{%
            \csname stack@ttop\the\innerloopcount\endcsname}
      \expandafter\xdef\csname stack@tttop\the\loopcount\endcsname{%
            \csname stack@tttop\the\innerloopcount\endcsname}
      \expandafter\xdef\csname stack@ttttop\the\loopcount\endcsname{%
            \csname stack@ttttop\the\innerloopcount\endcsname}
      \expandafter\xdef\csname stack@bottom\the\loopcount\endcsname{%
            \csname stack@bottom\the\innerloopcount\endcsname}
      \expandafter\xdef\csname stack@bbottom\the\loopcount\endcsname{%
            \csname stack@bbottom\the\innerloopcount\endcsname}
      \expandafter\xdef\csname stack@bbbottom\the\loopcount\endcsname{%
            \csname stack@bbbottom\the\innerloopcount\endcsname}
      \expandafter\xdef\csname stack@bbbbottom\the\loopcount\endcsname{%
            \csname stack@bbbbottom\the\innerloopcount\endcsname}
      \expandafter\xdef\csname res@count\the\loopcount\endcsname{%
            \csname res@count\the\innerloopcount\endcsname}
      \expandafter\xdef\csname hide@seq\the\innerloopcount\endcsname{kill}
      \ifnum\loopcount=\rule@num \hideruler \fi
   \fi
   \expandafter\xdef\csname @rd\the\innerloopcount\endcsname{\the\loopcount}
   \advance\loopcount by 1
   \ifnum\loopcount>\killseq@count 
      \ifnum\rule@num>0\xdef\rule@num{\csname @rd\rule@num\endcsname}\fi
      \ifnum\cons@num>0\xdef\cons@num{\csname @rd\cons@num\endcsname}\fi
   \else 
      \kill@seqnow
   \fi}
\def\get@sim#1#2@{\xdef\sim@char{#1} \xdef\last@{#2 &@}}
\def\getsim@char{%
   \advance\innerloopcount by 1
   \ifnum\innerloopcount>\m@x
   \else
      \expandafter\get@sim\last@
      \ifx\second@\sim@char \xdef\third@{2} \innerloopcount=\m@x \fi
      \getsim@char
   \fi}
\def\get@count(#1)#2@{\xdef\last@{#2 &@} \xdef\m@x{#1}}
\def\get@nums#1..#2@{\xdef\first@{#1} \xdef\second@{#2}}
\def\func@shading#1{%
   \clearfuncgroups
   \xdef\temp@{#1}
   \xdef\second@{charge}
   \ifx\temp@\second@
      \ifgerm@n
      \funcgroup{sauer ($-$)}{DE}{White}{Red}{upper}{up}
      \funcgroup{basisch ($+$)}{KRH}{White}{Blue}{upper}{up}
      \else
      \ifsp@nish
      \funcgroup{\'acido ($-$)}{DE}{White}{Red}{upper}{up}
      \funcgroup{b\'asico ($+$)}{KRH}{White}{Blue}{upper}{up}
      \else
      \funcgroup{acidic ($-$)}{DE}{White}{Red}{upper}{up}
      \funcgroup{basic ($+$)}{KRH}{White}{Blue}{upper}{up}
      \fi
      \fi
   \else
   \xdef\second@{hydropathy}
   \ifx\temp@\second@
      \ifgerm@n
      \funcgroup{sauer ($-$)}{DE}{White}{Red}{upper}{up}
      \funcgroup{basisch ($+$)}{KRH}{White}{Blue}{upper}{up}
      \funcgroup{polar ungeladen}{YSTGNQC}{Black}{Yellow}{upper}{up}
      \funcgroup{hydrophob unpolar}{AFPMWVIL}{White}{Green}{upper}{up}
      \else
      \ifsp@nish
      \funcgroup{\'acido ($-$)}{DE}{White}{Red}{upper}{up}
      \funcgroup{b\'asico ($+$)}{KRH}{White}{Blue}{upper}{up}
      \funcgroup{polar sin carga}{YSTGNQC}{Black}{Yellow}{upper}{up}
      \funcgroup{hidrof\'obico no polar}{AFPMWVIL}{White}{Green}{upper}{up}
      \else
      \funcgroup{acidic ($-$)}{DE}{White}{Red}{upper}{up}
      \funcgroup{basic ($+$)}{KRH}{White}{Blue}{upper}{up}
      \funcgroup{polar uncharged}{YSTGNQC}{Black}{Yellow}{upper}{up}
      \funcgroup{hydrophobic nonpolar}{AFPMWVIL}{White}{Green}{upper}{up}
      \fi
      \fi
   \else
   \xdef\second@{chemical}
   \ifx\temp@\second@
      \ifgerm@n
      \funcgroup{sauer ($-$)}{DE}{White}{Red}{upper}{up}
      \funcgroup{aliphatisch}{AGVIL}{White}{Black}{upper}{up}
      \funcgroup{aliphatisch (klein)}{AG}{White}{Gray}{upper}{up}
      \funcgroup{Amid}{NQ}{White}{Green}{upper}{up}
      \funcgroup{aromatisch}{FYW}{White}{Brown}{upper}{up}
      \funcgroup{basisch (+)}{KRH}{White}{Blue}{upper}{up}
      \funcgroup{Hydroxyl}{ST}{Black}{Magenta}{upper}{up}
      \funcgroup{Imin}{P}{Black}{Orange}{upper}{up}
      \funcgroup{Schwefel}{CM}{Black}{Yellow}{upper}{up}
      \else
      \ifsp@nish
      \funcgroup{\'acido ($-$)}{DE}{White}{Red}{upper}{up}
      \funcgroup{alif\'atico}{AGVIL}{White}{Black}{upper}{up}
      \funcgroup{alif\'atico (parvo)}{AG}{White}{Gray}{upper}{up}
      \funcgroup{amida}{NQ}{White}{Green}{upper}{up}
      \funcgroup{arom\'atico}{FYW}{White}{Brown}{upper}{up}
      \funcgroup{b\'asico ($+$)}{KRH}{White}{Blue}{upper}{up}
      \funcgroup{hidr\'oxido}{ST}{Black}{Magenta}{upper}{up}
      \funcgroup{imino}{P}{Black}{Orange}{upper}{up}
      \funcgroup{azufre}{CM}{Black}{Yellow}{upper}{up}
      \else
      \funcgroup{acidic ($-$)}{DE}{White}{Red}{upper}{up}
      \funcgroup{aliphatic}{AGVIL}{White}{Black}{upper}{up}
      \funcgroup{aliphatic (small)}{AG}{White}{Gray}{upper}{up}
      \funcgroup{amide}{NQ}{White}{Green}{upper}{up}
      \funcgroup{aromatic}{FYW}{White}{Brown}{upper}{up}
      \funcgroup{basic ($+$)}{KRH}{White}{Blue}{upper}{up}
      \funcgroup{hydroxyl}{ST}{Black}{Magenta}{upper}{up}
      \funcgroup{imino}{P}{Black}{Orange}{upper}{up}
      \funcgroup{sulfur}{CM}{Black}{Yellow}{upper}{up}
      \fi
      \fi
   \else
   \xdef\second@{structure}
   \ifx\temp@\second@
      \ifgerm@n
      \funcgroup{extern}{DEHKNQR}{Black}{Orange}{upper}{up}
      \funcgroup{ambivalent}{ACGPSTWY}{Black}{Yellow}{upper}{up}
      \funcgroup{intern}{FILMV}{White}{Green}{upper}{up}
      \else
      \ifsp@nish
      \funcgroup{externo}{DEHKNQR}{Black}{Orange}{upper}{up}
      \funcgroup{ambivalente}{ACGPSTWY}{Black}{Yellow}{upper}{up}
      \funcgroup{interno}{FILMV}{White}{Green}{upper}{up}
      \else
      \funcgroup{external}{DEHKNQR}{Black}{Orange}{upper}{up}
      \funcgroup{ambivalent}{ACGPSTWY}{Black}{Yellow}{upper}{up}
      \funcgroup{internal}{FILMV}{White}{Green}{upper}{up}
      \fi
      \fi
   \else
   \xdef\second@{standard area}
   \ifx\temp@\second@
      \ifgerm@n
      \funcgroup{\ 88,1 (G); Standard Seitenkettenfl\"ache %
                             (\AA$^2$)}%
                {G}{Black}{BrickRed}{upper}{up}
      \funcgroup{118,2 (A); 129,8 (S)}{AS}{Black}{Orange}{upper}{up}
      \funcgroup{146,1 (C); 146,8 (P)}%
                {CP}{Black}{Yellow}{upper}{up}
      \funcgroup{152,5 (T); 158,7 (D); 164,5 (V); 165,5 (N)}%
                {TDVN}{Black}{YellowGreen}{upper}{up}
      \funcgroup{181,0 (I); 186,2 (E)}{IE}{White}{PineGreen}{upper}{up}
      \funcgroup{193,1 (L); 193,2 (Q); 202,5 (H); 203,3 (M)}%
                {LQHM}{Black}{SkyBlue}{upper}{up}
      \funcgroup{222,8 (F); 225,8 (K)}{FK}{White}{RoyalPurple}{upper}{up}
      \funcgroup{238,8 (Y)}{Y}{White}{RedViolet}{upper}{up}
      \funcgroup{256,0 (R); 266,2 (W)}{RW}{White}{Black}{upper}{up}
      \else
      \ifsp@nish
      \funcgroup{\ 88.1 (G); Superficie est\'andar de la cadena lateral (\AA$^2$)}%
                {G}{Black}{BrickRed}{upper}{up}
      \funcgroup{118.2 (A); 129.8 (S)}{AS}{Black}{Orange}{upper}{up}
      \funcgroup{146.1 (C); 146.8 (P)}%
                {CP}{Black}{Yellow}{upper}{up}
      \funcgroup{152.5 (T); 158.7 (D); 164.5 (V); 165.5 (N)}%
                {TDVN}{Black}{YellowGreen}{upper}{up}
      \funcgroup{181.0 (I); 186.2 (E)}{IE}{White}{PineGreen}{upper}{up}
      \funcgroup{193.1 (L); 193.2 (Q); 202.5 (H); 203.3 (M)}%
                {LQHM}{Black}{SkyBlue}{upper}{up}
      \funcgroup{222.8 (F); 225.8 (K)}{FK}{White}{RoyalPurple}{upper}{up}
      \funcgroup{238.8 (Y)}{Y}{White}{RedViolet}{upper}{up}
      \funcgroup{256.0 (R); 266.2 (W)}{RW}{White}{Black}{upper}{up}
      \else
      \funcgroup{\ 88.1 (G); Standard sidechain area (\AA$^2$)}%
                {G}{Black}{BrickRed}{upper}{up}
      \funcgroup{118.2 (A); 129.8 (S)}{AS}{Black}{Orange}{upper}{up}
      \funcgroup{146.1 (C); 146.8 (P)}%
                {CP}{Black}{Yellow}{upper}{up}
      \funcgroup{152.5 (T); 158.7 (D); 164.5 (V); 165.5 (N)}%
                {TDVN}{Black}{YellowGreen}{upper}{up}
      \funcgroup{181.0 (I); 186.2 (E)}{IE}{White}{PineGreen}{upper}{up}
      \funcgroup{193.1 (L); 193.2 (Q); 202.5 (H); 203.3 (M)}%
                {LQHM}{Black}{SkyBlue}{upper}{up}
      \funcgroup{222.8 (F); 225.8 (K)}{FK}{White}{RoyalPurple}{upper}{up}
      \funcgroup{238.8 (Y)}{Y}{White}{RedViolet}{upper}{up}
      \funcgroup{256.0 (R); 266.2 (W)}{RW}{White}{Black}{upper}{up}
      \fi
      \fi
   \else
   \xdef\second@{accessible area}
   \ifx\temp@\second@
      \ifgerm@n
      \funcgroup{\ 13,9 (C); Zug\"angliche Seitenkettenfl\"ache %
                            (\AA$^2$)}%
                {CIV}{Black}{BrickRed}{upper}{up}
      \funcgroup{\ 23,0 (I); 23,5 (V); 25,2 (G)}%
                {IVG}{Black}{Orange}{upper}{up}
      \funcgroup{\ 28,7 (F); 29,0 (L); 30,5 (M); 31,5 (A)}%
                {FLMA}{Black}{Yellow}{upper}{up}
      \funcgroup{\ 41,7 (W); 44,2 (S); 46,0 (T); 46,7 (H)}%
                {WSTH}{Black}{YellowGreen}{upper}{up}
      \funcgroup{\ 53,7 (P)}{P}{White}{PineGreen}{upper}{up}
      \funcgroup{\ 59,1 (Y); 60,9 (D); 62,2 (N)}%
                {YDN}{Black}{SkyBlue}{upper}{up}
      \funcgroup{\ 72,3 (E); 74,0 (Q)}{EQ}{White}{RoyalPurple}{upper}{up}
      \funcgroup{\ 93,8 (R)}{R}{White}{RedViolet}{upper}{up}
      \funcgroup{110,3 (K)}{K}{White}{Black}{upper}{up}
      \else
      \ifsp@nish
      \funcgroup{\ 13.9 (C); Superficie accesible de la cadena lateral (\AA$^2$)}%
                {CIV}{Black}{BrickRed}{upper}{up}
      \funcgroup{\ 23.0 (I); 23.5 (V); 25.2 (G)}%
                {IVG}{Black}{Orange}{upper}{up}
      \funcgroup{\ 28.7 (F); 29.0 (L); 30.5 (M); 31.5 (A)}%
                {FLMA}{Black}{Yellow}{upper}{up}
      \funcgroup{\ 41.7 (W); 44.2 (S); 46.0 (T); 46.7 (H)}%
                {WSTH}{Black}{YellowGreen}{upper}{up}
      \funcgroup{\ 53.7 (P)}{P}{White}{PineGreen}{upper}{up}
      \funcgroup{\ 59.1 (Y); 60.9 (D); 62.2 (N)}%
                {YDN}{Black}{SkyBlue}{upper}{up}
      \funcgroup{\ 72.3 (E); 74.0 (Q)}{EQ}{White}{RoyalPurple}{upper}{up}
      \funcgroup{\ 93.8 (R)}{R}{White}{RedViolet}{upper}{up}
      \funcgroup{110.3 (K)}{K}{White}{Black}{upper}{up}
      \else
      \funcgroup{\ 13.9 (C); Accessible sidechain area (\AA$^2$)}%
                {CIV}{Black}{BrickRed}{upper}{up}
      \funcgroup{\ 23.0 (I); 23.5 (V); 25.2 (G)}%
                {IVG}{Black}{Orange}{upper}{up}
      \funcgroup{\ 28.7 (F); 29.0 (L); 30.5 (M); 31.5 (A)}%
                {FLMA}{Black}{Yellow}{upper}{up}
      \funcgroup{\ 41.7 (W); 44.2 (S); 46.0 (T); 46.7 (H)}%
                {WSTH}{Black}{YellowGreen}{upper}{up}
      \funcgroup{\ 53.7 (P)}{P}{White}{PineGreen}{upper}{up}
      \funcgroup{\ 59.1 (Y); 60.9 (D); 62.2 (N)}%
                {YDN}{Black}{SkyBlue}{upper}{up}
      \funcgroup{\ 72.3 (E); 74.0 (Q)}{EQ}{White}{RoyalPurple}{upper}{up}
      \funcgroup{\ 93.8 (R)}{R}{White}{RedViolet}{upper}{up}
      \funcgroup{110.3 (K)}{K}{White}{Black}{upper}{up}
      \fi
      \fi
   \else
   \xdef\second@{rasmol}
   \ifx\temp@\second@
      \funcgroup{Asp, Glu}{DE}{Red}{White}{upper}{up}
      \funcgroup{Arg, Lys, His}{KRH}{Blue}{White}{upper}{up}
      \funcgroup{Phe, Tyr, Trp}{FYW}{MidnightBlue}{White}{upper}{up}
      \funcgroup{Ala, Gly}{AG}{Gray}{White}{upper}{up}
      \funcgroup{Cys, Met}{CM}{Yellow}{White}{upper}{up}
      \funcgroup{Ser, Thr}{ST}{Orange}{White}{upper}{up}
      \funcgroup{Asn, Gln}{NQ}{Cyan}{White}{upper}{up}
      \funcgroup{Leu, Val, Ile}{LVI}{Green}{White}{upper}{up}
      \funcgroup{Pro}{P}{Apricot}{White}{upper}{up}
   \else
   \message{<Unknown shading mode - clearing `funcgroups'>}
   \fi\fi\fi\fi\fi\fi\fi
}
\def\shadeallresidues{\all@fshadetrue}
\def\get@fromstack#1;#2;#3;#4;#5@{%
   \xdef\first@{#1} \xdef\second@{#2}
   \xdef\third@{#3} \xdef\fourth@{#4} \xdef\last@{#5@}
}
\def\getregion@fromstack#1{%
   \xdef\temp@{#1}
   \xdef\first@{\csname stack@reg\temp@\endcsname}
   \expandafter\get@fromstack\first@
   \expandafter\xdef\csname style\temp@\endcsname{\first@}
   \expandafter\xdef\csname start\temp@\endcsname{\second@}
   \expandafter\xdef\csname stop\temp@\endcsname{\third@}
   \expandafter\xdef\csname all\temp@\endcsname{\fourth@}
   \expandafter\xdef\csname stack@reg\temp@\endcsname{\last@}
}
\def\sort@stack{%
   \expandafter\get@fromstack\last@ 
   \ifx\first@\ampers@nd  
      \xdef\tmpstack{\tmpstack\the\loopcount;\st@rt;\st@p;\@ll;&;&;&;&;@}
   \else 
   \ifnum\st@rt<\second@
      \xdef\tmpstack{\tmpstack\the\loopcount;\st@rt;\st@p;\@ll;%
                     \first@;\second@;\third@;\fourth@;\last@} 
   \else
      \xdef\tmpstack{\tmpstack\first@;\second@;\third@;\fourth@;} 
      \sort@stack
   \fi\fi
}
\def\get@regions#1..#2,#3@{%
   \xdef\st@rt{#1} \xdef\st@p{#2} \xdef\list@{#3}
   \xdef\last@{\csname stack@reg\seq@\endcsname}
   \xdef\tmpstack{}
   \sort@stack
   \expandafter\xdef\csname stack@reg\seq@\endcsname{\tmpstack}
}
\def\get@fromemphstack#1;#2;#3;#4@{%
   \xdef\first@{#1} \xdef\second@{#2}
   \xdef\third@{#3} \xdef\last@{#4@}
}
\def\getregion@fromemphstack#1{%
   \xdef\temp@{#1}
   \xdef\first@{\csname stack@emphreg\temp@\endcsname}
   \expandafter\get@fromemphstack\first@
   \expandafter\xdef\csname emphstart\temp@\endcsname{\first@}
   \expandafter\xdef\csname emphstop\temp@\endcsname{\second@}
   \expandafter\xdef\csname emphall\temp@\endcsname{\third@}
   \expandafter\xdef\csname stack@emphreg\temp@\endcsname{\last@}
}
\def\sort@emphstack{%
   \expandafter\get@fromemphstack\last@
   \ifx\first@\ampers@nd
      \xdef\tmpstack{\tmpstack\st@rt;\st@p;\@ll;&;&;&;@}
   \else
   \ifnum\st@rt<\second@
      \xdef\tmpstack{\tmpstack\st@rt;\st@p;\@ll;\first@;\second@;\third@;\last@}
   \else
      \xdef\tmpstack{\tmpstack\first@;\second@;\third@;} \sort@emphstack
   \fi\fi
}
\def\get@emphregions#1..#2,#3@{%
   \regionalemphtrue
   \xdef\st@rt{#1} \xdef\st@p{#2} \xdef\list@{#3}
   \xdef\last@{\csname stack@emphreg\seq@\endcsname}
   \xdef\tmpstack{}
   \sort@emphstack
   \expandafter\xdef\csname stack@emphreg\seq@\endcsname{\tmpstack}
}
\def\getregion@fromlowerstack#1{%
   \xdef\temp@{#1}
   \xdef\first@{\csname stack@lowerreg\temp@\endcsname}
   \expandafter\get@fromemphstack\first@
   \expandafter\xdef\csname lowerstart\temp@\endcsname{\first@}
   \expandafter\xdef\csname lowerstop\temp@\endcsname{\second@}
   \expandafter\xdef\csname lowerall\temp@\endcsname{\third@}
   \expandafter\xdef\csname stack@lowerreg\temp@\endcsname{\last@}
}
\def\sort@lowerstack{%
   \expandafter\get@fromemphstack\last@
   \ifx\first@\ampers@nd
      \xdef\tmpstack{\tmpstack\st@rt;\st@p;\@ll;&;&;&;@}
   \else
   \ifnum\st@rt<\second@
      \xdef\tmpstack{\tmpstack\st@rt;\st@p;\@ll;\first@;\second@;\third@;\last@}
   \else
      \xdef\tmpstack{\tmpstack\first@;\second@;\third@;} \sort@lowerstack
   \fi\fi
}
\def\get@lowerregions#1..#2,#3@{%
   \regionallowertrue
   \xdef\st@rt{#1} \xdef\st@p{#2} \xdef\list@{#3}
   \xdef\last@{\csname stack@lowerreg\seq@\endcsname}
   \xdef\tmpstack{}
   \sort@lowerstack
   \expandafter\xdef\csname stack@lowerreg\seq@\endcsname{\tmpstack}
}
\def\get@fromdomstack#1;#2;#3@{%
   \xdef\first@{#1} \xdef\second@{#2}
   \xdef\last@{#3@}
}
\def\sort@domstack{%
   \expandafter\get@fromdomstack\last@
   \ifx\first@\ampers@nd
      \xdef\tmpstack{\tmpstack\st@rt;\st@p;&;&;@}
   \else
   \ifnum\st@rt<\second@
      \xdef\tmpstack{\tmpstack\st@rt;\st@p;\first@;\second@;\last@}
   \else
      \xdef\tmpstack{\tmpstack\first@;\second@;} \sort@domstack
   \fi\fi
}
\def\get@domainregions#1..#2,#3@{%
   \xdef\st@rt{#1} \xdef\st@p{#2} \xdef\list@{#3}
   \xdef\last@{\stack@dom}
   \xdef\tmpstack{}
   \sort@domstack
   \xdef\stack@dom{\tmpstack}
}
\def\get@domain{%
   \expandafter\get@fromdomstack\last@
   \ifx\first@\ampers@nd
   \else
      \loopcount=\first@
      \loop
         \xdef\domain@num@list{\domain@num@list \the\loopcount,}
         \advance\loopcount by 1
      \ifnum\loopcount>\second@\else\repeat
      \get@domain
   \fi
}
\def\getregion@fromtintstack#1{%
   \xdef\temp@{#1}
   \xdef\first@{\csname stack@tintreg\temp@\endcsname}
   \expandafter\get@fromemphstack\first@
   \expandafter\xdef\csname tintstart\temp@\endcsname{\first@}
   \expandafter\xdef\csname tintstop\temp@\endcsname{\second@}
   \expandafter\xdef\csname tintall\temp@\endcsname{\third@}
   \expandafter\xdef\csname stack@tintreg\temp@\endcsname{\last@}
}
\def\get@tintregions#1..#2,#3@{%
   \regionaltinttrue
   \xdef\st@rt{#1} \xdef\st@p{#2} \xdef\list@{#3}
   \xdef\last@{\csname stack@tintreg\seq@\endcsname}
   \xdef\tmpstack{}
   \sort@emphstack
   \expandafter\xdef\csname stack@tintreg\seq@\endcsname{\tmpstack}
}
\def\getregion@fromframestack#1{%
   \xdef\temp@{#1}
   \xdef\first@{\csname stack@framereg\temp@\endcsname}
   \expandafter\get@fromemphstack\first@
   \expandafter\xdef\csname framestart\temp@\endcsname{\first@}
   \expandafter\xdef\csname framestop\temp@\endcsname{\second@}
   \expandafter\xdef\csname framestyle\temp@\endcsname{\third@}
   \expandafter\xdef\csname stack@framereg\temp@\endcsname{\last@}
}
\def\get@frameregions#1..#2,#3@{%
   \frame@true
   \xdef\st@rt{#1} \xdef\st@p{#2} \xdef\list@{#3}
   \xdef\last@{\csname stack@framereg\seq@\endcsname}
   \xdef\tmpstack{}
   \sort@emphstack
   \expandafter\xdef\csname stack@framereg\seq@\endcsname{\tmpstack}
}
\def\getregion@fromshadingstack#1{%
   \xdef\temp@{#1}
   \xdef\first@{\csname stack@shadingreg\temp@\endcsname}
   \expandafter\get@fromemphstack\first@
   \expandafter\xdef\csname shadingstart\temp@\endcsname{\first@}
   \expandafter\xdef\csname shadingstop\temp@\endcsname{\second@}
   \expandafter\xdef\csname shadingstyle\temp@\endcsname{\third@}
   \expandafter\xdef\csname stack@shadingreg\temp@\endcsname{\last@}
}
\def\get@shadingregions#1..#2,#3@{%
   \shading@true
   \xdef\st@rt{#1} \xdef\st@p{#2} \xdef\list@{#3}
   \xdef\last@{\csname stack@shadingreg\seq@\endcsname}
   \xdef\tmpstack{}
   \sort@emphstack
   \expandafter\xdef\csname stack@shadingreg\seq@\endcsname{\tmpstack}
}
\def\getregion@fromfstack#1{%
   \xdef\temp@{#1}
   \xdef\first@{\csname stack@\bottop@\temp@\endcsname}
   \expandafter\get@fromstack\first@
   \expandafter\xdef\csname text\bottop@\temp@\endcsname{\first@}
   \expandafter\xdef\csname start\bottop@\temp@\endcsname{\second@}
   \expandafter\xdef\csname stop\bottop@\temp@\endcsname{\third@}
   \ifx\fourth@\ampers@nd \xdef\fourth@{///} \fi
   \expandafter\xdef\csname style\bottop@\temp@\endcsname{\fourth@}
   \expandafter\xdef\csname stack@\bottop@\temp@\endcsname{\last@}
}
\def\sort@fstack{%
   \expandafter\get@fromstack\last@
   \ifx\first@\ampers@nd
      \xdef\tmpstack{\tmpstack\f@text@;\st@rt;\st@p;\style@;&;&;&;&;@}
   \else
   \ifnum\st@rt<\second@
      \xdef\tmpstack{\tmpstack\f@text@;\st@rt;\st@p;\style@;%
                     \first@;\second@;\third@;\fourth@;\last@}
   \else
      \xdef\tmpstack{\tmpstack\first@;\second@;\third@;\fourth@;}
      \sort@fstack
   \fi\fi
}
\def\get@fregions#1..#2,#3@{%
   \xdef\st@rt{#1}
   \ifx\temp@\y@
      \xdef\st@p{#1}
   \else
      \xdef\st@p{#2}
   \fi
   \xdef\list@{#3}
   \xdef\last@{\csname stack@\bottop@\seq@\endcsname}
   \xdef\tmpstack{}
   \sort@fstack
   \expandafter\xdef\csname stack@\bottop@\seq@\endcsname{\tmpstack}
}
\def\getarrow@shape#1#2#3#4&{%
   \xdef\first@@{#1}\xdef\second@@{#2}\xdef\third@@{#3}
   \if\first@@ v
      \if\second@@ =
      \else \xdef\style@{\first@@ v\third@@#4} \fi
   \else
   \if\first@@ v
      \if\second@@ =
      \else \xdef\style@{\first@@ v\third@@#4} \fi
   \else
   \if\third@@ v
      \if\second@@ =
      \else \xdef\style@{\first@@ v\third@@#4} \fi
   \else
   \if\third@@ v
      \if\second@@ =
      \else \xdef\style@{\first@@ v\third@@#4} \fi
   \fi\fi\fi\fi
}
\def\get@shape#1#2#3{%
   \xdef\first@@{#1}\xdef\second@@{#2}\xdef\third@@{#3}%
   \if\second@@ v \xdef\second@@{arrow}%
   \fi%
   \if\second@@ = \xdef\second@@{doublearrow}%
   \fi%
}
\def\getstyle@left#1#2#3#4@{%
   \ifstop@
      \xdef\style@@{\csname fstyle\bottop@\the\loopcount\endcsname}
   \else
   \xdef\style@@{#2}
   \xdef\temp@{-}
   \ifx\style@@\temp@ \xdef\style@@{#1#2-#4}
   \else
   \xdef\temp@{v}
   \ifx\style@@\temp@ \xdef\style@@{#1#2-#4}
   \else   
   \xdef\temp@{=}
   \ifx\style@@\temp@ \xdef\style@@{#1#2=#4}
   \else   
      \xdef\style@@{\csname fstyle\bottop@\the\loopcount\endcsname}
   \fi\fi\fi\fi
}
\def\getstyle@right#1#2#3#4@{%
   \xdef\style@@{#2}
   \xdef\temp@{-}
   \ifx\style@@\temp@
      \expandafter\xdef\csname fstyle\bottop@\the\loopcount\endcsname{-#2#3#4}
   \fi
   \xdef\temp@{v}
   \ifx\style@@\temp@
      \expandafter\xdef\csname fstyle\bottop@\the\loopcount\endcsname{-#2#3#4}
   \fi
   \xdef\temp@{=}
   \ifx\style@@\temp@
      \expandafter\xdef\csname fstyle\bottop@\the\loopcount\endcsname{=#2#3#4}
   \fi
}
\def\opt@color#1[#2]#3&{\xdef\fourth@{#1}\xdef\f@color{#2}}
\def\graph@opt@color#1[#2]:#3[#4][#5]#6&{%
   \xdef\fourth@{#1}
   \xdef\f@color{#2}
   \xdef\ffourth@{#3}
   \xdef\ff@color{#4}
   \xdef\fffourth@{#5}}
\def\arrow@col@width#1[#2][#3]#4&{\xdef\fourth@{#1}\xdef\f@color{#2}\xdef\rule@@thick{#3}}
\def\second@color#1,&{\xdef\back@color{#1}}
\def\two@opt@color#1,#2@{%
   \xdef\sixth@{#1}%
   \xdef\seventh@{#2}%
   \ifx\first@\ampers@nd%
   \else%
      \xdef\frame@color{#1}%
      \ifx\seventh@\ampers@nd%
         \xdef\back@color{#1}%
         \xdef\backtext@color{White}%
      \else%
         \expandafter\second@color\seventh@%
      \fi%
   \fi%
}
\def\test@get#1#2@{%
   \xdef\temp@{#1} \xdef\third@@{#2@}
   \ifx\temp@\ampers@nd
   \else
      \if \temp@ X
         \xdef\temp@{\second@@}
      \fi
      \ifx\temp@\second@@
         \xdef\third@@{\second@@}
      \else
         \expandafter\test@get\third@@
      \fi
   \fi
}
\def\get@@second@#1#2@{\xdef\second@@{#1}\xdef\first@@{#2@}}
\def\get@@third@#1]#2@{\xdef\third@@{#1&@}\xdef\fifth@@{#2@}}
\def\get@third@#1#2@{\xdef\third@@{#1}\xdef\fifth@@{#2@}}
\def\find@motif{%
   \expandafter\get@@second@\first@@
   \ifx\second@@\ampers@nd
   \else
      \ifx\second@@\d@t
      \else
         \advance\temp@count by 1
         \if \third@@ X
            \xdef\third@@{\second@@}
         \fi
         \ifx\third@@\br@cket
            \expandafter\get@@third@\fifth@@
            \expandafter\test@get\third@@
         \fi
         \ifx\second@@\third@@
            \ifx\nineth@@\n@
               \xdef\st@rt{\the\temp@count}
               \xdef\st@p{n}
            \fi
            \xdef\nineth@@{y}
            \expandafter\get@third@\fifth@@
            \ifx\third@@\ampers@nd
               \xdef\st@p{\the\temp@count}
               \xdef\list@{}
               \xdef\tmpstack{}
               \xdef\temp@{shade}
               \ifx\temp@\temp@@
                  \xdef\last@{\csname stack@reg\seq@\endcsname} 
                  \loopcount=\style@
                  \sort@stack
                  \expandafter\xdef\csname stack@reg\seq@\endcsname{\tmpstack}
               \else
               \xdef\temp@{emph}
               \ifx\temp@\temp@@
                  \regionalemphtrue
                  \xdef\last@{\csname stack@emphreg\seq@\endcsname} 
                  \sort@emphstack
                  \expandafter\xdef\csname stack@emphreg\seq@\endcsname{\tmpstack}
               \else
               \xdef\temp@{lower}
               \ifx\temp@\temp@@
                  \regionallowertrue
                  \xdef\last@{\csname stack@lowerreg\seq@\endcsname} 
                  \sort@lowerstack
                  \expandafter\xdef\csname stack@lowerreg\seq@\endcsname{\tmpstack}
               \else
               \xdef\temp@{tint}
               \ifx\temp@\temp@@
                  \regionaltinttrue
                  \xdef\last@{\csname stack@tintreg\seq@\endcsname}
                  \sort@emphstack
                  \expandafter\xdef\csname stack@tintreg\seq@\endcsname{\tmpstack}
               \else
               \xdef\temp@{frame}
               \ifx\temp@\temp@@
                  \frame@true
                  \xdef\last@{\csname stack@framereg\seq@\endcsname} 
                  \sort@emphstack
                  \expandafter\xdef\csname stack@framereg\seq@\endcsname{\tmpstack}
               \else
               \xdef\temp@{shading}
               \ifx\temp@\temp@@
                  \shading@true
                  \xdef\last@{\csname stack@shadingreg\seq@\endcsname} 
                  \sort@emphstack
                  \expandafter\xdef\csname stack@shadingreg\seq@\endcsname{\tmpstack}
               \else
               \xdef\temp@{ttttop}
               \ifx\temp@\temp@@
                  \xdef\last@{\csname stack@ttttop\seq@\endcsname} 
                  \sort@fstack
                  \expandafter\xdef\csname stack@ttttop\seq@\endcsname{\tmpstack}
               \else
               \xdef\temp@{tttop}
               \ifx\temp@\temp@@
                  \xdef\last@{\csname stack@tttop\seq@\endcsname} 
                  \sort@fstack
                  \expandafter\xdef\csname stack@tttop\seq@\endcsname{\tmpstack}
               \else
               \xdef\temp@{ttop}
               \ifx\temp@\temp@@
                  \xdef\last@{\csname stack@ttop\seq@\endcsname} 
                  \sort@fstack
                  \expandafter\xdef\csname stack@ttop\seq@\endcsname{\tmpstack}
               \else
               \xdef\temp@{top}
               \ifx\temp@\temp@@
                  \xdef\last@{\csname stack@top\seq@\endcsname}
                  \sort@fstack
                  \expandafter\xdef\csname stack@top\seq@\endcsname{\tmpstack}
               \else
               \xdef\temp@{bottom}
               \ifx\temp@\temp@@
                  \xdef\last@{\csname stack@bottom\seq@\endcsname} 
                  \sort@fstack
                  \expandafter\xdef\csname stack@bottom\seq@\endcsname{\tmpstack}
               \else
               \xdef\temp@{bbottom}
               \ifx\temp@\temp@@
                  \xdef\last@{\csname stack@bbottom\seq@\endcsname} 
                  \sort@fstack
                  \expandafter\xdef\csname stack@bbottom\seq@\endcsname{\tmpstack}
               \else
               \xdef\temp@{bbbottom}
               \ifx\temp@\temp@@
                  \xdef\last@{\csname stack@bbbottom\seq@\endcsname} 
                  \sort@fstack
                  \expandafter\xdef\csname stack@bbbottom\seq@\endcsname{\tmpstack}
               \else
               \xdef\temp@{bbbbottom}
               \ifx\temp@\temp@@
                  \xdef\last@{\csname stack@bbbbottom\seq@\endcsname} 
                  \sort@fstack
                  \expandafter\xdef\csname stack@bbbbottom\seq@\endcsname{\tmpstack}
               \fi
               \fi
               \fi
               \fi
               \fi
               \fi
               \fi
               \fi
               \fi
               \fi
               \fi
               \fi
               \fi
               \fi
               \xdef\fifth@@{\m@tif} \expandafter\get@third@\fifth@@
               \xdef\nineth@@{n}
            \fi
         \else
            \ifx\nineth@@\y@
               \xdef\first@@{\second@@\first@@}
               \advance\temp@count by -1
            \fi
            \xdef\nineth@@{n} \xdef\st@p{n}
            \xdef\fifth@@{\m@tif} \expandafter\get@third@\fifth@@
         \fi
      \fi
   \find@motif
   \fi
}
\def\prep@motif{%

   \def\get@motif@specs##1,##2,##3,##4,##5,&{%
      \xdef\seq@{##1} \xdef\temp@@{##2} \xdef\@ll{##3} \def\fourth@@{##4} \xdef\fifth@@{##5&@}
   }

   \loopcount=0
   \loop
      \advance\loopcount by 1
      \expandafter\xdef\csname stack@reg\the\loopcount\endcsname{\csname style\the\loopcount\endcsname;%
                                                                 \csname start\the\loopcount\endcsname;%
                                                                 \csname stop\the\loopcount\endcsname;%
                                                                 \csname all\the\loopcount\endcsname;%
                                                                 \csname stack@reg\the\loopcount\endcsname}
      \expandafter\xdef\csname stack@emphreg\the\loopcount\endcsname{\csname emphstart\the\loopcount\endcsname;%
                                                                 \csname emphstop\the\loopcount\endcsname;%
                                                                 \csname emphall\the\loopcount\endcsname;%
                                                                 \csname stack@emphreg\the\loopcount\endcsname}
      \expandafter\xdef\csname stack@lowerreg\the\loopcount\endcsname{\csname lowerstart\the\loopcount\endcsname;%
                                                                 \csname lowerstop\the\loopcount\endcsname;%
                                                                 \csname lowerall\the\loopcount\endcsname;%
                                                                 \csname stack@lowerreg\the\loopcount\endcsname}
      \expandafter\xdef\csname stack@tintreg\the\loopcount\endcsname{\csname tintstart\the\loopcount\endcsname;%
                                                                 \csname tintstop\the\loopcount\endcsname;%
                                                                 \csname tintall\the\loopcount\endcsname;%
                                                                 \csname stack@tintreg\the\loopcount\endcsname}
      \expandafter\xdef\csname stack@framereg\the\loopcount\endcsname{\csname framestart\the\loopcount\endcsname;%
                                                                 \csname framestop\the\loopcount\endcsname;%
                                                                 \csname framestyle\the\loopcount\endcsname;%
                                                                 \csname stack@framereg\the\loopcount\endcsname}
      \expandafter\xdef\csname stack@shadingreg\the\loopcount\endcsname{\csname shadingstart\the\loopcount\endcsname;%
                                                                 \csname shadingstop\the\loopcount\endcsname;%
                                                                 \csname shadingstyle\the\loopcount\endcsname;%
                                                                 \csname stack@shadingreg\the\loopcount\endcsname}
      \expandafter\xdef\csname stack@ttttop\the\loopcount\endcsname{\csname textttttop\the\loopcount\endcsname;%
                                                                 \csname startttttop\the\loopcount\endcsname;%
                                                                 \csname stopttttop\the\loopcount\endcsname;%
                                                                 \csname stylettttop\the\loopcount\endcsname;%
                                                                 \csname stack@ttttop\the\loopcount\endcsname}
      \expandafter\xdef\csname stack@tttop\the\loopcount\endcsname{\csname texttttop\the\loopcount\endcsname;%
                                                                 \csname starttttop\the\loopcount\endcsname;%
                                                                 \csname stoptttop\the\loopcount\endcsname;%
                                                                 \csname styletttop\the\loopcount\endcsname;%
                                                                 \csname stack@tttop\the\loopcount\endcsname}
      \expandafter\xdef\csname stack@ttop\the\loopcount\endcsname{\csname textttop\the\loopcount\endcsname;%
                                                                 \csname startttop\the\loopcount\endcsname;%
                                                                 \csname stopttop\the\loopcount\endcsname;%
                                                                 \csname stylettop\the\loopcount\endcsname;%
                                                                 \csname stack@ttop\the\loopcount\endcsname}
      \expandafter\xdef\csname stack@top\the\loopcount\endcsname{\csname texttop\the\loopcount\endcsname;%
                                                                 \csname starttop\the\loopcount\endcsname;%
                                                                 \csname stoptop\the\loopcount\endcsname;%
                                                                 \csname styletop\the\loopcount\endcsname;%
                                                                 \csname stack@top\the\loopcount\endcsname}
      \expandafter\xdef\csname stack@bottom\the\loopcount\endcsname{\csname textbottom\the\loopcount\endcsname;%
                                                                 \csname startbottom\the\loopcount\endcsname;%
                                                                 \csname stopbottom\the\loopcount\endcsname;%
                                                                 \csname stylebottom\the\loopcount\endcsname;%
                                                                 \csname stack@bottom\the\loopcount\endcsname}
      \expandafter\xdef\csname stack@bbottom\the\loopcount\endcsname{\csname textbbottom\the\loopcount\endcsname;%
                                                                 \csname startbbottom\the\loopcount\endcsname;%
                                                                 \csname stopbbottom\the\loopcount\endcsname;%
                                                                 \csname stylebbottom\the\loopcount\endcsname;%
                                                                 \csname stack@bbottom\the\loopcount\endcsname}
      \expandafter\xdef\csname stack@bbbottom\the\loopcount\endcsname{\csname textbbbottom\the\loopcount\endcsname;%
                                                                 \csname startbbbottom\the\loopcount\endcsname;%
                                                                 \csname stopbbbottom\the\loopcount\endcsname;%
                                                                 \csname stylebbbottom\the\loopcount\endcsname;%
                                                                 \csname stack@bbbottom\the\loopcount\endcsname}
      \expandafter\xdef\csname stack@bbbbottom\the\loopcount\endcsname{\csname textbbbbottom\the\loopcount\endcsname;%
                                                                 \csname startbbbbottom\the\loopcount\endcsname;%
                                                                 \csname stopbbbbottom\the\loopcount\endcsname;%
                                                                 \csname stylebbbbottom\the\loopcount\endcsname;%
                                                                 \csname stack@bbbbottom\the\loopcount\endcsname}
   \ifnum\loopcount<\seq@num\repeat
   \temp@@count=0
   \loop
      \advance\temp@@count by 1
      \xdef\first@@{\csname motif\the\temp@@count\endcsname}
      \expandafter\get@motif@specs\first@@
      \xdef\first@@{\csname sequence\seq@\endcsname &@}
      \temp@count=\csname res@count\seq@\endcsname
      \xdef\f@text@{\@ll}
      \xdef\style@{\fourth@@}
      \xdef\m@tif{\fifth@@} \expandafter\get@third@\fifth@@
      \xdef\nineth@@{n} \xdef\st@p{n}
      \find@motif
   \ifnum\temp@@count<\motif@num\repeat
   \loopcount=0
   \loop
      \advance\loopcount by 1
      \expandafter\getregion@fromstack{\the\loopcount}
      \expandafter\getregion@fromtintstack{\the\loopcount}
      \expandafter\getregion@fromemphstack{\the\loopcount}
      \expandafter\getregion@fromlowerstack{\the\loopcount}
      \expandafter\getregion@fromframestack{\the\loopcount}
      \expandafter\getregion@fromshadingstack{\the\loopcount}
      \xdef\bottop@{top}       \expandafter\getregion@fromfstack{\the\loopcount}
      \xdef\bottop@{ttop}      \expandafter\getregion@fromfstack{\the\loopcount}
      \xdef\bottop@{tttop}     \expandafter\getregion@fromfstack{\the\loopcount}
      \xdef\bottop@{ttttop}    \expandafter\getregion@fromfstack{\the\loopcount}
      \xdef\bottop@{bottom}    \expandafter\getregion@fromfstack{\the\loopcount}
      \xdef\bottop@{bbottom}   \expandafter\getregion@fromfstack{\the\loopcount}
      \xdef\bottop@{bbbottom}  \expandafter\getregion@fromfstack{\the\loopcount}
      \xdef\bottop@{bbbbottom} \expandafter\getregion@fromfstack{\the\loopcount}
   \ifnum\loopcount<\seq@num\repeat
}

\def\test@motif#1#2@{%
   \xdef\temp@{shade}
   \ifx\temp@\temp@@ \xdef\style@{\the\loopcount} \fi
   \xdef\temp@{emph}
   \ifx\temp@\temp@@ \xdef\style@{0} \fi
   \xdef\temp@{lower}
   \ifx\temp@\temp@@ \xdef\style@{0} \fi
   \xdef\temp@{tint}
   \ifx\temp@\temp@@ \xdef\style@{0} \fi
   \xdef\temp@{frame}
   \ifx\temp@\temp@@ \xdef\style@{0} \fi
   \xdef\temp@{shading}
   \ifx\temp@\temp@@ \xdef\style@{0} \fi
   \xdef\temp@{#1#2}
   \xdef\first@{#1}
   \ifx\first@\br@cket
      \xdef\label@motif{y}
      \temp@count=\motif@num
      \advance\temp@count by 1
      \xdef\motif@num{\the\temp@count} 
      \expandafter\xdef\csname motif\the\temp@count\endcsname{\seq@,\temp@@,\@ll,\style@,\temp@}
      \xdef\list@{&}
   \else
      \expandafter\check@char\first@
      \ifletter
         \xdef\label@motif{y}
         \temp@count=\motif@num
         \advance\temp@count by 1
         \xdef\motif@num{\the\temp@count} 
         \expandafter\xdef\csname motif\the\temp@count\endcsname{\seq@,\temp@@,\@ll,\style@,\temp@}
         \xdef\list@{&}
      \fi
   \fi   
}

\def\test@PDB#1:#2,#3,#4,#5,#6@{%
   \xdef\last@{#1[,]&}\expandafter\opt@color\last@
   \xdef\PDB@distance{no}
   \ifx\f@color\comm@
   \else
      \xdef\f@color{\f@color .&&@}
      \expandafter\coord@get\f@color
      \xdef\temp@{\c@@rd @}
      \expandafter\check@letter\temp@
      \ifletter
      \else
         \temp@count=\c@@rd
         \multiply\temp@count by \temp@count
         \xdef\PDB@distance{\the\temp@count}
      \fi
   \fi
   \xdef\PDB@hitpos{0}
   \xdef\PDB@stack{}
   \xdef\PDB@name{#2}
   \xdef\PDB@type{point}
   \ifx\fourth@\PDB@type
      \xdef\PDB@{y}
      \xdef\last@{#3[,]&}\expandafter\opt@color\last@
      \xdef\PDB@refnum@a{\fourth@}
      \ifx\f@color\C@lpha
         \xdef\PDB@back@side@a{CA}
      \else
         \xdef\PDB@back@side@a{side}
      \fi
      \xdef\PDB@refnum@b{0} \xdef\PDB@back@side@b{x}
      \xdef\PDB@refnum@c{0} \xdef\PDB@back@side@c{x}
      \ifx\PDB@distance\n@ \xdef\PDB@distance{3600}\fi
      \load@PDB 
      \expandafter\get@item\PDB@stack \xdef\list@{\first@@}
   \else
      \xdef\PDB@type{line}
      \ifx\fourth@\PDB@type
         \xdef\PDB@{y}
         \xdef\last@{#3[,]&}\expandafter\opt@color\last@
         \xdef\PDB@refnum{\fourth@}
         \ifx\f@color\C@lpha
            \xdef\PDB@back@side{CA}
         \else
            \xdef\PDB@back@side{side}
         \fi
         \xdef\last@{#4[,]&}\expandafter\opt@color\last@
         \ifnum\PDB@refnum<\fourth@ 
            \xdef\PDB@refnum@a{\PDB@refnum} 
            \xdef\PDB@refnum@b{\fourth@}
            \xdef\PDB@back@side@a{\PDB@back@side}
            \ifx\f@color\C@lpha
               \xdef\PDB@back@side@b{CA}
            \else
               \xdef\PDB@back@side@b{side}
            \fi
         \else
            \xdef\PDB@refnum@b{\PDB@refnum} 
            \xdef\PDB@refnum@a{\fourth@}
            \xdef\PDB@back@side@b{\PDB@back@side}
            \ifx\f@color\C@lpha
               \xdef\PDB@back@side@a{CA}
            \else
               \xdef\PDB@back@side@a{side}
            \fi
         \fi
         \xdef\PDB@refnum@c{0} \xdef\PDB@back@side@c{x}
         \ifx\PDB@distance\n@ \xdef\PDB@distance{100}\fi
         \load@PDB 
         \expandafter\get@item\PDB@stack \xdef\list@{\first@@}
      \else
         \xdef\PDB@type{plane}
         \ifx\fourth@\PDB@type
            \xdef\PDB@{y}
            \xdef\last@{#3[,]&}\expandafter\opt@color\last@
            \xdef\PDB@refnum{\fourth@}
            \ifx\f@color\C@lpha
               \xdef\PDB@back@side{CA}
            \else
               \xdef\PDB@back@side{side}
            \fi
            \xdef\last@{#4[,]&}\expandafter\opt@color\last@
            \ifnum\PDB@refnum<\fourth@ 
               \xdef\PDB@refnum@a{\PDB@refnum} 
               \xdef\PDB@refnum@b{\fourth@}
               \xdef\PDB@back@side@a{\PDB@back@side}
               \ifx\f@color\C@lpha
                  \xdef\PDB@back@side@b{CA}
               \else
                  \xdef\PDB@back@side@b{side}
               \fi
            \else
               \xdef\PDB@refnum@b{\PDB@refnum} 
               \xdef\PDB@refnum@a{\fourth@}
               \xdef\PDB@back@side@b{\PDB@back@side}
               \ifx\f@color\C@lpha
                  \xdef\PDB@back@side@a{CA}
               \else
                  \xdef\PDB@back@side@a{side}
               \fi
            \fi
            \xdef\last@{#5[,]&}\expandafter\opt@color\last@
            \xdef\PDB@refnum{\fourth@}
            \ifx\f@color\C@lpha
               \xdef\PDB@back@side{CA}
            \else
               \xdef\PDB@back@side{side}
            \fi
            \ifnum\PDB@refnum>\PDB@refnum@b
               \xdef\PDB@refnum@c{\PDB@refnum}
               \xdef\PDB@back@side@c{\PDB@back@side}
            \else
               \ifnum\PDB@refnum>\PDB@refnum@a
                  \xdef\PDB@refnum@c{\PDB@refnum@b}
                  \xdef\PDB@back@side@c{\PDB@back@side@b}
                  \xdef\PDB@refnum@b{\PDB@refnum}
                  \xdef\PDB@back@side@b{\PDB@back@side}
               \else
                  \xdef\PDB@refnum@c{\PDB@refnum@b}
                  \xdef\PDB@back@side@c{\PDB@back@side@b}
                  \xdef\PDB@refnum@b{\PDB@refnum@a}
                  \xdef\PDB@back@side@b{\PDB@back@side@a}
                  \xdef\PDB@refnum@a{\PDB@refnum}
                  \xdef\PDB@back@side@a{\PDB@back@side}
               \fi
            \fi
            \ifx\PDB@distance\n@ \xdef\PDB@distance{100}\fi
            \load@PDB 
            \expandafter\get@item\PDB@stack \xdef\list@{\first@@}
         \else
            \xdef\temp@{\list@ @}
            \expandafter\test@motif\temp@
         \fi
      \fi
   \fi
}

\def\test@fill#1:#2:#3&{%
   \xdef\last@{#1[,][,]&}\expandafter\arrow@col@width\last@%
   \xdef\second@@{\fourth@}%
   \xdef\last@{///}%
   \ifx\fourth@\last@%
     \xdef\second@@{empty}%
   \else
   \xdef\last@{translate}%
   \ifx\fourth@\last@%
     \xdef\second@@{translate}%
     \xdef\fill@char{#2[,]&}%
     \expandafter\opt@color\fill@char%
     \xdef\fill@char{\fourth@}%
   \else
   \xdef\last@{fill}%
   \ifx\fourth@\last@
     \xdef\second@@{fill}%
     \xdef\fill@char{#2[,]&}%
     \expandafter\opt@color\fill@char%
     \xdef\fill@char{\fourth@}%
   \else
  \xdef\last@{bar}%
   \ifx\fourth@\last@
     \xdef\second@@{bar}%
     \xdef\b@r{bar}%
     \ifx\f@color\comm@
        \xdef\g@min{,}\xdef\g@max{,}%
     \else
        \xdef\f@color{\f@color @}%
        \expandafter\get@item\f@color%
        \xdef\g@min{\fourth@} \xdef\g@max{\first@@}%
        \xdef\pm@shift{\fourth@}%
     \fi%
     \xdef\fill@char{#2[,]&}%
     \expandafter\opt@color\fill@char%
     \ifx\f@color\comm@\xdef\f@color{GrayDefault}\fi%
     \xdef\fill@char{\fourth@}%
     \xdef\box@color{\f@color,&@}%
     \expandafter\two@opt@color\box@color%
   \else
   \xdef\last@{color}%
   \ifx\fourth@\last@
     \xdef\second@@{color}%
     \ifx\f@color\comm@
        \xdef\g@min{,}\xdef\g@max{,}%
     \else
        \xdef\f@color{\f@color @}%
        \expandafter\get@item\f@color%
        \xdef\g@min{\fourth@} \xdef\g@max{\first@@}%
     \fi%
     \xdef\fill@char{#2[,]&}%
     \expandafter\opt@color\fill@char%
     \ifx\f@color\comm@\xdef\f@color{GrayDefault}\fi%
     \xdef\fill@char{\fourth@}%
   \else
     \xdef\last@{box}%
     \ifx\fourth@\last@%
        \xdef\second@@{box}%
        \ifx\f@color\comm@\xdef\f@color{White}\fi%
        \xdef\seventh@{\f@color @}\expandafter\check@letter\seventh@%
        \ifnumber%
           \ifx\rule@@thick\comm@%
              \xdef\rule@@thick{\f@color}%
              \xdef\box@color{White,&@}%
           \else%
              \xdef\seventh@{\rule@@thick}%
              \xdef\rule@@thick{\f@color}%
              \xdef\box@color{\seventh@,&@}%
           \fi%
        \else%
           \xdef\box@color{\f@color,&@}%
        \fi%
        \expandafter\two@opt@color\box@color%
        \xdef\fill@char{#2[,]&}%
        \expandafter\opt@color\fill@char%
        \xdef\fill@char{\fourth@}%
   \else
     \xdef\last@{plotcolor}%
     \ifx\fourth@\last@%
        \xdef\second@@{plotcolor}%
        \ifx\f@color\comm@%
           \xdef\pm@shift{0}%
        \else%
           \xdef\pm@shift{\f@color}%
        \fi%
        \xdef\fill@char{#2[,]&}%
        \expandafter\opt@color\fill@char%
        \xdef\fill@char{\fourth@}%
   \else
     \xdef\last@{plotbar}%
     \ifx\fourth@\last@%
        \xdef\second@@{plotbar}%
        \xdef\b@r{bar}%
        \ifx\f@color\comm@
           \xdef\pm@shift{0}%
        \else
           \xdef\pm@shift{\f@color}%
        \fi%
        \xdef\fill@char{#2[,]&}%
        \expandafter\opt@color\fill@char%
        \ifx\f@color\comm@\xdef\f@color{GrayDefault}\fi%
        \xdef\fill@char{\fourth@}%
        \xdef\box@color{\f@color,&@}%
         \expandafter\two@opt@color\box@color%
     \else
        \xdef\arrow@color{\f@color}%
        \ifx\arrow@color\comm@\xdef\arrow@color{Black}\fi%
        \xdef\seventh@{\f@color @}\expandafter\check@letter\seventh@%
        \ifnumber%
           \ifx\rule@@thick\comm@%
              \xdef\rule@@thick{\f@color}%
              \xdef\arrow@color{Black}%
           \else%
              \xdef\seventh@{\rule@@thick}%
              \xdef\rule@@thick{\f@color}%
              \xdef\arrow@color{\seventh@}%
           \fi%
        \fi%
        \xdef\fill@char{#2[,]&}%
        \expandafter\opt@color\fill@char%
        \xdef\fill@char{\fourth@}%
        \ifx\f@color\comm@\xdef\f@color{Black,White}\fi%
        \xdef\backtext@color{White}%
        \xdef\f@color{\f@color,&@}%
        \expandafter\two@opt@color\f@color%
        \xdef\f@color{\arrow@color}%
     \fi
   \fi
   \fi
   \fi
   \fi
   \fi
   \fi
   \fi
   \xdef\seventh@{\f@color @}\expandafter\check@letter\seventh@%
   \ifnumber%
      \ifx\rule@@thick\comm@%
         \xdef\rule@@thick{\f@color}%
         \xdef\f@color{Black}%
      \else%
         \xdef\seventh@{\rule@@thick}%
         \xdef\rule@@thick{\f@color}%
         \xdef\f@color{\seventh@}%
      \fi%
   \fi%
   \ifx\rule@@thick\comm@\xdef\rule@@thick{\the\rule@thick}\fi%
   \ifx\f@color\comm@\xdef\f@color{Black}\fi}
\def\clear@groups{%
    \expandafter\xdef\csname \prfx grpA\endcsname{ -1}
    \expandafter\xdef\csname \prfx grpB\endcsname{ -2}
    \expandafter\xdef\csname \prfx grpC\endcsname{ -3}
    \expandafter\xdef\csname \prfx grpD\endcsname{ -4}
    \expandafter\xdef\csname \prfx grpE\endcsname{ -5}
    \expandafter\xdef\csname \prfx grpF\endcsname{ -6}
    \expandafter\xdef\csname \prfx grpG\endcsname{ -7}
    \expandafter\xdef\csname \prfx grpH\endcsname{ -8}
    \expandafter\xdef\csname \prfx grpI\endcsname{ -9}
    \expandafter\xdef\csname \prfx grpJ\endcsname{-10}
    \expandafter\xdef\csname \prfx grpK\endcsname{-11}
    \expandafter\xdef\csname \prfx grpL\endcsname{-12}
    \expandafter\xdef\csname \prfx grpM\endcsname{-13}
    \expandafter\xdef\csname \prfx grpN\endcsname{-14}
    \expandafter\xdef\csname \prfx grpO\endcsname{-15}
    \expandafter\xdef\csname \prfx grpP\endcsname{-16}
    \expandafter\xdef\csname \prfx grpQ\endcsname{-17}
    \expandafter\xdef\csname \prfx grpR\endcsname{-18}
    \expandafter\xdef\csname \prfx grpS\endcsname{-19}
    \expandafter\xdef\csname \prfx grpT\endcsname{-20}
    \expandafter\xdef\csname \prfx grpU\endcsname{-21}
    \expandafter\xdef\csname \prfx grpV\endcsname{-22}
    \expandafter\xdef\csname \prfx grpW\endcsname{-23}
    \expandafter\xdef\csname \prfx grpX\endcsname{-24}
    \expandafter\xdef\csname \prfx grpY\endcsname{-25}
    \expandafter\xdef\csname \prfx grpZ\endcsname{-26}
    \expandafter\xdef\csname \prfx grp-\endcsname{-999}
    \expandafter\xdef\csname \prfx grp.\endcsname{-999}
    \expandafter\xdef\csname \prfx grp=\endcsname{-999}
}
\def\inactivate@chars{%
   \catcode`\#=12
   \catcode`\"=12
   \catcode`\~=12
   \catcode`\^=12
   \catcode`\_=12
   }
\def\numcount{\the\loopcount}
\def\Alphacount{\@Alph\loopcount}
\def\alphacount{\@alph\loopcount}
\def\romancount{\@roman\loopcount}
\def\Romancount{\@Roman\loopcount}
\def\cut@name#1.#2@{\global\xdef\file@n@me{#1}}
\def\coord@get#1.#2#3@{%
   \xdef\temp@{#2}
   \ifx\temp@\ampers@nd \xdef\temp@{0} \fi
   \xdef\c@@rd{#1\temp@}
}
\def\struc@get#1 #2 #3 #4 #5 #6 #7 #8 #9@{%
    \xdef\first@{#1} \xdef\second@{#2} \xdef\third@{#3}  \xdef\fourth@{#4}
    \xdef\fifth@{#5} \xdef\sixth@{#6}  \xdef\seventh@{#7}\xdef\eighth@{#8}
    \xdef\nineth@{#9}}
\def\get@PHD#1|#2|#3@{\xdef\PHD@line{\PHD@line #2}}
\def\write@PHDsec{%
 \expandafter\get@sim\last@
 \ifx\sim@char\c@mp
      \def\end@{\the\innerloopcount}
      \advance\innerloopcount by 1
 \else
      \ifnum\end@<\begin@ \xdef\end@{\begin@}\fi
      \if\c@mp .
      \else
      \if\c@mp L
      \else
      \if\c@mp H
        \ifx\show@Hsec\yes
         \loopcount=\first@
         \advance\loopcount by 1
         \xdef\first@{\the\loopcount}
         \ifx\m@p\yes
         \immediate\write\featurefile{%
         \string\feature{\bottop@Hsec}{\begin@..\end@}%
               {\label@Hsec}{\text@Hsec}}
         \else
         \immediate\write\featurefile{%
         \string\feature{\bottop@Hsec}{\st@rt}{\begin@..\end@}%
               {\label@Hsec}{\text@Hsec}}\fi
        \fi
      \else
      \if\c@mp E
        \ifx\show@Esec\yes
         \loopcount=\second@
         \advance\loopcount by 1
         \xdef\second@{\the\loopcount}
         \ifx\m@p\yes
         \immediate\write\featurefile{%
         \string\feature{\bottop@Esec}{\begin@..\end@}%
               {\label@Esec}{\text@Esec}}
         \else
         \immediate\write\featurefile{%
         \string\feature{\bottop@Esec}{\st@rt}{\begin@..\end@}%
               {\label@Esec}{\text@Esec}}\fi
        \fi
      \fi\fi\fi\fi
      \xdef\c@mp{\sim@char}
      \advance\innerloopcount by 1
      \xdef\begin@{\the\innerloopcount}
   \fi
 \ifx\sim@char\ampers@nd\else\write@PHDsec\fi
}
\def\write@PHDtopo{%
 \expandafter\get@sim\last@
 \ifx\sim@char\c@mp
      \def\end@{\the\innerloopcount}
      \advance\innerloopcount by 1
 \else
      \ifnum\end@<\begin@ \xdef\end@{\begin@}\fi
      \if\c@mp .
      \else
      \if\c@mp L
      \else
      \if\c@mp T
        \ifx\show@TMtop\yes
         \loopcount=\first@
         \advance\loopcount by 1
         \xdef\first@{\the\loopcount}
         \ifx\m@p\yes
         \immediate\write\featurefile{%
         \string\feature{\bottop@TMtop}{\begin@..\end@}%
               {\label@TMtop}{\text@TMtop}}
         \else
         \immediate\write\featurefile{%
         \string\feature{\bottop@TMtop}{\st@rt}{\begin@..\end@}%
               {\label@TMtop}{\text@TMtop}}\fi
        \fi
      \else
      \if\c@mp i
        \ifx\show@itop\yes
         \loopcount=\second@
         \advance\loopcount by 1
         \xdef\second@{\the\loopcount}
         \ifx\m@p\yes
         \immediate\write\featurefile{%
         \string\feature{\bottop@itop}{\begin@..\end@}%
               {\label@itop}{\text@itop}}
         \else
         \immediate\write\featurefile{%
         \string\feature{\bottop@itop}{\st@rt}{\begin@..\end@}%
               {\label@itop}{\text@itop}}\fi
        \fi
      \else
      \if\c@mp o
        \ifx\show@etop\yes
         \loopcount=\second@
         \advance\loopcount by 1
         \xdef\second@{\the\loopcount}
         \ifx\m@p\yes
         \immediate\write\featurefile{%
         \string\feature{\bottop@etop}{\begin@..\end@}%
               {\label@etop}{\text@etop}}
         \else
         \immediate\write\featurefile{%
         \string\feature{\bottop@etop}{\st@rt}{\begin@..\end@}%
               {\label@etop}{\text@etop}}\fi
        \fi
      \fi\fi\fi\fi\fi
      \xdef\c@mp{\sim@char}
      \advance\innerloopcount by 1
      \xdef\begin@{\the\innerloopcount}
 \fi
 \ifx\sim@char\ampers@nd\else\write@PHDtopo\fi
}
\def\include@T@coffee{%
    \bgroup
    \immediate\openin\structurefile = \TC@first@\relax
    \ifeof\structurefile
      \PackageError{TeXshade}
      {File `\TC@first@' not found}
      {\MessageBreak
       The file you specified is missing or you have \MessageBreak
       misspelled it. \MessageBreak\MessageBreak
       No T-Coffee shading will be displayed (using "similar"). \MessageBreak
       Type <return> to proceed. \MessageBreak
       Type X <return> to quit. \MessageBreak
}
       \immediate\closein\structurefile\egroup
       \simmodetrue  \funcmodefalse \T@coffeefalse
    \else
    \message{<T-Coffee shading>}
    \xdef\first@{} \xdef\temp@{*}
    \loop
       \read\structurefile to \readline
       \xdef\test@{\expandafter\string\readline}
       \ifx\test@\par@
       \else
          \xdef\structureline{\readline & & & & & & & & &@}
          \expandafter\struc@get\structureline
          \ifx\first@\temp@ \xdef\temp@{y}
             \read\structurefile to \readline
             \read\structurefile to \readline
          \fi
       \fi
    \ifx\temp@\y@ \else\repeat
    \loopcount=0
    \xdef\temp@{cons}
    \loop
       \read\structurefile to \readline
       \xdef\test@{\expandafter\string\readline}
       \ifx\test@\par@
       \else
          \xdef\structureline{\readline & & & & & & & & &@}
          \expandafter\struc@get\structureline     
          \if\second@ : 
             \advance\loopcount by 1\relax
             \expandafter\xdef\csname T@coffee\the\loopcount\endcsname{}
          \fi
       \fi
    \ifx\first@\temp@\expandafter\xdef\csname T@coffee0\endcsname{} \else\repeat
    \xdef\temp@{\the\loopcount}
    \loopcount=0
    \loop
       \read\structurefile to \readline
       \xdef\test@{\expandafter\string\readline}
       \ifx\test@\par@
       \else
          \xdef\structureline{\readline & & & & & & & & &@}
          \expandafter\struc@get\structureline
          \advance\loopcount by 1
          \ifnum\loopcount=\temp@ \loopcount=0 \fi
          \expandafter\xdef\csname T@coffee\the\loopcount\endcsname{\csname T@coffee\the\loopcount\endcsname \third@}
       \fi
    \ifeof\structurefile \else\repeat
    \closein\structurefile
    \loopcount=0
    \loop
       \expandafter\xdef\csname T@coffee\the\loopcount\endcsname{\csname T@coffee\the\loopcount\endcsname @}
       \advance\loopcount by 1
    \ifnum\loopcount=\temp@ \else\repeat
    \egroup
    \def\n@m@tch{{$\bullet$}}
    \expandafter\def\csname fg@color0\endcsname{TC0}
    \expandafter\def\csname fg@color1\endcsname{TC1}
    \expandafter\def\csname fg@color2\endcsname{TC2}
    \expandafter\def\csname fg@color3\endcsname{TC3}
    \expandafter\def\csname fg@color4\endcsname{TC4}
    \expandafter\def\csname fg@color5\endcsname{TC5}
    \expandafter\def\csname fg@color6\endcsname{TC6}
    \expandafter\def\csname fg@color7\endcsname{TC7}
    \expandafter\def\csname fg@color8\endcsname{TC8}
    \expandafter\def\csname fg@color9\endcsname{TC9}
    \expandafter\def\csname fg@textcolor0\endcsname{Black}
    \expandafter\def\csname fg@textcolor1\endcsname{Black}
    \expandafter\def\csname fg@textcolor2\endcsname{Black}
    \expandafter\def\csname fg@textcolor3\endcsname{Black}
    \expandafter\def\csname fg@textcolor4\endcsname{Black}
    \expandafter\def\csname fg@textcolor5\endcsname{Black}
    \expandafter\def\csname fg@textcolor6\endcsname{Black}
    \expandafter\def\csname fg@textcolor7\endcsname{Black}
    \expandafter\def\csname fg@textcolor8\endcsname{Black}
    \expandafter\def\csname fg@textcolor9\endcsname{Black}
   \fi
 }

\def\include@DSSP{%
  \xdef\first@{\csname optiondssp\the\loopcount\endcsname}
  \xdef\bottop@{\csname bottopdssp\the\loopcount\endcsname}
  \xdef\st@rt{\csname doseqdssp\the\loopcount\endcsname}
  \xdef\structurefilename{\csname filenamedssp\the\loopcount\endcsname}
  \bgroup
   \xdef\file@n@me{\structurefilename .@}
   \expandafter\cut@name\file@n@me
   \xdef\file@n@me{\file@n@me.sec}
   \ifx\first@\file@n@me
   \else
      \immediate\openin\alignfile = \file@n@me\relax
      \ifeof\alignfile \xdef\first@{make new} \fi
      \immediate\closein\alignfile
   \fi
   \xdef\temp@{make new}
   \ifx\first@\temp@
    \def\par{}
    \inactivate@chars
    \immediate\openin\structurefile = \structurefilename\relax
    \ifeof\structurefile
      \PackageError{TeXshade}
      {File `\structurefilename' not found}
      {\MessageBreak
       The `DSSP' file you specified is missing or you have \MessageBreak
       misspelled it. \MessageBreak\MessageBreak
       No labels for secondary structures will be displayed. \MessageBreak
       Type <return> to proceed. \MessageBreak
       Type X <return> to quit. \MessageBreak
}
       \immediate\closein\structurefile\egroup
    \else
    \message{[\structurefilename] ->}
    \xdef\second@{} \xdef\temp@{RESIDUE}
    \loop
       \read\structurefile to \readline
       \xdef\test@{\expandafter\string\readline}
       \xdef\structureline{\readline & & & & & & & & &@}
       \expandafter\struc@get\structureline
       \ifx\second@\temp@ \xdef\temp@{AA}\fi
    \ifx\third@\temp@ \else\repeat
    \immediate\openout\featurefile = \file@n@me
    \xdef\c@mp{+}
    \xdef\begin@{\csname seq@start\st@rt\endcsname}
    \xdef\end@{\csname seq@start\st@rt\endcsname}
    \xdef\st@rt@{\begin@}
    \expandafter\innerloopcount=\csname seq@start\st@rt\endcsname
    \advance\innerloopcount by -1
    \xdef\first@@{0} \xdef\second@@{0} \xdef\third@@{0}
    \xdef\fourth@@{0} \xdef\fifth@@{0} \xdef\sixth@@{0}
    \xdef\seventh@@{0}
    \loop
       \read\structurefile to \readline
       \xdef\test@{\expandafter\string\readline}
       \ifx\test@\par@
       \else
             \xdef\structureline{\readline & & & & & & & & &@}
             \expandafter\struc@get\structureline
             \if\c@mp + \xdef\c@mp{\fifth@}\fi
             \ifx\fifth@\c@mp
                \ifx\fc@DSSP\y@
                   \temp@count=\first@
                \else
                   \temp@count=\second@
                \fi
                \advance\temp@count by \st@rt@
                \advance\temp@count by -1
                \xdef\end@{\the\temp@count}
             \else
                   \ifnum\begin@>0
                      \ifnum\end@<\begin@ \xdef\end@{\begin@}\fi
                      \if\c@mp C
                      \else
                      \if\c@mp H
                        \ifx\show@Hdssp\yes
                         \loopcount=\first@@
                         \advance\loopcount by 1
                         \xdef\first@@{\the\loopcount}
                         \ifx\m@p\yes
                         \immediate\write\featurefile{%
                            \string\feature{\bottop@Hdssp}{\begin@..\end@}%
                            {\label@Hdssp}{\text@Hdssp}}
                         \else
                         \immediate\write\featurefile{%
                            \string\feature{\bottop@Hdssp}{\st@rt}{\begin@..\end@}%
                            {\label@Hdssp}{\text@Hdssp}}\fi
                        \fi
                      \else
                      \if\c@mp G
                        \ifx\show@Gdssp\yes
                         \loopcount=\second@@
                         \advance\loopcount by 1
                         \xdef\second@@{\the\loopcount}
                         \ifx\m@p\yes
                         \immediate\write\featurefile{%
                            \string\feature{\bottop@Gdssp}{\begin@..\end@}%
                            {\label@Gdssp}{\text@Gdssp}}
                         \else
                         \immediate\write\featurefile{%
                            \string\feature{\bottop@Gdssp}{\st@rt}{\begin@..\end@}%
                            {\label@Gdssp}{\text@Gdssp}}\fi
                        \fi
                      \else
                      \if\c@mp I
                        \ifx\show@Idssp\yes
                         \loopcount=\third@@
                         \advance\loopcount by 1
                         \xdef\third@@{\the\loopcount}
                         \ifx\m@p\yes
                         \immediate\write\featurefile{%
                            \string\feature{\bottop@Idssp}{\begin@..\end@}%
                            {\label@Idssp}{\text@Idssp}}
                         \else
                         \immediate\write\featurefile{%
                            \string\feature{\bottop@Idssp}{\st@rt}{\begin@..\end@}%
                            {\label@Idssp}{\text@Idssp}}\fi
                        \fi
                      \else
                      \if\c@mp E
                        \ifx\show@Edssp\yes
                         \loopcount=\fourth@@
                         \advance\loopcount by 1
                         \xdef\fourth@@{\the\loopcount}
                         \advance\innerloopcount by 1
                         \ifx\m@p\yes
                         \immediate\write\featurefile{%
                            \string\feature{\bottop@Edssp}{\begin@..\end@}%
                            {\label@Edssp}{\text@Edssp}}
                         \else
                         \immediate\write\featurefile{%
                            \string\feature{\bottop@Edssp}{\st@rt}{\begin@..\end@}%
                            {\label@Edssp}{\text@Edssp}}\fi
                        \fi
                      \else
                      \if\c@mp B
                        \ifx\show@Bdssp\yes
                         \loopcount=\fifth@@
                         \advance\loopcount by 1
                         \xdef\fifth@@{\the\loopcount}
                         \ifx\m@p\yes
                         \immediate\write\featurefile{%
                            \string\feature{\bottop@Bdssp}{\begin@..\end@}%
                            {\label@Bdssp}{\text@Bdssp}}
                         \else
                         \immediate\write\featurefile{%
                            \string\feature{\bottop@Bdssp}{\st@rt}{\begin@..\end@}%
                            {\label@Bdssp}{\text@Bdssp}}\fi
                        \fi
                      \else
                      \if\c@mp T
                        \ifx\show@Tdssp\yes
                         \loopcount=\sixth@@
                         \advance\loopcount by 1
                         \xdef\sixth@@{\the\loopcount}
                         \ifx\m@p\yes
                         \immediate\write\featurefile{%
                            \string\feature{\bottop@Tdssp}{\begin@..\end@}%
                            {\label@Tdssp}{\text@Tdssp}}
                         \else
                         \immediate\write\featurefile{%
                            \string\feature{\bottop@Tdssp}{\st@rt}{\begin@..\end@}%
                            {\label@Tdssp}{\text@Tdssp}}\fi
                        \fi
                      \else
                      \if\c@mp S
                        \ifx\show@Sdssp\yes
                         \loopcount=\seventh@@
                         \advance\loopcount by 1
                         \xdef\seventh@@{\the\loopcount}
                         \ifx\m@p\yes
                         \immediate\write\featurefile{%
                            \string\feature{\bottop@Sdssp}{\begin@..\end@}%
                            {\label@Sdssp}{\text@Sdssp}}
                         \else
                         \immediate\write\featurefile{%
                            \string\feature{\bottop@Sdssp}{\st@rt}{\begin@..\end@}%
                            {\label@Sdssp}{\text@Sdssp}}\fi
                        \fi
                      \fi\fi\fi\fi\fi\fi\fi\fi
                   \fi
                   \xdef\c@mp{\fifth@}
                   \ifx\fc@DSSP\y@
                      \temp@count=\first@
                   \else
                      \temp@count=\second@
                   \fi
                   \advance\temp@count by \st@rt@
                   \advance\temp@count by -1
                   \xdef\begin@{\the\temp@count}
             \fi
       \fi
    \ifeof\structurefile \else\repeat
    \closein\structurefile
    \immediate\closeout\featurefile
    \egroup
    \input{\file@n@me}
   \fi
   \else
    \egroup
    \message{using existing file:}
    \input{\file@n@me}
   \fi
 }

\def\include@HMMTOP{%
 \def\get@HMMTOP@TMs##1-##2 ##3@{%
     \xdef\temp@@{\temp@@\fourth@@##1\fourth@@##2}
     \xdef\structureline{##3 @}
  }
  \def\get@HMMTOP{%
     \ifnum\temp@count<\fifth@
        \advance\temp@count by 1
        \expandafter\get@HMMTOP@TMs\structureline
        \get@HMMTOP
     \fi
  }  
  \def\rem@ve@TM@info Transmembrane helices: ##1@{%
     \xdef\structureline{##1 @}
     \temp@count=0
     \get@HMMTOP
  }
 \xdef\bottop@{\csname bottopHMMTOP\the\loopcount\endcsname}
  \xdef\st@rt{\csname doseqHMMTOP\the\loopcount\endcsname}
  \xdef\first@{\csname optionHMMTOP\the\loopcount\endcsname}
  \xdef\structurefilename{\csname filenameHMMTOP\the\loopcount\endcsname}
  \bgroup
   \xdef\file@n@me{\structurefilename .@}
   \expandafter\cut@name\file@n@me
   \xdef\file@n@me{\file@n@me.top}
   \ifx\first@\file@n@me
   \else
      \immediate\openin\alignfile = \file@n@me\relax
      \ifeof\alignfile \xdef\first@{make new} \fi
      \immediate\closein\alignfile
   \fi
   \xdef\temp@{make new}
   \ifx\first@\temp@
    \def\par{}
    \inactivate@chars
    \immediate\openin\structurefile = \structurefilename\relax
    \ifeof\structurefile
      \PackageError{TeXshade}
      {File `\structurefilename' not found}
      {\MessageBreak
       The `HMMTOP' file you specified is missing or you have \MessageBreak
       misspelled it. \MessageBreak\MessageBreak
       No labels for secondary structures will be displayed. \MessageBreak
       Type <return> to proceed. \MessageBreak
       Type X <return> to quit. \MessageBreak
}
       \immediate\closein\structurefile\egroup
    \else
    \message{[\structurefilename] ->}
    \immediate\openout\featurefile = \file@n@me
    
    \xdef\first@{\csname fileseqHMMTOP\the\loopcount\endcsname @}
    \expandafter\check@letter\first@
    \ifletter
       \xdef\st@p{\csname fileseqHMMTOP\the\loopcount\endcsname}
    \else
       \xdef\first@{\csname fileseqHMMTOP\the\loopcount\endcsname}
       \ifnum\first@=0
          \xdef\st@p{0}
       \else
          \xdef\st@p{\csname fileseqHMMTOP\the\loopcount\endcsname}
       \fi
    \fi
    \xdef\temp@{yes} \innerloopcount=0
    \loop
       \read\structurefile to \readline
       \xdef\test@{\expandafter\string\readline}
       \ifx\test@\par@
       \else
          \xdef\structureline{\readline & & & & & & & & &@}
          \expandafter\struc@get\structureline
          \ifx\first@\@HP
             \ifx\temp@\yes \xdef\temp@{\readline} \fi
             \ifletter
                \ifx\st@p\third@ \xdef\temp@{\readline} \fi
             \else
             \ifnum\st@p=0
                \expandafter\ifx\csname seqname\st@rt\endcsname\third@
                   \xdef\temp@{\readline} \xdef\st@p{-1}
                \else
                   \expandafter\ifx\csname newseqname\st@rt\endcsname\third@
                      \xdef\temp@{\readline} \xdef\st@p{-1}
	               \fi
	            \fi
	         \else
	            \advance\innerloopcount by 1
	            \ifnum\st@p=\innerloopcount \xdef\temp@{\readline} \fi
	         \fi
	         \fi
	      \else
	      \xdef\first@@{Protein:}
	      \ifx\first@\first@@
	         \xdef\second@@{\second@}
	         \xdef\temp@@{>HP:} \xdef\fourth@@{ }
             \read\structurefile to \readline
             \xdef\structureline{\readline & & & & & & & & &@}
             \expandafter\struc@get\structureline
             \xdef\temp@@{\temp@@\fourth@@\second@\fourth@@\second@@}
             \read\structurefile to \readline
             \xdef\structureline{\readline & & & & & & & & &@}
             \expandafter\struc@get\structureline
             \xdef\temp@@{\temp@@\fourth@@\second@}
             \read\structurefile to \readline
             \xdef\structureline{\readline & & & & & & & & &@}
             \expandafter\struc@get\structureline
             \xdef\temp@@{\temp@@\fourth@@\fifth@}
             \read\structurefile to \readline
             \xdef\structureline{\readline @}
             \expandafter\rem@ve@TM@info\structureline
             \xdef\temp@@{\temp@@\fourth@@}
             \ifx\temp@\yes \xdef\temp@{\temp@@} \fi
             \ifletter
                \ifx\st@p\third@ \xdef\temp@{\temp@@} \fi
             \else
             \ifnum\st@p=0
                \expandafter\ifx\csname seqname\st@rt\endcsname\second@@
                   \xdef\temp@{\temp@@}
                \else
                   \expandafter\ifx\csname newseqname\st@rt\endcsname\second@@
                      \xdef\temp@{\temp@@}
	               \fi
	            \fi
	         \else
	            \advance\innerloopcount by 1
	            \ifnum\st@p=\innerloopcount \xdef\temp@{\temp@@} \fi
	         \fi
	         \fi
	      \fi
          \fi
       \fi
    \ifeof\structurefile \else\repeat
    \xdef\seq@line{\temp@ @}
    \innerloopcount=0
    \loop
       \advance\innerloopcount by 1
       \expandafter\seq@get\seq@line
       \xdef\seq@line{\seq@line @}
       \ifnum\innerloopcount=4 \xdef\c@mp{\first@} \fi
    \ifnum\innerloopcount=5 \xdef\st@p{\first@} \else \repeat
    \xdef\first@{IN}
    \ifx\c@mp\first@ \xdef\c@mp{i} \else \xdef\c@mp{e} \fi
    \xdef\begin@{1} \xdef\first@@{0} \xdef\second@@{0}
    \innerloopcount=0
    \loop
       \advance\innerloopcount by 1
       \expandafter\seq@get\seq@line
       \xdef\seq@line{\seq@line @}
       \temp@count=\first@
       \advance\temp@count by -1
       \xdef\end@{\the\temp@count}
       \loopcount=\second@@
       \advance\loopcount by 1
       \xdef\second@@{\the\loopcount}
       \if\c@mp i
          \ifx\show@i@HMMTOP\yes
          \immediate\write\featurefile{%
             \string\feature{\bottop@i@HMMTOP}{\st@rt}{\begin@..\end@}%
             {\label@i@HMMTOP}{\text@i@HMMTOP}}\fi
          \xdef\c@mp{e}
       \else
          \ifx\show@e@HMMTOP\yes
          \immediate\write\featurefile{%
             \string\feature{\bottop@e@HMMTOP}{\st@rt}{\begin@..\end@}%
             {\label@e@HMMTOP}{\text@e@HMMTOP}}\fi
          \xdef\c@mp{i}
       \fi
       \advance\temp@count by 1
       \xdef\begin@{\the\temp@count}
       \expandafter\seq@get\seq@line
       \xdef\seq@line{\seq@line @}
       \xdef\end@{\first@}
       \loopcount=\first@@
       \advance\loopcount by 1
       \xdef\first@@{\the\loopcount}
       \ifx\show@TM@HMMTOP\yes
          \immediate\write\featurefile{%
          \string\feature{\bottop@TM@HMMTOP}{\st@rt}{\begin@..\end@}%
          {\label@TM@HMMTOP}{\text@TM@HMMTOP}}\fi
       \temp@count=\end@ \advance\temp@count by 1 \xdef\begin@{\the\temp@count}
    \ifnum\innerloopcount=\st@p\else\repeat
    \closein\structurefile
    \immediate\closeout\featurefile
    \egroup
    
    \input{\file@n@me}
   \fi
   \else
    \egroup
    \message{using existing file:}
    \input{\file@n@me}
   \fi
 }

\def\include@stride{%
  \xdef\first@{\csname optionstride\the\loopcount\endcsname}
  \xdef\bottop@{\csname bottopstride\the\loopcount\endcsname}
  \xdef\st@rt{\csname doseqstride\the\loopcount\endcsname}
  \xdef\structurefilename{\csname filenamestride\the\loopcount\endcsname}
  \bgroup
   \xdef\file@n@me{\structurefilename .@}
   \expandafter\cut@name\file@n@me
   \xdef\file@n@me{\file@n@me.sec}
   \ifx\first@\file@n@me
   \else
      \immediate\openin\alignfile = \file@n@me\relax
      \ifeof\alignfile \xdef\first@{make new} \fi
      \immediate\closein\alignfile
   \fi
   \xdef\temp@{make new}
   \ifx\first@\temp@
    \def\par{}
    \inactivate@chars
    \immediate\openin\structurefile = \structurefilename\relax
    \ifeof\structurefile
      \PackageError{TeXshade}
      {File `\structurefilename' not found}
      {\MessageBreak
       The `STRIDE' file you specified is missing or you have \MessageBreak
       misspelled it. \MessageBreak\MessageBreak
       No labels for secondary structures will be displayed. \MessageBreak
       Type <return> to proceed. \MessageBreak
       Type X <return> to quit. \MessageBreak
}
       \immediate\closein\structurefile\egroup
    \else
    \message{[\structurefilename] ->}
    \immediate\openout\featurefile = \file@n@me
    \xdef\c@mp{+}
    \xdef\begin@{\csname seq@start\st@rt\endcsname}
    \xdef\end@{\csname seq@start\st@rt\endcsname}
    \xdef\st@rt@{\begin@}
    \expandafter\innerloopcount=\csname seq@start\st@rt\endcsname
    \advance\innerloopcount by -1
    \xdef\first@@{0} \xdef\second@@{0} \xdef\third@@{0}
    \xdef\fourth@@{0} \xdef\fifth@@{0} \xdef\sixth@@{0}
    \loop
       \read\structurefile to \readline
       \xdef\test@{\expandafter\string\readline}
       \ifx\test@\par@
       \else
          \xdef\structureline{\readline & & & & & & & & &@}
          \expandafter\struc@get\structureline
          \ifx\first@\@asg
             \if\c@mp + \xdef\c@mp{\sixth@}\fi
             \ifx\sixth@\c@mp
                \temp@count=\fifth@
                \advance\temp@count by \st@rt@
                \advance\temp@count by -1
                \xdef\end@{\the\temp@count}
             \else
                   \ifnum\begin@>0
                      \ifnum\end@<\begin@ \xdef\end@{\begin@}\fi
                      \if\c@mp C
                      \else
                      \if\c@mp H
                        \ifx\show@Hstride\yes
                         \loopcount=\first@@
                         \advance\loopcount by 1
                         \xdef\first@@{\the\loopcount}
                         \ifx\m@p\yes
                         \immediate\write\featurefile{%
                            \string\feature{\bottop@Hstride}{\begin@..\end@}%
                            {\label@Hstride}{\text@Hstride}}
                         \else
                         \immediate\write\featurefile{%
                            \string\feature{\bottop@Hstride}{\st@rt}{\begin@..\end@}%
                            {\label@Hstride}{\text@Hstride}}\fi
                        \fi
                      \else
                      \if\c@mp G
                        \ifx\show@Gstride\yes
                         \loopcount=\second@@
                         \advance\loopcount by 1
                         \xdef\second@@{\the\loopcount}
                         \ifx\m@p\yes
                         \immediate\write\featurefile{%
                            \string\feature{\bottop@Gstride}{\begin@..\end@}%
                            {\label@Gstride}{\text@Gstride}}
                         \else
                         \immediate\write\featurefile{%
                            \string\feature{\bottop@Gstride}{\st@rt}{\begin@..\end@}%
                            {\label@Gstride}{\text@Gstride}}\fi
                        \fi
                      \else
                      \if\c@mp I
                        \ifx\show@Istride\yes
                         \loopcount=\third@@
                         \advance\loopcount by 1
                         \xdef\third@@{\the\loopcount}
                         \ifx\m@p\yes
                         \immediate\write\featurefile{%
                            \string\feature{\bottop@Istride}{\begin@..\end@}%
                            {\label@Istride}{\text@Istride}}
                         \else
                         \immediate\write\featurefile{%
                            \string\feature{\bottop@Istride}{\st@rt}{\begin@..\end@}%
                            {\label@Istride}{\text@Istride}}\fi
                        \fi
                      \else
                      \if\c@mp E
                        \ifx\show@Estride\yes
                         \loopcount=\fourth@@
                         \advance\loopcount by 1
                         \xdef\fourth@@{\the\loopcount}
                         \advance\innerloopcount by 1
                         \ifx\m@p\yes
                         \immediate\write\featurefile{%
                            \string\feature{\bottop@Estride}{\begin@..\end@}%
                            {\label@Estride}{\text@Estride}}
                         \else
                         \immediate\write\featurefile{%
                            \string\feature{\bottop@Estride}{\st@rt}{\begin@..\end@}%
                            {\label@Estride}{\text@Estride}}\fi
                        \fi
                      \else
                      \if\c@mp B
                        \ifx\show@Bstride\yes
                         \loopcount=\fifth@@
                         \advance\loopcount by 1
                         \xdef\fifth@@{\the\loopcount}
                         \ifx\m@p\yes
                         \immediate\write\featurefile{%
                            \string\feature{\bottop@Bstride}{\begin@..\end@}%
                            {\label@Bstride}{\text@Bstride}}
                         \else
                         \immediate\write\featurefile{%
                            \string\feature{\bottop@Bstride}{\st@rt}{\begin@..\end@}%
                            {\label@Bstride}{\text@Bstride}}\fi
                        \fi
                      \else
                      \if\c@mp T
                        \ifx\show@Tstride\yes
                         \loopcount=\sixth@@
                         \advance\loopcount by 1
                         \xdef\sixth@@{\the\loopcount}
                         \ifx\m@p\yes
                         \immediate\write\featurefile{%
                            \string\feature{\bottop@Tstride}{\begin@..\end@}%
                            {\label@Tstride}{\text@Tstride}}
                         \else
                         \immediate\write\featurefile{%
                            \string\feature{\bottop@Tstride}{\st@rt}{\begin@..\end@}%
                            {\label@Tstride}{\text@Tstride}}\fi
                        \fi
                      \fi\fi\fi\fi\fi\fi\fi
                   \fi
                   \xdef\c@mp{\sixth@}
                   \temp@count=\fifth@
                   \advance\temp@count by \st@rt@
                   \advance\temp@count by -1
                   \xdef\begin@{\the\temp@count}
             \fi
          \fi
       \fi
    \ifeof\structurefile \else\repeat
    \closein\structurefile
    \immediate\closeout\featurefile
    \egroup
    \input{\file@n@me}
   \fi
   \else
    \egroup
    \message{using existing file:}
    \input{\file@n@me}
   \fi
 }
\def\include@PHD{%
  \xdef\first@{\csname optionphd\the\loopcount\endcsname}
  \xdef\bottop@{\csname bottopphd\the\loopcount\endcsname}
  \xdef\st@rt{\csname doseqphd\the\loopcount\endcsname}
  \xdef\m@de{\csname modephd\the\loopcount\endcsname}
  \xdef\structurefilename{\csname filenamephd\the\loopcount\endcsname}
  \bgroup
   \xdef\file@n@me{\structurefilename .@}
   \expandafter\cut@name\file@n@me
   \xdef\temp@{structure}
   \ifx\m@de\temp@
      \xdef\temp@{\file@n@me .sec}
      \immediate\openin\alignfile = \temp@\relax
      \ifeof\alignfile \xdef\first@{make new} \fi
      \immediate\closein\alignfile
   \else
   \xdef\temp@{topology}
   \ifx\m@de\temp@
      \xdef\temp@{\file@n@me .top}
      \immediate\openin\alignfile = \temp@\relax
      \ifeof\alignfile \xdef\first@{make new} \fi
      \immediate\closein\alignfile
   \else
      \message{<Unknown type - ignoring \noexpand\includePHD>}
      \xdef\first@{ignore}
   \fi\fi
   \xdef\temp@{make new}
   \ifx\first@\temp@
    \def\par{}
    \xdef\PHD@line{}
    \inactivate@chars
    \immediate\openin\structurefile=\structurefilename\relax
    \ifeof\structurefile
      \PackageError{TeXshade}%
      {File `\structurefilename' not found}%
      {\MessageBreak
       The `PHD' file you specified is missing or you have \MessageBreak
       misspelled it. \MessageBreak\MessageBreak
       No labels for secondary structures will be displayed. \MessageBreak
       Type <return> to proceed. \MessageBreak
       Type X <return> to quit. \MessageBreak
}
       \immediate\closein\structurefile\egroup
    \else
    \message{[\structurefilename] ->}
    \loop
       \read\structurefile to \readline
       \xdef\structureline{\readline & & & & & & & & &@}
       \expandafter\struc@get\structureline
       \xdef\temp@{structure}
       \ifx\m@de\temp@
          \xdef\temp@{SUB}
          \ifx\temp@\first@
             \xdef\temp@{sec}
             \ifx\temp@\second@
                \xdef\third@{\third@ @}
                \expandafter\get@PHD\third@
             \fi
          \else
             \ifx\temp@\second@
                \xdef\temp@{sec}
                \ifx\temp@\third@
                   \xdef\fourth@{\fourth@ @}
                   \expandafter\get@PHD\fourth@
                \fi
             \fi
          \fi
       \else
       \xdef\temp@{topology}
       \ifx\m@de\temp@
          \xdef\temp@{PHDThtm}
          \ifx\temp@\first@
             \xdef\second@{\second@ @}
             \expandafter\get@PHD\second@
          \fi
       \fi\fi
    \ifeof\structurefile \else\repeat
    \closein\structurefile
    \xdef\c@mp{+}
    \xdef\begin@{\csname seq@start\st@rt\endcsname}
    \xdef\end@{\csname seq@start\st@rt\endcsname}
    \expandafter\innerloopcount=\csname seq@start\st@rt\endcsname
    \advance\innerloopcount by -1
    \xdef\first@{0} \xdef\second@{0} \xdef\third@{0}
    \xdef\fourth@{0} \xdef\fifth@{0} \xdef\sixth@{0}
    \xdef\last@{\PHD@line &@}
    \xdef\temp@{structure}
    \ifx\temp@\m@de
       \xdef\file@n@me{\file@n@me .sec}
       \immediate\openout\featurefile = \file@n@me\relax
       \write@PHDsec
    \fi
    \xdef\temp@{topology}
    \ifx\temp@\m@de
       \xdef\file@n@me{\file@n@me .top}
       \immediate\openout\featurefile = \file@n@me\relax
       \write@PHDtopo
    \fi
    \immediate\closeout\featurefile
    \egroup
    \input{\file@n@me}
   \fi
   \else
    \egroup
    \xdef\temp@{ignore}
    \ifx\temp@\first@
    \else
       \message{using existing file:}
       \xdef\temp@{structure}
       \ifx\temp@\m@de \xdef\file@n@me{\file@n@me .sec} \fi
       \xdef\temp@{topology}
       \ifx\temp@\m@de \xdef\file@n@me{\file@n@me .top} \fi
       \input{\file@n@me}
    \fi
   \fi}
\def\show@DSSP{%
   \expandafter\get@item\first@
   \ifx\fourth@\ampers@nd
   \else
      \xdef\temp@{alpha}
      \ifx\fourth@\temp@ \xdef\show@Hdssp{\second@}\fi
      \xdef\temp@{3-10}
      \ifx\fourth@\temp@ \xdef\show@Gdssp{\second@}\fi
      \xdef\temp@{pi}
      \ifx\fourth@\temp@ \xdef\show@Idssp{\second@}\fi
      \xdef\temp@{beta}
      \ifx\fourth@\temp@ \xdef\show@Edssp{\second@}\fi
      \xdef\temp@{bridge}
      \ifx\fourth@\temp@ \xdef\show@Bdssp{\second@}\fi
      \xdef\temp@{turn}
      \ifx\fourth@\temp@ \xdef\show@Tdssp{\second@}\fi
      \xdef\temp@{bend}
      \ifx\fourth@\temp@ \xdef\show@Sdssp{\second@}\fi
      \show@DSSP
   \fi}
\def\show@HMMTOP{%
   \expandafter\get@item\first@
   \ifx\fourth@\ampers@nd
   \else
      \xdef\temp@{internal}
      \ifx\fourth@\temp@ \xdef\show@i@HMMTOP{\second@}\fi
      \xdef\temp@{external}
      \ifx\fourth@\temp@ \xdef\show@e@HMMTOP{\second@}\fi
      \xdef\temp@{TM}
      \ifx\fourth@\temp@ \xdef\show@TM@HMMTOP{\second@}\fi
      \show@HMMTOP
   \fi}
\def\show@STRIDE{%
   \expandafter\get@item\first@
   \ifx\fourth@\ampers@nd
   \else
      \xdef\temp@{alpha}
      \ifx\fourth@\temp@ \xdef\show@Hstride{\second@}\fi
      \xdef\temp@{3-10}
      \ifx\fourth@\temp@ \xdef\show@Gstride{\second@}\fi
      \xdef\temp@{pi}
      \ifx\fourth@\temp@ \xdef\show@Istride{\second@}\fi
      \xdef\temp@{beta}
      \ifx\fourth@\temp@ \xdef\show@Estride{\second@}\fi
      \xdef\temp@{bridge}
      \ifx\fourth@\temp@ \xdef\show@Bstride{\second@}\fi
      \xdef\temp@{turn}
      \ifx\fourth@\temp@ \xdef\show@Tstride{\second@}\fi
      \show@STRIDE
   \fi}
\def\show@PHDtopo{%
   \expandafter\get@item\first@
   \ifx\fourth@\ampers@nd
   \else
      \xdef\temp@{internal}
      \ifx\fourth@\temp@ \xdef\show@itop{\second@}\fi
      \xdef\temp@{external}
      \ifx\fourth@\temp@ \xdef\show@etop{\second@}\fi
      \xdef\temp@{TM}
      \ifx\fourth@\temp@ \xdef\show@TMtop{\second@}\fi
      \show@PHDtopo
   \fi}
\def\show@PHDsec{%
   \expandafter\get@item\first@
   \ifx\fourth@\ampers@nd
   \else
      \xdef\temp@{alpha}
      \ifx\fourth@\temp@ \xdef\show@Hsec{\second@}\fi
      \xdef\temp@{beta}
      \ifx\fourth@\temp@ \xdef\show@Esec{\second@}\fi
      \show@PHDsec
   \fi}
\def\get@triplet#1,#2@{%
   \xdef\third@{#1}
   \ifx\third@\ampers@nd
   \else
      \expandafter\xdef\csname @\third@\endcsname{\first@}
      \expandafter\xdef\csname rev@\first@\endcsname{\third@}
  \xdef\fourth@{#2,&,@}
  \expandafter\get@triplet\fourth@
   \fi}
\def\get@dom@count#1,#2@{\res@count=#1 \xdef\temp@{#2 @}}   
\def\get@name@number{%
   \xdef\second@{n}
   \loopcount=0
   \loop
      \advance\loopcount by 1
      \expandafter\ifx\csname newseqname\the\loopcount\endcsname\first@
         \xdef\first@{\the\loopcount} \loopcount=\seq@count 
         \xdef\second@{y}
      \fi
   \ifnum\loopcount=\seq@count \else \repeat
   \ifx\second@\n@
      \message{<Sequence name `\first@' was not found - using first sequence.>}
      \xdef\first@{1}
   \fi
}
\def\get@name@number@table{%
   \xdef\second@{n}%
   \loopcount=0%
   \loop%
      \advance\loopcount by 1%
      \expandafter\ifx\csname newseqname\the\loopcount\endcsname\first@%
         \xdef\first@{\the\loopcount}\loopcount=\seq@num%
         \xdef\second@{y}%
      \fi%
   \ifnum\loopcount=\seq@num\else\repeat%
   \ifx\second@\n@%
      \message{<Sequence name `\first@' was not found - using first sequence.>}%
      \xdef\first@{1}%
   \fi%
}
 
\def\calc@distance@point{%
   \loopcount=\xref@coord@a \advance\loopcount by -\x@coord\relax 
   \multiply\loopcount by \loopcount \xdef\x@distance{\the\loopcount}
   \loopcount=\yref@coord@a \advance\loopcount by -\y@coord\relax 
   \multiply\loopcount by \loopcount \xdef\y@distance{\the\loopcount}
   \loopcount=\zref@coord@a \advance\loopcount by -\z@coord\relax 
   \multiply\loopcount by \loopcount \xdef\z@distance{\the\loopcount}
   \advance\loopcount by \y@distance\relax
   \advance\loopcount by \x@distance\relax
   \ifnum\loopcount > \PDB@distance \else 
      \loopcount=\PDB@hitpos \advance\loopcount by 1 
      \ifnum\loopcount=\PDB@pos
      \else
         \xdef\PDB@stack{\PDB@stack\PDB@hitpos,\PDB@pos..}
      \fi
      \xdef\PDB@hitpos{\PDB@pos}
   \fi
}

\def\calc@distance@line{%
   \loopcount=\x@coord\relax \multiply\loopcount by \x@orient
   \temp@count=\y@coord \relax \multiply\temp@count by \y@orient
   \advance\loopcount by \temp@count 
   \temp@count=\z@coord \relax \multiply\temp@count by \z@orient
   \advance\loopcount by \temp@count 
   \advance\loopcount by -\c@nst@ 
   \xdef\z@hler{\the\loopcount}
   \multiply\loopcount by \x@orient \divide\loopcount by \f@ct@r \advance\loopcount by \xref@coord@a
   \advance\loopcount by -\x@coord
   \multiply\loopcount by \loopcount \xdef\x@distance{\the\loopcount}
   \loopcount=\z@hler   
   \multiply\loopcount by \y@orient \divide\loopcount by \f@ct@r \advance\loopcount by \yref@coord@a
   \advance\loopcount by -\y@coord
   \multiply\loopcount by \loopcount \xdef\y@distance{\the\loopcount}
   \loopcount=\z@hler   
   \multiply\loopcount by \z@orient \divide\loopcount by \f@ct@r \advance\loopcount by \zref@coord@a
   \advance\loopcount by -\z@coord
   \multiply\loopcount by \loopcount \xdef\z@distance{\the\loopcount}
   \advance\loopcount by \y@distance\relax
   \advance\loopcount by \x@distance\relax
   \ifnum\loopcount > \PDB@distance \else 
      \loopcount=\PDB@hitpos \advance\loopcount by 1 
      \ifnum\loopcount=\PDB@pos
      \else
         \xdef\PDB@stack{\PDB@stack\PDB@hitpos,\PDB@pos..}
      \fi
      \xdef\PDB@hitpos{\PDB@pos}
   \fi
}


\def\intr@@t{%
   \loopcount=\r@@t
   \temp@count=1
   \loop
      \advance\loopcount by \temp@count
      \divide\loopcount by 2
      \temp@count=\r@@t
      \divide\temp@count by \loopcount
   \ifnum\loopcount>\temp@count\repeat
}


\def\calc@distance@plane{%
   \loopcount=\x@coord\relax
   \multiply\loopcount by \x@orient
   \temp@count=\y@coord \relax
   \multiply\temp@count by \y@orient
   \advance\loopcount by \temp@count 
   \temp@count=\z@coord \relax
   \multiply\temp@count by \z@orient
   \advance\loopcount by \temp@count
   \advance\loopcount by \c@nst@
   \divide\loopcount by \f@ct@r  
   \multiply\loopcount by \loopcount
   \ifnum\loopcount > \PDB@distance \else 
      \loopcount=\PDB@hitpos \advance\loopcount by 1 
      \ifnum\loopcount=\PDB@pos
      \else
         \xdef\PDB@stack{\PDB@stack\PDB@hitpos,\PDB@pos..}
      \fi
      \xdef\PDB@hitpos{\PDB@pos}
   \fi
}

   
\def\load@PDB{%
    \bgroup
    \immediate\openin\structurefile = \PDB@name\relax
    \ifeof\structurefile
      \PackageError{TeXshade}
      {File `\PDB@name' not found}
      {\MessageBreak
       The file you specified is missing or you have \MessageBreak
       misspelled it. \MessageBreak\MessageBreak
       No PDB domain labeling will be displayed. \MessageBreak
       Type <return> to proceed. \MessageBreak
       Type X <return> to quit. \MessageBreak
}
       \immediate\closein\structurefile\egroup
    \else
    \message{<reading PDB: \PDB@name>}
    \xdef\first@{}
    \loop
       \read\structurefile to \readline
       \xdef\test@{\expandafter\string\readline}
       \ifx\test@\par@
       \else
          \xdef\structureline{\readline & & & & & & & & &@}
          \expandafter\struc@get\structureline
          \ifx\first@\@TOM 
             \ifx\PDB@back@side@a\C@lpha
                \ifx\third@\C@lpha
                   \xdef\temp@@@{\fifth@ &&@}
                   \expandafter\check@letter\temp@@@
                   \ifnumber
                      \ifnum\fifth@=\PDB@refnum@a
                         \xdef\sixth@{\sixth@ &.&&@} \expandafter\coord@get\sixth@ \xdef\xref@coord@a{\c@@rd}
                         \xdef\seventh@{\seventh@ &.&&@} \expandafter\coord@get\seventh@ \xdef\yref@coord@a{\c@@rd}
                         \xdef\eighth@{\eighth@ &.&&@} \expandafter\coord@get\eighth@ \xdef\zref@coord@a{\c@@rd}
                      \fi
                   \else
                      \ifnum\sixth@=\PDB@refnum@a
                         \xdef\seventh@{\seventh@ &.&&@} \expandafter\coord@get\seventh@ \xdef\xref@coord@a{\c@@rd}
                         \xdef\eighth@{\eighth@ &.&&@} \expandafter\coord@get\eighth@ \xdef\yref@coord@a{\c@@rd}
                         \xdef\nineth@{\nineth@ &.&&@} \expandafter\coord@get\nineth@ \xdef\zref@coord@a{\c@@rd}
                      \fi
                   \fi
                \fi
             \else
                \xdef\temp@@@{\fifth@ &&@}
                \expandafter\check@letter\temp@@@
                \ifnumber
                   \ifnum\fifth@=\PDB@refnum@a
                      \xdef\sixth@{\sixth@ &.&&@} \expandafter\coord@get\sixth@ \xdef\xref@coord@a{\c@@rd}
                      \xdef\seventh@{\seventh@ &.&&@} \expandafter\coord@get\seventh@ \xdef\yref@coord@a{\c@@rd}
                      \xdef\eighth@{\eighth@ &.&&@} \expandafter\coord@get\eighth@ \xdef\zref@coord@a{\c@@rd}
                   \fi
                \else
                   \ifnum\sixth@=\PDB@refnum@a
                      \xdef\seventh@{\seventh@ &.&&@} \expandafter\coord@get\seventh@ \xdef\xref@coord@a{\c@@rd}
                      \xdef\eighth@{\eighth@ &.&&@} \expandafter\coord@get\eighth@ \xdef\yref@coord@a{\c@@rd}
                      \xdef\nineth@{\nineth@ &.&&@} \expandafter\coord@get\nineth@ \xdef\zref@coord@a{\c@@rd}
                   \fi
                \fi
             \fi
             \ifx\PDB@back@side@b\C@lpha
                \ifx\third@\C@lpha
                   \xdef\temp@@@{\fifth@ &&@}
                   \expandafter\check@letter\temp@@@
                   \ifnumber
                      \ifnum\fifth@=\PDB@refnum@b
                         \xdef\sixth@{\sixth@ &.&&@} \expandafter\coord@get\sixth@ \xdef\xref@coord@b{\c@@rd}
                         \xdef\seventh@{\seventh@ &.&&@} \expandafter\coord@get\seventh@ \xdef\yref@coord@b{\c@@rd}
                         \xdef\eighth@{\eighth@ &.&&@} \expandafter\coord@get\eighth@ \xdef\zref@coord@b{\c@@rd}
                      \fi
                   \else
                      \ifnum\sixth@=\PDB@refnum@b
                         \xdef\seventh@{\seventh@ &.&&@} \expandafter\coord@get\seventh@ \xdef\xref@coord@b{\c@@rd}
                         \xdef\eighth@{\eighth@ &.&&@} \expandafter\coord@get\eighth@ \xdef\yref@coord@b{\c@@rd}
                         \xdef\nineth@{\nineth@ &.&&@} \expandafter\coord@get\nineth@ \xdef\zref@coord@b{\c@@rd}
                      \fi
                   \fi
                \fi
             \else
                \xdef\temp@@@{\fifth@ &&@}
                \expandafter\check@letter\temp@@@
                \ifnumber
                   \ifnum\fifth@=\PDB@refnum@b
                      \xdef\sixth@{\sixth@ &.&&@} \expandafter\coord@get\sixth@ \xdef\xref@coord@b{\c@@rd}
                      \xdef\seventh@{\seventh@ &.&&@} \expandafter\coord@get\seventh@ \xdef\yref@coord@b{\c@@rd}
                      \xdef\eighth@{\eighth@ &.&&@} \expandafter\coord@get\eighth@ \xdef\zref@coord@b{\c@@rd}
                   \fi
                \else
                   \ifnum\sixth@=\PDB@refnum@b
                      \xdef\seventh@{\seventh@ &.&&@} \expandafter\coord@get\seventh@ \xdef\xref@coord@b{\c@@rd}
                      \xdef\eighth@{\eighth@ &.&&@} \expandafter\coord@get\eighth@ \xdef\yref@coord@b{\c@@rd}
                      \xdef\nineth@{\nineth@ &.&&@} \expandafter\coord@get\nineth@ \xdef\zref@coord@b{\c@@rd}
                   \fi
                \fi
             \fi
             \ifx\PDB@back@side@c\C@lpha
                \ifx\third@\C@lpha
                   \xdef\temp@@@{\fifth@ &&@}
                   \expandafter\check@letter\temp@@@
                   \ifnumber
                      \ifnum\fifth@=\PDB@refnum@c
                         \xdef\sixth@{\sixth@ &.&&@} \expandafter\coord@get\sixth@ \xdef\xref@coord@c{\c@@rd}
                         \xdef\seventh@{\seventh@ &.&&@} \expandafter\coord@get\seventh@ \xdef\yref@coord@c{\c@@rd}
                         \xdef\eighth@{\eighth@ &.&&@} \expandafter\coord@get\eighth@ \xdef\zref@coord@c{\c@@rd}
                      \fi
                   \else
                      \ifnum\sixth@=\PDB@refnum@c
                         \xdef\seventh@{\seventh@ &.&&@} \expandafter\coord@get\seventh@ \xdef\xref@coord@c{\c@@rd}
                         \xdef\eighth@{\eighth@ &.&&@} \expandafter\coord@get\eighth@ \xdef\yref@coord@c{\c@@rd}
                         \xdef\nineth@{\nineth@ &.&&@} \expandafter\coord@get\nineth@ \xdef\zref@coord@c{\c@@rd}
                      \fi
                   \fi
                \fi
             \else
                \xdef\temp@@@{\fifth@ &&@}
                \expandafter\check@letter\temp@@@
                \ifnumber
                   \ifnum\fifth@=\PDB@refnum@c
                      \xdef\sixth@{\sixth@ &.&&@} \expandafter\coord@get\sixth@ \xdef\xref@coord@c{\c@@rd}
                      \xdef\seventh@{\seventh@ &.&&@} \expandafter\coord@get\seventh@ \xdef\yref@coord@c{\c@@rd}
                      \xdef\eighth@{\eighth@ &.&&@} \expandafter\coord@get\eighth@ \xdef\zref@coord@c{\c@@rd}
                   \fi
                \else
                   \ifnum\sixth@=\PDB@refnum@c
                      \xdef\seventh@{\seventh@ &.&&@} \expandafter\coord@get\seventh@ \xdef\xref@coord@c{\c@@rd}
                      \xdef\eighth@{\eighth@ &.&&@} \expandafter\coord@get\eighth@ \xdef\yref@coord@c{\c@@rd}
                      \xdef\nineth@{\nineth@ &.&&@} \expandafter\coord@get\nineth@ \xdef\zref@coord@c{\c@@rd}
                   \fi
                \fi
             \fi
          \fi
       \fi
    \ifeof\structurefile \else\repeat
    \closein\structurefile
    \ifx\PDB@type\@line@
       \loopcount=\xref@coord@a \advance\loopcount by -\xref@coord@b \xdef\x@orient{\the\loopcount}
       \loopcount=\yref@coord@a \advance\loopcount by -\yref@coord@b \xdef\y@orient{\the\loopcount}
       \loopcount=\zref@coord@a \advance\loopcount by -\zref@coord@b \xdef\z@orient{\the\loopcount}
       \loopcount=\xref@coord@a \multiply\loopcount by \x@orient
       \temp@count=\yref@coord@a \multiply\temp@count by \y@orient
       \advance\loopcount by \temp@count
       \temp@count=\zref@coord@a \multiply\temp@count by \z@orient
       \advance\loopcount by \temp@count 
       \xdef\c@nst@{\the\loopcount}
       \loopcount=\x@orient \multiply\loopcount by \x@orient
       \temp@count=\y@orient \multiply\temp@count by \y@orient
       \advance\loopcount by \temp@count
       \temp@count=\z@orient \multiply\temp@count by \z@orient
       \advance\loopcount by \temp@count
       \xdef\f@ct@r{\the\loopcount}
    \fi
    \ifx\PDB@type\@plane@
       \loopcount=\xref@coord@a \advance\loopcount by -\xref@coord@b \xdef\x@orient@r{\the\loopcount}
       \loopcount=\yref@coord@a \advance\loopcount by -\yref@coord@b \xdef\y@orient@r{\the\loopcount}
       \loopcount=\zref@coord@a \advance\loopcount by -\zref@coord@b \xdef\z@orient@r{\the\loopcount}
       \loopcount=\xref@coord@a \advance\loopcount by -\xref@coord@c \xdef\x@orient@s{\the\loopcount}
       \loopcount=\yref@coord@a \advance\loopcount by -\yref@coord@c \xdef\y@orient@s{\the\loopcount}
       \loopcount=\zref@coord@a \advance\loopcount by -\zref@coord@c \xdef\z@orient@s{\the\loopcount}
       \loopcount=\y@orient@r \multiply\loopcount by \z@orient@s
       \temp@count=\z@orient@r \multiply\temp@count by \y@orient@s
       \advance\loopcount by -\temp@count \divide\loopcount by 100 \xdef\x@orient{\the\loopcount}
       \loopcount=\z@orient@r \multiply\loopcount by \x@orient@s
       \temp@count=\x@orient@r \multiply\temp@count by \z@orient@s
       \advance\loopcount by -\temp@count \divide\loopcount by 100 \xdef\y@orient{\the\loopcount}
       \loopcount=\x@orient@r \multiply\loopcount by \y@orient@s
       \temp@count=\y@orient@r \multiply\temp@count by \x@orient@s
       \advance\loopcount by -\temp@count \divide\loopcount by 100 \xdef\z@orient{\the\loopcount}
       \loopcount=\x@orient \multiply\loopcount by \xref@coord@a
       \temp@count=\y@orient \multiply\temp@count by \yref@coord@a
       \advance\loopcount by \temp@count
       \temp@count=\z@orient \multiply\temp@count by \zref@coord@a
       \advance\loopcount by \temp@count 
       \multiply\loopcount by -1\relax 
       \xdef\c@nst@{\the\loopcount}
       \loopcount=\x@orient \multiply\loopcount by \loopcount
       \temp@count=\y@orient \multiply\temp@count by \temp@count
       \advance\loopcount by \temp@count
       \temp@count=\z@orient \multiply\temp@count by \temp@count
       \advance\loopcount by \temp@count
       \xdef\r@@t{\the\loopcount} \intr@@t
       \xdef\f@ct@r{\the\loopcount}
    \fi
    \immediate\openin\structurefile = \PDB@name\relax
    \xdef\first@{}
    \loop
       \read\structurefile to \readline
       \xdef\test@{\expandafter\string\readline}
       \ifx\test@\par@
       \else
          \xdef\structureline{\readline & & & & & & & & &@}
          \expandafter\struc@get\structureline
          \ifx\first@\@TOM
                \xdef\temp@@@{\fifth@ &&@}
                \expandafter\check@letter\temp@@@
                \ifnumber
                      \xdef\PDB@pos{\fifth@}
                      \xdef\sixth@{\sixth@ &.&&@} \expandafter\coord@get\sixth@ \xdef\x@coord{\c@@rd}
                      \xdef\seventh@{\seventh@ &.&&@} \expandafter\coord@get\seventh@ \xdef\y@coord{\c@@rd}
                      \xdef\eighth@{\eighth@ &.&&@} \expandafter\coord@get\eighth@ \xdef\z@coord{\c@@rd}
                \else
                      \xdef\PDB@pos{\sixth@}
                      \xdef\seventh@{\seventh@ &.&&@} \expandafter\coord@get\seventh@ \xdef\x@coord{\c@@rd}
                      \xdef\eighth@{\eighth@ &.&&@} \expandafter\coord@get\eighth@ \xdef\y@coord{\c@@rd}
                      \xdef\nineth@{\nineth@ &.&&@} \expandafter\coord@get\nineth@ \xdef\z@coord{\c@@rd}
                \fi
                \ifnum\PDB@hitpos=\PDB@pos
                \else
                   \ifx\PDB@type\@point@
                      \calc@distance@point
                   \else
                      \ifx\PDB@type\@line@
                         \calc@distance@line
                      \else
                         \ifx\PDB@type\@plane@
                            \calc@distance@plane
                         \fi
                      \fi
                   \fi
                \fi
          \fi
       \fi
    \ifeof\structurefile \xdef\PDB@stack{\PDB@stack\PDB@hitpos,&@}
    \else\repeat
    \closein\structurefile
    
   \fi
   \egroup
 }
 


%%%%% Definition of user commands

\def\clearfuncgroups{\xdef\prfx{func} \clear@groups \xdef\fgroup@num{0}}
\clearfuncgroups
\def\germanlanguage{\germ@ntrue \sp@nishfalse \def\cons@name{Konsensus}}
\def\spanishlanguage{\germ@nfalse \sp@nishtrue \def\cons@name{consenso}}
\def\englishlanguage{\germ@nfalse \sp@nishfalse \def\cons@name{consensus}}
\def\showlegend{\legend@true}
\def\hidelegend{\legend@false}
\def\movelegend#1#2{%
   \setlength\hspace@legend{#1}
   \setlength\vspace@legend{#2}
}
\newcommand{\showcaption}[2][bottom]{\def\cap@pos{#1}\def\c@p{#2}}
\def\shortcaption#1{\def\c@pshort{#1}}
\def\funcgroup#1#2#3#4#5#6{%
   \xdef\first@{#1}
   \loopcount=0 \innerloopcount=0
   \loop
      \advance\loopcount by 1
      \ifx\csname fgroup@name\the\loopcount\endcsname\first@
          \innerloopcount=\loopcount \loopcount=\fgroup@num
      \fi
   \ifnum\loopcount<\fgroup@num \repeat
   \ifnum\innerloopcount=0
     \ifnum\fgroup@num<9
      \innerloopcount=\fgroup@num
      \advance\innerloopcount by 1 \xdef\fgroup@num{\the\innerloopcount}
     \else \message{<Too many \noexpand\funcgroups>}
     \fi
   \fi
   \ifnum\innerloopcount>0
      \expandafter\xdef\csname fgroup@name\the\innerloopcount\endcsname{\first@}
      \expandafter\xdef\csname fg@textcolor\the\innerloopcount\endcsname{#3}
      \expandafter\xdef\csname fg@color\the\innerloopcount\endcsname{#4}
      \expandafter\xdef\csname funcm@tch\the\innerloopcount\endcsname{#5}
      \expandafter\def\csname func@style\the\innerloopcount\endcsname{%
          \csname text#6\endcsname}
      \xdef\prfx{func}
      \xdef\third@{#2&,@} \loopcount=\innerloopcount
      \expandafter\group@get\third@
  \fi}
\def\pepgroups#1{%
   \xdef\prfx{pep}
    \clear@groups
    \xdef\third@{#1&,@} \loopcount=0
    \loop \expandafter\group@get\third@ \advance\loopcount by 1
    \ifnum\loopcount<10 \repeat}
\def\DNAgroups#1{%
   \xdef\prfx{DNA}
    \clear@groups
    \xdef\third@{#1&,@} \loopcount=0
    \loop \expandafter\group@get\third@ \advance\loopcount by 1
    \ifnum\loopcount<10 \repeat}
\def\pepsims#1#2{\xdef\prfx{pep}
   \def\sim@set{\expandafter\residue@get\second@
                 \ifx\first@\ampers@nd
                 \else \advance\innerloopcount by 1
                 
                 \expandafter\xdef\csname simpair\third@\first@\endcsname{1}
                 \expandafter\xdef\csname simpair\first@\third@\endcsname{1}
                 
                 \xdef\second@{\csname sequence\the\loopcount\endcsname} \sim@set
                 \fi}
   \xdef\first@{#1} \make@upper \xdef\third@{\first@}
   \xdef\last@{#2} \xdef\second@{#2 &@} \innerloopcount=0 \sim@set
   \expandafter\xdef\csname \prfx sim\third@\endcsname{%
               (\the\innerloopcount)\last@}}
\def\DNAsims#1#2{\xdef\prfx{DNA}
   \def\sim@set{\expandafter\residue@get\second@
                 \ifx\first@\ampers@nd
                 \else \advance\innerloopcount by 1
                 \xdef\second@{\csname sequence\the\loopcount\endcsname} \sim@set
                 \fi}
   \xdef\first@{#1} \make@upper \xdef\third@{\first@}
   \xdef\last@{#2} \xdef\second@{#2 &@} \innerloopcount=0 \sim@set
   \expandafter\xdef\csname \prfx sim\third@\endcsname{%
               (\the\innerloopcount)\last@}}
\def\fingerprint#1{%
   \ifnum #1 >0
   \residuesperline*{#1}
   \def\finger@linenum{#1}
   \shownames{left}
   \hidenumbering
   \rulersteps{100}
   \nomatchresidues{}{Gray10}{}{}
   \loopcount=0
   \loop
      \advance\loopcount by 1
      \separationline{\the\loopcount}
   \ifnum\loopcount<\seq@count\repeat
   \fi}
\def\printPDBlist#1{%
     \xdef\list@{}
     \xdef\temp@{#1,,,:,,,,@} \expandafter\test@PDB\temp@
     \xdef\first@{\list@ &}
     \ifx\first@\ampers@nd
     \else
        \xdef\first@{\list@ @} 
        \loop
           \expandafter\get@item\first@
           \fourth@
        \ifx\first@@\ampers@nd\else{,\ }\repeat
     \fi
}
\def\messagePDBlist#1{%
     \xdef\list@{}
     \xdef\temp@{#1,,,:,,,,@} \expandafter\test@PDB\temp@
     \xdef\first@{\list@ &}
     \ifx\first@\ampers@nd
     \else
        \message{(#1:}
        \xdef\first@{\list@ @}%
        \loop%
           \expandafter\get@item\first@%
           \message{\fourth@}%
        \ifx\first@@\ampers@nd\else\repeat%
        \message{)}
     \fi
}
\def\shaderegion#1#2#3#4{%
   \regionalshadetrue
   \xdef\seq@{#1}
   \xdef\temp@{consensus} \ifx\seq@\temp@ \xdef\seq@{0} \fi
   \xdef\first@{\seq@ @} \expandafter\check@letter\first@
   \xdef\first@{\seq@}
   \ifletter \get@name@number \xdef\seq@{\first@} \fi
   \ifnum\seq@>\seq@count
     \message{<Ignored `\seq@' in \noexpand\shaderegion or \noexpand\shadeblock>}
   \else
     \ifx\@ll\yes \xdef\@ll{y}\else \xdef\@ll{n}\fi
     \loopcount=\seq@regions
     \advance\loopcount by 1
     \xdef\seq@regions{\the\loopcount}
     \expandafter\xdef\csname fgseqregion\the\loopcount\endcsname{#3}
     \expandafter\xdef\csname bgseqregion\the\loopcount\endcsname{#4}
     \xdef\list@{#2,&}
     \xdef\temp@@{shade}
     \xdef\temp@{#2,,,:,,,,@} \expandafter\test@PDB\temp@
     \ifx\list@\ampers@nd
     \else
        \loop
           \xdef\list@{\list@ @}
           \expandafter\get@regions\list@
        \ifx\list@\ampers@nd\else\repeat
     \fi
   \fi
}
\def\shadeblock#1#2#3#4{%
   \xdef\seq@{#1}
      \xdef\@ll{yes}
      \shaderegion{#1}{#2}{#3}{#4}
      \xdef\@ll{}
}
\def\tintregion#1#2{%
   \xdef\seq@{#1}
   \xdef\temp@{consensus} \ifx\seq@\temp@ \xdef\seq@{0} \fi
   \xdef\first@{\seq@ @} \expandafter\check@letter\first@
   \xdef\first@{\seq@}
   \ifletter \get@name@number \xdef\seq@{\first@} \fi
   \ifnum\seq@>\seq@count
     \message{<Ignored seq `\seq@' in \noexpand\tintregion or \noexpand\tintblock>}
   \else
     \ifx\@ll\yes \xdef\@ll{y}\else \xdef\@ll{n}\fi
     \xdef\list@{#2,&}
     \xdef\temp@@{tint} \loopcount=0
     \xdef\temp@{#2,,,:,,,,@} \expandafter\test@PDB\temp@
     \ifx\list@\ampers@nd
     \else
        \loop
           \xdef\list@{\list@ @}
           \expandafter\get@tintregions\list@
        \ifx\list@\ampers@nd\else\repeat
     \fi
   \fi
}
\def\tintblock#1#2{%
   \xdef\seq@{#1}
      \xdef\@ll{yes}
      \tintregion{#1}{#2}
      \xdef\@ll{}
}
\def\tintdefault#1{%
   \xdef\first@{#1}
   \xdef\second@{strong}
   \ifx\first@\second@
      \xdef\light@{LightLightLight}
   \else
   \xdef\second@{medium}
   \ifx\first@\second@
      \xdef\light@{LightLight}
   \else
   \xdef\second@{weak}
   \ifx\first@\second@
      \xdef\light@{Light}
   \else
      \xdef\light@{LightLight}
   \fi\fi\fi
}
\def\lowerregion#1#2{%
   \xdef\seq@{#1}
   \xdef\temp@{consensus} \ifx\seq@\temp@ \xdef\seq@{0} \fi
   \xdef\first@{\seq@ @} \expandafter\check@letter\first@
   \xdef\first@{\seq@}
   \ifletter \get@name@number \xdef\seq@{\first@} \fi
   \ifnum\seq@>\seq@count
     \message{<Ignored seq `\seq@' in \noexpand\lowerregion or \noexpand\lowerblock>}
   \else
     \ifx\@ll\yes \xdef\@ll{y}\else \xdef\@ll{n}\fi
     \xdef\list@{#2,&}
     \xdef\temp@@{lower} \loopcount=0
     \xdef\temp@{#2,,,:,,,,@} \expandafter\test@PDB\temp@
     \ifx\list@\ampers@nd
     \else
        \loop
           \xdef\list@{\list@ @}
           \expandafter\get@lowerregions\list@
        \ifx\list@\ampers@nd\else\repeat
     \fi
   \fi
}
\def\lowerblock#1#2{%
   \xdef\seq@{#1}
      \xdef\@ll{yes}
      \lowerregion{#1}{#2}
      \xdef\@ll{}
}
\def\emphregion#1#2{%
   \xdef\seq@{#1}
   \xdef\temp@{consensus} \ifx\seq@\temp@ \xdef\seq@{0} \fi
   \xdef\first@{\seq@ @} \expandafter\check@letter\first@
   \xdef\first@{\seq@}
   \ifletter \get@name@number \xdef\seq@{\first@} \fi
   \ifnum\seq@>\seq@count
     \message{<Ignored seq `\seq@' in \noexpand\emphregion or \noexpand\emphblock>}
   \else
     \ifx\@ll\yes \xdef\@ll{y}\else \xdef\@ll{n}\fi
     \xdef\list@{#2,&}
     \xdef\temp@@{emph} \loopcount=0
     \xdef\temp@{#2,,,:,,,,@} \expandafter\test@PDB\temp@
     \ifx\list@\ampers@nd
     \else
        \loop
           \xdef\list@{\list@ @}
           \expandafter\get@emphregions\list@
        \ifx\list@\ampers@nd\else\repeat
     \fi
   \fi
}
\def\emphblock#1#2{%
   \xdef\seq@{#1}
      \xdef\@ll{yes}
      \emphregion{#1}{#2}
      \xdef\@ll{}
}
\def\emphdefault#1{\def\res@style{\csname text#1\endcsname}}
\def\frameblock#1#2#3{%
   \xdef\seq@{#1}
   \xdef\temp@{consensus} \ifx\seq@\temp@ \xdef\seq@{0} \fi
   \xdef\first@{\seq@ @} \expandafter\check@letter\first@
   \xdef\first@{\seq@}
   \ifletter \get@name@number \xdef\seq@{\first@} \fi
   \ifnum\seq@>\seq@count
     \message{<Ignored seq `\seq@' in \noexpand\frameblock>}
   \else
     \xdef\@ll{#3}
     \xdef\list@{#2,&}
     \xdef\temp@@{frame} \loopcount=0
     \xdef\temp@{#2,,,:,,,,@} \expandafter\test@PDB\temp@
     \ifx\list@\ampers@nd
     \else
        \loop
           \xdef\list@{\list@ @}
           \expandafter\get@frameregions\list@
        \ifx\list@\ampers@nd\else\repeat
     \fi
   \fi
}
\def\changeshadingcolors#1#2#3{%
   \xdef\seq@{#1}
   \xdef\temp@{consensus} \ifx\seq@\temp@ \xdef\seq@{0} \fi
   \xdef\first@{\seq@ @} \expandafter\check@letter\first@
   \xdef\first@{\seq@}
   \ifletter \get@name@number \xdef\seq@{\first@} \fi
   \ifnum\seq@>\seq@count
     \message{<Ignored seq `\seq@' in \noexpand\changeshadingcolors>}
   \else
     \xdef\@ll{#3}
     \xdef\list@{#2,&}
     \xdef\temp@@{shading} \loopcount=0
     \xdef\temp@{#2,,,:,,,,@}    \expandafter\test@PDB\temp@
     \ifx\list@\ampers@nd
     \else
        \loop
           \xdef\list@{\list@ @}
           \expandafter\get@shadingregions\list@
        \ifx\list@\ampers@nd\else\repeat
     \fi
   \fi
}
\def\bargraphstretch#1{\def\b@r@stretch{#1}}
\def\colorscalestretch#1{\def\sc@le@stretch{#1}}
\def\rm@@measure#1#2@{%
   \xdef\test@{#1}
   \ifx\test@\ampers@nd
   \else
      \expandafter\check@char\test@
      \ifnumber\xdef\first@{\first@ #1}
         \xdef\second@{#2 &@}
         \expandafter\rm@@measure\second@
      \fi
   \fi
}
\def\rm@measure#1.#2@{\xdef\first@{#1.}\xdef\second@{#2 @}\xdef\third@{#1}\expandafter\rm@@measure\second@}
\def\pm@calc{%
    \temp@@length=100000sp
    \temp@@length=\g@min\temp@@length
    \innerloopcount=\temp@@length
    \xdef\min@{\the\innerloopcount}
    \arrow@height=\temp@@length
    \temp@@length=100000sp
    \temp@@length=\g@max\temp@@length
    \advance\temp@@length by -\arrow@height
    \innerloopcount=\temp@@length
    \divide\innerloopcount by 100
    \ifnum\innerloopcount=0 \innerloopcount=1 \fi
    \xdef\m@x{\the\innerloopcount}
    \xdef\test@{\g@min pt}
    \setlength\arrow@width{\test@}
    \xdef\test@{\g@max pt}
    \setlength\arrow@height{\test@}
    \advance\arrow@height by -\arrow@width
    \ifdim\arrow@width<0pt\temp@@length=-\arrow@width\xdef\test@{y}\else\temp@@length=\arrow@width\xdef\test@{n}\fi
    \ifdim\arrow@height>0pt
       \ifx\test@\n@ \xdef\test@{y} \else \xdef\test@{n} \fi
       \ifdim\temp@@length<\arrow@height\temp@@length=\arrow@height\fi
    \else
       \ifdim\temp@@length>-\arrow@height\temp@@length=\arrow@height\fi
    \fi
    \ifdim\temp@@length<100pt\arrow@width=100\arrow@width\arrow@height=100\arrow@height\else
    \ifdim\temp@@length<10pt\arrow@width=1000\arrow@width\arrow@height=1000\arrow@height\else
    \ifdim\temp@@length<1pt\arrow@width=10000\arrow@width\arrow@height=10000\arrow@height\else
    \ifdim\temp@@length<0.1pt\arrow@width=100000\arrow@width\arrow@height=100000\arrow@height\else
    \ifdim\temp@@length<0.01pt\arrow@width=1000000\arrow@width \arrow@height=1000000\arrow@height\else
    \ifdim\temp@@length<0.001pt\arrow@width=10000000\arrow@width \arrow@height=10000000\arrow@height\else
    \ifdim\temp@@length<0.0001pt\arrow@width=100000000\arrow@width \arrow@height=100000000\arrow@height\else
    \ifdim\temp@@length<0.00001pt\arrow@width=1000000000\arrow@width \arrow@height=1000000000\arrow@height
    \fi\fi\fi\fi\fi\fi\fi\fi
    \ifx\test@\y@
       \xdef\pm@{0}
    \else
       \xdef\test@{-\the\arrow@height @}
       \expandafter\rm@measure\test@
       \divide\arrow@width by \third@
       \xdef\pm@{\the\arrow@width @}
       \expandafter\rm@measure\pm@
       \xdef\pm@{\first@}
    \fi
}
\def\read@graph{%
    \bgroup
    \immediate\openin\structurefile = \fill@char\relax
    \ifeof\structurefile
      \PackageError{TeXshade}
      {File `\fill@char' not found}
      {\MessageBreak
       The file you specified is missing or you have \MessageBreak
       misspelled it. \MessageBreak\MessageBreak
       No feature graph will be displayed. \MessageBreak
       Type <return> to proceed. \MessageBreak
       Type X <return> to quit. \MessageBreak
}
       \immediate\closein\structurefile\egroup
    \else
    \ifx\g@min\comm@
    \def\par@{}
    \xdef\g@min{,} \xdef\g@max{,}
    \loop
       \read\structurefile to \readline
       \xdef\test@{\expandafter\string\readline}
       \ifx\test@\par@
       \else
       \xdef\second@{\test@ @}
       \expandafter\check@letter\second@
       \ifletter
          \xdef\second@{\expandafter\string\readline}
          \xdef\second@{\second@ @}
          \expandafter\firstchar@get\second@
          \if\first@ - \numbertrue\fi
       \fi
       \ifnumber
          \temp@@length=1pt
          \temp@@length=\test@\temp@@length
          \innerloopcount=\temp@@length
          \ifx\g@min\comm@ \xdef\min@{\test@ pt} \xdef\g@min{\test@}
          \else
             \ifdim\temp@@length<\min@\relax \xdef\min@{\the\temp@@length} \xdef\g@min{\test@} \fi\fi%
          \ifx\g@max\comm@ \xdef\m@x{\test@ pt} \xdef\g@max{\test@}
          \else
             \ifdim\temp@@length>\m@x\relax \xdef\m@x{\the\temp@@length} \xdef\g@max{\test@} \fi\fi%
       \fi
       \fi
    \ifeof\structurefile\else\repeat
    \fi
    \immediate\closein\structurefile
    \pm@calc
    \expandafter\temp@count=\csname seq@start\seq@\endcsname
    \advance\temp@count by -1
    \xdef\temp@@@{n}
    \immediate\openin\structurefile = \fill@char\relax
    \loop
       \read\structurefile to \readline
       \xdef\test@{\expandafter\string\readline}
       \ifx\test@\par@
       \else
       \xdef\second@{\test@ @}
       \expandafter\check@letter\second@
       \ifletter
          \xdef\second@{\expandafter\string\readline}
          \xdef\second@{\second@ @}
          \expandafter\firstchar@get\second@
          \if\first@ - \numbertrue
          \else
             \if\first@ N
                \expandafter\firstchar@get\third@
                \if\first@ a
                   \expandafter\firstchar@get\third@
                   \if\first@ N
                     \advance\temp@count by 1 
                     \ifnum\temp@count=0 \temp@count=1 \fi
                     \ifnum\temp@count<\st@rt
                     \else
                        \ifnum\temp@count>\st@p
                        \else
                           \ifx\temp@@@\n@
                              \xdef\temp@@@{N}
                           \else
                              \xdef\temp@@@{\temp@@@,N}
                           \fi
                        \fi
                     \fi
                   \fi
                \fi   
             \fi
          \fi
       \fi
       \ifnumber
          \advance\temp@count by 1 
          \ifnum\temp@count=0 \temp@count=1 \fi
          \ifnum\temp@count<\st@rt
          \else
             \ifnum\temp@count>\st@p
             \else
                \temp@@length=100000sp
                \temp@@length=\test@\temp@@length
                \innerloopcount=\temp@@length
                \advance\innerloopcount by 1
                \xdef\test@{\pm@ pt}
                \ifdim\test@=0pt
                   \advance\innerloopcount by -\min@
                \else
                   \ifx\b@r\n@
                      \advance\innerloopcount by -\min@
                   \fi
                \fi
                \divide\innerloopcount by \m@x
                \ifx\temp@@@\n@
                   \xdef\temp@@@{\the\innerloopcount}
                \else
                   \xdef\temp@@@{\temp@@@,\the\innerloopcount}
                \fi
             \fi
          \fi
       \fi
       \fi
    \ifeof\structurefile\else\repeat
    \immediate\closein\structurefile
    \xdef\temp@@@{\temp@@@,@} \xdef\temp@@{y}
    \fi
    \egroup
}
\def\sort@gstack{%
   \expandafter\get@fromstack\last@
   \ifx\first@\ampers@nd
      \xdef\tmpstack{\tmpstack\f@text@&;&;&;&;@}
   \else
   \ifnum\loopcount<\second@
      \xdef\tmpstack{\tmpstack\f@text@\first@;\second@;\third@;\fourth@;\last@}
   \else
      \xdef\tmpstack{\tmpstack\first@;\second@;\third@;\fourth@;}
      \sort@gstack
   \fi\fi
}
\def\get@gregion#1..#2,#3&{%
   \xdef\st@rt{#1} \xdef\st@p{#2} \xdef\list@{#3}
}
\def\do@bargraph{%
   \expandafter\get@gregion\list@
   \expandafter\ifnum\csname seq@start\seq@\endcsname>\st@rt
   \else
   \xdef\temp@@{n}
   \xdef\b@r{y}
   \xdef\pm@{0}
   \xdef\temp@@@{\fill@char,@}
   \expandafter\check@letter\temp@@@
   \ifletter 
      \read@graph
   \else
      \ifx\g@min\comm@ \xdef\g@min{0} \fi
      \ifx\g@max\comm@ \xdef\g@max{100} \fi
      \pm@calc
      \xdef\temp@@@{\fill@char,@}
      \xdef\temp@@{y}
   \fi
   \ifx\temp@@\y@
   \loopcount=\st@rt
   \xdef\tmpstack{}
   \loop
      \expandafter\get@item\temp@@@ 
      \xdef\temp@@@{\first@}
      \xdef\style@{bar[\pm@,0]:\fourth@[\f@color]}
      \xdef\tmpstack{\tmpstack\f@text@;\the\loopcount;\the\loopcount;\style@;}
      \advance\loopcount by 1
      \ifnum\loopcount=0 \loopcount=1 \fi
   \ifnum\loopcount>\st@p
   \else\repeat
      \xdef\f@text@{\tmpstack} \xdef\tmpstack{}
      \xdef\last@{\csname stack@\bottop@\seq@\endcsname}
      \sort@gstack
      \expandafter\xdef\csname stack@\bottop@\seq@\endcsname{\tmpstack}
   \fi
   \fi
   \xdef\list@{\list@ &}
   \ifx\list@\ampers@nd\else\do@bargraph\fi
}
\def\do@colorgraph{%
   \expandafter\get@gregion\list@
   \expandafter\ifnum\csname seq@start\seq@\endcsname>\st@rt
   \else
   \xdef\temp@@{n}
   \xdef\b@r{n}
   \xdef\pm@{0}
   \xdef\temp@@@{\fill@char,@}
   \expandafter\check@letter\temp@@@
   \ifletter 
      \read@graph
   \else
      \ifx\g@min\comm@ \xdef\g@min{0} \fi
      \ifx\g@max\comm@ \xdef\g@max{100} \fi
      \pm@calc
      \xdef\temp@@@{\fill@char,@}
      \xdef\temp@@{y}
   \fi
   \ifx\temp@@\y@
   \loopcount=\st@rt
   \xdef\tmpstack{}
   \loop
      \xdef\last@{\csname stack@\bottop@\seq@\endcsname}
      \expandafter\get@item\temp@@@ 
      \xdef\temp@@@{\first@}
      \ifx\fourth@\N@
         \xdef\style@{color:50[White]}
      \else
         \ifnum\fourth@<1 \xdef\fourth@{1} \fi
         \innerloopcount=\fourth@
         \advance\innerloopcount by 4
         \divide\innerloopcount by 5
         \multiply\innerloopcount by 5
         \ifnum\innerloopcount>100 \innerloopcount=100 \fi
         \xdef\style@{color:50[\f@color\the\innerloopcount]}
      \fi    
      \xdef\tmpstack{\tmpstack\f@text@;\the\loopcount;\the\loopcount;\style@;}
      \advance\loopcount by 1
      \ifnum\loopcount=0 \loopcount=1 \fi
   \ifnum\loopcount>\st@p
   \else\repeat
      \xdef\f@text@{\tmpstack} \xdef\tmpstack{}
      \xdef\last@{\csname stack@\bottop@\seq@\endcsname}
      \sort@gstack
      \expandafter\xdef\csname stack@\bottop@\seq@\endcsname{\tmpstack}
   \fi
   \fi
   \xdef\list@{\list@ &}
   \ifx\list@\ampers@nd\else\do@colorgraph\fi
}
\def\feature#1#2#3#4#5{%
   \xdef\bottop@{#1}
   \xdef\temp@{top}
   \ifx\bottop@\temp@ \topfeaturetrue\fi
   \xdef\temp@{ttop}
   \ifx\bottop@\temp@ \ttopfeaturetrue\fi
   \xdef\temp@{tttop}
   \ifx\bottop@\temp@ \tttopfeaturetrue\fi
   \xdef\temp@{ttttop}
   \ifx\bottop@\temp@ \ttttopfeaturetrue\fi
   \xdef\temp@{bottom}
   \ifx\bottop@\temp@ \bottomfeaturetrue\fi
   \xdef\temp@{bbottom}
   \ifx\bottop@\temp@ \bbottomfeaturetrue\fi
   \xdef\temp@{bbbottom}
   \ifx\bottop@\temp@ \bbbottomfeaturetrue\fi
   \xdef\temp@{bbbbottom}
   \ifx\bottop@\temp@ \bbbbottomfeaturetrue\fi
   \xdef\seq@{#2}
   \xdef\temp@{consensus} \ifx\seq@\temp@ \xdef\seq@{0} \fi
   \xdef\first@{\seq@ @} \expandafter\check@letter\first@
   \xdef\first@{\seq@}
   \ifletter \get@name@number \xdef\seq@{\first@} \fi
   \ifnum\seq@>\seq@count
      \message{<Ignored seq `\seq@' in \noexpand\feature>}
   \else
   \ifnum\seq@>-1
   \xdef\temp@{#4::&}\expandafter\test@fill\temp@
   \xdef\last@{bar}
   \ifx\second@@\last@
      \xdef\last@{hydrophobicity}
      \ifx\last@\fourth@
         \xdef\second@@{bh}
      \else
         \xdef\last@{molweight}
         \ifx\last@\fourth@
            \xdef\second@@{bm}
         \else
            \xdef\last@{charge}
            \ifx\last@\fourth@
               \xdef\second@@{bc}
            \else
               \xdef\last@{conservation}
               \ifx\last@\fourth@
                  \xdef\second@@{bcons}
               \fi
            \fi
         \fi
      \fi
   \fi
   \xdef\last@{color}
   \ifx\second@@\last@
      \xdef\last@{hydrophobicity}
      \ifx\last@\fourth@
         \xdef\second@@{ch}
      \else
         \xdef\last@{molweight}
         \ifx\last@\fourth@
            \xdef\second@@{cm}
         \else
            \xdef\last@{charge}
            \ifx\last@\fourth@
               \xdef\second@@{cc}
            \else
               \xdef\last@{conservation}
               \ifx\last@\fourth@
                  \xdef\second@@{ccons}
               \fi
            \fi
         \fi
      \fi
   \fi
   \xdef\last@{bar}
   \ifx\second@@\last@
      \xdef\list@{#3,&}
      \xdef\style@{#4}
      \def\f@text@{#5}
      \do@bargraph
      \xdef\temp@{bottom}
      \ifx\bottop@\temp@ \xdef\bottom@stretch{y}\fi
      \xdef\temp@{bbottom}
      \ifx\bottop@\temp@ \xdef\bbottom@stretch{y}\fi
      \xdef\temp@{bbbottom}
      \ifx\bottop@\temp@ \xdef\bbbottom@stretch{y}\fi
      \xdef\temp@{bbbbottom}
      \ifx\bottop@\temp@ \xdef\bbbbottom@stretch{y}\fi
   \else
   \xdef\last@{color}
   \ifx\second@@\last@
      \xdef\list@{#3,&}
      \xdef\style@{#4}
      \def\f@text@{#5}
      \do@colorgraph
   \else
   \xdef\f@@color{\f@color}
   \def\f@text@{#5}
   \xdef\f@color{\f@@color}
   \xdef\temp@{#4&}
   \ifx\temp@\ampers@nd
      \xdef\list@{#3,&}
      \xdef\style@{&}
      \def\f@text@{#5}
   \else
      \xdef\last@{restriction}
      \ifx\second@@\last@
         \xdef\temp@{\bottop@ @}
         \expandafter\firstchar@get\temp@
         \xdef\temp@{y}
         \if\first@ t
            \xdef\style@{fill:\kern0.9\box@width$\blacktriangledown$[\f@color]}
         \else
            \xdef\style@{fill:\kern0.9\box@width$\blacktriangle$[\f@color]}
         \fi
         \xdef\f@text@{\kern0.9\box@width#5}
      \else
      \xdef\last@{bh}
      \ifx\second@@\last@
         \ifx\f@color\gr@ydef@ult\xdef\f@color{Gray50}\fi
         \xdef\style@{plot[bar]:Hydro[\f@color][-53]}
      \else
      \xdef\last@{bm}
      \ifx\second@@\last@
         \ifx\f@color\gr@ydef@ult\xdef\f@color{Gray50}\fi
         \xdef\style@{plot[bar]:molw[\f@color][0]}
      \else
      \xdef\last@{bc}
      \ifx\second@@\last@
         \ifx\f@color\gr@ydef@ult\xdef\f@color{Gray50}\fi
         \xdef\style@{plot[bar]:charge[\f@color][-50]}
      \else
      \xdef\last@{bcons}
      \ifx\second@@\last@
         \ifx\f@color\gr@ydef@ult\xdef\f@color{Gray50}\fi
         \xdef\last@{T-Coffee}
         \ifx\f@color\last@
            \xdef\T@coffee@bcons{y}
            \xdef\f@color{Gray50}
         \fi
         \xdef\style@{cons[bar]:cons[\f@color][0]}
      \else
      \xdef\last@{ch}
      \ifx\second@@\last@
         \ifx\f@color\gr@ydef@ult\xdef\f@color{RedGreen}\fi
         \xdef\style@{plot[color]:Hydro[\f@color][-53]}
      \else
      \xdef\last@{cm}
      \ifx\second@@\last@
         \ifx\f@color\gr@ydef@ult\xdef\f@color{Gray}\fi
         \xdef\style@{plot[color]:molw[\f@color][0]}
      \else
      \xdef\last@{cc}
      \ifx\second@@\last@
         \ifx\f@color\gr@ydef@ult\xdef\f@color{RedBlue}\fi
         \xdef\style@{plot[color]:charge[\f@color][-50]}
      \else
      \xdef\last@{ccons}
      \ifx\second@@\last@
         \ifx\f@color\gr@ydef@ult\xdef\f@color{ColdHot}\fi
         \xdef\last@{T-Coffee}
         \ifx\f@color\last@
            \xdef\T@coffee@ccons{y}
            \xdef\f@color{TC}
         \fi
         \xdef\style@{cons[color]:cons[\f@color][0]}
      \else
         \xdef\style@{#4} \expandafter\getarrow@shape\temp@
      \fi
      \fi
      \fi
      \fi
      \fi
      \fi
      \fi
      \fi
      \fi
   \xdef\list@{#3,&}
   \xdef\@ll{\f@text@}
   \xdef\temp@@{\bottop@}
   \xdef\temp@{#3,,,:,,,,@} \expandafter\test@PDB\temp@
   \fi
   \loopcount=0 
   \ifx\list@\ampers@nd
   \else
      \loop
         \advance\loopcount by 1\relax
         \xdef\list@{\list@ @}
         \expandafter\get@fregions\list@
      \ifx\list@\ampers@nd\else\repeat
   \fi
   \fi\fi\fi
   \fi
}
\def\showfeaturename#1#2{\expandafter\xdef\csname featuretextn@me#1\endcsname{#2}}
\def\showfeaturestylename#1#2{\expandafter\xdef\csname featurestylesn@me#1\endcsname{#2}}
\def\hidefeaturename#1{\expandafter\xdef\csname featuretextn@me#1\endcsname{}}
\def\hidefeaturenames{\xdef\featuretextn@mettop{}\xdef\featuretextn@metop{}\xdef\featuretextn@mebottom{}\xdef\featuretextn@mebbottom{}
    \xdef\featuretextn@metttop{}\xdef\featuretextn@mettttop{}\xdef\featuretextn@mebbbbottom{}\xdef\featuretextn@mebbbottom{}}
\def\hidefeaturestylename#1{\expandafter\xdef\csname featurestylesn@me#1\endcsname{}}
\def\hidefeaturestylenames{\xdef\featurestylesn@mettop{}\xdef\featurestylesn@metop{}\xdef\featurestylesn@mebottom{}\xdef\featurestylesn@mebbottom{}
    \xdef\featurestylesn@metttop{}\xdef\featurestylesn@mettttop{}\xdef\featurestylesn@mebbbbottom{}\xdef\featurestylesn@mebbbottom{}}
\def\seqtype#1{\xdef\seq@type{#1}
               \if\seq@type P \xdef\prefix@{pep}
               \else \if\seq@type p \xdef\seq@type{P} \xdef\prefix@{pep}
               \else \xdef\seq@type{N} \xdef\prefix@{DNA} \fi\fi}
\def\nameseq#1#2{%
   \xdef\first@{#1 @} \expandafter\check@letter\first@
   \xdef\first@{#1}   \ifletter \get@name@number \fi
   \expandafter\xdef\csname newseqname\first@\endcsname{#2}
}
\newcommand\threshold[2][n]{%
   \xdef\first@{#1}
   \ifx\first@\n@
   \xdef\thresh@ld{#2}
   \else
      \all@shadetrue
      \ifnum\first@>#2
         \xdef\thresh@ld{#2}
         \xdef\all@thresh@ld{#1}
      \else
         \xdef\thresh@ld{#1}
         \xdef\all@thresh@ld{#2}
      \fi
   \fi
}
\def\constosingleseq#1{%
   \xdef\first@{#1 @} \expandafter\check@letter\first@
   \xdef\first@{#1}   \ifletter \get@name@number \fi
   \ifnum\first@>\seq@count
   \message{<Ignored seq `#1' in \noexpand\constosingleseq>}
   \else
   \ifnum\first@>0 \xdef\cons@num{\first@} \hideconsensus\fi\fi
}
\def\constoallseqs{\xdef\cons@num{0}}
\def\residuesperline{%
   \def\@rplfix*##1{%
      \res@perline=##1
      \ifnum\res@perline<1 \res@perline=1\fi \rpl@fixtrue}
   \def\@rplvar ##1{%
      \res@perline=##1
      \ifnum\res@perline<5 \res@perline=5\fi \rpl@fixfalse}
   \def\decide@{\ifx\l@@k * \let\next\@rplfix \else \let\next\@rplvar \fi \next}
   \futurelet\l@@k\decide@}
\def\numberingwidth#1{\def\num@width{#1}}
\def\charstretch#1{%
   \def\char@stretch{#1}
   \xdef\temp@{\char@stretch .00@} \expandafter\coord@get\temp@ \loopcount=\c@@rd
   \multiply \loopcount by 7
   \divide \loopcount by 10
   \ifnum\loopcount<10
      \xdef\temp@{0\the\loopcount @}
   \else
      \xdef\temp@{\the\loopcount @}
   \fi
   \expandafter\decimal@B\temp@
   \ifnum\loopcount>99
      \xdef\char@stretch@W{#1}
   \else
      \xdef\char@stretch@W{\temp@}
   \fi
}
\def\linestretch#1{\def\line@stretch{#1}}
\def\logostretch#1{%
   \def\logo@stretch{#1}
   \setlength\temp@@length{1000sp}
   \setlength\temp@@length{\logo@stretch\temp@@length}
   \loopcount=\temp@@length
   \xdef\logo@stretch@IOOO{\the\loopcount}
}
\def\noblockskip{\def\block@skip{\vspace{0pt}}}
\def\smallblockskip{\def\block@skip{\vspace{\baselineskip}}}
\def\medblockskip{\def\block@skip{\vspace{1.5\baselineskip}}}
\def\bigblockskip{\def\block@skip{\vspace{2\baselineskip}}}
\def\vblockspace#1{\def\block@skip{\vspace{#1}}}
\def\topspace#1{\def\t@sp@ce{#1}}
\def\ttopspace#1{\def\tt@sp@ce{#1}}
\def\tttopspace#1{\def\ttt@sp@ce{#1}}
\def\ttttopspace#1{\def\tttt@sp@ce{#1}}
\def\bottomspace#1{\def\b@sp@ce{#1}}
\def\bbottomspace#1{\def\bb@sp@ce{#1}}
\def\bbbottomspace#1{\def\bbb@sp@ce{#1}}
\def\bbbbottomspace#1{\def\bbbb@sp@ce{#1}}
\def\rulerspace#1{\def\ruler@sp@ce{#1}}
\def\fixblockspace{\fix@true}
\def\flexblockspace{\fix@false}
\def\nosepline{\def\seq@skip{\relax}}
\def\smallsepline{\def\seq@skip{\vspace{3pt}}\def\sep@space{3pt}}
\def\medsepline{\def\seq@skip{\vspace{6pt}}\def\sep@space{6pt}}
\def\bigsepline{\def\seq@skip{\vspace{12pt}}\def\sep@space{12pt}}
\def\vsepspace#1{\def\seq@skip{\vspace{#1}}\xdef\sep@space{#1}}
\def\separationline#1{%
   \xdef\first@@{#1}
   \xdef\first@{#1 @} \expandafter\check@letter\first@
   \xdef\first@{#1}
   \ifletter \get@name@number \xdef\first@@{\first@} \fi
   \ifnum\first@@>\seq@count \xdef\first@@{1}
   \else
      \ifnum\first@@<0 \xdef\first@@{1}
      \else
         \expandafter\def\csname seq@gap\first@@\endcsname{yes}
         \loopcount=\seq@gap@num
         \advance\loopcount by 1
         \xdef\seq@gap@num{\the\loopcount}
      \fi
   \fi}
\newcommand{\shadingmode}[2][-1]{%
   \T@coffeefalse
   \xdef\last@{#2}
   \xdef\first@{identical}
   \ifx\first@\last@
      \simmodefalse \funcmodefalse
      \xdef\second@{#1 @} \expandafter\check@letter\second@
      \ifnumber \all@shadetrue 
         \ifnum#1<0 \all@shadefalse
         \else
         \ifnum#1>100 \xdef\all@thresh@ld{100}
         \else 
            \xdef\all@thresh@ld{#1}
         \fi\fi
      \else
         \xdef\last@{#1}
         \xdef\first@{allmatchspecial}
         \ifx\last@\first@
            \xdef\all@thresh@ld{100} \all@shadetrue
         \fi
      \fi
   \else
   \xdef\first@{similar}
   \ifx\first@\last@
      \simmodetrue  \funcmodefalse
      \xdef\second@{#1 @} \expandafter\check@letter\second@
      \ifnumber \all@shadetrue 
         \ifnum#1<0 \all@shadefalse
         \else
         \ifnum#1>100 \xdef\all@thresh@ld{100}
         \else 
            \xdef\all@thresh@ld{#1}
         \fi\fi
      \else
         \xdef\last@{#1}
         \xdef\first@{allmatchspecial}
         \ifx\last@\first@
            \xdef\all@thresh@ld{100} \all@shadetrue
         \fi
      \fi
   \else
   \xdef\first@{functional}
   \ifx\first@\last@
      \if\seq@type N \message{<No functional shading on DNA sequences>}
      \else \simmodefalse \funcmodetrue \func@shading{#1} 
            \xdef\seq@type{P} \xdef\prefix@{pep} \fi
   \else
   \xdef\first@{T-Coffee}
   \ifx\first@\last@
      \simmodefalse \funcmodefalse \T@coffeetrue 
      \xdef\second@{#1 @} \expandafter\check@letter\second@
      \ifletter \xdef\TC@first@{#1}\include@T@coffee\fi
   \else
   \xdef\first@{diverse}
   \ifx\first@\last@
      \xdef\last@{#1}
      \ifnum\last@>\seq@count \xdef\last@{1}\fi
      \ifnum\last@<1 \xdef\last@{1}\fi
      \simmodetrue  \funcmodefalse
      \threshold{0}
      \xdef\divref@{\last@}
      \constosingleseq{\last@}
      \nomatchresidues{Black}{White}{lower}{up}
      \similarresidues{Black}{White}{lower}{up}
      \conservedresidues{Black}{White}{{.}}{up}
      \allmatchresidues{Black}{White}{{.}}{up}
      \gapchar{-}
      \hideconsensus
   \else
   \message{<Unknown shading mode - using `similar'>}
   \simmodetrue  \funcmodefalse
   \fi\fi\fi\fi\fi}
\def\hideallmatchpositions{\xdef\all@out{y}}
\def\showallmatchpositions{\xdef\all@out{n}}
\newcommand\allmatchspecial[1][100]{%
   \ifnum#1<0
      \xdef\all@thresh@ld{0}
   \else
   \ifnum#1>100
      \xdef\all@thresh@ld{100}
   \else
      \xdef\all@thresh@ld{#1}
   \fi\fi
   \all@shadetrue}
\def\allmatchspecialoff{\all@shadefalse}
\def\stopchar#1{\def\st@p@char{#1}}
\def\gapchar#1{%
   \xdef\first@{rule}\xdef\second@{#1}
   \ifx\first@\second@\def\gap@char{o}
   \else\def\gap@char{#1}\fi}
\def\gaprule#1{\def\gap@rulethick{#1}}
\def\domaingaprule#1{\def\domgap@rulethick{#1}}
\newcommand{\setends}[3][&]{%
  \xdef\start@seq{#2}
  \xdef\temp@{consensus}
  \ifx\start@seq\temp@ 
%%%     \message{<\noexpand\setends does not accept `consensus'>}
     \xdef\start@seq{0}
%%%%%%%%%%
            \xdef\second@{#3@} \expandafter\get@nums\second@ 
            \xdef\start@num{\first@} \xdef\end@num{\second@}
            \ifnum\start@num<1 \xdef\start@num{1}\fi
            \ifnum\end@num<\start@num \xdef\end@num{\start@num}\fi
            \loopcount=\end@num
            \advance\loopcount by 1
            \advance\loopcount by -\start@num
            \xdef\end@num{\the\loopcount}
            \start@false
%%%%%%%%%%
  \else
     \xdef\first@{\start@seq @} \expandafter\check@letter\first@
     \xdef\first@{\start@seq}
     \ifletter \get@name@number \xdef\start@seq{\first@} \fi
     \ifnum\start@seq>\seq@count 
         \message{<\noexpand\setends{} error: sequence `#2' not defined>}
         \xdef\start@seq{0}
     \else
         \ifnum\start@seq<1
            \message{<\noexpand\setends{} error: sequence `#2' not defined>}
            \xdef\start@seq{0}
         \else
            \xdef\second@{#3@} \expandafter\get@nums\second@ 
            \xdef\start@num{\first@} \xdef\end@num{\second@}
            \ifnum\start@num=0 \xdef\allow@zero{y}\fi
            \ifnum\end@num=0 \xdef\allow@zero{y}\fi
            \start@false
         \fi
     \fi
  \fi
  \xdef\first@{#1}\ifx\first@\ampers@nd\else\startnumber{#2}{#1}\fi
}
\newcommand{\startnumber}[3][&]{%
   \xdef\first@{#2} \xdef\second@{consensus}
   \ifx\first@\second@
      \ifnum#3=0 \xdef\allow@zero{y} \fi
      \expandafter\xdef\csname seq@start0\endcsname{#2}
      \cons@count=#3 \advance\cons@count by -1\relax 
      \expandafter\xdef\csname res@count0\endcsname{\the\cons@count}
   \else
      \xdef\first@{#2 @} \expandafter\check@letter\first@
      \xdef\first@{#2}   \ifletter \get@name@number \fi
      \ifnum\first@>\seq@count \message{<Ignored seq `#2' in \noexpand\startnumber>}
      \else
         \xdef\second@{#3} 
         \ifnum\second@=0 \xdef\allow@zero{y} \fi
         \expandafter\xdef\csname seq@start\first@\endcsname{\second@}
         \res@count=\second@
         \advance\res@count by -1
         \expandafter\xdef\csname res@count\first@\endcsname{\the\res@count}
      \fi
   \fi
   \xdef\first@{#1}\ifx\first@\ampers@nd\else\setends{#2}{#1}\fi
   }
\def\seqlength#1#2{%
   \xdef\first@{#1 @} \expandafter\check@letter\first@
   \xdef\first@{#1}   \ifletter \get@name@number \fi
   \ifnum\first@>\seq@count \message{<Ignored seq `#1' in \noexpand\seqlength>}
   \else
      \xdef\second@{#2} \ifnum\second@<0 \xdef\second@{1} \fi
      \expandafter\xdef\csname seq@len\first@\endcsname{\second@}
   \fi}
\newcommand\shownumbering[2][n]{%
   \xdef\first@{#1}\ifx\first@\n@\else\xdef\numbering@fg{#1}\fi
   \xdef\first@{#2}
   \xdef\second@{left}
   \ifx\first@\second@ \numbers@lefttrue \numbers@rightfalse \numbers@true \fi
   \xdef\second@{right}
   \ifx\first@\second@ \numbers@leftfalse \numbers@righttrue \numbers@true \fi
   \xdef\second@{leftright}
   \ifx\first@\second@ \numbers@lefttrue \numbers@righttrue \numbers@true \fi
   }
\def\hidenumbering{\numbers@false}
\def\hidenumber#1{\xdef\first@{#1,&,@} \hidenumber@}
\newcommand\shownames[2][n]{%
   \xdef\first@{#1}\ifx\first@\n@\else\xdef\names@fg{#1}\fi
   \xdef\first@{#2} \xdef\second@{left}
   \ifx\first@\second@ \names@rightfalse \else \names@righttrue \fi
   \names@true}
\def\hidenames{\names@false}
\def\hidename#1{\xdef\first@{#1,&,@} \hidename@}
\def\hideresidues{\hidechartrue}
\def\showresidues{\hidecharfalse}
\def\alignment#1{%
   \xdef\first@{#1}
   \xdef\temp@{left}
      \ifx\first@\temp@ \xdef\c@factor{0}
      \else
         \xdef\temp@{center}
         \ifx\first@\temp@ \xdef\c@factor{0.5}
         \else
            \xdef\temp@{right}
            \ifx\first@\temp@ \xdef\c@factor{1}
   \fi\fi\fi}
\def\donotshade#1{%
   \xdef\temp@{consensus}
   \xdef\first@{#1}
   \ifx\first@\temp@
      \consensuscolors{Black}{White}{Black}{White}{Black}{White}
   \else
      \xdef\first@{#1,&,@} \donot@shade
   \fi}
\def\hideseqs{\xdef\hide@seqs{y}}
\def\showseqs{\xdef\hide@seqs{n}}
\def\hideseq#1{\xdef\first@{#1,&,@} \hideseq@}
\def\killseq#1{\xdef\first@{#1,&,@} \killseq@}
\def\hidesequencelogo{\show@logofalse}
\def\hidesubfamilylogo{\show@sublogofalse}
\def\logo@group@get#1#2@{%
   \xdef\first@{#1}\xdef\third@{#2@}
   \ifx\first@\ampers@nd
   \else
      \ifnum`#1>96 \make@upper\fi
      \expandafter\xdef\csname logo@col\first@\endcsname{\second@}
      \expandafter\logo@group@get\third@
   \fi
}
\def\logocolor#1#2{%
   \xdef\logo@colors@set{yes}
   \xdef\second@{#2}
   \xdef\third@{#1&@}
   \expandafter\logo@group@get\third@
}
\newcommand\clearlogocolors[1][Black]{%
   \logocolor{ABCDEFGHIJKLMNOPQRSTUVWXYZ}{#1}
}
\def\dofrequencycorrection{\xdef\do@freq@correction{y}}
\def\undofrequencycorrection{\xdef\do@freq@correction{n}}
\newcommand\showlogoscale[2][Black]{\xdef\logo@scalecol{#1}\xdef\show@logoscale{#2}}
\def\hidelogoscale{\xdef\show@logoscale{n}}
\def\hidenegatives{\xdef\hide@negatives{y}\xdef\sublogo@tint{}}
\newcommand\shownegatives[1][medium]{%
   \xdef\hide@negatives{n}
   \xdef\first@{#1}
   \xdef\second@{full}
   \ifx\first@\second@
      \xdef\sublogo@tint{}
   \else
   \xdef\second@{strong}
   \ifx\first@\second@
      \xdef\sublogo@tint{Light}
   \else
   \xdef\second@{medium}
   \ifx\first@\second@
      \xdef\sublogo@tint{LightLight}
   \else
   \xdef\second@{weak}
   \ifx\first@\second@
      \xdef\sublogo@tint{LightLightLight}
   \else
      \xdef\sublogo@tint{LightLight}
   \fi\fi\fi\fi
}
\def\set@logocolors{%
   \xdef\second@{undefined}
   \ifx\first@\second@
      \if\logo@colors@set\n@
         \xdef\first@{standard}
      \fi
   \fi
   \xdef\second@{standard}
   \ifx\first@\second@
    \if\seq@type A
      \xdef\logo@colors@set{n}
    \else
      \if\seq@type P
         \xdef\first@{rasmol}
      \else
         \xdef\first@{nucleotide}
      \fi
    \fi
   \fi
   \xdef\second@{nucleotide}
   \ifx\first@\second@
      \clearlogocolors
      \logocolor{G}{Black}
      \logocolor{A}{Green}
      \logocolor{TU}{Red}
      \logocolor{C}{Blue}
   \else
   \xdef\second@{rasmol}
   \ifx\first@\second@
      \clearlogocolors
      \logocolor{DE}{Red}
      \logocolor{CM}{Yellow}
      \logocolor{KR}{Blue}
      \logocolor{ST}{Orange}
      \logocolor{FY}{MidnightBlue}
      \logocolor{NQ}{Cyan}
      \logocolor{G}{LightGray}
      \logocolor{LVI}{Green}
      \logocolor{A}{DarkGray}
      \logocolor{W}{CarnationPink}
      \logocolor{H}{CornflowerBlue}
      \logocolor{P}{Apricot}
      \logocolor{BZ}{LightMagenta}
   \else
   \xdef\second@{chemical}
   \ifx\first@\second@
      \clearlogocolors
      \logocolor{DE}{Red}
      \logocolor{VIL}{Black}
      \logocolor{AG}{Gray}
      \logocolor{NQ}{Green}
      \logocolor{FYW}{Brown}
      \logocolor{KRH}{Blue}
      \logocolor{ST}{Magenta}
      \logocolor{P}{Orange}
      \logocolor{CM}{Yellow}
   \else
   \xdef\second@{hydropathy}
   \ifx\first@\second@
      \clearlogocolors
      \logocolor{DE}{Red}
      \logocolor{KRH}{Blue}
      \logocolor{YSTGNQC}{Yellow}
      \logocolor{AFPMWVIL}{Green}
   \else
   \xdef\second@{structure}
   \ifx\first@\second@
      \clearlogocolors
      \logocolor{DEHKNQR}{Orange}
      \logocolor{ACGPSTWY}{Yellow}
      \logocolor{FILMV}{Green}
   \else
   \xdef\second@{standard area}
   \ifx\first@\second@
      \clearlogocolors
      \logocolor{G}{BrickRed}
      \logocolor{AS}{Orange}
      \logocolor{CP}{Yellow}
      \logocolor{TDVN}{YellowGreen}
      \logocolor{IE}{PineGreen}
      \logocolor{LQHM}{SkyBlue}
      \logocolor{FK}{RoyalPurple}
      \logocolor{Y}{RedViolet}
      \logocolor{RW}{Black}
   \else
   \xdef\second@{accessible area}
   \ifx\first@\second@
      \clearlogocolors
      \logocolor{C}{BrickRed}
      \logocolor{IVG}{Orange}
      \logocolor{FLMA}{Yellow}
      \logocolor{WSTH}{YellowGreen}
      \logocolor{P}{PineGreen}
      \logocolor{YDN}{SkyBlue}
      \logocolor{EQ}{RoyalPurple}
      \logocolor{R}{RedViolet}
      \logocolor{K}{Black}
   \fi\fi\fi\fi\fi\fi\fi
}
\newcommand\findsubfamily[2][n]{%
   \xdef\first@{#1}
   \ifx\first@\n@
   \else
      \xdef\subfamily@threshold{#1}
   \fi
   \xdef\subfamily@seq{#2}
}
\def\setsubfamily#1{%
   \xdef\sub@family@setting{#1}
   \loopcount=1
   \loop
      \expandafter\xdef\csname subfamily@num\the\loopcount\endcsname{1}
      \advance\loopcount by 1
   \ifnum\loopcount>\seq@count\else\repeat
   \xdef\subfamily@count{2}
   \res@count=0 
   \xdef\first@{#1,&,@}
   \setsubfamily@
   \clear@res@nums{1}
   \clear@res@nums{2}
   \expandafter\xdef\csname group@num2\endcsname{\the\res@count}
   \loopcount=\seq@count
   \advance\loopcount by -\res@count
   \expandafter\xdef\csname group@num1\endcsname{\the\loopcount}
}
\def\subfamilythreshold#1{\xdef\subfamily@threshold{#1}}
\newcommand\showsubfamilylogo[2][undefined]{%
   \xdef\first@{#1} \set@logocolors
   \xdef\first@{#2} \xdef\last@{top}
   \ifx\first@\last@\xdef\sublogo@top{0}\else\xdef\sublogo@top{1}\fi
   \show@sublogotrue
}
\def\relevance#1{%
   \def\sig@max{#1}
   \setlength\temp@@length{1000sp}
   \setlength\temp@@length{\sig@max\temp@@length}
   \loopcount=\temp@@length
   \xdef\sig@max{\the\loopcount}
}
\newcommand\showrelevance[2][Black]{\def\sig@color{#1}\def\sig@char{#2}\xdef\hide@sig{n}}
\def\hiderelevance{\xdef\hide@sig{y}}
\newcommand\showsequencelogo[2][undefined]{%
   \xdef\first@{#1} \set@logocolors
   \xdef\first@{#2} \xdef\last@{top}
   \ifx\first@\last@\xdef\logo@top{0}\else\xdef\logo@top{1}\fi
   \show@logotrue   
}
\newcommand\showconsensus[2][n]{%
   \xdef\text@scale{n}
   \xdef\box@scale{n}
   \xdef\first@{#1}
   \ifx\first@\n@
      \xdef\collect@cons@colors{no}
   \else
      \xdef\first@{#1,&,@}
      \expandafter\get@item\first@
      \xdef\c@nsc@l{\fourth@}
      \xdef\first@@{Gray}
      \ifx\c@nsc@l\first@@\xdef\text@scale{y}\else
      \xdef\first@@{RedBlue}
      \ifx\c@nsc@l\first@@\xdef\text@scale{y}\else
      \xdef\first@@{BlueRed}
      \ifx\c@nsc@l\first@@\xdef\text@scale{y}\else
      \xdef\first@@{RedGreen}
      \ifx\c@nsc@l\first@@\xdef\text@scale{y}\else
      \xdef\first@@{GreenRed}
      \ifx\c@nsc@l\first@@\xdef\text@scale{y}\else
      \xdef\first@@{ColdHot}
      \ifx\c@nsc@l\first@@\xdef\text@scale{y}\else
      \xdef\first@@{HotCold}
      \ifx\c@nsc@l\first@@\xdef\text@scale{y}
      \fi\fi\fi\fi\fi\fi\fi
      \expandafter\get@item\first@
      \ifx\fourth@\ampers@nd
         \xdef\c@nssc@le{White}
      \else
         \xdef\c@nssc@le{\fourth@}
         \xdef\first@{Gray}
         \ifx\c@nssc@le\first@\xdef\box@scale{y}\else
         \xdef\first@{RedBlue}
         \ifx\c@nssc@le\first@\xdef\box@scale{y}\else
         \xdef\first@{BlueRed}
         \ifx\c@nssc@le\first@\xdef\box@scale{y}\else
         \xdef\first@{RedGreen}
         \ifx\c@nssc@le\first@\xdef\box@scale{y}\else
         \xdef\first@{GreenRed}
         \ifx\c@nssc@le\first@\xdef\box@scale{y}\else
         \xdef\first@{ColdHot}
         \ifx\c@nssc@le\first@\xdef\box@scale{y}\else
         \xdef\first@{HotCold}
         \ifx\c@nssc@le\first@\xdef\box@scale{y}\else
         \fi\fi\fi\fi\fi\fi\fi
      \fi
      \xdef\collect@cons@colors{y}
   \fi
   \xdef\first@{#2} \xdef\last@{top}
   \ifx\first@\last@\xdef\cons@top{0}\else\xdef\cons@top{1}\fi
   \show@construe}
\def\consensuscolors#1#2#3#4#5#6{%
   \xdef\last@{\ampers@nd}
   \xdef\first@{#1&}\xdef\second@{#2&}
   \ifx\first@\last@\else\def\ConsTextNomatch{#1}\fi
   \ifx\second@\last@\else\def\ConsNomatch{#2}\fi
   \xdef\first@{#3&}\xdef\second@{#4&}
   \ifx\first@\last@\else\def\ConsTextMatch{#3}\fi
   \ifx\second@\last@\else\def\ConsMatch{#4}\fi
   \xdef\first@{#5&}\xdef\second@{#6&}
   \ifx\first@\last@\else\def\ConsTextAllmatch{#5}\fi
   \ifx\second@\last@\else\def\ConsAllmatch{#6}\fi
}
\def\defconsensus#1#2#3{%
   \xdef\second@{#1&}
   \ifx\second@\ampers@nd   \else \def\n@m@tch{#1}\fi
   \xdef\second@{#2&}
   \ifx\second@\ampers@nd   \else \def\m@tch{#2}\fi
   \xdef\second@{#3&}
   \ifx\second@\ampers@nd   \else \def\@llm@tch{#3}\fi}
\def\hideconsensus{\show@consfalse}
\def\nameconsensus#1{\def\cons@name{#1}}
\def\namesequencelogo#1{\def\logo@name@user{#1}}
\newcommand\namesubfamilylogo[2][]{\def\sublogo@name@neg{#1}\def\sublogo@name@user{#2}}
\def\hideleadinggaps{\sh@wg@psfalse}
\def\showleadinggaps{\sh@wg@pstrue}
\newcommand\showruler[3][n]{%
   \xdef\first@{#1}
   \ifx\first@\n@\else\xdef\ruler@fg{#1}\fi
   \xdef\first@{consensus} \xdef\second@{#3}
   \xdef\third@{bottom}    \xdef\fourth@{#2}
   \ifx\third@\fourth@ \xdef\rule@top{1}\else\xdef\rule@top{0}\fi
   \ifx\first@\second@ \xdef\rule@num{0}
   \else
     \xdef\first@{#3 @} \expandafter\check@letter\first@
     \xdef\first@{#3}   \ifletter \get@name@number \fi
     \ifnum\first@>\seq@count
     \else
        \ifnum\first@>0
           \xdef\rule@num{\first@}
        \fi
      \fi
   \fi
   \xdef\ruler@{}}
\def\hideruler{\xdef\rule@num{-1}}
\def\allowzero{\xdef\allow@zero{y}}
\def\disallowzero{\xdef\allow@zero{n}}
\def\rulersteps#1{%
   \xdef\ruler@step{#1}
   \ifnum#1<4 \xdef\ruler@rot{90}\fi
}
\def\rotateruler{\xdef\ruler@rot{90}}
\def\unrotateruler{\xdef\ruler@rot{0}}
\def\namerulerpos#1#2{%
   \expandafter\xdef\csname alt@ruler#1\endcsname{#2}
}
\def\featurerule#1{\setlength\rule@thick{#1}}
\def\orderseqs#1{%
   \def\order@loop{%
      \expandafter\check@letter\first@
      \ifletter
         \expandafter\get@item\first@
         \xdef\first@{\fourth@}
         \get@name@number
         \xdef\seq@order{\seq@order,\first@}
         \xdef\first@{\first@@ @}
         \order@loop
      \else
         \expandafter\get@digit\first@
         \ifx\fourth@\ampers@nd
         \else
            \xdef\seq@order{\seq@order,\fourth@}
            \order@loop
         \fi
      \fi}
   \xdef\first@{#1,&,@}
   \xdef\seq@order{}
   \order@loop
   \xdef\seq@order{\seq@order @}
   \expandafter\get@item\seq@order
   \xdef\seq@order{\first@@,@}
}
\def\setfamily#1#2{%
   \xdef\second@{#2&}
   \ifx\second@\ampers@nd
   \else
   \xdef\first@{#1}
   \xdef\second@{#2}
   \xdef\temp@{rm}
   \ifx\second@\temp@
      \xdef\third@{\rmdefault}
   \else
      \xdef\temp@{sf}
      \ifx\second@\temp@
         \xdef\third@{\sfdefault}
      \else
         \xdef\temp@{tt}
         \ifx\second@\temp@
            \xdef\third@{\ttdefault}
         \else
            \xdef\third@{\second@}
   \fi\fi\fi
   \xdef\temp@{featurenames}
   \ifx\first@\temp@ \xdef\ftext@family{\third@}
   \else
   \xdef\temp@{featurestylenames}
   \ifx\first@\temp@ \xdef\fstyles@family{\third@}
   \else
   \xdef\temp@{features}
   \ifx\first@\temp@ \xdef\featuretext@family{\third@}
   \else
   \xdef\temp@{featurestyles}
   \ifx\first@\temp@ \xdef\featurestyles@family{\third@}
   \else
   \xdef\temp@{numbering}
   \ifx\first@\temp@ \xdef\numbertext@family{\third@}
   \else
   \xdef\temp@{names}
   \ifx\first@\temp@ \xdef\namestext@family{\third@}
   \else
   \xdef\temp@{residues}
   \ifx\first@\temp@ \xdef\residues@family{\third@}
   \else
   \xdef\temp@{legend}
   \ifx\first@\temp@ \xdef\legend@family{\third@}
   \else
   \xdef\temp@{labels}
   \ifx\first@\temp@ \xdef\label@family{\third@}
   \else
   \xdef\temp@{ruler}
   \ifx\first@\temp@ \xdef\ruler@family{\second@} 
   \else
   \xdef\temp@{all}
   \ifx\first@\temp@
      \xdef\ftext@family{\third@}
      \xdef\fstyles@family{\third@}
      \xdef\featuretext@family{\third@}
      \xdef\featurestyles@family{\third@}
      \xdef\numbertext@family{\third@}
      \xdef\namestext@family{\third@}
      \xdef\residues@family{\third@}
      \xdef\legend@family{\third@}
      \xdef\label@family{\third@}
      \xdef\ruler@family{\second@}
   \fi\fi\fi\fi\fi\fi\fi\fi\fi\fi\fi
   \fi
}
\def\setseries#1#2{%
   \xdef\second@{#2&}
   \ifx\second@\ampers@nd
   \else
   \xdef\first@{#1}
   \xdef\second@{#2}
   \xdef\temp@{bf}
   \ifx\second@\temp@
      \xdef\third@{\bfdefault}
   \else
      \xdef\temp@{md}
      \ifx\second@\temp@
         \xdef\third@{\mddefault}
      \else
         \xdef\third@{\second@}
   \fi\fi
   \xdef\temp@{featurenames}
   \ifx\first@\temp@ \xdef\ftext@series{\third@}
   \else
   \xdef\temp@{featurestylenames}
   \ifx\first@\temp@ \xdef\fstyles@series{\third@}
   \else
   \xdef\temp@{features}
   \ifx\first@\temp@ \xdef\featuretext@series{\third@}
   \else
   \xdef\temp@{featurestyles}
   \ifx\first@\temp@ \xdef\featurestyles@series{\third@}
   \else
   \xdef\temp@{numbering}
   \ifx\first@\temp@ \xdef\numbertext@series{\third@}
   \else
   \xdef\temp@{names}
   \ifx\first@\temp@ \xdef\namestext@series{\third@}
   \else
   \xdef\temp@{residues}
   \ifx\first@\temp@ \xdef\residues@series{\third@}
   \else
   \xdef\temp@{legend}
   \ifx\first@\temp@ \xdef\legend@series{\third@}
   \else
   \xdef\temp@{labels}
   \ifx\first@\temp@ \xdef\label@series{\third@}
   \else
   \xdef\temp@{all}
   \ifx\first@\temp@
      \xdef\ftext@series{\third@}
      \xdef\fstyles@series{\third@}
      \xdef\featuretext@series{\third@}
      \xdef\featurestyles@series{\third@}
      \xdef\numbertext@series{\third@}
      \xdef\namestext@series{\third@}
      \xdef\residues@series{\third@}
      \xdef\legend@series{\third@}
      \xdef\label@series{\third@}
   \fi\fi\fi\fi\fi\fi\fi\fi\fi\fi
   \fi
}
\def\setshape#1#2{%
   \xdef\second@{#2&}
   \ifx\second@\ampers@nd
   \else
   \xdef\first@{#1}
   \xdef\second@{#2}
   \xdef\temp@{it}
   \ifx\second@\temp@
      \xdef\third@{\itdefault}
   \else
      \xdef\temp@{sl}
      \ifx\second@\temp@
         \xdef\third@{\sldefault}
      \else
         \xdef\temp@{sc}
         \ifx\second@\temp@
            \xdef\third@{\scdefault}
         \else
            \xdef\temp@{up}
            \ifx\second@\temp@
               \xdef\third@{\updefault}
            \else
               \xdef\third@{\second@}
   \fi\fi\fi\fi
   \xdef\temp@{featurenames}
   \ifx\first@\temp@ \xdef\ftext@shape{\third@}
   \else
   \xdef\temp@{featurestylenames}
   \ifx\first@\temp@ \xdef\fstyles@shape{\third@}
   \else
   \xdef\temp@{features}
   \ifx\first@\temp@ \xdef\featuretext@shape{\third@}
   \else
   \xdef\temp@{featurestyles}
   \ifx\first@\temp@ \xdef\featurestyles@shape{\third@}
   \else
   \xdef\temp@{numbering}
   \ifx\first@\temp@ \xdef\numbertext@shape{\third@}
   \else
   \xdef\temp@{names}
   \ifx\first@\temp@ \xdef\namestext@shape{\third@}
   \else
   \xdef\temp@{residues}
   \ifx\first@\temp@ \xdef\residues@shape{\third@}
   \else
   \xdef\temp@{legend}
   \ifx\first@\temp@ \xdef\legend@shape{\third@}
   \else
   \xdef\temp@{labels}
   \ifx\first@\temp@ \xdef\label@shape{\third@}
   \else
   \xdef\temp@{all}
   \ifx\first@\temp@
      \xdef\ftext@shape{\third@}
      \xdef\fstyles@shape{\third@}
      \xdef\featuretext@shape{\third@}
      \xdef\featurestyles@shape{\third@}
      \xdef\numbertext@shape{\third@}
      \xdef\namestext@shape{\third@}
      \xdef\residues@shape{\third@}
      \xdef\legend@shape{\third@}
      \xdef\label@shape{\third@}
   \fi\fi\fi\fi\fi\fi\fi\fi\fi\fi
   \fi
}
\def\setsize#1#2{%
   \xdef\second@{#2&}
   \ifx\second@\ampers@nd
   \else
   \xdef\first@{#1}
   \xdef\temp@{features}
   \ifx\first@\temp@
      \def\featuretext@size{\csname #2\endcsname}
   \else
   \xdef\temp@{featurestyles}
   \ifx\first@\temp@
      \def\featurestyles@size{\csname #2\endcsname}
   \else
   \xdef\temp@{featurenames}
   \ifx\first@\temp@
      \def\ftext@size{\csname #2\endcsname}
   \else
   \xdef\temp@{featurestylenames}
   \ifx\first@\temp@
      \def\fstyles@size{\csname #2\endcsname}
   \else
   \xdef\temp@{numbering}
   \ifx\first@\temp@
      \def\numbertext@size{\csname #2\endcsname}
   \else
   \xdef\temp@{names}
   \ifx\first@\temp@
      \def\namestext@size{\csname #2\endcsname}
   \else
   \xdef\temp@{legend}
   \ifx\first@\temp@
      \def\legend@size{\csname #2\endcsname}
   \else
   \xdef\temp@{labels}
   \ifx\first@\temp@
      \def\label@size{\csname #2\endcsname}
   \else
   \xdef\temp@{residues}
   \ifx\first@\temp@
      \def\residues@size{\csname #2\endcsname}
      \xdef\res@size{#2}
   \else
   \xdef\temp@{all}
   \ifx\first@\temp@
      \def\featuretext@size{\csname #2\endcsname}
      \def\featurestyles@size{\csname #2\endcsname}
      \def\ftext@size{\csname #2\endcsname}
      \def\fstyles@size{\csname #2\endcsname}
      \def\numbertext@size{\csname #2\endcsname}
      \def\namestext@size{\csname #2\endcsname}
      \def\legend@size{\csname #2\endcsname}
      \def\label@size{\csname #2\endcsname}
      \def\residues@size{\csname #2\endcsname}
      \xdef\res@size{#2}
   \fi\fi\fi\fi\fi\fi\fi\fi\fi\fi
   \xdef\temp@{Huge}
   \ifx\temp@\res@size
      \def\bottomruler@size{\csname Large\endcsname}
   \else
   \xdef\temp@{huge}
   \ifx\temp@\res@size
      \def\bottomruler@size{\csname large\endcsname}
   \else
   \xdef\temp@{LARGE}
   \ifx\temp@\res@size
      \def\bottomruler@size{\csname normalsize\endcsname}
   \else
   \xdef\temp@{Large}
   \ifx\temp@\res@size
      \def\bottomruler@size{\csname small\endcsname}
   \else
   \xdef\temp@{large}
   \ifx\temp@\res@size
      \def\bottomruler@size{\csname footnotesize\endcsname}
   \else
   \xdef\temp@{normalsize}
   \ifx\temp@\res@size
      \def\bottomruler@size{\csname scriptsize\endcsname}
   \else
      \def\bottomruler@size{\csname tiny\endcsname}
   \fi\fi\fi\fi\fi\fi
   \fi
   \xdef\temp@{ruler}
   \ifx\first@\temp@
      \def\bottomruler@size{\csname #2\endcsname}
   \fi
}
\def\setfont#1#2#3#4#5{%
   \setfamily{#1}{#2}\setseries{#1}{#3}
   \setshape{#1}{#4}\setsize{#1}{#5}}
\def\featurenamesrm{\setfamily{featurenames}{rm}}
\def\featurenamessf{\setfamily{featurenames}{sf}}
\def\featurenamestt{\setfamily{featurenames}{tt}}
\def\featurenamesmd{\setseries{featurenames}{md}}
\def\featurenamesbf{\setseries{featurenames}{bf}}
\def\featurenamesup{\setshape {featurenames}{up}}
\def\featurenamesit{\setshape {featurenames}{it}}
\def\featurenamessl{\setshape {featurenames}{sl}}
\def\featurenamessc{\setshape {featurenames}{sc}}
\def\featurenamestiny        {\setsize{featurenames}{tiny}}
\def\featurenamesscriptsize  {\setsize{featurenames}{scriptsize}}
\def\featurenamesfootnotesize{\setsize{featurenames}{footnotesize}}
\def\featurenamessmall       {\setsize{featurenames}{small}}
\def\featurenamesnormalsize  {\setsize{featurenames}{normalsize}}
\def\featurenameslarge       {\setsize{featurenames}{large}}
\def\featurenamesLarge       {\setsize{featurenames}{Large}}
\def\featurenamesLARGE       {\setsize{featurenames}{LARGE}}
\def\featurenameshuge        {\setsize{featurenames}{huge}}
\def\featurenamesHuge        {\setsize{featurenames}{Huge}}
\def\featurestylenamesrm{\setfamily{featurestylenames}{rm}}
\def\featurestylenamessf{\setfamily{featurestylenames}{sf}}
\def\featurestylenamestt{\setfamily{featurestylenames}{tt}}
\def\featurestylenamesmd{\setseries{featurestylenames}{md}}
\def\featurestylenamesbf{\setseries{featurestylenames}{bf}}
\def\featurestylenamesup{\setshape {featurestylenames}{up}}
\def\featurestylenamesit{\setshape {featurestylenames}{it}}
\def\featurestylenamessl{\setshape {featurestylenames}{sl}}
\def\featurestylenamessc{\setshape {featurestylenames}{sc}}
\def\featurestylenamestiny        {\setsize{featurestylenames}{tiny}}
\def\featurestylenamesscriptsize  {\setsize{featurestylenames}{scriptsize}}
\def\featurestylenamesfootnotesize{\setsize{featurestylenames}{footnotesize}}
\def\featurestylenamessmall       {\setsize{featurestylenames}{small}}
\def\featurestylenamesnormalsize  {\setsize{featurestylenames}{normalsize}}
\def\featurestylenameslarge       {\setsize{featurestylenames}{large}}
\def\featurestylenamesLarge       {\setsize{featurestylenames}{Large}}
\def\featurestylenamesLARGE       {\setsize{featurestylenames}{LARGE}}
\def\featurestylenameshuge        {\setsize{featurestylenames}{huge}}
\def\featurestylenamesHuge        {\setsize{featurestylenames}{Huge}}
\def\featuresrm{\setfamily{features}{rm}}
\def\featuressf{\setfamily{features}{sf}}
\def\featurestt{\setfamily{features}{tt}}
\def\featuresmd{\setseries{features}{md}}
\def\featuresbf{\setseries{features}{bf}}
\def\featuresup{\setshape {features}{up}}
\def\featuresit{\setshape {features}{it}}
\def\featuressl{\setshape {features}{sl}}
\def\featuressc{\setshape {features}{sc}}
\def\featurestiny        {\setsize{features}{tiny}}
\def\featuresscriptsize  {\setsize{features}{scriptsize}}
\def\featuresfootnotesize{\setsize{features}{footnotesize}}
\def\featuressmall       {\setsize{features}{small}}
\def\featuresnormalsize  {\setsize{features}{normalsize}}
\def\featureslarge       {\setsize{features}{large}}
\def\featuresLarge       {\setsize{features}{Large}}
\def\featuresLARGE       {\setsize{features}{LARGE}}
\def\featureshuge        {\setsize{features}{huge}}
\def\featuresHuge        {\setsize{features}{Huge}}
\def\featurestylesrm{\setfamily{featurestyles}{rm}}
\def\featurestylessf{\setfamily{featurestyles}{sf}}
\def\featurestylestt{\setfamily{featurestyles}{tt}}
\def\featurestylesmd{\setseries{featurestyles}{md}}
\def\featurestylesbf{\setseries{featurestyles}{bf}}
\def\featurestylesup{\setshape {featurestyles}{up}}
\def\featurestylesit{\setshape {featurestyles}{it}}
\def\featurestylessl{\setshape {featurestyles}{sl}}
\def\featurestylessc{\setshape {featurestyles}{sc}}
\def\featurestylestiny        {\setsize{featurestyles}{tiny}}
\def\featurestylesscriptsize  {\setsize{featurestyles}{scriptsize}}
\def\featurestylesfootnotesize{\setsize{featurestyles}{footnotesize}}
\def\featurestylessmall       {\setsize{featurestyles}{small}}
\def\featurestylesnormalsize  {\setsize{featurestyles}{normalsize}}
\def\featurestyleslarge       {\setsize{featurestyles}{large}}
\def\featurestylesLarge       {\setsize{featurestyles}{Large}}
\def\featurestylesLARGE       {\setsize{featurestyles}{LARGE}}
\def\featurestyleshuge        {\setsize{featurestyles}{huge}}
\def\featurestylesHuge        {\setsize{featurestyles}{Huge}}
\def\numberingrm{\setfamily{numbering}{rm}}
\def\numberingsf{\setfamily{numbering}{sf}}
\def\numberingtt{\setfamily{numbering}{tt}}
\def\numberingmd{\setseries{numbering}{md}}
\def\numberingbf{\setseries{numbering}{bf}}
\def\numberingup{\setshape {numbering}{up}}
\def\numberingit{\setshape {numbering}{it}}
\def\numberingsl{\setshape {numbering}{sl}}
\def\numberingsc{\setshape {numbering}{sc}}
\def\numberingtiny        {\setsize{numbering}{tiny}}
\def\numberingscriptsize  {\setsize{numbering}{scriptsize}}
\def\numberingfootnotesize{\setsize{numbering}{footnotesize}}
\def\numberingsmall       {\setsize{numbering}{small}}
\def\numberingnormalsize  {\setsize{numbering}{normalsize}}
\def\numberinglarge       {\setsize{numbering}{large}}
\def\numberingLarge       {\setsize{numbering}{Large}}
\def\numberingLARGE       {\setsize{numbering}{LARGE}}
\def\numberinghuge        {\setsize{numbering}{huge}}
\def\numberingHuge        {\setsize{numbering}{Huge}}
\def\namesrm{\setfamily{names}{rm}}
\def\namessf{\setfamily{names}{sf}}
\def\namestt{\setfamily{names}{tt}}
\def\namesmd{\setseries{names}{md}}
\def\namesbf{\setseries{names}{bf}}
\def\namesup{\setshape {names}{up}}
\def\namesit{\setshape {names}{it}}
\def\namessl{\setshape {names}{sl}}
\def\namessc{\setshape {names}{sc}}
\def\namestiny        {\setsize{names}{tiny}}
\def\namesscriptsize  {\setsize{names}{scriptsize}}
\def\namesfootnotesize{\setsize{names}{footnotesize}}
\def\namessmall       {\setsize{names}{small}}
\def\namesnormalsize  {\setsize{names}{normalsize}}
\def\nameslarge       {\setsize{names}{large}}
\def\namesLarge       {\setsize{names}{Large}}
\def\namesLARGE       {\setsize{names}{LARGE}}
\def\nameshuge        {\setsize{names}{huge}}
\def\namesHuge        {\setsize{names}{Huge}}
\def\residuesrm{\setfamily{residues}{rm}}
\def\residuessf{\setfamily{residues}{sf}}
\def\residuestt{\setfamily{residues}{tt}}
\def\residuesmd{\setseries{residues}{md}}
\def\residuesbf{\setseries{residues}{bf}}
\def\residuesup{\setshape {residues}{up}}
\def\residuesit{\setshape {residues}{it}}
\def\residuessl{\setshape {residues}{sl}}
\def\residuessc{\setshape {residues}{sc}}
\def\residuestiny        {\setsize{residues}{tiny}}
\def\residuesscriptsize  {\setsize{residues}{scriptsize}}
\def\residuesfootnotesize{\setsize{residues}{footnotesize}}
\def\residuessmall       {\setsize{residues}{small}}
\def\residuesnormalsize  {\setsize{residues}{normalsize}}
\def\residueslarge       {\setsize{residues}{large}}
\def\residuesLarge       {\setsize{residues}{Large}}
\def\residuesLARGE       {\setsize{residues}{LARGE}}
\def\residueshuge        {\setsize{residues}{huge}}
\def\residuesHuge        {\setsize{residues}{Huge}}
\def\legendrm{\setfamily{legend}{rm}}
\def\legendsf{\setfamily{legend}{sf}}
\def\legendtt{\setfamily{legend}{tt}}
\def\legendmd{\setseries{legend}{md}}
\def\legendbf{\setseries{legend}{bf}}
\def\legendup{\setshape {legend}{up}}
\def\legendit{\setshape {legend}{it}}
\def\legendsl{\setshape {legend}{sl}}
\def\legendsc{\setshape {legend}{sc}}
\def\legendtiny        {\setsize{legend}{tiny}}
\def\legendscriptsize  {\setsize{legend}{scriptsize}}
\def\legendfootnotesize{\setsize{legend}{footnotesize}}
\def\legendsmall       {\setsize{legend}{small}}
\def\legendnormalsize  {\setsize{legend}{normalsize}}
\def\legendlarge       {\setsize{legend}{large}}
\def\legendLarge       {\setsize{legend}{Large}}
\def\legendLARGE       {\setsize{legend}{LARGE}}
\def\legendhuge        {\setsize{legend}{huge}}
\def\legendHuge        {\setsize{legend}{Huge}}
\def\rulerrm{\setfamily{ruler}{rm}}
\def\rulersf{\setfamily{ruler}{sf}}
\def\rulertt{\setfamily{ruler}{tt}}
\def\rulertiny        {\setsize{ruler}{tiny}}
\def\rulerscriptsize  {\setsize{ruler}{scriptsize}}
\def\rulerfootnotesize{\setsize{ruler}{footnotesize}}
\def\rulersmall       {\setsize{ruler}{small}}
\def\rulernormalsize  {\setsize{ruler}{normalsize}}
\def\rulerlarge       {\setsize{ruler}{large}}
\def\rulerLarge       {\setsize{ruler}{Large}}
\def\rulerLARGE       {\setsize{ruler}{LARGE}}
\def\rulerhuge        {\setsize{ruler}{huge}}
\def\rulerHuge        {\setsize{ruler}{Huge}}
\def\funcshadingstyle#1#2#3#4#5{%
  \xdef\temp@{nomatch} \xdef\first@{#1}
  \ifx\temp@\first@
     \xdef\first@{0}
  \else
     \xdef\first@{\csname funcgrp#1\endcsname}
  \fi
  \ifnum\first@>-1
     \expandafter\xdef\csname fg@textcolor\first@\endcsname{#2}
     \expandafter\xdef\csname fg@color\first@\endcsname{#3}
     \expandafter\xdef\csname funcm@tch\first@\endcsname{#4}
     \expandafter\def\csname func@style\first@\endcsname{\csname text#5\endcsname}
  \fi}
\def\defshadingcolors#1{%  
   \expandafter\xdef\csname TextNomatch@#1\endcsname{\TextNomatch}
   \expandafter\xdef\csname Nomatch@#1\endcsname{\Nomatch}
   \expandafter\xdef\csname resn@m@tch@#1\endcsname{\resn@m@tch}
      
   \expandafter\xdef\csname TextSimilar@#1\endcsname{\TextSimilar}
   \expandafter\xdef\csname Similar@#1\endcsname{\Similar}
   \expandafter\xdef\csname ressimm@tch@#1\endcsname{\ressimm@tch}
   
   \expandafter\xdef\csname TextIdentical@#1\endcsname{\TextIdentical}
   \expandafter\xdef\csname Identical@#1\endcsname{\Identical}
   \expandafter\xdef\csname resm@tch@#1\endcsname{\resm@tch}
   
   \expandafter\xdef\csname TextAllmatch@#1\endcsname{\TextAllmatch}
   \expandafter\xdef\csname Allmatch@#1\endcsname{\Allmatch}
   \expandafter\xdef\csname res@llm@tch@#1\endcsname{\res@llm@tch}
   
   \expandafter\xdef\csname gap@fg@#1\endcsname{\gap@fg}
   \expandafter\xdef\csname gap@bg@#1\endcsname{\gap@bg}
   
}
\def\shadingcolors#1{%
  \gapcolors{Black}{White}
  \nomatchresidues{Black}{White}{upper}{up}
  \xdef\first@{#1} \xdef\second@{blues}
  \ifx\first@\second@
    \similarresidues{Black}{Magenta}{upper}{up}
    \conservedresidues{White}{RoyalBlue}{upper}{up}
    \allmatchresidues{Goldenrod}{RoyalPurple}{upper}{up}
  \else \xdef\second@{greens}
  \ifx\first@\second@
    \similarresidues{Black}{GreenYellow}{upper}{up}
    \conservedresidues{White}{PineGreen}{upper}{up}
    \allmatchresidues{YellowOrange}{OliveGreen}{upper}{up}
  \else \xdef\second@{reds}
  \ifx\first@\second@
    \similarresidues{Black}{YellowOrange}{upper}{up}
    \conservedresidues{White}{BrickRed}{upper}{up}
    \allmatchresidues{YellowGreen}{Mahagony}{upper}{up}
  \else \xdef\second@{black}
  \ifx\first@\second@
    \similarresidues{Black}{White}{upper}{sl}
    \conservedresidues{White}{Black}{upper}{up}
    \allmatchresidues{White}{Black}{upper}{sl}
  \else \xdef\second@{grays}
  \ifx\first@\second@
    \similarresidues{Black}{LightGray}{upper}{up}
    \conservedresidues{White}{DarkGray}{upper}{up}
    \allmatchresidues{White}{Black}{upper}{up}
  \else 
    \xdef\TextNomatch{\csname TextNomatch@#1\endcsname}
    \xdef\Nomatch    {\csname Nomatch@#1\endcsname}
    \xdef\resn@m@tch {\csname resn@m@tch@#1\endcsname}
    \xdef\TextSimilar{\csname TextSimilar@#1\endcsname}
    \xdef\Similar    {\csname Similar@#1\endcsname}
    \xdef\ressimm@tch{\csname ressimm@tch@#1\endcsname}
    \xdef\TextIdentical{\csname TextIdentical@#1\endcsname}
    \xdef\Identical    {\csname Identical@#1\endcsname}
    \xdef\resm@tch     {\csname resm@tch@#1\endcsname}
    \xdef\TextAllmatch{\csname TextAllmatch@#1\endcsname}
    \xdef\Allmatch    {\csname Allmatch@#1\endcsname}
    \xdef\res@llm@tch {\csname res@llm@tch@#1\endcsname}
    \xdef\gap@fg    {\csname gap@fg@#1\endcsname}
    \xdef\gap@bg    {\csname gap@bg@#1\endcsname}
    \xdef\domgap@fg {\csname domgap@fg@#1\endcsname}
    \xdef\domgap@bg {\csname domgap@bg@#1\endcsname}
  \fi\fi\fi\fi\fi
  \xdef\first@{#1}
  \expandafter\defshadingcolors{\first@}
}
\def\nomatchresidues#1#2#3#4  {\xdef\first@{#1&}\xdef\second@{#2&}\xdef\third@{#3&}
                             \xdef\last@{\ampers@nd}
                             \ifx\first@\last@\else\def\TextNomatch{#1}
                               \expandafter\def\csname fg@textcolor0\endcsname{#1}
                             \fi
                             \ifx\second@\last@\else\gdef\Nomatch{#2}
                               \expandafter\def\csname fg@color0\endcsname{#2}
                             \fi
                             \ifx\third@\last@\else\def\resn@m@tch{#3}
                             \fi
                             \xdef\first@{#4&}
                             \ifx\first@\last@\else
                               \def\no@style{\csname text#4\endcsname}
                               \expandafter\def\csname func@style0\endcsname%
                                  {\csname text#4\endcsname}\fi}
\def\similarresidues#1#2#3#4  {\xdef\first@{#1&}\xdef\second@{#2&}\xdef\third@{#3&}
                             \xdef\last@{\ampers@nd}
                             \ifx\first@\last@\else\def\TextSimilar{#1}\fi
                             \ifx\second@\last@\else\gdef\Similar{#2}\fi
                             \ifx\third@\last@\else\def\ressimm@tch{#3}\fi
                             \xdef\first@{#4&}
                             \ifx\first@\last@\else
                               \def\sim@style{\csname text#4\endcsname}\fi}
\def\conservedresidues#1#2#3#4{\xdef\first@{#1&}\xdef\second@{#2&}\xdef\third@{#3&}
                             \xdef\last@{\ampers@nd}
                             \ifx\first@\last@\else\def\TextIdentical{#1}\fi
                             \ifx\second@\last@\else\gdef\Identical{#2}\fi
                             \ifx\third@\last@\else\def\resm@tch{#3}\fi
                             \xdef\first@{#4&}
                             \ifx\first@\last@\else
                               \def\id@style{\csname text#4\endcsname}\fi}
\def\allmatchresidues#1#2#3#4 {\xdef\first@{#1&}\xdef\second@{#2&}\xdef\third@{#3&}
                             \xdef\last@{\ampers@nd}
                             \ifx\first@\last@\else\def\TextAllmatch{#1}\fi
                             \ifx\second@\last@\else\gdef\Allmatch{#2}\fi
                             \ifx\third@\last@\else\def\res@llm@tch{#3}\fi
                             \xdef\first@{#4&}
                             \ifx\first@\last@\else
                               \def\all@style{\csname text#4\endcsname}\fi}
\def\gapcolors#1#2           {\xdef\first@{#1&}\xdef\second@{#2&}
                             \xdef\last@{\ampers@nd}
                             \ifx\first@\last@\else\def\gap@fg{#1}
                                \expandafter\def\csname fg@textcolor*\endcsname{#1}\fi
                             \ifx\second@\last@\else\def\gap@bg{#2}
                                \expandafter\def\csname fg@color*\endcsname{#2}\fi}
\def\domaingapcolors#1#2     {\xdef\first@{#1&}\xdef\second@{#2&}
                             \xdef\last@{\ampers@nd}
                             \ifx\first@\last@\else\def\domgap@fg{#1}
                                \expandafter\def\csname fg@textcolor!\endcsname{#1}\fi
                             \ifx\second@\last@\else\def\domgap@bg{#2}
                                \expandafter\def\csname fg@color!\endcsname{#2}\fi}
\def\shadebox#1{%
   \xdef\first@{White}%
   \xdef\third@{#1}%
   \xdef\second@{nomatch}%
   \ifx\second@\third@
      \ifx\Nomatch\first@\white@box\else\textcolor{\Nomatch}{\box@rule}\fi%
   \else
   \xdef\second@{similar}%
   \ifx\second@\third@
      \ifx\Similar\first@\white@box\else\textcolor{\Similar}{\box@rule}\fi%
   \else
   \xdef\second@{conserved}%
   \ifx\second@\third@
      \ifx\Identical\first@\white@box\else\textcolor{\Identical}{\box@rule}\fi%
   \else
   \xdef\second@{allmatch}%
   \ifx\second@\third@
      \ifx\Allmatch\first@\white@box\else\textcolor{\Allmatch}{\box@rule}\fi%
   \else
   \ifx\third@\first@\white@box\else\textcolor{\third@}{\box@rule}\fi
   \fi\fi\fi\fi}
\def\featurenamescolor#1{%
   \expandafter\xdef\csname ftext@fg@ttttop\endcsname{#1}
   \expandafter\xdef\csname ftext@fg@tttop\endcsname{#1}
   \expandafter\xdef\csname ftext@fg@ttop\endcsname{#1}
   \expandafter\xdef\csname ftext@fg@top\endcsname{#1}
   \expandafter\xdef\csname ftext@fg@bottom\endcsname{#1}
   \expandafter\xdef\csname ftext@fg@bbottom\endcsname{#1}
   \expandafter\xdef\csname ftext@fg@bbbottom\endcsname{#1}
   \expandafter\xdef\csname ftext@fg@bbbbottom\endcsname{#1}
}
\def\featurestylenamescolor#1{%
   \expandafter\xdef\csname fstyles@fg@ttttop\endcsname{#1}
   \expandafter\xdef\csname fstyles@fg@tttop\endcsname{#1}
   \expandafter\xdef\csname fstyles@fg@ttop\endcsname{#1}
   \expandafter\xdef\csname fstyles@fg@top\endcsname{#1}
   \expandafter\xdef\csname fstyles@fg@bottom\endcsname{#1}
   \expandafter\xdef\csname fstyles@fg@bbottom\endcsname{#1}
   \expandafter\xdef\csname fstyles@fg@bbbottom\endcsname{#1}
   \expandafter\xdef\csname fstyles@fg@bbbbottom\endcsname{#1}
}
\def\featurenamecolor#1#2{\expandafter\xdef\csname ftext@fg@#1\endcsname{#2}}
\def\featurestylenamecolor#1#2{\expandafter\xdef\csname fstyles@fg@#1\endcsname{#2}}
\def\namescolor#1{\xdef\names@fg{#1}}
\def\namecolor#1#2{%
   \xdef\first@{consensus} \xdef\second@{#1}
   \ifx\first@\second@
      \expandafter\xdef\csname name@col0\endcsname{#2}
   \else
   \xdef\first@{featurenames} \xdef\second@{#1}
   \ifx\first@\second@
     \xdef\ftext@fg{#2}
   \else
   \xdef\first@{featurestylenames} \xdef\second@{#1}
   \ifx\first@\second@
     \xdef\fstyles@fg{#2}
   \else
      \xdef\first@{#1,&,@} \xdef\third@{#2} \namecolor@
   \fi\fi\fi
}
\def\numberingcolor#1{\xdef\numbering@fg{#1}}
\def\numbercolor#1#2{%
   \xdef\first@{consensus} \xdef\second@{#1}
   \ifx\first@\second@
      \expandafter\xdef\csname number@col0\endcsname{#2}
   \else
      \xdef\first@{#1,&,@} \xdef\third@{#2} \numbercolor@
   \fi
}
\def\legendcolor#1{\xdef\legend@fg{#1}}
\def\rulercolor#1{\xdef\ruler@fg{#1}}
\def\molweight#1#2{%
   \xdef\temp@{Da}%
   \xdef\second@{#2}%
   \ifx\second@\temp@\xdef\third@{Da}\else\xdef\third@{kDa}\fi%
   \xdef\first@{#1 @} \expandafter\check@letter\first@
   \xdef\first@{#1}   \ifletter \get@name@number \fi
   \xdef\first@{\csname @rd\first@\endcsname}%
   \loopcount=\csname mol@weight\first@\endcsname%
   \divide\loopcount by 10\relax%
   \innerloopcount=\loopcount%
   \hbox{%
   \ifnum\loopcount>1000%
      \divide\loopcount by 1000\relax%
      \pos@count=\loopcount%
      \multiply\loopcount by 1000\relax%
      \advance\innerloopcount by -\loopcount%
      \loopcount=\innerloopcount%
      \ifnum\loopcount>949\advance\pos@count by 1\relax\fi%
      \the\pos@count%
      \ifx\temp@\second@\ifgerm@n .\else {,}\fi\fi%
   \else%
      \ifx\second@\temp@ \else 0\fi%
   \fi%
   \ifx\second@\temp@%
      \the\loopcount%
      \loopcount=\csname mol@weight\first@\endcsname%
      \innerloopcount=\loopcount%
      \divide\loopcount by 10\relax%
      \multiply\loopcount by 10\relax%
      \advance\innerloopcount by -\loopcount\relax%
   \else%
      \divide\innerloopcount by 10\relax%
      \advance\innerloopcount by 5\relax%
      \divide\innerloopcount by 10\relax%
   \fi%
   \ifnum\innerloopcount>9\relax\innerloopcount=0\relax\fi%
   \ifgerm@n {,}\else .\fi%
   \the\innerloopcount~\third@}}
\newcommand\charge[2][o]{%
 \xdef\temp@{pep}%
 \ifx\prefix@\temp@%
   \xdef\first@{#2 @} \expandafter\check@letter\first@
   \xdef\first@{#2}   \ifletter \get@name@number \fi
   \xdef\second@{\csname @rd\first@\endcsname}%
   \loopcount=\csname ch@rge\second@\endcsname%
   \xdef\first@{#1}\make@lower%
   \if\first@ i\fi%
   \if\first@ o\advance\loopcount by \chargeNterm%
               \advance\loopcount by \chargeCterm\fi%
   \if\first@ n\advance\loopcount by \chargeNterm\fi%
   \if\first@ c\advance\loopcount by \chargeCterm\fi%
   \hbox{\ensuremath{%
   \ifnum\loopcount>0 +%
   \else\ifnum\loopcount=0 \pm%
   \else -\multiply\loopcount by -1\relax%
   \fi\fi%
   \innerloopcount=\loopcount%
   \divide\loopcount by 1000\relax%
   \the\loopcount%
   \multiply\loopcount by 1000\relax%
   \advance\innerloopcount by -\loopcount\relax%
   \divide\innerloopcount by 10\relax%
   \ifnum\innerloopcount=0%
   \else%
      \ifgerm@n {,}\else .\fi%
      \ifnum\innerloopcount<10 0\fi%
      \the\innerloopcount%
   \fi}}%
 \fi}
\def\TeXshade{%
   \setbox1=\hbox{\texttt{H}}%
   \def\logo@rule{\vrule depth0.25\ht1 height1.25\ht1 width\wd1}%
   \TeX%
   \logo@rule\kern-\wd1\textcolor{White}{\texttt{s}}%
   \logo@rule\kern-\wd1\textcolor{White}{\texttt{h}}%
   \texttt{a}%
   \logo@rule\kern-\wd1\textcolor{White}{\texttt{d}}%
   \texttt{e}}

\def\includeTCoffee#1{%
   \xdef\TC@first@{#1}\include@T@coffee}   
   
\def\firstcolumnDSSP{\xdef\fc@DSSP{y}}
\def\secondcolumnDSSP{\xdef\fc@DSSP{n}}
   
\newcommand{\includeDSSP}[3][existing]{%
  \temp@count=\dssp@num
  \advance\temp@count by 1
  \xdef\dssp@num{\the\temp@count}
  \xdef\first@{#1} \xdef\temp@{existing}
  \ifx\first@\temp@ \else\xdef\first@{make new}\fi
  \expandafter\xdef\csname optiondssp\the\temp@count\endcsname{\first@}
  \xdef\first@{#2 @} \expandafter\check@letter\first@
  \xdef\first@{#2}   \ifletter \get@name@number \fi
  \expandafter\xdef\csname doseqdssp\the\temp@count\endcsname{\first@}
  \expandafter\xdef\csname filenamedssp\the\temp@count\endcsname{#3}
  \expandafter\ifnum\csname doseqdssp\the\temp@count\endcsname>\seq@count
     \message{<Ignoring `#2' in \noexpand\includeDSSP>}
     \advance\temp@count by -1
     \xdef\dssp@num{\the\temp@count}
  \fi
}

\newcommand{\includeHMMTOP}[3][existing]{%
  \temp@count=\HMMTOP@num
  \advance\temp@count by 1
  \xdef\HMMTOP@num{\the\temp@count}
  \xdef\first@{#1} \xdef\temp@{existing}
  \ifx\first@\temp@ \else\xdef\first@{make new}\fi
  \expandafter\xdef\csname optionHMMTOP\the\temp@count\endcsname{\first@}
  \xdef\first@{#2[,]&}\expandafter\opt@color\first@
  \ifx\f@color\comm@ 
     \expandafter\xdef\csname fileseqHMMTOP\the\temp@count\endcsname{0} 
  \else
     \expandafter\xdef\csname fileseqHMMTOP\the\temp@count\endcsname{\f@color} 
  \fi
  \xdef\first@{\fourth@ @} \expandafter\check@letter\first@
  \xdef\first@{\fourth@}   \ifletter \get@name@number \fi       
  \expandafter\xdef\csname doseqHMMTOP\the\temp@count\endcsname{\first@}
  \expandafter\xdef\csname filenameHMMTOP\the\temp@count\endcsname{#3}
  \expandafter\ifnum\csname doseqHMMTOP\the\temp@count\endcsname>\seq@count
     \message{<Ignoring `#2' in \noexpand\includeHMMTOP>}
     \advance\temp@count by -1
     \xdef\HMMTOP@num{\the\temp@count}
  \fi
}

\newcommand{\includeSTRIDE}[3][existing]{%
  \temp@count=\stride@num
  \advance\temp@count by 1
  \xdef\stride@num{\the\temp@count}
  \xdef\first@{#1} \xdef\temp@{existing}
  \ifx\first@\temp@ \else\xdef\first@{make new}\fi
  \expandafter\xdef\csname optionstride\the\temp@count\endcsname{\first@}
  \xdef\first@{#2 @} \expandafter\check@letter\first@
  \xdef\first@{#2}   \ifletter \get@name@number \fi
  \expandafter\xdef\csname doseqstride\the\temp@count\endcsname{\first@}
  \expandafter\xdef\csname filenamestride\the\temp@count\endcsname{#3}
  \expandafter\ifnum\csname doseqstride\the\temp@count\endcsname>\seq@count
     \message{<Ignoring `#2' in \noexpand\includeSTRIDE>}
     \advance\temp@count by -1
     \xdef\stride@num{\the\temp@count}
  \fi
}
\newcommand{\includePHDsec}[3][existing]{%
  \temp@count=\PHD@num
  \advance\temp@count by 1
  \xdef\PHD@num{\the\temp@count}
  \xdef\first@{#1} \xdef\temp@{existing}
  \ifx\first@\temp@ \else\xdef\first@{make new}\fi
  \expandafter\xdef\csname optionphd\the\temp@count\endcsname{\first@}
  \xdef\first@{#2 @} \expandafter\check@letter\first@
  \xdef\first@{#2}   \ifletter \get@name@number \fi
  \expandafter\xdef\csname doseqphd\the\temp@count\endcsname{\first@}
  \expandafter\xdef\csname modephd\the\temp@count\endcsname{structure}
  \expandafter\xdef\csname filenamephd\the\temp@count\endcsname{#3}
  \expandafter\ifnum\csname doseqphd\the\temp@count\endcsname>\seq@count
     \message{<Ignoring `#2' in \noexpand\includePHDsec>}
     \advance\temp@count by -1
     \xdef\PHD@num{\the\temp@count}
  \fi
}
\newcommand{\includePHDtopo}[3][existing]{%
  \temp@count=\PHD@num
  \advance\temp@count by 1
  \xdef\PHD@num{\the\temp@count}
  \xdef\first@{#1} \xdef\temp@{existing}
  \ifx\first@\temp@ \else\xdef\first@{make new}\fi
  \expandafter\xdef\csname optionphd\the\temp@count\endcsname{\first@}
  \xdef\first@{#2 @} \expandafter\check@letter\first@
  \xdef\first@{#2}   \ifletter \get@name@number \fi
  \expandafter\xdef\csname doseqphd\the\temp@count\endcsname{\first@}
  \expandafter\xdef\csname modephd\the\temp@count\endcsname{topology}
  \expandafter\xdef\csname filenamephd\the\temp@count\endcsname{#3}
  \expandafter\ifnum\csname doseqphd\the\temp@count\endcsname>\seq@count
     \message{<Ignoring `#2' in \noexpand\includePHDtopo>}
     \advance\temp@count by -1
     \xdef\PHD@num{\the\temp@count}
  \fi
}
\def\appearance#1#2#3#4#5{%
   \xdef\first@{#1} \xdef\second@{#2}
   \xdef\temp@{PHDsec}
   \ifx\temp@\first@
     \xdef\temp@{alpha}
     \ifx\second@\temp@
      \def\bottop@Hsec{#3}
      \def\label@Hsec{#4}
      \def\text@Hsec{#5}
     \else
     \xdef\temp@{beta}
     \ifx\second@\temp@
      \def\bottop@Esec{#3}
      \def\label@Esec{#4}
      \def\text@Esec{#5}
     \fi\fi
   \else
   \xdef\temp@{PHDtopo}
   \ifx\temp@\first@
     \xdef\temp@{internal}
     \ifx\second@\temp@
      \def\bottop@itop{#3}
      \def\label@itop{#4}
      \def\text@itop{#5}
     \else
     \xdef\temp@{external}
     \ifx\second@\temp@
      \def\bottop@etop{#3}
      \def\label@etop{#4}
      \def\text@etop{#5}
     \else
     \xdef\temp@{TM}
     \ifx\second@\temp@
      \def\bottop@TMtop{#3}
      \def\label@TMtop{#4}
      \def\text@TMtop{#5}
     \fi\fi\fi
   \else
   \xdef\temp@{STRIDE}
   \ifx\temp@\first@
     \xdef\temp@{alpha}
     \ifx\second@\temp@
      \def\bottop@Hstride{#3}
      \def\label@Hstride{#4}
      \def\text@Hstride{#5}
     \else
     \xdef\temp@{3-10}
     \ifx\second@\temp@
      \def\bottop@Gstride{#3}
      \def\label@Gstride{#4}
      \def\text@Gstride{#5}
     \else
     \xdef\temp@{pi}
     \ifx\second@\temp@
      \def\bottop@Istride{#3}
      \def\label@Istride{#4}
      \def\text@Istride{#5}
     \else
     \xdef\temp@{beta}
     \ifx\second@\temp@
      \def\bottop@Estride{#3}
      \def\label@Estride{#4}
      \def\text@Estride{#5}
     \else
     \xdef\temp@{bridge}
     \ifx\second@\temp@
      \def\bottop@Bstride{#3}
      \def\label@Bstride{#4}
      \def\text@Bstride{#5}
     \else
     \xdef\temp@{turn}
     \ifx\second@\temp@
      \def\bottop@Tstride{#3}
      \def\label@Tstride{#4}
      \def\text@Tstride{#5}
     \fi\fi\fi\fi\fi\fi
   \else
   \xdef\temp@{DSSP}
   \ifx\temp@\first@
     \xdef\temp@{alpha}
     \ifx\second@\temp@
      \def\bottop@Hdssp{#3}
      \def\label@Hdssp{#4}
      \def\text@Hdssp{#5}
     \else
     \xdef\temp@{3-10}
     \ifx\second@\temp@
      \def\bottop@Gdssp{#3}
      \def\label@Gdssp{#4}
      \def\text@Gdssp{#5}
     \else
     \xdef\temp@{pi}
     \ifx\second@\temp@
      \def\bottop@Idssp{#3}
      \def\label@Idssp{#4}
      \def\text@Idssp{#5}
     \else
     \xdef\temp@{beta}
     \ifx\second@\temp@
      \def\bottop@Edssp{#3}
      \def\label@Edssp{#4}
      \def\text@Edssp{#5}
     \else
     \xdef\temp@{bridge}
     \ifx\second@\temp@
      \def\bottop@Bdssp{#3}
      \def\label@Bdssp{#4}
      \def\text@Bdssp{#5}
     \else
     \xdef\temp@{turn}
     \ifx\second@\temp@
      \def\bottop@Tdssp{#3}
      \def\label@Tdssp{#4}
      \def\text@Tdssp{#5}
     \else
     \xdef\temp@{bend}
     \ifx\second@\temp@
      \def\bottop@Sdssp{#3}
      \def\label@Sdssp{#4}
      \def\text@Sdssp{#5}
     \fi\fi\fi\fi\fi\fi\fi
   \else
   \xdef\temp@{HMMTOP}
   \ifx\temp@\first@
     \xdef\temp@{internal}
     \ifx\second@\temp@
      \def\bottop@i@HMMTOP{#3}
      \def\label@i@HMMTOP{#4}
      \def\text@i@HMMTOP{#5}
     \else
     \xdef\temp@{external}
     \ifx\second@\temp@
      \def\bottop@e@HMMTOP{#3}
      \def\label@e@HMMTOP{#4}
      \def\text@e@HMMTOP{#5}
     \else
     \xdef\temp@{TM}
     \ifx\second@\temp@
      \def\bottop@TM@HMMTOP{#3}
      \def\label@TM@HMMTOP{#4}
      \def\text@TM@HMMTOP{#5}
     \fi\fi\fi
   \fi\fi\fi\fi\fi
}

\def\showonDSSP#1{%
   \xdef\first@{#1,&,@} \xdef\second@{yes} \show@DSSP}
\def\hideonDSSP#1{%
   \xdef\first@{#1,&,@} \xdef\second@{no} \show@DSSP}

\def\showonSTRIDE#1{%
   \xdef\first@{#1,&,@} \xdef\second@{yes} \show@STRIDE}
\def\hideonSTRIDE#1{%
   \xdef\first@{#1,&,@} \xdef\second@{no} \show@STRIDE}

\def\showonPHDtopo#1{%
   \xdef\first@{#1,&,@} \xdef\second@{yes} \show@PHDtopo}
\def\hideonPHDtopo#1{%
   \xdef\first@{#1,&,@} \xdef\second@{no} \show@PHDtopo}

\def\showonPHDsec#1{%
   \xdef\first@{#1,&,@} \xdef\second@{yes} \show@PHDsec}
\def\hideonPHDsec#1{%
   \xdef\first@{#1,&,@} \xdef\second@{no} \show@PHDsec}

\def\showonHMMTOP#1{%
   \xdef\first@{#1,&,@} \xdef\second@{yes} \show@HMMTOP}
\def\hideonHMMTOP#1{%
   \xdef\first@{#1,&,@} \xdef\second@{no} \show@HMMTOP}

\def\codon#1#2{%
   \xdef\first@{#1}
   \xdef\second@{#2,&,@}
   \expandafter\get@triplet\second@}
\def\geneticcode#1{%
   \xdef\first@{#1}
   \xdef\temp@{standard}
   \ifx\first@\temp@
      \c@d@ns
   \else
  \input{#1.cod}
   \fi}
\newcommand{\backtranslabel}[2][tiny]{%
   \def\trans@size{\csname #1\endcsname}
   \xdef\first@{#2}
   \xdef\temp@{horizontal}
   \ifx\temp@\first@ \xdef\tr@nsstyle{0}\fi
   \xdef\temp@{zigzag}
   \ifx\temp@\first@ \xdef\tr@nsstyle{1}\fi
   \xdef\temp@{alternating}
   \ifx\temp@\first@ \xdef\tr@nsstyle{2}\fi
   \xdef\temp@{oblique}
   \ifx\temp@\first@ \xdef\tr@nsstyle{3}\fi
   \xdef\temp@{vertical}
   \ifx\temp@\first@ \xdef\tr@nsstyle{4}\fi
}
\newcommand{\backtranstext}[2][tiny]{%
   \def\transtext@size{\csname #1\endcsname}
   \xdef\first@{#2}
   \xdef\temp@{horizontal}
   \ifx\temp@\first@ \xdef\tr@nstextstyle{0}\fi
   \xdef\temp@{zigzag}
   \ifx\temp@\first@ \xdef\tr@nstextstyle{1}\fi
   \xdef\temp@{alternating}
   \ifx\temp@\first@ \xdef\tr@nstextstyle{2}\fi
   \xdef\temp@{oblique}
   \ifx\temp@\first@ \xdef\tr@nstextstyle{3}\fi
   \xdef\temp@{vertical}
   \ifx\temp@\first@ \xdef\tr@nstextstyle{4}\fi
}

\newcommand\exportconsensus[2][export.txt]{%
   \ifx\exp@rt\n@
      \xdef\first@{#2 @} \expandafter\check@letter\first@
      \xdef\first@{#2}   \ifletter \get@name@number \fi
      \xdef\exp@rt@num{\first@}
      \xdef\exp@rt{y}
      \immediate\openout\exp@rtfile = #1
   \fi
}

\def\setdomain#1#2{%
   \xdef\first@{#1}
   \xdef\temp@{consensus}
   \ifx\first@\temp@ \xdef\domain@seq{0}
   \else
      \xdef\first@{#1 @} \expandafter\check@letter\first@
      \xdef\first@{#1}   \ifletter \get@name@number \fi
      \xdef\domain@seq{\csname @rd\first@\endcsname}%
   \fi
   \xdef\list@{#2,&}
   \xdef\temp@{#2,,,:,,,,@} \expandafter\test@PDB\temp@
   \loop
        \xdef\list@{\list@ @}
        \expandafter\get@domainregions\list@
   \ifx\list@\ampers@nd\else\repeat
   \xdef\dom@in{y}
}

\def\decimal@IOO#1#2#3{#1#2\temp@#3}
\def\decimal@IO#1#2{#1\temp@#2}
\def\decimal@I#1{0\temp@#1}

\def\percentsimilarity#1#2{%
   \xdef\first@{#2@}\expandafter\check@letter\first@%
   \xdef\first@{#2}\xdef\second@@{#2}%
   \ifletter\get@name@number@table\xdef\second@@{\first@}\fi%
   \xdef\first@{#1@}\expandafter\check@letter\first@%
   \xdef\first@{#1}\xdef\first@@{#1}%
   \ifletter\get@name@number@table\xdef\first@@{\first@}\fi%
   \ifnum\first@@<\second@@%
      \loopcount=\csname simcount\first@@ @\second@@\endcsname%
      \temp@count=\csname poscount\first@@ @\second@@\endcsname%
   \else%
   \ifnum\first@@=\second@@ \loopcount=1 \temp@count=1%
   \else%
      \loopcount=\csname simcount\second@@ @\first@@\endcsname%
      \temp@count=\csname poscount\second@@ @\first@@\endcsname%
   \fi%
   \fi%
   \multiply\loopcount by 1000%
   \divide\loopcount by \temp@count%
   \xdef\first@{\the\loopcount}%
   \ifgerm@n\xdef\temp@{,}\else\xdef\temp@{.}\fi%
   \ifnum\first@=1000 100\temp@0%
   \else%
   \ifnum\first@>99 \expandafter\decimal@IOO\first@%
   \else%
   \ifnum\first@>9 \expandafter\decimal@IO\first@%
   \else%
   \ifnum\first@>0 \expandafter\decimal@I\first@%
   \else 0\temp@0%
   \fi%
   \fi%
   \fi%
   \fi%  
}

\def\percentidentity#1#2{%
   \xdef\first@{#2@}\expandafter\check@letter\first@%
   \xdef\first@{#2}\xdef\second@@{#2}%
   \ifletter\get@name@number@table\xdef\second@@{\first@}\fi%
   \xdef\first@{#1@}\expandafter\check@letter\first@%
   \xdef\first@{#1}\xdef\first@@{#1}%
   \ifletter\get@name@number@table\xdef\first@@{\first@}\fi%
   \ifnum\first@@<\second@@%
      \loopcount=\csname identcount\first@@ @\second@@\endcsname%
      \temp@count=\csname poscount\first@@ @\second@@\endcsname%
   \else%
   \ifnum\first@@=\second@@ \loopcount=1 \temp@count=1%
   \else%
      \loopcount=\csname identcount\second@@ @\first@@\endcsname%
      \temp@count=\csname poscount\second@@ @\first@@\endcsname%
   \fi%
   \fi%
   \multiply\loopcount by 1000%
   \divide\loopcount by \temp@count%
   \xdef\first@{\the\loopcount}%
   \ifgerm@n\xdef\temp@{,}\else\xdef\temp@{.}\fi%
   \ifnum\first@=1000 100\temp@0%
   \else%
   \ifnum\first@>99 \expandafter\decimal@IOO\first@%
   \else%
   \ifnum\first@>9 \expandafter\decimal@IO\first@%
   \else%
   \ifnum\first@>0 \expandafter\decimal@I\first@%
   \else 0\temp@0%
   \fi%
   \fi%
   \fi%
   \fi%  
}

\def\similaritytable{%
   \def\name@loop{%
      \advance\outerloopcount by 1
      \ifnum\outerloopcount>\seq@num\relax
      \else
      \immediate\write\exp@rtfile{&\string\multicolumn{1}{c}{\string\kern1ex\string\begin{rotopo}{90}\csname newseqname\the\outerloopcount\endcsname\string\end{rotopo}}}
      \name@loop\fi
   }
   \def\inner@loop{% 
      \advance\innerloopcount by 1
      \ifnum\innerloopcount>\seq@num\relax
      \else
      \ifnum\innerloopcount=\outerloopcount\relax \immediate\write\exp@rtfile{& {---\hss}}\fi
      \ifnum\innerloopcount>\outerloopcount\relax \immediate\write\exp@rtfile{& \string\percentsimilarity{\the\outerloopcount}{\the\innerloopcount}}\fi
      \ifnum\innerloopcount<\outerloopcount\relax \immediate\write\exp@rtfile{& \string\percentidentity{\the\outerloopcount}{\the\innerloopcount}}\fi
      \inner@loop\fi
   }
   \def\outer@loop{%
      \advance\outerloopcount by 1
      \ifnum\outerloopcount>\seq@num\relax
      \else
      \immediate\write\exp@rtfile{\csname newseqname\the\outerloopcount\endcsname}
      \innerloopcount=0 \inner@loop \immediate\write\exp@rtfile{&\string\\}
      \outer@loop\fi
   }
   \xdef\temp@{l|}
   \outerloopcount=0
   \loop
      \advance\outerloopcount by 1
      \xdef\temp@{\temp@ r}
   \ifnum\outerloopcount<\seq@num\repeat
   \advance\outerloopcount by 1 \xdef\seq@num@plus{\the\outerloopcount}
   \advance\outerloopcount by 1 \xdef\seq@num@plus@plus{\the\outerloopcount}
   \ifnum\seq@num<3 \xdef\temp@@{$\approx$} \else
   \ifnum\seq@num<5 \xdef\temp@@{simil.}\ifgerm@n\def\temp@@{\string\"{A}hnl.}\fi\ifsp@nish\xdef\temp@@{simil.}\fi\else
   \xdef\temp@@{similarity} 
      \ifgerm@n\ifnum\seq@num=5\def\temp@@{\string\"{A}hnlichk.}\else\def\temp@@{\string\"{A}hnlichkeit}\fi\fi\ifsp@nish\xdef\temp@@{similitud}\fi\fi
   \fi
   \xdef\temp@@@{identity}\ifgerm@n\def\temp@@@{Identit\string\"{a}t}\fi\ifsp@nish\xdef\temp@@@{identidad}\fi
   
   \immediate\openout\exp@rtfile = simtable.tmp
   \immediate\write\exp@rtfile{\string\begin{tabular}{\temp@ |r}}
   \immediate\write\exp@rtfile{\string\multicolumn{\seq@num@plus@plus}{c}{}\string\\[\name@@width]}
   \immediate\write\exp@rtfile{\string\multicolumn{1}{c}{}}
   \outerloopcount=0 \name@loop \immediate\write\exp@rtfile{&\string\\ \string\cline{2-\seq@num@plus}}
   \immediate\write\exp@rtfile{&\string\multicolumn{\seq@num}{|c|}{}&\string\\[-2ex]}
   \outerloopcount=0 \outer@loop 
   \immediate\write\exp@rtfile{\string\cline{2-\seq@num@plus}}
   \immediate\write\exp@rtfile{\string\multicolumn{\seq@num@plus}{c}{}&}
   \immediate\write\exp@rtfile{\string\multicolumn{1}{r}{\string\kern1.5ex\string\begin{rotopo}{90}\kern1em\% \temp@@\string\end{rotopo}}\string\\[-2ex]}
   \immediate\write\exp@rtfile{\string\multicolumn{1}{c}{}&\string\multicolumn{\seq@num}{l}{\% \temp@@@}&\string\\}
   \immediate\write\exp@rtfile{\string\end{tabular}}
   \immediate\closeout\exp@rtfile

   \input{simtable.tmp}
}

\def\identitytable{\similaritytable}

%%%%%  Calculate consensus

\def\check@sim{%
 \xdef\first@{\csname res\the\loopcount\endcsname}
 \xdef\first@{\csname \prefix@ grp\first@\endcsname}
 \newrestrue
 \ifnum\first@<0 \newresfalse
 \else
 \innerloopcount=\loopcount
 \ifnum\loopcount=\cons@num \innerloopcount=\seq@count \fi
 \ifnum\innerloopcount<\seq@count
 \loop
    \advance\innerloopcount by 1
    \xdef\second@{\csname res\the\innerloopcount\endcsname}
    \expandafter\ifx\csname \prefix@ grp\second@\endcsname\first@
       \newresfalse \innerloopcount=\seq@count \fi
 \ifnum\innerloopcount<\seq@count \repeat
 \fi
 \fi

 \ifnewres
  \pos@sum=0
  \innerloopcount=0
  \loop
    \advance\innerloopcount by 1
    \xdef\second@{\csname res\the\innerloopcount\endcsname}
    \expandafter\ifx\csname \prefix@ grp\second@\endcsname\first@
       \advance\pos@sum by 1 \fi
  \ifnum\innerloopcount<\seq@count \repeat

  \multiply\pos@sum by \seq@percent
  \expandafter\xdef\csname pos\the\loopcount\endcsname{\the\pos@sum}
  \expandafter\ifnum\csname pos\the\loopcount\endcsname<\thresh@ld
  \else
    \expandafter\ifnum\csname pos\the\loopcount\endcsname>\m@x
      \xdef\m@x{\csname pos\the\loopcount\endcsname}
      \xdef\cons@seq{\the\loopcount} \xdef\match@case{\c@se}
      \xdef\simgroup@{\first@}
    \else
     \expandafter\ifnum\csname pos\the\loopcount\endcsname=\m@x
      \xdef\match@case{0}
     \fi
    \fi
  \fi
 \fi

 \ifnum\loopcount=\cons@num \loopcount=1 \fi
 \advance\loopcount by -1
 \ifnum\loopcount>0 \check@sim \fi}

\def\check@ident{%
 \xdef\first@{\csname res\the\loopcount\endcsname}
 \newrestrue \expandafter\check@char\first@
 \ifletter
   \innerloopcount=\loopcount
   \ifnum\loopcount=\cons@num \innerloopcount=\seq@count \fi
   \ifnum\innerloopcount<\seq@count
     \loop
       \advance\innerloopcount by 1
       \expandafter\ifx\csname res\the\innerloopcount\endcsname\first@
          \newresfalse \innerloopcount=\seq@count  \fi
     \ifnum\innerloopcount<\seq@count \repeat
   \fi
 \else
    \newresfalse
%    \expandafter\xdef\csname res\the\loopcount\endcsname{\d@t}  %%%%% or \equ@l for domains!!!
 \fi

 \ifnewres
  \pos@sum=0
  \innerloopcount=0
  \loop
    \advance\innerloopcount by 1
    \expandafter\ifx\csname res\the\innerloopcount\endcsname\first@
       \advance\pos@sum by 1 \fi
  \ifnum\innerloopcount<\seq@count \repeat

  \expandafter\xdef\csname pos\the\loopcount\endcsname{\the\pos@sum}
  \expandafter\ifnum\csname pos\the\loopcount\endcsname=\seq@count
    \xdef\cons@seq{\the\loopcount} \xdef\match@case{2} \loopcount=1
  \else
   \multiply\pos@sum by \seq@percent
   \expandafter\xdef\csname pos\the\loopcount\endcsname{\the\pos@sum}
   \expandafter\ifnum\csname pos\the\loopcount\endcsname<\thresh@ld
   \else
    \expandafter\ifnum\csname pos\the\loopcount\endcsname>\m@x
      \xdef\m@x{\csname pos\the\loopcount\endcsname}
      \expandafter\ifnum\csname pos\the\loopcount\endcsname<\all@thresh@ld
         \xdef\cons@seq{\the\loopcount} \xdef\match@case{1}
      \else
         \xdef\cons@seq{\the\loopcount} \xdef\match@case{2}
      \fi      
    \else
     \expandafter\ifnum\csname pos\the\loopcount\endcsname=\m@x
      \xdef\match@case{0}
     \fi
    \fi
   \fi
  \fi
 \fi

 \ifnum\loopcount=\cons@num \loopcount=1 \fi
 \advance\loopcount by -1
 \ifnum\loopcount>0 \check@ident \fi}

\def\get@simchar{%
 \xdef\first@{\csname res\the\loopcount\endcsname}
 \newrestrue \expandafter\check@char\first@
 \ifletter
   \innerloopcount=\loopcount
   \ifnum\loopcount=\cons@num \innerloopcount=\seq@count \fi
   \ifnum\innerloopcount<\seq@count
     \loop
       \advance\innerloopcount by 1
       \expandafter\ifx\csname res\the\innerloopcount\endcsname\first@
          \newresfalse \innerloopcount=\seq@count  \fi
     \ifnum\innerloopcount<\seq@count \repeat
   \fi
 \else
    \newresfalse
 \fi

 \ifnewres
  \pos@sum=0
  \innerloopcount=0
  \loop
    \advance\innerloopcount by 1
    \expandafter\ifx\csname res\the\innerloopcount\endcsname\first@
       \xdef\second@{\csname res\the\innerloopcount\endcsname}
       \expandafter\ifx\csname \prefix@ grp\second@\endcsname\simgroup@
          \advance\pos@sum by 1 \fi
    \fi
  \ifnum\innerloopcount<\seq@count \repeat

  \expandafter\xdef\csname pos\the\loopcount\endcsname{\the\pos@sum}
  \expandafter\ifnum\csname pos\the\loopcount\endcsname>\m@x
      \xdef\m@x{\csname pos\the\loopcount\endcsname}
      \xdef\cons@seq{\the\loopcount}
  \fi
 \fi

 \ifnum\loopcount=\cons@num \loopcount=1 \fi
 \advance\loopcount by -1
 \ifnum\loopcount>0 \get@simchar \fi}

\def\unc@nserved{%
 \ifsimmode
    \ifnum\cons@num>0 \loopcount=\cons@num \else \loopcount=\seq@count \fi
    \xdef\match@case{0} \xdef\m@x{1} \check@sim
    \ifnum\match@case=0
       \loopcount=0
       \loop
          \advance\loopcount by 1
          \xdef\first@{noshade}
          \expandafter\ifx\csname hide@seq\the\loopcount\endcsname\first@
            \expandafter\ifx\csname res\the\loopcount\endcsname\d@t
             \expandafter\xdef\csname seq\the\loopcount\endcsname{%
                \csname seq\the\loopcount\endcsname3\gap@char}
            \else
            \expandafter\ifx\csname res\the\loopcount\endcsname\questi@n
             \expandafter\xdef\csname seq\the\loopcount\endcsname{%
                \csname seq\the\loopcount\endcsname3\st@p@char}
            \else
             \expandafter\ifx\csname res\the\loopcount\endcsname\equ@l
              \expandafter\xdef\csname seq\the\loopcount\endcsname{%
                 \csname seq\the\loopcount\endcsname!\dom@char} \xdef\g@p{y}
             \else
             \expandafter\xdef\csname seq\the\loopcount\endcsname{%
                \csname seq\the\loopcount\endcsname%
                3\csname res\the\loopcount\endcsname}
             \fi
             \fi
             \fi
          \else
           \xdef\first@{\csname res\the\loopcount\endcsname}
           \ifx\first@\equ@l
             \def\first@{\dom@char} \def\third@{9}
             \expandafter\ifx\csname lower@seq\the\loopcount\endcsname\y@
                   \expandafter\xdef\csname lower@seq\the\loopcount\endcsname{n}
             \xdef\first@{;\first@}\fi
             \expandafter\ifx\csname tint@seq\the\loopcount\endcsname\y@
                   \expandafter\xdef\csname tint@seq\the\loopcount\endcsname{n}
             \xdef\first@{=\first@}\fi
             \expandafter\ifx\csname emph@seq\the\loopcount\endcsname\y@
                   \expandafter\xdef\csname emph@seq\the\loopcount\endcsname{n}
             \xdef\first@{,\first@}\fi
             \expandafter\xdef\csname seq\the\loopcount\endcsname{%
                              \csname seq\the\loopcount\endcsname\third@\first@} \xdef\g@p{y}
           \else
             \ifx\first@\d@t
                 \expandafter\ifnum\csname res@count\the\loopcount\endcsname%
                      <\csname seq@start\the\loopcount\endcsname
                    \ifsh@wg@ps
                       \def\first@{\gap@char}\def\third@{7}
                    \else
                       \def\first@{{}} \def\third@{8}
                    \fi
                 \else
                    \expandafter\ifnum\csname res@count\the\loopcount\endcsname%
                        <\csname seq@len\the\loopcount\endcsname
                       \def\third@{7}\def\first@{\gap@char}
                    \else
                    \ifsh@wg@ps
                       \def\first@{\gap@char}\def\third@{7}
                    \else
                       \def\first@{{}} \def\third@{8}
                    \fi
                    \fi
                 \fi
                \expandafter\ifx\csname lower@seq\the\loopcount\endcsname\y@
                   \expandafter\xdef\csname lower@seq\the\loopcount\endcsname{n}
                \xdef\first@{;\first@}\fi
                \expandafter\ifx\csname tint@seq\the\loopcount\endcsname\y@
                   \expandafter\xdef\csname tint@seq\the\loopcount\endcsname{n}
                \xdef\first@{=\first@}\fi
                \expandafter\ifx\csname emph@seq\the\loopcount\endcsname\y@
                   \expandafter\xdef\csname emph@seq\the\loopcount\endcsname{n}
                \xdef\first@{,\first@}\fi
                \expandafter\xdef\csname seq\the\loopcount\endcsname{%
                                 \csname seq\the\loopcount\endcsname\third@\first@}
             \else
             \ifx\first@\questi@n
                 \expandafter\ifnum\csname res@count\the\loopcount\endcsname%
                      <\csname seq@start\the\loopcount\endcsname
                    \def\first@{{}} \def\third@{8}
                 \else
                    \expandafter\ifnum\csname res@count\the\loopcount\endcsname%
                        <\csname seq@len\the\loopcount\endcsname
                       \def\third@{7}\def\first@{\st@p@char}
                    \else
                       \def\first@{{}} \def\third@{8}
                    \fi
                 \fi
                \expandafter\ifx\csname lower@seq\the\loopcount\endcsname\y@
                   \expandafter\xdef\csname lower@seq\the\loopcount\endcsname{n}
                \xdef\first@{;\first@}\fi
                \expandafter\ifx\csname tint@seq\the\loopcount\endcsname\y@
                   \expandafter\xdef\csname tint@seq\the\loopcount\endcsname{n}
                \xdef\first@{=\first@}\fi
                \expandafter\ifx\csname emph@seq\the\loopcount\endcsname\y@
                   \expandafter\xdef\csname emph@seq\the\loopcount\endcsname{n}
                \xdef\first@{,\first@}\fi
                \expandafter\xdef\csname seq\the\loopcount\endcsname{%
                                 \csname seq\the\loopcount\endcsname\third@\first@}
             \else
               \ifnum\divref@=\loopcount\else 
               \xdef\low@up{lower} \ifx\resn@m@tch\low@up \make@lower \else
               \xdef\low@up{upper} \ifx\resn@m@tch\low@up
               \else \xdef\first@{\resn@m@tch} \fi\fi\fi
               \expandafter\ifx\csname lower@seq\the\loopcount\endcsname\y@
                  \expandafter\xdef\csname lower@seq\the\loopcount\endcsname{n}
                  \xdef\first@{;\first@}\fi
               \expandafter\ifx\csname tint@seq\the\loopcount\endcsname\y@
                  \expandafter\xdef\csname tint@seq\the\loopcount\endcsname{n}
                  \xdef\first@{=\first@}\fi
               \expandafter\ifx\csname emph@seq\the\loopcount\endcsname\y@
                  \expandafter\xdef\csname emph@seq\the\loopcount\endcsname{n}
                  \xdef\first@{,\first@}\fi
               \expandafter\xdef\csname seq\the\loopcount\endcsname{%
                  \csname seq\the\loopcount\endcsname3\first@}
             \fi
             \fi
           \fi
          \fi
       \ifnum\loopcount<\seq@count \repeat
       \xdef\low@up{lower} \ifx\n@m@tch\low@up \xdef\first@{{ }} \else
       \xdef\low@up{upper} \ifx\n@m@tch\low@up \xdef\first@{{ }}
       \else \xdef\first@{\n@m@tch} \fi\fi
       \expandafter\ifx\csname lower@seq0\endcsname\y@
          \expandafter\xdef\csname lower@seq0\endcsname{n}
          \xdef\first@{;\first@}\fi
       \expandafter\ifx\csname tint@seq0\endcsname\y@
          \expandafter\xdef\csname tint@seq0\endcsname{n}
          \xdef\first@{=\first@}\fi
       \expandafter\ifx\csname emph@seq0\endcsname\y@
          \expandafter\xdef\csname emph@seq0\endcsname{n}
          \xdef\first@{,\first@}\fi
       \ifx\g@p\y@
          \xdef\consensus{\consensus 8{}} 
       \else
          \xdef\consensus{\consensus 4\first@} 
          \expandafter\ifx\csname res\cons@num\endcsname\d@t
          \else\xdef\constopo{\constopo 0}\fi
       \fi
    \else
     \ifnum\cons@num>0
       \xdef\tmp@{\csname res\cons@num\endcsname}
     \else
       \xdef\m@x{0} \loopcount=\seq@count \get@simchar
       \xdef\tmp@{\csname res\cons@seq\endcsname}
     \fi
     \xdef\second@{\csname \prefix@ grp\tmp@\endcsname}
     \loopcount=0
     \loop
        \advance\loopcount by 1
        \xdef\first@{noshade}
        \expandafter\ifx\csname hide@seq\the\loopcount\endcsname\first@
             \xdef\first@{\csname res\the\loopcount\endcsname}
             \ifx\first@\d@t\def\first@{\gap@char}\fi
             \ifx\first@\questi@n\def\first@{\st@p@char}\fi
             \ifx\first@\equ@l\def\first@{\dom@char} \xdef\g@p{y} \fi
             \expandafter\ifx\csname lower@seq\the\loopcount\endcsname\y@
                \expandafter\xdef\csname lower@seq\the\loopcount\endcsname{n}
                \xdef\first@{;\first@}\fi
             \expandafter\ifx\csname tint@seq\the\loopcount\endcsname\y@
                \expandafter\xdef\csname tint@seq\the\loopcount\endcsname{n}
                \xdef\first@{=\first@}\fi
             \expandafter\ifx\csname emph@seq\the\loopcount\endcsname\y@
                \expandafter\xdef\csname emph@seq\the\loopcount\endcsname{n}
                \xdef\first@{,\first@}\fi
             \expandafter\xdef\csname seq\the\loopcount\endcsname{%
                \csname seq\the\loopcount\endcsname3\first@} 
        \else
          \xdef\first@{\csname res\the\loopcount\endcsname}
          \xdef\last@{\csname res\the\loopcount\endcsname}
          \expandafter\ifnum\csname \prefix@ grp\last@\endcsname=\second@
              \xdef\third@{2}
              \ifnum\divref@=\loopcount\else 
              \xdef\low@up{lower} \ifx\ressimm@tch\low@up \make@lower \else
              \xdef\low@up{upper} \ifx\ressimm@tch\low@up
              \else \xdef\first@{\ressimm@tch} \fi\fi\fi
          \else
            \ifx\first@\equ@l
              \def\first@{\dom@char} \def\third@{9}  \xdef\g@p{y}
            \else
              \ifx\first@\d@t
                 \expandafter\ifnum\csname res@count\the\loopcount\endcsname%
                      <\csname seq@start\the\loopcount\endcsname
                    \ifsh@wg@ps
                       \def\first@{\gap@char}\def\third@{7}
                    \else
                       \def\first@{{}} \def\third@{8}
                    \fi
                 \else
                    \expandafter\ifnum\csname res@count\the\loopcount\endcsname%
                         <\csname seq@len\the\loopcount\endcsname
                       \def\third@{7}\def\first@{\gap@char}
                    \else
                    \ifsh@wg@ps
                       \def\first@{\gap@char}\def\third@{7}
                    \else
                       \def\first@{{}} \def\third@{8}
                    \fi
                    \fi
                 \fi
              \else
              \ifx\first@\questi@n
                 \expandafter\ifnum\csname res@count\the\loopcount\endcsname%
                      <\csname seq@start\the\loopcount\endcsname
                       \def\first@{{}} \def\third@{8}
                 \else
                    \expandafter\ifnum\csname res@count\the\loopcount\endcsname%
                         <\csname seq@len\the\loopcount\endcsname
                       \def\third@{7}\def\first@{\st@p@char}
                    \else
                       \def\first@{{}} \def\third@{8}
                    \fi
                 \fi
              \else
                 \xdef\third@{3}
                 \ifnum\divref@=\loopcount\else 
                 \xdef\low@up{lower} \ifx\resn@m@tch\low@up \make@lower \else
                 \xdef\low@up{upper} \ifx\resn@m@tch\low@up
                 \else \xdef\first@{\resn@m@tch} \fi\fi\fi
              \fi
              \fi
            \fi
          \fi
          \expandafter\ifx\csname lower@seq\the\loopcount\endcsname\y@
                \expandafter\xdef\csname lower@seq\the\loopcount\endcsname{n}
                \xdef\first@{;\first@}\fi
          \expandafter\ifx\csname tint@seq\the\loopcount\endcsname\y@
                \expandafter\xdef\csname tint@seq\the\loopcount\endcsname{n}
                \xdef\first@{=\first@}\fi
          \expandafter\ifx\csname emph@seq\the\loopcount\endcsname\y@
                \expandafter\xdef\csname emph@seq\the\loopcount\endcsname{n}
                \xdef\first@{,\first@}\fi
          \expandafter\xdef\csname seq\the\loopcount\endcsname{%
              \csname seq\the\loopcount\endcsname\third@\first@}
        \fi
     \ifnum\loopcount<\seq@count \repeat
     \xdef\first@{\tmp@}
     \xdef\low@up{lower} \ifx\m@tch\low@up \make@lower \else
     \xdef\low@up{upper} \ifx\m@tch\low@up
     \else \xdef\first@{\m@tch} \fi\fi
     \expandafter\ifx\csname lower@seq0\endcsname\y@
                \expandafter\xdef\csname lower@seq0\endcsname{n}
                \xdef\first@{;\first@}\fi
     \expandafter\ifx\csname tint@seq0\endcsname\y@
                \expandafter\xdef\csname tint@seq0\endcsname{n}
                \xdef\first@{=\first@}\fi
     \expandafter\ifx\csname emph@seq0\endcsname\y@
                \expandafter\xdef\csname emph@seq0\endcsname{n}
                \xdef\first@{,\first@}\fi
     \ifx\g@p\y@
        \xdef\consensus{\consensus 8{}}
     \else
        \xdef\consensus{\consensus5\first@}
        \xdef\constopo{\constopo 1}
     \fi
    \fi
 \else
 \iffuncmode
    \loopcount=0
    \loop
       \advance\loopcount by 1
       \xdef\first@{\csname res\the\loopcount\endcsname}
       \ifx\first@\equ@l
          \def\first@{\dom@char} \def\third@{!}  \xdef\g@p{y}
          \expandafter\ifx\csname lower@seq\the\loopcount\endcsname\y@
                   \expandafter\xdef\csname lower@seq\the\loopcount\endcsname{n}
                   \xdef\first@{;\first@}\fi
          \expandafter\ifx\csname tint@seq\the\loopcount\endcsname\y@
                   \expandafter\xdef\csname tint@seq\the\loopcount\endcsname{n}
                   \xdef\first@{=\first@}\fi
          \expandafter\ifx\csname emph@seq\the\loopcount\endcsname\y@
                   \expandafter\xdef\csname emph@seq\the\loopcount\endcsname{n}
                   \xdef\first@{,\first@}\fi
          \expandafter\xdef\csname seq\the\loopcount\endcsname{%
             \csname seq\the\loopcount\endcsname\third@\first@}
       \else
       \ifx\first@\d@t
            \expandafter\ifnum\csname res@count\the\loopcount\endcsname%
                 <\csname seq@start\the\loopcount\endcsname
                    \ifsh@wg@ps
                       \def\first@{\gap@char}\def\third@{*}
                    \else
                       \def\first@{{}} \def\third@{/}
                    \fi
            \else
               \expandafter\ifnum\csname res@count\the\loopcount\endcsname%
                    <\csname seq@len\the\loopcount\endcsname
                  \def\first@{\gap@char}\def\third@{*}
               \else
                    \ifsh@wg@ps
                       \def\first@{\gap@char}\def\third@{*}
                    \else
                       \def\first@{{}} \def\third@{/}
                    \fi
               \fi
             \fi
          \expandafter\ifx\csname lower@seq\the\loopcount\endcsname\y@
                   \expandafter\xdef\csname lower@seq\the\loopcount\endcsname{n}
                   \xdef\first@{;\first@}\fi
          \expandafter\ifx\csname tint@seq\the\loopcount\endcsname\y@
                   \expandafter\xdef\csname tint@seq\the\loopcount\endcsname{n}
                   \xdef\first@{=\first@}\fi
          \expandafter\ifx\csname emph@seq\the\loopcount\endcsname\y@
                   \expandafter\xdef\csname emph@seq\the\loopcount\endcsname{n}
                   \xdef\first@{,\first@}\fi
          \expandafter\xdef\csname seq\the\loopcount\endcsname{%
             \csname seq\the\loopcount\endcsname\third@\first@}
       \else
       \ifx\first@\questi@n
            \expandafter\ifnum\csname res@count\the\loopcount\endcsname%
                 <\csname seq@start\the\loopcount\endcsname
                  \def\first@{{}} \def\third@{/}
            \else
               \expandafter\ifnum\csname res@count\the\loopcount\endcsname%
                    <\csname seq@len\the\loopcount\endcsname
                  \def\first@{\st@p@char}\def\third@{*}
               \else
                  \def\first@{{}} \def\third@{/}
               \fi
             \fi
          \expandafter\ifx\csname lower@seq\the\loopcount\endcsname\y@
                   \expandafter\xdef\csname lower@seq\the\loopcount\endcsname{n}
                   \xdef\first@{;\st@p@char}\fi
          \expandafter\ifx\csname tint@seq\the\loopcount\endcsname\y@
                   \expandafter\xdef\csname tint@seq\the\loopcount\endcsname{n}
                   \xdef\first@{=\st@p@char}\fi
          \expandafter\ifx\csname emph@seq\the\loopcount\endcsname\y@
                   \expandafter\xdef\csname emph@seq\the\loopcount\endcsname{n}
                   \xdef\first@{,\st@p@char}\fi
          \expandafter\xdef\csname seq\the\loopcount\endcsname{%
             \csname seq\the\loopcount\endcsname\third@\first@}
       \else
          \xdef\low@up{lower} \ifx\resn@m@tch\low@up \make@lower \else
          \xdef\low@up{upper} \ifx\resn@m@tch\low@up
          \else \xdef\first@{\resn@m@tch} \fi\fi
          \expandafter\ifx\csname lower@seq\the\loopcount\endcsname\y@
                   \expandafter\xdef\csname lower@seq\the\loopcount\endcsname{n}
                   \xdef\first@{;\first@}\fi
          \expandafter\ifx\csname tint@seq\the\loopcount\endcsname\y@
                   \expandafter\xdef\csname tint@seq\the\loopcount\endcsname{n}
                   \xdef\first@{=\first@}\fi
          \expandafter\ifx\csname emph@seq\the\loopcount\endcsname\y@
                   \expandafter\xdef\csname emph@seq\the\loopcount\endcsname{n}
                   \xdef\first@{,\first@}\fi
          \expandafter\xdef\csname seq\the\loopcount\endcsname{%
             \csname seq\the\loopcount\endcsname0\first@}
       \fi
       \fi
       \fi
    \ifnum\loopcount<\seq@count \repeat
 \else
    \loopcount=0
    \loop
        \advance\loopcount by 1
        \xdef\first@{noshade}
        \expandafter\ifx\csname hide@seq\the\loopcount\endcsname\first@
             \xdef\first@{\csname res\the\loopcount\endcsname}
             \ifx\first@\d@t\def\first@{\gap@char}\fi
             \ifx\first@\questi@n\def\first@{\st@p@char}\fi
             \ifx\first@\equ@l\def\first@{\dom@char}  \xdef\g@p{y} \fi
             \expandafter\ifx\csname lower@seq\the\loopcount\endcsname\y@
                \expandafter\xdef\csname lower@seq\the\loopcount\endcsname{n}
                \xdef\first@{;\first@}\fi
             \expandafter\ifx\csname tint@seq\the\loopcount\endcsname\y@
                \expandafter\xdef\csname tint@seq\the\loopcount\endcsname{n}
                \xdef\first@{=\first@}\fi
             \expandafter\ifx\csname emph@seq\the\loopcount\endcsname\y@
                \expandafter\xdef\csname emph@seq\the\loopcount\endcsname{n}
                \xdef\first@{,\first@}\fi
             \expandafter\xdef\csname seq\the\loopcount\endcsname{%
                \csname seq\the\loopcount\endcsname%
                3\first@}
        \else
         \xdef\first@{\csname res\the\loopcount\endcsname}
         \ifx\first@\equ@l
            \def\first@{\dom@char} \def\third@{9}  \xdef\g@p{y}
           \expandafter\ifx\csname lower@seq\the\loopcount\endcsname\y@
                   \expandafter\xdef\csname lower@seq\the\loopcount\endcsname{n}
                   \xdef\first@{;\first@}\fi
           \expandafter\ifx\csname tint@seq\the\loopcount\endcsname\y@
                   \expandafter\xdef\csname tint@seq\the\loopcount\endcsname{n}
                   \xdef\first@{=\first@}\fi
           \expandafter\ifx\csname emph@seq\the\loopcount\endcsname\y@
                   \expandafter\xdef\csname emph@seq\the\loopcount\endcsname{n}
                   \xdef\first@{,\first@}\fi
           \expandafter\xdef\csname seq\the\loopcount\endcsname{%
              \csname seq\the\loopcount\endcsname\third@\first@}
         \else
         \ifx\first@\d@t
            \expandafter\ifnum\csname res@count\the\loopcount\endcsname%
                 <\csname seq@start\the\loopcount\endcsname
                    \ifsh@wg@ps
                       \def\first@{\gap@char}\def\third@{7}
                    \else
                       \def\first@{{}} \def\third@{8}
                    \fi
            \else
               \expandafter\ifnum\csname res@count\the\loopcount\endcsname%
                    <\csname seq@len\the\loopcount\endcsname
                       \def\third@{7}\def\first@{\gap@char}
               \else
                    \ifsh@wg@ps
                       \def\first@{\gap@char}\def\third@{7}
                    \else
                       \def\first@{{}} \def\third@{8}
                    \fi
               \fi
            \fi
           \expandafter\ifx\csname lower@seq\the\loopcount\endcsname\y@
                   \expandafter\xdef\csname lower@seq\the\loopcount\endcsname{n}
                   \xdef\first@{;\first@}\fi
           \expandafter\ifx\csname tint@seq\the\loopcount\endcsname\y@
                   \expandafter\xdef\csname tint@seq\the\loopcount\endcsname{n}
                   \xdef\first@{=\first@}\fi
           \expandafter\ifx\csname emph@seq\the\loopcount\endcsname\y@
                   \expandafter\xdef\csname emph@seq\the\loopcount\endcsname{n}
                   \xdef\first@{,\first@}\fi
           \expandafter\xdef\csname seq\the\loopcount\endcsname{%
              \csname seq\the\loopcount\endcsname\third@\first@}
         \else
         \ifx\first@\questi@n
            \expandafter\ifnum\csname res@count\the\loopcount\endcsname%
                 <\csname seq@start\the\loopcount\endcsname
                       \def\first@{{}} \def\third@{8}
            \else
               \expandafter\ifnum\csname res@count\the\loopcount\endcsname%
                    <\csname seq@len\the\loopcount\endcsname
                       \def\third@{7}\def\first@{\st@p@char}
               \else
                       \def\first@{{}} \def\third@{8}
               \fi
            \fi
           \expandafter\ifx\csname lower@seq\the\loopcount\endcsname\y@
                   \expandafter\xdef\csname lower@seq\the\loopcount\endcsname{n}
                   \xdef\first@{;\first@}\fi
           \expandafter\ifx\csname tint@seq\the\loopcount\endcsname\y@
                   \expandafter\xdef\csname tint@seq\the\loopcount\endcsname{n}
                   \xdef\first@{=\first@}\fi
           \expandafter\ifx\csname emph@seq\the\loopcount\endcsname\y@
                   \expandafter\xdef\csname emph@seq\the\loopcount\endcsname{n}
                   \xdef\first@{,\first@}\fi
           \expandafter\xdef\csname seq\the\loopcount\endcsname{%
              \csname seq\the\loopcount\endcsname\third@\first@}
         \else
           \ifnum\divref@=\loopcount\else 
           \xdef\low@up{lower} \ifx\resn@m@tch\low@up \make@lower \else
           \xdef\low@up{upper} \ifx\resn@m@tch\low@up
           \else \xdef\first@{\resn@m@tch} \fi\fi\fi
           \expandafter\ifx\csname lower@seq\the\loopcount\endcsname\y@
                   \expandafter\xdef\csname lower@seq\the\loopcount\endcsname{n}
                   \xdef\first@{;\first@}\fi
           \expandafter\ifx\csname tint@seq\the\loopcount\endcsname\y@
                   \expandafter\xdef\csname tint@seq\the\loopcount\endcsname{n}
                   \xdef\first@{=\first@}\fi
           \expandafter\ifx\csname emph@seq\the\loopcount\endcsname\y@
                   \expandafter\xdef\csname emph@seq\the\loopcount\endcsname{n}
                   \xdef\first@{,\first@}\fi
           \expandafter\xdef\csname seq\the\loopcount\endcsname{%
             \csname seq\the\loopcount\endcsname3\first@}
         \fi
         \fi
         \fi
        \fi
    \ifnum\loopcount<\seq@count \repeat
    \xdef\low@up{lower} \ifx\n@m@tch\low@up \xdef\first@{{ }} \else
    \xdef\low@up{upper} \ifx\n@m@tch\low@up \xdef\first@{{ }}
    \else \xdef\first@{\n@m@tch} \fi\fi
    \expandafter\ifx\csname lower@seq0\endcsname\y@
                \expandafter\xdef\csname lower@seq0\endcsname{n}
                \xdef\first@{;\first@}\fi
    \expandafter\ifx\csname tint@seq0\endcsname\y@
                \expandafter\xdef\csname tint@seq0\endcsname{n}
                \xdef\first@{=\first@}\fi
    \expandafter\ifx\csname emph@seq0\endcsname\y@
                \expandafter\xdef\csname emph@seq0\endcsname{n}
                \xdef\first@{,\first@}\fi
    \ifx\g@p\y@
       \xdef\consensus{\consensus 8{}}
    \else
       \xdef\consensus{\consensus 4\first@}
       \xdef\constopo{\constopo 0}
    \fi
 \fi\fi}

\def\c@nserved{%
 \xdef\tmp@{\csname res\cons@seq\endcsname}
 \loopcount=0
 \loop
    \advance\loopcount by 1
    \xdef\first@{noshade}
    \expandafter\ifx\csname hide@seq\the\loopcount\endcsname\first@
             \xdef\first@{\csname res\the\loopcount\endcsname}
             \ifx\first@\d@t\def\first@{\gap@char}\fi
             \ifx\first@\questi@n\def\first@{\st@p@char}\fi
             \ifx\first@\equ@l\def\first@{\dom@char} \xdef\g@p{y} \fi
             \expandafter\ifx\csname lower@seq\the\loopcount\endcsname\y@
                \expandafter\xdef\csname lower@seq\the\loopcount\endcsname{n}
                \xdef\first@{;\first@}\fi
             \expandafter\ifx\csname tint@seq\the\loopcount\endcsname\y@
                \expandafter\xdef\csname tint@seq\the\loopcount\endcsname{n}
                \xdef\first@{=\first@}\fi
             \expandafter\ifx\csname emph@seq\the\loopcount\endcsname\y@
                \expandafter\xdef\csname emph@seq\the\loopcount\endcsname{n}
                \xdef\first@{,\first@}\fi
             \expandafter\xdef\csname seq\the\loopcount\endcsname{%
                \csname seq\the\loopcount\endcsname%
                3\first@}
    \else
    \xdef\second@{\csname res\the\loopcount\endcsname}
    \ifx\tmp@\second@
       \xdef\third@{1}
    \else
       \xdef\third@{3}
       \ifsimmode
          \xdef\last@{\csname \prefix@ sim\tmp@\endcsname &@}
          \expandafter\get@count\last@
          \innerloopcount=0 \getsim@char
       \fi
    \fi
    \xdef\first@{\csname res\the\loopcount\endcsname}
    \ifcase\third@ \or
    \ifnum\divref@=\loopcount\else 
    \xdef\low@up{lower} \ifx\resm@tch\low@up \make@lower \else
    \xdef\low@up{upper} \ifx\resm@tch\low@up
    \else \xdef\first@{\resm@tch} \fi\fi\fi
    \or
    \ifnum\divref@=\loopcount\else 
    \xdef\low@up{lower} \ifx\ressimm@tch\low@up \make@lower \else
    \xdef\low@up{upper} \ifx\ressimm@tch\low@up
    \else \xdef\first@{\ressimm@tch} \fi\fi\fi
    \else
    \ifnum\divref@=\loopcount\else 
    \xdef\low@up{lower} \ifx\resn@m@tch\low@up \make@lower \else
    \xdef\low@up{upper} \ifx\resn@m@tch\low@up
    \else \xdef\first@{\resn@m@tch} \fi\fi\fi
    \fi
    \ifx\first@\equ@l \def\first@{\dom@char} \def\third@{9}  \xdef\g@p{y} \fi
    \ifx\first@\d@t
            \expandafter\ifnum\csname res@count\the\loopcount\endcsname%
                 <\csname seq@start\the\loopcount\endcsname
		    \ifsh@wg@ps
		       \def\first@{\gap@char}\def\third@{7}
		    \else
                       \def\first@{{}} \def\third@{8}
		    \fi
            \else
               \expandafter\ifnum\csname res@count\the\loopcount\endcsname%
                    <\csname seq@len\the\loopcount\endcsname
		       \def\third@{7}\def\first@{\gap@char}
               \else
                    \ifsh@wg@ps
                       \def\first@{\gap@char}\def\third@{7}
                    \else
                       \def\first@{{}} \def\third@{8}
                    \fi
               \fi
            \fi
    \fi
    \ifx\first@\questi@n
            \expandafter\ifnum\csname res@count\the\loopcount\endcsname%
                 <\csname seq@start\the\loopcount\endcsname
                      \def\first@{{}} \def\third@{8}
            \else
               \expandafter\ifnum\csname res@count\the\loopcount\endcsname%
                    <\csname seq@len\the\loopcount\endcsname
		               \def\third@{7}\def\first@{\st@p@char}
               \else
                       \def\first@{{}} \def\third@{8}
               \fi
            \fi
    \fi
    \expandafter\ifx\csname lower@seq\the\loopcount\endcsname\y@
                \expandafter\xdef\csname lower@seq\the\loopcount\endcsname{n}
                \xdef\first@{;\first@}\fi
    \expandafter\ifx\csname tint@seq\the\loopcount\endcsname\y@
                \expandafter\xdef\csname tint@seq\the\loopcount\endcsname{n}
                \xdef\first@{=\first@}\fi
    \expandafter\ifx\csname emph@seq\the\loopcount\endcsname\y@
                \expandafter\xdef\csname emph@seq\the\loopcount\endcsname{n}
                \xdef\first@{,\first@}\fi
    \expandafter\xdef\csname seq\the\loopcount\endcsname{%
       \csname seq\the\loopcount\endcsname\third@\first@}
    \fi
 \ifnum\loopcount<\seq@count \repeat
 \xdef\first@{\csname res\cons@seq\endcsname}
 \xdef\second@{lower} \ifx\m@tch\second@ \make@lower \else
 \xdef\second@{upper} \ifx\m@tch\second@
 \else \xdef\first@{\m@tch} \fi\fi
 \expandafter\ifx\csname lower@seq0\endcsname\y@
                \expandafter\xdef\csname lower@seq0\endcsname{n}
                \xdef\first@{;\first@}\fi
 \expandafter\ifx\csname tint@seq0\endcsname\y@
                \expandafter\xdef\csname tint@seq0\endcsname{n}
                \xdef\first@{=\first@}\fi
 \expandafter\ifx\csname emph@seq0\endcsname\y@
                \expandafter\xdef\csname emph@seq0\endcsname{n}
                \xdef\first@{,\first@}\fi
 \ifx\g@p\y@
    \xdef\consensus{\consensus 8{}}
 \else
    \xdef\consensus{\consensus 5\first@}
    \xdef\constopo{\constopo 2}
 \fi}
 
\def\allm@tch{%
 \xdef\tmp@{\csname res\cons@seq\endcsname}
 \loopcount=0
 \loop
    \advance\loopcount by 1
    \xdef\first@{noshade}
    \expandafter\ifx\csname hide@seq\the\loopcount\endcsname\first@
             \xdef\first@{\csname res\the\loopcount\endcsname}
             \ifx\first@\d@t\def\first@{\gap@char}\fi
             \ifx\first@\questi@n\def\first@{\st@p@char}\fi
             \ifx\first@\equ@l\def\first@{\dom@char} \xdef\g@p{y} \fi
             \expandafter\ifx\csname lower@seq\the\loopcount\endcsname\y@
                \expandafter\xdef\csname lower@seq\the\loopcount\endcsname{n}
                \xdef\first@{;\first@}\fi
             \expandafter\ifx\csname tint@seq\the\loopcount\endcsname\y@
                \expandafter\xdef\csname tint@seq\the\loopcount\endcsname{n}
                \xdef\first@{=\first@}\fi
             \expandafter\ifx\csname emph@seq\the\loopcount\endcsname\y@
                \expandafter\xdef\csname emph@seq\the\loopcount\endcsname{n}
                \xdef\first@{,\first@}\fi
             \expandafter\xdef\csname seq\the\loopcount\endcsname{%
                \csname seq\the\loopcount\endcsname%
                3\first@}
    \else
    \xdef\second@{\csname res\the\loopcount\endcsname}
    \ifx\tmp@\second@
       \ifall@shade \xdef\third@{0} \else \xdef\third@{1} \fi
    \else
       \xdef\third@{3}
       \ifsimmode
          \xdef\last@{\csname \prefix@ sim\tmp@\endcsname &@}
          \expandafter\get@count\last@
          \innerloopcount=0 \getsim@char
       \fi
    \fi
    \xdef\first@{\csname res\the\loopcount\endcsname}
    \ifcase\third@ 
    \ifnum\divref@=\loopcount\else 
    \xdef\low@up{lower} \ifx\res@llm@tch\low@up \make@lower \else
    \xdef\low@up{upper} \ifx\res@llm@tch\low@up
    \else \xdef\first@{\res@llm@tch} \fi\fi\fi
    \or
    \ifnum\divref@=\loopcount\else 
    \xdef\low@up{lower} \ifx\res@llm@tch\low@up \make@lower \else
    \xdef\low@up{upper} \ifx\res@llm@tch\low@up
    \else \xdef\first@{\res@llm@tch} \fi\fi\fi
    \or
    \ifnum\divref@=\loopcount\else 
    \xdef\low@up{lower} \ifx\ressimm@tch\low@up \make@lower \else
    \xdef\low@up{upper} \ifx\ressimm@tch\low@up
    \else \xdef\first@{\ressimm@tch} \fi\fi\fi
    \else
    \ifnum\divref@=\loopcount\else 
    \xdef\low@up{lower} \ifx\resn@m@tch\low@up \make@lower \else
    \xdef\low@up{upper} \ifx\resn@m@tch\low@up
    \else \xdef\first@{\resn@m@tch} \fi\fi\fi
    \fi
    \ifx\first@\equ@l \def\first@{\dom@char} \def\third@{9}  \xdef\g@p{y} \fi
    \ifx\first@\d@t
            \expandafter\ifnum\csname res@count\the\loopcount\endcsname%
                 <\csname seq@start\the\loopcount\endcsname
		    \ifsh@wg@ps
		       \def\first@{\gap@char}\def\third@{7}
		    \else
                       \def\first@{{}} \def\third@{8}
		    \fi
            \else
               \expandafter\ifnum\csname res@count\the\loopcount\endcsname%
                    <\csname seq@len\the\loopcount\endcsname
		       \def\third@{7}\def\first@{\gap@char}
               \else
                    \ifsh@wg@ps
                       \def\first@{\gap@char}\def\third@{7}
                    \else
                       \def\first@{{}} \def\third@{8}
                    \fi
               \fi
            \fi
    \fi
    \ifx\first@\questi@n
            \expandafter\ifnum\csname res@count\the\loopcount\endcsname%
                 <\csname seq@start\the\loopcount\endcsname
                       \def\first@{{}} \def\third@{8}
            \else
               \expandafter\ifnum\csname res@count\the\loopcount\endcsname%
                    <\csname seq@len\the\loopcount\endcsname
		                \def\third@{7}\def\first@{\st@p@char}
               \else
                       \def\first@{{}} \def\third@{8}
               \fi
            \fi
    \fi
    \expandafter\ifx\csname lower@seq\the\loopcount\endcsname\y@
                \expandafter\xdef\csname lower@seq\the\loopcount\endcsname{n}
                \xdef\first@{;\first@}\fi
    \expandafter\ifx\csname tint@seq\the\loopcount\endcsname\y@
                \expandafter\xdef\csname tint@seq\the\loopcount\endcsname{n}
                \xdef\first@{=\first@}\fi
    \expandafter\ifx\csname emph@seq\the\loopcount\endcsname\y@
                \expandafter\xdef\csname emph@seq\the\loopcount\endcsname{n}
                \xdef\first@{,\first@}\fi
    \expandafter\xdef\csname seq\the\loopcount\endcsname{%
       \csname seq\the\loopcount\endcsname\third@\first@}
    \fi
 \ifnum\loopcount<\seq@count \repeat
 \xdef\first@{\csname res\cons@seq\endcsname}
 \xdef\second@{lower} \ifx\@llm@tch\second@ \make@lower \else
 \xdef\second@{upper} \ifx\@llm@tch\second@
 \else \xdef\first@{\@llm@tch} \fi\fi
 \expandafter\ifx\csname lower@seq0\endcsname\y@
                \expandafter\xdef\csname lower@seq0\endcsname{n}
                \xdef\first@{;\first@}\fi
 \expandafter\ifx\csname tint@seq0\endcsname\y@
                \expandafter\xdef\csname tint@seq0\endcsname{n}
                \xdef\first@{=\first@}\fi
 \expandafter\ifx\csname emph@seq0\endcsname\y@
                \expandafter\xdef\csname emph@seq0\endcsname{n}
                \xdef\first@{,\first@}\fi
 \ifx\g@p\y@
    \xdef\consensus{\consensus 8{}}
 \else
    \xdef\consensus{\consensus 6\first@}
    \xdef\constopo{\constopo 3}
 \fi}

\def\functi@nal{%
     \ifnum\cons@num>0
       \xdef\first@{\csname res\cons@num\endcsname}
     \else
       \xdef\first@{\csname res\cons@seq\endcsname}
     \fi
     \xdef\second@{\csname funcgrp\first@\endcsname}
     \loopcount=0
     \loop
        \advance\loopcount by 1
        \xdef\third@{noshade}
        \expandafter\ifx\csname hide@seq\the\loopcount\endcsname\third@
             \xdef\first@{\csname res\the\loopcount\endcsname}
             \expandafter\ifx\csname lower@seq\the\loopcount\endcsname\y@
                \expandafter\xdef\csname lower@seq\the\loopcount\endcsname{n}
                \xdef\first@{;\first@}\fi
             \expandafter\ifx\csname tint@seq\the\loopcount\endcsname\y@
                \expandafter\xdef\csname tint@seq\the\loopcount\endcsname{n}
                \xdef\first@{=\first@}\fi
             \expandafter\ifx\csname emph@seq\the\loopcount\endcsname\y@
                \expandafter\xdef\csname emph@seq\the\loopcount\endcsname{n}
                \xdef\first@{,\first@}\fi
             \expandafter\xdef\csname seq\the\loopcount\endcsname{%
                \csname seq\the\loopcount\endcsname%
                0\first@}
        \else
        \xdef\first@{\csname res\the\loopcount\endcsname}
         \ifx\first@\equ@l
            \def\first@{\dom@char} \def\third@{!} \xdef\g@p{y}
         \else
         \ifx\first@\d@t
            \expandafter\ifnum\csname res@count\the\loopcount\endcsname%
                 <\csname seq@start\the\loopcount\endcsname
                    \ifsh@wg@ps
                       \def\first@{\gap@char}\def\third@{*}
                    \else
                       \def\first@{{}} \def\third@{/}
                    \fi
            \else
               \expandafter\ifnum\csname res@count\the\loopcount\endcsname%
                    <\csname seq@len\the\loopcount\endcsname
                  \def\first@{\gap@char}\def\third@{*}
               \else
                    \ifsh@wg@ps
                       \def\first@{\gap@char}\def\third@{*}
                    \else
                       \def\first@{{}} \def\third@{/}
                    \fi
               \fi
            \fi
         \else
         \ifx\first@\questi@n
            \expandafter\ifnum\csname res@count\the\loopcount\endcsname%
                 <\csname seq@start\the\loopcount\endcsname
                    \def\first@{{}} \def\third@{/}
            \else
               \expandafter\ifnum\csname res@count\the\loopcount\endcsname%
                    <\csname seq@len\the\loopcount\endcsname
                    \def\first@{\st@p@char}\def\third@{*}
               \else
                    \def\first@{{}} \def\third@{/}
               \fi
            \fi
         \else
           \expandafter\ifnum\csname funcgrp\first@\endcsname=\second@
              \xdef\low@up{lower}
           \expandafter\ifx\csname funcm@tch\second@\endcsname\low@up
                       \make@lower \fi
              \xdef\third@{\second@}
           \else \xdef\third@{0}
              \xdef\low@up{lower} \ifx\resn@m@tch\low@up \make@lower \else
              \xdef\low@up{upper} \ifx\resn@m@tch\low@up
              \else \xdef\first@{\resn@m@tch} \fi\fi
         \fi
         \fi
         \fi
        \fi
        \expandafter\ifx\csname lower@seq\the\loopcount\endcsname\y@
                \expandafter\xdef\csname lower@seq\the\loopcount\endcsname{n}
                \xdef\first@{;\first@}\fi
        \expandafter\ifx\csname tint@seq\the\loopcount\endcsname\y@
                \expandafter\xdef\csname tint@seq\the\loopcount\endcsname{n}
                \xdef\first@{=\first@}\fi
        \expandafter\ifx\csname emph@seq\the\loopcount\endcsname\y@
                \expandafter\xdef\csname emph@seq\the\loopcount\endcsname{n}
                \xdef\first@{,\first@}\fi
        \expandafter\xdef\csname seq\the\loopcount\endcsname{%
          \csname seq\the\loopcount\endcsname\third@\first@}
        \fi
     \ifnum\loopcount<\seq@count \repeat}

\def\all@funcshade{%
   \loopcount=0
   \loop
     \advance\loopcount by 1
     \xdef\first@{\csname res\the\loopcount\endcsname}
     \xdef\second@{\csname funcgrp\first@\endcsname}
     \ifnum\second@<0 \xdef\second@{0} \fi
     \ifx\first@\equ@l \def\first@{\dom@char} \def\second@{!}  \xdef\g@p{y} \fi
     \ifx\first@\d@t
            \expandafter\ifnum\csname res@count\the\loopcount\endcsname%
                 <\csname seq@start\the\loopcount\endcsname
                    \ifsh@wg@ps
                       \def\first@{\gap@char}\def\second@{*}
                    \else
                       \def\first@{{}} \def\second@{/}
                    \fi
            \else
               \expandafter\ifnum\csname res@count\the\loopcount\endcsname%
                    <\csname seq@len\the\loopcount\endcsname
                  \def\first@{\gap@char}\def\second@{*}
               \else
                    \ifsh@wg@ps
                       \def\first@{\gap@char}\def\second@{*}
                    \else
                       \def\first@{{}} \def\second@{/}
                    \fi
               \fi
            \fi
     \fi
     \ifx\first@\questi@n
            \expandafter\ifnum\csname res@count\the\loopcount\endcsname%
                 <\csname seq@start\the\loopcount\endcsname
                    \def\first@{{}} \def\second@{/}
            \else
               \expandafter\ifnum\csname res@count\the\loopcount\endcsname%
                    <\csname seq@len\the\loopcount\endcsname
                    \def\first@{\st@p@char}\def\second@{*}
               \else
                    \def\first@{{}} \def\second@{/}
               \fi
            \fi
     \fi
     \xdef\low@up{lower}
     \expandafter\ifx\csname funcm@tch\second@\endcsname\low@up
                 \make@lower \fi
     \xdef\third@{noshade}
     \expandafter\ifx\csname hide@seq\the\loopcount\endcsname\third@
             \expandafter\ifx\csname lower@seq\the\loopcount\endcsname\y@
                \expandafter\xdef\csname lower@seq\the\loopcount\endcsname{n}
                \xdef\first@{;\first@}\fi
             \expandafter\ifx\csname tint@seq\the\loopcount\endcsname\y@
                \expandafter\xdef\csname tint@seq\the\loopcount\endcsname{n}
                \xdef\first@{=\first@}\fi
             \expandafter\ifx\csname emph@seq\the\loopcount\endcsname\y@
                \expandafter\xdef\csname emph@seq\the\loopcount\endcsname{n}
                \xdef\first@{,\first@}\fi
             \expandafter\xdef\csname seq\the\loopcount\endcsname{%
                \csname seq\the\loopcount\endcsname%
                0\first@}
     \else
             \expandafter\ifx\csname lower@seq\the\loopcount\endcsname\y@
                \expandafter\xdef\csname lower@seq\the\loopcount\endcsname{n}
                \xdef\first@{;\first@}\fi
             \expandafter\ifx\csname tint@seq\the\loopcount\endcsname\y@
                \expandafter\xdef\csname tint@seq\the\loopcount\endcsname{n}
                \xdef\first@{=\first@}\fi
             \expandafter\ifx\csname emph@seq\the\loopcount\endcsname\y@
                \expandafter\xdef\csname emph@seq\the\loopcount\endcsname{n}
                \xdef\first@{,\first@}\fi
             \expandafter\xdef\csname seq\the\loopcount\endcsname{%
                \csname seq\the\loopcount\endcsname%
                \second@\first@}
     \fi
   \ifnum\loopcount<\seq@count \repeat}

\def\T@coffee@shade{%
   \loopcount=0
   \loop
     \advance\loopcount by 1
     \xdef\first@{\csname res\the\loopcount\endcsname}
     \xdef\second@{\csname TC@num\the\loopcount\endcsname}
     \ifx\first@\equ@l \def\first@{\dom@char} \def\second@{!}  \xdef\g@p{y} \fi
     \ifx\first@\d@t
            \expandafter\ifnum\csname res@count\the\loopcount\endcsname%
                 <\csname seq@start\the\loopcount\endcsname
                    \ifsh@wg@ps
                       \def\first@{\gap@char}\def\second@{*}
                    \else
                       \def\first@{{}} \def\second@{/}
                    \fi
            \else
               \expandafter\ifnum\csname res@count\the\loopcount\endcsname%
                    <\csname seq@len\the\loopcount\endcsname
                  \def\first@{\gap@char}\def\second@{*}
               \else
                    \ifsh@wg@ps
                       \def\first@{\gap@char}\def\second@{*}
                    \else
                       \def\first@{{}} \def\second@{/}
                    \fi
               \fi
            \fi
     \fi
     \ifx\first@\questi@n
            \expandafter\ifnum\csname res@count\the\loopcount\endcsname%
                 <\csname seq@start\the\loopcount\endcsname
                       \def\first@{{}} \def\second@{/}
            \else
               \expandafter\ifnum\csname res@count\the\loopcount\endcsname%
                    <\csname seq@len\the\loopcount\endcsname
                       \def\first@{\st@p@char}\def\second@{*}
               \else
                       \def\first@{{}} \def\second@{/}
               \fi
            \fi
     \fi
     \xdef\third@{noshade}
     \expandafter\ifx\csname hide@seq\the\loopcount\endcsname\third@
             \expandafter\ifx\csname lower@seq\the\loopcount\endcsname\y@
                \expandafter\xdef\csname lower@seq\the\loopcount\endcsname{n}
                \xdef\first@{;\first@}\fi
             \expandafter\ifx\csname tint@seq\the\loopcount\endcsname\y@
                \expandafter\xdef\csname tint@seq\the\loopcount\endcsname{n}
                \xdef\first@{=\first@}\fi
             \expandafter\ifx\csname emph@seq\the\loopcount\endcsname\y@
                \expandafter\xdef\csname emph@seq\the\loopcount\endcsname{n}
                \xdef\first@{,\first@}\fi
             \expandafter\xdef\csname seq\the\loopcount\endcsname{%
                \csname seq\the\loopcount\endcsname%
                *\first@}
     \else
             \expandafter\ifx\csname lower@seq\the\loopcount\endcsname\y@
                \expandafter\xdef\csname lower@seq\the\loopcount\endcsname{n}
                \xdef\first@{;\first@}\fi
             \expandafter\ifx\csname tint@seq\the\loopcount\endcsname\y@
                \expandafter\xdef\csname tint@seq\the\loopcount\endcsname{n}
                \xdef\first@{=\first@}\fi
             \expandafter\ifx\csname emph@seq\the\loopcount\endcsname\y@
                \expandafter\xdef\csname emph@seq\the\loopcount\endcsname{n}
                \xdef\first@{,\first@}\fi
             \expandafter\xdef\csname seq\the\loopcount\endcsname{%
                \csname seq\the\loopcount\endcsname%
                \second@\first@}
     \fi
   \ifnum\loopcount<\seq@count \repeat
   \xdef\first@{\csname res0\endcsname}
   \expandafter\ifx\csname lower@seq0\endcsname\y@
                \expandafter\xdef\csname lower@seq0\endcsname{n}
                \xdef\first@{;\first@}\fi
   \expandafter\ifx\csname tint@seq0\endcsname\y@
                \expandafter\xdef\csname tint@seq0\endcsname{n}
                \xdef\first@{=\first@}\fi
   \expandafter\ifx\csname emph@seq0\endcsname\y@
                \expandafter\xdef\csname emph@seq0\endcsname{n}
                \xdef\first@{,\first@}\fi
   \xdef\second@{\csname TC@num0\endcsname}
   \ifx\g@p\y@
      \xdef\consensus{\consensus 8{}}
   \else
      \xdef\consensus{\consensus\second@\n@m@tch}
      \xdef\constopo{\constopo\second@}
   \fi
   }

\def\getregion@fromstack@first{%
      \expandafter\getregion@fromstack{\the\loopcount}
      \expandafter\ifx\csname start\the\loopcount\endcsname\ampers@nd
      \else
      \expandafter\ifnum\csname stop\the\loopcount\endcsname<\first@@
         \getregion@fromstack@first
      \else
         \expandafter\ifx\csname all\the\loopcount\endcsname\y@
            \innerloopcount=\csname style\the\loopcount\endcsname
         \fi
         \expandafter\xdef\csname shade@style\the\loopcount\endcsname{%
                          \csname style\the\loopcount\endcsname}
      \fi
      \fi
}

\def\calc@regshade{%
      \loopcount=-1 \innerloopcount=0
      \loop
         \advance\loopcount by 1
         \expandafter\xdef\csname shade@style\the\loopcount\endcsname{y}
         \ifnum\loopcount=0 \xdef\first@@{\the\cons@count}
         \else \xdef\first@@{\csname res@count\the\loopcount\endcsname} \fi
         \expandafter\ifx\csname start\the\loopcount\endcsname\ampers@nd
         \else
         \expandafter\ifnum\csname start\the\loopcount\endcsname>\first@@
         \else
            \expandafter\ifnum\csname stop\the\loopcount\endcsname<\first@@
               \getregion@fromstack@first
            \else
               \expandafter\ifx\csname all\the\loopcount\endcsname\y@
                  \innerloopcount=\csname style\the\loopcount\endcsname
               \fi
               \expandafter\xdef\csname shade@style\the\loopcount\endcsname{%
                                \csname style\the\loopcount\endcsname}
               \expandafter\ifnum\csname stop\the\loopcount\endcsname=\first@@
                  \expandafter\getregion@fromstack{\the\loopcount}
               \fi
            \fi
         \fi\fi
      \ifnum\loopcount<\seq@count \repeat
      \loopcount=0
      \expandafter\ifx\csname shade@style\the\loopcount\endcsname\y@
      \else
         \xdef\consensus{\consensus&\csname shade@style\the\loopcount\endcsname)}
      \fi
      \loop
         \advance\loopcount by 1
         \expandafter\ifx\csname shade@style\the\loopcount\endcsname\y@
            \ifnum\innerloopcount>0
               \expandafter\xdef\csname seq\the\loopcount\endcsname{%
                                \csname seq\the\loopcount\endcsname%
                                &\the\innerloopcount)}
            \fi
         \else
             \expandafter\xdef\csname seq\the\loopcount\endcsname{%
                              \csname seq\the\loopcount\endcsname%
                              &\csname shade@style\the\loopcount\endcsname)}
         \fi
      \ifnum\loopcount<\seq@count \repeat
}

\def\getregion@fromshadingstack@first{%
      \expandafter\getregion@fromshadingstack{\the\loopcount}
      \expandafter\ifx\csname shadingstart\the\loopcount\endcsname\ampers@nd
      \else
      \expandafter\ifnum\csname shadingstop\the\loopcount\endcsname<\first@@
         \getregion@fromshadingstack@first
      \fi
      \fi
}

\def\calc@shading{%
      \xdef\shading@style{&}
      \loopcount=-1
      \loop
         \advance\loopcount by 1
         \ifnum\loopcount=0 \xdef\first@@{\the\cons@count}
         \else \xdef\first@@{\csname res@count\the\loopcount\endcsname} \fi
         \expandafter\ifx\csname shadingstart\the\loopcount\endcsname\ampers@nd
         \else
         \expandafter\ifnum\csname shadingstart\the\loopcount\endcsname>\first@@
         \else
            \expandafter\ifnum\csname shadingstop\the\loopcount\endcsname<\first@@
               \getregion@fromshadingstack@first
            \else
               \xdef\shading@style{\csname shadingstyle\the\loopcount\endcsname}
               \expandafter\ifnum\csname shadingstop\the\loopcount\endcsname=\first@@
                  \expandafter\getregion@fromshadingstack{\the\loopcount}
               \fi
            \fi
         \fi\fi
      \ifnum\loopcount<\seq@count \repeat
      \loopcount=0
      \loop
         \advance\loopcount by 1
         \ifx\shading@style\ampers@nd
         \else
             \expandafter\xdef\csname seq\the\loopcount\endcsname{%
                              \csname seq\the\loopcount\endcsname�\shading@style)}
         \fi
      \ifnum\loopcount<\seq@count \repeat
}

\def\getregion@fromemphstack@first{%
      \expandafter\getregion@fromemphstack{\the\loopcount}
      \expandafter\ifx\csname emphstart\the\loopcount\endcsname\ampers@nd
      \else
      \expandafter\ifnum\csname emphstop\the\loopcount\endcsname<\first@@
         \getregion@fromemphstack@first
      \else
         \expandafter\ifx\csname emphall\the\loopcount\endcsname\y@
            \innerloopcount=1
         \fi
         \expandafter\xdef\csname emph@seq\the\loopcount\endcsname{y}
      \fi
      \fi
}

\def\calc@regemph{%
      \loopcount=-1 \innerloopcount=0
      \loop
         \advance\loopcount by 1
         \ifnum\loopcount=0 \xdef\first@@{\the\cons@count}
         \else \xdef\first@@{\csname res@count\the\loopcount\endcsname} \fi
         \expandafter\ifx\csname emphstart\the\loopcount\endcsname\ampers@nd
         \else
         \expandafter\ifnum\csname emphstart\the\loopcount\endcsname>\first@@
         \else
            \expandafter\ifnum\csname emphstop\the\loopcount\endcsname<\first@@
              \getregion@fromemphstack@first
            \else
               \expandafter\ifx\csname emphall\the\loopcount\endcsname\y@
                  \innerloopcount=1
               \fi
               \expandafter\xdef\csname emph@seq\the\loopcount\endcsname{y}
               \expandafter\ifnum\csname emphstop\the\loopcount\endcsname=\first@@
                  \expandafter\getregion@fromemphstack{\the\loopcount}
               \fi
            \fi
         \fi\fi
      \ifnum\loopcount<\seq@count \repeat
      \loopcount=0
      \expandafter\ifx\csname emph@seq\the\loopcount\endcsname\y@
         \expandafter\xdef\csname emph@seq\the\loopcount\endcsname{y}
      \fi
      \ifnum\innerloopcount>0
         \loop
            \advance\loopcount by 1
            \expandafter\xdef\csname emph@seq\the\loopcount\endcsname{y}
         \ifnum\loopcount<\seq@count \repeat
      \fi
}

\def\getregion@fromlowerstack@first{%
      \expandafter\getregion@fromlowerstack{\the\loopcount}
      \expandafter\ifx\csname lowerstart\the\loopcount\endcsname\ampers@nd
      \else
      \expandafter\ifnum\csname lowerstop\the\loopcount\endcsname<\first@@
         \getregion@fromlowerstack@first
      \else
         \expandafter\ifx\csname lowerall\the\loopcount\endcsname\y@
            \innerloopcount=1
         \fi
         \expandafter\xdef\csname lower@seq\the\loopcount\endcsname{y}
      \fi
      \fi
}

\def\calc@reglower{%
      \loopcount=-1 \innerloopcount=0
      \loop
         \advance\loopcount by 1
         \ifnum\loopcount=0 \xdef\first@@{\the\cons@count}
         \else \xdef\first@@{\csname res@count\the\loopcount\endcsname} \fi
         \expandafter\ifx\csname lowerstart\the\loopcount\endcsname\ampers@nd
         \else
         \expandafter\ifnum\csname lowerstart\the\loopcount\endcsname>\first@@
         \else
            \expandafter\ifnum\csname lowerstop\the\loopcount\endcsname<\first@@
              \getregion@fromlowerstack@first
            \else
               \expandafter\ifx\csname lowerall\the\loopcount\endcsname\y@
                  \innerloopcount=1
               \fi
               \expandafter\xdef\csname lower@seq\the\loopcount\endcsname{y}
               \expandafter\ifnum\csname lowerstop\the\loopcount\endcsname=\first@@
                  \expandafter\getregion@fromlowerstack{\the\loopcount}
               \fi
            \fi
         \fi\fi
      \ifnum\loopcount<\seq@count \repeat
      \loopcount=0
      \expandafter\ifx\csname lower@seq\the\loopcount\endcsname\y@
         \expandafter\xdef\csname lower@seq\the\loopcount\endcsname{y}
      \fi
      \ifnum\innerloopcount>0
         \loop
            \advance\loopcount by 1
            \expandafter\xdef\csname lower@seq\the\loopcount\endcsname{y}
         \ifnum\loopcount<\seq@count \repeat
      \fi
}

\def\getregion@fromtintstack@first{%
      \expandafter\getregion@fromtintstack{\the\loopcount}
      \expandafter\ifx\csname tintstart\the\loopcount\endcsname\ampers@nd
      \else
      \expandafter\ifnum\csname tintstop\the\loopcount\endcsname<\first@@
         \getregion@fromtintstack@first
      \else
         \expandafter\ifx\csname tintall\the\loopcount\endcsname\y@
            \innerloopcount=1
         \fi
         \expandafter\xdef\csname tint@seq\the\loopcount\endcsname{y}
      \fi
      \fi
}

\def\calc@regtint{%
      \loopcount=-1 \innerloopcount=0
      \loop
         \advance\loopcount by 1
         \ifnum\loopcount=0 \xdef\first@@{\the\cons@count}
         \else \xdef\first@@{\csname res@count\the\loopcount\endcsname} \fi
         \expandafter\ifx\csname tintstart\the\loopcount\endcsname\ampers@nd
         \else
         \expandafter\ifnum\csname tintstart\the\loopcount\endcsname>\first@@
         \else
            \expandafter\ifnum\csname tintstop\the\loopcount\endcsname<\first@@
              \getregion@fromtintstack@first
            \else
               \expandafter\ifx\csname tintall\the\loopcount\endcsname\y@
                  \innerloopcount=1
               \fi
               \expandafter\xdef\csname tint@seq\the\loopcount\endcsname{y}
               \expandafter\ifnum\csname tintstop\the\loopcount\endcsname=\first@@
                  \expandafter\getregion@fromtintstack{\the\loopcount}
               \fi
            \fi
         \fi\fi
      \ifnum\loopcount<\seq@count \repeat
      \loopcount=0
      \expandafter\ifx\csname tint@seq\the\loopcount\endcsname\y@
         \expandafter\xdef\csname tint@seq\the\loopcount\endcsname{y}
      \fi
      \ifnum\innerloopcount>0
         \loop
            \advance\loopcount by 1
            \expandafter\xdef\csname tint@seq\the\loopcount\endcsname{y}
         \ifnum\loopcount<\seq@count \repeat
      \fi
}

\def\getregion@fromframestack@first{%
      \expandafter\getregion@fromframestack{\the\loopcount}
      \expandafter\ifx\csname framestart\the\loopcount\endcsname\ampers@nd
      \else
      \expandafter\ifnum\csname framestop\the\loopcount\endcsname<\first@@
         \getregion@fromframestack@first
      \else
         \ifnum\frame@on=0
            \xdef\frame@on{1}
            \xdef\frame@{1}
            \expandafter\xdef\csname fr@style\the\loopcount\endcsname{%
                             \csname framestyle\the\loopcount\endcsname}
            \innerloopcount=\pos@count
            \advance\innerloopcount by -\frame@pos
            \xdef\styleframe{\styleframe&\the\innerloopcount;&;}
            \xdef\frame@pos{\the\pos@count}
         \fi
         \expandafter\ifnum\csname framestop\the\loopcount\endcsname=\first@@
            \expandafter\getregion@fromframestack{\the\loopcount}
            \ifnum\frame@on=1
               \xdef\frame@on{0}
               \innerloopcount=\pos@count \advance\innerloopcount by 1
               \advance\innerloopcount by -\frame@pos
               \xdef\styleframe{\styleframe&\the\innerloopcount;%
                                \csname fr@style\the\loopcount\endcsname;}
               \innerloopcount=\pos@count \advance\innerloopcount by 1
               \xdef\frame@pos{\the\innerloopcount}
            \fi
         \fi
         \ifnum\pos@count=\res@perline
            \ifnum\frame@on=1
               \innerloopcount=\pos@count
               \advance\innerloopcount by 1
               \advance\innerloopcount by -\frame@pos
               \xdef\styleframe{\styleframe&\the\innerloopcount;%
                                \csname fr@style\the\loopcount\endcsname;}
               \innerloopcount=\pos@count \advance\innerloopcount by 1
               \xdef\frame@pos{\the\innerloopcount}
            \fi
         \fi
         \expandafter\ifnum\csname res@count\the\loopcount\endcsname=\end@num\relax
            \ifnum\frame@on=1
               \innerloopcount=\pos@count
               \advance\innerloopcount by 1
               \advance\innerloopcount by -\frame@pos
               \xdef\styleframe{\styleframe&\the\innerloopcount;%
                                \csname fr@style\the\loopcount\endcsname;}
               \innerloopcount=\pos@count \advance\innerloopcount by 1
               \xdef\frame@pos{\the\innerloopcount}
            \fi
         \fi
      \fi
      \fi
}

\def\calc@frame{%
%      \advance\pos@count by -1
      \loopcount=-1 \innerloopcount=0
      \loop
         \advance\loopcount by 1
         \ifnum\loopcount=0 \xdef\first@@{\the\cons@count}
         \else \xdef\first@@{\csname res@count\the\loopcount\endcsname} \fi
         \expandafter\ifx\csname framestart\the\loopcount\endcsname\ampers@nd
         \else
         \expandafter\ifnum\csname framestart\the\loopcount\endcsname>\first@@
         \else
            \expandafter\ifnum\csname framestop\the\loopcount\endcsname<\first@@
              \getregion@fromframestack@first
            \else
               \ifnum\frame@on=0
                  \xdef\frame@on{1}
                  \xdef\frame@{1}
                  \expandafter\xdef\csname fr@style\the\loopcount\endcsname{%
                                   \csname framestyle\the\loopcount\endcsname}
                  \innerloopcount=\pos@count 
                  \advance\innerloopcount by -\frame@pos
                  \xdef\styleframe{\styleframe&\the\innerloopcount;&;}
                  \xdef\frame@pos{\the\pos@count}
               \fi
               \expandafter\ifnum\csname framestop\the\loopcount\endcsname=\first@@
                  \expandafter\getregion@fromframestack{\the\loopcount}
                  \ifnum\frame@on=1
                     \xdef\frame@on{0}
                     \innerloopcount=\pos@count \advance\innerloopcount by 1
                     \advance\innerloopcount by -\frame@pos
                     \xdef\styleframe{\styleframe&\the\innerloopcount;%
                                      \csname fr@style\the\loopcount\endcsname;}
                     \innerloopcount=\pos@count \advance\innerloopcount by 1
                     \xdef\frame@pos{\the\innerloopcount}
                  \fi
               \fi
               \ifnum\pos@count=\res@perline
                  \ifnum\frame@on=1
                     \innerloopcount=\pos@count 
                     \advance\innerloopcount by 1
                     \advance\innerloopcount by -\frame@pos
                     \xdef\styleframe{\styleframe&\the\innerloopcount;%
                                      \csname fr@style\the\loopcount\endcsname;}
                     \innerloopcount=\pos@count \advance\innerloopcount by 1
                     \xdef\frame@pos{\the\innerloopcount}
                  \fi
               \fi
               \expandafter\ifnum\csname res@count\the\loopcount\endcsname=\end@num\relax
                  \ifnum\frame@on=1
                     \innerloopcount=\pos@count
                     \advance\innerloopcount by 1
                     \advance\innerloopcount by -\frame@pos
                     \xdef\styleframe{\styleframe&\the\innerloopcount;%
                                      \csname fr@style\the\loopcount\endcsname;}
                     \innerloopcount=\pos@count \advance\innerloopcount by 1
                     \xdef\frame@pos{\the\innerloopcount}
                  \fi
               \fi
            \fi
         \fi\fi
      \ifnum\loopcount<\seq@count \repeat
%      \advance\pos@count by 1 
}

\def\get@nextres#1#2:{%
    \xdef\first@{#1}
    \xdef\temp@{#2:}
    \ifx\first@\gap@char \expandafter\get@nextres\temp@
    \else
       \if\first@ @
          \xdef\temp@{}
       \else
          \xdef\temp@{+\first@}
          \expandafter\xdef\csname last@res\bottop@\endcsname{}
       \fi
    \fi}
\def\get@@nextres#1#2:{%
    \xdef\first@{#1}
    \xdef\temp@@{#2:}
    \ifx\first@\gap@char \expandafter\get@@nextres\temp@@
    \else
       \if\first@ @
          \xdef\temp@@{}
       \else
          \xdef\temp@@{+\first@}
          \expandafter\xdef\csname last@res\bottop@\endcsname{}
       \fi
    \fi}

\def\char@get#1#2@{\xdef\first@{#1} \xdef\tr@nsl@ted{#2@}} 
\def\trans@now#1#2@{%
   \xdef\first@{#1}
   \xdef\tr@nsl@ted{#2@}
   \ifx\first@\ampers@nd
   \else
      \expandafter\check@char\first@
      \ifletter
         \xdef\triplet@{\triplet@\first@}
         \advance\triple@count by 1
         \ifnum\triple@count=1
            \xdef\out@{\out@{-}}
         \fi
         \ifnum\triple@count=3
            \expandafter\ifx\csname @\triplet@\endcsname\relax
            \expandafter\xdef\csname @\triplet@\endcsname{?} \fi
            \xdef\out@{\out@\csname @\triplet@\endcsname\out@@{-}}
            \triple@count=0
            \xdef\triplet@{}
            \xdef\out@@{}
         \fi
      \fi
      \if\first@ -
         \ifnum\triple@count<2
            \xdef\out@{\out@{-}}
         \else
            \xdef\out@@{\out@@{-}}
         \fi
      \fi
      \if\first@ +
         \expandafter\char@get\tr@nsl@ted
         \xdef\triplet@{\triplet@\first@}
         \advance\triple@count by 1
         \ifnum\triple@count=3
            \expandafter\ifx\csname @\triplet@\endcsname\relax
            \expandafter\xdef\csname @\triplet@\endcsname{?} \fi
            \xdef\out@{\out@\csname @\triplet@\endcsname\out@@}
            \triple@count=0
            \xdef\triplet@{}
            \xdef\out@@{}
         \fi
      \fi
      \if\first@ 2
         \loop
            \expandafter\char@get\tr@nsl@ted
            \ifx\first@\ampers@nd
               \lettertrue
               \xdef\tr@nsl@ted{&@}
            \else
               \expandafter\check@char\first@
            \fi
         \ifletter\else\xdef\out@{\out@{-}}\repeat
         \xdef\out@{\out@{-}}
      \fi
   \expandafter\trans@now\tr@nsl@ted
   \fi
}

\def\do@translation{%
   \xdef\triplet@{}
   \xdef\out@{}
   \xdef\out@@{}
   \xdef\tr@nsl@ted{\tr@nsl@ted &@}
   \triple@count=0
   \expandafter\trans@now\tr@nsl@ted
   \xdef\tr@nsl@ted{\out@}
}

\def\trans@pep#1#2@{%
   \xdef\first@{#1}
   \xdef\tr@nsl@ted{#2@}
   \ifx\first@\ampers@nd
   \else
      \expandafter\check@char\first@
  \ifletter
     \xdef\out@{\out@\csname rev@\first@\endcsname}
  \else
     \xdef\out@{\out@{-}{-}{-}}
  \fi
   \expandafter\trans@pep\tr@nsl@ted
   \fi
}

\def\rev@translation{%
   \xdef\out@{}
   \xdef\tr@nsl@ted{\tr@nsl@ted &@}
   \expandafter\trans@pep\tr@nsl@ted
   \xdef\tr@nsl@ted{\out@}
}


\def\sum@up{%
   \advance\innerloopcount by 1
   \xdef\second@@@{\csname res\the\innerloopcount\endcsname}
   \xdef\third@@@{\csname cons\first@@@\second@@@\endcsname}   
   \advance\temp@count by \third@@@   
   \ifnum\innerloopcount<\seq@count\sum@up\fi
}

\def\sum@up@cons{%
   \innerloopcount=\outerloopcount
   \xdef\first@@@{\csname res\the\outerloopcount\endcsname}
   \sum@up
   \advance\outerloopcount by 1\relax
   \ifnum\outerloopcount<\seq@count
      \sum@up@cons
   \else
      \innerloopcount=\seq@count
      \advance\innerloopcount by -1
      \multiply\innerloopcount by \seq@count
      \multiply\temp@count by 2
      \multiply\temp@count by \m@trixf@ctor
      \divide\temp@count by \innerloopcount
      
      \ifnum\temp@count>100 \temp@count=100 \fi
      \ifnum\temp@count<0   \temp@count=0 \fi
      \xdef\cons@val{\the\temp@count}
      \ifx\first@\equ@l \xdef\cons@val{N} \fi
      
      \ifx\T@coffee@bcons\y@
         \xdef\cons@val{\csname TC@num0\endcsname}
         \if\cons@val * 
            \xdef\cons@val{99}
         \fi
      \fi
      \ifx\T@coffee@ccons\y@
         \xdef\cons@val{\csname TC@num0\endcsname}
         \if\cons@val * 
            \xdef\cons@val{99}
         \fi
      \fi
   \fi
}

\def\collect@cons@res{%
   \xdef\temp@{\temp@\csname res\the\innerloopcount\endcsname}
   \advance\innerloopcount by 1
   \ifnum\innerloopcount>\seq@count\relax
      \outerloopcount=1\relax
      \temp@count=0\relax
      \sum@up@cons
   \else
      \collect@cons@res
   \fi
}

\def\sum@@up{%
   \advance\innerloopcount by 1
   \xdef\second@@@{\csname res\the\innerloopcount\endcsname}
   
   \ifx\first@@@\d@t
   \else
    \ifx\first@@@\questi@n
    \else
     \ifx\first@@@\equ@l
     \else       
         \ifx\second@@@\d@t
         \else
          \ifx\second@@@\questi@n
          \else
           \ifx\second@@@\equ@l
           \else
               \xdef\third@@@{\csname poscount\the\outerloopcount @\the\innerloopcount\endcsname}
               \temp@@count=\third@@@
               \advance\temp@@count by 1
               \expandafter\xdef\csname poscount\the\outerloopcount @\the\innerloopcount\endcsname{\the\temp@@count}
               
               \xdef\third@@@{\csname simcount\the\outerloopcount @\the\innerloopcount\endcsname}
               \temp@@count=\third@@@
               \xdef\third@@@{\csname simpair\first@@@\second@@@\endcsname}
               \advance\temp@@count by \third@@@
               \expandafter\xdef\csname simcount\the\outerloopcount @\the\innerloopcount\endcsname{\the\temp@@count}
               
               \ifx\first@@@\second@@@
                  \xdef\third@@@{\csname identcount\the\outerloopcount @\the\innerloopcount\endcsname}
                  \temp@@count=\third@@@
                  \advance\temp@@count by 1
                  \expandafter\xdef\csname identcount\the\outerloopcount @\the\innerloopcount\endcsname{\the\temp@@count}
               \fi
           \fi
          \fi
         \fi
     \fi
    \fi
   \fi
   
   \ifnum\innerloopcount<\seq@count\sum@@up\fi
}

\def\sum@up@sim{%
   \innerloopcount=\outerloopcount
   \xdef\first@@@{\csname res\the\outerloopcount\endcsname}
   \sum@@up
   \advance\outerloopcount by 1\relax
   \ifnum\outerloopcount<\seq@count
      \sum@up@sim
   \fi
}

\def\collect@similarity{%
   \xdef\temp@{\temp@\csname res\the\innerloopcount\endcsname}
   \advance\innerloopcount by 1
   \ifnum\innerloopcount>\seq@count\relax
      \outerloopcount=1\relax
      \temp@count=0\relax
      \sum@up@sim
   \else
      \collect@similarity
   \fi
}


\def\sum@up@bits{%
   \ifx\first@@@\second@@@
      \ifx\first@@@\d@t
         \temp@count=0\relax
      \else
         \temp@count=100\relax
      \fi
   \else
      \temp@count=0
%      \xdef\third@@@{\csname cons\first@@@\second@@@\endcsname}\temp@count=\third@@@   %%% or if only identical =0
   \fi
   \multiply\temp@count by \last@ 
   \divide\temp@count by 100  
   \expandafter\ifx\csname info@\subfamily@seq @\the\outerloopcount\endcsname\relax
      \expandafter\xdef\csname info@\subfamily@seq @\the\outerloopcount\endcsname{0}
   \fi
   \xdef\third@@@{\csname info@\subfamily@seq @\the\outerloopcount\endcsname} 
   \advance\temp@count by \third@@@ 
   \expandafter\xdef\csname info@\subfamily@seq @\the\outerloopcount\endcsname{\the\temp@count}
}
\def\sum@up@info{%
   \ifnum\outerloopcount=\subfamily@seq \advance\outerloopcount by 1 \fi
   \ifnum\outerloopcount>\seq@count
   \else
      \xdef\first@@@{\csname res\the\outerloopcount\endcsname}
      \sum@up@bits
      \advance\outerloopcount by 1\relax
   \fi
   \ifnum\outerloopcount>\seq@count\else\sum@up@info\fi
}
\def\collect@info{%
   \outerloopcount=1\relax
   \temp@count=0\relax
   \xdef\second@@@{\csname res\subfamily@seq\endcsname}   
   \sum@up@info
}

\def\calc@grouping{%
   \xdef\second@{\csname res@num\d@t\endcsname}
   \innerloopcount=\seq@count
   \advance\innerloopcount by -\second@
   \xdef\second@{\the\innerloopcount}
   \xdef\seventh@{0}
   \innerloopcount=0
   \outerloopcount=65
   \loop
      \xdef\first@@{\csname ch@r@\the\outerloopcount\endcsname}
      \xdef\first@{\csname res@num\first@@\endcsname}
      \ifnum\first@>0
         \temp@count=\csname log2@\first@\endcsname
         \advance\temp@count by -\csname log2@\second@\endcsname
         \advance\temp@count by \csname res@corr\first@@\endcsname
         \multiply\temp@count by -\first@
         \divide\temp@count by \second@
         \advance\innerloopcount by \temp@count\relax
         \temp@count=\second@              %%%
         \advance\temp@count by -\first@   %%%
         \advance\temp@count by \seventh@  %%%
         \xdef\seventh@{\the\temp@count}   %%%
      \fi
      \expandafter\xdef\csname res@num\first@@\endcsname{0}
      \advance\outerloopcount by 1
   \ifnum\outerloopcount>90\else\repeat
   \expandafter\xdef\csname res@num\d@t\endcsname{0}
   \if\seq@type P
      \temp@count=\csname log2@20\endcsname 
      \ifnum\sig@max=100000 \xdef\sig@max{2321} \fi
   \else
      \temp@count=\csname log2@4\endcsname
      \ifnum\sig@max=100000 \xdef\sig@max{1000} \fi
   \fi
   \xdef\bit@max{\the\temp@count}
   \advance\temp@count by -\innerloopcount
   \multiply\temp@count by \second@
   \divide\temp@count by \seq@count
   \xdef\last@{\the\temp@count}
   \expandafter\xdef\csname bit@pos\the\loopcount\endcsname{\last@}
   \temp@count=\bit@total
   \advance\temp@count by \last@
   \xdef\bit@total{\the\temp@count}
   \collect@info
}
\def\do@grouping{%
   \ifnum\loopcount>\total@pos
   \else
      \triple@count=1
      \loop
         \xdef\first@{\csname seq\the\triple@count\endcsname}
         \expandafter\dis@get\first@
         \ifx\first@\ampers@nd
         \else
            \innerloopcount=\csname res@num\first@\endcsname
            \advance\innerloopcount by 1 
            \expandafter\xdef\csname res@num\first@\endcsname{\the\innerloopcount}
         \fi
         \advance\triple@count by 1\relax 
      \ifnum\triple@count>\seq@count
         \calc@grouping
         \advance\loopcount by 1
         \do@grouping
      \else
      \repeat 
   \fi
}
\def\total@frequency@correction{%
   \innerloopcount=0 
   \outerloopcount=65
   \loop
      \xdef\first@@{\csname ch@r@\the\outerloopcount\endcsname}
      \xdef\first@{\csname res@corr\first@@\endcsname}
      \advance\innerloopcount by \first@\relax  
      \advance\outerloopcount by 1
   \ifnum\outerloopcount>90\else\repeat
   \xdef\res@num@total{\the\innerloopcount}
   \outerloopcount=65
   \loop
      \xdef\first@@{\csname ch@r@\the\outerloopcount\endcsname}
      \xdef\first@{\csname res@corr\first@@\endcsname}
      \innerloopcount=\first@\relax  
      \ifnum\innerloopcount>0
         \multiply\innerloopcount by 1000\relax   
         \divide\innerloopcount by \res@num@total\relax  
         \innerloopcount=\csname log2@\the\innerloopcount\endcsname  
         \advance\innerloopcount by -5644\relax  
         \multiply\innerloopcount by -1\relax
      \fi 
      \expandafter\xdef\csname res@corr\first@@\endcsname{\the\innerloopcount}
      \ifnum\the\innerloopcount>\corr@max \xdef\corr@max{\the\innerloopcount}\fi
      \advance\outerloopcount by 1  
   \ifnum\outerloopcount>90\else\repeat
}
\def\calc@total@frequency{%
   \ifnum\loopcount>\total@pos
   \else
      \triple@count=1
      \loop
         \xdef\first@{\csname seq\the\triple@count\endcsname}
         \expandafter\tot@get\first@ 
         \ifx\first@\ampers@nd
         \else
            \innerloopcount=\csname res@corr\first@\endcsname 
            \advance\innerloopcount by 1 
            \expandafter\xdef\csname res@corr\first@\endcsname{\the\innerloopcount}
         \fi
         \advance\triple@count by 1\relax 
      \ifnum\triple@count>\seq@count
         \advance\loopcount by 1
         \calc@total@frequency
      \else
      \repeat 
   \fi
}
\def\prep@logo{%
    \if\seq@type P
     \ifx\do@freq@correction\y@
       \xdef\do@freq@correction{n}
       \loopcount=1
       \loop
          \expandafter\xdef\csname seq\the\loopcount\endcsname{\csname sequence\the\loopcount\endcsname &@}
          \advance\loopcount by 1
       \ifnum\loopcount>\seq@count\else\repeat
       \xdef\corr@max{0}
       \loopcount=1
       \calc@total@frequency   
       \total@frequency@correction 
     \fi
    \fi
}
\def\define@subfamilies{%
  \xdef\first@{\csname group@num1\endcsname}
  \advance\loopcount by -1 
  \xdef\last@{\the\loopcount}
  \outerloopcount=\bit@total
  \divide\outerloopcount by \last@
  \xdef\bit@mean{\the\outerloopcount}
  \multiply\outerloopcount by \subfamily@threshold\relax
  \divide\outerloopcount by 100\relax
  \xdef\sub@threshold{\the\outerloopcount}
  \ifnum\first@=0
    \clear@res@nums{1}
    \clear@res@nums{2}
    \immediate\write\featurefile{TeXshade subfamily logo data file for \alignfilename}
    \immediate\write\featurefile{--}
    \immediate\write\featurefile{Average information content [1000*bits per position]: \bit@mean}
    \immediate\write\featurefile{Subfamily threshold setting [percent]: \subfamily@threshold}
    \immediate\write\featurefile{=> \bit@mean\space * 0.\subfamily@threshold\space = \sub@threshold\space [1000*bits per position]}
    \immediate\write\featurefile{}
    \expandafter\xdef\csname subfamily@num\subfamily@seq\endcsname{2}
    \immediate\write\featurefile{Automatic subfamily assignment around sequence no. \subfamily@seq:} 
    \immediate\write\featurefile{[sequence pair: average shared information *1000, subfamily number]}
    \temp@count=1 
    \outerloopcount=1
    \loop
       \ifnum\outerloopcount=\subfamily@seq \advance\outerloopcount by 1 \fi
       \ifnum\outerloopcount>\seq@count
       \else
          \xdef\first@{\csname info@\subfamily@seq @\the\outerloopcount\endcsname}
          \loopcount=\first@\relax 
          \divide\loopcount by \last@\relax 
          \ifnum\loopcount>\sub@threshold
             \expandafter\xdef\csname subfamily@num\the\outerloopcount\endcsname{2}
             \advance\temp@count by 1\relax 
          \else
             \expandafter\xdef\csname subfamily@num\the\outerloopcount\endcsname{1}
          \fi    
          \xdef\first@@{\csname subfamily@num\the\outerloopcount\endcsname}          
          \immediate\write\featurefile{\subfamily@seq-\the\outerloopcount:\space\the\loopcount,\space subfamily: \first@@}
          \advance\outerloopcount by 1 
       \fi
    \ifnum\outerloopcount>\seq@count\else\repeat
    \immediate\write\featurefile{}
    \immediate\write\featurefile{//}
    \ifnum\temp@count>0
       \expandafter\xdef\csname group@num2\endcsname{\the\temp@count}
    \else
       \message{<No subfamilies found (check threshold) - hiding subfamily logo>}
       \show@sublogofalse
       \immediate\write\featurefile{<No subfamilies found (check threshold) - hiding subfamily logo>}
    \fi
    \loopcount=\seq@count 
    \advance\loopcount by -\temp@count
    \ifnum\loopcount>0
       \expandafter\xdef\csname group@num1\endcsname{\the\loopcount}
    \else
       \message{<No subfamilies found (check threshold) - hiding subfamily logo>}
       \immediate\write\featurefile{<No subfamilies found (check threshold) - hiding subfamily logo>}
       \show@sublogofalse
    \fi
  \else
    \immediate\write\featurefile{TeXshade subfamily logo data file for \alignfilename}
    \immediate\write\featurefile{--}
    \immediate\write\featurefile{Average information content [1000*bits per position]: \bit@mean}
    \immediate\write\featurefile{Subfamily threshold setting [percent]: \subfamily@threshold}
    \immediate\write\featurefile{=> \bit@mean\space * 0.\subfamily@threshold\space = \sub@threshold\space [1000*bits per position]}
    \immediate\write\featurefile{}
    \immediate\write\featurefile{User defined subfamily: \sub@family@setting}
    \immediate\write\featurefile{}
    \immediate\write\featurefile{//}
  \fi
}
\def\prep@sublogo{%
    \message{<Calculating subfamily logo...>}
    \immediate\openout\featurefile = sublogo.txt 
    \if\seq@type P
     \ifx\do@freq@correction\y@
       \xdef\do@freq@correction{n}
       \loopcount=1
       \loop
          \xdef\first@{\csname @rd\the\loopcount\endcsname} %%%
          \expandafter\xdef\csname seq\the\loopcount\endcsname{\csname sequence\first@\endcsname &@}
          \advance\loopcount by 1
       \ifnum\loopcount>\seq@count\else\repeat
       \xdef\corr@max{0}
       \loopcount=1
       \calc@total@frequency   
       \total@frequency@correction 
     \fi
    \fi
    \loopcount=1
    \loop
          \xdef\first@{\csname @rd\the\loopcount\endcsname} %%%
          \expandafter\xdef\csname seq\the\loopcount\endcsname{\csname sequence\first@\endcsname &@}
       \advance\loopcount by 1
    \ifnum\loopcount>\seq@count\else\repeat
    \loopcount=1
    \do@grouping 
    \define@subfamilies
    \ifshow@sublogo
       \loopcount=1
       \loop
          \xdef\first@{\csname @rd\the\loopcount\endcsname} %%%
          \expandafter\xdef\csname seq\the\loopcount\endcsname{\csname sequence\first@\endcsname &@}
          \advance\loopcount by 1
       \ifnum\loopcount>\seq@count\else\repeat
       \xdef\first@{\csname group@num1\endcsname}
       \xdef\first@@{\csname log2@\first@\endcsname}
       \xdef\second@{\csname group@num2\endcsname}
       \xdef\second@@{\csname log2@\second@\endcsname}
       \loopcount=\first@ 
       \advance\loopcount by \second@ 
       \xdef\third@{\the\loopcount}
       \xdef\third@@{\csname log2@\third@\endcsname}
       \loopcount=\first@@ 
       \advance\loopcount by -\third@@ 
       \multiply\loopcount by \first@ 
       \divide\loopcount by \third@
       \xdef\first@@@{\the\loopcount}
       \loopcount=\second@@ 
       \advance\loopcount by -\third@@ 
       \multiply\loopcount by \second@ 
       \divide\loopcount by \third@
       \xdef\second@@@{\the\loopcount}
       \loopcount=\bit@max 
       \advance\loopcount by \first@@@
       \advance\loopcount by \second@@@
       \xdef\group@correction{\the\loopcount}
       \loopcount=1 \total@count=0
       \do@sublogo 
       \immediate\closeout\featurefile  
       \openin\sublogofile = sublogo.txt\relax
       \read@header     
    \else
       \immediate\closeout\featurefile  
    \fi 
}
\def\read@header{%
   \read\sublogofile to \first@\relax
   \xdef\first@{\expandafter\string\first@}
   \ifx\first@\par@ \read@header
   \else
      \xdef\first@{\first@ @}
      \expandafter\seq@get\first@
      \ifx\first@\he@derend
      \else\read@header
      \fi
   \fi
}
\def\read@sublogo{%
   \xdef\stack@sublogo{}%
   \loopcount=1 %
   \loop%
      \ifeof\sublogofile%
         \advance\loopcount by \res@perline%
      \else%
         \read\sublogofile to \first@\relax%
         \xdef\first@{\expandafter\string\first@}%
         \ifx\first@\par@%
            \advance\loopcount by \res@perline%
         \else
            \xdef\first@{\first@ @}%
            \expandafter\sublogo@get\first@%
            \xdef\stack@sublogo{\stack@sublogo\first@}%
            \advance\loopcount by 1%
         \fi%
      \fi%
   \ifnum\loopcount=\res@perline\else\repeat%
}
\def\calc@sublogo{%
   \xdef\seventh@{\csname res@num.2\endcsname}
   \xdef\first@{\csname group@num2\endcsname}
   \innerloopcount=\first@
   \advance\innerloopcount by -\seventh@
   \xdef\seventh@{\the\innerloopcount}
   \xdef\eighth@{\csname res@num.1\endcsname}
   \xdef\first@{\csname group@num1\endcsname}
   \innerloopcount=\first@
   \advance\innerloopcount by -\eighth@
   \xdef\eighth@{\the\innerloopcount}
   \xdef\nineth@{n}
   \xdef\pos@max{0}
   \xdef\pos@min{0}
   \outerloopcount=65
   \loop
      \xdef\first@@{\csname ch@r@\the\outerloopcount\endcsname}
      \xdef\first@{\csname res@num\first@@ 1\endcsname} \xdef\temp@{\first@}
      \temp@count=\first@ 
      \ifnum\seventh@>0 \multiply\temp@count by \seventh@ \fi 
      \xdef\first@{\the\temp@count}
      \xdef\second@{\csname res@num\first@@ 2\endcsname}
      \temp@count=\temp@
      \advance\temp@count by \second@\relax 
      \ifnum\temp@count>0
         \temp@count=\second@ 
         \ifnum\eighth@>0 \multiply\temp@count by \eighth@ \fi 
         \advance\temp@count by -\first@ 
         \xdef\last@{\csname bit@pos\the\loopcount\endcsname}
         \multiply\temp@count by \last@ 
         \ifnum\seventh@>0 \divide\temp@count by \seventh@ \fi 
         \ifnum\eighth@>0 \divide\temp@count by \eighth@ \fi 
         \multiply\temp@count by \bit@max  
         \divide\temp@count by \group@correction   
         \ifx\hide@negatives\y@
            \ifnum\temp@count>0 \else \temp@count=1 \fi%
         \else
            \ifnum\temp@count=0 \temp@count=1 \fi%
         \fi 
         \expandafter\xdef\csname res@val\first@@\endcsname{\the\temp@count}
         \expandafter\xdef\csname res@num\first@@ 1\endcsname{0}
         \expandafter\xdef\csname res@num\first@@ 2\endcsname{0}
         \xdef\nineth@{y}
      \else 
         \expandafter\xdef\csname res@val\first@@\endcsname{0}
         \expandafter\xdef\csname res@num\first@@ 1\endcsname{0}
         \expandafter\xdef\csname res@num\first@@ 2\endcsname{0}
      \fi
      \ifnum\temp@count>\pos@max \xdef\pos@max{\the\temp@count}\fi
      \ifnum\temp@count<\pos@min \xdef\pos@min{\the\temp@count}\fi
      \advance\outerloopcount by 1
   \ifnum\outerloopcount>90\else\repeat
   \expandafter\xdef\csname res@val\d@t\endcsname{0}
   \expandafter\xdef\csname res@num\d@t 1\endcsname{0}
   \expandafter\xdef\csname res@num\d@t 2\endcsname{0}
   \temp@count=\pos@min\relax 
   \multiply\temp@count by -1\relax 
   \xdef\pos@min{\the\temp@count}
   \ifnum\pos@min>\pos@max \xdef\pos@max{\pos@min}\fi
   \temp@count=\pos@max
   \multiply\temp@count by 100 \relax
   \divide\temp@count by \bit@max \relax
   \xdef\sublogo@num{\sublogo@num\the\temp@count,}
   \ifnum\pos@max<\sig@max \xdef\sublogo@sig{\sublogo@sig n} \else  \xdef\sublogo@sig{\sublogo@sig y} \fi
   \ifx\nineth@\y@ 
      \xdef\tmpstack{&:&,}
      \outerloopcount=65
      \loop
         \xdef\first@@{\csname ch@r@\the\outerloopcount\endcsname}
         \xdef\second@@{\csname res@val\first@@\endcsname}
         \ifnum\second@@=0 
         \else
            \xdef\third@{\tmpstack @} \xdef\tmpstack{} 
            \sort@logostack
         \fi
         \advance\outerloopcount by 1
      \ifnum\outerloopcount>90 
         \immediate\write\featurefile{\expandafter\string\tmpstack&:&,} 
      \else\repeat
   \else
      \immediate\write\featurefile{\expandafter\string O:0,&:&,} 
   \fi
}
\def\do@sublogo{%
   \ifnum\loopcount>\total@pos 
   \xdef\sublogo@sig{\sublogo@sig &@}
   \else 
      \triple@count=1
      \loop
         \xdef\first@{\csname seq\the\triple@count\endcsname}
         \expandafter\dis@get\first@
         \ifnum\triple@count=\start@seq
            \ifx\first@\d@t\else \advance\total@count by 1\relax\fi  
         \fi
         \ifnum\loopcount<\start@number
         \else
         \ifnum\total@count>\end@num
         \else
            \ifx\first@\ampers@nd
            \else
               \xdef\second@{\csname subfamily@num\the\triple@count\endcsname}
               \innerloopcount=\csname res@num\first@\second@\endcsname
               \advance\innerloopcount by 1 
               \expandafter\xdef\csname res@num\first@\second@\endcsname{\the\innerloopcount}
            \fi
         \fi
         \fi
         \advance\triple@count by 1\relax 
      \ifnum\triple@count>\seq@count
         \ifnum\loopcount<\start@number
         \else
            \ifnum\total@count>\end@num
            \else
               \calc@sublogo 
            \fi
         \fi
         \advance\loopcount by 1   
         \do@sublogo
      \else
      \repeat
   \fi
}
\def\calc@logo{%
   \xdef\second@{\csname res@num\d@t\endcsname}
   \innerloopcount=\seq@count
   \advance\innerloopcount by -\second@
   \xdef\second@{\the\innerloopcount}
   \xdef\seventh@{0}
   \innerloopcount=0
   \outerloopcount=65
   \loop
      \xdef\first@@{\csname ch@r@\the\outerloopcount\endcsname}
      \xdef\first@{\csname res@num\first@@\endcsname}
      \ifnum\first@>0
         \temp@count=\csname log2@\first@\endcsname
         \advance\temp@count by -\csname log2@\second@\endcsname
         \advance\temp@count by \csname res@corr\first@@\endcsname     %%%% correct for background freq.
         \multiply\temp@count by -\first@
         \divide\temp@count by \second@
         \advance\innerloopcount by \temp@count\relax
      \fi
      \advance\outerloopcount by 1
   \ifnum\outerloopcount>90\else\repeat
   \if\seq@type P
      \temp@count=\csname log2@20\endcsname
   \else
      \temp@count=\csname log2@4\endcsname
   \fi
   \advance\temp@count by -\innerloopcount
   \multiply\temp@count by \second@
   \divide\temp@count by \seq@count
   \xdef\last@{\the\temp@count}
   \xdef\nineth@{n}
   \outerloopcount=65
   \loop
      \xdef\first@@{\csname ch@r@\the\outerloopcount\endcsname}
      \xdef\first@{\csname res@num\first@@\endcsname}
      \ifnum\first@>0
         \temp@count=\last@
         \multiply\temp@count by \first@\relax 
         \divide\temp@count by \second@\relax 
         \ifnum\temp@count=0 %
            \temp@count=1 %
         \fi%
         \expandafter\xdef\csname res@val\first@@\endcsname{\the\temp@count}
         \expandafter\xdef\csname res@num\first@@\endcsname{0}
         \xdef\nineth@{y}
      \else
         \expandafter\xdef\csname res@val\first@@\endcsname{0}
      \fi
      \advance\outerloopcount by 1
   \ifnum\outerloopcount>90\else\repeat
   \expandafter\xdef\csname res@val\d@t\endcsname{0}
   \expandafter\xdef\csname res@num\d@t\endcsname{0}
   \ifx\nineth@\y@
      \xdef\tmpstack{&:&,}
      \outerloopcount=65
      \loop
         \xdef\first@@{\csname ch@r@\the\outerloopcount\endcsname}
         \xdef\second@@{\csname res@val\first@@\endcsname}
         \ifnum\second@@>0
            \xdef\third@{\tmpstack @} \xdef\tmpstack{}
            \sort@logostack
         \fi
         \advance\outerloopcount by 1
      \ifnum\outerloopcount>90
         \xdef\stack@sequencelogo{\stack@sequencelogo\tmpstack&:&,}
      \else\repeat
   \else
      \xdef\stack@sequencelogo{\stack@sequencelogo O:0,&:&,}
   \fi
}
\def\get@fromlogostack#1:#2,#3@{%
   \xdef\first@{#1} \xdef\second@{#2} \xdef\third@{#3}
}
\def\sort@logostack{%
   \expandafter\get@fromlogostack\third@
   \ifx\first@\ampers@nd 
      \xdef\tmpstack{\tmpstack\first@@:\second@@,}
   \else
   \ifnum\second@@<0 
      \ifnum\second@>0 
         \xdef\tmpstack{\tmpstack\first@@:\second@@,\first@:\second@,\third@}
      \else       
         \temp@count=\second@@\relax \multiply\temp@count by -1\relax  
         \triple@count=\second@\relax \multiply\triple@count by -1\relax  
         \ifnum\temp@count<\triple@count
            \xdef\tmpstack{\tmpstack\first@@:\second@@,\first@:\second@,\third@}
         \else
            \xdef\tmpstack{\tmpstack\first@:\second@,}
            \xdef\third@@{\third@ .}
            \ifx\third@@\d@t
               \xdef\third@{&:&,@}
            \else 
               \xdef\third@{\third@ @}
            \fi
            \sort@logostack
         \fi
      \fi
   \else
      \ifnum\second@@<\second@ 
         \xdef\tmpstack{\tmpstack\first@@:\second@@,\first@:\second@,\third@}
      \else
         \xdef\tmpstack{\tmpstack\first@:\second@,}
         \xdef\third@@{\third@ .}
         \ifx\third@@\d@t
            \xdef\third@{&:&,@}
         \else 
            \xdef\third@{\third@ @}
         \fi
         \sort@logostack
      \fi
   \fi
   \fi
}


\def\calc@feature{%
%      \advance\pos@count by -1
      \loopcount=-1 \innerloopcount=0
      \loop
         \advance\loopcount by 1
         \ifnum\loopcount=0 \xdef\first@@{\the\cons@count}
         \else \xdef\first@@{\csname res@count\the\loopcount\endcsname} \fi
         \expandafter\ifx\csname start\bottop@\the\loopcount\endcsname\ampers@nd
         \else
         \expandafter\ifnum\csname start\bottop@\the\loopcount\endcsname>\first@@
         \else
            \expandafter\ifnum\csname stop\bottop@\the\loopcount\endcsname<\first@@
               \expandafter\getregion@fromfstack{\the\loopcount}
            \else
               \innerloopcount=\loopcount
               \expandafter\ifnum\csname featureon\bottop@\endcsname=0
                  \expandafter\xdef\csname featureon\bottop@\endcsname{1}
                  \expandafter\xdef\csname feature@\bottop@\endcsname{1}
                  \expandafter\xdef\csname ftext\bottop@\the\loopcount\endcsname{%
                                   \csname text\bottop@\the\loopcount\endcsname}
                  \expandafter\xdef\csname fstyle\bottop@\the\loopcount\endcsname{%
                                   \csname style\bottop@\the\loopcount\endcsname}
                  \innerloopcount=\pos@count
                  \advance\innerloopcount by -\csname featurepos\bottop@\endcsname
                  \expandafter\xdef\csname textfeature\bottop@\endcsname{%
                                   \csname textfeature\bottop@\endcsname%
                                   &\the\innerloopcount;{};}
                  \expandafter\xdef\csname stylefeature\bottop@\endcsname{%
                                   \csname stylefeature\bottop@\endcsname%
                                   &\the\innerloopcount;&;}
                  \expandafter\xdef\csname featurepos\bottop@\endcsname{\the\pos@count}
                  \xdef\temp@@@{n}
                  \xdef\fourth@{}
                  \xdef\temp@{\csname ftext\bottop@\the\loopcount\endcsname}
                  \xdef\temp@{\temp@[,]:[,][]&}\expandafter\graph@opt@color\temp@
                  \ifx\fourth@\tr@ns \xdef\temp@@@{y} \fi
                  \xdef\fourth@{}
                  \xdef\temp@{\csname fstyle\bottop@\the\loopcount\endcsname}
                  \xdef\temp@{\temp@[,]:[,][]&}\expandafter\graph@opt@color\temp@
                  \ifx\fourth@\tr@ns \xdef\temp@@@{y} \fi
                  \ifx\temp@@@\y@
                    \ifnum\loopcount=0
                     \message{<No translations of the consensus sequence>}
                     \expandafter\xdef\csname ftext\bottop@\the\loopcount\endcsname{%
                                  No consensus translations!}
                     \expandafter\xdef\csname fstyle\bottop@\the\loopcount\endcsname{%
                                          ///}
                    \else
                     \if\seq@type P
                        \expandafter\xdef\csname collect@res\bottop@\endcsname{yes}
                        \expandafter\xdef\csname tr@nsseq\bottop@\endcsname{\the\loopcount}
                        \expandafter\xdef\csname tr@nslate\bottop@\endcsname{%
                        \csname res\the\loopcount\endcsname}
                     \else
                        \expandafter\xdef\csname collect@res\bottop@\endcsname{yes}
                        \expandafter\xdef\csname tr@nsseq\bottop@\endcsname{\the\loopcount}
                        \expandafter\ifx\csname res\the\loopcount\endcsname\gap@char
                           \expandafter\xdef\csname triple@count\bottop@\endcsname{0}
                           \expandafter\xdef\csname tr@nslate\bottop@\endcsname{-}
                        \else
                           \expandafter\xdef\csname triple@count\bottop@\endcsname{1}
                           \expandafter\xdef\csname tr@nslate\bottop@\endcsname{%
                           \csname res\the\loopcount\endcsname}
                           \expandafter\xdef\csname last@@res\bottop@\endcsname{%
                           \csname res\the\loopcount\endcsname}
                        \fi
                     \fi
                    \fi
                  \fi
                  \xdef\temp@{plot}
                  \ifx\temp@\fourth@
                    \ifnum\loopcount=0
                     \message{<No bar graphs/scales with the consensus sequence>}
                     \expandafter\xdef\csname ftext\bottop@\the\loopcount\endcsname{}
                     \expandafter\xdef\csname fstyle\bottop@\the\loopcount\endcsname{%
                                          ///}
                    \else
                     \expandafter\xdef\csname collect@val\bottop@\endcsname{yes}
                     \expandafter\xdef\csname v@lseq\bottop@\endcsname{\the\loopcount}
                     \expandafter\xdef\csname ffourth@\bottop@\endcsname{\ffourth@}
                     \xdef\temp@{\ffourth@\csname res\the\loopcount\endcsname}
                     \expandafter\xdef\csname v@l\bottop@\endcsname{\csname \temp@\endcsname}
                    \fi
                  \fi
                  \xdef\temp@{cons}
                  \ifx\temp@\fourth@
                     \expandafter\xdef\csname collect@cons@graph\bottop@\endcsname{yes}
                     \expandafter\xdef\csname v@lseq\bottop@\endcsname{\the\loopcount}
                     \expandafter\xdef\csname ffourth@\bottop@\endcsname{\ffourth@}
                     \innerloopcount=1
                     \collect@cons@res
                     \expandafter\xdef\csname v@l\bottop@\endcsname{\cons@val}
                  \fi
               \else
                  \ifnum\pos@count=1\relax
                     \xdef\temp@{\csname fstyle\bottop@\the\loopcount\endcsname @}
                     \expandafter\getstyle@right\temp@
                     \xdef\temp@@@{n}
                     \xdef\fourth@{}
                     \xdef\temp@{\csname ftext\bottop@\the\loopcount\endcsname}
                     \xdef\temp@{\temp@[,]:[,][]&}\expandafter\graph@opt@color\temp@
                     \ifx\fourth@\tr@ns \xdef\temp@@@{y} \fi
                     \xdef\fourth@{}
                     \xdef\temp@{\csname fstyle\bottop@\the\loopcount\endcsname}
                     \xdef\temp@{\temp@[,]:[,][]&}\expandafter\graph@opt@color\temp@
                     \ifx\fourth@\tr@ns \xdef\temp@@@{y} \fi
                     \ifx\temp@@@\y@
                        \if\seq@type P
                           \expandafter\xdef\csname tr@nslate\bottop@\endcsname{%
                           \csname res\the\loopcount\endcsname}
                        \else
                           \expandafter\ifnum\csname triple@count\bottop@\endcsname=2
                              \expandafter\ifx\csname res\the\loopcount\endcsname\gap@char
                              \else
                                 \expandafter\xdef\csname tr@nslate\bottop@\endcsname{%
                                 +\csname last@res\bottop@\endcsname%
                                 \csname tr@nslate\bottop@\endcsname}
                              \fi
                           \fi
                           \expandafter\ifnum\csname triple@count\bottop@\endcsname=1
                              \expandafter\xdef\csname last@res\bottop@\endcsname{%
                              \csname last@@res\bottop@\endcsname}
                              \expandafter\ifx\csname res\the\loopcount\endcsname\gap@char
                                 \expandafter\xdef\csname tr@nslate\bottop@\endcsname{%
                                  +\csname last@res\bottop@\endcsname%
                                  \csname tr@nslate\bottop@\endcsname}
                              \fi
                           \fi
                        \fi
                     \fi
                     \xdef\temp@{plot}
                     \ifx\temp@\fourth@
                        \expandafter\ifx\csname res\the\loopcount\endcsname\gap@char
                           \expandafter\xdef\csname v@l\bottop@\endcsname{N}
                        \else
                           \xdef\temp@{\ffourth@\csname res\the\loopcount\endcsname}
                           \expandafter\xdef\csname v@l\bottop@\endcsname{%
                           \csname \temp@\endcsname}
                        \fi
                     \fi 
                     \xdef\temp@{cons}
                     \ifx\temp@\fourth@
                        \innerloopcount=1
                        \collect@cons@res
                        \expandafter\xdef\csname v@l\bottop@\endcsname{\cons@val}
                     \fi
                  \fi
               \fi
               \expandafter\ifnum\csname stop\bottop@\the\loopcount\endcsname=\first@@
                  \expandafter\getregion@fromfstack{\the\loopcount}
                  \xdef\temp@@@{n}
                  \xdef\fourth@{}
                  \xdef\temp@{\csname ftext\bottop@\the\loopcount\endcsname}
                  \xdef\temp@{\temp@[,]:[,][]&}\expandafter\graph@opt@color\temp@
                  \ifx\fourth@\tr@ns
                     \ifx\f@color\comm@
                        \xdef\f@color{} \else \xdef\f@color{[\f@color]}
                     \fi
                     \xdef\tr@nsl@ted{\csname tr@nslate\bottop@\endcsname}
                     \if\seq@type N \do@translation \else \rev@translation \fi
                     \xdef\temp@{translate:\tr@nsl@ted\f@color}
                     \xdef\temp@@@{y}
                  \else
                     \xdef\temp@{\csname ftext\bottop@\the\loopcount\endcsname}
                  \fi
                  \xdef\fourth@{}
                  \xdef\temp@@{\csname fstyle\bottop@\the\loopcount\endcsname}
                  \xdef\temp@@{\temp@@[,]:[,][]&}\expandafter\graph@opt@color\temp@@
                  \ifx\fourth@\tr@ns
                     \ifx\f@color\comm@
                         \xdef\f@color{} \else \xdef\f@color{[\f@color]}
                     \fi
                     \xdef\tr@nsl@ted{\csname tr@nslate\bottop@\endcsname}
                     \if\seq@type N \do@translation \else \rev@translation \fi
                     \xdef\temp@@{translate:\tr@nsl@ted\f@color}
                     \xdef\temp@@@{y}
                  \else
                     \xdef\temp@@{plot}
                     \ifx\temp@@\fourth@
                        \xdef\temp@@{plot\f@color[\fffourth@]:\csname v@l\bottop@\endcsname[\ff@color]}
                        \expandafter\xdef\csname collect@val\bottop@\endcsname{no}
                        \expandafter\xdef\csname v@l\bottop@\endcsname{}
                     \else
                     \xdef\temp@@{cons}
                     \ifx\temp@@\fourth@
                        \xdef\temp@@{plot\f@color[\fffourth@]:\csname v@l\bottop@\endcsname[\ff@color]}
                        \expandafter\xdef\csname collect@cons@graph\bottop@\endcsname{no}
                        \expandafter\xdef\csname v@l\bottop@\endcsname{}
                     \else
                        \xdef\temp@@{\csname fstyle\bottop@\the\loopcount\endcsname}
                     \fi
                     \fi
                  \fi
                  \ifx\temp@@@\y@
                     \expandafter\xdef\csname collect@res\bottop@\endcsname{no}
                     \expandafter\xdef\csname tr@nslate\bottop@\endcsname{}
                  \fi
                  \expandafter\ifnum\csname featureon\bottop@\endcsname=1
                     \expandafter\xdef\csname featureon\bottop@\endcsname{0}
                     \innerloopcount=\pos@count \advance\innerloopcount by 1
                     \advance\innerloopcount by -\csname featurepos\bottop@\endcsname
                     \expandafter\xdef\csname textfeature\bottop@\endcsname{%
                                      \csname textfeature\bottop@\endcsname%
                                      &\the\innerloopcount;\temp@;}
                     \expandafter\xdef\csname stylefeature\bottop@\endcsname{%
                                      \csname stylefeature\bottop@\endcsname%
                                      &\the\innerloopcount;\temp@@;}
                     \innerloopcount=\pos@count \advance\innerloopcount by 1
                     \expandafter\xdef\csname %
                     featurepos\bottop@\endcsname{\the\innerloopcount}
                  \fi
               \fi
               \ifnum\pos@count=\res@perline
                  \expandafter\ifnum\csname featureon\bottop@\endcsname=1
                     \innerloopcount=\pos@count
                     \advance\innerloopcount by 1
                     \advance\innerloopcount by -\csname featurepos\bottop@\endcsname
                     \xdef\temp@{\csname fstyle\bottop@\the\loopcount\endcsname @}
                     \expandafter\getstyle@left\temp@
                     \xdef\temp@@@{n}
                     \xdef\fourth@{}
                     \xdef\temp@{\csname ftext\bottop@\the\loopcount\endcsname}
                     \xdef\temp@{\temp@[,]:[,][]&}\expandafter\graph@opt@color\temp@
                     \ifx\fourth@\tr@ns
                        \ifx\f@color\comm@
                           \xdef\f@color{} \else \xdef\f@color{[\f@color]}
                        \fi
                        \expandafter\ifnum\csname triple@count\bottop@\endcsname=2
                           \if\seq@type N
                              \xdef\temp@{\csname sequence\the\loopcount\endcsname:}
                              \expandafter\get@nextres\temp@
                              \xdef\tr@nsl@ted{\csname tr@nslate\bottop@\endcsname\temp@}
                              \do@translation
                              \xdef\temp@@@{2}
                           \else
                              \xdef\tr@nsl@ted{\csname tr@nslate\bottop@\endcsname}
                              \rev@translation
                             \xdef\temp@@@{}
                           \fi
                           \xdef\temp@{translate:\tr@nsl@ted\f@color}
                        \else
                           \xdef\tr@nsl@ted{\csname tr@nslate\bottop@\endcsname}
                           \if\seq@type N \do@translation \else \rev@translation \fi
                           \xdef\temp@{translate:\tr@nsl@ted\f@color}
                           \xdef\temp@@@{}
                        \fi
                     \else \xdef\temp@{\csname ftext\bottop@\the\loopcount\endcsname}
                     \fi
                     \xdef\fourth@{}
                     \xdef\temp@@{\csname fstyle\bottop@\the\loopcount\endcsname}
                     \xdef\temp@@{\temp@@[,]:[,][]&}\expandafter\graph@opt@color\temp@@
                     \ifx\fourth@\tr@ns
                        \ifx\f@color\comm@
                           \xdef\f@color{} \else \xdef\f@color{[\f@color]}
                        \fi
                        \expandafter\ifnum\csname triple@count\bottop@\endcsname=2
                           \if\seq@type N
                              \xdef\temp@@{\csname sequence\the\loopcount\endcsname:}
                              \expandafter\get@@nextres\temp@@
                              \xdef\tr@nsl@ted{\csname tr@nslate\bottop@\endcsname\temp@@}
                              \do@translation
                              \xdef\temp@@@{2}
                           \else
                              \xdef\tr@nsl@ted{\csname tr@nslate\bottop@\endcsname}
                              \rev@translation
                              \xdef\temp@@@{}
                           \fi
                           \xdef\temp@@{translate:\tr@nsl@ted\f@color}
                        \else
                           \xdef\tr@nsl@ted{\csname tr@nslate\bottop@\endcsname}
                           \if\seq@type N \do@translation \else \rev@translation \fi
                           \xdef\temp@@{translate:\tr@nsl@ted\f@color}
                           \xdef\temp@@@{}
                        \fi
                     \else
                        \xdef\temp@@{plot}
                        \ifx\temp@@\fourth@
                          \xdef\temp@@{plot\f@color[\fffourth@]:\csname v@l\bottop@\endcsname[\ff@color]}
                        \else
                        \xdef\temp@@{cons}
                        \ifx\temp@@\fourth@
                           \xdef\temp@@{plot\f@color[\fffourth@]:\csname v@l\bottop@\endcsname[\ff@color]}
                        \else
                           \xdef\temp@@{\style@@}
                        \fi
                        \fi
                     \fi
                     \ifx\temp@@@\n@
                     \else
                        \expandafter\xdef\csname tr@nslate\bottop@\endcsname{\temp@@@}
                     \fi
                     \expandafter\xdef\csname textfeature\bottop@\endcsname{%
                                      \csname textfeature\bottop@\endcsname%
                                      &\the\innerloopcount;\temp@;}
                     \expandafter\xdef\csname stylefeature\bottop@\endcsname{%
                                      \csname stylefeature\bottop@\endcsname%
                                      &\the\innerloopcount;\temp@@;}
                     \innerloopcount=\pos@count \advance\innerloopcount by 1
                     \expandafter\xdef\csname %
                     featurepos\bottop@\endcsname{\the\innerloopcount}
                  \fi
               \fi
            \fi
         \fi
         \fi
      \ifnum\loopcount<\seq@count \repeat
%      \advance\pos@count by 1
}

\def\add@to@rule@tens{%
    \advance\innerloopcount by \ruler@step\relax
    \ifnum\innerloopcount=0 
          \ifx\allow@zero\n@ \innerloopcount=\ruler@step \fi
    \fi
    \expandafter\ifnum\csname res@count\rule@num\endcsname>\innerloopcount \add@to@rule@tens
    \else
       \xdef\rule@tens{\the\innerloopcount}
    \fi
}


\def\c@nsensus{%
 \ifnum\pos@count>\res@perline
    \else
    \loopcount=0
    \ifx\g@p\n@
       \global\advance\cons@count by 1\relax
    \fi
    \ifnum\cons@count=0\relax
       \ifx\allow@zero\n@ \global\advance\cons@count by 1 \fi
    \fi
    \ifT@coffee 
            \xdef\TC@line{\csname T@coffee0\endcsname}
            \expandafter\TC@get\TC@line
    \fi
    \expandafter\xdef\csname res@count0\endcsname{\the\cons@count}
    \loop
         \advance\loopcount by 1
         \ifT@coffee
            \xdef\TC@line{\csname T@coffee\the\loopcount\endcsname}
            \expandafter\TC@get\TC@line
            \xdef\TC@first@{\csname TC@num\the\loopcount\endcsname}
         \fi
         \xdef\seq@line{\csname sequence\the\loopcount\endcsname}
         \expandafter\residue@get\seq@line
         \xdef\first@{\csname res\the\loopcount\endcsname}
         \innerloopcount=\csname res@num\first@\endcsname
         \advance\innerloopcount by 1
         \expandafter\xdef\csname res@num\first@\endcsname{\the\innerloopcount}
         \expandafter\check@char\first@
         \ifx\dom@in\y@ %%%%%%%%%%%*****
                \xdef\temp@{\csname dom@num\the\loopcount\endcsname}
                \expandafter\get@dom@count\temp@
                \expandafter\xdef\csname dom@num\the\loopcount\endcsname{\temp@}    
         \fi            %%%%%%%%%%%*****
         \ifletter
             \ifx\dom@in\y@
             \else
                \res@count=\csname res@count\the\loopcount\endcsname
                \advance\res@count by 1
                \ifnum\res@count=0\relax
                   \ifx\allow@zero\n@ \advance\res@count by 1 \fi
             \fi
             \fi
             \expandafter\xdef\csname res@count\the\loopcount\endcsname{\the\res@count}
             \ifnum\loopcount=\exp@rt@num \xdef\sixth@{\the\res@count}\xdef\seventh@{y}\fi
             \ifnum\loopcount=\rule@num\relax 
              \expandafter\ifnum\csname res@count\rule@num\endcsname>\rule@tens 
                 \innerloopcount=\rule@tens \add@to@rule@tens
              \fi  
              \expandafter\ifnum\csname res@count\rule@num\endcsname=\rule@tens
                \expandafter\ifx\csname alt@ruler\rule@tens\endcsname\relax
                   \xdef\temp@{\rule@tens}
                \else
                   \xdef\temp@{\csname alt@ruler\rule@tens\endcsname}
                \fi
                \xdef\ruler@{\ruler@ !<\temp@>}
              \else
                \xdef\ruler@{\ruler@ -}
              \fi
              \xdef\temp@{\csname res@count\rule@num\endcsname}
              \expandafter\ifx\csname alt@ruler\temp@\endcsname\relax
              \else
                \expandafter\xdef\csname alt@ruler\temp@\endcsname{\temp@}
              \fi
             \fi
             \ifx\collect@valtop\yes
             \ifnum\v@lseqtop=\loopcount
                \xdef\v@ltop{\v@ltop,\csname \ffourth@top\first@\endcsname}
             \fi\fi
             \ifx\collect@valttop\yes
             \ifnum\v@lseqttop=\loopcount
                \xdef\v@lttop{\v@lttop,\csname \ffourth@ttop\first@\endcsname}
             \fi\fi
             \ifx\collect@valtttop\yes
             \ifnum\v@lseqtttop=\loopcount
                \xdef\v@ltttop{\v@ltttop,\csname \ffourth@tttop\first@\endcsname}
             \fi\fi
             \ifx\collect@valttttop\yes
             \ifnum\v@lseqttttop=\loopcount
                \xdef\v@lttttop{\v@lttttop,\csname \ffourth@ttttop\first@\endcsname}
             \fi\fi
             \ifx\collect@valbottom\yes
             \ifnum\v@lseqbottom=\loopcount
                \xdef\v@lbottom{\v@lbottom,\csname \ffourth@bottom\first@\endcsname}
             \fi\fi
             \ifx\collect@valbbottom\yes
             \ifnum\v@lseqbbottom=\loopcount
                \xdef\v@lbbottom{\v@lbbottom,\csname \ffourth@bbottom\first@\endcsname}
             \fi\fi
             \ifx\collect@valbbbottom\yes
             \ifnum\v@lseqbbbottom=\loopcount
                \xdef\v@lbbbottom{\v@lbbbottom,\csname \ffourth@bbbottom\first@\endcsname}
             \fi\fi
             \ifx\collect@valbbbbottom\yes
             \ifnum\v@lseqbbbbottom=\loopcount
                \xdef\v@lbbbbottom{\v@lbbbbottom,\csname \ffourth@bbbbottom\first@\endcsname}
             \fi\fi
             \ifx\collect@restop\yes
             \ifnum\tr@nsseqtop=\loopcount
                \xdef\last@restop{\last@@restop}
                \xdef\last@@restop{\first@}
                \xdef\tr@nslatetop{\tr@nslatetop\first@}
                \innerloopcount=\triple@counttop
                \advance\innerloopcount by 1
                \ifnum\innerloopcount>3 \innerloopcount=1 \fi
                \xdef\triple@counttop{\the\innerloopcount}
             \fi\fi
             \ifx\collect@resttop\yes
             \ifnum\tr@nsseqttop=\loopcount
                \xdef\last@resttop{\last@@resttop}
                \xdef\last@@resttop{\first@}
                \xdef\tr@nslatettop{\tr@nslatettop\first@}
                \innerloopcount=\triple@countttop
                \advance\innerloopcount by 1
                \ifnum\innerloopcount>3 \innerloopcount=1 \fi
                \xdef\triple@countttop{\the\innerloopcount}
             \fi\fi
             \ifx\collect@restttop\yes
             \ifnum\tr@nsseqtttop=\loopcount
                \xdef\last@restttop{\last@@restttop}
                \last@@restttop{\first@}
                \xdef\tr@nslatetttop{\tr@nslatetttop\first@}
                \innerloopcount=\triple@counttttop
                \advance\innerloopcount by 1
                \ifnum\innerloopcount>3 \innerloopcount=1 \fi
                \xdef\triple@counttttop{\the\innerloopcount}
             \fi\fi
             \ifx\collect@resttttop\yes
             \ifnum\tr@nsseqttttop=\loopcount
                \xdef\last@resttttop{\last@@resttttop}
                \xdef\last@@resttttop{\first@}
                \xdef\tr@nslatettttop{\tr@nslatettttop\first@}
                \innerloopcount=\triple@countttttop
                \advance\innerloopcount by 1
                \ifnum\innerloopcount>3 \innerloopcount=1 \fi
                \xdef\triple@countttttop{\the\innerloopcount}
             \fi\fi
             \ifx\collect@resbottom\yes
             \ifnum\tr@nsseqbottom=\loopcount
                 \xdef\last@resbottom{\last@@resbottom}
                 \xdef\last@@resbottom{\first@}
                 \xdef\tr@nslatebottom{\tr@nslatebottom\first@}
                 \innerloopcount=\triple@countbottom
                 \advance\innerloopcount by 1
                 \ifnum\innerloopcount>3 \innerloopcount=1 \fi
                 \xdef\triple@countbottom{\the\innerloopcount}
             \fi\fi
             \ifx\collect@resbbottom\yes
             \ifnum\tr@nsseqbbottom=\loopcount
                 \xdef\last@resbbottom{\last@@resbbottom}
                 \xdef\last@@resbbottom{\first@}
                 \xdef\tr@nslatebbottom{\tr@nslatebbottom\first@}
                 \innerloopcount=\triple@countbbottom
                 \advance\innerloopcount by 1
                 \ifnum\innerloopcount>3 \innerloopcount=1 \fi
                 \xdef\triple@countbbottom{\the\innerloopcount}
             \fi\fi
             \ifx\collect@resbbbottom\yes
             \ifnum\tr@nsseqbbbottom=\loopcount
                 \xdef\last@resbbbottom{\last@@resbbbottom}
                 \xdef\last@@resbbbottom{\first@}
                 \xdef\tr@nslatebbbottom{\tr@nslatebbbottom\first@}
                 \innerloopcount=\triple@countbbbottom
                 \advance\innerloopcount by 1
                 \ifnum\innerloopcount>3 \innerloopcount=1 \fi
                 \xdef\triple@countbbbottom{\the\innerloopcount}
             \fi\fi
             \ifx\collect@resbbbbottom\yes
             \ifnum\tr@nsseqbbbbottom=\loopcount
                 \xdef\last@resbbbbottom{\last@@resbbbbottom}
                 \xdef\last@@resbbbbottom{\first@}
                 \xdef\tr@nslatebbbbottom{\tr@nslatebbbbottom\first@}
                 \innerloopcount=\triple@countbbbbottom
                 \advance\innerloopcount by 1
                 \ifnum\innerloopcount>3 \innerloopcount=1 \fi
                 \xdef\triple@countbbbbottom{\the\innerloopcount}
             \fi\fi
          \innerloopcount=\csname mol@weight\the\loopcount\endcsname
          \advance\innerloopcount by \csname \prefix@ mw\first@\endcsname
          \expandafter\xdef\csname mol@weight\the\loopcount\endcsname{%
                           \the\innerloopcount}
          \innerloopcount=\csname ch@rge\the\loopcount\endcsname
          \advance\innerloopcount by \csname pepcharge\first@\endcsname
          \expandafter\xdef\csname ch@rge\the\loopcount\endcsname{%
                           \the\innerloopcount}
         \else
          \ifnum\loopcount=\rule@num \xdef\ruler@{\ruler@ -} \fi
          \ifx\collect@valtop\yes
          \ifnum\v@lseqtop=\loopcount
             \xdef\v@ltop{\v@ltop,N}
          \fi\fi
          \ifx\collect@valttop\yes
          \ifnum\v@lseqttop=\loopcount
             \xdef\v@lttop{\v@lttop,N}
          \fi\fi
          \ifx\collect@valtttop\yes
          \ifnum\v@lseqtttop=\loopcount
             \xdef\v@ltttop{\v@ltttop,N}
          \fi\fi
          \ifx\collect@valttttop\yes
          \ifnum\v@lseqttttop=\loopcount
             \xdef\v@lttttop{\v@lttttop,N}
          \fi\fi
          \ifx\collect@valbottom\yes
          \ifnum\v@lseqbottom=\loopcount
             \xdef\v@lbottom{\v@lbottom,N}
          \fi\fi
          \ifx\collect@valbbottom\yes
          \ifnum\v@lseqbbottom=\loopcount
             \xdef\v@lbbottom{\v@lbbottom,N}
          \fi\fi
          \ifx\collect@valbbbottom\yes
          \ifnum\v@lseqbbbottom=\loopcount
             \xdef\v@lbbbottom{\v@lbbbottom,N}
          \fi\fi
          \ifx\collect@valbbbbottom\yes
          \ifnum\v@lseqbbbbottom=\loopcount
             \xdef\v@lbbbbottom{\v@lbbbbottom,N}
          \fi\fi
          \ifx\collect@restop\yes
          \ifnum\tr@nsseqtop=\loopcount
              \xdef\tr@nslatetop{\tr@nslatetop -}
          \fi\fi
          \ifx\collect@resttop\yes
          \ifnum\tr@nsseqttop=\loopcount
              \xdef\tr@nslatettop{\tr@nslatettop -}
          \fi\fi
          \ifx\collect@restttop\yes
          \ifnum\tr@nsseqtttop=\loopcount
              \xdef\tr@nslatetttop{\tr@nslatetttop -}
          \fi\fi
          \ifx\collect@resttttop\yes
          \ifnum\tr@nsseqttttop=\loopcount
              \xdef\tr@nslatettttop{\tr@nslatettttop -}
          \fi\fi
          \ifx\collect@resbottom\yes
          \ifnum\tr@nsseqbottom=\loopcount
              \xdef\tr@nslatebottom{\tr@nslatebottom -}
          \fi\fi
          \ifx\collect@resbbottom\yes
          \ifnum\tr@nsseqbbottom=\loopcount
              \xdef\tr@nslatebbottom{\tr@nslatebbottom -}
          \fi\fi
          \ifx\collect@resbbbottom\yes
          \ifnum\tr@nsseqbbbottom=\loopcount
              \xdef\tr@nslatebbbottom{\tr@nslatebbbottom -}
          \fi\fi
          \ifx\collect@resbbbbottom\yes
          \ifnum\tr@nsseqbbbbottom=\loopcount
              \xdef\tr@nslatebbbbottom{\tr@nslatebbbbottom -}
          \fi\fi
         \fi
    \ifnum\loopcount<\seq@count \repeat
         \ifx\collect@cons@graphtop\yes
                \outerloopcount=1\relax
                \temp@count=0\relax
                \sum@up@cons
                \xdef\v@ltop{\v@ltop,\cons@val}
         \fi
         \ifx\collect@cons@graphttop\yes
                \outerloopcount=1\relax
                \temp@count=0\relax
                \sum@up@cons
                \xdef\v@lttop{\v@lttop,\cons@val}
         \fi
         \ifx\collect@cons@graphtttop\yes
                \outerloopcount=1\relax
                \temp@count=0\relax
                \sum@up@cons
                \xdef\v@ltttop{\v@ltttop,\cons@val}
         \fi
         \ifx\collect@cons@graphttttop\yes
                \outerloopcount=1\relax
                \temp@count=0\relax
                \sum@up@cons
                \xdef\v@lttttop{\v@lttttop,\cons@val}
         \fi
         \ifx\collect@cons@graphbottom\yes
                \outerloopcount=1\relax
                \temp@count=0\relax
                \sum@up@cons
                \xdef\v@lbottom{\v@lbottom,\cons@val}
         \fi
         \ifx\collect@cons@graphbbottom\yes
                \outerloopcount=1\relax
                \temp@count=0\relax
                \sum@up@cons
                \xdef\v@lbbottom{\v@lbbottom,\cons@val}
         \fi
         \ifx\collect@cons@graphbbbottom\yes
                \outerloopcount=1\relax
                \temp@count=0\relax
                \sum@up@cons
                \xdef\v@lbbbottom{\v@lbbbottom,\cons@val}
         \fi
         \ifx\collect@cons@graphbbbbottom\yes
                \outerloopcount=1\relax
                \temp@count=0\relax
                \sum@up@cons
                \xdef\v@lbbbbottom{\v@lbbbbottom,\cons@val}
         \fi
         \ifx\collect@cons@colors\y@
                \outerloopcount=1\relax
                \temp@count=0\relax
                \sum@up@cons
                \outerloopcount=\cons@val
                \advance\outerloopcount by 4
                \divide\outerloopcount by 5
                \multiply\outerloopcount by 5
                \ifnum\outerloopcount<5\relax\outerloopcount=5\fi
                \xdef\c@nscol{\c@nscol\the\outerloopcount,}
         \fi
         \ifx\exp@rt\y@
             \ifx\seventh@\y@
                \outerloopcount=1\relax
                \temp@count=0\relax
                \sum@up@cons
                \outerloopcount=\cons@val
                \advance\outerloopcount by 4
                \divide\outerloopcount by 5
                \multiply\outerloopcount by 5
                \ifnum\outerloopcount<5\relax\outerloopcount=5\fi
                \xdef\seventh@{n}
                \immediate\write\exp@rtfile{\string color col\the\outerloopcount, resi \sixth@}
             \fi
         \fi

    \xdef\g@p{n}
    \ifregionalshadenow \calc@regshade \fi
    \ifregionaltintnow  \calc@regtint  \fi
    \ifregionalemphnow  \calc@regemph  \fi
    \ifregionallowernow \calc@reglower \fi
    \ifframenow         \calc@frame    \fi
    \ifshadingnow\iffuncmode\else\ifT@coffee\else   \calc@shading \fi\fi\fi
    \iftopfeaturenow       \xdef\bottop@{top}       \calc@feature \fi
    \ifttopfeaturenow      \xdef\bottop@{ttop}      \calc@feature \fi
    \iftttopfeaturenow     \xdef\bottop@{tttop}     \calc@feature \fi
    \ifttttopfeaturenow    \xdef\bottop@{ttttop}    \calc@feature \fi
    \ifbottomfeaturenow    \xdef\bottop@{bottom}    \calc@feature \fi
    \ifbbottomfeaturenow   \xdef\bottop@{bbottom}   \calc@feature \fi
    \ifbbbottomfeaturenow  \xdef\bottop@{bbbottom}  \calc@feature \fi
    \ifbbbbottomfeaturenow \xdef\bottop@{bbbbottom} \calc@feature \fi
    \ifnum\seq@count>1 \innerloopcount=1 \collect@similarity \fi
    \advance\pos@count by 1 
     \ifshow@logo         \calc@logo     \fi
     \ifall@fshade \all@funcshade
     \else
      \ifT@coffee \T@coffee@shade
      \else
       \ifnum\cons@num>0 \loopcount=\cons@num \else \loopcount=\seq@count \fi
       \xdef\match@case{0} \xdef\m@x{1}
       \iffuncmode
          \xdef\prfx{\prefix@} \xdef\prefix@{func} \xdef\c@se{3} \check@sim
          \xdef\prefix@{\prfx}
       \else \xdef\c@se{1} \check@ident \fi
       \ifcase\match@case \unc@nserved \or \c@nserved \or \allm@tch \else \functi@nal \fi
      \fi
     \fi
     \ifnum\rule@num=0 
         \ifx\g@p\y@  
           \xdef\ruler@{\ruler@ -}
         \else
           \ifnum\cons@count=\rule@tens
              \expandafter\ifx\csname alt@ruler\rule@tens\endcsname\relax
                 \xdef\temp@{\rule@tens}
              \else
                 \xdef\temp@{\csname alt@ruler\rule@tens\endcsname}
              \fi
              \xdef\ruler@{\ruler@ !<\temp@>}
              \innerloopcount=\rule@tens \advance\innerloopcount by \ruler@step\relax
              \ifnum\innerloopcount=0 
                 \ifx\allow@zero\n@ \innerloopcount=\ruler@step \fi
              \fi
              \xdef\rule@tens{\the\innerloopcount}
           \else
              \xdef\ruler@{\ruler@ -}
           \fi 
           \expandafter\ifx\csname alt@ruler\the\cons@count\endcsname\relax
           \else
             \expandafter\xdef\csname alt@ruler\the\cons@count\endcsname{\the\cons@count}
           \fi
         \fi
     \fi 
    \expandafter\ifnum\csname res@count\start@seq\endcsname<\end@num\relax 
      \c@nsensus 
    \else 
      \global\stop@true
      \loopcount=\pos@count \advance\loopcount by -1 \relax
      \res@perline=\loopcount
      \iftopfeature       \xdef\bottop@{top}       \calc@feature \fi
      \ifttopfeature      \xdef\bottop@{ttop}      \calc@feature \fi
      \iftttopfeature     \xdef\bottop@{tttop}     \calc@feature \fi
      \ifttttopfeature    \xdef\bottop@{ttttop}    \calc@feature \fi
      \ifbottomfeature    \xdef\bottop@{bottom}    \calc@feature \fi
      \ifbbottomfeature   \xdef\bottop@{bbottom}   \calc@feature \fi
      \ifbbbottomfeature  \xdef\bottop@{bbbottom}  \calc@feature \fi
      \ifbbbbottomfeature \xdef\bottop@{bbbbottom} \calc@feature \fi
      \pos@count=0
    \fi
 \fi}
 
\def\prep@reexp@rtfile{
   \loopcount=0
   \loop
      \advance\loopcount by 5
      \immediate\write\exp@rtfile{\string set_color col\the\loopcount, \csname\c@nsc@l\the\loopcount\endcsname}
   \ifnum\loopcount>95\else\repeat
}

\def\c@unt{%
       \advance\loopcount by 1
       \ifT@coffee
          \xdef\seq@line{\csname T@coffee\the\loopcount\endcsname}
          \expandafter\TC@get\seq@line
       \fi
       \xdef\seq@line{\csname sequence\the\loopcount\endcsname}
       \expandafter\residue@get\seq@line
       \xdef\first@{\csname res\the\loopcount\endcsname}
       \expandafter\check@char\first@
       \ifletter
           \res@count=\csname res@count\the\loopcount\endcsname
           \advance\res@count by 1
           \ifnum\res@count=0
              \ifx\allow@zero\n@ \advance\res@count by 1 \fi
           \fi
           \ifx\dom@in\y@ 
              \xdef\temp@{\csname dom@num\the\loopcount\endcsname}
              \expandafter\get@dom@count\temp@ 
              \expandafter\xdef\csname dom@num\the\loopcount\endcsname{\temp@}
           \fi %%%%%%%%%%%*****
           \expandafter\xdef\csname res@count\the\loopcount\endcsname{\the\res@count}
           \ifnum\rule@num=\loopcount
              \temp@count=\csname res@count\the\loopcount\endcsname
              \advance\temp@count by \ruler@step  
              \divide\temp@count by \ruler@step
              \multiply\temp@count by \ruler@step
              \xdef\rule@tens{\the\temp@count}  
           \fi
       \fi
       \ifnum\loopcount<\seq@count \c@unt\fi}
\def\count@first{%
    \advance\end@count by 1
    \ifnum\end@count<\start@number
       \loopcount=0
       \ifT@coffee
          \xdef\seq@line{\csname T@coffee\the\loopcount\endcsname}
          \expandafter\TC@get\seq@line
       \fi
       \c@unt
       \count@first
    \fi}
\def\findc@nsensus{%
 \loopcount=0
 \loop
 \advance\loopcount by 1
    \expandafter\xdef\csname sequence\the\loopcount\endcsname{%
                     \csname sequence\the\loopcount\endcsname @}
 \ifnum\loopcount<\seq@count \repeat
 \end@count=0 \count@first \end@count=0 \xdef\start@number{0}
 \topfeaturenowfalse     \bottomfeaturenowfalse
 \ttopfeaturenowfalse    \bbottomfeaturenowfalse
 \tttopfeaturenowfalse   \bbbottomfeaturenowfalse
 \ttttopfeaturenowfalse  \bbbbottomfeaturenowfalse
 \innerloopcount=\cons@count
 \advance\innerloopcount by \res@perline \advance\innerloopcount by 1
 \loopcount=0
    \ifregionalshade
         \expandafter\ifx\csname start\the\loopcount\endcsname\ampers@nd
         \else
            \expandafter\ifnum\csname start\the\loopcount\endcsname>%
            \innerloopcount
            \else
                  \regionalshadenowtrue
            \fi
         \fi
    \fi
    \ifregionaltint
         \expandafter\ifx\csname tintstart\the\loopcount\endcsname\ampers@nd
         \else
            \expandafter\ifnum\csname tintstart\the\loopcount\endcsname>%
            \innerloopcount
            \else
                  \regionaltintnowtrue
            \fi
         \fi
    \fi
    \ifregionalemph
         \expandafter\ifx\csname emphstart\the\loopcount\endcsname\ampers@nd
         \else
            \expandafter\ifnum\csname emphstart\the\loopcount\endcsname>%
            \innerloopcount
            \else
                  \regionalemphnowtrue
            \fi
         \fi
    \fi
    \ifregionallower
         \expandafter\ifx\csname lowerstart\the\loopcount\endcsname\ampers@nd
         \else
            \expandafter\ifnum\csname lowerstart\the\loopcount\endcsname>%
            \innerloopcount
            \else
                  \regionallowernowtrue
            \fi
         \fi
    \fi
    \ifframe@
         \expandafter\ifx\csname framestart\the\loopcount\endcsname\ampers@nd
         \else
            \expandafter\ifnum\csname framestart\the\loopcount\endcsname>%
            \innerloopcount
            \else
               \framenowtrue
            \fi
         \fi
    \fi
    \ifshading@
         \expandafter\ifx\csname shadingstart\the\loopcount\endcsname\ampers@nd
         \else
            \expandafter\ifnum\csname shadingstart\the\loopcount\endcsname>%
            \innerloopcount
            \else
               \shadingnowtrue
            \fi
         \fi
    \fi
    \iftopfeature
       \xdef\bottop@{top}
         \expandafter\ifx\csname start\bottop@\the\loopcount\endcsname\ampers@nd
         \else
            \expandafter\ifnum\csname start\bottop@\the\loopcount\endcsname>%
            \innerloopcount
            \else
               \topfeaturenowtrue
            \fi
         \fi
    \fi
    \ifttopfeature
       \xdef\bottop@{ttop}
         \expandafter\ifx\csname start\bottop@\the\loopcount\endcsname\ampers@nd
         \else
            \expandafter\ifnum\csname start\bottop@\the\loopcount\endcsname>%
            \innerloopcount
            \else
               \ttopfeaturenowtrue
            \fi
         \fi
    \fi
    \iftttopfeature
       \xdef\bottop@{tttop}
         \expandafter\ifx\csname start\bottop@\the\loopcount\endcsname\ampers@nd
         \else
            \expandafter\ifnum\csname start\bottop@\the\loopcount\endcsname>%
            \innerloopcount
            \else
               \tttopfeaturenowtrue
            \fi
         \fi
    \fi
    \ifttttopfeature
       \xdef\bottop@{ttttop}
         \expandafter\ifx\csname start\bottop@\the\loopcount\endcsname\ampers@nd
         \else
            \expandafter\ifnum\csname start\bottop@\the\loopcount\endcsname>%
            \innerloopcount
            \else
               \ttttopfeaturenowtrue
            \fi
         \fi
    \fi
    \ifbottomfeature
       \xdef\bottop@{bottom}
         \expandafter\ifx\csname start\bottop@\the\loopcount\endcsname\ampers@nd
         \else
            \expandafter\ifnum\csname start\bottop@\the\loopcount\endcsname>%
            \innerloopcount
            \else
               \bottomfeaturenowtrue
            \fi
         \fi
    \fi
    \ifbbottomfeature
       \xdef\bottop@{bbottom}
         \expandafter\ifx\csname start\bottop@\the\loopcount\endcsname\ampers@nd
         \else
            \expandafter\ifnum\csname start\bottop@\the\loopcount\endcsname>%
            \innerloopcount
            \else
               \bbottomfeaturenowtrue
            \fi
         \fi
    \fi
    \ifbbbottomfeature
       \xdef\bottop@{bbbottom}
         \expandafter\ifx\csname start\bottop@\the\loopcount\endcsname\ampers@nd
         \else
            \expandafter\ifnum\csname start\bottop@\the\loopcount\endcsname>%
            \innerloopcount
            \else
               \bbbottomfeaturenowtrue
            \fi
         \fi
    \fi
    \ifbbbbottomfeature
       \xdef\bottop@{bbbbottom}
         \expandafter\ifx\csname start\bottop@\the\loopcount\endcsname\ampers@nd
         \else
            \expandafter\ifnum\csname start\bottop@\the\loopcount\endcsname>%
            \innerloopcount
            \else
               \bbbbottomfeaturenowtrue
            \fi
         \fi
    \fi
\loop
 \advance\loopcount by 1
    \ifnumbers@left
       \innerloopcount=\csname seq@len\the\loopcount\endcsname\relax 
       \expandafter\ifnum\csname res@count\the\loopcount\endcsname=%
                   \innerloopcount
          \res@count=\innerloopcount
       \else
          \res@count=\csname res@count\the\loopcount\endcsname
          \advance\res@count by 1
          \ifnum\res@count=0
             \ifx\allow@zero\n@ \advance\res@count by 1 \fi
          \fi
       \fi
       \ifx\dom@in\y@
            \xdef\temp@{\csname dom@num\the\loopcount\endcsname}
            \expandafter\get@dom@count\temp@
            \expandafter\xdef\csname dom@num\the\loopcount\endcsname{\the\res@count,\temp@}
       \fi         %%%%%%%%%%*****
       \expandafter\xdef\csname res@count\the\loopcount\endcsname{\the\res@count}
       \ifnames@right
          \expandafter\xdef\csname seq\the\loopcount\endcsname{%
              \csname res@count\the\loopcount\endcsname)}
       \else
          \expandafter\xdef\csname seq\the\loopcount\endcsname{%
              <\csname newseqname\the\loopcount\endcsname>
              \csname res@count\the\loopcount\endcsname)}
       \fi
       \ifx\dom@in\y@
       \else
          \expandafter\ifnum\csname res@count\the\loopcount\endcsname=%
                      \innerloopcount
          \else
             \res@count=\csname res@count\the\loopcount\endcsname
             \advance\res@count by -1
             \ifnum\res@count=0
                \ifx\allow@zero\n@ \advance\res@count by -1 \fi
             \fi
	         \expandafter\xdef\csname res@count\the\loopcount\endcsname{\the\res@count}
          \fi
       \fi
    \else
       \ifnames@right
          \expandafter\xdef\csname seq\the\loopcount\endcsname{}
       \else
          \expandafter\xdef\csname seq\the\loopcount\endcsname{%
              <\csname newseqname\the\loopcount\endcsname>}
       \fi
    \fi
    \ifx\dom@in\y@
         \xdef\temp@{\csname dom@num@break\the\loopcount\endcsname}
         \expandafter\get@dom@count\temp@
         \expandafter\xdef\csname dom@num@break\the\loopcount\endcsname{\temp@}
         \innerloopcount=\res@count
    \else        
         \innerloopcount=\csname res@count\the\loopcount\endcsname
         \advance\innerloopcount by \res@perline
    \fi
    \advance\innerloopcount by 1
    \ifregionalshade
         \expandafter\ifx\csname start\the\loopcount\endcsname\ampers@nd
         \else
            \expandafter\ifnum\csname start\the\loopcount\endcsname>%
            \innerloopcount
            \else
               \regionalshadenowtrue
            \fi
         \fi
    \fi
    \ifregionaltint
         \expandafter\ifx\csname tintstart\the\loopcount\endcsname\ampers@nd
         \else
            \expandafter\ifnum\csname tintstart\the\loopcount\endcsname>%
            \innerloopcount
            \else
                  \regionaltintnowtrue
            \fi
         \fi
    \fi
    \ifregionalemph
         \expandafter\ifx\csname emphstart\the\loopcount\endcsname\ampers@nd
         \else
            \expandafter\ifnum\csname emphstart\the\loopcount\endcsname>%
            \innerloopcount
            \else
               \regionalemphnowtrue
            \fi
         \fi
    \fi
    \ifregionallower
         \expandafter\ifx\csname lowerstart\the\loopcount\endcsname\ampers@nd
         \else
            \expandafter\ifnum\csname lowerstart\the\loopcount\endcsname>%
            \innerloopcount
            \else
               \regionallowernowtrue
            \fi
         \fi
    \fi
    \ifframe@
         \expandafter\ifx\csname framestart\the\loopcount\endcsname\ampers@nd
         \else
            \expandafter\ifnum\csname framestart\the\loopcount\endcsname>%
            \innerloopcount
            \else
               \framenowtrue
            \fi
         \fi
    \fi
    \ifshading@
         \expandafter\ifx\csname shadingstart\the\loopcount\endcsname\ampers@nd
         \else
            \expandafter\ifnum\csname shadingstart\the\loopcount\endcsname>%
            \innerloopcount
            \else
               \shadingnowtrue
            \fi
         \fi
    \fi
    \iftopfeature
       \xdef\bottop@{top}
         \expandafter\ifx\csname start\bottop@\the\loopcount\endcsname\ampers@nd
         \else
            \expandafter\ifnum\csname start\bottop@\the\loopcount\endcsname>%
            \innerloopcount
            \else
               \topfeaturenowtrue
            \fi
         \fi
    \fi
    \ifttopfeature
       \xdef\bottop@{ttop}
         \expandafter\ifx\csname start\bottop@\the\loopcount\endcsname\ampers@nd
         \else
            \expandafter\ifnum\csname start\bottop@\the\loopcount\endcsname>%
            \innerloopcount
            \else
               \ttopfeaturenowtrue
            \fi
         \fi
    \fi
    \iftttopfeature
       \xdef\bottop@{tttop}
         \expandafter\ifx\csname start\bottop@\the\loopcount\endcsname\ampers@nd
         \else
            \expandafter\ifnum\csname start\bottop@\the\loopcount\endcsname>%
            \innerloopcount
            \else
               \tttopfeaturenowtrue
            \fi
         \fi
    \fi
    \ifttttopfeature
       \xdef\bottop@{ttttop}
         \expandafter\ifx\csname start\bottop@\the\loopcount\endcsname\ampers@nd
         \else
            \expandafter\ifnum\csname start\bottop@\the\loopcount\endcsname>%
            \innerloopcount
            \else
               \ttttopfeaturenowtrue
            \fi
         \fi
    \fi
    \ifbottomfeature
       \xdef\bottop@{bottom}
         \expandafter\ifx\csname start\bottop@\the\loopcount\endcsname\ampers@nd
         \else
            \expandafter\ifnum\csname start\bottop@\the\loopcount\endcsname>%
            \innerloopcount
            \else
               \bottomfeaturenowtrue
            \fi
         \fi
    \fi
    \ifbbottomfeature
       \xdef\bottop@{bbottom}
         \expandafter\ifx\csname start\bottop@\the\loopcount\endcsname\ampers@nd
         \else
            \expandafter\ifnum\csname start\bottop@\the\loopcount\endcsname>%
            \innerloopcount
            \else
               \bbottomfeaturenowtrue
            \fi
         \fi
    \fi
    \ifbbbottomfeature
       \xdef\bottop@{bbbottom}
         \expandafter\ifx\csname start\bottop@\the\loopcount\endcsname\ampers@nd
         \else
            \expandafter\ifnum\csname start\bottop@\the\loopcount\endcsname>%
            \innerloopcount
            \else
               \bbbottomfeaturenowtrue
            \fi
         \fi
    \fi
    \ifbbbbottomfeature
       \xdef\bottop@{bbbbottom}
         \expandafter\ifx\csname start\bottop@\the\loopcount\endcsname\ampers@nd
         \else
            \expandafter\ifnum\csname start\bottop@\the\loopcount\endcsname>%
            \innerloopcount
            \else
               \bbbbottomfeaturenowtrue
            \fi
         \fi
    \fi
\ifnum\loopcount<\seq@count \repeat
 \c@nsensus}

%%%%%  Output routines

\def\white@box{%
   \bgroup
   \fboxsep-0.5pt\fboxrule0.5pt
   \fcolorbox{Black}{White}{\box@hstrut\box@wstrut}\egroup}
\def\box@rule{\vrule depth\box@depth height\box@height width\box@width}
\def\box@hstrut{\vrule depth\box@depth height\box@height width 0pt}
\def\box@wstrut{\vrule depth 0pt height 0pt width\box@width}
\def\do@legend{%
      \baselineskip=1.2\baselineskip
      \xdef\first@{White}
      \fontfamily{\legend@family}%
      \fontseries{\legend@series}%
      \fontshape{\legend@shape}%
      \ifT@coffee
         \noindent%
         \ifnames@\ifnames@right\else\hbox to \name@width{\hss}\fi\fi%
         \ifnumbers@\ifnumbers@left\hbox to \number@width{\hss}\fi\fi%
	     \hbox to \hspace@legend{\hss}%
         \def\res@@style{\csname no@style\endcsname}%
         \def\temp@{X}\xdef\low@up{lower}%
         \expandafter\ifx\csname n@m@tch\the\loopcount\endcsname\low@up%
         \def\temp@{x}\fi%
         \ifhidechar\xdef\temp@{}\fi%
         \textcolor{TC0}{\box@rule}%
         \kern-\box@width\textcolor{Black}{\hbox to \box@width{%
               \legend@size{\res@@style{\hss\temp@\hss}}}}%
         \kern2ex\textcolor{\legend@fg}{%
         \ifgerm@n\kern2ex\legend@size{schlecht}%
         \else\ifsp@nish\kern2ex\legend@size{mala}%
         \else\kern2ex\legend@size{bad}\fi\fi}
         \newline\hbox{}%
         \noindent%
         \ifnames@\ifnames@right\else\hbox to \name@width{\hss}\fi\fi%
         \ifnumbers@\ifnumbers@left\hbox to \number@width{\hss}\fi\fi%
	     \hbox to \hspace@legend{\hss}%
         \textcolor{TC1}{\box@rule}%
         \kern-\box@width\textcolor{Black}{\hbox to \box@width{%
               \legend@size{\res@@style{\hss\temp@\hss}}}}%
         \kern2ex\textcolor{\legend@fg}{\legend@size{}}
         \newline\hbox{}%
         \noindent%
         \ifnames@\ifnames@right\else\hbox to \name@width{\hss}\fi\fi%
         \ifnumbers@\ifnumbers@left\hbox to \number@width{\hss}\fi\fi%
	     \hbox to \hspace@legend{\hss}%
         \textcolor{TC2}{\box@rule}%
         \kern-\box@width\textcolor{Black}{\hbox to \box@width{%
               \legend@size{\res@@style{\hss\temp@\hss}}}}%
         \kern2ex\textcolor{\legend@fg}{\legend@size{}}
         \newline\hbox{}%
         \noindent%
         \ifnames@\ifnames@right\else\hbox to \name@width{\hss}\fi\fi%
         \ifnumbers@\ifnumbers@left\hbox to \number@width{\hss}\fi\fi%
	     \hbox to \hspace@legend{\hss}%
         \textcolor{TC3}{\box@rule}%
         \kern-\box@width\textcolor{Black}{\hbox to \box@width{%
               \legend@size{\res@@style{\hss\temp@\hss}}}}%
         \kern2ex\textcolor{\legend@fg}{\legend@size{}}
         \newline\hbox{}%
         \noindent%
         \ifnames@\ifnames@right\else\hbox to \name@width{\hss}\fi\fi%
         \ifnumbers@\ifnumbers@left\hbox to \number@width{\hss}\fi\fi%
	     \hbox to \hspace@legend{\hss}%
         \textcolor{TC4}{\box@rule}%
         \kern-\box@width\textcolor{Black}{\hbox to \box@width{%
               \legend@size{\res@@style{\hss\temp@\hss}}}}%
         \kern2ex\textcolor{\legend@fg}{%
         \ifgerm@n\kern2ex\legend@size{Grad der}%
         \else\ifsp@nish\kern2ex\legend@size{conservaci\'on}%
         \else\kern2ex\legend@size{level of}\fi\fi}
         \newline\hbox{}%
         \noindent%
         \ifnames@\ifnames@right\else\hbox to \name@width{\hss}\fi\fi%
         \ifnumbers@\ifnumbers@left\hbox to \number@width{\hss}\fi\fi%
	     \hbox to \hspace@legend{\hss}%
         \textcolor{TC5}{\box@rule}%
         \kern-\box@width\textcolor{Black}{\hbox to \box@width{%
               \legend@size{\res@@style{\hss\temp@\hss}}}}%
         \kern2ex\textcolor{\legend@fg}{%
         \ifgerm@n\kern2ex\legend@size{Konservierung}%
         \else\ifsp@nish\kern2ex\legend@size{}%
         \else\kern2ex\legend@size{conservation}\fi\fi}
         \newline\hbox{}%
         \noindent%
         \ifnames@\ifnames@right\else\hbox to \name@width{\hss}\fi\fi%
         \ifnumbers@\ifnumbers@left\hbox to \number@width{\hss}\fi\fi%
	     \hbox to \hspace@legend{\hss}%
         \textcolor{TC6}{\box@rule}%
         \kern-\box@width\textcolor{Black}{\hbox to \box@width{%
               \legend@size{\res@@style{\hss\temp@\hss}}}}%
         \kern2ex\textcolor{\legend@fg}{\legend@size{}}
         \newline\hbox{}%
         \noindent%
         \ifnames@\ifnames@right\else\hbox to \name@width{\hss}\fi\fi%
         \ifnumbers@\ifnumbers@left\hbox to \number@width{\hss}\fi\fi%
	     \hbox to \hspace@legend{\hss}%
         \textcolor{TC7}{\box@rule}%
         \kern-\box@width\textcolor{Black}{\hbox to \box@width{%
               \legend@size{\res@@style{\hss\temp@\hss}}}}%
         \kern2ex\textcolor{\legend@fg}{\legend@size{}}
         \newline\hbox{}%
         \noindent%
         \ifnames@\ifnames@right\else\hbox to \name@width{\hss}\fi\fi%
         \ifnumbers@\ifnumbers@left\hbox to \number@width{\hss}\fi\fi%
	     \hbox to \hspace@legend{\hss}%
         \textcolor{TC8}{\box@rule}%
         \kern-\box@width\textcolor{Black}{\hbox to \box@width{%
               \legend@size{\res@@style{\hss\temp@\hss}}}}%
         \kern2ex\textcolor{\legend@fg}{\legend@size{}}
         \newline\hbox{}%
         \noindent%
         \ifnames@\ifnames@right\else\hbox to \name@width{\hss}\fi\fi%
         \ifnumbers@\ifnumbers@left\hbox to \number@width{\hss}\fi\fi%
	     \hbox to \hspace@legend{\hss}%
         \textcolor{TC9}{\box@rule}%
         \kern-\box@width\textcolor{Black}{\hbox to \box@width{%
               \legend@size{\res@@style{\hss\temp@\hss}}}}%
         \kern2ex\textcolor{\legend@fg}{%
         \ifgerm@n\kern2ex\legend@size{gut}%
         \else\ifsp@nish\kern2ex\legend@size{buena}%
         \else\kern2ex\legend@size{good}\fi\fi}
         \newline\hbox{}%
      \else
      \iffuncmode
        \ifnum\fgroup@num>0
         \loopcount=0
         \loop
         \advance\loopcount by 1
         \xdef\third@{\csname fg@color\the\loopcount\endcsname}%
         \noindent%
         \ifnames@\ifnames@right\else\hbox to \name@width{\hss}\fi\fi%
         \ifnumbers@\ifnumbers@left\hbox to \number@width{\hss}\fi\fi%
	 \hbox to \hspace@legend{\hss}%
         \ifx\third@\first@\white@box\else\textcolor{\third@}{\box@rule}\fi%
         \xdef\third@{\csname fg@textcolor\the\loopcount\endcsname}%
         \def\res@@style{\csname func@style\the\loopcount\endcsname}%
         \def\temp@{X}\xdef\low@up{lower}%
         \expandafter\ifx\csname funcm@tch\the\loopcount\endcsname\low@up%
         \def\temp@{x}\fi%
         \ifhidechar\xdef\temp@{}\fi%
         \kern-\box@width\textcolor{\third@}{\hbox to \box@width{%
               \legend@size{\res@@style{\hss\temp@\hss}}}}%
         \kern2ex\textcolor{\legend@fg}{%
         \legend@size{\csname fgroup@name\the\loopcount\endcsname}}
         \newline\hbox{}%
         \ifnum\loopcount<\fgroup@num \repeat
        \fi
      \else
         \noindent
         \ifnames@\ifnames@right\else\hbox to \name@width{\hss}\fi\fi%
         \ifnumbers@\ifnumbers@left \hbox to \number@width{\hss}\fi\fi%
	 \hbox to \hspace@legend{\hss}%
         \ifx\Nomatch\first@\white@box\else\textcolor{\Nomatch}{\box@rule}\fi%
         \def\res@@style{\csname no@style\endcsname}%
         \def\temp@{X}\xdef\low@up{lower}\ifx\resn@m@tch\low@up\def\temp@{x}\fi%
         \ifhidechar\xdef\temp@{}\fi%
         \kern-\box@width\textcolor{\TextNomatch}{\hbox to \box@width{%
               \legend@size{\res@@style{\hss\temp@\hss}}}}%
         \textcolor{\legend@fg}{%
         \ifgerm@n\kern2ex\legend@size{nicht konserviert}%
         \else\ifsp@nish\kern2ex\legend@size{no conservado}%
         \else\kern2ex\legend@size{non conserved}\fi\fi}
         \newline\hbox{}\noindent%
         \ifsimmode%
         \ifnames@\ifnames@right\else\hbox to \name@width{\hss}\fi\fi
         \ifnumbers@\ifnumbers@left \hbox to \number@width{\hss}\fi\fi
	 \hbox to \hspace@legend{\hss}%
         \ifx\Similar\first@\white@box\else\textcolor{\Similar}{\box@rule}\fi
         \def\res@@style{\csname sim@style\endcsname}%
         \def\temp@{X}\xdef\low@up{lower}\ifx\ressimm@tch\low@up\def\temp@{x}\fi%
         \ifhidechar\xdef\temp@{}\fi%
         \kern-\box@width\textcolor{\TextSimilar}{\hbox to \box@width{%
               \legend@size{\res@@style{\hss\temp@\hss}}}}%
         \textcolor{\legend@fg}{%
         \ifgerm@n\kern2ex\legend@size{\"ahnlich}%
         \else\ifsp@nish\kern2ex\legend@size{similar}%
         \else\kern2ex\legend@size{similar}\fi\fi}
         \newline\hbox{}\noindent%
         \fi%
         \ifnames@\ifnames@right\else\hbox to \name@width{\hss}\fi\fi
         \ifnumbers@\ifnumbers@left \hbox to \number@width{\hss}\fi\fi
	 \hbox to \hspace@legend{\hss}%
         \ifx\Identical\first@\white@box\else\textcolor{\Identical}{\box@rule}\fi
         \def\res@@style{\csname id@style\endcsname}%
         \def\temp@{X}\xdef\low@up{lower}\ifx\resm@tch\low@up\def\temp@{x}\fi%
         \ifhidechar\xdef\temp@{}\fi%
         \kern-\box@width\textcolor{\TextIdentical}{\hbox to \box@width{%
               \legend@size{\res@@style{\hss\temp@\hss}}}}%
         \textcolor{\legend@fg}{%
         \ifgerm@n\kern2ex\legend@size{$\ge$\,\thresh@ld\% konserviert}%
         \else\ifsp@nish\kern2ex\legend@size{$\ge$\,\thresh@ld\% conservado}%
         \else\kern2ex\legend@size{$\ge$\,\thresh@ld\% conserved}\fi\fi}
         \newline\hbox{}\noindent%
         \ifall@shade%
         \ifnames@\ifnames@right\else\hbox to \name@width{\hss}\fi\fi
         \ifnumbers@\ifnumbers@left \hbox to \number@width{\hss}\fi\fi
	 \hbox to \hspace@legend{\hss}%
         \ifx\Allmatch\first@\white@box\else\textcolor{\Allmatch}{\box@rule}\fi
         \def\res@@style{\csname all@style\endcsname}%
         \def\temp@{X}\xdef\low@up{lower}\ifx\res@llm@tch\low@up\def\temp@{x}\fi%
         \ifhidechar\xdef\temp@{}\fi%
         \kern-\box@width\textcolor{\TextAllmatch}{\hbox to \box@width{%
               \legend@size{\res@@style{\hss\temp@\hss}}}}%
         \ifnum\all@thresh@ld=100
         \textcolor{\legend@fg}{%
         \ifgerm@n\kern2ex\legend@size{alle identisch}%
         \else\ifsp@nish\kern2ex\legend@size{todos id\'enticos}%
         \else\kern2ex\legend@size{all match}\fi\fi}
         \else
         \textcolor{\legend@fg}{%
         \ifgerm@n\kern2ex\legend@size{$\ge$\,\all@thresh@ld\% konserviert}%
         \else\ifsp@nish\kern2ex\legend@size{$\ge$\,\all@thresh@ld\% conservado}%
         \else\kern2ex\legend@size{$\ge$\,\all@thresh@ld\% conserved}\fi\fi}
         \fi
         \newline\hbox{}\noindent
         \fi
      \fi
      \fi
}
\def\put@name<#1>#2@{%
   \ifnames@%
   \expandafter\ifx\csname hide@name\the\loopcount\endcsname\yes%
      \xdef\temp@{}\else\xdef\temp@{#1}\fi%
   \expandafter\ifx\csname name@col\the\loopcount\endcsname\yes%
           \def\second@{\names@fg}%
      \else\def\second@{\csname name@col\the\loopcount\endcsname}\fi%
   \fontfamily{\namestext@family}%
   \fontseries{\namestext@series}%
   \fontshape{\namestext@shape}%
   \selectfont%
   \textcolor{\second@}{%
   \hbox to \name@width{\@kern\namestext@size{\temp@}\hss}}\fi%
   \xdef\first@{#2@}%
}
\def\put@number#1)#2@{%
   \ifnumbers@%
   \expandafter\ifx\csname hide@number\the\loopcount\endcsname\yes%
      \xdef\temp@{}\else\xdef\temp@{#1}\fi%
   \expandafter\ifx\csname number@col\the\loopcount\endcsname\yes%
           \def\second@{\numbering@fg}%
      \else\def\second@{\csname number@col\the\loopcount\endcsname}\fi%
   \fontfamily{\numbertext@family}%
   \fontseries{\numbertext@series}%
   \fontshape{\numbertext@shape}%
   \selectfont%
   \textcolor{\second@}{%
   \hbox to \number@width{\hss\numbertext@size{\temp@}\@kern}}\fi%
   \xdef\first@{#2@}%
}
\def\special@shading#1)#2#3#4@{%
   \xdef\second@{@\second@#1}%
   \xdef\third@{\second@}%
   \xdef\first@{#4@}%
   \xdef\second@{#3}%
   \xdef\last@{#2}%
    \def\res@@style{\csname relax\endcsname}%
}
\def\special@shade#1)#2#3#4@{%
   \xdef\second@{\second@#1}%
   \xdef\boxc@l@r{\csname bgseqregion\second@\endcsname}%
   \xdef\textc@l@r{\csname fgseqregion\second@\endcsname}%
   \xdef\first@{#4@}%
   \xdef\second@{#3}%
    \def\res@@style{\csname relax\endcsname}%
}
\def\get@second@#1#2@{\xdef\second@{#1}\xdef\first@{#2@}}
\def\next@char#1#2#3@{%
   \xdef\first@{#3@}%
   \xdef\second@{#2}%
   \xdef\last@{#1}%
   \xdef\temp@@{}%
   \xdef\third@{}%
   \ifx\last@\p@r@gr@ph\expandafter\special@shading\first@\fi%
   \ifx\last@\ampers@nd\def\last@{0}\expandafter\special@shade\first@%
   \else%
   \ifT@coffee%
         \xdef\boxc@l@r{\csname fg@color#1\endcsname}%
         \xdef\textc@l@r{\csname fg@textcolor#1\endcsname}%
         \def\res@@style{\csname func@style#1\endcsname}%
         \if\last@ *\def\last@{0}\fi
         \if\last@ /\def\last@{10}\fi
         \if\last@ !\def\last@{11}\fi
   \else%
   \iffuncmode%
         \xdef\boxc@l@r{\csname fg@color#1\endcsname}%
         \xdef\textc@l@r{\csname fg@textcolor#1\endcsname}%
         \def\res@@style{\csname func@style#1\endcsname}%
         \if\last@ *\def\last@{0}\fi
         \if\last@ /\def\last@{10}\fi
         \if\last@ !\def\last@{11}\fi
   \else%
   \ifcase\last@\xdef\boxc@l@r{\csname Allmatch\third@\endcsname}%
             \xdef\textc@l@r{\csname TextAllmatch\third@\endcsname}%
             \def\res@@style{\all@style}%
         \or\xdef\boxc@l@r{\csname Identical\third@\endcsname}%
            \xdef\textc@l@r{\csname TextIdentical\third@\endcsname}%
             \def\res@@style{\id@style}%
         \or\xdef\boxc@l@r{\csname Similar\third@\endcsname}%
            \xdef\textc@l@r{\csname TextSimilar\third@\endcsname}%
             \def\res@@style{\sim@style}%
         \or\xdef\boxc@l@r{\csname Nomatch\third@\endcsname}%
            \xdef\textc@l@r{\csname TextNomatch\third@\endcsname}%
             \def\res@@style{\no@style}%
         \or\xdef\boxc@l@r{\csname ConsNomatch\third@\endcsname}%
            \xdef\textc@l@r{\csname ConsTextNomatch\third@\endcsname}%
             \def\res@@style{\csname relax\endcsname}%
         \or\xdef\boxc@l@r{\csname ConsMatch\third@\endcsname}%
            \xdef\textc@l@r{\csnam ConsTextMatch\third@\endcsname}%
             \def\res@@style{\csname relax\endcsname}%
         \or\xdef\boxc@l@r{\csname ConsAllmatch\third@\endcsname}%
            \xdef\textc@l@r{\csname ConsTextAllmatch\third@\endcsname}%
             \def\res@@style{\csname relax\endcsname}%
         \or\xdef\boxc@l@r{\csname gap@bg\third@\endcsname}%
            \xdef\textc@l@r{\csname gap@fg\third@\endcsname}%
             \def\res@@style{\csname relax\endcsname}%
         \or\xdef\boxc@l@r{White}\xdef\textc@l@r{White}%
             \def\res@@style{\csname relax\endcsname}%
         \else\xdef\boxc@l@r{\csname domgap@bg\endcsname}%
              \xdef\textc@l@r{\csname domgap@fg\endcsname}%
             \def\res@@style{\csname relax\endcsname}%
   \fi\fi\fi\fi%
   \ifx\second@\comm@%
         \def\temp@{\res@style}%
         \expandafter\get@second@\first@%
   \else%
         \def\temp@{\csname relax\endcsname}%
   \fi%
   \ifx\second@\equ@l%
         \xdef\temp@@{\light@}%
         \expandafter\get@second@\first@%
   \fi%
   \ifx\second@\semic@n%
         \expandafter\get@second@\first@%
         \xdef\first@@{\first@}%
         \xdef\first@{\second@}%
         \make@lower%
         \xdef\second@{\first@}%
         \xdef\first@{\first@@}%
   \fi%
   \textcolor{\temp@@\boxc@l@r}{\box@rule}%
   \ifhidechar%
      \ifx\second@\o@%
         \def\second@{\gap@rule}%
         \hbox to -\box@width{\hss\textcolor{\temp@@\textc@l@r}%
                             {\residues@size{\res@@style{\temp@{\second@}}}}\hss}%
                              \kern\box@width%
      \fi%
   \else%
   \ifx\second@\o@\def\second@{\gap@rule}\fi%
   \hbox to -\box@width{\hss\textcolor{\temp@@\textc@l@r}%
                       {\residues@size{\res@@style{\temp@{\second@}}}}\hss}%
                        \kern\box@width\fi%
}
\def\put@char{%
   \ifnum\innerloopcount>\res@perline
   \else
      \expandafter\next@char\first@
      \advance\innerloopcount by 1
      \put@char%
   \fi}
\def\next@cons#1#2#3@{%
   \xdef\last@{#1}%
   \xdef\third@{}%
   \ifx\last@\p@r@gr@ph\expandafter\special@shading\first@\fi%
   \ifx\last@\ampers@nd\def\last@{0}\expandafter\special@shade\first@%
   \else%
   \ifx\collect@cons@colors\y@%
      \if\last@ *%
         \xdef\boxc@l@r{\gap@bg}\xdef\textc@l@r{\gap@bg}%
      \else%
      \if\last@ !%
         \xdef\boxc@l@r{\domgap@bg}\xdef\textc@l@r{\domgap@bg}%
      \else%
      \if\last@ /%
         \xdef\boxc@l@r{White}\xdef\textc@l@r{White}%
      \else%
      \expandafter\get@item\first@@@%
      \xdef\first@@@{\first@}%
      \ifx\box@scale\y@%
         \xdef\boxc@l@r{\c@nssc@le\fourth@}%
      \else%
         \xdef\first@{T-Coffee}%
         \ifx\first@\c@nssc@le%
            \xdef\boxc@l@r{TC\last@}%         
         \else
            \xdef\boxc@l@r{\c@nssc@le}%
         \fi%
      \fi%
      \ifx\text@scale\y@%
         \xdef\textc@l@r{\c@nsc@l\fourth@}%
      \else%
         \xdef\first@{T-Coffee}%
         \ifx\first@\c@nsc@l%
            \xdef\textc@l@r{TC\last@}%         
         \else
            \xdef\textc@l@r{\c@nsc@l}%
         \fi%
      \fi\fi\fi\fi%
      \def\res@@style{\csname relax\endcsname}%
   \else%
   \ifT@coffee%
      \if\last@ *%
         \xdef\boxc@l@r{\gap@bg}\xdef\textc@l@r{\gap@bg}%
      \else%
         \if\last@ !%
            \xdef\boxc@l@r{\domgap@bg}\xdef\textc@l@r{\domgap@bg}%
         \else%
            \if\last@ /%
               \xdef\boxc@l@r{White}\xdef\textc@l@r{White}%
            \else%
               \xdef\boxc@l@r{White}\xdef\textc@l@r{TC\last@}%
               \def\res@@style{\csname relax\endcsname}%
            \fi
         \fi%
      \fi%
   \else%
   \ifcase#1\xdef\boxc@l@r{\csname Allmatch\third@\endcsname}%
            \xdef\textc@l@r{\csname TextAllmatch\third@\endcsname}%
             \def\res@@style{\all@style}%
         \or\xdef\boxc@l@r{\csname Identical\third@\endcsname}%
            \xdef\textc@l@r{\csname TextIdentical\third@\endcsname}%
             \def\res@@style{\id@style}%
         \or\xdef\boxc@l@r{\csname Similar\third@\endcsname}%
            \xdef\textc@l@r{\csname TextSimilar\third@\endcsname}%
             \def\res@@style{\sim@style}%
         \or\xdef\boxc@l@r{\csname Nomatch\third@\endcsname}%
            \xdef\textc@l@r{\csname TextNomatch\third@\endcsname}%
             \def\res@@style{\no@style}%
         \or\xdef\boxc@l@r{\ConsNomatch}\xdef\textc@l@r{\ConsTextNomatch}%
             \def\res@@style{\csname relax\endcsname}%
         \or\xdef\boxc@l@r{\ConsMatch}\xdef\textc@l@r{\ConsTextMatch}%
             \def\res@@style{\csname relax\endcsname}%
         \or\xdef\boxc@l@r{\ConsAllmatch}\xdef\textc@l@r{\ConsTextAllmatch}%
             \def\res@@style{\csname relax\endcsname}%
         \or\xdef\boxc@l@r{\gap@bg}\xdef\textc@l@r{\gap@fg}%
             \def\res@@style{\csname relax\endcsname}%
         \or\xdef\boxc@l@r{White}\xdef\textc@l@r{White}%
             \def\res@@style{\csname relax\endcsname}%
         \else\xdef\boxc@l@r{\domgap@bg}\xdef\textc@l@r{\domgap@fg}%
             \def\res@@style{\csname relax\endcsname}%
   \fi\fi\fi\fi%
   \xdef\first@{#3@}%
   \xdef\second@{#2}%
   \ifx\second@\comm@%
         \def\temp@{\res@style}%
         \expandafter\get@second@\first@%
   \else%
         \def\temp@{\csname relax\endcsname}%
   \fi%
   \ifx\second@\semic@n%
         \expandafter\get@second@\first@%
         \xdef\first@@{\first@}%
         \xdef\first@{\second@}%
         \make@lower%
         \xdef\second@{\first@}%
         \xdef\first@{\first@@}%
   \fi%
   \textcolor{\boxc@l@r}{\box@rule}%
   \ifhidechar%
      \ifx\second@\o@%
         \def\second@{\gap@rule}%
         \hbox to -\box@width{\hss\textcolor{\textc@l@r}%
                             {\residues@size{\res@@style{\temp@{\second@}}}}\hss}%
                              \kern\box@width%
      \fi%
   \else%
   \ifx\second@\o@\def\second@{\gap@rule}\fi%
   \hbox to -\box@width{\hss\textcolor{\textc@l@r}%
                       {\residues@size{\res@@style{\temp@{\second@}}}}\hss}%
                        \kern\box@width\fi%
}
\def\put@cons{%
   \ifnum\innerloopcount>\res@perline
   \else
      \expandafter\next@cons\first@
      \advance\innerloopcount by 1
      \put@cons%
   \fi}
\def\put@line{%
   \ifnames@right\else\def\@kern{\kern0em}\expandafter\put@name\first@\fi
   \ifnumbers@left\def\@kern{\kern1em}\expandafter\put@number\first@\fi
   \fontfamily{\residues@family}%
   \fontseries{\residues@series}%
   \fontshape{\residues@shape}%
   \selectfont%
   \ifx\cons@now\y@%
      \innerloopcount=1\relax\put@cons%
   \else
      \innerloopcount=1\relax\put@char%
   \fi
   \ifnumbers@right\def\@kern{\kern0em}\expandafter\put@number\first@\fi
   \ifnames@right\def\@kern{\kern1em}\expandafter\put@name\first@\fi
   \newline\hbox{}%
}

\def\set@consensus{%
      \ifnames@right
        \ifnumbers@left
          \ifnumbers@right
           \xdef\consensus{)\consensus)<\cons@name>}%
          \else
           \xdef\consensus{)\consensus<\cons@name>}%
          \fi
        \else
           \xdef\consensus{\consensus)<\cons@name>}%
        \fi
      \else
        \ifnumbers@left
          \ifnumbers@right
           \xdef\consensus{<\cons@name>)\consensus)}%
          \else
           \xdef\consensus{<\cons@name>)\consensus}%
          \fi
        \else
           \xdef\consensus{<\cons@name>\consensus)}%
        \fi
      \fi}

\def\get@rulenum<#1>#2@{%
   \xdef\first@{#2@}%
   \xdef\fill@char{#1[,]&}%
   \expandafter\opt@color\fill@char%
   \ifx\f@color\comm@\xdef\f@color{\ruler@fg}\fi%
   \ifcase\rule@top
     \ifnum\ruler@rot=0 %
      \xdef\temp@{tt}%
      \ifx\ruler@family\temp@%
         \def\third@{\bottomruler@size\ensuremath{\,\stackrel{\mathtt{\fourth@}}{\textcolor{\ruler@fg}{.}}}}%
      \else
      \xdef\temp@{sf}%
      \ifx\ruler@family\temp@%
         \def\third@{\bottomruler@size\ensuremath{\,\stackrel{\mathsf{\fourth@}}{\textcolor{\ruler@fg}{.}}}}%
      \else
      \xdef\temp@{rm}%
      \ifx\ruler@family\temp@%
         \def\third@{\bottomruler@size\ensuremath{\,\stackrel{\mathrm{\fourth@}}{\textcolor{\ruler@fg}{.}}}}%
      \fi\fi\fi
     \else
      \xdef\temp@{tt}%
      \ifx\ruler@family\temp@%
         \def\third@{\tt\bottomruler@size\,\,\,\,%
           \begin{rotopo}{90}\hbox to \ruler@width{\,\textcolor{\ruler@fg}{\ensuremath{\cdot}}\,\fourth@\hss}\end{rotopo}}%
      \else
      \xdef\temp@{sf}%
      \ifx\ruler@family\temp@%
         \def\third@{\sf\bottomruler@size\,\,\,\,%
           \begin{rotopo}{90}\hbox to \ruler@width{\,\textcolor{\ruler@fg}{\ensuremath{\cdot}}\,\fourth@\hss}\end{rotopo}}%
      \else
      \xdef\temp@{rm}%
      \ifx\ruler@family\temp@%
         \def\third@{\rm\bottomruler@size\,\,\,\,%
           \begin{rotopo}{90}\hbox to \ruler@width{\,\textcolor{\ruler@fg}{\ensuremath{\cdot}}\,\fourth@\hss}\end{rotopo}}%
      \fi\fi\fi
     \fi
   \else
     \ifnum\ruler@rot=0 %
      \xdef\temp@{tt}%
      \ifx\ruler@family\temp@%
         \def\third@{\bottomruler@size\ensuremath{\,\stackrel{\textcolor{\ruler@fg}{.}}{\mathtt{\fourth@}}}}%
      \else
      \xdef\temp@{sf}%
      \ifx\ruler@family\temp@%
         \def\third@{\bottomruler@size\ensuremath{\,\stackrel{\textcolor{\ruler@fg}{.}}{\mathsf{\fourth@}}}}%
      \else
      \xdef\temp@{rm}%
      \ifx\ruler@family\temp@%
         \def\third@{\bottomruler@size\ensuremath{\,\stackrel{\textcolor{\ruler@fg}{.}}{\mathrm{\fourth@}}}}%
      \fi\fi\fi
     \else
      \xdef\temp@{tt}%
      \ifx\ruler@family\temp@%
         \def\third@{\tt\bottomruler@size\,\,\,\,%
           \begin{rotopo}{90}\hbox to \ruler@width{\hss\fourth@\,\textcolor{\ruler@fg}{\ensuremath{\cdot}}}\end{rotopo}}%
      \else
      \xdef\temp@{sf}%
      \ifx\ruler@family\temp@%
         \def\third@{\sf\bottomruler@size\,\,\,\,%
           \begin{rotopo}{90}\hbox to \ruler@width{\hss\fourth@\,\textcolor{\ruler@fg}{\ensuremath{\cdot}}}\end{rotopo}}%
      \else
      \xdef\temp@{rm}%
      \ifx\ruler@family\temp@%
         \def\third@{\rm\bottomruler@size\,\,\,\,%
           \begin{rotopo}{90}\hbox to \ruler@width{\hss\fourth@\,\textcolor{\ruler@fg}{\ensuremath{\cdot}}}\end{rotopo}}%
      \fi\fi\fi
     \fi
   \fi}

\def\next@rulechar#1#2@{%
   \xdef\third@{#1}%
   \xdef\first@{#2@}%
   \xdef\second@{!}%
   \ifx\third@\second@ \expandafter\get@rulenum\first@%
   \else \xdef\third@{}\xdef\f@color{Black}\fi
   \textcolor{\f@color}{\hbox to \box@width{\hss\third@\hss}}%
   }

\def\put@rulechar{%
   \ifnum\innerloopcount>\res@perline
   \else
      \expandafter\next@rulechar\first@
      \advance\innerloopcount by 1
      \put@rulechar%
   \fi}

\def\put@ruler{%
   \ifnames@right%
        \ifnumbers@left%
          \ifnumbers@right%
            \xdef\ruler@{)\ruler@)<>}%
          \else%
            \xdef\ruler@{)\ruler@<>}%
          \fi%
        \else%
            \xdef\ruler@{\ruler@)<>}%
        \fi%
   \else%
        \ifnumbers@left%
          \ifnumbers@right%
            \xdef\ruler@{<>)\ruler@)}%
          \else%
            \xdef\ruler@{<>)\ruler@}%
          \fi%
        \else%
            \xdef\ruler@{<>\ruler@)}%
        \fi%
   \fi%
   \xdef\first@{\ruler@ @}%
   \ifnum\rule@top=1%
      \vspace{\ruler@sp@ce}%
      \vspace{-\baselineskip}%
      \newline\hbox{}%
   \fi%
   \ifnames@right\else\def\@kern{\kern0em}\expandafter\put@name\first@\fi%
   \ifnumbers@left\def\@kern{\kern1em}\expandafter\put@number\first@\fi%
   \vspace{-0.25\baselineskip}%
   \ifnum\rule@top=0%
      \vspace{\ruler@sp@ce}%
   \fi%
   \fontfamily{\ruler@family}%
   \fontseries{m}%
   \fontshape{n}%
   \selectfont%
   \innerloopcount=1\relax\put@rulechar%
   \newline\hbox{}%
}
\def\get@firstfill#1#2&{\xdef\second@@{#1}\xdef\fill@char{#2&}}
\def\get@firstv@l#1,#2&{\xdef\second@@{#1}\xdef\fill@char{#2&}}
\def\get@tripletfill#1#2#3#4&{%
   \multiply\temp@count by -1%
   \def\second@@{#1}\def\second@@@{#2}\def\second@@@@{#3}\def\fill@char{#4&}}
\def\putfeature@style#1{%
   \residues@size%
   \setbox1=\hbox{\ensuremath{\overrightarrow{\hbox{}}}}%
   \arrow@height=\ht1%
   \arrow@width=\wd1%
   \xdef\second@@{#1}%
   \xdef\last@{\second@@::&}\expandafter\test@fill\last@%
   \xdef\last@{empty}%
   \ifx\second@@\last@%
      \hbox to \second@\box@width{\hss}%
   \else%
    \xdef\last@{translate}%
    \ifx\second@@\last@%
     \fontfamily{\featurestyles@family}%
     \fontseries{\featurestyles@series}%
     \fontshape{\featurestyles@shape}%
     \selectfont%
     \xdef\fill@char{\fill@char &}%
     \if\seq@type N%
        \loop%
            \expandafter\get@firstfill\fill@char%
            \if\second@@ -\def\second@@{\hss}\fi%
            \hbox to \box@width{\hss\textcolor{\f@color}{\featurestyles@size{\second@@}}\hss}%
        \ifx\fill@char\ampers@nd\else\repeat%
     \else%
     \temp@count=1%
     \loop%
      \expandafter\get@tripletfill\fill@char%
      \if\second@@ -\ifnum\tr@nsstyle>0\hbox to \box@width{\hss}\fi%
      \else%
        \ifcase\tr@nsstyle%
           \hbox{\trans@size\textcolor{\f@color}%
                {\hss\second@@\second@@@\second@@@@\,\hss}}%
        \or%
           \ifnum\temp@count=1%
              \vbox{\trans@size%
              \hbox to \box@width{\textcolor{\f@color}{\second@@\hss}}%
              \hbox to \box@width{\textcolor{\f@color}{\hss\second@@@\hss}}%
              \hbox to \box@width{\textcolor{\f@color}{\hss\second@@@@}}}%
           \else%
              \vbox{\trans@size%
              \hbox to \box@width{\textcolor{\f@color}{\hss\second@@@@}}%
              \hbox to \box@width{\textcolor{\f@color}{\hss\second@@@\hss}}%
              \hbox to \box@width{\textcolor{\f@color}{\second@@\hss}}}%
           \fi%
        \or%
           \ifnum\temp@count=1%
              \vbox{\trans@size%
              \hbox to \box@width{\textcolor{\f@color}%
                                 {\hss\second@@\second@@@\second@@@@\hss}}%
              \hbox to \box@width{\textcolor{\f@color}{\hss}}}%
           \else%
              \vbox{\trans@size%
              \hbox to \box@width{\textcolor{\f@color}{\hss}}%
              \hbox to \box@width{\textcolor{\f@color}%
                                 {\hss\second@@\second@@@\second@@@@\hss}}}%
           \fi%
        \or%
           \vbox{\trans@size%
           \hbox to \box@width{\textcolor{\f@color}{\second@@\hss}}%
           \hbox to \box@width{\textcolor{\f@color}{\hss\second@@@\hss}}%
           \hbox to \box@width{\textcolor{\f@color}{\hss\second@@@@}}}%
        \or%
           \vbox{\trans@size%
           \hbox to \box@width{\textcolor{\f@color}{\hss\second@@\hss}}%
           \hbox to \box@width{\textcolor{\f@color}{\hss\second@@@\hss}}%
           \hbox to \box@width{\textcolor{\f@color}{\hss\second@@@@\hss}}}%
        \fi%
      \fi%
     \ifx\fill@char\ampers@nd\else\repeat%
     \fi%
    \else%
     \xdef\last@{brace}%
     \ifx\second@@\last@%
      \xdef\last@{top}%
      \textcolor{\f@color}{%
      \ifx\bottop@\last@%
         \ensuremath{\overbrace{\hbox to \second@\box@width{\hss%
             \rule[0.2\arrow@height]{0pt}{0pt}\hss}}}%
      \else%
      \xdef\last@{ttop}%
      \ifx\bottop@\last@%
         \ensuremath{\overbrace{\hbox to \second@\box@width{\hss%
             \rule[0.2\arrow@height]{0pt}{0pt}\hss}}}%
      \else%
      \xdef\last@{tttop}%
      \ifx\bottop@\last@%
         \ensuremath{\overbrace{\hbox to \second@\box@width{\hss%
             \rule[0.2\arrow@height]{0pt}{0pt}\hss}}}%
      \else%
      \xdef\last@{ttttop}%
      \ifx\bottop@\last@%
         \ensuremath{\overbrace{\hbox to \second@\box@width{\hss%
             \rule[0.2\arrow@height]{0pt}{0pt}\hss}}}%
      \else%
         \raisebox{1.75\arrow@height}{\ensuremath{\underbrace{\hbox to %
             \second@\box@width{}}}}%
      \fi\fi\fi\fi}%
     \else%
      \xdef\last@{fill}%
      \ifx\second@@\last@%
       \fontfamily{\featurestyles@family}%
       \fontseries{\featurestyles@series}%
       \fontshape{\featurestyles@shape}%
       \selectfont%
       \ifx\bottop@\temp@\xdef\last@{0.35}\else\xdef\last@{0}\fi%
       \kern0.15\box@width%
       \loopcount=0\relax%
       \loop%
         \advance\loopcount by 1\relax%
         \raisebox{\last@\arrow@height}{%
            \hbox to \box@width{\hss{\textcolor{\f@color}{\featurestyles@size\fill@char}}\hss}}%
       \ifnum\loopcount<\second@ \repeat%
      \else%
      \xdef\last@{bar}%
      \ifx\second@@\last@%
       \setlength\arrow@width{\pm@shift\box@height}%
       \setlength\arrow@width{\b@r@stretch\arrow@width}%
       \ifx\fill@char\N@%
          \raisebox{\pm@shift\box@height}{\vrule width\box@width}%
       \else%
          \kern-0.6\box@width%
          \ifx\frame@color\back@color%
          \else%
             \setlength\arrow@height{\box@height}%
             \advance\arrow@height by -\pm@shift\box@height%
             \setlength\arrow@height{\b@r@stretch\arrow@height}%
             \raisebox{\arrow@width}{%
                \hbox to \box@width{\hss{\textcolor{\back@color}{\vrule depth\arrow@width%
                height\arrow@height width\box@width}}\hss}}%
             \kern-\box@width%
          \fi%
          \setlength\arrow@height{\fill@char\box@height}%
          \setlength\arrow@height{\b@r@stretch\arrow@height}%
          \divide\arrow@height by 100\relax%
          \ifdim\arrow@height<0pt%
             \arrow@height=-\arrow@height%
             \raisebox{\arrow@width}{%
                \hbox to \box@width{\hss{\textcolor{\frame@color}{\vrule depth\arrow@height width0.8\box@width}}\hss}}%
          \else%
             \raisebox{\arrow@width}{%
                \hbox to \box@width{\hss{\textcolor{\frame@color}{\vrule height\arrow@height width0.8\box@width}}\hss}}\fi%
          \kern-\box@width%
          \raisebox{\arrow@width}{\textcolor{Black}{\vrule height0.25pt depth0.25pt width\box@width}}%
       \fi%
      \else%
      \xdef\last@{color}%
      \ifx\second@@\last@%
       \setlength\arrow@height{\fill@char\box@height}%
       \setlength\arrow@height{\sc@le@stretch\arrow@height}%
       \divide\arrow@height by 100\relax%
       \raisebox{0.2\box@height}{%
         \hbox to \box@width{\hss{\textcolor{\f@color}{\vrule height\arrow@height width\box@width}}\hss}}%
      \else%
      \xdef\last@{plotcolor}%
      \ifx\second@@\last@%
        \xdef\fill@char{\fill@char,&}%
        \loop%
            \expandafter\get@firstv@l\fill@char%
            \ifx\second@@\N@\hbox to \box@width{\hss}%
            \else
              \loopcount=\second@@%
              \ifx\T@coffee@ccons\y@%
              \else%
                 \advance\loopcount by -\pm@shift%
                 \advance\loopcount by 4%
                 \divide\loopcount by 5%
                 \multiply\loopcount by 5%
                 \ifnum\loopcount>100\loopcount=100\fi%
                 \ifnum\loopcount<5\loopcount=5\fi%
              \fi%
              \setlength\arrow@height{50\box@height}%
              \divide\arrow@height by 100\relax%
              \setlength\arrow@height{\sc@le@stretch\arrow@height}%
              \raisebox{0.2\box@height}{%
                 \hbox to \box@width{\hss{\textcolor{\f@color\the\loopcount}{\vrule height\arrow@height width\box@width}}\hss}}%
            \fi
        \ifx\fill@char\ampers@nd\else\repeat%
      \else%
      \xdef\last@{plotbar}%
      \ifx\second@@\last@%
        \xdef\fill@char{\fill@char,&}%
        \ifnum\pm@shift>0%
           \setlength\arrow@width{0pt}%
        \else
           \setlength\arrow@width{-\pm@shift\box@height}%
           \divide\arrow@width by 100%
           \setlength\arrow@width{\b@r@stretch\arrow@width}%
        \fi%
        \loop%
            \expandafter\get@firstv@l\fill@char%
            \ifx\second@@\N@\hbox to \box@width{\hss}%
            \else\relax%
               \ifx\frame@color\back@color%
               \else%
                  \setlength\arrow@height{\b@r@stretch\box@height}%
                  \hbox to \box@width{\hss{\textcolor{\back@color}%
                      {\vrule height\arrow@height width\box@width}}\hss}%
                  \kern-\box@width%
               \fi%
               \ifx\T@coffee@bcons\y@%
                  \ifnum\second@@=99 \xdef\second@@{0}\fi%
               \fi
               \setlength\arrow@height{\second@@\box@height}%
               \ifx\T@coffee@bcons\y@%
                  \setlength\arrow@height{11\arrow@height}%
               \fi
               \divide\arrow@height by 100\relax%
               \setlength\arrow@height{\b@r@stretch\arrow@height}%
               \ifdim\arrow@height<0pt%
                  \arrow@height=-\arrow@height%
                  \raisebox{\arrow@width}{%
                     \hbox to \box@width{\hss{\textcolor{\frame@color}%
                        {\vrule depth\arrow@height width0.8\box@width}}\hss}}%
               \else%
                  \raisebox{\arrow@width}{%
                     \hbox to \box@width{\hss{\textcolor{\frame@color}%
                        {\vrule height\arrow@height width0.8\box@width}}\hss}}\fi%
               \kern-\box@width%
               \raisebox{\arrow@width}{\textcolor{Black}%
                        {\vrule height0.25pt depth0.25pt width\box@width}}%
            \fi
        \ifx\fill@char\ampers@nd\else\repeat%
      \else%
      \xdef\last@{helix}%
      \ifx\second@@\last@%
       \fontfamily{cmr}%
       \fontseries{m}%
       \fontshape{it}%
       \selectfont%
       \ifx\bottop@\temp@\xdef\last@{0.35}\else\xdef\last@{0}\fi%
       \kern0.15\box@width%
       \setbox1=\hbox{\ensuremath{\helixhook}\kern-1.13exo\kern-1.02ex}%
       \arrow@width=\second@\box@width%
       \divide\arrow@width by \wd1%
       \arrow@width=2\wd1%
       \loop%
         \textcolor{\f@color}{\raisebox{-0.25ex}{\ensuremath{\helixhook}}%
                              \kern-1.13ex\raisebox{0.3ex}{o}}\kern-1.02ex%
         \advance\arrow@width by \wd1\relax%
       \ifdim\arrow@width<\second@\box@width \repeat%
       \textcolor{\f@color}{\raisebox{-0.25ex}{\ensuremath{\helixhook}}}%
      \else%
      \xdef\last@{box}%
      \ifx\second@@\last@%
         \fontfamily{\featurestyles@family}%
         \fontseries{\featurestyles@series}%
         \fontshape{\featurestyles@shape}%
         \selectfont%
         \kern-\second@\box@width%
         \bgroup%
             \textcolor{\back@color}{%
                \vrule width\second@\box@width height\box@height depth\box@depth}%
             \kern-\second@\box@width%
             \fboxsep-\rule@@thick\fboxrule\rule@@thick%
             \textcolor{\frame@color}{%
                \fbox{\makebox[\second@\box@width]%
                     {\vrule\@height\box@height\@depth\box@depth \@width\z@}}}%
         \egroup%
         \setbox1=\hbox{\residues@size{\fill@char}}%
         \temp@count=\wd1 \xdef\wd@{\the\temp@count}%
         \width@tmp=\second@\box@width%
         \temp@count=\width@tmp%
         \xdef\sb@{\the\temp@count}%
         \ifnum\wd@>\sb@ \xdef\fill@char{}\fi%
         \hbox to -\second@\box@width{\hss\textcolor{\f@color}%
                       {\residues@size{\fill@char}}\hss}%
     \else%
      \expandafter\get@shape\second@@%
      \xdef\last@{arrow}%
      \ifx\second@@\last@%
        \kern-0.75\box@width%
        \ifx\bottop@\temp@ \xdef\last@{0.35}\else\xdef\last@{-0.55}\fi%
        \textcolor{\f@color}{%
        \raisebox{\last@\arrow@height}{%
         \if\first@@ ,%
            \rule{0.1\arrow@height}{\arrow@height}\kern-0.35\arrow@height%
         \else%
            \if\first@@ |%
               \rule{0.1\arrow@height}{2\arrow@height}\kern-0.35\arrow@height%
            \else%
               \if\first@@ `\xdef\first@@{'}\fi%
               \if\first@@ '%
                  \rule[\arrow@height]%
                     {0.1\arrow@height}{\arrow@height}\kern-0.35\arrow@height%
               \else%
                  \if\first@@ -%
                     \rule{0pt}{0pt}\kern-0.35\arrow@height%
                  \fi%
               \fi%
            \fi%
         \fi%
         \if\third@@ v%
            \if\first@@ v%
               \ensuremath{\overleftarrow{\hbox to %
               \second@\box@width{\rule[0.4\arrow@height]{0pt}{0pt}\hss}}}%
               \kern-\arrow@width%
               \ensuremath{\overrightarrow{\hbox%
                          {\rule[0.4\arrow@height]{0pt}{0pt}\hss}}}%
            \else%
               \ensuremath{\overrightarrow{\hbox to %
               \second@\box@width{\rule[0.4\arrow@height]{0pt}{0pt}\hss}}}%
            \fi%
         \else%
            \if\first@@ v%
               \ensuremath{\overleftarrow{\hbox to %
                   \second@\box@width{\rule[0.4\arrow@height]{0pt}{0pt}\hss}}}%
            \else
               \kern0.35\arrow@height%
               \rule[0.9\arrow@height]{\second@\box@width}{0.1\arrow@height}%
               \kern0.35\arrow@height%
            \fi
            \if\third@@ ,%
               \kern-0.4\arrow@height\rule{0.1\arrow@height}{\arrow@height}%
            \else%
               \if\third@@ |%
                  \kern-0.4\arrow@height\rule{0.1\arrow@height}{2\arrow@height}%
               \else%
                  \if\third@@ `\xdef\third@@{'}\fi%
                  \if\third@@ '%
                     \kern-0.4\arrow@height%
                     \rule[\arrow@height]{0.1\arrow@height}{\arrow@height}%
                  \fi%
               \fi%
            \fi%
         \fi}}%
      \else%
       \xdef\last@{doublearrow}%
       \ifx\second@@\last@%
          \setbox1=\hbox{\ensuremath{\Rightarrow}}%
          \arrow@height=\ht1%
          \arrow@width=\wd1%
          \setbox1=\hbox to \second@\box@width{}%
          \width@tmp=\wd1%
          \kern-0.75\box@width%
        \xdef\temp@{top}%
        \ifx\bottop@\temp@ \xdef\last@{0.25}%
        \else%
           \xdef\temp@{ttop}%
           \ifx\bottop@\temp@ \xdef\last@{0.25}%
           \else%
             \xdef\temp@{tttop}%
             \ifx\bottop@\temp@ \xdef\last@{0.25}%
             \else%
               \xdef\temp@{ttttop}%
               \ifx\bottop@\temp@ \xdef\last@{0.25}%
               \else%
                 \xdef\last@{-0.25}\fi\fi\fi\fi%
        \textcolor{\f@color}{%
        \raisebox{\last@\arrow@height}{%
         \if\first@@ ,%
            \rule[-0.5\arrow@height]{0.1\arrow@height}{1.5\arrow@height}%
            \kern-0.1\arrow@height%
         \else%
            \if\first@@ |%
               \rule[-0.5\arrow@height]{0.1\arrow@height}{2.25\arrow@height}%
               \kern-0.1\arrow@height%
            \else%
               \if\first@@ `\xdef\first@@{'}\fi%
               \if\first@@ '%
                  \rule[0.4\arrow@height]%
                     {0.1\arrow@height}{1.5\arrow@height}%
                  \kern-0.1\arrow@height%
               \else%
                  \if\first@@ <%
                     \ensuremath{\Leftarrow}\kern-0.5\arrow@width%
                     \advance\width@tmp by -0.5\arrow@width
                  \else
                     \rule{0pt}{0pt}%
                  \fi%
               \fi%
            \fi%
         \fi%
         \if\third@@ >%
            \advance\width@tmp by -0.5\arrow@width%
            \rule[0.37\arrow@height]{\width@tmp}{0.1\arrow@height}%
            \kern-\width@tmp%
            \rule[0.9\arrow@height]{\width@tmp}{0.1\arrow@height}%
            \kern-0.5\arrow@width\ensuremath{\Rightarrow}%
         \else%
            \rule[0.37\arrow@height]{\width@tmp}{0.1\arrow@height}%
            \kern-\width@tmp%
            \rule[0.9\arrow@height]{\width@tmp}{0.1\arrow@height}%
            \if\third@@ ,%
               \kern-0.05\arrow@height%
               \rule[-0.5\arrow@height]{0.1\arrow@height}{1.5\arrow@height}%
            \else%
               \if\third@@ |%
                  \kern-0.05\arrow@height%
                  \rule[-0.5\arrow@height]{0.1\arrow@height}{2.25\arrow@height}%
               \else%
                  \if\third@@ `\xdef\third@@{'}\fi%
                  \if\third@@ '%
                     \kern-0.05\arrow@height%
                     \rule[0.4\arrow@height]{0.1\arrow@height}{1.5\arrow@height}%
                  \fi%
               \fi%
            \fi%
         \fi}}%
       \else
        \loopcount=0\relax%
        \width@tmp=\arrow@height%
        \temp@@length=\rule@@thick%
        \advance\width@tmp by -0.5\temp@@length%
        \if\first@@ o\xdef\first@@{O}\fi%
        \if\third@@ o\xdef\third@@{O}\fi%
        \textcolor{\f@color}{%
         \if\first@@ ,%
         \rule{\temp@@length}{\arrow@height}\kern-\temp@@length%
         \else%
            \if\first@@ |%
               \rule{\temp@@length}{2\arrow@height}\kern-\temp@@length%
            \else%
              \if\first@@ O%
                \raisebox{0.06ex}{\ensuremath{\bullet}}\kern-0.55ex%
                \rule[\width@tmp]{0.65ex}{\temp@@length}%
              \else%
               \if\first@@ <%
                 \raisebox{0.06ex}{\ensuremath{\blacktriangleleft}}\kern-0.35ex%
               \else
                \if\first@@ `\xdef\first@@{'}\fi%
                \if\first@@ '%
                  \advance\width@tmp by 0.5\temp@@length%
                  \rule[\width@tmp]{\temp@@length}{\arrow@height}\kern-\temp@@length%
                  \advance\width@tmp by -0.5\temp@@length%
                \fi%
               \fi%
              \fi%
            \fi%
         \fi%
         \loopcount=\second@%
         \if\first@@ <\advance\loopcount by -1\relax\ifnum\loopcount<0 \loopcount=0\fi\fi%
         \if\third@@ >\advance\loopcount by -1\relax\ifnum\loopcount<0 \loopcount=0\fi\fi%
         \if\first@@ O\advance\loopcount by -1\relax\ifnum\loopcount<0 \loopcount=0\fi\fi%
         \if\third@@ O\advance\loopcount by -1\relax\ifnum\loopcount<0 \loopcount=0\fi\fi%
         \xdef\second@{\the\loopcount}%
         \rule[\width@tmp]{\second@\box@width}{\temp@@length}%
         \setbox1=\hbox{\residues@size{\fill@char}}%
         \kern-\second@\box@width%
         \hbox to \second@\box@width{\textcolor{\backtext@color}{\hss\rule[\width@tmp]{1.2\wd1}{\temp@@length}\hss}}%
         \kern-\second@\box@width%
         \hbox to \second@\box@width{\textcolor{\frame@color}{\residues@size{\hss\fill@char\hss}}}%
         \if\third@@ ,%
         \kern-\temp@@length\rule{\temp@@length}{\arrow@height}%
         \else%
            \if\third@@ |%
               \kern-\temp@@length\rule{\temp@@length}{2\arrow@height}%
            \else%
              \if\third@@ O%
                \rule[\width@tmp]{0.65ex}{\temp@@length}%
                \kern-0.55ex\raisebox{0.06ex}{\ensuremath{\bullet}}%
              \else%
               \if\third@@ >%
                \kern-0.35ex\raisebox{0.06ex}{\ensuremath{\blacktriangleright}}%
               \else
                \if\third@@ `\xdef\third@@{'}\fi%
                \if\third@@ '%
                  \advance\width@tmp by 0.5\temp@@length%
                  \kern-\temp@@length\rule[\width@tmp]{\temp@@length}{\arrow@height}%
                \fi%
               \fi%
              \fi%
            \fi
         \fi}%
       \fi
      \fi
      \fi
      \fi
      \fi
     \fi
     \fi
     \fi
    \fi
   \fi
   \fi
   \fi
}
\def\next@featuretext&#1;#2;#3@{%
   \xdef\first@{#3@}%
   \xdef\last@{fill:#2:&}\expandafter\test@fill\last@%
   \ifx\f@color\comm@\xdef\f@color{Black}\fi%
   \fontfamily{\featuretext@family}%
   \fontseries{\featuretext@series}%
   \fontshape{\featuretext@shape}%
   \selectfont%
   \xdef\last@{translate}%
   \ifx\last@\second@@%
     \xdef\fill@char{\fill@char &}%
     \if\seq@type N
  \loop%
  \expandafter\get@firstfill\fill@char%
  \if\second@@ -\def\second@@{\hss}\fi%
      \hbox to \box@width{\hss\textcolor{\f@color}{%
                      \featuretext@size{\strut\second@@}}\hss}%
  \ifx\fill@char\ampers@nd\else\repeat%
 \else
   \hbox to #1\box@width{\hss%
    \temp@count=1%
    \loop%
      \expandafter\get@tripletfill\fill@char%
      \if\second@@ -\ifnum\tr@nstextstyle>0\hbox to \box@width{\hss}\fi%
  \else%
   \ifcase\tr@nstextstyle%
     \hbox{\transtext@size\textcolor{\f@color}%
      {\hss\second@@\second@@@\second@@@@\,\hss}}%
   \or%
     \ifnum\temp@count=1%
        \vbox{\transtext@size%
          \hbox to \box@width{\textcolor{\f@color}{\second@@\hss}}%
          \hbox to \box@width{\textcolor{\f@color}{\hss\second@@@\hss}}%
          \hbox to \box@width{\textcolor{\f@color}{\hss\second@@@@}}}%
     \else%
        \vbox{\transtext@size%
          \hbox to \box@width{\textcolor{\f@color}{\hss\second@@@@}}%
          \hbox to \box@width{\textcolor{\f@color}{\hss\second@@@\hss}}%
          \hbox to \box@width{\textcolor{\f@color}{\second@@\hss}}}%
     \fi%
   \or%
     \ifnum\temp@count=1%
        \vbox{\transtext@size%
          \hbox to \box@width{\textcolor{\f@color}%
                             {\hss\second@@\second@@@\second@@@@\hss}}%
          \hbox to \box@width{\textcolor{\f@color}{\hss}}}%
     \else%
        \vbox{\transtext@size%
          \hbox to \box@width{\textcolor{\f@color}{\hss}}%
          \hbox to \box@width{\textcolor{\f@color}%
                             {\hss\second@@\second@@@\second@@@@\hss}}}%
     \fi%
   \or%
        \vbox{\transtext@size%
          \hbox to \box@width{\textcolor{\f@color}{\second@@\hss}}%
          \hbox to \box@width{\textcolor{\f@color}{\hss\second@@@\hss}}%
          \hbox to \box@width{\textcolor{\f@color}{\hss\second@@@@}}}%
   \or%
        \vbox{\transtext@size%
          \hbox to \box@width{\textcolor{\f@color}{\hss\second@@\hss}}%
          \hbox to \box@width{\textcolor{\f@color}{\hss\second@@@\hss}}%
          \hbox to \box@width{\textcolor{\f@color}{\hss\second@@@@\hss}}}%
   \fi%
  \fi%
    \ifx\fill@char\ampers@nd\else\repeat%
   \hss}
 \fi%
   \else%
      \textcolor{\f@color}{%
      \hbox to #1\box@width{\hss\featuretext@size{\strut\fourth@}\hss}}%
   \fi%
   \advance\loopcount by -#1%
}
\def\put@featuretext{%
   \if\first@ @%
   \else
      \expandafter\next@featuretext\first@%
      \put@featuretext%
   \fi}
\def\next@featurestyle&#1;#2;#3@{%
   \xdef\first@{#2}%
   \xdef\second@{#1}%
   \ifx\first@\ampers@nd \hbox to \second@\box@width{\hss}%
   \else%
   \hbox to \second@\box@width%
           {\hss\expandafter\putfeature@style{\first@}\hss}\fi%
   \xdef\first@{#3@}%
   \advance\loopcount by -\second@%
}
\def\put@featurestyle{%
   \if\first@ @%
   \else
      \expandafter\next@featurestyle\first@%
      \put@featurestyle%
   \fi
}
\def\put@feature{%
   \vspace{-\baselineskip}%
   \newline\hbox{}%
   \xdef\temp@{ttttop}%
   \ifx\temp@\bottop@%
      \ifnames@right\else\ifnames@%
      \fontfamily{\ftext@family}\fontseries{\ftext@series}\fontshape{\ftext@shape}%
      \selectfont%
      \textcolor{\csname ftext@fg@\bottop@\endcsname}{\hbox to \name@width{\ftext@size\csname featuretextn@me\bottop@\endcsname\hss}}\fi\fi%
      \ifnumbers@left\ifnumbers@\hbox to \number@width{\hss}\fi\fi%
      \xdef\first@{\csname textfeature\bottop@\endcsname @}%
      \loopcount=\res@perline%
      \put@featuretext%
      \ifnumbers@right\ifnumbers@\hbox to \number@width{\hss}\fi\fi%
      \ifnames@right\ifnames@%
      \fontfamily{\ftext@family}\fontseries{\ftext@series}\fontshape{\ftext@shape}%
      \selectfont%
      \kern1em\hbox to \loopcount\box@width{\hss}%
      \textcolor{\csname ftext@fg@\bottop@\endcsname}{\hbox to \name@width{\ftext@size\csname featuretextn@me\bottop@\endcsname\hss}}\fi\fi%
   \else%
   \xdef\temp@{tttop}%
   \ifx\temp@\bottop@%
      \ifnames@right\else\ifnames@%
      \fontfamily{\ftext@family}\fontseries{\ftext@series}\fontshape{\ftext@shape}%
      \selectfont%
      \textcolor{\csname ftext@fg@\bottop@\endcsname}{\hbox to \name@width{\ftext@size\csname featuretextn@me\bottop@\endcsname\hss}}\fi\fi%
      \ifnumbers@left\ifnumbers@\hbox to \number@width{\hss}\fi\fi%
      \xdef\first@{\csname textfeature\bottop@\endcsname @}%
      \loopcount=\res@perline%
      \put@featuretext%
      \ifnumbers@right\ifnumbers@\hbox to \number@width{\hss}\fi\fi%
      \ifnames@right\ifnames@%
      \fontfamily{\ftext@family}\fontseries{\ftext@series}\fontshape{\ftext@shape}%
      \selectfont%
      \kern1em\hbox to \loopcount\box@width{\hss}%
      \textcolor{\csname ftext@fg@\bottop@\endcsname}{\hbox to \name@width{\ftext@size\csname featuretextn@me\bottop@\endcsname\hss}}\fi\fi%
   \else%
   \xdef\temp@{ttop}%
   \ifx\temp@\bottop@%
      \ifnames@right\else\ifnames@%
      \fontfamily{\ftext@family}\fontseries{\ftext@series}\fontshape{\ftext@shape}%
      \selectfont%
      \textcolor{\csname ftext@fg@\bottop@\endcsname}{\hbox to \name@width{\ftext@size\csname featuretextn@me\bottop@\endcsname\hss}}\fi\fi%
      \ifnumbers@left\ifnumbers@\hbox to \number@width{\hss}\fi\fi%
      \xdef\first@{\csname textfeature\bottop@\endcsname @}%
      \loopcount=\res@perline%
      \put@featuretext%
      \ifnumbers@right\ifnumbers@\hbox to \number@width{\hss}\fi\fi%
      \ifnames@right\ifnames@%
      \fontfamily{\ftext@family}\fontseries{\ftext@series}\fontshape{\ftext@shape}%
      \selectfont%
      \kern1em\hbox to \loopcount\box@width{\hss}%
      \textcolor{\csname ftext@fg@\bottop@\endcsname}{\hbox to \name@width{\ftext@size\csname featuretextn@me\bottop@\endcsname\hss}}\fi\fi%
   \else%
      \xdef\temp@{top}%
      \ifx\temp@\bottop@%
         \ifnames@right\else\ifnames@%
         \fontfamily{\ftext@family}\fontseries{\ftext@series}\fontshape{\ftext@shape}%
         \selectfont%
         \textcolor{\csname ftext@fg@\bottop@\endcsname}{\hbox to \name@width{\ftext@size\csname featuretextn@me\bottop@\endcsname\hss}}\fi\fi%
         \ifnumbers@left\ifnumbers@\hbox to \number@width{\hss}\fi\fi%
         \xdef\first@{\csname textfeature\bottop@\endcsname @}%
         \loopcount=\res@perline%
         \put@featuretext%
         \ifnumbers@right\ifnumbers@\hbox to \number@width{\hss}\fi\fi%
         \ifnames@right\ifnames@%
         \fontfamily{\ftext@family}\fontseries{\ftext@series}\fontshape{\ftext@shape}%
         \selectfont%
         \kern1em\hbox to \loopcount\box@width{\hss}%
         \textcolor{\csname ftext@fg@\bottop@\endcsname}{\hbox to \name@width{\ftext@size\csname featuretextn@me\bottop@\endcsname\hss}}\fi\fi%
      \else
         \ifnames@right\else\ifnames@%
         \fontfamily{\fstyles@family}\fontseries{\fstyles@series}\fontshape{\fstyles@shape}%
         \selectfont%
         \textcolor{\csname fstyles@fg@\bottop@\endcsname}{\hbox to \name@width{\fstyles@size\csname featurestylesn@me\bottop@\endcsname\hss}}\fi\fi%
         \ifnumbers@left\ifnumbers@\hbox to \number@width{\hss}\fi\fi%
         \xdef\first@{\csname stylefeature\bottop@\endcsname @}%
         \loopcount=\res@perline%
         \put@featurestyle%
         \ifnumbers@right\ifnumbers@\hbox to \number@width{\hss}\fi\fi%
         \ifnames@right\ifnames@%
         \fontfamily{\fstyles@family}\fontseries{\fstyles@series}\fontshape{\fstyles@shape}%
         \selectfont%
         \kern1em\hbox to \loopcount\box@width{\hss}%
         \textcolor{\csname fstyles@fg@\bottop@\endcsname}{\hbox to \name@width{\fstyles@size\csname featurestylesn@me\bottop@\endcsname\hss}}\fi\fi%
   \fi\fi\fi\fi%
   \newline\hbox{}%
   \ifx\temp@\bottop@%
      \ifnames@right\else\ifnames@%
      \fontfamily{\fstyles@family}\fontseries{\fstyles@series}\fontshape{\fstyles@shape}%
      \selectfont%
      \textcolor{\csname fstyles@fg@\bottop@\endcsname}{\hbox to \name@width{\fstyles@size\csname featurestylesn@me\bottop@\endcsname\hss}}\fi\fi%
      \ifnumbers@left\ifnumbers@\hbox to \number@width{\hss}\fi\fi%
      \xdef\first@{\csname stylefeature\bottop@\endcsname @}%
      \loopcount=\res@perline%
      \put@featurestyle%
      \ifnumbers@right\ifnumbers@\hbox to \number@width{\hss}\fi\fi%
      \ifnames@right\ifnames@%
      \fontfamily{\fstyles@family}\fontseries{\fstyles@series}\fontshape{\fstyles@shape}%
      \selectfont%
      \kern1em\hbox to \loopcount\box@width{\hss}%
      \textcolor{\csname fstyles@fg@\bottop@\endcsname}{\hbox to \name@width{\fstyles@size\csname featurestylesn@me\bottop@\endcsname\hss}}\fi\fi%
   \else%
      \xdef\temp@{ttop}%
      \ifx\temp@\bottop@%
         \ifnames@right\else\ifnames@%
         \fontfamily{\fstyles@family}\fontseries{\fstyles@series}\fontshape{\fstyles@shape}%
         \selectfont%
         \textcolor{\csname fstyles@fg@\bottop@\endcsname}{\hbox to \name@width{\fstyles@size\csname featurestylesn@me\bottop@\endcsname\hss}}\fi\fi%
         \ifnumbers@left\ifnumbers@\hbox to \number@width{\hss}\fi\fi%
         \xdef\first@{\csname stylefeature\bottop@\endcsname @}%
         \loopcount=\res@perline%
         \put@featurestyle%
         \ifnumbers@right\ifnumbers@\hbox to \number@width{\hss}\fi\fi%
         \ifnames@right\ifnames@%
         \fontfamily{\fstyles@family}\fontseries{\fstyles@series}\fontshape{\fstyles@shape}%
         \selectfont%
         \kern1em\hbox to \loopcount\box@width{\hss}%
         \textcolor{\csname fstyles@fg@\bottop@\endcsname}{\hbox to \name@width{\fstyles@size\csname featurestylesn@me\bottop@\endcsname\hss}}\fi\fi%
      \else
      \xdef\temp@{tttop}%
      \ifx\temp@\bottop@%
         \ifnames@right\else\ifnames@%
         \fontfamily{\fstyles@family}\fontseries{\fstyles@series}\fontshape{\fstyles@shape}%
         \selectfont%
         \textcolor{\csname fstyles@fg@\bottop@\endcsname}{\hbox to \name@width{\fstyles@size\csname featurestylesn@me\bottop@\endcsname\hss}}\fi\fi%
         \ifnumbers@left\ifnumbers@\hbox to \number@width{\hss}\fi\fi%
         \xdef\first@{\csname stylefeature\bottop@\endcsname @}%
         \loopcount=\res@perline%
         \put@featurestyle%
         \ifnumbers@right\ifnumbers@\hbox to \number@width{\hss}\fi\fi%
         \ifnames@right\ifnames@%
         \fontfamily{\fstyles@family}\fontseries{\fstyles@series}\fontshape{\fstyles@shape}%
         \selectfont%
         \kern1em\hbox to \loopcount\box@width{\hss}%
         \textcolor{\csname fstyles@fg@\bottop@\endcsname}{\hbox to \name@width{\fstyles@size\csname featurestylesn@me\bottop@\endcsname\hss}}\fi\fi%
      \else
      \xdef\temp@{ttttop}%
      \ifx\temp@\bottop@%
         \ifnames@right\else\ifnames@%
         \fontfamily{\fstyles@family}\fontseries{\fstyles@series}\fontshape{\fstyles@shape}%
         \selectfont%
         \textcolor{\csname fstyles@fg@\bottop@\endcsname}{\hbox to \name@width{\fstyles@size\csname featurestylesn@me\bottop@\endcsname\hss}}\fi\fi%
         \ifnumbers@left\ifnumbers@\hbox to \number@width{\hss}\fi\fi%
         \xdef\first@{\csname stylefeature\bottop@\endcsname @}%
         \loopcount=\res@perline%
         \put@featurestyle%
         \ifnumbers@right\ifnumbers@\hbox to \number@width{\hss}\fi\fi%
         \ifnames@right\ifnames@%
         \fontfamily{\fstyles@family}\fontseries{\fstyles@series}\fontshape{\fstyles@shape}%
         \selectfont%
         \kern1em\hbox to \loopcount\box@width{\hss}%
         \textcolor{\csname fstyles@fg@\bottop@\endcsname}{\hbox to \name@width{\fstyles@size\csname featurestylesn@me\bottop@\endcsname\hss}}\fi\fi%
      \else
         \ifnames@right\else\ifnames@%
         \fontfamily{\ftext@family}\fontseries{\ftext@series}\fontshape{\ftext@shape}%
         \selectfont%
         \textcolor{\csname ftext@fg@\bottop@\endcsname}{\hbox to \name@width{\ftext@size\csname featuretextn@me\bottop@\endcsname\hss}}\fi\fi%
         \ifnumbers@left\ifnumbers@\hbox to \number@width{\hss}\fi\fi%
         \xdef\first@{\csname textfeature\bottop@\endcsname @}%
         \xdef\first@{\csname textfeature\bottop@\endcsname @}%
         \loopcount=\res@perline%
         \put@featuretext%
         \ifnumbers@right\ifnumbers@\hbox to \number@width{\hss}\fi\fi%
         \ifnames@right\ifnames@%
         \fontfamily{\ftext@family}\fontseries{\ftext@series}\fontshape{\ftext@shape}%
         \selectfont%
         \kern1em\hbox to \loopcount\box@width{\hss}%
         \textcolor{\csname ftext@fg@\bottop@\endcsname}{\hbox to \name@width{\ftext@size\csname featuretextn@me\bottop@\endcsname\hss}}\fi\fi%
   \fi\fi\fi\fi%
   \newline\hbox{}%
}
\def\put@@@frame#1{%
   \xdef\last@{#1[,]&}\expandafter\opt@color\last@%
   \xdef\second@@{\fourth@}%
   \ifx\f@color\comm@%
      \xdef\third@@{0.2\box@width}%
   \else%
      \xdef\third@@{\f@color}%
   \fi%
   \setlength\arrow@width{\temp@@length}%
   \advance\arrow@width by -\third@@%
   \textcolor{\second@@}{%
      \rule{\second@\box@width}{\third@@}%
      \kern-\second@\box@width%
      \rule{\third@@}{\arrow@width}%
      \kern-\third@@%
      \rule[\arrow@width]{\second@\box@width}{\third@@}%
      \kern-\third@@%
      \rule{\third@@}{\arrow@width}}%
}
\def\next@frame&#1;#2;#3@{%
   \xdef\first@{#2}%
   \xdef\second@{#1}%
   \ifx\first@\ampers@nd \hbox to \second@\box@width{\hss}%
   \else%
      \expandafter\put@@@frame{\first@}%
   \fi%
   \xdef\first@{#3@}%
}
\def\put@@frame{%
   \if\first@ @%
   \else
      \expandafter\next@frame\first@%
      \put@@frame%
   \fi%
}
\def\put@frame{%
   \ifx\hide@seqs\n@%
   \ifshow@cons\ifnum\cons@top=1 \vspace{-\baselineskip}\fi\fi%
   \temp@count=\seq@count%
   \ifx\hide@seqs\y@
      \advance\temp@count by -1
   \else
   \loopcount=1%
   \xdef\first@{true}%
   \loop%
      \expandafter\ifx\csname hide@seq\the\loopcount\endcsname\first@%
         \advance\temp@count by -1\fi%
      \advance\loopcount by 1%
   \ifnum\loopcount>\seq@count\else\repeat%
   \fi
   \ifnum\temp@count>0%
   \setlength\temp@@length{\box@height}%
   \advance\temp@@length by \box@depth%
   \setlength\arrow@height{0.5\temp@@length}%
   \setlength\temp@@length{\temp@count\temp@@length}%
   \advance\temp@@length by \arrow@height%
   \setlength\arrow@width{\sep@space}%
   \setlength\arrow@width{\seq@gap@num\arrow@width}%
   \advance\temp@@length by \arrow@width%
   \vspace{\arrow@height}%
   \vspace{-\temp@@length}%
   \vspace{-\baselineskip}%
   \newline\hbox{}%
   \ifnames@right\else\ifnames@\hbox to \name@width{\hss}\fi\fi%
   \ifnumbers@left\ifnumbers@\hbox to \number@width{\hss}\fi\fi%
   \xdef\first@{\styleframe @}%
   \put@@frame%
   \vspace{-\arrow@height}%
   \ifshow@cons\ifnum\cons@top=1 \vspace{\baselineskip}\fi\fi%
   \newline\hbox{}%
   \fi%
   \fi%
}
\def\decimal@A#1#2@{%
   \def\decimal@AB##1##2@{\xdef\temp@{##1.##2}}%
   \xdef\temp@@{#1}%
   \xdef\temp@{#2@}%
   \expandafter\decimal@AB\temp@%
   \xdef\temp@{\temp@@\temp@}%
}
\def\decimal@B#1#2@{\xdef\temp@{#1.#2}}
\def\decimal@C#1#2@{\xdef\temp@{0.#1#2}}
\def\decimal@D#1#2@{\xdef\temp@{0.0#1#2}}
\def\decimal@E#1#2@{\xdef\temp@{0.00#1#2}}
\def\correct@CGSO{%
   \setlength\temp@@length{0.12\logo@height}%
   \multiply\temp@@length by \third@ %
   \if\seq@type P %
      \divide\temp@@length by 4322 % 
   \else%
      \divide\temp@@length by 4000 %
   \fi%
   \loopcount=\third@%
   \multiply\loopcount by 94 %
   \divide\loopcount by 100 %
   \xdef\third@{\the\loopcount}%
}
\def\correct@Q{%
   \setlength\temp@@length{0.063\logo@height}%
   \multiply\temp@@length by \third@%
   \if\seq@type P %
      \divide\temp@@length by 4322 % 
   \else%
      \divide\temp@@length by 4000 %
   \fi%
   \multiply\temp@@length by 10 %
   \loopcount=\third@%
   \multiply\loopcount by 83 %
   \divide\loopcount by 100 %
   \xdef\third@{\the\loopcount}%
}
\def\correct@JUV{%
   \setlength\temp@@length{0.12\logo@height}%
   \multiply\temp@@length by \third@ %
   \if\seq@type P %
      \divide\temp@@length by 4322 % 
   \else%
      \divide\temp@@length by 4000 %
   \fi%
   \loopcount=\third@%
   \multiply\loopcount by 97 %
   \divide\loopcount by 100 %
   \xdef\third@{\the\loopcount}%
}
\def\next@logo#1:#2,#3@{%
   \xdef\first@{#1}\xdef\second@{#2}\xdef\last@{#3}%
   \ifx\first@\ampers@nd%
     \ifx\clear@logo\n@%
     \ifx\hide@sig\n@%
      \expandafter\firstchar@get\sublogo@sig%
      \xdef\sublogo@sig{\third@}%
      \ifx\first@\y@%
         \loopcount=\temp@count%
         \ifnum\loopcount<0 \loopcount=0\fi%
         \xdef\fourth@{\the\loopcount @}%
         \ifnum\loopcount>9999 \expandafter\decimal@A\fourth@ \else%
         \ifnum\loopcount>999  \expandafter\decimal@B\fourth@ \else%
         \ifnum\loopcount>99   \expandafter\decimal@C\fourth@ \else%
         \ifnum\loopcount>9    \expandafter\decimal@D\fourth@ \else%
                               \expandafter\decimal@E\fourth@%
         \fi\fi\fi\fi%
         \raisebox{\temp@\logo@height}{%
         \hbox to \box@width{\textcolor{\sig@color}{\hss\residues@size{\sig@char}\hss}}}%
         \kern-\box@width%
      \fi%
     \fi%
     \fi%
     \temp@count=0%
     \kern\box@width%
     \advance\outerloopcount by 1 %
   \else%
    \ifnum#2=1%
      \temp@count=0%
    \else%
      \if\seq@type N%
         \loopcount=\second@%
         \multiply\loopcount by 2%
         \xdef\second@{\the\loopcount}%
      \fi%
      \xdef\third@{\second@}%
      \if\first@ C\correct@CGSO \else%
      \if\first@ G\correct@CGSO \else%
      \if\first@ S\correct@CGSO \else%
      \if\first@ O\correct@CGSO \else%
      \if\first@ Q\correct@Q    \else%
      \if\first@ J\correct@JUV  \else%
      \if\first@ U\correct@JUV  \else%
      \if\first@ V\correct@JUV  \else%
                  \setlength\temp@@length{0pt}%
      \fi\fi\fi\fi\fi\fi\fi\fi%
      \loopcount=\third@
      \multiply\loopcount by \logo@stretch@IOOO%
      \ifnum\loopcount>0 %
         \divide\loopcount by 1000 \xdef\fl@g{}%
         \xdef\tint@{}%
      \else%
         \divide\loopcount by -1000 \xdef\fl@g{-}%
         \xdef\tint@{\sublogo@tint}%
      \fi%
      \ifnum\loopcount=0\relax\loopcount=1\relax\fi
      \xdef\third@{\the\loopcount @}%
      \ifnum\loopcount>9999 \expandafter\decimal@A\third@ \else%
      \ifnum\loopcount>999  \expandafter\decimal@B\third@ \else%
      \ifnum\loopcount>99   \expandafter\decimal@C\third@ \else%
      \ifnum\loopcount>9    \expandafter\decimal@D\third@ \else%
                            \expandafter\decimal@E\third@%
      \fi\fi\fi\fi%
      \xdef\third@{\fl@g\temp@}%
      \ifnum\temp@count<0 %
         \ifnum\second@<0 \else\temp@count=0 \fi%
      \fi%
      \ifnum\temp@count>0 %
         \loopcount=\temp@count \xdef\fl@g{}%
      \else
         \loopcount=-\temp@count \xdef\fl@g{-}%
      \fi%
      \xdef\fourth@{\the\loopcount @}%
      \ifnum\loopcount>9999 \expandafter\decimal@A\fourth@ \else%
      \ifnum\loopcount>999  \expandafter\decimal@B\fourth@ \else%
      \ifnum\loopcount>99   \expandafter\decimal@C\fourth@ \else%
      \ifnum\loopcount>9    \expandafter\decimal@D\fourth@ \else%
                             \expandafter\decimal@E\fourth@%
      \fi\fi\fi\fi%
      \xdef\fourth@{\fl@g\temp@}%
      \if\first@ W%  
         \hbox to \box@width{\hss%
         \raisebox{\temp@@length}{\raisebox{\fourth@\logo@height}{\scalebox{\char@stretch@W}[\third@]%
         {\hbox to \box@width{\textcolor{\tint@\csname logo@col\first@\endcsname}{\hss\residues@size{W}\hss}}}}}\hss}%
         \kern-\box@width%
      \else%
         \hbox to \box@width{\hss%
         \raisebox{\temp@@length}{\raisebox{\fourth@\logo@height}{\scalebox{\char@stretch}[\third@]%
         {\hbox to \box@width{\textcolor{\tint@\csname logo@col\first@\endcsname}{\hss\residues@size{\first@}\hss}}}}}\hss}%
         \kern-\box@width%
      \fi%
      \advance\temp@count by \second@%
    \fi%
   \fi%
   \xdef\temp@{\last@.}%
   \ifx\temp@\d@t%
   \else%
      \xdef\last@{#3@}%
      \ifnum\outerloopcount>\res@perline%
      \else
         \expandafter\next@logo\last@%
      \fi%
   \fi%
}

\def\put@logo{%
   \ifx\clear@logo\y@%
   \ifnum\logo@top=1 %
      \vspace{-0.5\baselineskip}%
   \else
      \vspace{-0.75\baselineskip}%
   \fi%
   \else%
   \ifnum\sublogo@top=1 %
      \vspace{-0.5\baselineskip}%
   \else
      \vspace{-0.75\baselineskip}%
   \fi%
   \fi%
   \newline\hbox{}%
   \ifnames@right%
   \else%
      \fontfamily{\namestext@family}\fontseries{\namestext@series}\fontshape{\namestext@shape}%
      \selectfont%
      \ifnames@
         \textcolor{\names@fg}{%
            \raisebox{2\logo@height}{\hbox to \name@width{\namestext@size{\logo@name}\hss}}%
            \ifx\clear@logo\n@
            \ifx\hide@negatives\n@%
               \kern-\name@width\raisebox{-3\logo@height}{\hbox to \name@width{\namestext@size{\sublogo@name@neg}\hss}}%
            \fi%
            \fi}%
      \fi%
   \fi%
   \ifnumbers@left\ifnumbers@\hbox to \number@width{\hss}\fi\fi%
   \fontfamily{cmss}\fontseries{m}\fontshape{n}\selectfont%
   \ifx\show@logoscale\n@%
      \if\seq@type P%
         \ifnum\corr@max>600\relax%
            \rule[5.4\logo@height]{0pt}{0pt}%
         \else%
            \rule[4.4\logo@height]{0pt}{0pt}%
         \fi%
      \else%
         \rule[4.0688\logo@height]{0pt}{0pt}%
      \fi%
   \else%
   \xdef\first@{right}%
   \ifx\show@logoscale\first@%
   \else%
      \textcolor{\logo@scalecol}{%
      \kern-0.7\box@width%
      \rule[-0.05\logo@height]{0.4\box@width}{0.1\box@width}\kern-0.4\box@width%
      \rule[2\logo@height]{0.4\box@width}{0.1\box@width}\kern-0.4\box@width%
      \rule[4\logo@height]{0.4\box@width}{0.1\box@width}%
      \if\seq@type P%
         \kern-0.4\box@width%
         \rule[1\logo@height]{0.4\box@width}{0.1\box@width}\kern-0.4\box@width%
         \rule[3\logo@height]{0.4\box@width}{0.1\box@width}%
         \ifnum\corr@max>600\relax%
            \kern-0.4\box@width\rule[5\logo@height]{0.4\box@width}{0.1\box@width}\fi%
         \setbox1=\hbox{\bottomruler@size 2}%
         \kern-1.2\box@width%
         \ifdim\logo@height>1.25\ht1 \relax%
            \raisebox{-0.5\ht1}{\raisebox{1\logo@height}{\hbox to 0.6\box@width{\bottomruler@size\hss 1}}}%
            \kern-0.6\box@width%
            \raisebox{-0.5\ht1}{\raisebox{3\logo@height}{\hbox to 0.6\box@width{\bottomruler@size\hss 3}}}%
            \kern-0.6\box@width%
            \ifnum\corr@max>600\relax%
               \raisebox{-0.5\ht1}{\raisebox{5\logo@height}{\hbox to 0.6\box@width{\bottomruler@size\hss 5}}}%
               \kern-0.6\box@width%
            \fi
         \fi
         \raisebox{-0.5\ht1}{\raisebox{2\logo@height}{\hbox to 0.6\box@width{\bottomruler@size\hss 2}}}%
         \kern-0.6\box@width%
         \raisebox{-0.5\ht1}{\raisebox{4\logo@height}{\hbox to 0.6\box@width{\bottomruler@size\hss 4}}}%
         \kern0.6\box@width%
         \ifnum\corr@max>600\relax%
            \rule[5.4\logo@height]{0pt}{0pt}%
         \else
            \rule[4.4\logo@height]{0pt}{0pt}%
         \fi
      \else%
         \setbox1=\hbox{\bottomruler@size 2}%
         \kern-1.2\box@width%
         \raisebox{-0.5\ht1}{\raisebox{2\logo@height}{\hbox to 0.6\box@width{\bottomruler@size\hss 1}}}%
         \kern-0.6\box@width%
         \raisebox{-0.5\ht1}{\raisebox{4\logo@height}{\hbox to 0.6\box@width{\bottomruler@size\hss 2}}}%
         \kern0.6\box@width%
         \rule[4.0688\logo@height]{0pt}{0pt}%
      \fi%
      \kern-0.1\box@width%
      \ifnum\corr@max>600\relax%
         \rule{0.1\box@width}{5\logo@height}%
      \else%
         \rule{0.1\box@width}{4\logo@height}%
      \fi%
      \ifx\clear@logo\n@%
      \ifx\hide@negatives\n@%
      \kern-0.4\box@width%
      \rule[-2\logo@height]{0.4\box@width}{0.1\box@width}\kern-0.4\box@width%
      \rule[-4\logo@height]{0.4\box@width}{0.1\box@width}%
      \if\seq@type P%
         \kern-0.4\box@width%
         \rule[-\logo@height]{0.4\box@width}{0.1\box@width}\kern-0.4\box@width%
         \rule[-3\logo@height]{0.4\box@width}{0.1\box@width}%
         \ifnum\corr@max>600\relax%
            \kern-0.4\box@width\rule[-5\logo@height]{0.4\box@width}{0.1\box@width}\fi%
         \setbox1=\hbox{\bottomruler@size 2}%
         \kern-1.2\box@width%
         \ifdim\logo@height>1.25\ht1 \relax%
            \raisebox{-0.5\ht1}{\raisebox{-\logo@height}{\hbox to 0.6\box@width{\bottomruler@size\hss -1}}}%
            \kern-0.6\box@width%
            \raisebox{-0.5\ht1}{\raisebox{-3\logo@height}{\hbox to 0.6\box@width{\bottomruler@size\hss -3}}}%
            \kern-0.6\box@width%
            \ifnum\corr@max>600\relax%
               \raisebox{-0.5\ht1}{\raisebox{-5\logo@height}{\hbox to 0.6\box@width{\bottomruler@size\hss -5}}}%
               \kern-0.6\box@width%
            \fi
         \fi
         \raisebox{-0.5\ht1}{\raisebox{-2\logo@height}{\hbox to 0.6\box@width{\bottomruler@size\hss -2}}}%
         \kern-0.6\box@width%
         \raisebox{-0.5\ht1}{\raisebox{-4\logo@height}{\hbox to 0.6\box@width{\bottomruler@size\hss -4}}}%
         \kern0.6\box@width%
         \ifnum\corr@max>600\relax%
            \rule[-5.4\logo@height]{0pt}{0pt}%
         \else
            \rule[-4.4\logo@height]{0pt}{0pt}%
         \fi
      \else%
         \setbox1=\hbox{\bottomruler@size 2}%
         \kern-1.2\box@width%
         \raisebox{-0.5\ht1}{\raisebox{-2\logo@height}{\hbox to 0.6\box@width{\bottomruler@size\hss -1}}}%
         \kern-0.6\box@width%
         \raisebox{-0.5\ht1}{\raisebox{-4\logo@height}{\hbox to 0.6\box@width{\bottomruler@size\hss -2}}}%
         \kern0.6\box@width%
         \rule[-4.0688\logo@height]{0pt}{0pt}%
      \fi%
      \kern-0.1\box@width%
      \ifnum\corr@max>600\relax%
         \rule[-5\logo@height]{0.1\box@width}{5\logo@height}%
      \else%
         \rule[-4\logo@height]{0.1\box@width}{4\logo@height}%
      \fi%
      \fi%
      \fi
      \kern0.3\box@width}%
   \fi%
   \fi%
   \temp@count=0%
   \outerloopcount=1 \expandafter\next@logo\last@%
   \ifx\clear@logo\n@%
      \ifx\hide@negatives\n@%
         \textcolor{\logo@scalecol}{%
         \kern-\res@perline\box@width\rule[-0.03\box@width]{\res@perline\box@width}{0.06\box@width}}%
      \fi%
   \fi%
   \def\@kern{\kern1em}%
   \ifx\show@logoscale\n@%
   \else%
   \xdef\first@{left}%
   \ifx\show@logoscale\first@%
   \kern0.1\box@width%
   \else%
      \def\@kern{\kern2em}%
      \textcolor{\logo@scalecol}{%
      \kern0.3\box@width%
      \rule[-0.05\logo@height]{0.4\box@width}{0.1\box@width}\kern-0.4\box@width%
      \rule[2\logo@height]{0.4\box@width}{0.1\box@width}\kern-0.4\box@width%
      \rule[4\logo@height]{0.4\box@width}{0.1\box@width}%
      \if\seq@type P%
         \kern-0.4\box@width%
         \rule[1\logo@height]{0.4\box@width}{0.1\box@width}\kern-0.4\box@width%
         \rule[3\logo@height]{0.4\box@width}{0.1\box@width}%
         \ifnum\corr@max>600\relax%
            \kern-0.4\box@width\rule[5\logo@height]{0.4\box@width}{0.1\box@width}\fi%
         \setbox1=\hbox{\bottomruler@size 2}%
         \kern0.2\box@width%
         \ifdim\logo@height>1.25\ht1 \relax%
            \raisebox{-0.5\ht1}{\raisebox{1\logo@height}{\hbox to 0.6\box@width{\bottomruler@size 1 \hss}}}%
            \kern-0.6\box@width%
            \raisebox{-0.5\ht1}{\raisebox{3\logo@height}{\hbox to 0.6\box@width{\bottomruler@size 3 \hss}}}%
            \kern-0.6\box@width%
            \ifnum\corr@max>600\relax%
               \raisebox{-0.5\ht1}{\raisebox{5\logo@height}{\hbox to 0.6\box@width{\bottomruler@size 5 \hss}}}%
               \kern-0.6\box@width%
            \fi
         \fi
         \raisebox{-0.5\ht1}{\raisebox{2\logo@height}{\hbox to 0.6\box@width{\bottomruler@size 2 \hss}}}%
         \kern-0.6\box@width%
         \raisebox{-0.5\ht1}{\raisebox{4\logo@height}{\hbox to 0.6\box@width{\bottomruler@size 4 \hss}}}%
         \kern-0.8\box@width%
         \ifnum\corr@max>600\relax%
            \rule[5.4\logo@height]{0pt}{0pt}%
         \else
            \rule[4.4\logo@height]{0pt}{0pt}%
         \fi
      \else%
         \setbox1=\hbox{\bottomruler@size 2}%
         \kern0.2\box@width%
         \raisebox{-0.5\ht1}{\raisebox{2\logo@height}{\hbox to 0.6\box@width{\bottomruler@size 1 \hss}}}%
         \kern-0.6\box@width%
         \raisebox{-0.5\ht1}{\raisebox{4\logo@height}{\hbox to 0.6\box@width{\bottomruler@size 2 \hss}}}%
         \kern-0.8\box@width%
         \rule[4.0688\logo@height]{0pt}{0pt}%
      \fi%
      \kern-0.4\box@width%
      \ifnum\corr@max>600\relax%
         \rule{0.1\box@width}{5\logo@height}%
      \else%
         \rule{0.1\box@width}{4\logo@height}%
      \fi%
      \ifx\clear@logo\n@%
      \ifx\hide@negatives\n@%
      \kern-0.1\box@width%
      \rule[-2\logo@height]{0.4\box@width}{0.1\box@width}\kern-0.4\box@width%
      \rule[-4\logo@height]{0.4\box@width}{0.1\box@width}%
      \if\seq@type P%
         \kern-0.4\box@width%
         \rule[-\logo@height]{0.4\box@width}{0.1\box@width}\kern-0.4\box@width%
         \rule[-3\logo@height]{0.4\box@width}{0.1\box@width}%
         \ifnum\corr@max>600\relax%
            \kern-0.4\box@width\rule[-5\logo@height]{0.4\box@width}{0.1\box@width}\fi%
         \setbox1=\hbox{\bottomruler@size 2}%
         \kern0.2\box@width%
         \ifdim\logo@height>1.25\ht1 \relax%
            \raisebox{-0.5\ht1}{\raisebox{-\logo@height}{\hbox to 0.6\box@width{\bottomruler@size -1 \hss}}}%
            \kern-0.6\box@width%
            \raisebox{-0.5\ht1}{\raisebox{-3\logo@height}{\hbox to 0.6\box@width{\bottomruler@size -3 \hss}}}%
            \kern-0.6\box@width%
            \ifnum\corr@max>600\relax%
               \raisebox{-0.5\ht1}{\raisebox{-5\logo@height}{\hbox to 0.6\box@width{\bottomruler@size -5 \hss}}}%
               \kern-0.6\box@width%
            \fi
         \fi
         \raisebox{-0.5\ht1}{\raisebox{-2\logo@height}{\hbox to 0.6\box@width{\bottomruler@size -2 \hss}}}%
         \kern-0.6\box@width%
         \raisebox{-0.5\ht1}{\raisebox{-4\logo@height}{\hbox to 0.6\box@width{\bottomruler@size -4 \hss}}}%
         \kern-0.8\box@width%
         \ifnum\corr@max>600\relax%
            \rule[-5.4\logo@height]{0pt}{0pt}%
         \else
            \rule[-4.4\logo@height]{0pt}{0pt}%
         \fi
      \else%
         \setbox1=\hbox{\bottomruler@size 2}%
         \kern0.2\box@width%
         \raisebox{-0.5\ht1}{\raisebox{-2\logo@height}{\hbox to 0.6\box@width{\bottomruler@size -1 \hss}}}%
         \kern-0.6\box@width%
         \raisebox{-0.5\ht1}{\raisebox{-4\logo@height}{\hbox to 0.6\box@width{\bottomruler@size -2 \hss}}}%
         \kern-0.8\box@width%
         \rule[-4.0688\logo@height]{0pt}{0pt}%
      \fi%
      \kern-0.4\box@width%
      \ifnum\corr@max>600\relax%
         \rule[-5\logo@height]{0.1\box@width}{5\logo@height}%
      \else%
         \rule[-4\logo@height]{0.1\box@width}{4\logo@height}%
      \fi%
      \fi%
      \fi%
      \kern-0.3\box@width}%
   \fi%
   \fi%
   \ifnumbers@right\ifnumbers@\hbox to \number@width{\hss}\def\@kern{\kern1em}\fi\fi%
   \ifnames@right%
      \fontfamily{\namestext@family}\fontseries{\namestext@series}\fontshape{\namestext@shape}%
      \selectfont%
      \ifnames@
         \textcolor{\names@fg}{%
            \raisebox{2\logo@height}{\hbox to \name@width{\namestext@size{\@kern\logo@name}\hss}}%
            \ifx\clear@logo\n@
            \ifx\hide@negatives\n@%
               \kern-\name@width\raisebox{-3\logo@height}{\hbox to \name@width{\namestext@size{\@kern\sublogo@name@neg}\hss}}%
            \fi%
            \fi}%
      \fi%
   \fi%
   \ifx\clear@logo\y@\xdef\stack@sequencelogo{}%
   \else%
      \ifx\hide@negatives\n@ \newline\hbox{}\fi%
   \fi%
   \newline\hbox{}%
}

\def\set@lines{%
   \pos@count=1
   \xdef\frame@pos{1}\xdef\bar@pos{0}%
   \xdef\featurepostop{1}    \xdef\featureposbottom{1}%
   \xdef\featureposttop{1}   \xdef\featureposbbottom{1}%
   \xdef\featurepostttop{1}  \xdef\featureposbbbottom{1}%
   \xdef\featureposttttop{1} \xdef\featureposbbbbottom{1}%
   \findc@nsensus%
   \noindent%
   \ifnum\feature@ttttop=1
      \ifnum\featureonttttop=0 \xdef\feature@ttttop{0} \fi
      \xdef\bottop@{ttttop}%
      \put@feature%
      \vspace{\tttt@sp@ce}%
      \vspace{-\baselineskip}%
      \newline\hbox{}%
   \else
      \iffix@\ifttttopfeature \vspace{\tttt@sp@ce}\newline\hbox{}\newline\hbox{}\fi\fi
   \fi
   \ifnum\feature@tttop=1
      \ifnum\featureontttop=0 \xdef\feature@tttop{0} \fi
      \xdef\bottop@{tttop}%
      \put@feature%
      \vspace{\ttt@sp@ce}%
      \vspace{-\baselineskip}%
      \newline\hbox{}%
   \else
      \iffix@\iftttopfeature \vspace{\ttt@sp@ce}\newline\hbox{}\newline\hbox{}\fi\fi
   \fi
   \ifnum\feature@ttop=1
      \ifnum\featureonttop=0 \xdef\feature@ttop{0} \fi
      \xdef\bottop@{ttop}%
      \put@feature%
      \vspace{\tt@sp@ce}%
      \vspace{-\baselineskip}%
      \newline\hbox{}%
   \else
      \iffix@\ifttopfeature \vspace{\tt@sp@ce}\newline\hbox{}\newline\hbox{}\fi\fi
   \fi
   \ifnum\feature@top=1
      \ifnum\featureontop=0 \xdef\feature@top{0} \fi
      \xdef\bottop@{top}%
      \put@feature%
      \vspace{\t@sp@ce}%
      \vspace{-\baselineskip}%
      \newline\hbox{}%
   \else
      \iffix@\iftopfeature \vspace{\t@sp@ce}\newline\hbox{}\newline\hbox{}\fi\fi
   \fi
   \ifnum\rule@num<0 \else \ifnum\rule@top=0 \loopcount=0 \put@ruler \fi\fi
   \ifshow@logo\ifnum\logo@top=0
      \xdef\last@{\stack@sequencelogo @}%
      \ifx\logo@name@user\ampers@nd\xdef\logo@name{logo}\else\xdef\logo@name{\logo@name@user}\fi
      \xdef\clear@logo{y}%
      \put@logo%
      \xdef\clear@logo{n}%
   \fi\fi
   \ifshow@sublogo\ifnum\sublogo@top=0
      \read@sublogo%
      \xdef\last@{\stack@sublogo @}%
      \ifx\sublogo@name@user\ampers@nd\xdef\logo@name{subfamily}\else\xdef\logo@name{\sublogo@name@user}\fi
      \put@logo%
   \fi\fi
   \set@consensus
   \ifshow@cons\ifnum\cons@top=0
      \xdef\cons@now{y}%
      \xdef\first@{\consensus @}%
      \xdef\first@@@{\c@nscol @}%
      \loopcount=0\relax%
      \put@line%
      \xdef\c@nscol{}%
      \xdef\cons@now{no}%
   \fi\fi
   \ifx\hide@seqs\n@%
   \loopcount=0 %
   \loop
     \advance\loopcount by 1 %
     \xdef\second@{\csname hide@seq\the\loopcount\endcsname}%
     \xdef\first@{noshade}\nosh@defalse%
     \ifx\second@\first@ \nosh@detrue \xdef\second@{false}\fi
     \xdef\first@{false}%
     \ifx\second@\first@
       \ifnames@right
         \ifnumbers@right
            \expandafter\xdef\csname seq\the\loopcount\endcsname{%
               \csname seq\the\loopcount\endcsname%
               \csname res@count\the\loopcount\endcsname)%
               <\csname newseqname\the\loopcount\endcsname>}%
         \else
            \expandafter\xdef\csname seq\the\loopcount\endcsname{%
               \csname seq\the\loopcount\endcsname%
               <\csname newseqname\the\loopcount\endcsname>}%
         \fi
        \else
         \ifnumbers@right
            \expandafter\xdef\csname seq\the\loopcount\endcsname{%
               \csname seq\the\loopcount\endcsname%
               \csname res@count\the\loopcount\endcsname)}%
         \fi
       \fi
       \xdef\first@{\csname seq\the\loopcount\endcsname @}%
       \expandafter\ifx\csname seq@gap\the\loopcount\endcsname\yes%
       \ifnum\finger@linenum=0\seq@skip%
       \else%
           \ifnum\loopcount<\seq@count\seq@skip\fi%
       \fi%
     \fi%
     \put@line%
     \fi%
   \ifnum\loopcount<\seq@count\repeat%
   \fi%
   \ifshow@cons\ifnum\cons@top=1 %
      \xdef\cons@now{y}%
      \xdef\first@{\consensus @}%
      \xdef\first@@@{\c@nscol @}%
      \loopcount=0\relax%
      \put@line%
      \xdef\c@nscol{}%
      \xdef\cons@now{no}%
   \fi\fi%
   \ifnum\frame@=1 %
      \ifnum\frame@on=0 \xdef\frame@{0}\fi%
      \put@frame%
   \fi%
   \ifshow@logo\ifnum\logo@top=1 %
      \xdef\last@{\stack@sequencelogo @}%
      \ifx\logo@name@user\ampers@nd\xdef\logo@name{logo}\else\xdef\logo@name{\logo@name@user}\fi
      \xdef\clear@logo{y}%
      \put@logo%
      \xdef\clear@logo{n}%
   \fi\fi%
   \ifshow@sublogo\ifnum\sublogo@top=1
      \read@sublogo%
      \xdef\last@{\stack@sublogo @}
      \ifx\sublogo@name@user\ampers@nd\xdef\logo@name{subfamily}\else\xdef\logo@name{\sublogo@name@user}\fi
      \put@logo%
   \fi\fi%
   \ifnum\rule@num<0 %
   \else%
      \ifnum\rule@top=1 %
         \loopcount=0\relax%
         \ifnum\ruler@rot=0 %
         \else \vspace{\ruler@width}\vspace{-1.75\baselineskip}\newline\hbox{}%
         \fi
         \put@ruler%
         \ifnum\ruler@rot=0 \vspace{0.25\baselineskip}\fi%
      \fi%
   \fi%
   \ifnum\feature@bottom=1 %
      \ifnum\featureonbottom=0 \xdef\feature@bottom{0}\fi%
      \xdef\bottop@{bottom}%
      \vspace{\b@sp@ce}%
      \if\bottom@stretch y%
         \vspace{-\box@height}%
         \vspace{\b@r@stretch\box@height}%
      \fi%
      \put@feature%
   \else
      \iffix@
      \if\bottom@stretch y%
         \vspace{-\box@height}%
         \vspace{\b@r@stretch\box@height}%
      \fi%
      \ifbottomfeature%
         \vspace{\b@sp@ce}\newline\hbox{}\newline\hbox{}%
      \fi%
      \fi%
   \fi%
   \ifnum\feature@bbottom=1 %
      \ifnum\featureonbbottom=0 \xdef\feature@bbottom{0}\fi%
      \xdef\bottop@{bbottom}%
      \vspace{\bb@sp@ce}%
      \if\bbottom@stretch y%
         \vspace{-\box@height}%
         \vspace{\b@r@stretch\box@height}%
      \fi%
      \put@feature%
   \else
      \iffix@
      \if\bbottom@stretch y%
         \vspace{-\box@height}%
         \vspace{\b@r@stretch\box@height}%
      \fi%
      \ifbbottomfeature
         \vspace{\bb@sp@ce}\newline\hbox{}\newline\hbox{}%
      \fi
      \fi
   \fi
   \ifnum\feature@bbbottom=1 %
      \ifnum\featureonbbbottom=0 \xdef\feature@bbbottom{0}\fi%
      \xdef\bottop@{bbbottom}%
      \vspace{\bbb@sp@ce}%
      \if\bbbottom@stretch y%
         \vspace{-\box@height}%
         \vspace{\b@r@stretch\box@height}%
      \fi%
      \put@feature%
   \else
      \iffix@
      \if\bbbottom@stretch y%
         \vspace{-\box@height}%
         \vspace{\b@r@stretch\box@height}%
      \fi%
      \ifbbbottomfeature
         \vspace{\bbb@sp@ce}\newline\hbox{}\newline\hbox{}%
      \fi
      \fi
   \fi
   \ifnum\feature@bbbbottom=1 %
      \ifnum\featureonbbbbottom=0 \xdef\feature@bbbbottom{0}\fi%
      \xdef\bottop@{bbbbottom}%
      \vspace{\bbbb@sp@ce}%
      \if\bbbbottom@stretch y%
         \vspace{-\box@height}%
         \vspace{\b@r@stretch\box@height}%
      \fi%
      \put@feature%
   \else
      \iffix@
      \if\bbbbottom@stretch y%
         \vspace{-\box@height}%
         \vspace{\b@r@stretch\box@height}%
      \fi%
      \ifbbbbottomfeature
         \vspace{\bbbb@sp@ce}\newline\hbox{}\newline\hbox{}%
      \fi
      \fi
   \fi
   \xdef\consensus{} \xdef\ruler@{}
   \xdef\styleframe{}         
   \xdef\textfeaturetop{}     \xdef\textfeaturebottom{}
   \xdef\textfeaturettop{}    \xdef\textfeaturebbottom{}
   \xdef\textfeaturetttop{}   \xdef\textfeaturebbbottom{}
   \xdef\textfeaturettttop{}  \xdef\textfeaturebbbbottom{}
   \xdef\stylefeaturetop{}    \xdef\stylefeaturebottom{}
   \xdef\stylefeaturettop{}   \xdef\stylefeaturebbottom{}
   \xdef\stylefeaturetttop{}  \xdef\stylefeaturebbbottom{}
   \xdef\stylefeaturettttop{} \xdef\stylefeaturebbbbottom{}
   \loopcount=0
   \loop
         \advance\loopcount by 1
         \xdef\seq@line{\csname sequence\the\loopcount\endcsname}
         \expandafter\remove@fromseq\seq@line
   \ifnum\loopcount<\seq@count \repeat}

\def\block@output{%
   \expandafter\ifnum\csname res@count\start@seq\endcsname<\end@num\relax
     \message{.}
     \ifx\out@put\y@
        \vbox{\set@lines}\par
        \block@skip
     \else
        \pos@count=1
        \findc@nsensus
        \loopcount=0
        \loop
           \advance\loopcount by 1\relax
           \xdef\seq@line{\csname sequence\the\loopcount\endcsname}
           \expandafter\remove@fromseq\seq@line
           \ifT@coffee
              \xdef\TC@cons{\csname TC0\endcsname}
              \expandafter\remove@fromTC\TC@cons
           \fi
        \ifnum\loopcount<\seq@count \repeat
     \fi
     \ifstop@
     \else
        \advance\seq@pointer by -\res@perline
        \ifnum\seq@pointer>\res@perline \block@output \fi
     \fi
   \fi}

%%%%% Basic input routines

\def\savedseqlength#1#2#3{%
   \expandafter\xdef\csname savelength#1seq#2\endcsname{#3}}

\def\set@savedseqlength{%
   \loopcount=0
   \loop
      \advance\loopcount by 1
      \expandafter\ifx%
         \csname savelength\@lign@count seq\the\loopcount\endcsname\relax 
      \else
         \expandafter\xdef\csname seq@len\the\loopcount\endcsname{%
            \csname savelength\@lign@count seq\the\loopcount\endcsname}
      \fi
   \ifnum\loopcount<\seq@count\repeat
}
   
\def\save@lengths{%
   \loopcount=0
   \loop
       \advance\loopcount by 1\relax
       \immediate\write\@auxout{%
       \string\savedseqlength{\@lign@count}{\the\loopcount}% 
           {\csname res@count\the\loopcount\endcsname}}
   \ifnum\loopcount<\seq@count \repeat
}
   
\def\do@cleanup	 
	 \else
            \xdef\second@{\second@\noexpand\string\first@}
	 \fi\fi
      \fi
   \fi
   \ifx\third@\@t \else \do@cleanup \fi
}
    
\def\cleanup@name{%
   \xdef\third@{\csname seqname\the\loopcount\endcsname @}
   \xdef\second@{}
   \do@cleanup
   \expandafter\xdef\csname newseqname\the\loopcount\endcsname{\second@}
}
    
\def\clear@seq{%
    \loopcount=0
    \loop
       \advance\loopcount by 1
       \expandafter\def\csname sequence\the\loopcount\endcsname{}
    \ifnum\loopcount<\seq@count \repeat
    \xdef\consensus{}
    \xdef\constopo{}
    \xdef\frame@on{0}
    \xdef\textfeaturetop{}       \xdef\featureontop{0}
    \xdef\textfeaturettop{}      \xdef\featureonttop{0}
    \xdef\textfeaturetttop{}     \xdef\featureontttop{0}
    \xdef\textfeaturettttop{}    \xdef\featureonttttop{0}
    \xdef\textfeaturebottom{}    \xdef\featureonbottom{0}
    \xdef\textfeaturebbottom{}   \xdef\featureonbbottom{0}
    \xdef\textfeaturebbbottom{}  \xdef\featureonbbbottom{0}
    \xdef\textfeaturebbbbottom{} \xdef\featureonbbbbottom{0}
    \xdef\styleframe{} 
    \xdef\stylefeaturetop{}
    \xdef\stylefeaturettop{}
    \xdef\stylefeaturetttop{}
    \xdef\stylefeaturettttop{}
    \xdef\stylefeaturebottom{}
    \xdef\stylefeaturebbottom{}
    \xdef\stylefeaturebbbottom{}
    \xdef\stylefeaturebbbbottom{}}
\def\guess@protein{\seqtype{P}\message{<Seqtype guess: protein>}}
\def\guess@DNA{\seqtype{N}\message{<Seqtype guess: nucleotide>}}
\def\allmatch@out{%
    \xdef\leave@in{n}
    \loopcount=1
    \xdef\seq@line{\csname seq\the\loopcount\endcsname}
    \expandafter\res@get\seq@line
    \expandafter\xdef\csname seq\the\loopcount\endcsname{\seq@line}
    \ifnum\loopcount=\start@seq\relax
       \expandafter\check@char\first@
       \ifletter
          \advance\temp@@count by 1
          \ifnum\temp@@count=\end@num\relax 
             \advance\temp@count by 1
             \xdef\second@{\the\temp@count}
             \advance\temp@count by -1
          \fi
       \fi
    \fi
    \ifx\first@\ampers@nd
       \ifx\domain@active\y@
          \xdef\stack@dom{\stack@dom\the\temp@count;}
       \fi   
       \xdef\stack@dom{\stack@dom &;&;@}
    \else
       \xdef\first@@{\first@}
       \advance\temp@count by 1
       \loop
          \advance\loopcount by 1
          \xdef\seq@line{\csname seq\the\loopcount\endcsname}
          \expandafter\res@get\seq@line
          \expandafter\xdef\csname seq\the\loopcount\endcsname{\seq@line}
          \ifx\first@@\first@\else\xdef\leave@in{y}\fi
          \ifnum\loopcount=\start@seq\relax
             \expandafter\check@char\first@
             \ifletter
                \advance\temp@@count by 1
                \ifnum\temp@@count=\end@num\relax \xdef\second@{\the\temp@count}\fi
             \fi
          \fi
       \ifnum\loopcount<\seq@count\repeat
       \ifx\leave@in\y@
          \ifx\domain@active\n@
             \ifnum\start@seq=0
                \xdef\domain@active{y}
                \xdef\stack@dom{\stack@dom\the\temp@count;}
             \else
                \ifnum\temp@@count<\start@num\relax
                \else
                   \ifnum\temp@@count>\end@num\relax
                   \else
                      \xdef\domain@active{y}
                      \xdef\stack@dom{\stack@dom\the\temp@count;}
                   \fi
                \fi
             \fi
          \else
             \ifnum\start@seq>0
                \ifnum\temp@@count<\end@num\relax
                \else
                   \xdef\domain@active{n}
                   \xdef\stack@dom{\stack@dom\second@;}
                \fi
             \fi
          \fi
       \else
          \ifx\domain@active\y@
             \xdef\domain@active{n}
             \advance\temp@count by -1
             \xdef\stack@dom{\stack@dom\the\temp@count;}
             \advance\temp@count by 1
          \fi
       \fi
       \allmatch@out
    \fi
}
\def\domain@count{%
    \xdef\@ddg@p{n}
    \loopcount=0
    \ifT@coffee
       \xdef\seq@line{\csname T@cof\the\loopcount\endcsname}
       \expandafter\res@get\seq@line
       \expandafter\xdef\csname TC@num\the\loopcount\endcsname{\first@}
       \expandafter\xdef\csname T@cof\the\loopcount\endcsname{\seq@line}
    \fi
    \loop
       \advance\loopcount by 1
       \xdef\seq@line{\csname seq\the\loopcount\endcsname}
       \expandafter\res@get\seq@line
       \expandafter\xdef\csname res\the\loopcount\endcsname{\first@}
       \expandafter\xdef\csname seq\the\loopcount\endcsname{\seq@line}
       \ifT@coffee
          \xdef\seq@line{\csname T@cof\the\loopcount\endcsname}
          \expandafter\res@get\seq@line
          \expandafter\xdef\csname TC@num\the\loopcount\endcsname{\first@}
          \expandafter\xdef\csname T@cof\the\loopcount\endcsname{\seq@line}
       \fi
    \ifnum\loopcount<\seq@count\repeat
    \ifx\first@\ampers@nd
    \else
     \ifnum\next@domain@num>-99999\relax
         \ifnum\domain@seq=0
             \temp@count=\csname dom@count0\endcsname 
             \advance\temp@count by 1
             \expandafter\xdef\csname dom@count0\endcsname{\the\temp@count}
         \fi
         \loopcount=0
         \loop
           \advance\loopcount by 1
            \xdef\first@{\csname res\the\loopcount\endcsname}
            \expandafter\check@char\first@
            \ifletter
               \temp@count=\csname dom@count\the\loopcount\endcsname 
               \advance\temp@count by 1
               \expandafter\xdef\csname dom@count\the\loopcount\endcsname{\the\temp@count}
               \if\seq@type A \relax
                  \temp@count=\char@num
                  \ifnum\temp@count>96 \advance\temp@count by -32 \fi
                  \ifnum\temp@count=69 \guess@protein
                  \else
                  \ifnum\temp@count=70 \guess@protein
                  \else
                  \ifnum\temp@count=73 \guess@protein
                  \else
                  \ifnum\temp@count=76 \guess@protein
                  \else
                  \ifnum\temp@count=80 \guess@protein
                  \else
                  \ifnum\temp@count=81 \guess@protein
                  \fi\fi\fi\fi\fi\fi
               \fi
            \else
               \ifx\d@m@in\y@
                  \ifnum\loopcount=\domain@seq \xdef\@ddg@p{y} \fi
               \fi
            \fi
         \ifnum\loopcount<\seq@count\repeat 
         \ifx\@ddg@p\y@ 
           \ifx\g@p\n@
            \loopcount=0
            \loop
            \advance\loopcount by 1
               \ifT@coffee
                  \expandafter\xdef\csname T@coffee\the\loopcount\endcsname{%
                                   \csname T@coffee\the\loopcount\endcsname \csname TC@num\the\loopcount\endcsname}
               \fi
               \expandafter\xdef\csname sequence\the\loopcount\endcsname{%
                                \csname sequence\the\loopcount\endcsname \csname res\the\loopcount\endcsname}
               \expandafter\xdef\csname dom@num\the\loopcount\endcsname{%
                                \csname dom@num\the\loopcount\endcsname \csname dom@count\the\loopcount\endcsname,}
            \ifnum\loopcount<\seq@count\repeat
            \advance\total@count by 1
            \advance\outerloopcount by 1 \ifnum\outerloopcount=\res@perline \outerloopcount=0 \fi
           \fi
         \fi 
         \expandafter\ifnum\csname dom@count\domain@seq\endcsname=\next@domain@num
            \xdef\d@m@in{y}
            \advance\total@count by 1
            \advance\outerloopcount by 1 \ifnum\outerloopcount=\res@perline \outerloopcount=0 \fi
            \loopcount=0
            \ifx\g@p\y@
               \ifT@coffee
                  \expandafter\xdef\csname T@coffee\the\loopcount\endcsname{%
                                   \csname T@coffee\the\loopcount\endcsname \csname TC@num\the\loopcount\endcsname}
               \fi
            \fi
            \ifT@coffee
               \expandafter\xdef\csname T@coffee\the\loopcount\endcsname{%
                                \csname T@coffee\the\loopcount\endcsname \csname TC@num\the\loopcount\endcsname}
            \fi
            \loop
               \advance\loopcount by 1
               \ifx\g@p\y@
                  \ifT@coffee
                     \expandafter\xdef\csname T@coffee\the\loopcount\endcsname{%
                                      \csname T@coffee\the\loopcount\endcsname \csname TC@num\the\loopcount\endcsname}
                  \fi
                  \expandafter\xdef\csname sequence\the\loopcount\endcsname{%
                                   \csname sequence\the\loopcount\endcsname\equ@l}
                  \expandafter\xdef\csname dom@num\the\loopcount\endcsname{%
                                   \csname dom@num\the\loopcount\endcsname \csname dom@count\the\loopcount\endcsname,}
                  \ifnum\outerloopcount=0
                     \expandafter\xdef\csname dom@num@break\the\loopcount\endcsname{%
                                      \csname dom@num@break\the\loopcount\endcsname \csname dom@count\the\loopcount\endcsname,}
                  \fi
               \fi
               \ifT@coffee
                  \expandafter\xdef\csname T@coffee\the\loopcount\endcsname{%
                                   \csname T@coffee\the\loopcount\endcsname \csname TC@num\the\loopcount\endcsname}
               \fi
               \expandafter\xdef\csname sequence\the\loopcount\endcsname{%
                                \csname sequence\the\loopcount\endcsname \csname res\the\loopcount\endcsname}
               \expandafter\xdef\csname dom@num\the\loopcount\endcsname{%
                                \csname dom@num\the\loopcount\endcsname \csname dom@count\the\loopcount\endcsname,}
               \ifnum\outerloopcount=0
                  \expandafter\xdef\csname dom@num@break\the\loopcount\endcsname{%
                                   \csname dom@num@break\the\loopcount\endcsname \csname dom@count\the\loopcount\endcsname,}
               \fi
            \ifnum\loopcount<\seq@count\repeat
            \ifx\@ddg@p\n@
               \expandafter\get@item\domain@num@list
               \xdef\domain@num@list{\first@}
            \fi
            \ifx\fourth@\ampers@nd \xdef\next@domain@num{-99999}
            \else
               \loopcount=\next@domain@num
               \advance\loopcount by 1
               \ifnum\loopcount=\fourth@
                  \xdef\g@p{n}
               \else
                  \xdef\g@p{y}
                     \advance\total@count by 1
                     \advance\outerloopcount by 1
                     \ifnum\outerloopcount=\res@perline
                        \outerloopcount=0
                        \loopcount=0
                        \loop
                           \advance\loopcount by 1
                           \expandafter\xdef\csname dom@num@break\the\loopcount\endcsname{%
                                            \csname dom@num@break\the\loopcount\endcsname \csname dom@count\the\loopcount\endcsname,}
                        \ifnum\loopcount<\seq@count\repeat
                     \fi
               \fi
               \xdef\next@domain@num{\fourth@} \xdef\end@num{\fourth@}
            \fi
         \fi 
         \domain@count
     \fi
    \fi
}
\def\residue@count{%
    \expandafter\res@get\seq@line
    \if\first@\ampers@nd
    \else
         \advance\total@count by 1
         \advance\innerloopcount by 1
         \ifstart@ \advance\res@count by 1 \fi
         \ifnum\start@seq=0
            \advance\end@count by 1
%%%%%%%%%%            
            \ifnum\end@count=\start@num\relax
               \xdef\start@number{\the\innerloopcount}
               \start@true \res@count=1
            \fi
%%%%%%%%%%
         \else
            \expandafter\check@char\first@
            \ifletter 
               \advance\end@count by 1
               \if\seq@type A \relax
                  \temp@count=\char@num
                  \ifnum\temp@count>96 \advance\temp@count by -32 \fi
                  \ifnum\temp@count=69 \guess@protein
                  \else
                  \ifnum\temp@count=70 \guess@protein
                  \else
                  \ifnum\temp@count=73 \guess@protein
                  \else
                  \ifnum\temp@count=76 \guess@protein
                  \else
                  \ifnum\temp@count=80 \guess@protein
                  \else
                  \ifnum\temp@count=81 \guess@protein
                  \fi\fi\fi\fi\fi\fi
               \fi
            \fi
            \ifnum\end@count=\start@num\relax
               \xdef\start@number{\the\innerloopcount}
               \start@true \res@count=1
            \fi
      \fi
    \residue@count
    \fi}
\def\clean@seq{%
    \expandafter\res@get\seq@line
    \if\first@\ampers@nd \xdef\seq@line{\seq@@line}
    \else
       \expandafter\check@char\first@
       \ifnum\char@num>64
          \ifnum\char@num>96 \make@upper \fi
          \xdef\seq@@line{\seq@@line\first@}
       \else
       \ifnum\char@num=46 
          \xdef\seq@@line{\seq@@line\d@t}
       \else
       \ifnum\char@num=45 
          \xdef\seq@@line{\seq@@line\d@t}
       \else
       \ifnum\char@num=42 
          \xdef\seq@@line{\seq@@line\questi@n}
       \fi
       \fi
       \fi
       \fi
       \clean@seq
    \fi}
\def\read@loop{%
    \read\alignfile to \inline
    \xdef\last@{\expandafter\string\inline}
    \ifx\last@\par@
      \else
         \xdef\inline{\inline @}
         \expandafter\seq@get\inline
         \ifstop@
         \else
         \innerloopcount=\csname @rd\the\loopcount\endcsname\relax
         \expandafter\ifx\csname seq@name\the\loopcount\endcsname\first@
             \xdef\seq@@line{} \xdef\seq@line{\seq@line &@} \clean@seq
             \expandafter\xdef\csname sequence\the\innerloopcount\endcsname{%
                  \csname sequence\the\innerloopcount\endcsname\seq@line}
             \advance\loopcount by 1
             \ifnum\loopcount>\seq@count
                \loopcount=1
             \fi
         \fi
         \fi
       \fi
    \ifeof\alignfile
     \ifx\all@out\y@
       \xdef\domain@seq{0}
       \loopcount=0
       \loop
          \advance\loopcount by 1
          \expandafter\xdef\csname seq\the\loopcount\endcsname{\csname sequence\the\loopcount\endcsname &@}
       \ifnum\loopcount<\seq@count\repeat
       \xdef\domain@active{n}
       \xdef\stack@dom{}
       \temp@count=0
       \temp@@count=0
       \allmatch@out
       \xdef\dom@in{y}
     \fi
     \ifx\dom@in\y@
       \loopcount=0
       \ifT@coffee
          \xdef\temp@{\csname T@coffee\the\loopcount\endcsname}
          \expandafter\get@temp\temp@
          \expandafter\xdef\csname T@cof\the\loopcount\endcsname{\temp@ &@}
          \expandafter\xdef\csname T@coffee\the\loopcount\endcsname{}
       \fi
       \expandafter\xdef\csname dom@count\the\loopcount\endcsname{0}
       \loop
          \advance\loopcount by 1
          \ifT@coffee
             \xdef\temp@{\csname T@coffee\the\loopcount\endcsname}
             \expandafter\get@temp\temp@
             \expandafter\xdef\csname T@cof\the\loopcount\endcsname{\temp@ &@}
             \expandafter\xdef\csname T@coffee\the\loopcount\endcsname{}
          \fi
          \expandafter\xdef\csname seq\the\loopcount\endcsname{\csname sequence\the\loopcount\endcsname &@}
          \expandafter\xdef\csname sequence\the\loopcount\endcsname{}
          \expandafter\xdef\csname dom@num\the\loopcount\endcsname{}
          \expandafter\xdef\csname dom@num@break\the\loopcount\endcsname{}
          \expandafter\xdef\csname dom@count\the\loopcount\endcsname{\csname res@count\the\loopcount\endcsname}
       \ifnum\loopcount<\seq@count\repeat
       \xdef\domain@num@list{}
       \xdef\last@{\stack@dom}
       \get@domain
       \xdef\domain@num@list{\domain@num@list &,@}
       \expandafter\get@item\domain@num@list
       \xdef\next@domain@num{\fourth@} \xdef\domain@num@list{\first@}
       \xdef\d@m@in{n}
       \domain@count
       \loopcount=0
       \ifT@coffee
          \expandafter\xdef\csname T@coffee\the\loopcount\endcsname{\csname T@coffee\the\loopcount\endcsname @}
       \fi
       \loop
          \advance\loopcount by 1
          \ifT@coffee
             \expandafter\xdef\csname T@coffee\the\loopcount\endcsname{\csname T@coffee\the\loopcount\endcsname @}
          \fi
          \expandafter\xdef\csname dom@num\the\loopcount\endcsname{\csname dom@num\the\loopcount\endcsname @}
          \expandafter\xdef\csname dom@num@break\the\loopcount\endcsname{%
                           \csname dom@num@break\the\loopcount\endcsname \csname dom@count\the\loopcount\endcsname,@}
       \ifnum\loopcount<\seq@count\repeat
       \if\seq@type A \guess@DNA \fi
       \advance\seq@pointer by \total@count\relax
       \xdef\total@pos{\the\total@count}
    
     \else   

       \ifnum\start@seq=0
          \xdef\seq@line{\csname sequence1\endcsname &@} 
       \else
          \xdef\seq@line{\csname sequence\start@seq\endcsname &@} 
       \fi
       \res@count=0 
       \innerloopcount=0
       \residue@count 
       \if\seq@type A \guess@DNA \fi
       \advance\seq@pointer by \res@count\relax 
       \xdef\total@pos{\the\total@count}
       \ifx\end@num\n@\xdef\end@num{\the\end@count}\fi
       \total@count=\end@count\relax
       \xdef\first@{\csname res@count\start@seq\endcsname}
       \advance\total@count by \first@\relax
       \xdef\first@{\the\total@count}
       \ifnum\end@num>\total@count\xdef\end@num{\the\total@count}\fi
     \fi       
     \ifshow@sublogo \prep@sublogo \fi
     \ifshow@logo  \prep@logo  \fi
     \ifx\label@motif\y@  \prep@motif \fi
     \ifnum\seq@pointer>\res@perline \block@output \fi 
    \else \read@loop
    \fi}
\def\read@fasta{%
    \read\alignfile to \inline
    \xdef\last@{\expandafter\string\inline}
    \ifx\last@\par@
      \else
         \xdef\seq@line{\inline}
         \xdef\inline{\inline @}
	 \expandafter\firstchar@get\inline
         \ifstop@
         \else
	 \ifx\first@\gre@ter
	    \advance\loopcount by 1\relax
	    \expandafter\seq@get\third@
	    \xdef\seq@name{\first@}
	    \xdef\second@{\first@ &}
	    \ifx\second@\ampers@nd \xdef\seq@name{seq\the\loopcount}\fi
	    \innerloopcount=\csname @rd\the\loopcount\endcsname\relax
	 \else
	    \expandafter\ifx\csname seq@name\the\loopcount\endcsname\seq@name
              \xdef\seq@@line{} \xdef\seq@line{\seq@line &@} \clean@seq
              \expandafter\xdef\csname sequence\the\innerloopcount\endcsname{%
                  \csname sequence\the\innerloopcount\endcsname\seq@line}
            \fi
	 \fi
         \fi
       \fi
    \ifeof\alignfile
     \ifx\all@out\y@
       \xdef\domain@seq{0}
       \loopcount=0
       \loop
          \advance\loopcount by 1
          \expandafter\xdef\csname seq\the\loopcount\endcsname{\csname sequence\the\loopcount\endcsname &@}
       \ifnum\loopcount<\seq@count\repeat
       \xdef\domain@active{n}
       \xdef\stack@dom{}
       \temp@count=0
       \temp@@count=0
       \allmatch@out
       \xdef\dom@in{y}
     \fi
     \ifx\dom@in\y@
       \loopcount=0
       \ifT@coffee
          \xdef\temp@{\csname T@coffee\the\loopcount\endcsname}
          \expandafter\get@temp\temp@
          \expandafter\xdef\csname T@cof\the\loopcount\endcsname{\temp@ &@}
          \expandafter\xdef\csname T@coffee\the\loopcount\endcsname{}
       \fi
       \expandafter\xdef\csname dom@count\the\loopcount\endcsname{0}
       \loop
          \advance\loopcount by 1
          \ifT@coffee
             \xdef\temp@{\csname T@coffee\the\loopcount\endcsname}
             \expandafter\get@temp\temp@
             \expandafter\xdef\csname T@cof\the\loopcount\endcsname{\temp@ &@} 
             \expandafter\xdef\csname T@coffee\the\loopcount\endcsname{}
          \fi
          \expandafter\xdef\csname seq\the\loopcount\endcsname{\csname sequence\the\loopcount\endcsname &@}
          \expandafter\xdef\csname sequence\the\loopcount\endcsname{}
          \expandafter\xdef\csname dom@num\the\loopcount\endcsname{}
          \expandafter\xdef\csname dom@num@break\the\loopcount\endcsname{}
          \expandafter\xdef\csname dom@count\the\loopcount\endcsname{\csname res@count\the\loopcount\endcsname}
       \ifnum\loopcount<\seq@count\repeat
       \xdef\domain@num@list{}
       \xdef\last@{\stack@dom}
       \get@domain
       \xdef\domain@num@list{\domain@num@list &,@}
       \expandafter\get@item\domain@num@list
       \xdef\next@domain@num{\fourth@} \xdef\domain@num@list{\first@}
       \xdef\d@m@in{n}
       \domain@count
       \loopcount=0
       \ifT@coffee
          \expandafter\xdef\csname T@coffee\the\loopcount\endcsname{\csname T@coffee\the\loopcount\endcsname @}
       \fi
       \loop
          \advance\loopcount by 1
          \ifT@coffee
             \expandafter\xdef\csname T@coffee\the\loopcount\endcsname{\csname T@coffee\the\loopcount\endcsname @}
          \fi
          \expandafter\xdef\csname dom@num\the\loopcount\endcsname{\csname dom@num\the\loopcount\endcsname @}
          \expandafter\xdef\csname dom@num@break\the\loopcount\endcsname{%
                           \csname dom@num@break\the\loopcount\endcsname \csname dom@count\the\loopcount\endcsname,@}
       \ifnum\loopcount<\seq@count\repeat
       \if\seq@type A \guess@DNA \fi
       \advance\seq@pointer by \total@count\relax 
       \xdef\total@pos{\the\total@count}
    
     \else   
    
       \ifnum\start@seq=0
          \xdef\seq@line{\csname sequence1\endcsname &@} 
       \else
          \xdef\seq@line{\csname sequence\start@seq\endcsname &@} 
       \fi
       \res@count=0 
       \innerloopcount=0
       \residue@count
       \if\seq@type A \guess@DNA \fi
       \advance\seq@pointer by \res@count
       \xdef\total@pos{\the\total@count}
       \ifx\end@num\n@\xdef\end@num{\the\end@count}\fi
       \total@count=\end@count\relax
       \xdef\first@{\csname res@count\start@seq\endcsname}
       \advance\total@count by \first@\relax
       \xdef\first@{\the\total@count}
       \ifnum\end@num>\total@count\xdef\end@num{\the\total@count}\fi
     \fi
     \ifshow@sublogo \prep@sublogo \fi
     \ifshow@logo  \prep@logo  \fi
     \ifx\label@motif\y@  \prep@motif \fi
     \ifnum\seq@pointer>\res@perline \block@output \fi
    \else 
       \read@fasta 
    \fi}
\def\read@lines{%
    \openin\alignfile = \alignfilename
    \clear@seq
    \ifnum\start@seq>0
       \xdef\start@seq{\csname @rd\start@seq\endcsname} \fi
    \loopcount=\start@num
    \advance\loopcount by -\csname res@count\start@seq\endcsname\relax
    \expandafter\ifnum\csname res@count\start@seq\endcsname<\start@num\relax
       \xdef\start@num{\the\loopcount}
    \else \start@true \fi
    \res@count=0 \seq@pointer=0 \end@count=0 \total@count=0
    \xdef\start@number{0}
    \ifx\exp@rt\y@ \prep@reexp@rtfile \fi
    \ifx\f@st@\y@
       \loopcount=0
       \read@fasta
    \else
       \loopcount=1
       \read@loop
    \fi
    \ifnum\seq@pointer>0
       \ifstop@
       \else
          \res@count=\res@perline
          \res@perline=\seq@pointer
          \block@output
       \fi
    \fi
    \ifshow@sublogo \closein\sublogofile \fi
    \ifx\exp@rt\y@ \closeout\exp@rtfile \fi
    \closein\alignfile
    \message{)}
}


%%%%%  Read alignment, decide whether MSF or ALN, interpret
    
\def\readalignfile#1{%
 \def\alignfilename{#1}
 \xdef\first@{byhand}
 \ifx\alignfilename\first@
 \else
 \openin\alignfile = #1
 \ifeof\alignfile
    \PackageError{TeXshade}
    {File `#1' not found}
    {\MessageBreak
     The alignment file you specified is missing or you have \MessageBreak
     misspelled it. \MessageBreak\MessageBreak
     Stop here, otherwise you're likely getting in trouble. \MessageBreak
     Type X <return> to quit. \MessageBreak
}
 \else
    \message{(\alignfilename :}
    \xdef\seq@type{A} \xdef\he@der{no} \xdef\f@st@{no} 
    \xdef\first@line{y}
    \seq@count=0 \loopcount=0 \innerloopcount=0 \temp@count=0
    \loop
       \read\alignfile to \inline 
       \xdef\test@{\expandafter\string\inline}
       \ifx\test@\par@ \innerloopcount=0
       \else
          \xdef\msfline{\inline & & & & @}
          \expandafter\inf@@get\msfline
          \ifx\first@\@msf \expandafter\type@get\msfline \fi
          \ifx\second@\@msf   \expandafter\type@get\msfline \fi
          \ifx\third@\@msf \expandafter\type@get\msfline \fi
          \ifx\first@\n@me \advance\loopcount by 1\relax
             \expandafter\xdef\csname seqname\the\loopcount\endcsname{\second@}
             \expandafter\xdef\csname seq@len\the\loopcount\endcsname{\fourth@}
          \fi
          \ifx\first@\he@derend
             \ifnum\loopcount>0 \xdef\he@der{yes} \fi
       \fi
       \xdef\alnline{\inline @} 
       \expandafter\check@letter\alnline
       \ifletter
          \expandafter\seq@get\alnline
          \advance\innerloopcount by 1\relax
          \seq@count=\innerloopcount
          \expandafter\xdef\csname newseqname\the\seq@count\endcsname{\first@}
	   \else
	     \expandafter\firstchar@get\alnline
	     \ifx\first@\gre@ter
	        \ifx\first@line\y@ \xdef\f@st@{y}\fi
                \expandafter\seq@get\third@
                \advance\temp@count by 1\relax
	        \xdef\second@{\first@ &}
	        \ifx\second@\ampers@nd \xdef\first@{seq\the\temp@count}\fi
                \expandafter\xdef\csname seq@name\the\temp@count\endcsname{\first@}
             \fi
          \fi
       \fi
      \xdef\first@line{n}
    \ifeof\alignfile \else\repeat
    \closein\alignfile
    \xdef\first@{no}
    \ifx\he@der\first@
      \loopcount=0
      \ifx\f@st@\y@
        \seq@count=\temp@count \loopcount=0
        \loop
          \advance\loopcount by 1
          \expandafter\xdef\csname seqname\the\loopcount\endcsname{%
                              \csname seq@name\the\loopcount\endcsname}
          \expandafter\xdef\csname newseqname\the\loopcount\endcsname{%
                              \csname seq@name\the\loopcount\endcsname}
          \expandafter\xdef\csname seq@len\the\loopcount\endcsname{99999999}
        \ifnum\loopcount<\seq@count \repeat
      \else
        \loop
          \advance\loopcount by 1
          \expandafter\xdef\csname seqname\the\loopcount\endcsname{%
                              \csname newseqname\the\loopcount\endcsname}
          \expandafter\xdef\csname seq@len\the\loopcount\endcsname{99999999}
        \ifnum\loopcount<\seq@count \repeat
      \fi
    \else
       \seq@count=\loopcount \loopcount=0 \box@width=0pt
       \loop
          \advance\loopcount by 1
          \expandafter\xdef\csname newseqname\the\loopcount\endcsname{%
                              \csname seqname\the\loopcount\endcsname}
       \ifnum\loopcount<\seq@count \repeat
    \fi
    \set@savedseqlength
    \loopcount=0 \xdef\seq@order{}
    \expandafter\xdef\csname stack@dom\endcsname{&;&;&;&;@}
    \expandafter\xdef\csname stack@reg\the\loopcount\endcsname{&;&;&;&;@}
    \expandafter\xdef\csname stack@tintreg\the\loopcount\endcsname{&;&;&;@}
    \expandafter\xdef\csname stack@emphreg\the\loopcount\endcsname{&;&;&;@}
    \expandafter\xdef\csname stack@lowerreg\the\loopcount\endcsname{&;&;&;@}
    \expandafter\xdef\csname stack@framereg\the\loopcount\endcsname{&;&;&;@}
    \expandafter\xdef\csname stack@shadingreg\the\loopcount\endcsname{&;&;&;@}
    \expandafter\xdef\csname stack@top\the\loopcount\endcsname{&;&;&;&;@}
    \expandafter\xdef\csname stack@ttop\the\loopcount\endcsname{&;&;&;&;@}
    \expandafter\xdef\csname stack@tttop\the\loopcount\endcsname{&;&;&;&;@}
    \expandafter\xdef\csname stack@ttttop\the\loopcount\endcsname{&;&;&;&;@}
    \expandafter\xdef\csname stack@bottom\the\loopcount\endcsname{&;&;&;&;@}
    \expandafter\xdef\csname stack@bbottom\the\loopcount\endcsname{&;&;&;&;@}
    \expandafter\xdef\csname stack@bbbottom\the\loopcount\endcsname{&;&;&;&;@}
    \expandafter\xdef\csname stack@bbbbottom\the\loopcount\endcsname{&;&;&;&;@}
    \expandafter\xdef\csname seq@start\the\loopcount\endcsname{1}
    \expandafter\xdef\csname name@col\the\loopcount\endcsname{yes}
    \expandafter\xdef\csname number@col\the\loopcount\endcsname{yes}
    \loop
       \advance\loopcount by 1
       \expandafter\xdef\csname @rd\the\loopcount\endcsname{\the\loopcount}
       \expandafter\xdef\csname res@count\the\loopcount\endcsname{0}
       \cleanup@name
       \expandafter\xdef\csname tint@seq\the\loopcount\endcsname{n}
       \expandafter\xdef\csname emph@seq\the\loopcount\endcsname{n}
       \expandafter\xdef\csname lower@seq\the\loopcount\endcsname{n}
       \expandafter\xdef\csname hide@seq\the\loopcount\endcsname{false}
       \expandafter\xdef\csname hide@@@seq\the\loopcount\endcsname{false}
       \expandafter\xdef\csname hide@name\the\loopcount\endcsname{no}
       \expandafter\xdef\csname name@col\the\loopcount\endcsname{yes}
       \expandafter\xdef\csname hide@number\the\loopcount\endcsname{no}
       \expandafter\xdef\csname number@col\the\loopcount\endcsname{yes}
       \expandafter\xdef\csname stack@reg\the\loopcount\endcsname{&;&;&;&;@}
       \expandafter\xdef\csname stack@tintreg\the\loopcount\endcsname{&;&;&;@}
       \expandafter\xdef\csname stack@emphreg\the\loopcount\endcsname{&;&;&;@}
       \expandafter\xdef\csname stack@lowerreg\the\loopcount\endcsname{&;&;&;@}
       \expandafter\xdef\csname stack@framereg\the\loopcount\endcsname{&;&;&;@}
       \expandafter\xdef\csname stack@shadingreg\the\loopcount\endcsname{&;&;&;@}
       \expandafter\xdef\csname stack@top\the\loopcount\endcsname{&;&;&;&;@}
       \expandafter\xdef\csname stack@ttop\the\loopcount\endcsname{&;&;&;&;@}
       \expandafter\xdef\csname stack@tttop\the\loopcount\endcsname{&;&;&;&;@}
       \expandafter\xdef\csname stack@ttttop\the\loopcount\endcsname{&;&;&;&;@}
       \expandafter\xdef\csname stack@bottom\the\loopcount\endcsname{&;&;&;&;@}
       \expandafter\xdef\csname stack@bbottom\the\loopcount\endcsname{&;&;&;&;@}
       \expandafter\xdef\csname stack@bbbottom\the\loopcount\endcsname{&;&;&;&;@}
       \expandafter\xdef\csname stack@bbbbottom\the\loopcount\endcsname{&;&;&;&;@}
       \expandafter\xdef\csname seq@gap\the\loopcount\endcsname{no}
       \expandafter\xdef\csname seq@start\the\loopcount\endcsname{1}
       \expandafter\xdef\csname mol@weight\the\loopcount\endcsname{0}
       \expandafter\xdef\csname ch@rge\the\loopcount\endcsname{0}
       \ifnum\loopcount=1 \xdef\seq@order{\the\loopcount}
       \else              \xdef\seq@order{\seq@order,\the\loopcount}\fi
    \ifnum\loopcount<\seq@count \repeat
    \xdef\seq@order{\seq@order,@}
 \killseq@count=\seq@count
 \seq@percent=100
 \ifnum\seq@count>0 \divide\seq@percent by \seq@count \fi
 \fi
 \fi
}

%%%%%  TeXshade

\def\calc@widths{%
   \fontfamily{cmss}\fontseries{m}\fontshape{n}
   \selectfont
   \setbox1=\hbox{\residues@size{A}}\logo@height=\ht1
   \setlength\logo@height{\logo@stretch\logo@height}
   \fontfamily{\residues@family}
   \fontseries{\residues@series}
   \fontshape{\residues@shape}
   \selectfont
   \setbox1=\hbox{\residues@size{W}}\box@width=1.15\wd1
   \global\setlength\box@width{\char@stretch\box@width}
   \box@height=1.2\ht1
   \setbox1=\hbox{\residues@size{g}}\box@depth=1.1\dp1
   \global\setlength\box@height{\line@stretch\box@height}
   \global\setlength\box@depth{\line@stretch\box@depth}
   \baselineskip=\box@height \advance\baselineskip by \box@depth
   \lineskip=0pt
   \ifshow@cons\setbox1=\hbox{\cons@name}\name@width=\wd1\else\name@width=0pt\fi
   \loopcount=0
   \expandafter\getregion@fromstack{\the\loopcount}
   \expandafter\getregion@fromtintstack{\the\loopcount}
   \expandafter\getregion@fromemphstack{\the\loopcount}
   \expandafter\getregion@fromlowerstack{\the\loopcount}
   \expandafter\getregion@fromframestack{\the\loopcount}
   \expandafter\getregion@fromshadingstack{\the\loopcount}
   \xdef\bottop@{top}       \expandafter\getregion@fromfstack{\the\loopcount}
   \xdef\bottop@{ttop}      \expandafter\getregion@fromfstack{\the\loopcount}
   \xdef\bottop@{tttop}     \expandafter\getregion@fromfstack{\the\loopcount}
   \xdef\bottop@{ttttop}    \expandafter\getregion@fromfstack{\the\loopcount}
   \xdef\bottop@{bottom}    \expandafter\getregion@fromfstack{\the\loopcount}
   \xdef\bottop@{bbottom}   \expandafter\getregion@fromfstack{\the\loopcount}
   \xdef\bottop@{bbbottom}  \expandafter\getregion@fromfstack{\the\loopcount}
   \xdef\bottop@{bbbbottom} \expandafter\getregion@fromfstack{\the\loopcount}
   \fontfamily{\namestext@family}
   \fontseries{\namestext@series}
   \fontshape{\namestext@shape}
   \selectfont
   \ifshow@logo
      \ifx\logo@name@user\ampers@nd
         \setbox1=\hbox{\namestext@size logo\kern1em}
      \else
         \setbox1=\hbox{\namestext@size\logo@name@user\kern1em}
      \fi
      \ifnum\wd1>\name@width \name@width=\wd1 \fi
   \fi
   \ifshow@sublogo
      \ifx\sublogo@name@user\ampers@nd
         \setbox1=\hbox{\namestext@size family\kern1em}
      \else
         \setbox1=\hbox{\namestext@size\sublogo@name@user\kern1em}
      \fi
      \ifnum\wd1>\name@width \name@width=\wd1 \fi
      \ifx\hide@negatives\n@
         \setbox1=\hbox{\namestext@size\sublogo@name@neg\kern1em}
      \fi
      \ifnum\wd1>\name@width \name@width=\wd1 \fi
   \fi
   \loopcount=1
   \loop
       \setbox1=\hbox{\namestext@size\csname newseqname\the\loopcount\endcsname}
       \ifnum\wd1>\name@width \name@width=\wd1 \fi
       \ifnum\ht1>\box@height
          \box@height=1.1\ht1
          \global\setlength\box@height{\line@stretch\box@height}
          \baselineskip=\box@height \advance\baselineskip by \box@depth
       \fi
       \ifnum\dp1>\box@depth
          \box@depth=1.1\dp1
          \global\setlength\box@depth{\line@stretch\box@depth}
          \baselineskip=\box@height \advance\baselineskip by \box@depth
       \fi
       \expandafter\getregion@fromstack{\the\loopcount}
       \expandafter\getregion@fromtintstack{\the\loopcount}
       \expandafter\getregion@fromemphstack{\the\loopcount}
       \expandafter\getregion@fromlowerstack{\the\loopcount}
       \expandafter\getregion@fromframestack{\the\loopcount}
       \expandafter\getregion@fromshadingstack{\the\loopcount}
       \xdef\bottop@{top}       \expandafter\getregion@fromfstack{\the\loopcount}
       \xdef\bottop@{ttop}      \expandafter\getregion@fromfstack{\the\loopcount}
       \xdef\bottop@{tttop}     \expandafter\getregion@fromfstack{\the\loopcount}
       \xdef\bottop@{ttttop}    \expandafter\getregion@fromfstack{\the\loopcount}
       \xdef\bottop@{bottom}    \expandafter\getregion@fromfstack{\the\loopcount}
       \xdef\bottop@{bbottom}   \expandafter\getregion@fromfstack{\the\loopcount}
       \xdef\bottop@{bbbottom}  \expandafter\getregion@fromfstack{\the\loopcount}
       \xdef\bottop@{bbbbottom} \expandafter\getregion@fromfstack{\the\loopcount}
       \innerloopcount = \csname seq@len\the\loopcount\endcsname
       \advance\innerloopcount by \csname seq@start\the\loopcount\endcsname
       \advance\innerloopcount by -1
       \expandafter\xdef\csname seq@len\the\loopcount\endcsname{\the\innerloopcount}
       \advance\loopcount by 1
   \ifnum\loopcount>\seq@count \else \repeat
   \advance\name@width by -2em
   \global\xdef\name@@width{\the\name@width}
   \advance\name@width by 3em
   \fontfamily{\numbertext@family}
   \fontseries{\numbertext@series}
   \fontshape{\numbertext@shape}
   \selectfont
   \setbox1=\hbox{\bottomruler@size\num@width\,-}
   \xdef\ruler@width{\the\wd1}
   \setbox1=\hbox{\numbertext@size\num@width}
   \number@width=\wd1
   \advance\number@width by 1em
   \loopcount=\linewidth
   \ifnames@   \advance\loopcount by -\name@width \fi
   \width@tmp=1.6\box@width
   \ifnumbers@
      \ifnumbers@left  
         \advance\loopcount by -\number@width
      \else
         \xdef\first@{left}
         \ifx\show@logoscale\first@
            \advance\loopcount by -\width@tmp
         \fi
      \fi
      \ifnumbers@right  
         \advance\loopcount by -\number@width
      \else
         \xdef\first@{right}
         \ifx\show@logoscale\first@
            \advance\loopcount by -\width@tmp
         \fi
      \fi
   \else
      \xdef\first@{left}
      \ifx\show@logoscale\first@
         \advance\loopcount by -\width@tmp
      \fi
      \xdef\first@{right}
      \ifx\show@logoscale\first@
         \advance\loopcount by -\width@tmp
      \fi
      \xdef\first@{leftright}
      \ifx\show@logoscale\first@
         \advance\loopcount by -\width@tmp
         \advance\loopcount by -\width@tmp
      \fi
   \fi
   \divide\loopcount by \box@width
   \divide\loopcount by 5 \multiply\loopcount by 5
   \ifrpl@fix\else
   \ifnum\res@perline>\loopcount \res@perline=\loopcount\fi\fi
   \ifnum\finger@linenum>0
      \width@tmp=\linewidth
      \ifnames@   \advance\width@tmp by -\name@width \fi
      \ifnumbers@ \advance\width@tmp by -\number@width \fi
      \divide\width@tmp by \finger@linenum
      \global\setlength\box@width{\width@tmp}
   \fi
   \center@fill=\linewidth
   \ifnames@   \advance\center@fill by -\name@width \fi
   \ifnumbers@ \advance\center@fill by -\number@width \fi
   \width@tmp=\box@width \multiply\width@tmp by \res@perline
   \advance\center@fill by -\width@tmp
   \ifx\out@put\y@\leftskip\c@factor\center@fill\fi
   \ifx\logo@colors@set\n@
     \if\seq@type N
      \clearlogocolors[Yellow]
      \logocolor{G}{Black}
      \logocolor{A}{Green}
      \logocolor{TU}{Red}
      \logocolor{C}{Blue}
     \else
      \clearlogocolors
      \logocolor{DE}{Red}
      \logocolor{CM}{Yellow}
      \logocolor{KR}{Blue}
      \logocolor{ST}{Orange}
      \logocolor{FY}{MidnightBlue}
      \logocolor{NQ}{Cyan}
      \logocolor{G}{LightGray}
      \logocolor{LVI}{Green}
      \logocolor{A}{DarkGray}
      \logocolor{W}{CarnationPink}
      \logocolor{H}{CornflowerBlue}
      \logocolor{P}{Apricot}
      \logocolor{BZ}{LightMagenta}
     \fi   
   \fi
}

\def\multiple@dssp{%
   \advance\loopcount by 1
   \include@DSSP
   \ifnum\loopcount<\dssp@num\multiple@dssp\fi}

\def\multiple@stride{%
   \advance\loopcount by 1
   \include@stride
   \ifnum\loopcount<\stride@num\multiple@stride\fi}

\def\multiple@PHD{%
   \advance\loopcount by 1
   \include@PHD
   \ifnum\loopcount<\PHD@num\multiple@PHD\fi}

\def\multiple@HMMTOP{%
   \advance\loopcount by 1
   \include@HMMTOP
   \ifnum\loopcount<\HMMTOP@num\multiple@HMMTOP\fi}

\newenvironment{texshade}[2][&]%
  {\standarddefinitions
   \c@d@ns
   \xdef\first@{#1}\ifx\first@\ampers@nd\else\input{#1}\fi
   \readalignfile{#2}
   }%
  {\ifnum\seq@count>0
      \loopcount=0
      \ifnum\loopcount<\dssp@num \multiple@dssp\fi
      \loopcount=0
      \ifnum\loopcount<\stride@num \multiple@stride\fi
      \loopcount=0
      \ifnum\loopcount<\PHD@num \multiple@PHD\fi
      \loopcount=0
      \ifnum\loopcount<\HMMTOP@num \multiple@HMMTOP\fi
      \loopcount=1 \kill@seqnow
      \reorder@seqs\seq@order \seq@count=\killseq@count
      \global\xdef\seq@num{\the\seq@count}
      \clear@sim@count
      \ifnum\csname res@count\start@seq\endcsname<0
       \ifnum\start@num>0
         \loopcount=\start@num
         \advance\loopcount by -1
         \xdef\start@num{\the\loopcount}
       \fi
      \fi
      \ifnum\rule@num>0  
         \loopcount=\csname res@count\rule@num\endcsname
         \divide\loopcount by \ruler@step
         \multiply\loopcount by \ruler@step
         \ifnum\loopcount<0 
            \ifnum\ruler@step<3
               \advance\loopcount by \ruler@step 
            \fi
         \else 
            \advance\loopcount by \ruler@step 
         \fi
         \xdef\rule@tens{\the\loopcount} 
      \else
         \loopcount=\cons@count  
         \divide\loopcount by \ruler@step
         \multiply\loopcount by \ruler@step
         \ifnum\loopcount<0 
            \ifnum\ruler@step<3
               \advance\loopcount by \ruler@step 
            \fi
         \else 
            \advance\loopcount by \ruler@step 
         \fi
         \xdef\rule@tens{\the\loopcount}  
      \fi
      \xdef\first@{top}
      \ifx\cap@pos\first@
         \xdef\@captype{figure}
         \ifx\c@pshort\n@
           \caption{\c@p}
         \else
           \caption[\c@pshort]{\c@p}
         \fi
      \fi
      \bgroup
         \ifx\out@put\y@\bigskip\fi
         \iffuncmode \show@consfalse \fi
         \ifall@fshade \iffuncmode \else \all@fshadefalse \fi\fi
         \ifnum\finger@linenum>0
             \show@consfalse
             \hidechartrue
             \message{<Fingerprinting---please wait>}
         \fi
         \calc@widths
         \read@lines
	     \save@lengths
         \iflegend@ 
            \vspace{\vspace@legend}
            \setbox1=\vbox{\do@legend}
            \vbox{\do@legend}\par
            \ifnum\ht1<-\vspace@legend
            \vspace{-\ht1}\vspace{-\vspace@legend}
            \fi
         \fi
      \egroup
      \xdef\first@{bottom}
      \ifx\cap@pos\first@ 
         \vspace{-\baselineskip}
         \xdef\@captype{figure}
         \ifx\c@pshort\n@
           \caption{\c@p}
         \else
           \caption[\c@pshort]{\c@p}
         \fi
      \fi
   \fi}

\def\standarddefinitions{%
\xdef\prfx{pep}\clear@groups\clear@sims
\xdef\prfx{DNA}\clear@groups\clear@sims
\clearfuncgroups
\clear@simpairs
\shadingcolors{blues}
\shadingcolors{reds}
\shadingcolors{greens}
\shadingcolors{grays}
\shadingcolors{black}
\loopcount=\@lign@count
\advance\loopcount by 1\relax
\xdef\@lign@count{\the\loopcount}
\start@true \xdef\start@num{1} \xdef\start@seq{0}
\stop@false \xdef\end@num{n} \xdef\seq@regions{0}
\cons@count=0
\expandafter\xdef\csname res@count0\endcsname{0}
\xdef\allow@zero{n} \xdef\c@ns@shift{0}
\regionalshadefalse\regionalemphfalse\regionallowerfalse\regionaltintfalse
\frame@false\shading@false
\xdef\ruler@rot{0}
\topfeaturefalse    \bottomfeaturefalse
\ttopfeaturefalse   \bbottomfeaturefalse
\tttopfeaturefalse  \bbbottomfeaturefalse
\ttttopfeaturefalse \bbbbottomfeaturefalse
\all@fshadefalse\hidecharfalse
\xdef\finger@linenum{0}
\xdef\hide@seqs{n}
\xdef\dssp@num{0} \xdef\stride@num{0} \xdef\PHD@num{0} \xdef\HMMTOP@num{0}
\xdef\bottop@{top}      \expandafter\xdef\csname feature@\bottop@\endcsname{0}
\xdef\bottop@{ttop}     \expandafter\xdef\csname feature@\bottop@\endcsname{0}
\xdef\bottop@{tttop}    \expandafter\xdef\csname feature@\bottop@\endcsname{0}
\xdef\bottop@{ttttop}   \expandafter\xdef\csname feature@\bottop@\endcsname{0}
\xdef\bottop@{bottom}   \expandafter\xdef\csname feature@\bottop@\endcsname{0}
\xdef\bottop@{bbottom}  \expandafter\xdef\csname feature@\bottop@\endcsname{0}
\xdef\bottop@{bbbottom} \expandafter\xdef\csname feature@\bottop@\endcsname{0}
\xdef\bottop@{bbbbottom}\expandafter\xdef\csname feature@\bottop@\endcsname{0}
\xdef\frame@{0}
\xdef\show@Hdssp{no} \xdef\show@Gdssp{no} \xdef\show@Idssp{no}
\xdef\show@Edssp{no} \xdef\show@Bdssp{no} \xdef\show@Tdssp{no}
\xdef\show@Sdssp{no}
\xdef\show@Hstride{no} \xdef\show@Gstride{no} \xdef\show@Istride{no}
\xdef\show@Estride{no} \xdef\show@Bstride{no} \xdef\show@Tstride{no}
\xdef\show@itop{no} \xdef\show@etop{no} \xdef\show@TMtop{no}
\xdef\show@i@HMMTOP{no} \xdef\show@e@HMMTOP{no} \xdef\show@TM@HMMTOP{no}
\xdef\show@Hsec{no} \xdef\show@Esec{no}
\xdef\collect@restop{no}     \xdef\collect@resttop{no}
\xdef\collect@restttop{no}   \xdef\collect@resttttop{no}
\xdef\collect@resbottom{no}  \xdef\collect@resbbottom{no}
\xdef\collect@resbbbottom{no}\xdef\collect@resbbbbottom{no}
\xdef\tr@nslatetop{}         \xdef\tr@nslatettop{}
\xdef\tr@nslatetttop{}       \xdef\tr@nslatettttop{}
\xdef\tr@nslatebottom{}      \xdef\tr@nslatebbottom{}
\xdef\tr@nslatebbbottom{}    \xdef\tr@nslatebbbbottom{}
\xdef\tr@nsseqtop{0}         \xdef\tr@nsseqttop{0}
\xdef\tr@nsseqtttop{0}       \xdef\tr@nsseqttttop{0}
\xdef\tr@nsseqbottom{0}      \xdef\tr@nsseqbbottom{0}
\xdef\tr@nsseqbbbottom{0}    \xdef\tr@nsseqbbbbottom{0}
\xdef\triple@counttop{0}     \xdef\triple@countttop{0}
\xdef\triple@counttttop{0}   \xdef\triple@countttttop{0}
\xdef\triple@countbottom{0}  \xdef\triple@countbbottom{0}
\xdef\triple@countbbbottom{0}\xdef\triple@countbbbbottom{0}
\xdef\last@@restop{}         \xdef\last@@resttop{}
\xdef\last@@restttop{}       \xdef\last@@resttttop{}
\xdef\last@@resbottom{}      \xdef\last@@resbbottom{}
\xdef\last@@resbbbottom{}    \xdef\last@@resbbbbottom{}
\xdef\out@put{y}             \xdef\m@p{no}
\xdef\t@sp@ce{0mm}           \xdef\tt@sp@ce{0mm}
\xdef\ttt@sp@ce{0mm}         \xdef\tttt@sp@ce{0mm}
\xdef\b@sp@ce{0mm}           \xdef\bb@sp@ce{0mm}
\xdef\bbb@sp@ce{0mm}         \xdef\bbbb@sp@ce{0mm}
\xdef\ruler@sp@ce{0mm}
\xdef\seq@gap@num{0}        
\xdef\h@ndalign{no}         \xdef\sep@space{0pt}
\xdef\c@pshort{n}
\xdef\bottom@stretch{n}     \xdef\bbottom@stretch{n}
\xdef\bbbottom@stretch{n}   \xdef\bbbbottom@stretch{n}
\xdef\c@nscol{}             \xdef\c@nssc@le{ColdHot}
\xdef\collect@cons@colors{no} \xdef\cons@now{no}
\xdef\res@numA{0}         \xdef\res@numB{0}
\xdef\res@numC{0}         \xdef\res@numD{0}
\xdef\res@numE{0}         \xdef\res@numF{0}
\xdef\res@numG{0}         \xdef\res@numH{0}
\xdef\res@numI{0}         \xdef\res@numJ{0}
\xdef\res@numK{0}         \xdef\res@numL{0}
\xdef\res@numM{0}         \xdef\res@numN{0}
\xdef\res@numO{0}         \xdef\res@numP{0}
\xdef\res@numQ{0}         \xdef\res@numR{0}
\xdef\res@numS{0}         \xdef\res@numT{0}
\xdef\res@numU{0}         \xdef\res@numV{0}
\xdef\res@numW{0}         \xdef\res@numX{0}
\xdef\res@numY{0}         \xdef\res@numZ{0}
\expandafter\xdef\csname res@num\questi@n\endcsname{0}
\expandafter\xdef\csname res@num\d@t\endcsname{0}
\expandafter\xdef\csname res@num\equ@l\endcsname{0}
\xdef\res@corrA{0}        \xdef\res@corrB{0}
\xdef\res@corrC{0}        \xdef\res@corrD{0}
\xdef\res@corrE{0}        \xdef\res@corrF{0}
\xdef\res@corrG{0}        \xdef\res@corrH{0}
\xdef\res@corrI{0}        \xdef\res@corrJ{0}
\xdef\res@corrK{0}        \xdef\res@corrL{0}
\xdef\res@corrM{0}        \xdef\res@corrN{0}
\xdef\res@corrO{0}        \xdef\res@corrP{0}
\xdef\res@corrQ{0}        \xdef\res@corrR{0}
\xdef\res@corrS{0}        \xdef\res@corrT{0}
\xdef\res@corrU{0}        \xdef\res@corrV{0}
\xdef\res@corrW{0}        \xdef\res@corrX{0}
\xdef\res@corrY{0}        \xdef\res@corrZ{0}
\expandafter\xdef\csname res@corr\questi@n\endcsname{0}
\expandafter\xdef\csname res@corr\d@t\endcsname{0}
\expandafter\xdef\csname res@corr\equ@l\endcsname{0}
\xdef\corr@max{0}         \xdef\do@freq@correction{n}
\xdef\res@total{0}
\xdef\stack@sequencelogo{} \xdef\bit@total{0}
\clearlogocolors \xdef\logo@colors@set{n}
\show@sublogofalse    \xdef\stack@sublogo{}
\xdef\clear@logo{n}       \xdef\subfamily@threshold{50}
\xdef\subfamily@seq{1}    \xdef\sig@max{100000}
\xdef\sublogo@num{}       \xdef\sublogo@sig{}
\xdef\subfamily@count{0}  \xdef\sublogo@name@neg{}
\xdef\logo@name@user{&}   \xdef\sublogo@name@user{&}
\expandafter\xdef\csname group@num1\endcsname{0}
\expandafter\xdef\csname group@num2\endcsname{0}
\xdef\exp@rt@num{0}       \xdef\exp@rt{n}
\xdef\divref@{0}
\xdef\all@thresh@ld{100}  \all@shadefalse
\hidefeaturenames         \hidefeaturestylenames
\xdef\T@coffee@ccons{n}   \xdef\T@coffee@bcons{n}
\xdef\dom@in{no}          \xdef\label@motif{no}
\xdef\motif@num{0}        \xdef\all@out{n}
\hfuzz9999pt

%%%%%%%%%%%%%%%%%%%%%%%%%%%%%%%%%%%%%%%%%%%%%%%%%%%%%%%%%%%%%%%%%%%%%%%%%%%%%%%%
%%%%%                                                                      %%%%%
%%%%%   Default parameter settings for the  LaTeX  ``TeXshade''  package   %%%%%
%%%%%                                                                      %%%%%
%%%%%%%%%%%%%%%%%%%%%%%%%%%%%%%%%%%%%%%%%%%%%%%%%%%%%%%%%%%%%%%%%%%%%%%%%%%%%%%%
%%%%%                                                                      %%%%%
%%%%%   Under any circumstances:                                           %%%%%
%%%%%                                                                      %%%%%
%%%%%                   DO NOT CHANGE ANY SETTINGS HERE !!!                %%%%%
%%%%%                                                                      %%%%%
%%%%%   Please define your personal parameter file!  Store your new file   %%%%%
%%%%%   together with this style-file in the same directory and load the   %%%%%
%%%%%   file by naming it as an optional parameter in the `texshade' en-   %%%%%
%%%%%   vironment. The file `texshade.def' can be used as a template for   %%%%%
%%%%%   the new creation. See the manual for further help.                 %%%%%
%%%%%                                                                      %%%%%
%%%%%                              THANK YOU !!!                           %%%%%
%%%%%                                                                      %%%%%
%%%%%%%%%%%%%%%%%%%%%%%%%%%%%%%%%%%%%%%%%%%%%%%%%%%%%%%%%%%%%%%%%%%%%%%%%%%%%%%%

\shadingmode{identical}             %  Shade identical residues only
\shadingcolors{blues}               %  Select the blue color scheme for shading
\constoallseqs                      %  Calculate consensus considering all seqs
\threshold{50}                      %  Consensus threshold percentage is 50
\residuesperline{999}               %  As many residues as possible per line
\numberingwidth{9999}               %  Assign space for 4 digit numbering
\charstretch{1}                     %  Do not stretch character width
\linestretch{1}                     %  Do not stretch lines vertically
\gapchar{.}                         %  . is printed in sequence gaps
\gaprule{0.3pt}                     %  If a rule is printed in gaps use 0.3 pt
\gapcolors{Black}{White}            %  Gap symbols appear `Black on White'
\domaingaprule{1pt}                 %  If a rule is set between domains use 1 pt
\domaingapcolors{Red}{White}        %  Domain separation line appears `Red on White'
\numberingcolor{Black}              %  Numbering color is `Black'
\shownumbering{right}               %  Show sequence numbering on the left
\namescolor{Black}                  %  Names' color is `Black'
\featurenamescolor{Black}           %  Feature names' color is `Black'
\featurestylenamescolor{Black}      %  Feature names' color is `Black'
\shownames{left}                    %  Show sequence names on the right
\consensuscolors{Black}{White}      %  All consensus symbols/letters
                {Black}{White}      %  appear `Black on White'
                {Black}{White}      %
\showconsensus{bottom}              %  Show consensus line at bottom with
\hidesequencelogo                   %  Do not show a sequence logo as consensus
\showlogoscale{leftright}           %  Show vertical scale bar if logo is on
\logostretch{1}                     %  Do not stretch sequence logo vertically
\hidesubfamilylogo                  %  Do not show a subfamily logo
\subfamilythreshold{50}             %  Set subfamily threshold to 50%
\shownegatives                      %  Show negative values in subfamily logo
\showrelevance[Black]{*}            %  Label relevant subfamily deviations
\defconsensus{{}}{*}{!}             %  Blank =no match; * =match; ! =all match
\showleadinggaps                    %  Show gap symbols before sequence start
\rulercolor{Black}                  %  Ruler's color is `Black'
\hideruler                          %  Do not show the ruler
\rulersteps{10}                     %  Ruler ticks every 10 residues
\legendcolor{Black}                 %  Legend text color is `Black'
\hidelegend                         %  Do not show the legend
\alignment{center}                  %  Center alignment on page
\medsepline                         %  Medium height if separation line is on
\medblockskip                       %  Medium skip between sequence blocks
\flexblockspace                     %  Use optimized space between blocks
\featurerule{0.5ex}                 %  Set feature rule thickness to 1/2 ex
\bargraphstretch{1}                 %  Do not stretch bars in feature graphs
\colorscalestretch{1}               %  Do not stretch color scales in features
\backtranstext{horizontal}          %  Horizontal triplets in feature texts
\backtranslabel{alternating}        %  Alternating triplets in feature styles
\setfamily{residues}{tt}            %  Use typewriter family for residues
\setseries{residues}{md}            %  Use normal series for residues
\setshape {residues}{up}            %  Use upright shape for residues
\setsize  {residues}{normalsize}    %  Use normal size for residues
\setfamily{numbering}{tt}           %  Use typewriter family for numbering
\setseries{numbering}{md}           %  Use normal series for numbering
\setshape {numbering}{up}           %  Use upright shape for numbering
\setsize  {numbering}{normalsize}   %  Use normal size for numbering
\setfamily{names}{tt}               %  Use typewriter family for names
\setseries{names}{md}               %  Use normal series for names
\setshape {names}{up}               %  Use upright shape for names
\setsize  {names}{normalsize}       %  Use normal size for names
\setfamily{featurenames}{tt}        %  Use typewriter family for feature names
\setseries{featurenames}{md}        %  Use normal series for feature names
\setshape {featurenames}{up}        %  Use upright shape for feature names
\setsize  {featurenames}{normalsize}%  Use normal size for feature  names
\setfamily{featurestylenames}{tt}   %  Use typewriter family for feature style names
\setseries{featurestylenames}{md}   %  Use normal series for feature style names
\setshape {featurestylenames}{up}   %  Use upright shape for feature style names
\setsize  {featurestylenames}{normalsize}%  Use normal size for feature style names
\setfamily{features}{rm}            %  Use roman family for feature texts
\setseries{features}{md}            %  Use normal series for feature texts
\setshape {features}{it}            %  Use italics shape for feature texts
\setsize  {features}{normalsize}    %  Use normal size for feature texts
\setfamily{featurestyles}{tt}       %  Use typewriter family for feature styles
\setseries{featurestyles}{md}       %  Use normal series for feature styles
\setshape {featurestyles}{up}       %  Use upright shape for feature styles
\setsize {featurestyles}{normalsize}%  Use normal size for feature styles
\setfamily{legend}{tt}              %  Use typewriter family for legend texts
\setseries{legend}{md}              %  Use normal series for legend texts
\setshape {legend}{up}              %  Use upright shape for legend texts
\setsize  {legend}{normalsize}      %  Use normal size for legend texts
\setfamily{ruler}{sf}               %  Use sans serif font for ruler numbers
\tintdefault{medium}                %  Use medium tint intensity
\emphdefault{it}                    %  Use italics to emphasize regions
\showonPHDsec{alpha,beta}           %  Show helices and strands (PHD input)
\showonPHDtopo{internal,external,TM}%  Show int., ext. and TM's (PHD input)
\showonHMMTOP{TM}                   %  Show TM's (not int/ext on HMMTOP input)
\showonSTRIDE{alpha,3-10,pi,beta}   %  Show helices and strands (STRIDE input)
\showonDSSP{alpha,3-10,pi,beta}     %  Show helices and strands (DSSP input)
\secondcolumnDSSP                   %  Use numbering from 2. column in DSSP
\appearance{PHDtopo}{internal}      % \
           {bottom}{'-'}            %  \
           {int.\ \Alphacount}      %   |
\appearance{PHDtopo}{external}      %   |
           {top}{,-,}               %   |
           {ext.\ \Alphacount}      %   |
\appearance{PHDtopo}{TM}{top}       %   |
     {box[LightGray]:TM\numcount}{} %   |
\appearance{HMMTOP}{internal}       %   |
           {bottom}{'-'}            %   |
           {int.\ \Alphacount}      %   |
\appearance{HMMTOP}{external}       %   |
           {top}{,-,}               %   |
           {ext.\ \Alphacount}      %   |
\appearance{HMMTOP}{TM}{top}        %   |
           {helix}{TM\numcount}     %   |
\appearance{PHDsec}{alpha}{top}     %   |
          {box:$\alpha$\numcount}{} %   |
\appearance{PHDsec}{beta}{top}      %   |
          {-->}{$\beta$\numcount}   %   |
\appearance{STRIDE}{alpha}{top}     %   |
          {box:$\alpha$\numcount}{} %   |    Definitions for the appearance of
\appearance{STRIDE}{3-10}{top}      %    \
           {fill:$\circ$}{3$_{10}$} %     >  secondary structures included from|
\appearance{STRIDE}{pi}             %    /
           {top}{---}{$\pi$}        %   |    PHD-, STRIDE-, or DSSP-files.
\appearance{STRIDE}{beta}{top}      %   |
          {-->}{$\beta$\numcount}   %   |
\appearance{STRIDE}{bridge}         %   |
           {top}{fill:$\uparrow$}{} %   |
\appearance{STRIDE}{turn}           %   |
           {top}{,-,}{turn}         %   |
\appearance{DSSP}{alpha}{top}       %   |
          {box:$\alpha$\numcount}{} %   |
\appearance{DSSP}{3-10}{top}        %   |
           {fill:$\circ$}{3$_{10}$} %   |
\appearance{DSSP}{pi}               %   |
           {top}{---}{$\pi$}        %   |
\appearance{DSSP}{beta}{top}        %   |
          {-->}{$\beta$\numcount}   %   |
\appearance{DSSP}{bridge}           %   |
           {top}{fill:$\uparrow$}{} %   |
\appearance{DSSP}{turn}             %   |
           {top}{,-,}{turn}         %   |
\appearance{DSSP}{bend}{top}        %  /
           {fill:$\diamond$}{}      % /

\pepgroups{FYW,ILVM,RK,DE,GA,ST,NQ} %  Amino acid grouping due to similarity

\pepsims{F}{YW}                     %  Y and W are similar to F
\pepsims{Y}{WF}                     %  W and F are similar to Y
\pepsims{W}{YF}                     %  Y and F are similar to W

\pepsims{I}{LVM}                    %  L, V and M are similar to I
\pepsims{L}{VMI}                    %  V, M and I are similar to L
\pepsims{V}{MIL}                    %  M, I and L are similar to V

\pepsims{R}{KH}                     %  K and H are similar to R
\pepsims{K}{HR}                     %  H and R are similar to K
\pepsims{H}{RK}                     %  R and K are similar to H

\pepsims{A}{GS}                     %  G and S are similar to A
\pepsims{G}{A}                      %  A (but not S) is similar to G

\pepsims{S}{TA}                     %  T and A are similar to S
\pepsims{T}{S}                      %  S (but not A) is similar to T

\pepsims{D}{EN}                     %  E and N (but not Q) are similar to D
\pepsims{E}{DQ}                     %  D and Q (but not N) are similar to E
\pepsims{N}{QD}                     %  Q and D (but not E) are similar to N
\pepsims{Q}{NE}                     %  N and E (but not D) are similar to Q

\DNAgroups{GAR,CTY}                 %  Nucleotide grouping due to similarity

\DNAsims{A}{GR}                     %  G and R are similar to A
\DNAsims{G}{AR}                     %  A and R are similar to G
\DNAsims{R}{AG}                     %  A and G are similar to R

\DNAsims{C}{TY}                     %  T and Y are similar to C
\DNAsims{T}{CY}                     %  C and Y are similar to T
\DNAsims{Y}{CT}                     %  C and T are similar to Y

}
\catcode`\@=12
%</texshade>
%    \end{macrocode}
%    \subsection{\file{texshade.def}}
%    \begin{macrocode}
%<*definitions>
%%%%%%%%%%%%%%%%%%%%%%%%%%%%%%%%%%%%%%%%%%%%%%%%%%%%%%%%%%%%%%%%%%%%%%%%%%%%%%%%
%%%%%                                                                      %%%%%
%%%%%   Default parameter settings for the  LaTeX  ``TeXshade''  package   %%%%%
%%%%%                                                                      %%%%%
%%%%%%%%%%%%%%%%%%%%%%%%%%%%%%%%%%%%%%%%%%%%%%%%%%%%%%%%%%%%%%%%%%%%%%%%%%%%%%%%
%%%%%                                                                      %%%%%
%%%%%   This example file contains all standard settings of the TeXshade   %%%%%
%%%%%   package. It can be used as a template for the creation of perso-   %%%%%
%%%%%   nal parameter files.  All TeXshade user commands are allowed and   %%%%%
%%%%%   functional when specified here.                                    %%%%%
%%%%%                                                                      %%%%%
%%%%%   To activate these settings for your alignment load  this file by   %%%%%
%%%%%   naming it as optional parameter at the beginning of the texshade   %%%%%
%%%%%   environment, e.g.                                                  %%%%%
%%%%%                                                                      %%%%%
%%%%%          \begin{texshade}[myparameterfile]{alignmentfile}            %%%%%
%%%%%             .                                                        %%%%%
%%%%%             .                                                        %%%%%
%%%%%          \end{texshade}                                              %%%%%
%%%%%                                                                      %%%%%
%%%%%%%%%%%%%%%%%%%%%%%%%%%%%%%%%%%%%%%%%%%%%%%%%%%%%%%%%%%%%%%%%%%%%%%%%%%%%%%%


\shadingmode{identical}             %  Shade identical residues only
\shadingcolors{blues}               %  Select the blue color scheme for shading
\constoallseqs                      %  Calculate consensus considering all seqs
\threshold{50}                      %  Consensus threshold percentage is 50
\residuesperline{999}               %  As many residues as possible per line
\numberingwidth{9999}               %  Assign space for 4 digit numbering
\charstretch{1}                     %  Do not stretch character width
\linestretch{1}                     %  Do not stretch lines vertically
\gapchar{.}                         %  . is printed in sequence gaps
\gaprule{0.3pt}                     %  If a rule is printed in gaps use 0.3 pt
\gapcolors{Black}{White}            %  Gap symbols appear `Black on White'
\domaingaprule{1pt}                 %  If a rule is set between domains use 1 pt
\domaingapcolors{Red}{White}        %  Domain separation line appears `Red on White'
\numberingcolor{Black}              %  Numbering color is `Black'
\shownumbering{right}               %  Show sequence numbering on the left
\namescolor{Black}                  %  Names' color is `Black'
\featurenamescolor{Black}           %  Feature names' color is `Black'
\featurestylenamescolor{Black}      %  Feature names' color is `Black'
\shownames{left}                    %  Show sequence names on the right
\consensuscolors{Black}{White}      %  All consensus symbols/letters
                {Black}{White}      %  appear `Black on White'
                {Black}{White}      %
\showconsensus{bottom}              %  Show consensus line at bottom with
\hidesequencelogo                   %  Do not show a sequence logo as consensus
\showlogoscale{leftright}           %  Show vertical scale bar if logo is on
\logostretch{1}                     %  Do not stretch sequence logo vertically
\hidesubfamilylogo                  %  Do not show a subfamily logo
\subfamilythreshold{50}             %  Set subfamily threshold to 50%
\shownegatives                      %  Show negative values in subfamily logo
\showrelevance[Black]{*}            %  Label relevant subfamily deviations
\defconsensus{{}}{*}{!}             %  Blank =no match; * =match; ! =all match
\showleadinggaps                    %  Show gap symbols before sequence start
\rulercolor{Black}                  %  Ruler's color is `Black'
\hideruler                          %  Do not show the ruler
\rulersteps{10}                     %  Ruler ticks every 10 residues
\legendcolor{Black}                 %  Legend text color is `Black'
\hidelegend                         %  Do not show the legend
\alignment{center}                  %  Center alignment on page
\medsepline                         %  Medium height if separation line is on
\medblockskip                       %  Medium skip between sequence blocks
\flexblockspace                     %  Use optimized space between blocks
\featurerule{0.5ex}                 %  Set feature rule thickness to 1/2 ex
\bargraphstretch{1}                 %  Do not stretch bars in feature graphs
\colorscalestretch{1}               %  Do not stretch color scales in features
\backtranstext{horizontal}          %  Horizontal triplets in feature texts
\backtranslabel{alternating}        %  Alternating triplets in feature styles
\setfamily{residues}{tt}            %  Use typewriter family for residues
\setseries{residues}{md}            %  Use normal series for residues
\setshape {residues}{up}            %  Use upright shape for residues
\setsize  {residues}{normalsize}    %  Use normal size for residues
\setfamily{numbering}{tt}           %  Use typewriter family for numbering
\setseries{numbering}{md}           %  Use normal series for numbering
\setshape {numbering}{up}           %  Use upright shape for numbering
\setsize  {numbering}{normalsize}   %  Use normal size for numbering
\setfamily{names}{tt}               %  Use typewriter family for names
\setseries{names}{md}               %  Use normal series for names
\setshape {names}{up}               %  Use upright shape for names
\setsize  {names}{normalsize}       %  Use normal size for names
\setfamily{featurenames}{tt}        %  Use typewriter family for feature names
\setseries{featurenames}{md}        %  Use normal series for feature names
\setshape {featurenames}{up}        %  Use upright shape for feature names
\setsize  {featurenames}{normalsize}%  Use normal size for feature  names
\setfamily{featurestylenames}{tt}   %  Use typewriter family for feature style names
\setseries{featurestylenames}{md}   %  Use normal series for feature style names
\setshape {featurestylenames}{up}   %  Use upright shape for feature style names
\setsize  {featurestylenames}{normalsize}%  Use normal size for feature style names
\setfamily{features}{rm}            %  Use roman family for feature texts
\setseries{features}{md}            %  Use normal series for feature texts
\setshape {features}{it}            %  Use italics shape for feature texts
\setsize  {features}{normalsize}    %  Use normal size for feature texts
\setfamily{featurestyles}{tt}       %  Use typewriter family for feature styles
\setseries{featurestyles}{md}       %  Use normal series for feature styles
\setshape {featurestyles}{up}       %  Use upright shape for feature styles
\setsize {featurestyles}{normalsize}%  Use normal size for feature styles
\setfamily{legend}{tt}              %  Use typewriter family for legend texts
\setseries{legend}{md}              %  Use normal series for legend texts
\setshape {legend}{up}              %  Use upright shape for legend texts
\setsize  {legend}{normalsize}      %  Use normal size for legend texts
\setfamily{ruler}{sf}               %  Use sans serif font for ruler numbers
\tintdefault{medium}                %  Use medium tint intensity
\emphdefault{it}                    %  Use italics to emphasize regions
\showonPHDsec{alpha,beta}           %  Show helices and strands (PHD input)
\showonPHDtopo{internal,external,TM}%  Show int., ext. and TM's (PHD input)
\showonHMMTOP{TM}                   %  Show TM's (not int/ext on HMMTOP input)
\showonSTRIDE{alpha,3-10,pi,beta}   %  Show helices and strands (STRIDE input)
\showonDSSP{alpha,3-10,pi,beta}     %  Show helices and strands (DSSP input)
\secondcolumnDSSP                   %  Use numbering from 2. column in DSSP
\appearance{PHDtopo}{internal}      % \
           {bottom}{'-'}            %  \
           {int.\ \Alphacount}      %   |
\appearance{PHDtopo}{external}      %   |
           {top}{,-,}               %   |
           {ext.\ \Alphacount}      %   |
\appearance{PHDtopo}{TM}{top}       %   |
     {box[LightGray]:TM\numcount}{} %   |
\appearance{HMMTOP}{internal}       %   |
           {bottom}{'-'}            %   |
           {int.\ \Alphacount}      %   |
\appearance{HMMTOP}{external}       %   |
           {top}{,-,}               %   |
           {ext.\ \Alphacount}      %   |
\appearance{HMMTOP}{TM}{top}        %   |
           {helix}{TM\numcount}     %   |
\appearance{PHDsec}{alpha}{top}     %   |
          {box:$\alpha$\numcount}{} %   |
\appearance{PHDsec}{beta}{top}      %   |
          {-->}{$\beta$\numcount}   %   |
\appearance{STRIDE}{alpha}{top}     %   |
          {box:$\alpha$\numcount}{} %   |    Definitions for the appearance of
\appearance{STRIDE}{3-10}{top}      %    \
           {fill:$\circ$}{3$_{10}$} %     >  secondary structures included from|
\appearance{STRIDE}{pi}             %    /
           {top}{---}{$\pi$}        %   |    PHD-, STRIDE-, or DSSP-files.
\appearance{STRIDE}{beta}{top}      %   |
          {-->}{$\beta$\numcount}   %   |
\appearance{STRIDE}{bridge}         %   |
           {top}{fill:$\uparrow$}{} %   |
\appearance{STRIDE}{turn}           %   |
           {top}{,-,}{turn}         %   |
\appearance{DSSP}{alpha}{top}       %   |
          {box:$\alpha$\numcount}{} %   |
\appearance{DSSP}{3-10}{top}        %   |
           {fill:$\circ$}{3$_{10}$} %   |
\appearance{DSSP}{pi}               %   |
           {top}{---}{$\pi$}        %   |
\appearance{DSSP}{beta}{top}        %   |
          {-->}{$\beta$\numcount}   %   |
\appearance{DSSP}{bridge}           %   |
           {top}{fill:$\uparrow$}{} %   |
\appearance{DSSP}{turn}             %   |
           {top}{,-,}{turn}         %   |
\appearance{DSSP}{bend}{top}        %  /
           {fill:$\diamond$}{}      % /

\pepgroups{FYW,ILVM,RK,DE,GA,ST,NQ} %  Amino acid grouping due to similarity

\pepsims{F}{YW}                     %  Y and W are similar to F
\pepsims{Y}{WF}                     %  W and F are similar to Y
\pepsims{W}{YF}                     %  Y and F are similar to W

\pepsims{I}{LVM}                    %  L, V and M are similar to I
\pepsims{L}{VMI}                    %  V, M and I are similar to L
\pepsims{V}{MIL}                    %  M, I and L are similar to V

\pepsims{R}{KH}                     %  K and H are similar to R
\pepsims{K}{HR}                     %  H and R are similar to K
\pepsims{H}{RK}                     %  R and K are similar to H

\pepsims{A}{GS}                     %  G and S are similar to A
\pepsims{G}{A}                      %  A (but not S) is similar to G

\pepsims{S}{TA}                     %  T and A are similar to S
\pepsims{T}{S}                      %  S (but not A) is similar to T

\pepsims{D}{EN}                     %  E and N (but not Q) are similar to D
\pepsims{E}{DQ}                     %  D and Q (but not N) are similar to E
\pepsims{N}{QD}                     %  Q and D (but not E) are similar to N
\pepsims{Q}{NE}                     %  N and E (but not D) are similar to Q

\DNAgroups{GAR,CTY}                 %  Nucleotide grouping due to similarity

\DNAsims{A}{GR}                     %  G and R are similar to A
\DNAsims{G}{AR}                     %  A and R are similar to G
\DNAsims{R}{AG}                     %  A and G are similar to R

\DNAsims{C}{TY}                     %  T and Y are similar to C
\DNAsims{T}{CY}                     %  C and Y are similar to T
\DNAsims{Y}{CT}                     %  C and T are similar to Y
%</definitions>
%    \end{macrocode}
%    \begin{macrocode}
%<*AQPDNA>
AQPDNA.MSF  MSF: 979   Type: N  Freitag, 12. Februar 1999 Check: 2594 ..
Name: AQP1nuc.SEQ  Len: 807  Check:  8330    Weight: 1.00
Name: AQP2nuc.SEQ  Len: 813  Check:  7220    Weight: 1.00
Name: AQP3nuc.SEQ  Len: 855  Check:  7590    Weight: 1.00
Name: AQP4nuc.SEQ  Len: 960  Check:  8696    Weight: 1.00
Name: AQP5nuc.SEQ  Len: 795  Check:  758    Weight: 1.00
//
              1                                                      60
AQP1nuc.SEQ  ATGGCCAGCGAAATCAAGAAGAAGC.................TCTTCT........GGAG
AQP2nuc.SEQ  ATGTG....GGAACTCAG.........................ATCCAT...........
AQP3nuc.SEQ  ATG........AACC........GTTGCGGGG.AGATG.....CTCC.............
AQP4nuc.SEQ  ATGAGTGACGGAGCTGCAGCGAGGCGGTGGGGTAAGTGTGGACCTCCCTGCAGCAGAGAG
AQP5nuc.SEQ  ATGAAAAA.GGAGGTGTG.........................CTCCCT...........

              61                                                      120
AQP1nuc.SEQ  GGC..TGTGGTGGCT.....GAGTTCCTGGCCATGA.CCCTCTTCG..............
AQP2nuc.SEQ  ...................................AGCCTTCTCCCGAGCAGTGCTGGCT
AQP3nuc.SEQ  .ACATCC.....GCTACCGG......CTG.........CTTCGCCA....GGCTCTGGCG
AQP4nuc.SEQ  AGCATCATGGTGGCTTTCAAAGGCGTCTGGACTCAAGCCTTCTGGAAGGCGGTCACAGCA
AQP5nuc.SEQ  ...................................TGCCTTCTTCAAGGCGGTGTTCGCA

              121                                                      180
AQP1nuc.SEQ  ....TCTTCATCAGCATCGGTTCTGCCCTA...GGCTT.....CAATTACCCACTGGAGA
AQP2nuc.SEQ  GAGTTCTTGGCCACGCTCCTTTTTGTCTTCTTTGGCCTTGGCTCAGCCCTCCA.....GT
AQP3nuc.SEQ  GAGTGCCTGGGGACCCTCATCCTTGTGATGTTCGGCTGTGGTTCCGTGGCTCAA.GTGGT
AQP4nuc.SEQ  GAGTTCCTGGCCATGCTCATCTTTGTTCTGCTCAGCGTGGGATCCACCATTAACTGGGGT
AQP5nuc.SEQ  GAGTTCCTGGCCACCCTCATCTTCGTCTTCTTTGGCCTGGGCTCAGCACTCAA.....GT

              181                                                      240
AQP1nuc.SEQ  GA...AACCAGACGCTGGTCCA.GGACAATGTGAAGGTGTCACTGGCCTTTGGTCTGAGC
AQP2nuc.SEQ  GGGCCAGCT....CCCCACCCTC...TGTGCTCCAGATCGCCGTGGCCTTTGGTCTGGGC
AQP3nuc.SEQ  GCTCAGCCGAGGGACCCATG.GTGG.CTTCCTCACCATCAACTTGGCTTTTGGCTTCGCT
AQP4nuc.SEQ  GGCTCAGAGAACCCCCTACCTGTGGACATGGTCCTCATCTCCCTCTGCTTTGGACTCAGC
AQP5nuc.SEQ  GGCCCTCGG....CTCTGCCCAC...CATTCTGCAAATCTCAATTGCCTTTGGCCTGGCC

              241                                                      300
AQP1nuc.SEQ  ATCGCTACTCTGGCCCAAAGTGTGGGTCACATCAGTGGTGCTCACTCCAACCCAGCGGTC
AQP2nuc.SEQ  ATCGGCATCCTGGTTCAGGCTCTGGGCCATGTCAGCGGGGCACACATCAACCCCGCCGTG
AQP3nuc.SEQ  GTCACCCTTGCCATCTTGGTGGCTGGCCAAGTGTCTGGAGCCCACTTGAACCCTGCTGTG
AQP4nuc.SEQ  ATTGCCACCATGGTTCAGTGCTTCGGCCACATCAGCGGTGGCCACATCAACCCAGCGGTG
AQP5nuc.SEQ  ATAGGTACCTTAGCCCAAGCTCTGGGACCTGTGAGTGGTGGCCACATCAATCCAGCCATT

              301                                                      360
AQP1nuc.SEQ  ACACTGGGGCTTCTGCTCAGCTGTCAGATCAGCATCCTCCGGGCTGTCA.TGTATATCAT
AQP2nuc.SEQ  ACTGTGGCATGCCTGGTGGGTTGCCATGTCTCCTTCCTTCGAGCTGCCT.TCTATGTGGC
AQP3nuc.SEQ  ACCTTTGCAATG.TGCTTCCTGGCACGAGAGCCCTGGATCAAGCTGCCCATCTACACACT
AQP4nuc.SEQ  ACAGTGGCCATGGTGTGCACACGAAAGATCAGCATCGCCAAGTCTGTCT.TCTACATCAC
AQP5nuc.SEQ  ACTCTGGCCCTCTTAATAGGAAACCAGATCTCGCTGCTCCGAGCTGTCT.TCTACGTGGC

              361                                                      420
AQP1nuc.SEQ  CGCCCAGTGTGTGGGAGCCATCGTTGCCTCCGCCATCCTCTCCGGCATCACCTCCTCCCT
AQP2nuc.SEQ  TGCCCAGCTGCTGGGCGCCGTGGCTGGGGCTGCCATCCTCCATGAGATTAC.TCCAGTAG
AQP3nuc.SEQ  GGCACAGACCCTCGGGGCCTTCTTGGGTGCTGGGATTGTTTTTGGGCT..CTACTA..TG
AQP4nuc.SEQ  TGCGCAGTGCCTGGGGGCCATCATCGGAGCTGGGATCCTCTACCTGGTCAC.ACCCCCCA
AQP5nuc.SEQ  AGCCCAGCTGGTGGGCGCCATTGCTGGGGCAGGCATCCTGTACTGGCTGGC.GCCACTCA

              421                                                      480
AQP1nuc.SEQ  GCTCGAGAACTCACTTGGCCGA.AATGACCTGGCTCGAGGTGTGAACTCCGGCCAGGGCC
AQP2nuc.SEQ  AAATCCGTGGGGACCTGGCTGTCAATGCTCTCCACAACAACGCCACAGCTGGCCAGGCTG
AQP3nuc.SEQ  ATGCAATCTGGGCCTTTGCTGGCAATGAGCT.........TGTTGTCTCCGGCC.....C
AQP4nuc.SEQ  GCGTGGTGGGAGGATTGGGAGTCACCACGGTTCATGGAAACCTCACTGCTGGCCATGGGC
AQP5nuc.SEQ  ATGCCCGGGGTAACCTGGCCGTCAATGCGCTGAACAACAACACAACGCCTGGCAAGGCCA

              481                                                      540
AQP1nuc.SEQ  TGGGCATTGAGATCATTGGCACCCTGCAGCTGGTGCTGTGCGT.TCTGGCTACCACTGAC
AQP2nuc.SEQ  TGACTGTAGAGCTCTTCCTGACCATGCAGCTGGTGCTGTGCAT.CTTTGCCTCCACCGAC
AQP3nuc.SEQ  CAATGGCACAGCTGGTATC..TTTGCCACCTATCCCTCTGGACACTTGGATATGGTCAAT
AQP4nuc.SEQ  TCCTGGTGGAGCTAATAATCACTTTCCAGCTGGTATTCACCAT.TTTTGCCAGCTGTGAT
AQP5nuc.SEQ  TGGTGGTGGAGTTAATCTTGACTTTCCAGCTAGCCCTCTGCAT.CTTCTCCTCCACCGAC

              541                                                      600
AQP1nuc.SEQ  CGGAGGCGCCGAGACTTAGGTGGCTCAGCCCCACTTGCCATTGGCTTGTCTGTGGCTCTT
AQP2nuc.SEQ  GAGCGCCGCGGTGACAACCTGGGTAGCCCTGCCCTCTCCATTGGTTTCTCTGTTACCCTG
AQP3nuc.SEQ  GGCTTCTTTGATCAGTTCATAGGCACAGCAGCCCTTATTGTGTGTGTGCTGGCCATTGTT
AQP4nuc.SEQ  TCCAAACGGACTGATGTTACTGGTTCCGTTGCTTTAGCAATTGGGTTTTCCGTTGCAATT
AQP5nuc.SEQ  TCTCGCCGAACCAGCCCTGTGGGCTCCCCAGCCTTATCCATTGGCTTGTCTGTCACACTG

              601                                                      660
AQP1nuc.SEQ  GGACACCTGCTGGCCATTGACTACACTGGCTGTGGGATCAACCCTGCCCGGTCATT.TGG
AQP2nuc.SEQ  GGCCACCTCCTTGGGATCTATTTCACCGGTTGCTCCATGAATCCAGCCCGCTCCCT.GGC
AQP3nuc.SEQ  GACC..CTTATAACAACCCTGTGCCCCGGGGCCTGGAGGCCTTCACTGTGGGCCTTGTGG
AQP4nuc.SEQ  GGACATTTGTTTGCAATCAATTATACCGGAGCCAGCATGAATCCAGCTCGATCCTT.TGG
AQP5nuc.SEQ  GGCCATCTTGTGGGGATCTACTTCACCGGCTGTTCCATGAACCCAGCCCGATCTTT.CGG

              661                                                      720
AQP1nuc.SEQ  CTCTGCTGTGCTCACCCGCAACTTCTCAAAC...CACTGGATTTTCTGGGTGGGACCATT
AQP2nuc.SEQ  TCCAGCAGTTGTCACTGGCAAGTTTGATGA...TCACTGGGTCTTCTGGATCGGACCCCT
AQP3nuc.SEQ  TCCTG.....GTCATTGGGACCTCCATGGGCTTCAATTCTGGCTATGCCGTCAACCCAGC
AQP4nuc.SEQ  CCCTGCAGTTATCATGGGAAACTGGGAAAAC...CACTGGATATATTGGGTTGGACCAAT
AQP5nuc.SEQ  CCCTGCGGTGGTCATGAACCGGTTCAGCCCCTCTCACTGGGTCTTCTGGGTAGGGCCTAT

              721                                                      780
AQP1nuc.SEQ  CATTGGGAGTGCCCTGGCAGTGCTGATCTATGACTTCATC..CTGGCCCCACGC..AGC.
AQP2nuc.SEQ  GGTGGGCGCCATCATCGGCTCCCTCCTCTACAACTAC..CTGCTGTTC..........CC
AQP3nuc.SEQ  T.....CGTGACTTTGG..ACCTCGCCTTTTCACTGCCCTGGCTGGC......TGGGGTT
AQP4nuc.SEQ  CATAGGCGCTGTGCTGGCAGGTGCACTTTACGAGTATGTCTTCTGTCCTGACGTGGAGCT
AQP5nuc.SEQ  TGTGGGGGCCATGCTGGCGGCCATCCTCTATTTCTAC..CTGCTCTTC..........CC

              781                                                      840
AQP1nuc.SEQ  ..AGCG.........................ACTTTACAG.............ACCGCAT
AQP2nuc.SEQ  C.....TCGGCAAAG...AGCCTGCAGGAGCGCTTGGCAGTGCTCAAGGG.......CCT
AQP3nuc.SEQ  CAGAAGTC.TTTACGACTGGCC...AGAACTGGTGGTGGGTACCCATCGTCTCTCCACTC
AQP4nuc.SEQ  CAAACGTCGCCTAAAGGAAGCCTTCAGCAAAGCTGCACAGCAGACGAAAGGGAGCTACAT
AQP5nuc.SEQ  C.....TCCTCTCTG...AGCCTCCATGATCGCGTGGCTGTCGTCAAAGG.......CAC

              841                                                      900
AQP1nuc.SEQ  GAAGGTGTGGACCAGT...GGCCAAGTGGA.....GGAGTATGACCTGGATGC.......
AQP2nuc.SEQ  GGAGCCCGACACCGACTGGGA.......GGAACGTGAAGTGCGG..CGGCGGCAGTCGGT
AQP3nuc.SEQ  CTGGGTTC.CATTGGTGGTGTCTTCGTGT.ACCAGCT..CATGAT.TGGCTGCCACC..T
AQP4nuc.SEQ  GGAGGTGGAGGACAACCGGAGCCAAGTGGAGACAGAAGACTTGATCCTGAAGCCCGGGGT
AQP5nuc.SEQ  ATA...TGA.GCCGG..AGGA.......GGACTGGGAAGATCAT..CGAGAGGAGAGGAA

              901                                                      960
AQP1nuc.SEQ  ........TGAT.GATATCAACTCCAGGGTGGAGATGAAG....................
AQP2nuc.SEQ  GGAGC......TC..CACTCTCCTCAGAG...................CCTGCCTCGCG.
AQP3nuc.SEQ  GGAGCA.GCCCCCGCCTTCCACT..GAGGCAGAGAATGTGAAGCTGG.CCCACATGAAGC
AQP4nuc.SEQ  GGTGCATGTGATCGACATTGACCGTGGAGACGAGAAGAAGGGGAAGGACTCGTCTGGAGA
AQP5nuc.SEQ  GAAG............ACCATC....GAG........................CTGACG.

              961                    979
AQP1nuc.SEQ  ..........CCCAAATAG
AQP2nuc.SEQ  .GCAGCAAGGCCTG....A
AQP3nuc.SEQ  ACAAGGA..GCAGATCTGA
AQP4nuc.SEQ  GGTATTATCTTCTGTATGA
AQP5nuc.SEQ  .GCA.CA....CTG....A

%</AQPDNA>
%    \end{macrocode}
%    \begin{macrocode}
%<*AQPpro>
AQPpro.MSF  MSF: 356   Type: P  Freitag, 12. Februar 1999 Check: 2586 ..
Name: AQP1.PRO  Len: 269  Check:  5367    Weight: 1.00
Name: AQP2.PRO  Len: 271  Check:  6176    Weight: 1.00
Name: AQP3.PRO  Len: 285  Check:  2893    Weight: 1.00
Name: AQP4.PRO  Len: 323  Check:  9737    Weight: 1.00
Name: AQP5.PRO  Len: 265  Check:  8413    Weight: 1.00
//
          1                                                      60
AQP1.PRO  MAS........................EIKKKLFWRAVVAEFLAMTLFVFISIGSALGFN
AQP2.PRO  MW.........................ELRSIAFSRAVLAEFLATLLFVFFGLGSALQWA
AQP3.PRO  M.........NRCG.....EMLHIRYR......LLRQALAECLGTLILVMFGCGSVAQVV
AQP4.PRO  MSDGAAARRWGKCGPPCSRESIMVAFKGVWTQAFWKAVTAEFLAMLIFVLLSVGSTINWG
AQP5.PRO  MK........................KEVCSLAFFKAVFAEFLATLIFVFFGLGSALKWP

          61                                                      120
AQP1.PRO  YPLERNQTLVQDNVKVSLAFGLSIATLAQSVGHISGAHSNPAVTLGLLLSCQISILRAVM
AQP2.PRO  ...SS....PPSVLQIAVAFGLGIGILVQALGHVSGAHINPAVTVACLVGCHVSFLRAAF
AQP3.PRO  LSRGTH....GGFLTINLAFGFAVTLAILVAGQVSGAHLNPAVTFAMCFLAREPWIKLPI
AQP4.PRO  ...GSENPLPVDMVLISLCFGLSIATMVQCFGHISGGHINPAVTVAMVCTRKISIAKSVF
AQP5.PRO  ...SA....LPTILQISIAFGLAIGTLAQALGPVSGGHINPAITLALLIGNQISLLRAVF

          121                                                      180
AQP1.PRO  YIIAQCVGAIVASAILSGI..........TSSLLENSLGRNDLARGVNSGQ.....GLGI
AQP2.PRO  YVAAQLLGAVAGAAILHEI..........TPVEIRGDLAVNALHNNATAGQ.....AVTV
AQP3.PRO  YTLAQTLGAFLGAGIVFGLYYDAIWAFAGNELVVSGPNGTAGIFATYPSGHLDMVNGFFD
AQP4.PRO  YITAQCLGAIIGAGILYLV..........TPPSVVGGLGVTTVHGNLTAGH.....GLLV
AQP5.PRO  YVAAQLVGAIAGAGILYWL..........APLNARGNLAVNALNNNTTPGK.....AMVV

          181                                                      240
AQP1.PRO  EIIGTLQLVLCVLATTDR.RRRDLGGSAPLAIGLSVALGHLLAIDYTGCGINPARSFGSA
AQP2.PRO  ELFLTMQLVLCIFASTDE.RRGDNLGSPALSIGFSVTLGHLLGIYFTGCSMNPARSLAPA
AQP3.PRO  QFIGTAALIVCVLAIVDPYNNPVPRGLEAFTVGLVVLVIGTSMGFNSGYAVNPARDFGPR
AQP4.PRO  ELIITFQLVFTIFASCDS.KRTDVTGSVALAIGFSVAIGHLFAINYTGASMNPARSFGPA
AQP5.PRO  ELILTFQLALCIFSSTDS.RRTSPVGSPALSIGLSVTLGHLVGIYFTGCSMNPARSFGPA

          241                                                      300
AQP1.PRO  VLTR..NFS.N......HWIFWVGPFIGSALAVL..IYDFILAPRSSDFTDRMK......
AQP2.PRO  VVTG..KFD.D......HWVFWIGPLVGAIIGSL..LYNYLLFPSAKSLQERL..AVLKG
AQP3.PRO  LFTALAGWGSEVFTTGQNW..WWVPIVSPLLGSIGGVFVYQL..................
AQP4.PRO  VIMG..NWE.N......HWIYWVGPIIGAVLAGA..LYEYV.FCPDVELKRRLKEAFSKA
AQP5.PRO  VVMN..RFSPS......HWVFWVGPIVGAMLAAI..LYFYLLFPSSLSLHDRV..AVVKG

          301                                                      354
AQP1.PRO  ......VWTS.....GQVEEYDLDAD.......DINSRVEMKPK..........
AQP2.PRO  .LEPDTDWEEREVRRRQ..SVELHSPQSLPRG...................SKA
AQP3.PRO  .................MIGCHLEQPPPSTEAENV.KLAHMKHKE.......QI
AQP4.PRO  AQQTKGSYMEVEDNRSQVETEDLILKPGVVHVIDIDRGDEKKGKDSSGEVLSSV
AQP5.PRO  TYEPEEDWEDHREERKK..TIELTAH............................

%</AQPpro>
%    \end{macrocode}
%    \begin{macrocode}
%<*AQP2spec>
AQP2bt  SIAFSRAVLAEFLATLLFVFFGLGSALNWPQALPSVLQIAMAFGLAIGTLVQALGHVSGA
AQP2cf  SVAFSRAVFAEFLATLLFVFFGLGSALNWPQALPSVLQIAMAFGLGIGTLVQALGHVSGA
AQP2dd  SIAFSRAVFSEFLATLLFVFFGLGSALNWPQALPSVLQIAMAFGLAIGTLVQALGHISGA
AQP2ec  SIAFSRAVLAEFLATLLFVFFGLGSALNWPQAMPSVLQIAMAFGLAIGTLVQALGHVSGA
AQP2em  SIAFSRAVFSEFLATLLFVFFGLGSALNWPQALPSVLQIAMAFGLAIGTLVQTLGHISGA


AQP2bt  HINPAVTVACLVGCHVSFLRAVFYVAAQLLGAVAGAALLHEITPPAIRG
AQP2cf  HINPAVTVACLVGCHVSFLRAAFYVAAQLLGAVAGAALLHEITPPHVRG
AQP2dd  HINPAVTVACLVGCHVSFLRATFYLAAQLLGAVAGAAILHEITPPDIRG
AQP2ec  HINPAVTVACLVGCHVSFLRAAFYVAAQLLGAVAGAALLHEITPPDIRR
AQP2em  HINPAVTVACLVGCHVSFLRATFYLAAQLLGAVAGAALLHELTPPDIRG

%</AQP2spec>
%    \end{macrocode}
%    \begin{macrocode}
%<*AQP1topo>    
\feature{bottom}{1}{1..14}{'-'}{int.\ A}
\feature{top}{1}{15..32}{box[LightGray]:TM1}{}
\feature{top}{1}{33..49}{,-,}{ext.\ B}
\feature{top}{1}{50..68}{box[LightGray]:TM2}{}
\feature{bottom}{1}{69..81}{'-'}{int.\ C}
\feature{top}{1}{82..106}{box[LightGray]:TM3}{}
\feature{top}{1}{107..136}{,-,}{ext.\ D}
\feature{top}{1}{137..154}{box[LightGray]:TM4}{}
\feature{bottom}{1}{155..168}{'-'}{int.\ E}
\feature{top}{1}{169..186}{box[LightGray]:TM5}{}
\feature{top}{1}{187..211}{,-,}{ext.\ F}
\feature{top}{1}{212..230}{box[LightGray]:TM6}{}
\feature{bottom}{1}{231..269}{'-'}{int.\ G}
%</AQP1topo>
%    \end{macrocode}
%    \begin{macrocode}
%<*AQP1PHD>

From phd@EMBL-Heidelberg.de Wed Nov 25 10:24:25 1998
Date: Tue, 24 Nov 1998 17:45:25 +0100
From: Protein Prediction <phd@EMBL-Heidelberg.de>
To: eric.beitz@uni-tuebingen.de
Subject: PredictProtein




The following information has been received by the server:
~~~~~~~~~~~~~~~~~~~~~~~~~~~~~~~~~~~~~~~~~~~~~~~~~~~~~~~~~

________________________________________________________________________________

reference predict_h25873 (Tue Nov 24 17:43:21 MET 1998)
from eric.beitz@uni-tuebingen.de
password(###)
resp MAIL
orig HTML
prediction of: -secondary structure   (PHDsec)-solvent accessibility (PHDacc)-
return msf format
# no description
MASEIKKKLFWRAVVAEFLAMTLFVFISIGSALGFNYPLERNQTLVQDNVKVSLAFGLSIATLAQSVGHISGAHSNPAVT
LGLLLSCQISILRAVMYIIAQCVGAIVASAILSGITSSLLENSLGRNDLARGVNSGQGLGIEIIGTLQLVLCVLATTDRR
RRDLGGSAPLAIGLSVALGHLLAIDYTGCGINPARSFGSAVLTRNFSNHWIFWVGPFIGSALAVLIYDFILAPRSSDFTD
RMKVWTSGQVEEYDLDADDINSRVEMKPK

________________________________________________________________________________





Result of PROSITE search (Amos Bairoch):    
~~~~~~~~~~~~~~~~~~~~~~~~~~~~~~~~~~~~~~~

please quote: A Bairoch, P Bucher & K Hofmann: The PROSITE database,
its status in 1997. Nucl. Acids Res., 1997, 25, 217-221.

________________________________________________________________________________


--------------------------------------------------------

--------------------------------------------------------

Pattern-ID: ASN_GLYCOSYLATION PS00001 PDOC00001
Pattern-DE: N-glycosylation site
Pattern:    N[^P][ST][^P]
   42       NQTL
   250      NFSN

Pattern-ID: GLYCOSAMINOGLYCAN PS00002 PDOC00002
Pattern-DE: Glycosaminoglycan attachment site
Pattern:    SG.G
   135      SGQG

Pattern-ID: PKC_PHOSPHO_SITE PS00005 PDOC00005
Pattern-DE: Protein kinase C phosphorylation site
Pattern:    [ST].[RK]
   157      TDR
   398      TDR

Pattern-ID: CK2_PHOSPHO_SITE PS00006 PDOC00006
Pattern-DE: Casein kinase II phosphorylation site
Pattern:    [ST].{2}[DE]
   118      SLLE
   383      SRVE

Pattern-ID: MYRISTYL PS00008 PDOC00008
Pattern-DE: N-myristoylation site
Pattern:    G[^EDRKHPFYW].{2}[STAGCN][^P]
   30       GSALGF
   92       GLSIAT
   179      GLLLSC
   288      GAIVAS
   407      GITSSL
   544      GVNSGQ
   722      GLSVAL
   917      GINPAR
   1141     GSALAV

Pattern-ID: PROKAR_LIPOPROTEIN PS00013 PDOC00013
Pattern-DE: Prokaryotic membrane lipoprotein lipid attachment site
Pattern:    [^DERK]{6}[LIVMFWSTAG]{2}[LIVMFYSTAGCQ][AGS]C
   77       PAVTLGLLLSC

Pattern-ID: MIP PS00221 PDOC00193
Pattern-DE: MIP family signature
Pattern:    [HNQA].NP[STA][LIVMF][ST][LIVMF][GSTAFY]
   74       HSNPAVTLG




________________________________________________________________________________





Result of ProDom domain search (Corpet, Gouzy, Kahn):
~~~~~~~~~~~~~~~~~~~~~~~~~~~~~~~~~~~~~~~~~~~~~~~~~~~~

- please quote: ELL Sonnhammer & D Kahn, Prot. Sci., 1994, 3, 482-492

________________________________________________________________________________


--- ------------------------------------------------------------
--- Results from running BLAST against PRODOM domains
---
--- PLEASE quote:
---       F Corpet, J Gouzy, D Kahn (1998).  The ProDom database
---       of protein domain families. Nucleic Ac Res 26:323-326.
---
--- BEGIN of BLASTP output
BLASTP 1.4.7 [16-Oct-94] [Build 17:06:52 Oct 31 1994]

Reference:  Altschul, Stephen F., Warren Gish, Webb Miller, Eugene W. Myers,
and David J. Lipman (1990).  Basic local alignment search tool.  J. Mol. Biol.
215:403-10.

Query=  prot (#) ppOld, no description /home/phd/server/work/predict_h25873
        (269 letters)

Database:  /home/phd/ut/prodom/prodom_34_2
           53,597 sequences; 6,740,067 total letters.
Searching..................................................done

                                                                     Smallest
                                                                       Sum
                                                              High  Probability
Sequences producing High-scoring Segment Pairs:              Score  P(N)      N

 390 p34.2 (45) MIP(6) AQP1(4) GLPF(4)  // PROTEIN INTRIN...   270  2.0e-32   1
 45663 p34.2 (1) AQPZ_ECOLI // AQUAPORIN Z.                     90  3.2e-13   2
 45611 p34.2 (1) AQP2_HUMAN // AQUAPORIN-CD (AQP-CD) (WAT...   136  6.0e-13   1
 304 p34.2 (61) AQP2(10) GLPF(6) MIP(5)  // PROTEIN CHANN...   121  9.2e-11   1
 45607 p34.2 (1) PMIP_NICAL // POLLEN-SPECIFIC MEMBRANE I...    80  1.2e-07   2
 45606 p34.2 (1) BIB_DROME // NEUROGENIC PROTEIN BIG BRAIN.     80  1.2e-05   2
 2027 p34.2 (15) GLPF(9) AQP3(2)  // PROTEIN FACILITATOR ...    60  3.4e-05   2
 45615 p34.2 (1) GLPF_STRPN // GLYCEROL UPTAKE FACILITATO...    63  0.024     1
 45638 p34.2 (1) AQP5_HUMAN // AQUAPORIN 5.                     61  0.044     1



>390 p34.2 (45) MIP(6) AQP1(4) GLPF(4)  // PROTEIN INTRINSIC CHANNEL WATER
  AQUAPORIN TONOPLAST MEMBRANE FOR PLASMA LENS
  Length = 88

 Score = 270 (125.3 bits), Expect = 2.0e-32, P = 2.0e-32
 Identities = 47/67 (70%), Positives = 56/67 (83%)

Query:   156 TTDRRRRDLGGSAPLAIGLSVALGHLLAIDYTGCGINPARSFGSAVLTRNFSNHWIFWVG 215
             T D+RR  +GGSAPL IG SVALGHL+ I YTGCG+NPARSFG AV+T NF+NHW++WVG
Sbjct:    22 TDDKRRGSVGGSAPLPIGFSVALGHLIGIPYTGCGMNPARSFGPAVVTGNFTNHWVYWVG 81

Query:   216 PFIGSAL 222
             P IG+ L
Sbjct:    82 PIIGAVL 88

 Score = 95 (44.1 bits), Expect = 2.3e-06, P = 2.3e-06
 Identities = 20/33 (60%), Positives = 23/33 (69%)

Query:   136 GQGLGIEIIGTLQLVLCVLATTDRRRRDLGGSA 168
             GQ L +EIIGT QLV CV ATTD +RR   G +
Sbjct:     1 GQNLVVEIIGTFQLVYCVFATTDDKRRGSVGGS 33


>45663 p34.2 (1) AQPZ_ECOLI // AQUAPORIN Z.
  Length = 96

 Score = 90 (41.8 bits), Expect = 3.2e-13, Sum P(2) = 3.2e-13
 Identities = 18/36 (50%), Positives = 25/36 (69%)

Query:   166 GSAPLAIGLSVALGHLLAIDYTGCGINPARSFGSAV 201
             G AP+AIGL++ L HL++I  T   +NPARS   A+
Sbjct:    25 GFAPIAIGLALTLIHLISIPVTNTSVNPARSTAVAI 60

 Score = 63 (29.2 bits), Expect = 3.2e-13, Sum P(2) = 3.2e-13
 Identities = 11/25 (44%), Positives = 14/25 (56%)

Query:   210 WIFWVGPFIGSALAVLIYDFILAPR 234
             W FWV P +G  +  LIY  +L  R
Sbjct:    71 WFFWVVPIVGGIIGGLIYRTLLEKR 95


>45611 p34.2 (1) AQP2_HUMAN // AQUAPORIN-CD (AQP-CD) (WATER CHANNEL PROTEIN FOR
  RENAL COLLECTING DUCT) (ADH WATER CHANNEL) (AQUAPORIN 2) (COLLECTING DUCT
  WATER CHANNEL PROTEIN) (WCH-CD).
  Length = 49

 Score = 136 (63.1 bits), Expect = 6.0e-13, P = 6.0e-13
 Identities = 23/42 (54%), Positives = 34/42 (80%)

Query:    50 VKVSLAFGLSIATLAQSVGHISGAHSNPAVTLGLLLSCQISI 91
             +++++AFGL I TL Q++GHISGAH NPAVT+  L+ C +S+
Sbjct:     8 LQIAMAFGLGIGTLVQALGHISGAHINPAVTVACLVGCHVSV 49


>304 p34.2 (61) AQP2(10) GLPF(6) MIP(5)  // PROTEIN CHANNEL WATER AQUAPORIN
  INTRINSIC DUCT COLLECTING FOR TONOPLAST WCH-CD
  Length = 43

 Score = 121 (56.1 bits), Expect = 9.2e-11, P = 9.2e-11
 Identities = 24/43 (55%), Positives = 31/43 (72%)

Query:    70 ISGAHSNPAVTLGLLLSCQISILRAVMYIIAQCVGAIVASAIL 112
             ISG H NPAVT+GLL+  +   LRAV YI AQ +GA+  +A+L
Sbjct:     1 ISGGHINPAVTIGLLIGGRFPFLRAVFYIAAQLLGAVAGAALL 43


>45607 p34.2 (1) PMIP_NICAL // POLLEN-SPECIFIC MEMBRANE INTEGRAL PROTEIN.
  Length = 69

 Score = 80 (37.1 bits), Expect = 1.2e-07, Sum P(2) = 1.2e-07
 Identities = 17/54 (31%), Positives = 32/54 (59%)

Query:   149 LVLCVLATTDRRRRDLGGSAPLAIGLSVALGHLLAIDYTGCGINPARSFGSAVL 202
             L++ V++      R +G  A +A+G+++ L   +A   +G  +NPARS G A++
Sbjct:    13 LLMFVISGVATDDRAIGQVAGIAVGMTITLNVFVAGPISGASMNPARSIGPAIV 66

 Score = 34 (15.8 bits), Expect = 1.2e-07, Sum P(2) = 1.2e-07
 Identities = 8/18 (44%), Positives = 11/18 (61%)

Query:   136 GQGLGIEIIGTLQLVLCV 153
             GQ L IEII +  L+  +
Sbjct:     1 GQSLAIEIIISFLLMFVI 18


>45606 p34.2 (1) BIB_DROME // NEUROGENIC PROTEIN BIG BRAIN.
  Length = 119

 Score = 80 (37.1 bits), Expect = 1.2e-05, Sum P(2) = 1.2e-05
 Identities = 15/34 (44%), Positives = 24/34 (70%)

Query:     1 MASEIKKKLFWRAVVAEFLAMTLFVFISIGSALG 34
             M +EI+   FWR++++E LA  ++VFI  G+A G
Sbjct:    55 MQAEIRTLEFWRSIISECLASFMYVFIVCGAAAG 88

 Score = 39 (18.1 bits), Expect = 1.2e-05, Sum P(2) = 1.2e-05
 Identities = 9/17 (52%), Positives = 12/17 (70%)

Query:    53 SLAFGLSIATLAQSVGH 69
             +LA GL++ATL Q   H
Sbjct:   103 ALASGLAMATLTQCFLH 119


>2027 p34.2 (15) GLPF(9) AQP3(2)  // PROTEIN FACILITATOR GLYCEROL UPTAKE
  AQUAPORIN DIFFUSION UPTAKE/EFFLUX PEPX 5'REGION ORF1
  Length = 55

 Score = 60 (27.8 bits), Expect = 3.4e-05, Sum P(2) = 3.4e-05
 Identities = 17/46 (36%), Positives = 20/46 (43%)

Query:   156 TTDRRRRDLGGSAPLAIGLSVALGHLLAIDYTGCGINPARSFGSAV 201
             T D      GG  PL +G  V    +     TG  INPAR FG  +
Sbjct:    10 TDDGNNVPSGGLHPLMVGFLVMGIGMSLGGTTGYAINPARDFGPRI 55

 Score = 37 (17.2 bits), Expect = 3.4e-05, Sum P(2) = 3.4e-05
 Identities = 7/10 (70%), Positives = 8/10 (80%)

Query:   149 LVLCVLATTD 158
             L+ CVLA TD
Sbjct:     2 LIACVLALTD 11


>45615 p34.2 (1) GLPF_STRPN // GLYCEROL UPTAKE FACILITATOR PROTEIN.
  Length = 26

 Score = 63 (29.2 bits), Expect = 0.025, P = 0.024
 Identities = 13/23 (56%), Positives = 18/23 (78%)

Query:   205 NFSNHWIFWVGPFIGSALAVLIY 227
             ++S  WI  VGP IG+ALAVL++
Sbjct:     1 DWSYAWIPVVGPVIGAALAVLVF 23


>45638 p34.2 (1) AQP5_HUMAN // AQUAPORIN 5.
  Length = 27

 Score = 61 (28.3 bits), Expect = 0.045, P = 0.044
 Identities = 11/19 (57%), Positives = 18/19 (94%)

Query:    50 VKVSLAFGLSIATLAQSVG 68
             ++++LAFGL+I TLAQ++G
Sbjct:     8 LQIALAFGLAIGTLAQALG 26


Parameters:
  E=0.1
  B=500

  V=500
  -ctxfactor=1.00

  Query                        -----  As Used  -----    -----  Computed  ----
  Frame  MatID Matrix name     Lambda    K       H      Lambda    K       H
   +0      0   BLOSUM62        0.322   0.138   0.394    same    same    same

  Query
  Frame  MatID  Length  Eff.Length   E    S W   T  X     E2  S2
   +0      0      269       269     0.10 69 3  11 22    0.22 33


Statistics:
  Query          Expected         Observed           HSPs       HSPs
  Frame  MatID  High Score       High Score       Reportable  Reported
   +0      0    59 (27.4 bits)  270 (125.3 bits)       14         14

  Query         Neighborhd  Word      Excluded    Failed   Successful  Overlaps
  Frame  MatID   Words      Hits        Hits    Extensions Extensions  Excluded
   +0      0      5349     3124825      609708     2510548     4569         2

  Database:  /home/phd/ut/prodom/prodom_34_2
    Release date:  unknown
    Posted date:  12:24 PM MET DST May 06, 1998
  # of letters in database:  6,740,067
  # of sequences in database:  53,597
  # of database sequences satisfying E:  9
  No. of states in DFA:  564 (111 KB)
  Total size of DFA:  226 KB (256 KB)
  Time to generate neighborhood:  0.03u 0.00s 0.03t  Real: 00:00:00
  Time to search database:  9.80u 0.03s 9.83t  Real: 00:00:10
  Total cpu time:  9.90u 0.06s 9.96t  Real: 00:00:10
--- END of BLASTP output
--- ------------------------------------------------------------
---
--- Again: these results were obtained based on the domain data-
--- base collected by Daniel Kahn and his coworkers in Toulouse.
---
--- PLEASE quote:
---       F Corpet, J Gouzy, D Kahn (1998).  The ProDom database
---       of protein domain families. Nucleic Ac Res 26:323-326.
---
--- The general WWW page is on:
----      ---------------------------------------
---       http://www.toulouse.inra.fr/prodom.html
----      ---------------------------------------
---
--- For WWW graphic interfaces to PRODOM, in particular for your
--- protein family, follow the following links (each line is ONE
--- single link for your protein!!):
---
http://www.toulouse.inra.fr/prodom/cgi-bin/ReqProdomII.pl?id_dom1=390 ==> multiple alignment, consensus, PDB and PROSITE links of domain 390
http://www.toulouse.inra.fr/prodom/cgi-bin/ReqProdomII.pl?id_dom2=390 ==> graphical output of all proteins having domain 390
http://www.toulouse.inra.fr/prodom/cgi-bin/ReqProdomII.pl?id_dom1=45663 ==> multiple alignment, consensus, PDB and PROSITE links of domain 45663
http://www.toulouse.inra.fr/prodom/cgi-bin/ReqProdomII.pl?id_dom2=45663 ==> graphical output of all proteins having domain 45663
http://www.toulouse.inra.fr/prodom/cgi-bin/ReqProdomII.pl?id_dom1=45611 ==> multiple alignment, consensus, PDB and PROSITE links of domain 45611
http://www.toulouse.inra.fr/prodom/cgi-bin/ReqProdomII.pl?id_dom2=45611 ==> graphical output of all proteins having domain 45611
http://www.toulouse.inra.fr/prodom/cgi-bin/ReqProdomII.pl?id_dom1=304 ==> multiple alignment, consensus, PDB and PROSITE links of domain 304
http://www.toulouse.inra.fr/prodom/cgi-bin/ReqProdomII.pl?id_dom2=304 ==> graphical output of all proteins having domain 304
http://www.toulouse.inra.fr/prodom/cgi-bin/ReqProdomII.pl?id_dom1=45607 ==> multiple alignment, consensus, PDB and PROSITE links of domain 45607
http://www.toulouse.inra.fr/prodom/cgi-bin/ReqProdomII.pl?id_dom2=45607 ==> graphical output of all proteins having domain 45607
http://www.toulouse.inra.fr/prodom/cgi-bin/ReqProdomII.pl?id_dom1=45606 ==> multiple alignment, consensus, PDB and PROSITE links of domain 45606
http://www.toulouse.inra.fr/prodom/cgi-bin/ReqProdomII.pl?id_dom2=45606 ==> graphical output of all proteins having domain 45606
http://www.toulouse.inra.fr/prodom/cgi-bin/ReqProdomII.pl?id_dom1=2027 ==> multiple alignment, consensus, PDB and PROSITE links of domain 2027
http://www.toulouse.inra.fr/prodom/cgi-bin/ReqProdomII.pl?id_dom2=2027 ==> graphical output of all proteins having domain 2027
http://www.toulouse.inra.fr/prodom/cgi-bin/ReqProdomII.pl?id_dom1=45615 ==> multiple alignment, consensus, PDB and PROSITE links of domain 45615
http://www.toulouse.inra.fr/prodom/cgi-bin/ReqProdomII.pl?id_dom2=45615 ==> graphical output of all proteins having domain 45615
http://www.toulouse.inra.fr/prodom/cgi-bin/ReqProdomII.pl?id_dom1=45638 ==> multiple alignment, consensus, PDB and PROSITE links of domain 45638
http://www.toulouse.inra.fr/prodom/cgi-bin/ReqProdomII.pl?id_dom2=45638 ==> graphical output of all proteins having domain 45638
---
--- NOTE: if you want to use the link, make sure the entire line
---       is pasted as URL into your browser!
---
--- END of PRODOM
--- ------------------------------------------------------------

________________________________________________________________________________


--- Database used for sequence comparison:
--- SEQBASE    RELEASE 34.0 OF EMBL/SWISS-PROT WITH  59021 SEQUENCES




The alignment that has been used as input to the network is:
~~~~~~~~~~~~~~~~~~~~~~~~~~~~~~~~~~~~~~~~~~~~~~~~~~~~~~~~~~~

________________________________________________________________________________

--- ------------------------------------------------------------
--- MAXHOM multiple sequence alignment
--- ------------------------------------------------------------
---
--- MAXHOM ALIGNMENT HEADER: ABBREVIATIONS FOR SUMMARY
--- ID           : identifier of aligned (homologous) protein
--- STRID        : PDB identifier (only for known structures)
--- PIDE         : percentage of pairwise sequence identity
--- WSIM         : percentage of weighted similarity
--- LALI         : number of residues aligned
--- NGAP         : number of insertions and deletions (indels)
--- LGAP         : number of residues in all indels
--- LSEQ2        : length of aligned sequence
--- ACCNUM       : SwissProt accession number
--- NAME         : one-line description of aligned protein
---
--- MAXHOM ALIGNMENT HEADER: SUMMARY
ID         STRID  IDE WSIM LALI NGAP LGAP LEN2 ACCNUM NAME
aqp1_rat          100  100  269    0    0  269 P29975 PROXIMAL TUBULE) (AQUAPOR
aqp1_mouse         98   99  269    0    0  269 Q02013 PROXIMAL TUBULE) (AQUAPOR
aqp1_human         93   97  269    0    0  269 P29972 PROXIMAL TUBULE) (AQUAPOR
aqp1_bovin         90   95  269    1    2  271 P47865 PROXIMAL TUBULE) (AQUAPOR
aqp1_sheep         90   94  269    2    3  272 P56401 PROXIMAL TUBULE) (AQUAPOR
aqpa_ranes         78   89  268    2    5  272 P50501 AQUAPORIN FA-CHIP.
aqp2_dasno         49   73  109    1    7  109 P79164 PROTEIN) (WCH-CD) (FRAGME
aqp2_bovin         49   73  109    1    7  109 P79099 PROTEIN) (WCH-CD) (FRAGME
aqp2_canfa         48   72  109    1    7  109 P79144 PROTEIN) (WCH-CD) (FRAGME
aqp2_rabit         48   73  109    1    7  109 P79213 PROTEIN) (WCH-CD) (FRAGME
aqp2_elema         47   72  109    1    7  109 P79168 PROTEIN) (WCH-CD) (FRAGME
aqp2_horse         47   72  109    1    7  109 P79165 PROTEIN) (WCH-CD) (FRAGME
aqp2_proha         47   73  109    1    7  109 P79229 PROTEIN) (WCH-CD) (FRAGME
mip_rat            46   73  259    1    7  261 P09011 LENS FIBER MAJOR INTRINSI
aqp2_oryaf         46   72  109    1    7  109 P79200 PROTEIN) (WCH-CD) (FRAGME
mip_mouse          46   73  261    1    7  263 P51180 LENS FIBER MAJOR INTRINSI
mip_ranpi          45   73  261    1    7  263 Q06019 LENS FIBER MAJOR INTRINSI
mip_bovin          45   73  261    1    7  263 P06624 LENS FIBER MAJOR INTRINSI
mip_human          45   73  261    1    7  263 P30301 LENS FIBER MAJOR INTRINSI
mip_chick          45   72  110    1    1  112 P28238 LENS FIBER MAJOR INTRINSI
aqp5_rat           44   71  262    2    8  265 P47864 AQUAPORIN 5.
aqp5_human         44   71  262    2    8  265 P55064 AQUAPORIN 5.
aqp2_human         44   72  261    2    8  271 P41181 PROTEIN) (WCH-CD).
aqp4_human         43   70  266    2    5  323 P55087 AQUAPORIN 4 (WCH4) (MERCU
aqp4_rat           43   70  266    2    5  323 P47863 AQUAPORIN 4 (WCH4) (MERCU
aqp4_mouse         43   69  265    3    6  322 P55088 AQUAPORIN 4 (WCH4) (MERCU
aqp2_rat           42   71  261    2    8  271 P34080 PROTEIN) (WCH-CD).
aqp2_mouse         42   71  261    2    8  271 P56402 PROTEIN) (WCH-CD).
wc2a_arath         42   67  248    4   12  287 P43286 PLASMA MEMBRANE INTRINSIC
aqp6_human         42   68  260    2    9  282 Q13520 AQUAPORIN 6 (AQUAPORIN-2
wc2c_arath         41   66  248    4   12  285 P30302 INTRINSIC PROTEIN) (WSI-T
wc2b_arath         41   66  248    4   12  285 P43287 PLASMA MEMBRANE INTRINSIC
wc1c_arath         41   65  238    4   10  286 Q08733 (TMP-B).
wc1b_arath         41   65  238    4   10  286 Q06611 (TMP-A).
tipw_lyces         40   65  237    4   10  286 Q08451 (RIPENING-ASSOCIATED MEMB
wc1a_arath         40   64  238    4   10  286 P43285 PLASMA MEMBRANE INTRINSIC
tipw_pea           40   64  237    4   11  289 P25794 RESPONSIVE PROTEIN 7A).
tipa_arath         38   64  250    3    9  268 P26587 TONOPLAST INTRINSIC PROTE
aqua_atrca         38   64  246    4   10  282 P42767 AQUAPORIN.
dip_antma          38   65  242    2    4  250 P33560 PROBABLE TONOPLAST INTRIN
aqpz_ecoli         37   59  220    4   17  231 P48838 AQUAPORIN Z (BACTERIAL NO
tip2_tobac         37   64  242    2    4  250 P24422 TONOPLAST INTRINSIC PROTE
tip1_tobac         37   64  242    2    4  250 P21653 TONOPLAST INTRINSIC PROTE
tipg_arath         33   62  241    2    4  251 P25818 TONOPLAST INTRINSIC PROTE
bib_drome          33   60  260    4   10  700 P23645 NEUROGENIC PROTEIN BIG BR
tipr_arath         33   62  243    2    4  253 P21652 TONOPLAST INTRINSIC PROTE
tipa_phavu         33   62  246    2    4  256 P23958 TONOPLAST INTRINSIC PROTE
tipg_orysa         32   62  240    2    5  250 P50156 TONOPLAST INTRINSIC PROTE
---
--- MAXHOM ALIGNMENT: IN MSF FORMAT
MSF of: /home/phd/server/work/predict_h25873-22040.hssp from:    1 to:  269
 /home/phd/server/work/predict_h25873-22040.msfRet  MSF:  269  Type: P 24-Nov-98  17:44:5  Check: 3448  ..


 Name: predict_h258    Len:   269  Check: 8331  Weight:  1.00
 Name: aqp1_rat        Len:   269  Check: 8331  Weight:  1.00
 Name: aqp1_mouse      Len:   269  Check: 7552  Weight:  1.00
 Name: aqp1_human      Len:   269  Check: 6501  Weight:  1.00
 Name: aqp1_bovin      Len:   269  Check: 7067  Weight:  1.00
 Name: aqp1_sheep      Len:   269  Check: 7582  Weight:  1.00
 Name: aqpa_ranes      Len:   269  Check: 4844  Weight:  1.00
 Name: aqp2_dasno      Len:   269  Check: 8933  Weight:  1.00
 Name: aqp2_bovin      Len:   269  Check: 9649  Weight:  1.00
 Name: aqp2_canfa      Len:   269  Check: 8990  Weight:  1.00
 Name: aqp2_rabit      Len:   269  Check: 8787  Weight:  1.00
 Name: aqp2_elema      Len:   269  Check: 9381  Weight:  1.00
 Name: aqp2_horse      Len:   269  Check: 8993  Weight:  1.00
 Name: aqp2_proha      Len:   269  Check: 8855  Weight:  1.00
 Name: mip_rat         Len:   269  Check: 9773  Weight:  1.00
 Name: aqp2_oryaf      Len:   269  Check: 8554  Weight:  1.00
 Name: mip_mouse       Len:   269  Check: 9723  Weight:  1.00
 Name: mip_ranpi       Len:   269  Check: 5937  Weight:  1.00
 Name: mip_bovin       Len:   269  Check: 1430  Weight:  1.00
 Name: mip_human       Len:   269  Check:  372  Weight:  1.00
 Name: mip_chick       Len:   269  Check: 4658  Weight:  1.00
 Name: aqp5_rat        Len:   269  Check: 9033  Weight:  1.00
 Name: aqp5_human      Len:   269  Check: 6547  Weight:  1.00
 Name: aqp2_human      Len:   269  Check: 6209  Weight:  1.00
 Name: aqp4_human      Len:   269  Check: 2589  Weight:  1.00
 Name: aqp4_rat        Len:   269  Check: 4412  Weight:  1.00
 Name: aqp4_mouse      Len:   269  Check: 2845  Weight:  1.00
 Name: aqp2_rat        Len:   269  Check: 5748  Weight:  1.00
 Name: aqp2_mouse      Len:   269  Check: 6526  Weight:  1.00
 Name: wc2a_arath      Len:   269  Check: 4866  Weight:  1.00
 Name: aqp6_human      Len:   269  Check: 9404  Weight:  1.00
 Name: wc2c_arath      Len:   269  Check: 6187  Weight:  1.00
 Name: wc2b_arath      Len:   269  Check: 7328  Weight:  1.00
 Name: wc1c_arath      Len:   269  Check: 8575  Weight:  1.00
 Name: wc1b_arath      Len:   269  Check: 9544  Weight:  1.00
 Name: tipw_lyces      Len:   269  Check: 9283  Weight:  1.00
 Name: wc1a_arath      Len:   269  Check:  598  Weight:  1.00
 Name: tipw_pea        Len:   269  Check: 9253  Weight:  1.00
 Name: tipa_arath      Len:   269  Check: 6544  Weight:  1.00
 Name: aqua_atrca      Len:   269  Check: 2848  Weight:  1.00
 Name: dip_antma       Len:   269  Check: 9619  Weight:  1.00
 Name: aqpz_ecoli      Len:   269  Check: 5641  Weight:  1.00
 Name: tip2_tobac      Len:   269  Check:  490  Weight:  1.00
 Name: tip1_tobac      Len:   269  Check:  622  Weight:  1.00
 Name: tipg_arath      Len:   269  Check: 3231  Weight:  1.00
 Name: bib_drome       Len:   269  Check: 7687  Weight:  1.00
 Name: tipr_arath      Len:   269  Check: 4476  Weight:  1.00
 Name: tipa_phavu      Len:   269  Check: 5563  Weight:  1.00
 Name: tipg_orysa      Len:   269  Check: 3537  Weight:  1.00

//


              1                                                   50
predict_h258  MASEIKKKLF WRAVVAEFLA MTLFVFISIG SALGFNYPLE RNQTLVQDNV
aqp1_rat      MASEIKKKLF WRAVVAEFLA MTLFVFISIG SALGFNYPLE RNQTLVQDNV
aqp1_mouse    MASEIKKKLF WRAVVAEFLA MTLFVFISIG SALGFNYPLE RNQTLVQDNV
aqp1_human    MASEFKKKLF WRAVVAEFLA TTLFVFISIG SALGFKYPVG NNQTAVQDNV
aqp1_bovin    MASEFKKKLF WRAVVAEFLA MILFIFISIG SALGFHYPIK SNQTtvQDNV
aqp1_sheep    MASEFKKKLF WRAVVAEFLA MILFIFISIG SALGFHYPIK SNQTtvQDNV
aqpa_ranes    MASEFKKKAF WRAVIAEFLA MILFVFISIG AALGFNFPIE EKANQtqDIV
aqp2_dasno    ......SVAF SRAVLAEFLA TLIFVFFGLG SALSWPQALP S.......VL
aqp2_bovin    ......SIAF SRAVLAEFLA TLLFVFFGLG SALNWPQALP S.......VL
aqp2_canfa    ......SVAF SRAVFAEFLA TLLFVFFGLG SALNWPQALP S.......VL
aqp2_rabit    ......SIAF SRAVFAEFLA TLLFVFFGLG SALNWPSALP S.......TL
aqp2_elema    ......SIAF SRAVFSEFLA TLLFVFFGLG SALNWPQALP S.......VL
aqp2_horse    ......SIAF SRAVLAEFLA TLLFVFFGLG SALNWPQAMP S.......VL
aqp2_proha    ......SIAF SRAVLSEFLA TLLFVFFGLG SALNWPQALP S.......VL
mip_rat       ...ELRSASF WRAIFAEFFA TLFYVFFGLG SSLRWA.... ...PGPLHVL
aqp2_oryaf    ......SIAF SKAVFSEFLA TLLFVFFGLG SALNWPQALP S.......GL
mip_mouse     .MWELRSASF WRAIFAEFFA TLFYVFFGLG ASLRWA.... ...PGPLHVL
mip_ranpi     .MWEFRSFSF WRAVFAEFFG TMFYVFFGLG ASLKWAAGPA .......NVL
mip_bovin     .MWELRSASF WRAICAEFFA SLFYVFFGLG ASLRWA.... ...PGPLHVL
mip_human     .MWELRSASF WRAIFAEFFA TLFYVFFGLG SSLRWA.... ...PGPLHVL
mip_chick     .......... .......... .......... .......... ..........
aqp5_rat      MKKEVCSLAF FKAVFAEFLA TLIFVFFGLG SALKWPSALP T.......IL
aqp5_human    MKKEVCSVAF LKAVFAEFLA TLIFVFFGLG SALKWPSALP T.......IL
aqp2_human    .MWELRSIAF SRAVFAEFLA TLLFVFFGLG SALNWPQALP S.......VL
aqp4_human    AFKGVWTQAF WKAVTAEFLA MLIFVLLSLG STINWG...G TEKPLPVDMV
aqp4_rat      AFKGVWTQAF WKAVTAEFLA MLIFVLLSVG STINWG...G SENPLPVDMV
aqp4_mouse    AFKGVWTQAF WKAVSAEFLA TLIFVL.GVG STINWG...G SENPLPVDMV
aqp2_rat      .MWELRSIAF SRAVLAEFLA TLLFVFFGLG SALQWASSPP S.......VL
aqp2_mouse    .MWELRSIAY CRAVLAEFLA TLLFVFFGLG SALQWASSPP S.......VL
wc2a_arath    DGAELKKWSF YRAVIAEFVA TLLFLYITVL TVIGYKIQSD TDAGGVdgIL
aqp6_human    MLACRLWKAI SRALFAEFLA TGLYVFFGVG SVMRWPTALP S.......VL
wc2c_arath    DAEELTKWSL YRAVIAEFVA TLLFLYVTVL TVIGYKIQSD TKAGGVdgIL
wc2b_arath    DADELTKWSL YRAVIAEFVA TLLFLYITVL TVIGYKIQSD TKAGGVdgIL
wc1c_arath    EPGELSSWSF YRAGIAEFIA TFLFLYITVL TVMGVKRA.. PNMCASVGIQ
wc1b_arath    EPGELASWSF WRAGIAEFIA TFLFLYITVL TVMGVKR..S PNMCASVGIQ
tipw_lyces    EPGELSSWSF YRAGIAEFMA TFLFLYITIL TVMGLKRSDS LCSSV..GIQ
wc1a_arath    EPGELSSWSF WRAGIAEFIA TFLFLYITVL TVMGVKR..S PNMCASVGIQ
tipw_pea      EPSELTSWSF YRAGIAEFIA TFLFLYITVL TVMGVVRESS KCKTV..GIQ
tipa_arath    RADEATHPDS IRATLAEFLS TFVFVFAAEG SILSLDKLYW EHAAHAGTni
aqua_atrca    DMGELKLWSF WRAAIAEFIA TLLFLYITVA TVIGYKKETD PCASVGL..L
dip_antma     SIGDSFSVAS IKAYVAEFIA TLLFVFAGVG SAIAYNKLTS DAALDPAGLV
aqpz_ecoli    .........M FRKLAAECFG TFWLVFGGCG SAVLAAGFPE ....LGIGFA
tip2_tobac    SIGDSFSVGS LKAYVAEFIA TLLFVFAGVG SAIAYNKLTA DAALDPAGLV
tip1_tobac    SIGDSFSVGS LKAYVAEFIA TLLFVFAGVG SAIAYNKLTA DAALDPAGLV
tipg_arath    RPDEATRPDA LKAALAEFIS TLIFVVAGSG SGMAFNKLTE NGATTPSGLV
bib_drome     MQAEIRTLEF WRSIISECLA SFMYVFIVCG AAAGVGVGAS VSSVL....L
tipr_arath    RPDEATRPDA LKAALAEFIS TLIFVVAGSG SGMAFNKLTE NGATTPSGLV
tipa_phavu    RTDEATHPDS MRASLAEFAS TFIFVFAGEG SGLALVKIYQ DSAFSAGELL
tipg_orysa    SHQEVYHPGA LKAALAEFIS TLIFVFAGQG SGMAFSKLTG GGATTPAGLI

              51                                                 100
predict_h258  KVSLAFGLSI ATLAQSVGHI SGAHSNPAVT LGLLLSCQIS ILRAVMYIIA
aqp1_rat      KVSLAFGLSI ATLAQSVGHI SGAHSNPAVT LGLLLSCQIS ILRAVMYIIA
aqp1_mouse    KVSLAFGLSI ATLAQSVGHI SGAHLNPAVT LGLLLSCQIS ILRAVMYIIA
aqp1_human    KVSLAFGLSI ATLAQSVGHI SGAHLNPAVT LGLLLSCQIS IFRALMYIIA
aqp1_bovin    KVSLAFGLSI ATLAQSVGHI SGAHLNPAVT LGLLLSCQIS VLRAIMYIIA
aqp1_sheep    KVSLAFGLSI ATLAQSVGHI SGAHLNPAVT LGLLLSCQIS ILRAIMYIIA
aqpa_ranes    KVSLAFGISI ATMAQSVGHV SGAHLNPAVT LGCLLSCQIS ILKAVMYIIA
aqp2_dasno    QIALAFGLAI GTLVQALGHV SGAHINPAVT VACLVGCHVS FLRAAFYVAA
aqp2_bovin    QIAMAFGLAI GTLVQALGHV SGAHINPAVT VACLVGCHVS FLRAVFYVAA
aqp2_canfa    QIAMAFGLGI GTLVQALGHV SGAHINPAVT VACLVGCHVS FLRAAFYVAA
aqp2_rabit    QIAMAFGLGI GTLVQALGHV SGAHINPAVT VACLVGCHVS FLRAAFYVAA
aqp2_elema    QIAMAFGLAI GTLVQTLGHI SGAHINPAVT VACLVGCHVS FLRATFYLAA
aqp2_horse    QIAMAFGLAI GTLVQALGHV SGAHINPAVT VACLVGCHVS FLRAAFYVAA
aqp2_proha    QIAMAFGLAI GTLVQTLGHI SGAHINPAVT IACLVGCHVS FLRALFYLAA
mip_rat       QVALAFGLAL ATLVQTVGHI SGAHVNPAVT FAFLVGSQMS LLRAFCYIAA
aqp2_oryaf    QIAMAFGLAI GTLVQTLGHI SGAHINPAVT VACLVGCHVS FLRAIFYVAA
mip_mouse     QVALAFGLAL ATLVQTVGHI SGAHVNPAVT FAFLVGSQMS LLRAFCYIAA
mip_ranpi     VIALAFGLVL ATMVQSIGHV SGAHINPAVT FAFLIGSQMS LFRAIFYIAA
mip_bovin     QVALAFGLAL ATLVQAVGHI SGAHVNPAVT FAFLVGSQMS LLRAICYMVA
mip_human     QVAMAFGLAL ATLVQSVGHI SGAHVNPAVT FAFLVGSQMS LLRAFCYMAA
mip_chick     .......... .......... .......... .......... ..........
aqp5_rat      QISIAFGLAI GTLAQALGPV SGGHINPAIT LALLIGNQIS LLRAVFYVAA
aqp5_human    QIALAFGLAI GTLAQALGPV SGGHINPAIT LALLVGNQIS LLRAFFYVAA
aqp2_human    QIAMAFGLGI GTLVQALGHI SGAHINPAVT VACLVGCHVS VLRAAFYVAA
aqp4_human    LISLCFGLSI ATMVQCFGHI SGGHINPAVT VAMVCTRKIS IAKSVFYIAA
aqp4_rat      LISLCFGLSI ATMVQCFGHI SGGHINPAVT VAMVCTRKIS IAKSVFYITA
aqp4_mouse    LISLCFGLSI ATMVQCLGHI SGGHINPAVT VAMVCTRKIS IAKSVFYIIA
aqp2_rat      QIAVAFGLGI GILVQALGHV SGAHINPAVT VACLVGCHVS FLRAAFYVAA
aqp2_mouse    QIAVAFGLGI GTLVQALGHV SGAHINPAVT VACLVGCHVS FLRAAFYVAA
wc2a_arath    GIAWAFGGMI FILVYCTAGI SGGHINPAVT FGLFLARKVS LPRALLYIIA
aqp6_human    QIAITFNLVT AMAVQVTWKT SGAHANPAVT LAFLVGSHIS LPRAVAYVAA
wc2c_arath    GIAWAFGGMI FILVYCTAGI SGGHINPAVT FGLFLARKVS LIRAVLYMVA
wc2b_arath    GIAWAFGGMI FILVYCTAGI SGGHINPAVT FGLFLARKVS LIRAVLYMVA
wc1c_arath    GIAWAFGGMI FALVYCTAGI SGGHINPAVT FGLFLARKLS LTRAVFYIVM
wc1b_arath    GIAWAFGGMI FALVYCTAGI SGGHINPAVT FGLFLARKLS LTRAVYYIVM
tipw_lyces    GVAWAFGGMI FALVYCTAGI SGGHINPAVT FGLFLARKLS LTRAVFYMVM
wc1a_arath    GIAWAFGGMI FALVYCTAGI SGGHINPAVT FGLFLARKLS LTRALYYIVM
tipw_pea      GIAWAFGGMI FALVYCTAGI SGGHINPAVT FGLFLARKLS LTRAIFYMVM
tipa_arath    LVALAHAFAL FAAVSAAINV SGGHVNPAVT FGALVGGRVT AIRAIYYWIA
aqua_atrca    GIAWSFGGMI FVLVYCTAGI SGGHINPAVT FGLFLARKVS LLRALVYMIA
dip_antma     AVAVAHAFAL FVGVSMAANV SGGHLNPAVT LGLAVGGNIT ILTGLFYWIA
aqpz_ecoli    GVALAFGLTV LTMAFAVGHI SGGHFNPAVT IGLWAGGRFP AKEVVGYVIA
tip2_tobac    AVAVAHAFAL FVGVSIAANI SGGHLNPAVT LGLAVGGNIT ILTGFFYWIA
tip1_tobac    AVAVAHAFAL FVGVSIAANI SGGHLNPAVT LGLAVGGNIT ILTGFFYWIA
tipg_arath    AAAVAHAFGL FVAVSVGANI SGGHVNPAVT FGAFIGGNIT LLRGILYWIA
bib_drome     ATALASGLAM ATLTQCFLHI SGAHINPAVT LALCVVRSIS PIRAAMYITA
tipr_arath    AAAVAHAFGL FVAVSVGANI SGGHVNPAVT FGAFIGGNIT LLRGILYWIA
tipa_phavu    ALALAHAFAL FAAVSASMHV SGGHVNPAVS FGALIGGRIS VIRAVYYWIA
tipg_orysa    AAAVAHAFAL FVAVSVGANI SGGHVNPAVT FGAFVGGNIT LFRGLLYWIA

              101                                                150
predict_h258  QCVGAIVASA ILSGITSSLL ENSLGRNDLA RGVNSGQGLG IEIIGTLQLV
aqp1_rat      QCVGAIVASA ILSGITSSLL ENSLGRNDLA RGVNSGQGLG IEIIGTLQLV
aqp1_mouse    QCVGAIVATA ILSGITSSLV DNSLGRNDLA HGVNSGQGLG IEIIGTLQLV
aqp1_human    QCVGAIVATA ILSGITSSLT GNSLGRNDLA DGVNSGQGLG IEIIGTLQLV
aqp1_bovin    QCVGAIVATA ILSGITSSLP DNSLGLNALA PGVNSGQGLG IEIIGTLQLV
aqp1_sheep    QCVGAIVATV ILSGITSSLP DNSLGLNALA PGVNSGQGLG IEIIGTLQLV
aqpa_ranes    QCLGAVVATA ILSGITSGLE NNSLGLNGLS PGVSAGQGLG VEILVTFQLV
aqp2_dasno    QLLGAVAGAA ILHEITPPDV RG........ .......... ..........
aqp2_bovin    QLLGAVAGAA LLHEITPPAI RG........ .......... ..........
aqp2_canfa    QLLGAVAGAA LLHEITPPHV RG........ .......... ..........
aqp2_rabit    QLLGAVAGAA LLHEITPAEV RG........ .......... ..........
aqp2_elema    QLLGAVAGAA LLHELTPPDI RG........ .......... ..........
aqp2_horse    QLLGAVAGAA LLHEITPPDI RR........ .......... ..........
aqp2_proha    QLLGAVAGAA LLHELTPPDI RG........ .......... ..........
mip_rat       QLLGAVAGAA VLYSVTPPAV RGNLALNTLH AGVSVGQATT VEIFLTLQFV
aqp2_oryaf    QLLGAVAGAA LLHELTPPDI RG........ .......... ..........
mip_mouse     QLLGAVAGAA VLYSVTPPAV RGNLALNTLH TGVSVGQATT VEIFLTLQFV
mip_ranpi     QLLGAVAGAA VLYGVTPAAI RGNLALNTLH PGVSLGQATT VEIFLTLQFV
mip_bovin     QLLGAVAGAA VLYSVTPPAV RGNLALNTLH PGVSVGQATI VEIFLTLQFV
mip_human     QLLGAVAGAA VLYSVTPPAV RGNLALNTLH PAVSVGQATT VEIFLTLQFV
mip_chick     .......... .......... .......... .......... ..........
aqp5_rat      QLVGAIAGAG ILYWLAPLNA RGNLAVNALN NNTTPGKAMV VELILTFQLA
aqp5_human    QLVGAIAGAG ILYGVAPLNA RGNLAVNALN NNTTQGQAMV VELILTFQLA
aqp2_human    QLLGAVAGAA LLHEITPADI RGDLAVNALS NSTTAGQAVT VELFLTLQLV
aqp4_human    QCLGAIIGAG ILYLVTPPSV VGGLGVTMVH GNLTAGHGLL VELIITFQLV
aqp4_rat      QCLGAIIGAG ILYLVTPPSV VGGLGVTTVH GNLTAGHGLL VELIITFQLV
aqp4_mouse    QCLGAIIGAG ILYLVTPPSV VGGLGVTTVH GNLTAGHGLL VELIITFQLV
aqp2_rat      QLLGAVAGAA ILHEITPVEI RGDLAVNALH NNATAGQAVT VELFLTMQLV
aqp2_mouse    QLLGAVAGAA ILHEITPVEI RGDLAVNALH NNATAGQAVT VELFLTMQLV
wc2a_arath    QCLGAICGVG FVKAFQSSYY TRYGGgnSLA DGYSTGTGLA AEIIGTFVLV
aqp6_human    QLVGATVGAA LLYGVMPGDI RETLGINVVR NSVSTGQAVA VELLLTLQLV
wc2c_arath    QCLGAICGVG FVKAFQSSHY VNYGGgnFLA DGYNTGTGLA AEIIGTFVLV
wc2b_arath    QCLGAICGVG FRQSFQSSYY DRYGGgnSLA DGYNTGTGLA AEIIGTFVLV
wc1c_arath    QCLGAICGAG VVKGFQPNPY QtgGGANTVA HGYTKGSGLG AEIIGTFVLV
wc1b_arath    QCLGAICGAG VVKGFQPKQY QagGGANTIA HGYTKGSGLG AEIIGTFVLV
tipw_lyces    QCLGAICGAG VVKGFMVGPY QrgGGANVVN PGYTKGDGLG AEIIGTFVLV
wc1a_arath    QCLGAICGAG VVKGFQPKQY QagGGANTVA HGYTKGSGLG AEIIGTFVLV
tipw_pea      QVLGAICGAG VVKGFEGKQR FGDLNgnFVA PGYTKGDGLG AEIVGTFILV
tipa_arath    QLLGAILACL LLRLTTNGMR PVGFR...LA SGVGAVNGLV LEIILTFGLV
aqua_atrca    QCAGAICGVG LVKAFMKGPY NqgGGANSVA LGYNKGTAFG AELIGTFVLV
dip_antma     QCLGSTVACL LLKFVTNGL. ..SVPTHGVA AGMDAIQGVV MEIIITFALV
aqpz_ecoli    QVVGGIVAAA LLYLIASGKT GFDAAASGFA sgYSMLSALV VELVLSAGFL
tip2_tobac    QLLGSTVACL LLKYVTNGL. ..AVPTHGVA AGLNGFQGVV MEIIITFALV
tip1_tobac    QLLGSTVACL LLKYVTNGL. ..AVPTHGVA AGLNGLQGVV MEIIITFALV
tipg_arath    QLLGSVVACL ILKFATGGLA VPAFG...LS AGVGVLNAFV FEIVMTFGLV
bib_drome     QCGGGIAGAA LLYGVTVPGY QGNLQAasHS AALAAWERFG VEFILTSLVV
tipr_arath    QLLGSVVACL ILKFATGGLA VPPFG...LS AGVGVLNAFV FEIVMTFGLV
tipa_phavu    QLLGSIVAAL VLRLVTNNMR PSGF...HVS PGVGVGHMFI LEVVMTFGLM
tipg_orysa    QLLGSTVACF LLRFSTGGLA TGTFGL.... TGVSVWEALV LEIVMTFGLV

              151                                                200
predict_h258  LCVLATTDRR RRDLGGSAPL AIGLSVALGH LLAIDYTGCG INPARSFGSA
aqp1_rat      LCVLATTDRR RRDLGGSAPL AIGLSVALGH LLAIDYTGCG INPARSFGSA
aqp1_mouse    LCVLATTDRR RRDLGGSAPL AIGLSVALGH LLAIDYTGCG INPARSFGSA
aqp1_human    LCVLATTDRR RRDLGGSAPL AIGLSVALGH LLAIDYTGCG INPARSFGSA
aqp1_bovin    LCVLATTDRR RRDLGGSGPL AIGFSVALGH LLAIDYTGCG INPARSFGSS
aqp1_sheep    LCVLATTDRR RrdLGDSGPL AIGFSVALGH LLAIDYTGCG INPARSFGSS
aqpa_ranes    LCVVAVTDRR RHDVSGSVPL AIGLSVALGH LIAIDYTGCG MNPARSFGSA
aqp2_dasno    .......... .......... .......... .......... ..........
aqp2_bovin    .......... .......... .......... .......... ..........
aqp2_canfa    .......... .......... .......... .......... ..........
aqp2_rabit    .......... .......... .......... .......... ..........
aqp2_elema    .......... .......... .......... .......... ..........
aqp2_horse    .......... .......... .......... .......... ..........
aqp2_proha    .......... .......... .......... .......... ..........
mip_rat       LCIFATYDER RNGRMGSVAL AVGFSLTLGH LFGMYYTGAG MNPARSFAPA
aqp2_oryaf    .......... .......... .......... .......... ..........
mip_mouse     LCIFATYDER RNGRMGSVAL AVGFSLTLGH LFGMYYTGAG MNPARSFAPA
mip_ranpi     LCIFATYDER RNGRLGSVSL AIGFSLTLGH LFGLYYTGAS MNPARSFAPA
mip_bovin     LCIFATYDER RNGRLGSVAL AVGFSLTLGH LFGMYYTGAG MNPARSFAPA
mip_human     LCIFATYDER RNGQLGSVAL AVGFSLALGH LFGMYYTGAG MNPARSFAPA
mip_chick     ........DR HDGRPGSAAL PVGFSLALGH LFGIPFTGAG MNPARSFAPA
aqp5_rat      LCIFSSTDSR RTSPVGSPAL SIGLSVTLGH LVGIYFTGCS MNPARSFGPA
aqp5_human    LCIFASTDSR RTSPVGSPAL SIGLSVTLGH LVGIYFTGCS MNPARSFGPA
aqp2_human    LCIFASTDER RGENPGTPAL SIGFSVALGH LLGIHYTGCS MNPARSLAPA
aqp4_human    FTIFASCDSK RTDVTGSIAL AIGFSVAIGH LFAINYTGAS MNPARSFGPA
aqp4_rat      FTIFASCDSK RTDVTGSVAL AIGFSVAIGH LFAINYTGAS MNPARSFGPA
aqp4_mouse    FTVFASCDSK RTDVTGSIAL AIGFSVAIGH LFAINYTGAS MNPARSFGPA
aqp2_rat      LCIFASTDER RGDNLGSPAL SIGFSVTLGH LLGIYFTGCS MNPARSLAPA
aqp2_mouse    LCIFASTDER RSDNLGSPAL SIGFSVTLGH LLGIYFTGCS MNPARSLAPA
wc2a_arath    YTVFSATDPK RSavPVLAPL PIGFAVFMVH LATIPITGTG INPARSFGAA
aqp6_human    LCVFASTDSR QTS..GSPAT MIGISWALGH LIGILFTGCS MNPARSFGPA
wc2c_arath    YTVFSATDPK RNavPVLAPL PIGFAVFMVH LATIPITGTG INPARSFGAA
wc2b_arath    YTVFSATDPK RNavPVLAPL PIGFAVFMVH LATIPITGTG INPARSFGAS
wc1c_arath    YTVFSATDAK RSavPILAPL PIGFAVFLVH LATIPITGTG INPARSLGAA
wc1b_arath    YTVFSATDAK RNavPILAPL PIGFAVFLVH LATIPITGTG INPARSLGAA
tipw_lyces    YTVFSATDAK RNavPILAPL PIGFAVFLVH LATIPITGTG INPARSLGAA
wc1a_arath    YTVFSATDAK RNavPILAPL PIGFAVFLVH LATIPITATG INPARSLGAA
tipw_pea      YTVFSATDAK RSavPILAPL PIGFAVFLVH LATIPITGTG INPARSLGAA
tipa_arath    YVVYStiDPK RGSLGIIAPL AIGLIVGANI LVGGPFSGAS MNPARAFGPA
aqua_atrca    YTVFSATDPK RSavPILAPL PIGFAVFMVH LATIPITGTG INPARSFGAA
dip_antma     YTVYAtaDPK KGSLGVIAPI AIGFIVGANI LAAGPFSGGS MNPARSFGPA
aqpz_ecoli    LVIHGATDKF APA..GFAPI AIGLALTLIH LISIPVTNTS VNPARSTAVA
tip2_tobac    YTVYAtaDPK KGSLGTIAPI AIGFIVGANI LAAGPFSGGS MNPARSFGPA
tip1_tobac    YTVYAtaDPK KGSLGTIAPI AIGFIVGANI LAAGPFSGGS MNPARSFGPA
tipg_arath    YTVYAtiDPK NGSLGTIAPI AIGFIVGANI LAGGAFSGAS MNPAVAFGPA
bib_drome     LCYFVSTDPM KKFMGNS.AA SIGCAYSACC FVSMPYLN.. ..PARSLGPS
tipr_arath    YTVYAtiDPK NGSLGTIAPI AIGFIVGANI LAGGAFSGAS MNPAVAFGPA
tipa_phavu    YTVYGtiDPK RGAVSYIAPL AIGLIVGANI LVGGPFDGAC MNPALAFGPS
tipg_orysa    YTVYAtvDPK KGSLGTIAPI AIGFIVGANI LVGGAFDGAS MNPAVSFGPA

              201                                                250
predict_h258  VLTRNFSNHW IFWVGPFIGS ALAVLIYDFI LAPRSSDFTD RMKVWTSGQV
aqp1_rat      VLTRNFSNHW IFWVGPFIGS ALAVLIYDFI LAPRSSDFTD RMKVWTSGQV
aqp1_mouse    VLTRNFSNHW IFWVGPFIGG ALAVLIYDFI LAPRSSDFTD RMKVWTSGQV
aqp1_human    VITHNFSNHW IFWVGPFIGG ALAVLIYDFI LAPRSSDLTD RVKVWTSGQV
aqp1_bovin    VITHNFQDHW IFWVGPFIGA ALAVLIYDFI LAPRSSDLTD RVKVWTSGQV
aqp1_sheep    VITHNFQDHW IFWVGPFIGA ALAVLIYDFI LAPRSSDLTD RVKVWTSGQV
aqpa_ranes    VLTKNFTYHW IFWVGPMIGG AAAAIIYDFI LAPRTSDLTD RMKVWTNGQV
aqp2_dasno    .......... .......... .......... .......... ..........
aqp2_bovin    .......... .......... .......... .......... ..........
aqp2_canfa    .......... .......... .......... .......... ..........
aqp2_rabit    .......... .......... .......... .......... ..........
aqp2_elema    .......... .......... .......... .......... ..........
aqp2_horse    .......... .......... .......... .......... ..........
aqp2_proha    .......... .......... .......... .......... ..........
mip_rat       ILTRNFSNHW VYWVGPIIGG GLGSLLYDFL LFPRLKSVSE RLSILKGARP
aqp2_oryaf    .......... .......... .......... .......... ..........
mip_mouse     ILTRNFSNHW VYWVGPIIGG GLGSLLYDFL LFPRLKSVSE RLSILKGARP
mip_ranpi     VLTRNFTNHW VYWVGPIIGG ALGGLVYDFI LFPRMRGLSE RLSILKGARP
mip_bovin     ILTRNFTNHW VYWVGPVIGA GLGSLLYDFL LFPRLKSVSE RLSILKGSRP
mip_human     ILTGNFTNHW VYWVGPIIGG GLGSLLYDFL LFPRLKSISE RLSVLKGAKP
mip_chick     VITRNFTNHW VFWAGPLLGA ALAALLYELA LCPRARSMAE RLAV.LRGEP
aqp5_rat      VVMNRFssHW VFWVGPIVGA MLAAILYFYL LFPSSLSLHD RVAVVKGTYE
aqp5_human    VVMNRFsaHW VFWVGPIVGA VLAAILYFYL LFPNSLSLSE RVAIIKGTYE
aqp2_human    VVTGKFDDHW VFWIGPLVGA ILGSLLYNYV LFPPAKSLSE RLAVLKGLEp
aqp4_human    VIMGNWENHW IYWVGPIIGA VLAGGLYEYV FCPDVEFKRR FKEAFSKaqT
aqp4_rat      VIMGNWENHW IYWVGPIIGA VLAGALYEYV FCPDVELKRR LKEAFSKaqT
aqp4_mouse    VIMGNWANHW IYWVGPIMGA VLAGALYEYV FCPDVELKRR LKEAFSKaqT
aqp2_rat      VVTGKFDDHW VFWIGPLVGA IIGSLLYNYL LFPSAKSLQE RLAVLKGLEp
aqp2_mouse    VVTGKFDDHW VFWIGPLVGA IIGSLLYNYL LFPSTKSLQE RLAVLKGLEp
wc2a_arath    VIYnpWDDHW IFWVGPFIGA AIAAFYHQFV LRASGSKSLG SFRSAANV..
aqp6_human    IIIGKFTVHW VFWVGPLMGA LLASLIYNFV LFPDTKTLAQ RLAILTGTVE
wc2c_arath    VIFnpWDDHW IFWVGPFIGA TIAAFYHQFV LRASGSKSLG SFRSAANV..
wc2b_arath    VIYnpWDDHW IFWVGPFIGA AIAAFYHQFV LRASGSKSLG SFRSAANV..
wc1c_arath    IIYnaWDDHW IFWVGPFIGA ALAALYHQLV IRAIPFKSRS ..........
wc1b_arath    IIFnaWDDHW VFWVGPFIGA ALAALYHVIV IRAIPFKSRS ..........
tipw_lyces    IIYnaWNDHW IFWVGPMIGA ALAAIYHQII IRAMPFHRS. ..........
wc1a_arath    IIYnsWDDHW VFWVGPFIGA ALAALYHVVV IRAIPFKSRS ..........
tipw_pea      IVFngWNDHW IFWVGPFIGA ALAALYHQVV IRAIPFKSK. ..........
tipa_arath    LVGWRWHDHW IYWVGPFIGS ALAALIYEYM VIPTEPPTHH AHGVHQPLAP
aqua_atrca    VIyrVWDDHW IFWVGPFVGA LAAAAYHQYV LRAAAIKALG SFRSNPTN..
dip_antma     VASGDFSQNW IYWAGPLIGG ALAGFIYGDV FITAHAPLPT SEDYA.....
aqpz_ecoli    IFQgaLEQLW FFWVVPIVGG IIGGLIYRTL LEKRD..... ..........
tip2_tobac    VVAGDFSQNW IYWAGPLIGG GLAGFIYGDV FIGCHTPLPT SEDYA.....
tip1_tobac    VVAGDFSQNW IYWAGPLIGG GLAGFIYGDV FIGCHTPLPT SEDYA.....
tipg_arath    VVSWTWTNHW VYWAGPLVGG GIAGLIYEVF FINTTHEQLP TTDY......
bib_drome     FVLNKWDSHW VYWFGPLVGG MASGLVYEYI FNSRNRNLRH NKGSIDNDSS
tipr_arath    VVSWTWTNHW VYWAGPLVGG GIAGLIYEVF FINTTHTSSS NHRLLN....
tipa_phavu    LVGWQWHQHW IFWVGPLLGA ALAALVYEYA VIPIEPPPHH HQPLATEDY.
tipg_orysa    LVSWSWESQW VYWVGPLIGG GLAGVIYEVL FISHTHEQLP TTDY......

              251              269
predict_h258  EEYDLDADDI NSRVEMKPK
aqp1_rat      EEYDLDADDI NSRVEMKPK
aqp1_mouse    EEYDLDADDI NSRVEMKPK
aqp1_human    EEYDLDADDI NSRVEMKPK
aqp1_bovin    EEYDLDADDI NSRVEMKPK
aqp1_sheep    EEYDLDADDI NSRVEMKPK
aqpa_ranes    EEYELDGDD. NTRVEMKPK
aqp2_dasno    .......... .........
aqp2_bovin    .......... .........
aqp2_canfa    .......... .........
aqp2_rabit    .......... .........
aqp2_elema    .......... .........
aqp2_horse    .......... .........
aqp2_proha    .......... .........
mip_rat       SDSNGQPEGT GEPVELKTQ
aqp2_oryaf    .......... .........
mip_mouse     SDSNGQPEGT GEPVELKTQ
mip_ranpi     AEPEGQQEAT GEPIELKTQ
mip_bovin     SESNGQPEVT GEPVELKTQ
mip_human     DVSNGQPEVT GEPVELNTQ
mip_chick     PAAAPPPEPP AEPLELKTQ
aqp5_rat      PEEDWEDHRE ERKKTIELT
aqp5_human    PDEDWEEQRE ERKKTMELT
aqp2_human    tDWEEREVRR RQSVELHSP
aqp4_human    KGSYMEVEDN RSQVETDDL
aqp4_rat      KGSYMEVEDN RSQVETEDL
aqp4_mouse    KGSYMEVEDN RSQVETEDL
aqp2_rat      tDWEEREVRR RQSVELHSP
aqp2_mouse    tDWEEREVRR RQSVELHSP
wc2a_arath    .......... .........
aqp6_human    VGTGARAGAE PLKKESQPG
wc2c_arath    .......... .........
wc2b_arath    .......... .........
wc1c_arath    .......... .........
wc1b_arath    .......... .........
tipw_lyces    .......... .........
wc1a_arath    .......... .........
tipw_pea      .......... .........
tipa_arath    EDY....... .........
aqua_atrca    .......... .........
dip_antma     .......... .........
aqpz_ecoli    .......... .........
tip2_tobac    .......... .........
tip1_tobac    .......... .........
tipg_arath    .......... .........
bib_drome     SIHSEDELNY DMDMEKPNK
tipr_arath    .......... .........
tipa_phavu    .......... .........
tipg_orysa    .......... .........


________________________________________________________________________________




   Prediction of:           

    - secondary structure,          by PHDsec       
    - solvent accessibility,        by PHDacc       
    - and helical transmembrane regions,    by PHDhtm       

   PHD: Profile fed neural network systems from HeiDelberg
   ~~~~~~~~~~~~~~~~~~~~~~~~~~~~~~~~~~~~~~~~~~~~~~~~~~~~~~~

   Author:             Burkhard Rost        
                       EMBL, Heidelberg, FRG
                       Meyerhofstrasse 1, 69 117 Heidelberg
                       Internet: Predict-Help@EMBL-Heidelberg.DE

   All rights reserved.





   ~~~~~~~~~~~~~~~~~~~~~~~~~~~~~~~~~~~~~~~~~    
   Secondary structure prediction by PHDsec:
   ~~~~~~~~~~~~~~~~~~~~~~~~~~~~~~~~~~~~~~~~~    

   Author:             Burkhard Rost        
                       EMBL, Heidelberg, FRG
                       Meyerhofstrasse 1, 69 117 Heidelberg
                       Internet: Rost@EMBL-Heidelberg.DE        

   All rights reserved.




About the network method
~~~~~~~~~~~~~~~~~~~~~~~

The network procedure is described in detail in:
1) Rost, Burkhard; Sander, Chris:
  Prediction of protein structure at better than 70% accuracy.
  J. Mol. Biol., 1993, 232, 584-599.            

A brief description is given in:
  Rost, Burkhard; Sander, Chris:
  Improved prediction of protein secondary structure by use of se-
  quence profiles and neural networks.
  Proc. Natl. Acad. Sci. U.S.A., 1993, 90, 7558-7562.           

The PHD mail server is described in:
2) Rost, Burkhard; Sander, Chris; Schneider, Reinhard:
  PHD - an automatic mail server for protein secondary structure
  prediction.
  CABIOS, 1994, 10, 53-60.

The latest improvement steps (up to 72%) are explained in:
3) Rost, Burkhard; Sander, Chris:
  Combining evolutionary information and neural networks to predict
  protein secondary structure.
  Proteins, 1994,  19, 55-72.

To be quoted for publications of PHD output:
  Papers 1-3 for the prediction of secondary structure and the pre-
  diction server.



About the input to the network
~~~~~~~~~~~~~~~~~~~~~~~~~~~~~~

The prediction is performed by a system of neural networks.
The input is a multiple sequence alignment. It is taken from an HSSP
file (produced by the program MaxHom:
  Sander, Chris & Schneider, Reinhard: Database of Homology-Derived
  Structures and the Structural Meaning of Sequence Alignment.
  Proteins, 1991, 9, 56-68.

For optimal results the alignment should contain sequences with varying
degrees of sequence similarity relative to the input protein.
The following is an ideal situation:

+-----------------+----------------------+
|   sequence:     |  sequence identity   |
+-----------------+----------------------+
| target sequence |  100 %               |
| aligned seq. 1  |   90 %               |
| aligned seq. 2  |   80 %               |
|      ...        |   ...                |
| aligned seq. 7  |   30 %               |
+-----------------+----------------------+



Estimated Accuracy of Prediction
~~~~~~~~~~~~~~~~~~~~~~~~~~~~~~~~

A careful cross validation test on some 250 protein chains (in total
about 55,000 residues) with less than 25% pairwise sequence identity
gave the following results:

++================++-----------------------------------------+
|| Qtotal = 72.1% ||      ("overall three state accuracy")   |
++================++-----------------------------------------+

+----------------------------+-----------------------------+
| Qhelix (% of observed)=70% | Qhelix (% of predicted)=77% |
| Qstrand(% of observed)=62% | Qstrand(% of predicted)=64% |
| Qloop  (% of observed)=79% | Qloop  (% of predicted)=72% |
+----------------------------+-----------------------------+
..........................................................................

These percentages are defined by:
~~~~~~~~~~~~~~~~~~~~~~~~~~~~~~~~

|                    number of correctly predicted residues
|Qtotal =            ---------------------------------------      (*100)
|                          number of all residues
|
|                    no of res correctly predicted to be in helix
|Qhelix (% of obs) = -------------------------------------------- (*100)
|                    no of all res observed to be in helix
|
|
|                    no of res correctly predicted to be in helix
|Qhelix (% of pred)= -------------------------------------------- (*100)
|                    no of all residues predicted to be in helix

..........................................................................

Averaging over single chains
~~~~~~~~~~~~~~~~~~~~~~~~~~~

The most reasonable way to compute the overall accuracies is the above
quoted percentage of correctly predicted residues.  However, since the
user is mainly interested in the expected performance of the prediction
for a particular protein, the mean value when averaging over protein
chains might be of help as well.  Computing first the three state
accuracy for each protein chain, and then averaging over 250 chains
yields the following average:

+-------------------------------====--+
| Qtotal/averaged over chains = 72.2% |
+-------------------------------====--+
| standard deviation          =  9.3% |
+-------------------------------------+

..........................................................................

Further measures of performance
~~~~~~~~~~~~~~~~~~~~~~~~~~~~~~

Matthews correlation coefficient:

+---------------------------------------------+
| Chelix = 0.63, Cstrand = 0.53, Cloop = 0.52 |
+---------------------------------------------+
..........................................................................

Average length of predicted secondary structure segments:

.           +------------+----------+
.           |  predicted | observed |
+-----------+------------+----------+
| Lhelix  = |    10.3    |    9.3   |
| Lstrand = |     5.0    |    5.3   |
| Lloop   = |     7.2    |    5.9   |
+-----------+------------+----------+
..........................................................................

The accuracy matrix in detail:

+---------------------------------------+
|    number of residues with H, E, L    |
+---------+------+------+------+--------+
|         |net H |net E |net L |sum obs |
+---------+------+------+------+--------+
| obs H   |12447 | 1255 | 3990 |  17692 |
| obs E   |  949 | 7493 | 3750 |  12192 |
| obs L   | 2604 | 2875 |19962 |  25441 |
+---------+------+------+------+--------+
| sum Net |16000 |11623 |27702 |  55325 |
+---------+------+------+------+--------+

Note: This table is to be read in the following manner:
     12447 of all residues predicted to be in helix, were observed to
     be in helix, 949 however belong to observed strands, 2604 to
     observed loop regions.  The term "observed" refers to the DSSP
     assignment of secondary structure calculated from 3D coordinates
     of experimentally determined structures (Dictionary of Secondary
     Structure  of Proteins: Kabsch & Sander (1983) Biopolymers, 22,
     2577-2637).



Position-specific reliability index
~~~~~~~~~~~~~~~~~~~~~~~~~~~~~~~~~~

The network predicts the three secondary structure types using real
numbers from the output units. The prediction is assigned by choosing
the maximal unit ("winner takes all").  However, the real numbers
contain additional information.
E.g. the difference between the maximal and the second largest output
unit can be used to derive a "reliability index".  This index is given
for each residue along with the prediction.  The index is scaled to
have values between 0 (lowest reliability), and 9 (highest).
The accuracies (Qtot) to be expected for residues with values above a
particular value of the index are given below as well as the fraction
of such residues (%res).:

+------+-----+-----+-----+-----+-----+-----+-----+-----+-----+-----+
| index|  0  |  1  |  2  |  3  |  4  |  5  |  6  |  7  |  8  |  9  |
| %res |100.0| 99.2| 90.4| 80.9| 71.6| 62.5| 52.8| 42.3| 29.8| 14.1|
+------+-----+-----+-----+-----+-----+-----+-----+-----+-----+-----+
|      |     |     |     |     |     |     |     |     |     |     |
| Qtot | 72.1| 72.3| 74.8| 77.7| 80.3| 82.9| 85.7| 88.5| 91.1| 94.2|
|      |     |     |     |     |     |     |     |     |     |     |
+------+-----+-----+-----+-----+-----+-----+-----+-----+-----+-----+
| H%obs| 70.4| 70.6| 73.7| 77.1| 80.1| 83.1| 86.0| 89.3| 92.5| 96.4|
| E%obs| 61.5| 61.7| 63.7| 66.6| 69.1| 71.7| 74.6| 77.0| 77.8| 68.1|
|      |     |     |     |     |     |     |     |     |     |     |
| H%prd| 77.8| 78.0| 80.0| 82.6| 84.7| 86.9| 89.2| 91.3| 93.1| 95.4|
| E%prd| 64.5| 64.7| 67.8| 71.0| 74.2| 77.6| 81.4| 85.1| 89.8| 93.5|
+------+-----+-----+-----+-----+-----+-----+-----+-----+-----+-----+

The above table gives the cumulative results, e.g. 62.5% of all
residues have a reliability of at least 5.  The overall three-state
accuracy for this subset of almost two thirds of all residues is 82.9%.
For this subset, e.g., 83.1% of the observed helices are correctly
predicted, and 86.9% of all residues predicted to be in helix are
correct.

..........................................................................

The following table gives the non-cumulative quantities, i.e. the
values per reliability index range.  These numbers answer the question:
how reliable is the prediction for all residues labeled with the
particular index i.

+------+-----+-----+-----+-----+-----+-----+-----+-----+-----+
| index|  1  |  2  |  3  |  4  |  5  |  6  |  7  |  8  |  9  |
| %res |  8.8|  9.5|  9.3|  9.1|  9.7| 10.5| 12.5| 15.7| 14.1|
+------+-----+-----+-----+-----+-----+-----+-----+-----+-----+
|      |     |     |     |     |     |     |     |     |     |
| Qtot | 46.6| 50.6| 57.7| 62.6| 67.9| 74.2| 82.2| 88.3| 94.2|
|      |     |     |     |     |     |     |     |     |     |
+------+-----+-----+-----+-----+-----+-----+-----+-----+-----+
| H%obs| 36.8| 42.3| 49.5| 55.2| 61.7| 69.9| 78.8| 87.4| 96.4|
| E%obs| 44.7| 44.5| 52.1| 55.4| 60.9| 68.0| 75.9| 81.0| 68.1|
|      |     |     |     |     |     |     |     |     |     |
| H%prd| 49.9| 52.5| 60.3| 64.2| 69.2| 77.5| 85.4| 89.9| 95.4|
| E%prd| 41.7| 47.1| 53.6| 57.0| 64.0| 71.6| 78.8| 88.8| 93.5|
+------+-----+-----+-----+-----+-----+-----+-----+-----+-----+

For example, for residues with Relindex = 5 64% of all predicted betha-
strand residues are correctly identified.





   ~~~~~~~~~~~~~~~~~~~~~~~~~~~~~~~~~~~~~~~~~~~      
   Solvent accessibility prediction by PHDacc:
   ~~~~~~~~~~~~~~~~~~~~~~~~~~~~~~~~~~~~~~~~~~~      

   Author:             Burkhard Rost        
                       EMBL, Heidelberg, FRG
                       Meyerhofstrasse 1, 69 117 Heidelberg
                       Internet: Rost@EMBL-Heidelberg.DE        

   All rights reserved.




About the network method
~~~~~~~~~~~~~~~~~~~~~~~

The network for prediction of secondary structure is described in
detail in:
  Rost, Burkhard; Sander, Chris:
  Prediction of protein structure at better than 70% accuracy.
  J. Mol. Biol., 1993, 232, 584-599.

The analysis of the prediction of solvent exposure is given in:
  Rost, Burkhard; Sander, Chris:
  Conservation and prediction of solvent accessibility in protein
  families.  Proteins, 1994, 20, 216-226.

To be quoted for publications of PHD exposure prediction:
  Both papers quoted above.



Definition of accessibility
~~~~~~~~~~~~~~~~~~~~~~~~~~

For training the residue solvent accessibility the DSSP (Dictionary of
Secondary Structure of Proteins; Kabsch & Sander (1983) Biopolymers, 22,
2577-2637) values of accessible surface area have been used.  The
prediction provides values for the relative solvent accessibility.  The
normalisation is the following:

|                           ACCESSIBILITY (from DSSP in Angstrom)
|RELATIVE_ACCESSIBILITY =   ------------------------------------- * 100
|                               MAXIMAL_ACC (amino acid type i)

where MAXIMAL_ACC (i) is the maximal accessibility of amino acid type i.
The maximal values are:

+----+----+----+----+----+----+----+----+----+----+----+----+
|  A |  B |  C |  D |  E |  F |  G |  H |  I |  K |  L |  M |
| 106| 160| 135| 163| 194| 197|  84| 184| 169| 205| 164| 188|
+----+----+----+----+----+----+----+----+----+----+----+----+
|  N |  P |  Q |  R |  S |  T |  V |  W |  X |  Y |  Z |
| 157| 136| 198| 248| 130| 142| 142| 227| 180| 222| 196|
+----+----+----+----+----+----+----+----+----+----+----+

Notation: one letter code for amino acid, B stands for D or N; Z stands
  for E or Q; and X stands for undetermined.

The relative solvent accessibility can be used to estimate the number
of water molecules (W) in contact with the residue:

W = ACCESSIBILITY /10

The prediction is given in 10 states for relative accessibility, with

RELATIVE_ACCESSIBILITY = (PREDICTED_ACC * PREDICTED_ACC)

where PREDICTED_ACC = 0 - 9.



Estimated Accuracy of Prediction
~~~~~~~~~~~~~~~~~~~~~~~~~~~~~~~~

A careful cross validation test on some 238 protein chains (in total
about 62,000 residues) with less than 25% pairwise sequence identity
gave the following results:


Correlation
...........

The correlation between observed and predicted solvent accessibility
is:

-----------
corr = 0.53
-----------

This value ought to be compared to the worst and best case prediction
scenario: random prediction (corr = 0.0) and homology modelling
(corr = 0.66).  (Note: homology modelling yields a relative accurate
prediction in 3D if, and only if, a significantly identical sequence
has a known 3D structure.)


3-state accuracy
................

Often the relative accessibility is projected onto, e.g., 3 states:
  b  = buried       (here defined as < 9% relative accessibility),
  i  = intermediate ( 9% <= rel. acc. < 36% ),
  e  = exposed      ( rel. acc. >= 36% ).

A projection onto 3 states or 2 states (buried/exposed) enables the
compilation of a 3- and 2-state prediction accuracy.  PHD reaches an
overall 3-state accuracy of:
  Q3 = 57.5%
(compared to 35% for random prediction and 70% for homology modelling).

In detail:

+-----------------------------------+-------------------------+
| Qburied       (% of observed)=77% | Qb (% of predicted)=60% |
| Qintermediate (% of observed)= 9% | Qi (% of predicted)=44% |
| Qexposed      (% of observed)=78% | Qe (% of predicted)=56% |
+-----------------------------------+-------------------------+


10-state accuracy
.................

The network predicts relative solvent accessibility in 10 states, with
state i (i = 0-9) corresponding to a relative solvent accessibility of
i*i %.  The 10-state accuracy of the network is:

  Q10 = 24.5%

..........................................................................

These percentages are defined by:
~~~~~~~~~~~~~~~~~~~~~~~~~~~~~~~~

|                     number of correctly predicted residues
|Q3               = ---------------------------------------      (*100)
|                           number of all residues
|
|                     no of res. correctly predicted to be buried
|Qburied (% of obs) = ------------------------------------------- (*100)
|                     no of all res. observed to be buried
|
|
|                     no of res. correctly predicted to be buried
|Qburied (% of pred)= ------------------------------------------- (*100)
|                     no of all residues predicted to be buried

..........................................................................

Averaging over single chains
~~~~~~~~~~~~~~~~~~~~~~~~~~~

The most reasonable way to compute the overall accuracies is the above
quoted percentage of correctly predicted residues.  However, since the
user is mainly interested in the expected performance of the prediction
for a particular protein, the mean value when averaging over protein
chains might be of help as well.  Computing first the correlation
between observed and predicted accessibility for each protein chan, and
then averaging over all 238 chains yields the following average:

+-------------------------------====--+
| corr/averaged over chains   = 0.53  |
+-------------------------------====--+
| standard deviation          = 0.11  |
+-------------------------------------+

..........................................................................

Further details of performance accuracy
~~~~~~~~~~~~~~~~~~~~~~~~~~~~~~~~~~~~~~

The accuracy matrix in detail:
..............................

-------+----------------------------------------------------+-----------
\ PHD |    0    1   2   3    4    5     6     7    8    9  |  SUM  %obs
-------+----------------------------------------------------+-----------
OBS  0 | 8611  140   8  44   82  169   772   334   27    0  | 10187 16.6
OBS  1 | 4367  164   0  50  106  231   738   346   44    3  |  6049  9.8
OBS  2 | 3194  168   1  68  125  303   951   513   42    7  |  5372  8.7
OBS  3 | 2760  159   8  80  136  327  1246   746   58   19  |  5539  9.0
OBS  4 | 2312  144   2  72  166  396  1615  1245  124   19  |  6095  9.9
OBS  5 | 1873   96   3  84  138  425  1979  1834  187   27  |  6646 10.8
OBS  6 | 1387   67   1  60   80  278  2237  2627  231   51  |  7019 11.4
OBS  7 | 1082   35   0  32   56  225  1871  3107  302   60  |  6770 11.0
OBS  8 |  660   25   0  27   43  136  1206  2374  325   87  |  4883  7.9
OBS  9 |  325   20   2  27   29   74   648  1159  366  214  |  2864  4.7
-------+----------------------------------------------------+-----------
SUM    |26571 1018  25 544  961 2564 13263 14285 1706  487  |
%pred  | 43.3  1.7 0.0 0.9  1.6  4.2  21.6  23.3  2.8  0.8  |
-------+----------------------------------------------------+-----------

Note: This table is to be read in the following manner:
     8611 of all residues predicted to be in exposed by 0%, were
     observed with 0% relative accessibility.  However, 325 of all
     residues predicted to have 0% are observed as completely exposed
     (obs = 9 -> rel. acc. >= 81%).  The term "observed" refers to the
     DSSP compilation of area of solvent accessibility calculated from
     3D coordinates of experimentally determined structures (Diction-
     ary of Secondary Structure  of Proteins: Kabsch & Sander (1983)
     Biopolymers, 22, 2577-2637).


Accuracy for each amino acid:
.............................

+---+------------------------------+-----+-------+------+
|AA |   Q3 b%o b%p i%o i%p e%o e%p | Q10 |  corr |    N |
+---+------------------------------+-----+-------+------+
| A | 59.0  87  60   2  38  66  57 |  31 | 0.530 | 5054 |
| C | 62.0  91  67   5  39  25  21 |  34 | 0.244 |  893 |
| D | 56.5  21  45   6  49  94  57 |  20 | 0.321 | 3536 |
| E | 60.8   9  40   3  41  98  61 |  21 | 0.347 | 3743 |
| F | 63.3  94  67   9  46  29  37 |  27 | 0.366 | 2436 |
| G | 52.1  75  51   1  31  67  53 |  22 | 0.405 | 4787 |
| H | 50.9  63  53  23  45  71  50 |  18 | 0.442 | 1366 |
| I | 64.9  95  68   6  41  30  38 |  34 | 0.360 | 3437 |
| K | 66.6   2  11   2  37  98  67 |  23 | 0.267 | 3652 |
| L | 61.6  93  65   8  44  31  40 |  31 | 0.368 | 5016 |
| M | 60.1  92  64   5  39  45  44 |  29 | 0.452 | 1371 |
| N | 55.5  45  45   8  38  87  59 |  17 | 0.410 | 2923 |
| P | 53.0  48  48   9  39  83  56 |  18 | 0.364 | 2920 |
| Q | 54.3  27  44   7  44  92  56 |  20 | 0.344 | 2225 |
| R | 49.9  15  47  36  47  76  51 |  18 | 0.372 | 2765 |
| S | 55.6  69  53   3  51  81  56 |  22 | 0.464 | 3981 |
| T | 51.8  61  51   8  38  78  53 |  21 | 0.432 | 3740 |
| V | 61.1  93  65   5  40  39  42 |  34 | 0.418 | 4156 |
| W | 56.2  85  62  20  49  29  27 |  21 | 0.318 |  891 |
| Y | 49.7  73  52  33  49  36  38 |  19 | 0.359 | 2301 |
+---+------------------------------+-----+-------+------+

Abbreviations:

AA:   amino acid in one-letter code
b%o, i%o, e%o:   = Qburied, Qintermediate, Qexposed (% of observed),
     i.e. percentage of correct prediction in each state, see above
b%p, i%p, e%p:   = Qburied, Qintermediate, Qexposed (% of predicted),
     i.e. probability of correct prediction in each state, see above
b%o:  = Qburied (% of observed), see above
Q10:  percentage of correctly predicted residues in each of the 10
     states of predicted relative accessibility.
corr: correlation between predicted and observed rel. acc.
N:    number of residues in data set


Accuracy for different secondary structure:
...........................................

+--------+------------------------------+----+-------+-------+
| type   |   Q3 b%o b%p i%o i%p e%o e%p |Q10 |  corr |     N |
+--------+------------------------------+----+-------+-------+
| helix  | 59.5  79  64   8  44  80  56 | 27 | 0.574 | 20100 |
| strand | 61.3  84  73   9  46  69  37 | 35 | 0.524 | 13356 |
| loop   | 54.4  64  43  11  44  78  61 | 18 | 0.442 | 27968 |
+--------+------------------------------+----+-------+-------+

Abbreviations as before.



Position-specific reliability index
~~~~~~~~~~~~~~~~~~~~~~~~~~~~~~~~~~

The network predicts the 10 states for relative accessibility using real
numbers from the output units. The prediction is assigned by choosing
the maximal unit ("winner takes all").  However, the real numbers
contain additional information.
E.g. the difference between the maximal and the second largest output
unit (with the constraint that the second largest output is compiled
among all units at least 2 positions off the maximal unit) can be used
to derive a "reliability index".  This index is given for each residue
along with the prediction.  The index is scaled to have values between
0 (lowest reliability), and 9 (highest).
The accuracies (Q3, corr, asf.) to be expected for residues with values
above a particular value of the index are given below as well as the
fraction of such residues (%res).:

+---+------------------------------+----+-------+-------+
|RI |   Q3 b%o b%p i%o i%p e%o e%p |Q10 |  corr |  %res |
+---+------------------------------+----+-------+-------+
| 0 | 57.5  77  60   9  44  78  56 | 24 | 0.535 | 100.0 |
| 1 | 59.1  76  63   9  45  82  57 | 25 | 0.560 |  91.2 |
| 2 | 61.7  79  66   4  47  87  58 | 27 | 0.594 |  77.1 |
| 3 | 66.6  87  70   1  51  89  63 | 30 | 0.650 |  57.1 |
| 4 | 70.0  89  72   0  83  91  67 | 32 | 0.686 |  45.8 |
| 5 | 72.9  92  75   0   0  93  70 | 34 | 0.722 |  35.6 |
| 6 | 76.3  95  77   0   0  93  75 | 36 | 0.769 |  24.7 |
| 7 | 79.0  97  79   0   0  93  78 | 39 | 0.803 |  16.0 |
| 8 | 80.9  98  80   0   0  91  81 | 43 | 0.824 |   9.6 |
| 9 | 81.2  99  80   0   0  88  83 | 45 | 0.828 |   5.9 |
+---+------------------------------+----+-------+-------+

Abbreviations as before.

The above table gives the cumulative results, e.g. 45.8% of all
residues have a reliability of at least 4.  The correlation for this
most reliably predicted half of the residues is 0.686, i.e. a value
comparable to what could be expected if homology modelling were
possible.  For this subset of 45.8% of all residues, 89% of the buried
residues are correctly predicted, and 72% of all residues predicted to
be buried are correct.

..........................................................................

The following table gives the non-cumulative quantities, i.e. the
values per reliability index range.  These numbers answer the question:
how reliable is the prediction for all residues labeled with the
particular index i.

+---+------------------------------+----+-------+-------+
|RI |   Q3 b%o b%p i%o i%p e%o e%p |Q10 |  corr |  %res |
+---+------------------------------+----+-------+-------+
| 0 | 40.9  79  40  16  41  21  40 | 14 | 0.175 |   8.8 |
| 1 | 45.4  61  46  28  44  48  44 | 17 | 0.278 |  14.1 |
| 2 | 47.4  53  52  10  46  80  44 | 19 | 0.343 |  19.9 |
| 3 | 52.9  75  59   4  50  77  47 | 23 | 0.439 |  11.4 |
| 4 | 60.0  81  63   0  83  84  56 | 25 | 0.547 |  10.1 |
| 5 | 65.2  82  70   0   0  93  62 | 28 | 0.607 |  10.9 |
| 6 | 71.3  90  72   0   0  94  70 | 31 | 0.692 |   8.8 |
| 7 | 76.0  94  76   0   0  95  75 | 34 | 0.762 |   6.3 |
| 8 | 80.5  97  81   0   0  94  79 | 39 | 0.808 |   3.8 |
| 9 | 81.2  99  80   0   0  88  83 | 45 | 0.828 |   5.9 |
+---+------------------------------+----+-------+-------+

For example, for residues with RI = 4 83% of all predicted intermediate
residues are correctly predicted as such.





   ~~~~~~~~~~~~~~~~~~~~~~~~~~~~~~~~~~~~~~~~~~~~~~~~~~~~~~~
   Prediction of helical transmembrane segments by PHDhtm:
   ~~~~~~~~~~~~~~~~~~~~~~~~~~~~~~~~~~~~~~~~~~~~~~~~~~~~~~~

   Author:             Burkhard Rost        
                       EMBL, Heidelberg, FRG
                       Meyerhofstrasse 1, 69 117 Heidelberg
                       Internet: Rost@EMBL-Heidelberg.DE        

   All rights reserved.




About the network method
~~~~~~~~~~~~~~~~~~~~~~~

The PHD mail server is described in:
  Rost, Burkhard; Sander, Chris; Schneider, Reinhard:
  PHD - an automatic mail server for protein secondary structure
  prediction.
  CABIOS, 1994, 10, 53-60.      

To be quoted for publications of PHDhtm output:
  Rost, Burkhard; Casadio, Rita; Fariselli, Piero; Sander, Chris:
  Prediction of helical transmembrane segments at 95% accuracy.
  Protein Science, 1995, 4, 521-533.                



Estimated Accuracy of Prediction
~~~~~~~~~~~~~~~~~~~~~~~~~~~~~~~~

A cross validation test on 69 helical trans-membrane  proteins (in total
about 30,000 residues) with less than 25% pairwise sequence identity
gave the following results:

++================++-----------------------------------------+
|| Qtotal = 94.7% ||      ("overall two state accuracy")     |
++================++-----------------------------------------+

+----------------------------+-----------------------------+
| Qhelix (% of observed)=92% | Qhelix (% of predicted)=83% |
| Qloop  (% of observed)=96% | Qloop  (% of predicted)=97% |
+----------------------------+-----------------------------+

..........................................................................

These percentages are defined by:
~~~~~~~~~~~~~~~~~~~~~~~~~~~~~~~~

|                    number of correctly predicted residues
|Qtotal =            ---------------------------------------      (*100)
|                          number of all residues
|
|                    no of res correctly predicted to be in helix
|Qhelix (% of obs) = -------------------------------------------- (*100)
|                    no of all res observed to be in helix
|
|
|                    no of res correctly predicted to be in helix
|Qhelix (% of pred)= -------------------------------------------- (*100)
|                    no of all residues predicted to be in helix

..........................................................................

Further measures of performance
~~~~~~~~~~~~~~~~~~~~~~~~~~~~~~

Matthews correlation coefficient:

+---------------------------------------------+
| Chelix = 0.84, Cloop = 0.84                 |
+---------------------------------------------+
..........................................................................

Average length of predicted secondary structure segments:

|           +------------+----------+
|           |  predicted | observed |
+-----------+------------+----------+
| Lhelix  = |    24.6    |   22.2   |
+-----------+------------+----------+
..........................................................................

The accuracy matrix in detail:

+---------------------------------+
|    number of residues with H, L |
+---------+------+-------+--------+
|         |net H | net L |sum obs |
+---------+------+-------+--------+
| obs H   | 5214 |   492 |   5706 |
| obs L   | 1050 | 22423 |  23473 |
+---------+------+-------+--------+
| sum Net | 6264 | 22915 |  29179 |
+---------+------+-------+--------+

Note: This table is to be read in the following manner:
     5214 of all residues predicted to be in a helical trans-membrane
     region, were observed to be in the lipid bilayer, 1050 however
     were observed either inside or outside of the protein, i.e. in
     loop (or non-membrane) regions. The term "observed" refers to DSSP
     assignment of secondary structure calculated from 3D coordinates
     of experimentally determined structures (Dictionary of Secondary
     Structure  of Proteins: Kabsch & Sander (1983) Biopolymers, 22,
     2577-2637) where these were available.  For all other proteins,
     the assignment of trans-membrane segments has been taken from the
     Swissprot data bank (Bairoch, A.; Boeckmann, B.: The SWISS-PROT
     protein sequence data bank. Nucl. Acids Res. 20: 2019-2022, 1992).

..........................................................................

Overlap between predicted and observed segments:

+-----------------+---------------+----------------+
| segment overlap | % of observed | % of predicted |
|   Sov helix     |      95.6%    |      95.5%     |
|   Sov loop      |      83.6%    |      97.2%     |
+-----------------+---------------+----------------+
|   Sov total     |      86.0%    |      96.8%     |
+-----------------+---------------+----------------+

     Definition of Sov in: Rost et al., JMB, 1994, 235, 13-26.

     As helical trans-membrane segments are longer than globular heli-
     ces, correctly predicted segments can easily be made out.  PHDhtm
     misses 5 out of 258 observed segments, predicts 6 where non is
     observed and 3 times the predicted helical segment overlaps two
     observed regions.  Thus, in total more than 95% of all segments
     are correctly predicted.

..........................................................................

Entropy of prediction (information measure):

+-----------------+
| I = 0.64        |
+-----------------+

     (For comparison: homology modelling of globular proteins in three
     states: I=0.62.)
     Definition of Sov in: Rost et al., JMB, 1994, 235, 13-26.



Position-specific reliability index
~~~~~~~~~~~~~~~~~~~~~~~~~~~~~~~~~~

The network predicts two states: helical trans-membrane region and rest
using two output units.  The prediction is assigned by choosing the ma-
ximal unit ("winner takes all").  However, the real numbers of the out-
put units contain additional information.
E.g. the difference between the two output units can be used to derive
a "reliability index".  This index is given for each residue along with
the prediction.  The index is scaled to have values between 0 (lowest
reliability), and 9 (highest).
The accuracies (Qtot) to be expected for residues with values above a
particular value of the index are given below as well as the fraction
of such residues (%res).:

+------+-----+-----+-----+-----+-----+-----+-----+-----+-----+-----+
| index|  0  |  1  |  2  |  3  |  4  |  5  |  6  |  7  |  8  |  9  |
| %res |100.0| 98.8| 97.3| 95.9| 94.1| 92.3| 89.9| 86.2| 75.0| 66.8|
+------+-----+-----+-----+-----+-----+-----+-----+-----+-----+-----+
|      |     |     |     |     |     |     |     |     |     |     |
| Qtot | 94.7| 95.2| 95.6| 96.2| 96.7| 97.2| 97.7| 98.4| 99.4| 99.8|
|      |     |     |     |     |     |     |     |     |     |     |
+------+-----+-----+-----+-----+-----+-----+-----+-----+-----+-----+
| H%obs| 91.8| 92.9| 93.8| 94.4| 95.0| 95.7| 96.2| 96.8| 95.5| 78.7|
| L%obs| 95.3| 95.7| 96.1| 96.6| 97.0| 97.5| 98.1| 98.8| 99.7|100.0|
|      |     |     |     |     |     |     |     |     |     |     |
| H%prd| 82.7| 83.8| 85.0| 86.7| 88.1| 89.7| 91.4| 93.8| 96.3| 97.1|
| L%prd| 97.9| 98.3| 98.5| 98.7| 98.8| 99.0| 99.2| 99.4| 99.7| 99.9|
+------+-----+-----+-----+-----+-----+-----+-----+-----+-----+-----+

The above table gives the cumulative results, e.g. 92.3% of all
residues have a reliability of at least 5.  The overall two-state
accuracy for this subset is 97.2%.  For this subset, e.g., 95.7% of
the observed helical trans-membrane residues are correctly predicted,
and 89.7% of all residues predicted to be in helical trans-membrane
segment are correct.







The resulting network (PHD) prediction is:
~~~~~~~~~~~~~~~~~~~~~~~~~~~~~~~~~~~~~~~~~

________________________________________________________________________________



 PHD: Profile fed neural network systems from HeiDelberg
 ~~~~~~~~~~~~~~~~~~~~~~~~~~~~~~~~~~~~~~~~~~~~~~~~~~~~~~~

 Prediction of:         
    secondary structure,               by PHDsec        
    solvent accessibility,             by PHDacc        
    and helical transmembrane regions,     by PHDhtm        

 Author:                                    
    Burkhard Rost                           
    EMBL, 69012 Heidelberg, Germany                 
    Internet: Rost@EMBL-Heidelberg.DE               

 All rights reserved.



 The network systems are described in:                              

 PHDsec:    B Rost & C Sander: JMB, 1993, 232, 584-599.     
        B Rost & C Sander: Proteins, 1994, 19, 55-72.       
 PHDacc:    B Rost & C Sander: Proteins, 1994, 20, 216-226.     
 PHDhtm:    B Rost et al.:     Prot. Science, 1995, 4, 521-533. 



 Some statistics
 ~~~~~~~~~~~~~~~

 Percentage of amino acids:
 +--------------+--------+--------+--------+--------+--------+
 | AA:          |    L   |    A   |    S   |    G   |    I   |
 | % of AA:     |   13.0 |   10.0 |    9.7 |    8.9 |    8.6 |
 +--------------+--------+--------+--------+--------+--------+
 | AA:          |    V   |    R   |    T   |    F   |    D   |
 | % of AA:     |    7.8 |    5.2 |    4.5 |    4.5 |    4.5 |
 +--------------+--------+--------+--------+--------+--------+
 | AA:          |    N   |    Q   |    E   |    P   |    K   |
 | % of AA:     |    4.1 |    3.0 |    3.0 |    2.6 |    2.6 |
 +--------------+--------+--------+--------+--------+--------+
 | AA:          |    Y   |    M   |    W   |    H   |    C   |
 | % of AA:     |    1.9 |    1.9 |    1.5 |    1.5 |    1.5 |
 +--------------+--------+--------+--------+--------+--------+

 Percentage of secondary structure predicted:
 +--------------+--------+--------+--------+
 | SecStr:      |    H   |    E   |    L   |
 | % Predicted: |   43.9 |   16.7 |   39.4 |
 +--------------+--------+--------+--------+

 According to the following classes:
    all-alpha:   %H>45 and %E< 5; all-beta : %H<5 and %E>45
    alpha-beta : %H>30 and %E>20; mixed:    rest,
 this means that the predicted class is:           mixed class



 PHD output for your protein
 ~~~~~~~~~~~~~~~~~~~~~~~~~~~

 Tue Nov 24 17:44:57 1998
 Jury on:       10    different architectures (version   5.94_317 ).
 Note: differently trained architectures, i.e., different versions can
 result in different predictions.



 About the protein
 ~~~~~~~~~~~~~~~~~

 HEADER     /home/phd/server/work/predict_h25873-220
 COMPND
 SOURCE
 AUTHOR
 SEQLENGTH   269
 NCHAIN        1 chain(s) in predict_h25873-22040 data set
 NALIGN       48
 (=number of aligned sequences in HSSP file)



 Abbreviations: PHDsec
 ~~~~~~~~~~~~~~~~~~~~~

 sequence:
    AA : amino acid sequence
 secondary structure:
    HEL: H=helix, E=extended (sheet), blank=other (loop)
    PHD: Profile network prediction HeiDelberg
    Rel: Reliability index of prediction (0-9)
 detail:
    prH: 'probability' for assigning helix
    prE: 'probability' for assigning strand
    prL: 'probability' for assigning loop
         note: the 'probabilites' are scaled to the interval 0-9, e.g.,
               prH=5 means, that the first output node is 0.5-0.6
 subset:
    SUB: a subset of the prediction, for all residues with an expected
         average accuracy > 82% (tables in header)
         note: for this subset the following symbols are used:
      L: is loop (for which above " " is used)
    ".": means that no prediction is made for this residue, as the
         reliability is:  Rel < 5

 Abbreviations: PHDacc
 ~~~~~~~~~~~~~~~~~~~~~

    SS : secondary structure
    HEL: H=helix, E=extended (sheet), blank=other (loop)
 solvent accessibility:
    3st: relative solvent accessibility (acc) in 3 states:
         b = 0-9%, i = 9-36%, e = 36-100%.
    PHD: Profile network prediction HeiDelberg
    Rel: Reliability index of prediction (0-9)
    O_3: observed relative acc. in 3 states: B, I, E
         note: for convenience a blank is used intermediate (i).
    P_3: predicted relative accessibility in 3 states
    10st:relative accessibility in 10 states:
         = n corresponds to a relative acc. of n*n %
 subset:
    SUB: a subset of the prediction, for all residues with an expected
         average correlation > 0.69 (tables in header)
         note: for this subset the following symbols are used:
    "I": is intermediate (for which above " " is used)
    ".": means that no prediction is made for this residue, as the
         reliability is: Rel < 4


 Abbreviations: PHDhtm
 ~~~~~~~~~~~~~~~~~~~~~

 secondary structure:
    HL:  T=helical transmembrane region, blank=other (loop)
    PHD: Profile network prediction HeiDelberg
    PHDF:filtered prediction, i.e., too long transmembrane segments
         are split, too short ones are deleted
    Rel: Reliability index of prediction (0-9)
 detail:
    prH: 'probability' for assigning helical transmembrane region
    prL: 'probability' for assigning loop
         note: the 'probabilites' are scaled to the interval 0-9, e.g.,
               prH=5 means, that the first output node is 0.5-0.6
 subset:
    SUB: a subset of the prediction, for all residues with an expected
         average accuracy > 82% (tables in header)
         note: for this subset the following symbols are used:
      L: is loop (for which above " " is used)
    ".": means that no prediction is made for this residue, as the
         reliability is:  Rel < 5



 protein:       predict        length      269

                  ....,....1....,....2....,....3....,....4....,....5....,....6
         AA      |MASEIKKKLFWRAVVAEFLAMTLFVFISIGSALGFNYPLERNQTLVQDNVKVSLAFGLSI|
         PHD sec |       HHHHHHHHHHHHHHHHHHHHHHHHHHEE            HHHHHHHHHHHHH|
         Rel sec |998443148899999999999998997676530312469989998623353579999999|
 detail:
         prH sec |001223468899999999999998888777653112210000000145566788999999|
         prE sec |000011000000000000000001001111233542100000000000323211000000|
         prL sec |998665420100000000000000000011112244578988998753100000000000|
 subset: SUB sec |LLL.....HHHHHHHHHHHHHHHHHHHHHHH......LLLLLLLLL...H.HHHHHHHHH|

 ACCESSIBILITY
 3st:    P_3 acc |eeeebee bbb bbbbbbbbbbbbbbbbbbbbbebeee eeeeeeeeebbbbbbbbbbbb|
 10st:   PHD acc |997706650005000000000000000000000607775779776677000000000000|
         Rel acc |735421110541467608662789996343122133420454330023453975664547|
 subset: SUB acc |e.ee.....bb.bbbb.bbb.bbbbbb.b.......e..eee......bb.bbbbbbbbb|
                  ....,....7....,....8....,....9....,....10...,....11...,....12
         AA      |ATLAQSVGHISGAHSNPAVTLGLLLSCQISILRAVMYIIAQCVGAIVASAILSGITSSLL|
         PHD sec |HHHHHHHHHE      HHHHEHHHHHHHHHHHHHHHHHHHHHHHHHHHHHHHHHH     |
         Rel sec |999996412122653167703135552356779999999999999999999998467213|
 detail:
         prH sec |998986544334223477843456665567779999999999999999999998611343|
         prE sec |001001123420010000145432101221110000000000000000000000000000|
         prL sec |000001232245765521000000123210000000000000000000000000278555|
 subset: SUB sec |HHHHHH......LL..HHH....HHH..HHHHHHHHHHHHHHHHHHHHHHHHHH.LL...|

 ACCESSIBILITY
 3st:    P_3 acc |bbbbebbbebbbbbb bbbbbbbbbbbebbbbbbbbbbbbbbbbbbbbbbbbeebbeeeb|
 10st:   PHD acc |000060006000000500000000000600000000000000000000000067006760|
         Rel acc |456515321655013144869663400154551757478936465465467713401400|
 subset: SUB acc |bbbb.b...bbb....bbbbbbb.b...bbbb.bbbbbbb.bbbbbbbbbbb..b..e..|
                  ....,....13...,....14...,....15...,....16...,....17...,....18
         AA      |ENSLGRNDLARGVNSGQGLGIEIIGTLQLVLCVLATTDRRRRDLGGSAPLAIGLSVALGH|
         PHD sec |       HHH       EEEEEEEEEEEEEEEEEEE            E E   HHHHHH|
         Rel sec |359985212134223651899898866789799875436658889963211351457756|
 detail:
         prH sec |320002345432332111000000000000100000221120000000001113567767|
         prE sec |100000000000011014899888877789789886100000000013544222221111|
         prL sec |568986543466545763100000011100000112567768889975454564210111|
 subset: SUB sec |.LLLLL.........LL.EEEEEEEEEEEEEEEEEE..LLLLLLLLL.....L..HHHHH|

 ACCESSIBILITY
 3st:    P_3 acc |eeebbbebbbeebeebeebbbbbbbbbbbbbbbbbbbeeeeeeeebbbbbbbbbbbbbbb|
 10st:   PHD acc |677000600077076077000000000000000000077767767000000000000000|
         Rel acc |133100124043040233247198656399879530035414413123255869586654|
 subset: SUB acc |........b.e..e.....bb.bbbbb.bbbbbb....ee.ee......bbbbbbbbbbb|
                  ....,....19...,....20...,....21...,....22...,....23...,....24
         AA      |LLAIDYTGCGINPARSFGSAVLTRNFSNHWIFWVGPFIGSALAVLIYDFILAPRSSDFTD|
         PHD sec |HEEEE E          HHHEEEE    EEEEEE HHHHHHHHHHHHHEEEEE       |
         Rel sec |321341126989622145152653534229996251699999999973147525556642|
 detail:
         prH sec |521100000000145432463121122000000114789999999875421111121124|
         prE sec |244564431000000000015765121358997510000000000013467642110000|
         prL sec |233234457889754567411012655530002364200000000010010136667765|
 subset: SUB sec |........LLLLL....H.H.EE.L....EEEE.L.HHHHHHHHHHH...EE.LLLLL..|

 ACCESSIBILITY
 3st:    P_3 acc |bbbbebbbbbbebb bbbbbbbbeebeebbbbbbbbbbbbbbbbbbbbbbbbeeeee ee|
 10st:   PHD acc |000060000006005000000007606600000000000000000000000076777577|
         Rel acc |754424240102242141047612131118967874356346635751777031345044|
 subset: SUB acc |bbbb.b.b.....b..b..bbb.......bbbbbbb.bb.bbb.bbb.bbb....ee.ee|
                  ....,....25...,....26...,....27...,....28...,....29...,....30
         AA      |RMKVWTSGQVEEYDLDADDINSRVEMKPK|
         PHD sec |HHHHHH                       |
         Rel sec |66775259975467555457776422699|
 detail:
         prH sec |77887520012221222221111100000|
         prE sec |00000000000000000000001233200|
         prL sec |11112379987678777678887655799|
 subset: SUB sec |HHHHH.LLLLL.LLLLL.LLLLL...LLL|

 ACCESSIBILITY
 3st:    P_3 acc |ebebbeeeeeeeeeeeeeeeeeebeeeee|
 10st:   PHD acc |60700787677777677777767067789|
         Rel acc |10411563134335144444514212559|
 subset: SUB acc |..e..ee...e..e.eeeeee.e...eee|


 PHDhtm Helical transmembrane prediction
        note: PHDacc and PHDsec are reliable for water-
              soluble globular proteins, only.  Thus,
              please take the  predictions above with
              particular caution wherever transmembrane
              helices are predicted by PHDhtm!


 PHDhtm
---
--- PhdTopology REFINEMENT AND TOPOLOGY PREDICTION: SYMBOLS
--- AA           : amino acid in one-letter code
--- PHD htm      : HTM's predicted by the PHD neural network
---                system (T=HTM, ' '=not HTM)
--- Rel htm      : Reliability index of prediction (0-9, 0 is low)
--- detail       : Neural network output in detail
--- prH htm      : 'Probability' for assigning a helical trans-
---                membrane region (HTM)
--- prL htm      : 'Probability' for assigning a non-HTM region
---          note: 'Probabilites' are scaled to the interval
---                0-9, e.g., prH=5 means, that the first
---                output node is 0.5-0.6
--- subset       : Subset of more reliable predictions
--- SUB htm      : All residues for which the expected average
---                accuracy is > 82% (tables in header).
---          note: for this subset the following symbols are used:
---             L: is loop (for which above ' ' is used)
---           '.': means that no prediction is made for this,
---                residue as the reliability is:  Rel < 5
--- other        : predictions derived based on PHDhtm
--- PHDFhtm      : filtered prediction, i.e., too long HTM's are
---                split, too short ones are deleted
--- PHDRhtm      : refinement of neural network output
--- PHDThtm      : topology prediction based on refined model
---                symbols used:
---             i: intra-cytoplasmic
---             T: transmembrane region
---             o: extra-cytoplasmic
---
--- PhdTopology REFINEMENT AND TOPOLOGY PREDICTION
                  ....,....1....,....2....,....3....,....4....,....5....,....6
         AA      |MASEIKKKLFWRAVVAEFLAMTLFVFISIGSALGFNYPLERNQTLVQDNVKVSLAFGLSI|
         PHD htm |              TTTTTTTTTTTTTTTTTTT               TTTTTTTTTTTT|
 detail:         |                                                            |
         prH htm |000000000001136788999999999988875321110000000123678889999988|
         prL htm |999999999998863211000000000011124678889999999876321110000011|
  other:         |                                                            |
         PHDFhtm |              TTTTTTTTTTTTTTTTTTT                TTTTTTTTTTT|
         PHDRhtm |              TTTTTTTTTTTTTTTTTT                 TTTTTTTTTTT|
         PHDThtm |iiiiiiiiiiiiiiTTTTTTTTTTTTTTTTTToooooooooooooooooTTTTTTTTTTT|
 subset:         |                                                            |
         SUB htm |............................................................|
                  ....,....7....,....8....,....9....,....10...,....11...,....12
         AA      |ATLAQSVGHISGAHSNPAVTLGLLLSCQISILRAVMYIIAQCVGAIVASAILSGITSSLL|
         PHD htm |TTTTTTTTTTTTTTTTTTTTTTTTTTTTTTTTTTTTTTTTTTTTTTTTTTTTTTT     |
 detail:         |                                                            |
         prH htm |888888877777666677788888888888888888888888888888888876543211|
         prL htm |111111122222333322211111111111111111111111111111111123456788|
  other:         |                                                            |
         PHDFhtm |TTTTTTTTTTTTTTTT    TTTTTTTTTTTTTTTTTTTTTTTTTTTTTTTTT       |
         PHDRhtm |TTTTTTTT             TTTTTTTTTTTTTTTTTTTTTTTTT              |
         PHDThtm |TTTTTTTTiiiiiiiiiiiiiTTTTTTTTTTTTTTTTTTTTTTTTToooooooooooooo|
 subset:         |                                                            |
         SUB htm |............................................................|
                  ....,....13...,....14...,....15...,....16...,....17...,....18
         AA      |ENSLGRNDLARGVNSGQGLGIEIIGTLQLVLCVLATTDRRRRDLGGSAPLAIGLSVALGH|
         PHD htm |               TTTTTTTTTTTTTTTTTTT             TTTTTTTTTTTTT|
 detail:         |                                                            |
         prH htm |000000000001234567788888999988887643211111111235788899998888|
         prL htm |999999999998765432211111000011112356788888888764211100001111|
  other:         |                                                            |
         PHDFhtm |               TTTTTTTTTTTTTTTTTTT             TTTTTTTTTTTTT|
         PHDRhtm |                TTTTTTTTTTTTTTTTTT              TTTTTTTTTTTT|
         PHDThtm |ooooooooooooooooTTTTTTTTTTTTTTTTTTiiiiiiiiiiiiiiTTTTTTTTTTTT|
 subset:         |                                                            |
         SUB htm |............................................................|
                  ....,....19...,....20...,....21...,....22...,....23...,....24
         AA      |LLAIDYTGCGINPARSFGSAVLTRNFSNHWIFWVGPFIGSALAVLIYDFILAPRSSDFTD|
         PHD htm |TTTTTTTTT            TTTTTTTTTTTTTTTTTTTTTTTTTTTTTTT        |
 detail:         |                                                            |
         prH htm |888887765443432233334566777777788888888888888888887542100000|
         prL htm |111112234556567766665433222222211111111111111111112457899999|
  other:         |                                                            |
         PHDFhtm |TTTTTTTTT            TTTTTTTTTTTTTTTTTTTTTTTTTTTTTTT        |
         PHDRhtm |TTTTTT                         TTTTTTTTTTTTTTTTTTT          |
         PHDThtm |TTTTTToooooooooooooooooooooooooTTTTTTTTTTTTTTTTTTTiiiiiiiiii|
 subset:         |                                                            |
         SUB htm |............................................................|
                  ....,....25...,....26...,....27...,....28...,....29...,....30
         AA      |RMKVWTSGQVEEYDLDADDINSRVEMKPK|
         PHD htm |                             |
 detail:         |                             |
         prH htm |00000000000000000000000000000|
         prL htm |99999999999999999999999999999|
  other:         |                             |
         PHDFhtm |                             |
         PHDRhtm |                             |
         PHDThtm |iiiiiiiiiiiiiiiiiiiiiiiiiiiii|
 subset:         |                             |
         SUB htm |.............................|
---
--- PhdTopology REFINEMENT AND TOPOLOGY PREDICTION END
---

________________________________________________________________________________



________________________________________________________________________________


-----------------------------------------------------------------------------
---   PredictProtein: NEWS from January, 1997                             ---
---                                                                       ---
---   Dear user,                                                          ---
---                                                                       ---
---      as of  January 1, 1997,  EMBL has effectively decided to not     ---
---   support the PredictProtein service by personal resources.  I do     ---
---   maintain the program, so to speak, in my private time. However,     ---
---   my contract obliges me to do science, instead.   Unfortunately,     ---
---   the computer environment at  EMBL  is at the same time starting     ---
---   to become increasingly unstable.   Consequence of these two re-     ---
---   cent developments is that the  PredictProtein service is not as     ---
---   stable as it was.                                                   ---
---                                                                       ---
---      I apologise for the problems this may cause.  In particular,     ---
---   I apologise for my inability to reply to the 20-30 daily,  per-     ---
---   sonal mails, and suggest to  re-submit requests after 24 hours!     ---
---                                                                       ---
---   Hoping  that I shall  find  a more convenient  solution for the     ---
---   future of the PredictProtein I remain with my best regards,         ---
---                                                                       ---
---   Burkhard Rost                                                       ---
-----------------------------------------------------------------------------
---   PredictProtein: NEWS from April, 1998                               ---
---                                                                       ---
--------------------------------                                          ---
---   MOVING PredictProtein                                               ---
---   There appears to be light on the horizon! PP will may be having     ---
---   many hickups over the next months (as I shall leave EMBL). How-     ---
---   ever, the server seems to have a fair chance of survival thanks     ---
---   to a major support that is being raised by Columbia University,     ---
---   New York, U.S.A.).   I hope that this will settle the issue for     ---
---   the years to come ...                                               ---
--------------------------------                                          ---
---   WARNING                                                             ---
---   After a  major  rewriting of most of the PP code over the last,     ---
---   I am afraid  that  not all errors have been traced by me,  yet.     ---
---   Thus, please have mercy and report any bug you'll encounter!        ---
---                                             THANKS, Burkhard Rost     ---
--------------------------------                                          ---
---   NEW PREDICTION DEFAULTS                                             ---
---   * Coiled-coil regions: now by default the program COILS written by  ---
---     Andrei Lupas is run on your sequence. An output is returned if a  ---
---     coiled-coil region has been detected.                             ---
---   * Functional sequence motifs: now by default the PROSITE database   ---
---     written by Amos Bairoch, Philip Bucher and Kay Hofmann is scanned ---
---     for sequence motifs. An output is returned if any motif has been  ---
---     detected.                                                         ---
--------------------------------                                          ---
---   see http://www.embl-heidelberg.de/predictprotein/ppNews.html        ---
---     for a description of the following new options.                   ---
---   NEW INPUT OPTION                                                    ---
---   * Your input sequence(s) in FASTA-list format ("# FASTA list ")     ---
---   NEW OUTPUT OPTIONS                                                  ---
---   * Return also BLASTP output ("return blast")                        ---
---   * Return prediction additionally in RDB format ("return phd rdb")   ---
---   * Return topits hssp  ("return topits hssp")                        ---
---   * Return topits strip ("return topits strip")                       ---
---   * Return topits own   ("return topits own")                         ---
---   * Return no coils     ("return no coils")                           ---
---   * Return no prosite   ("return no prosite")                         ---
-----------------------------------------------------------------------------
%</AQP1PHD>
%    \end{macrocode}
%    \begin{macrocode}
%<*AQPHMMsgl>
>HP: 269 AQP1  IN    6   14    33    54    73    94   112   139   156   165   184   211   230 
>HP: 271 AQP2  IN    6   17    35    44    65    86   104   131   148   157   176   203   224 
>HP: 285 AQP3  IN    6   22    41    50    72   103   122   153   172   185   207   238   260 
>HP: 323 AQP4  IN    6   37    57    70    92   123   147   160   177   186   205   232   254 
>HP: 265 AQP5  IN    6   13    32    59    78    87   110   131   149   158   177   204   228 
%</AQPHMMsgl>
%    \end{macrocode}
%    \begin{macrocode}
%<*AQPHMMext>
Protein: AQP1
Length:  269
N-terminus:  IN 
Number of transmembrane helices: 6
Transmembrane helices: 14-33 54-73 94-112 139-156 165-184 211-230 

Total entropy of the model:  17.0025
Entropy of the best path:  17.0049

The best path:

     seq  MASEIKKKLF WRAVVAEFLA MTLFVFISIG SALGFNYPLE RNQTLVQDNV    50
     pred IIIIiiiiii iiiHHHHHHH HHHHHHHHHH HHHooooooo oooooooooo 

     seq  KVSLAFGLSI ATLAQSVGHI SGAHSNPAVT LGLLLSCQIS ILRAVMYIIA   100
     pred oooHHHHHHH HHHHHHHHHH HHHiiiiiii iiiiiiiiii iiiHHHHHHH 

     seq  QCVGAIVASA ILSGITSSLL ENSLGRNDLA RGVNSGQGLG IEIIGTLQLV   150
     pred HHHHHHHHHH HHoooooooo oooooooooo ooooooooHH HHHHHHHHHH 

     seq  LCVLATTDRR RRDLGGSAPL AIGLSVALGH LLAIDYTGCG INPARSFGSA   200
     pred HHHHHHiiii iiiiHHHHHH HHHHHHHHHH HHHHoooooo oooooooooo 

     seq  VLTRNFSNHW IFWVGPFIGS ALAVLIYDFI LAPRSSDFTD RMKVWTSGQV   250
     pred oooooooooo HHHHHHHHHH HHHHHHHHHH iiiiiiiiii iiiiiIIIII 

     seq  EEYDLDADDI NSRVEMKPK  269
     pred IIIIIIIIII IIIIIIIII

Protein: AQP2
Length:  271
N-terminus:  IN 
Number of transmembrane helices: 6
Transmembrane helices: 17-35 44-65 86-104 131-148 157-176 203-224 

Total entropy of the model:  17.0017
Entropy of the best path:  17.0046

The best path:

     seq  MWELRSIAFS RAVLAEFLAT LLFVFFGLGS ALQWASSPPS VLQIAVAFGL    50
     pred IIIIIIiiii iiiiiiHHHH HHHHHHHHHH HHHHHooooo oooHHHHHHH 

     seq  GIGILVQALG HVSGAHINPA VTVACLVGCH VSFLRAAFYV AAQLLGAVAG   100
     pred HHHHHHHHHH HHHHHiiiii iiiiiiiiii iiiiiHHHHH HHHHHHHHHH 

     seq  AAILHEITPV EIRGDLAVNA LHNNATAGQA VTVELFLTMQ LVLCIFASTD   150
     pred HHHHoooooo oooooooooo oooooooooo HHHHHHHHHH HHHHHHHHii 

     seq  ERRGDNLGSP ALSIGFSVTL GHLLGIYFTG CSMNPARSLA PAVVTGKFDD   200
     pred iiiiiiHHHH HHHHHHHHHH HHHHHHoooo oooooooooo oooooooooo 

     seq  HWVFWIGPLV GAIIGSLLYN YLLFPSAKSL QERLAVLKGL EPDTDWEERE   250
     pred ooHHHHHHHH HHHHHHHHHH HHHHiiiiii iiiiiiiiiI IIIIIIIIII 

     seq  VRRRQSVELH SPQSLPRGSK A  271
     pred IIIIIIIIII IIIIIIIIII I

Protein: AQP3
Length:  285
N-terminus:  IN 
Number of transmembrane helices: 6
Transmembrane helices: 22-41 50-72 103-122 153-172 185-207 238-260 

Total entropy of the model:  17.0059
Entropy of the best path:  17.0075

The best path:

     seq  MNRCGEMLHI RYRLLRQALA ECLGTLILVM FGCGSVAQVV LSRGTHGGFL    50
     pred IIIIIIiiii iiiiiiiiii iHHHHHHHHH HHHHHHHHHH HooooooooH 

     seq  TINLAFGFAV TLAILVAGQV SGAHLNPAVT FAMCFLAREP WIKLPIYTLA   100
     pred HHHHHHHHHH HHHHHHHHHH HHiiiiiiii iiiiiiiiii iiiiiiiiii 

     seq  QTLGAFLGAG IVFGLYYDAI WAFAGNELVV SGPNGTAGIF ATYPSGHLDM   150
     pred iiHHHHHHHH HHHHHHHHHH HHoooooooo oooooooooo oooooooooo 

     seq  VNGFFDQFIG TAALIVCVLA IVDPYNNPVP RGLEAFTVGL VVLVIGTSMG   200
     pred ooHHHHHHHH HHHHHHHHHH HHiiiiiiii iiiiHHHHHH HHHHHHHHHH 

     seq  FNSGYAVNPA RDFGPRLFTA LAGWGSEVFT TGQNWWWVPI VSPLLGSIGG   250
     pred HHHHHHHooo oooooooooo oooooooooo oooooooHHH HHHHHHHHHH 

     seq  VFVYQLMIGC HLEQPPPSTE AENVKLAHMK HKEQI  285
     pred HHHHHHHHHH iiiiiiiiii iiiiiIIIII IIIII

Protein: AQP4
Length:  323
N-terminus:  IN 
Number of transmembrane helices: 6
Transmembrane helices: 37-57 70-92 123-147 160-177 186-205 232-254 

Total entropy of the model:  17.0058
Entropy of the best path:  17.0091

The best path:

     seq  MSDGAAARRW GKCGPPCSRE SIMVAFKGVW TQAFWKAVTA EFLAMLIFVL    50
     pred IIIIIIIIII IIIIIIIIII Iiiiiiiiii iiiiiiHHHH HHHHHHHHHH 

     seq  LSVGSTINWG GSENPLPVDM VLISLCFGLS IATMVQCFGH ISGGHINPAV   100
     pred HHHHHHHooo oooooooooH HHHHHHHHHH HHHHHHHHHH HHiiiiiiii 

     seq  TVAMVCTRKI SIAKSVFYIT AQCLGAIIGA GILYLVTPPS VVGGLGVTTV   150
     pred iiiiiiiiii iiiiiiiiii iiHHHHHHHH HHHHHHHHHH HHHHHHHooo 

     seq  HGNLTAGHGL LVELIITFQL VFTIFASCDS KRTDVTGSVA LAIGFSVAIG   200
     pred oooooooooH HHHHHHHHHH HHHHHHHiii iiiiiHHHHH HHHHHHHHHH 

     seq  HLFAINYTGA SMNPARSFGP AVIMGNWENH WIYWVGPIIG AVLAGALYEY   250
     pred HHHHHooooo oooooooooo oooooooooo oHHHHHHHHH HHHHHHHHHH 

     seq  VFCPDVELKR RLKEAFSKAA QQTKGSYMEV EDNRSQVETE DLILKPGVVH   300
     pred HHHHiiiiii iiiiiiiiiI IIIIIIIIII IIIIIIIIII IIIIIIIIII 

     seq  VIDIDRGDEK KGKDSSGEVL SSV  323
     pred IIIIIIIIII IIIIIIIIII III

Protein: AQP5
Length:  265
N-terminus:  IN 
Number of transmembrane helices: 6
Transmembrane helices: 13-32 59-78 87-110 131-149 158-177 204-228 

Total entropy of the model:  17.0020
Entropy of the best path:  17.0052

The best path:

     seq  MKKEVCSLAF FKAVFAEFLA TLIFVFFGLG SALKWPSALP TILQISIAFG    50
     pred IIIIIIIIii iiHHHHHHHH HHHHHHHHHH HHoooooooo oooooooooo 

     seq  LAIGTLAQAL GPVSGGHINP AITLALLIGN QISLLRAVFY VAAQLVGAIA   100
     pred ooooooooHH HHHHHHHHHH HHHHHHHHii iiiiiiHHHH HHHHHHHHHH 

     seq  GAGILYWLAP LNARGNLAVN ALNNNTTPGK AMVVELILTF QLALCIFSST   150
     pred HHHHHHHHHH oooooooooo oooooooooo HHHHHHHHHH HHHHHHHHHi 

     seq  DSRRTSPVGS PALSIGLSVT LGHLVGIYFT GCSMNPARSF GPAVVMNRFS   200
     pred iiiiiiiHHH HHHHHHHHHH HHHHHHHooo oooooooooo oooooooooo 

     seq  PSHWVFWVGP IVGAMLAAIL YFYLLFPSSL SLHDRVAVVK GTYEPEEDWE   250
     pred oooHHHHHHH HHHHHHHHHH HHHHHHHHii iiiiiiiiii iiiIIIIIII 

     seq  DHREERKKTI ELTAH  265
     pred IIIIIIIIII IIIII
%</AQPHMMext>
%    \end{macrocode}
%    \begin{macrocode}
%<*TCoffee>
T-COFFEE, Version_5.31(Fri Oct 26 17:01:36 2007)
Cedric Notredame 
CPU TIME:1 sec.
SCORE=54
*
 BAD AVG GOOD
*
AQP1.PRO   :  55
AQP2.PRO   :  58
AQP3.PRO   :  46
AQP4.PRO   :  55
AQP5.PRO   :  58
cons       :  54

AQP1.PRO       1 65------------------------3777666677777766666666556666666554   36 
AQP2.PRO       1 -2------------------------2776667777777777777667666666666654   35 
AQP3.PRO       1 5---------9644-----33322111------457777665555556555657555544   40 
AQP4.PRO       1 63DGAAARRW9644PPCSR33322113667667677777666666666666767666665   60 
AQP5.PRO       1 65------------------------4877667667777777776666666667666665   36 
cons           1 54--------9644-----33322113777667667777766666666666666666554   60 


AQP1.PRO      37 111221111223454555555555444444444554455555555544544445555444   96 
AQP2.PRO      36 ---33----323444444444444444454444455556666555555555455565555   88 
AQP3.PRO      41 111211----11444444444433333232333344455565555544444445555444   96 
AQP4.PRO      61 ---331111223554555555544444454455555666666666555555455555555  117 
AQP5.PRO      37 ---43----323555555555554444454455555555666555555555555565555   89 
cons          61 111331111323454545444444444444444455555665555555555455555555  120 


AQP1.PRO      97 5555555556555655555----------4444445555555555555555-----5555  141 
AQP2.PRO      89 6556666666666665555----------5555555566666666666666-----5556  133 
AQP3.PRO      97 4445555555555554443YYDAIWAFAG2322222333333344443454LDMVN4445  156 
AQP4.PRO     118 5556666666666665544----------4445545555555555665666-----6666  162 
AQP5.PRO      90 5555555556666665554----------4455555666666666666677-----6676  134 
cons         121 5555556666666655554----------4444444555555555555566-----5566  180 


AQP1.PRO     142 6677777777777766666666-5666656666666555666556677777778777766  200 
AQP2.PRO     134 5566777777777766666555-5565556666666555556556788888888888777  192 
AQP3.PRO     157 7666777777777777765444V3454455554444322222224577777777777665  216 
AQP4.PRO     163 7777777777777777776666-5555555555666555555555677777777777766  221 
AQP5.PRO     135 7777888788887777776666-5666666666666555555556677777777777776  193 
cons         181 7677777777777777666555-5565555565556544555555677777777777766  240 


AQP1.PRO     201 7766--666---------66666666665555666666555566544455555220111-  248 
AQP2.PRO     193 7766--666---------6666666665555555555555556664456666633-222-  239 
AQP3.PRO     217 5544LA2220EVFTTGQN223333434333334433332322------------------  258 
AQP4.PRO     222 6555--555---------55655666656666655555555454-33345544100111-  268 
AQP5.PRO     194 6666--5440--------5555556655555555555544445664444555533-222T  242 
cons         241 6665--5550--------55555555555555555555444455544455555220221-  300 


AQP1.PRO     249 22222--2111111111----11000----------------------------3  269 
AQP2.PRO     240 33333--42222222223--3333333233322-------------------223  271 
AQP3.PRO     259 --------------------2111222122211122-111112221-------12  285 
AQP4.PRO     269 11111KG22111111123VE3222333233322122D111112221SSGEVL224  323 
AQP5.PRO     243 23333--33222223334--3333333----------------------------  265 
cons         301 23322--32222222223--3222223233322122-111112221------223  355 
%</TCoffee>
%    \end{macrocode}
%    \begin{macrocode}
%<*Standard>
%%%%%%%%%%%%%%%%%%%%%%%%%%%%%%%%%%%%%%%%%%%%%%%%%%%%%%%%%%%%%%%%%%%%%%%%%%%%%%%%
%%%%%                                                                      %%%%%
%%%%%                   Standard genetic code definitions                  %%%%%
%%%%%                                                                      %%%%%
%%%%%      (The last codon of each list is used for backtranslations       %%%%%
%%%%%        from protein to DNA sequences---therefore the wobbles)        %%%%%
%%%%%                                                                      %%%%%
%%%%%              These definitions are default in TeXshade.              %%%%%
%%%%%     There is no need to load them. This is an example file only.     %%%%%
%%%%%                                                                      %%%%%
%%%%%%%%%%%%%%%%%%%%%%%%%%%%%%%%%%%%%%%%%%%%%%%%%%%%%%%%%%%%%%%%%%%%%%%%%%%%%%%%

\codon{A}{GCA,GCG,GCC,GCT,GCU,GCN}
\codon{C}{TGC,TGT,UGC,UGU,TGY}
\codon{D}{GAC,GAT,GAU,GAY}
\codon{E}{GAA,GAG,GAR}
\codon{F}{TTC,TTT,UUC,UUU,TTY}
\codon{G}{GGA,GGG,GGC,GGT,GGU,GGN}
\codon{H}{CAC,CAT,CAY}
\codon{I}{ATA,ATC,ATT,AUA,AUC,AUU,ATH}
\codon{K}{AAA,AAG,AAG,AAR}
\codon{L}{CTA,CTG,CTC,CTT,TTA,TTG,CUG,CUG,CUC,CUU,UUA,UUG,YTN}
\codon{M}{ATG,AUG,ATG}
\codon{N}{AAC,AAT,AAU,AAY}
\codon{P}{CCA,CCG,CCC,CCT,CCU,CCN}
\codon{Q}{CAA,CAG,CAR}
\codon{R}{AGA,AGG,CGA,CGG,CGC,CGT,CGU,MGN}
\codon{S}{TCT,TCC,TCG,TCA,AGT,AGC,UCU,UCC,UCG,UCA,AGU,WSN}
\codon{T}{ACT,ACC,ACG,ACA,ACU,ACN}
\codon{V}{GTA,GTG,GTC,GTT,GUA,GUG,GUC,GUU,GTN}
\codon{W}{TGG,UGG,TGG}
\codon{Y}{TAC,TAT,UAC,UAU,TAY}
\codon{.}{TAA,TAG,TGA,UAA,UAG,UGA,TRR}
%</Standard>
%    \end{macrocode}
%    \begin{macrocode}
%<*Ciliate>
%%%%%%%%%%%%%%%%%%%%%%%%%%%%%%%%%%%%%%%%%%%%%%%%%%%%%%%%%%%%%%%%%%%%%%%%%%%%%%%%
%%%%%                                                                      %%%%%
%%%%%            Ciliate macronuclear genetic code definitions             %%%%%
%%%%%                                                                      %%%%%
%%%%%    Only exchanges compared to the standard code must be defined.     %%%%%
%%%%%                                                                      %%%%%
%%%%%      (The last codon of the list is used for backtranslations        %%%%%
%%%%%       from protein to DNA sequences---therefore the wobbles)         %%%%%
%%%%%                                                                      %%%%%
%%%%%                                                                      %%%%%
%%%%%    Activate these definitions for your alignment by the following    %%%%%
%%%%%    command in the texshade environment:                              %%%%%
%%%%%                                                                      %%%%%
%%%%%                         \geneticcode{ciliate}                        %%%%%
%%%%%                                                                      %%%%%
%%%%%%%%%%%%%%%%%%%%%%%%%%%%%%%%%%%%%%%%%%%%%%%%%%%%%%%%%%%%%%%%%%%%%%%%%%%%%%%%

\codon{Q}{TAA,TAG,UAA,UAG,YAR}
%</Ciliate>
%    \end{macrocode}
%    \Finale
\endinput
