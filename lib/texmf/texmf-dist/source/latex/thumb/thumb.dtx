% \iffalse % thumb.dtx
%<*copyright>
% thumb macros for use with the LaTeX book class.
% $Header: thumb.dtx,v 1.0 97/12/24 14:43:14 sdc Exp $
%$%%%%%%%%%%%%%%%%%%%%%%%%%%%%%%%%%%%%%%%%%%%%%%%%%%%%%%%%%%%%%%%%%%%%%%
% Copyright (C) 1997 Christian Holm.
%
% The thumb package is free software; you can redistribute it
% and/or modify it under the terms of the GNU General Public License
% as published by the Free Software Foundation; either version 2 of
% the License, or (at your option) any later version.
%
% The thumb package is distributed in the hope that it will be
% useful, but WITHOUT ANY WARRANTY; without even the implied warranty
% of MERCHANTABILITY or FITNESS FOR A PARTICULAR PURPOSE.  See the 
% GNU General Public License for more details.
%
% You should have received a copy of the GNU General Public License
% along with this program; if not, write to the Free Software
% Foundation, Inc., 675 Mass Ave, Cambridge, MA 02139, USA.
%%%%%%%%%%%%%%%%%%%%%%%%%%%%%%%%%%%%%%%%%%%%%%%%%%%%%%%%%%%%%%%%%%%%%%%%
%% @LaTeX-style-file{
%%    Author     = "Christian Holm",
%%    Version    = "1.0",
%%    Date       = "1997/12/24",
%%    Time       = "14:43:14",
%%    Filename   = "thumb.dtx",
%%    Address    = "Niels Bohr Institute of Physics
%%                  University of Copenhagen
%%                  Denmark",
%%    Email      = "cholm@fys.ku.dk (Internet)",
%%    CodeTable  = "ISO/ASCII",
%%    Keywords   = "LaTeX2e, thumb, minitoc, fancyhdr",
%%    Supported  = "yes",
%%    Abstract   = "LaTeX package for providing support for the
%%                  inclusion of ``thumb'' marks."  
%% }
%%%%%%%%%%%%%%%%%%%%%%%%%%%%%%%%%%%%%%%%%%%%%%%%%%%%%%%%%%%%%%%%%%%%%%%%
%</copyright>
%<*driver>
\documentclass{ltxdoc}
    \EnableCrossrefs
%    \DisableCrossrefs    % use \DisableCrossrefs if the
                          % index is ready

   \RecordChanges
%   \OnlyDescription
  \CodelineIndex

   \typeout{Expect some Under- and overfull boxes}

\begin{document}
  \DocInput{thumb.dtx}
\end{document}%
%</driver>
%
% \fi
%
%\def\fileversion{v1.0} 
%\def\filedate{97/12/24}
%\def\docdate {97/12/24}
%\def\bs{{\tt\char'134}}
%\newcommand{\Lcs}[1]{{\bs\tt#1}}
%\def\SpecialOptIndex#1{%
%    \index{#1\actualchar{\protect\ttfamily#1}
%           (option)\encapchar usage}%
%    \index{options:\levelchar{\protect\ttfamily#1}\encapchar
%           usage}}
%\def\DescribeOption{\leavevmode
%   \begingroup\MakePrivateLetters\DescribeOpt}
%\def\DescribeOpt#1{\endgroup
%              \marginpar{\raggedleft\PrintDescribeMacro{#1}}%
%              \SpecialOptIndex{#1}\ignorespaces}
%\def\SpecialCountIndex#1{%
%    \index{#1\actualchar{\protect\ttfamily#1}
%           (counter)\encapchar usage}%
%    \index{counters:\levelchar{\protect\ttfamily#1}\encapchar
%           usage}}
%\def\DescribeCount{\leavevmode
%   \begingroup\MakePrivateLetters\DescribeCou}
%\def\DescribeCou#1{\endgroup
%              \marginpar{\raggedleft\PrintDescribeMacro{#1}}%
%              \SpecialCountIndex{#1}\ignorespaces}
%\def\SpecialPageIndex#1{%
%    \index{#1\actualchar{\protect\ttfamily#1}
%           (page style)\encapchar usage}%
%    \index{page styles:\levelchar{\protect\ttfamily#1}\encapchar
%           usage}}
%\def\DescribePage{\leavevmode
%   \begingroup\MakePrivateLetters\DescribePg}
%\def\DescribePg#1{\endgroup
%              \marginpar{\raggedleft\PrintDescribeMacro{#1}}%
%              \SpecialPageIndex{#1}\ignorespaces}
%\def\SpecialBoxIndex#1{%
%    \index{#1\actualchar{\protect\ttfamily#1}
%           (box)\encapchar usage}%
%    \index{box:\levelchar{\protect\ttfamily#1}\encapchar
%           usage}}
%\def\DescribeBox{\leavevmode
%   \begingroup\MakePrivateLetters\DescribeBx}
%\def\DescribeBx#1{\endgroup
%              \marginpar{\raggedleft\PrintDescribeMacro{#1}}%
%              \SpecialPageBox{#1}\ignorespaces}
%
% \DoNotIndex{\@ne,\advance,\begin,\char,\closein,\def,\DeclareOption}
% \DoNotIndex{\divide,\docdate,\else,\end,\fi,\filedate,\fileversion}
% \DoNotIndex{\framebox,\global,\ifdim,\ifeof,\ifnum,\ifx,\let,\loop}
% \DoNotIndex{\makebox,\mbox,\mutliply,\NeedsTeXFormat,\newcommand}
% \DoNotIndex{\newcount,\newdim,\newenvironment,\newif,\newsavebox}
% \DoNotIndex{\noindent,\openin,\PackageWarning,\par,\ProcessOptions}
% \DoNotIndex{\ProvidesPackage,\put,\relax,\repeat,\RequirePackage}
% \DoNotIndex{\rule,\savebox,\space,\tempa,\textsc,\tt,\undefined}
% \DoNotIndex{\usebox}
% \iffalse
% \CheckSum{881}
% \fi
% \changes{v1.0}{97/12/24}{The creation of \textsf{thumb}}
%
% \title{The \textsf{thumb}--Package\thanks{%
%    This file has version number \fileversion{} dated \filedate{}.
%    The documentation was last revised on \docdate.
%       }}
% \author{Christian Holm\thanks{This package id largely based on the
% work of Piet van Oosrum, in the package \textsf{fancyhdr}.}\\
% Niels Bohr Institute of Physics\\
% University of Copenhagen}
%
% \date{\today}
%
% \maketitle
%
% \begin{abstract}
%    This style option contains the definitions that are necessary to
%    make a \textit{thumb} index, providing an easy reference methode
%    for large books, manuals, and reference guides. 
% \end{abstract}
%
% \tableofcontents
%
% \section{Use of this package}
%
% This package puts running thumb marks in the margin, moving downward
% as the chapters increase. Also, it is possible to have an
% ``Overview'' page, contaning the chapter names and a thumb mark
% corrosponding to every chapter, and where it is positioned in the
% chapter. I think this package could be usefull for people writing
% referance guides, where quick look--up is important, or for
% anthologies or the like.
%
% This package is intented for large documents only, and should be
% used with a document class, such as \texttt{book} or \texttt{report},
% \emph{not} \texttt{article}. The reason why this is limited so, is
% because, it doesn't make sense to have thumbmarks in a docuemnt
% smaller then 30 pages.
%
% \subsection{What to do in your document source}
% 
% To use this package, include it in a document with
% \begin{quote}
%   \Lcs{usepackage[}\textsl{options}\texttt{]\{thumb\}}
% \end{quote}
%
% \DescribeOption{minitoc}
% Here \textsl{option}, can be \texttt{minitoc} for miniture table of
% contents at each chapter. This depends on package \textsf{minitoc},
% so you need to have this installed, if you want to use this feature.
%
% \DescribeOption{box}\DescribeOption{filledbox}\DescribeOption{oval}
% \DescribeOption{filledoval}
% Option \texttt{minitoc}, can be combined with \emph{one} of the
% below, or not at all. In any case, of the options below, \emph{only
% one} or less can be used.
%
% These options specify the \textit{style} of the ``thumbs''. They
% are shown in figure~\ref{tab:styles}
% \begin{figure}[htbp]
%   \setlength{\unitlength}{1cm}
%   \begin{picture}(5,10)
%     \put(0,0){\mbox{\texttt{filledbox} (defualt)}}
%     \put(5,0){\rule{3cm}{1cm}}
%     \put(0,2){\mbox{\texttt{box}}}
%     \put(5,2){\framebox(3,1)}
%     \put(0,4){\mbox{\texttt{oval}}}
%     \put(6.5,4.5){\oval(3,1)}
%     \put(0,6){\mbox{\texttt{filledoval}}}
%     \put(5,6){\circle*{0.5}}
%     \put(5,5.77){\rule{3cm}{0.5cm}}
%   \end{picture}
%   \caption{``thumb'' styles}\label{tab:styles}
% \end{figure}
%
% Notice the style \texttt{filledoval} \emph{cannot} be made any
% higher. This is due the the primitive drawning mechanisms of
% \LaTeX{} without extension packages as \textsf{epic} etc..
% 
% The \texttt{box} and \texttt{oval} style, will extend the thumbs on
% the ``overview page'', to reach over the chapter name and
% number. This will effecually ``frame'' the chapter name with the
% thumb.
%
% \subsection{Style Commands}
%
% While the general style of the ``thumbs'' are selected by option to
% the package, it is also possible to say something more via commands.
%
% \DescribeMacro{\Overviewpage} 
%
% First of, there is the ``Overview page''. To trigger this page, put
% \Lcs{Overviewpage} at the begining of your document. I think it
% should come just before the preface and the table of contents, but
% you may disagree.
%
% This page will contain, the thumbs for all the capters. plus their
% names, printed next to there corrosponding thumb mark, and the
% title, author(s) and date of the document, in a seperate cloumn of
% its own.
%
% If you use the overview page, you should proberly run you docuemnt
% through \LaTeX{} \emph{at least} three times. This is because, the
% first time, \LaTeX{} must create a file, the second time, it will
% have to adjust the space between the thumb marks, and thirdly, it
% should type set it all proberly. However, this is not that strange,
% in fact, packages like \textsf{minitoc}, may sometimes require 5
% runs, until the utput comes out right.
%
% The use of the ``Overview page'' does, however put some restrains on
% your formatting, though they are minor, and I think a have made a
% suttable compensation. More on this later.
%
% \subsection{Manipulating the ``thumbs''}
%
% There are two ways in which you can maniplute the thumb marks. It
% should come as no surpise, that these are the height and width of
% the thumbs. All you ghave to do is to set this \textsl{lengths} to
% the value, you desire. By default, they are calculated from the
% paper and text dimensions. 
%
% \DescribeMacro{\thumbwidth}\DescribeMacro{\thumbheight}
% The \textsl{lenghts} you should change are \Lcs{thumbheight} and
% \Lcs{thumbwidth}. The should be cahnge by using the \LaTeX{} command
% \Lcs{setlength\{}\textsl{length}\texttt{\}\{}\textsl{dimension}\texttt{\}} 
% where \textsl{dimension} is a valid \LaTeX{} dimension (a number
% plus a unit). 
%
% \StopEventually{}
% \iffalse
%<*package>
% \fi
%
% \section{Implementation}
% 
% First some package identification:
%    \begin{macrocode}
\def\fileversion{v1.0} 
\def\filedate{97/12/24}
\ProvidesPackage{thumb}[\filedate\space\fileversion\space thumb
                          package.]
\NeedsTeXFormat{LaTeX2e}[1994/06/01]
%    \end{macrocode} 
%
% \subsection{Initialization}
%
% \begin{macro}{\ifhave@fancyhdr}
% Then we check to se if package \textsf{fancyhdr} is avaliable. This
% is done by a new conditional \Lcs{ifhave@fancyhdr}:
%    \begin{macrocode}
\newif\ifhave@fancyhdr
\openin\@ne fancyhdr.sty
\ifeof\@ne \else \have@fancyhdrtrue \fi
\closein\@ne \relax
%    \end{macrocode}
% \end{macro}
% If we found {\tt fancyhdr.sty} we use it.  
% Notice the rest of the package is \emph{conditional}, in the sense,
% that if \textsf{fancyhdr} isn't installed, nothing will be defined.
%     \begin{macrocode}
\ifhave@fancyhdr 
  \RequirePackage{fancyhdr}
%    \end{macrocode}
% \DescribeCount{thumb}
% Initialize the counter \texttt{thumb}, which will be used to measure
% out the distance between the ``thumbs''. 
%    \begin{macrocode}
  % Define primitive counters
  \newcount\thumbchap
  \newcount\thumbHskip
  % Define lengths
  \setlength{\unitlength}{1sp}
  \newlength{\thumbheight}
  \newlength{\thumbwidth}
  \newlength{\thumbskip}
  \newlength{\overviewskip}
  % Define counters
  \newcounter{thumbheight}
  \newcounter{thumbwidth}
  \newcounter{lthumbskip}
  \newcounter{rthumbskip}
  \newcounter{overviewskip}
  % the height of the thumbs
  \setlength{\thumbheight}{1cm}
  % the width of the thumbs
  \newdimen\thumbtmp
  \thumbtmp\paperwidth
  \advance\thumbtmp-\textwidth
  \divide\thumbtmp2
  \setlength{\thumbwidth}{\thumbtmp}
  % the skip of the overview titles
  \thumbtmp\textwidth
  \divide\thumbtmp2
  \setlength{\overviewskip}{\thumbtmp}
  % Set counters
  \AtBeginDocument{
    \setcounter{thumbheight}{\thumbheight}
    \setcounter{thumbwidth}{\thumbwidth}
    \setcounter{overviewskip}{\overviewskip}
  }
%    \end{macrocode}
%
% \subsection{The \Lcs{chapter} macro}
%
% \begin{macro}{\chapter}
% The chapter needs to be redefined, so that the ``overview'' page can
% be produced in a nice way. The page style is set to
% \texttt{thumbplain} so that the \texttt{plain} page style can be
% used for page that should contain \emph{no} fancy stuff at all. In
% any other respect, the \Lcs{chapter} works as usual.
%    \begin{macrocode}
  \renewcommand\chapter{\if@openright\cleardoublepage\else\clearpage\fi
                      \thispagestyle{thumbplain}%
                      \global\@topnum\z@
                      \@afterindentfalse
                      \secdef\@chapter\@schapter}
%    \end{macrocode}
% \end{macro}
% \begin{macro}{\@chapter}
% However it is only the \emph{un--stared}
% form \Lcs{chapter\{...\}} that needs this modfication. The stared
% form should not create any line in the ``overview'' page, nor should
% it have a ``table of contents'' if the \texttt{minitoc} option is
% used. This have to be done before the options decleration, because
% style \textsf{minitoc} changes the \Lcs{chapter} macro. 
%    \begin{macrocode}
%\let\sv@chapter\@chapter
%\def\@chapter[#1]#2{\sv@chapter[{#1}]{#2}\relax%}
  \def\@chapter[#1]#2{%
    \ifnum \c@secnumdepth >\m@ne
      \if@mainmatter
        \refstepcounter{chapter}%
        \typeout{\@chapapp\space\thechapter.}%
        \addcontentsline{toc}{chapter}%
          {\protect\numberline{\thechapter}#1}%
      \else
        \addcontentsline{toc}{chapter}{#1}%
      \fi
    \else
      \addcontentsline{toc}{chapter}{#1}%
    \fi
    \chaptermark{#1}%
    \addtocontents{lof}{\protect\addvspace{10\p@}}%
    \addtocontents{lot}{\protect\addvspace{10\p@}}%
    \if@twocolumn
      \@topnewpage[\@makechapterhead{#2}]%
    \else
      \@makechapterhead{#2}%
      \@afterheading
    \fi
    \relax
%    \end{macrocode}
% The above code is taken from \texttt{book.cls}. No changes so far.
%
% Now we put an entry in the \textsl{filename}\texttt{.ovr} file, so
% that the ``overview'' page can input it later. Notice this implies
% \LaTeX{} should be run \emph{at least} twice.
%    \begin{macrocode}
    \ifOverview
      \def\ovrout{\noexpand\secname{\@arabic\c@chapter}{#2}{\thepage}}
      \immediate\write\thumbovr{\ovrout}
    \fi
    \advance\thumbchap1
%    \end{macrocode}
% If the \texttt{minitoc} option is given, a table of contents for
% this chapter is output. Notice this implies \LaTeX{} should be run
% \emph{at least} three times.
%    \begin{macrocode}
    \ifx\thumbmini\undefined
    \else
      \minitoc
    \fi
  }
%    \end{macrocode}
% \end{macro}
%
% \begin{macro}{\preface}
% This macro is for including a preface (a chapter with no number, but
% a entry in the table of contents and the overview page.
% 
% The macro \Lcs{Prefacename} is set to \Lcs{prefacename} if it is
% defined (as in the \textsf{babel} package), else it is set to
% ``Preface''. This means non--english users should redefine the macro
% \Lcs{prefacename}, \emph{not} \Lcs{Prefacename}.
%    \begin{macrocode}
  \ifx\prefacename\undefined
    \newcommand{\Prefacename}{Preface}
  \else
    \newcommand{\Prefacename}{\prefacename}
  \fi
%    \end{macrocode}
% A new \textsl{if} is defined, so that the overview page may use the
% \Lcs{Prefacename} \emph{only} if a preface is present. This means
% the preface should come before the overview page. 
%    \begin{macrocode}
  \newif\ifPreface
  \Prefacefalse
%    \end{macrocode}
% The command \Lcs{preface} takes an optional argument, so
% non--english users may specify another name for the preface, e.g.\
% ``Vorrede''. A new right--handed page is started, and a line is
% added to the table of contents. 
%    \begin{macrocode}
  \newcommand{\preface}[1][\Prefacename]{%
    \Prefacetrue
    \cleardoublepage
    \thispagestyle{thumbplain}
    \addcontentsline{toc}{chapter}{#1}%
%    \end{macrocode}
% Marks are set, and apropiate
% vertical space is added. Next, we take into account wether the
% document is set in two column or not.
%    \begin{macrocode}
    \chaptermark{#1}%
    \vspace*{50\p@}%
    \if@twocolumn
      \@topnewpage[{\Huge \bfseries #1}]%
    \else
      {
      \interlinepenalty\@M
      \Huge \bfseries #1
      \par\nobreak
      \vskip 40\p@
      }
      \@afterheading
    \fi
    \relax
%    \end{macrocode}
% Some commands for the overview page is defined, etc.
%    \begin{macrocode}
    \renewcommand{\Overviewname}{#1}
    \newcount\tempa
    \tempa\thepage
    \newcommand{\Overviewpage}{\the\tempa}
  }
%    \end{macrocode}    
% \end{macro}
% \subsection{Options}
%
% The options are declared. These are \texttt{minitoc, filledbox, box,
% filledoval,} and \texttt{oval}.
%
%  \begin{macro}{minitoc}
% If option \texttt{minitoc} is set, we check to see if the file
% \texttt{minito.sty} is avaliable. If it is we load it. 
%
% \begin{macro}{\ifhave@minitoc}
% To accomplish this we define a new \Lcs{if}
%    \begin{macrocode}
  \DeclareOption{minitoc}{
    \newif\ifhave@minitoc
      \openin\@ne minitoc.sty
      \ifeof\@ne \else \have@minitoctrue \fi
    \closein\@ne \relax
    \ifhave@minitoc 
      \def\thumbmini{}
%    \end{macrocode}
% Notice if you use some other language then English, you will have
% to redefine the macro \Lcs{contentsname} yourself to something
% apropriate\footnote{For example Danish users should say
% \Lcs{renewcommand\{}\Lcs{contentsname\}\{Kort oversigt\}}.}.
%
% If the option \texttt{minitoc} is not set, then put out a warning.
%    \begin{macrocode}
    \else
      \PackageWarning{thumb}{style `minitoc� not avaliable \return
        option `minitoc' will not be used}
    \fi
  }
%    \end{macrocode}
% \end{macro}
% \end{macro}
%
% \subsubsection{The ``thumb'' styles}
% 
% Next the options for the thumbs styles is set. Notice, in every
% option there is a definiton of three different marks
% \Lcs{rthumb}, \Lcs{lthumb} and \Lcs{ovrthumb}. These are used on
% right--, left--handed pages and on the overview page, respectivly. 
%
% First some commands to use wit the options.
%
% Most of the dimensions are defined here also, but often not set
%  until at \Lcs{begin\{document\}}, since various packages such as
% \texttt{a4wide.sty}, may change the size of the paper etc.
%
% \begin{macro}{thumbfilledbox}
%    \begin{macrocode}
  \newcommand{\thumbfilledbox}{%
    \thumbtmp\paperwidth
    \advance\thumbtmp-\textwidth
    \divide\thumbtmp2
    \divide\thumbtmp3
    \setlength{\thumbskip}{\thumbtmp}
    \typeout{Expect some overfull hbox'es in the overview page}
    \AtBeginDocument{
       \setcounter{lthumbskip}{\thumbskip}
       \setcounter{rthumbskip}{\thumbskip}
       \addtocounter{lthumbskip}{\value{thumbwidth}}
    }
%    \end{macrocode}
% From then on, it's is just a question of defing the thumb marks.
% 
% First the thumbmarks for right--handed pages.
%
%    \begin{macrocode}
    \def\rthumb{%
      \thumbHskip\thumbchap
      \multiply\thumbHskip\value{thumbhskip}
      \begin{picture}(0,0)
        \put(\value{rthumbskip},-\the\thumbHskip){    
          \rule{\thumbwidth}{\thumbheight}}
      \end{picture}
    }
%    \end{macrocode}
% Then the thumbmarks for left--handed pages.
%
%    \begin{macrocode}
    \def\lthumb{%
      \thumbHskip\thumbchap
      \multiply\thumbHskip\value{thumbhskip}
      \begin{picture}(0,0)
        \put(-\value{lthumbskip},-\the\thumbHskip){    
          \rule{\thumbwidth}{\thumbheight}}
      \end{picture}
    }
%    \end{macrocode}
% And finally the thumbmarks for overview page.
%
%    \begin{macrocode}
    \def\ovrthumb{%
      \hspace{\thumbskip}\rule{\thumbwidth}{\thumbheight}
    }
  }
%    \end{macrocode}
% \end{macro}
% \begin{macro}{thumbbox}
% This makes the ``thumbs'' black frames. This is pretty much the same
% as above, expect for \Lcs{lthumb}, which I had to define in a
% roundabout way.
%    \begin{macrocode}
  \newcommand{\thumbbox}{
    \thumbtmp\paperwidth
    \advance\thumbtmp-\textwidth
    \divide\thumbtmp6
    \setlength{\thumbskip}{\thumbtmp}
    \typeout{Expect some overfull hbox'es in the overvew page}
    \AtBeginDocument{
      \setcounter{rthumbskip}{\thumbskip}
      \setcounter{lthumbskip}{\thumbtmp}
      \newcounter{thumbovrwidth}
      \setcounter{thumbovrwidth}{\value{overviewskip}}
      \addtocounter{thumbovrwidth}{\value{overviewskip}}
    }
    \def\rthumb{%
      \thumbHskip\thumbchap
      \multiply\thumbHskip\value{thumbhskip}
      \begin{picture}(0,0)
        \put(\value{rthumbskip},-\the\thumbHskip){    
          \framebox(\value{thumbwidth},\value{thumbheight})}
      \end{picture}
    }
%    \end{macrocode}
% The left--handed pages thumbs are a bit tricky. Since \LaTeX{} have
% some problem starting a picture \emph{outside} the page, I had to
% make up the left thumbs, from lines, instead of using the
% \Lcs{framebox} command, which would have been easier, but didn't work!
%
%    \begin{macrocode}
    \def\lthumb{%
      \thumbHskip\thumbchap
      \multiply\thumbHskip\value{thumbhskip}
      \advance\thumbHskip-\value{thumbheight}
      \begin{picture}(0,0)
        \put(-\value{rthumbskip},-\the\thumbHskip)
          {\line(-1,0){\value{thumbwidth}}}
        \put(-\value{lthumbskip},-\the\thumbHskip){%
          \line(0,-1){\value{thumbheight}}}
        \advance\thumbHskip\value{thumbheight}
        \put(-\value{rthumbskip},-\the\thumbHskip)
          {\line(-1,0){\value{thumbwidth}}}
      \end{picture}
    }
%    \end{macrocode}
% The thumbmarks for overview page are actually a bit interresting,
% since it produces a frame around the text put on the overview page
% for each chapter. The thumbs are therefore wider on the overview
% page then in the rest of the document (half the textwidth plus some
% more, aloving it to go beyod the page).
%
%    \begin{macrocode}
    \def\ovrthumb{%
      \hspace{-\overviewskip}\hspace{-2em}\raisebox{-2ex}{
      \framebox(\value{thumbovrwidth},\value{thumbheight})}
    }
  }
%    \end{macrocode}
% \end{macro}
% \begin{macro}{thumbfilledoval}
% This makes the ``thumbs'' filled ovals. Notice, these \emph{can not}
% be higher then ca.\ 0.7cm (ca.\ 0.3 inches), due to \LaTeX{}
% handling of the macro \Lcs{circle}. \Lcs{HOLD} is used to prevent
% redifintion of \Lcs{unitlenght}.
%
% \textbf{Notice:} This style may not come out completly right. That
% is due to some rounding errors, which I didn't try to prevent,
% since I think it is only a minor bug\footnote{OK, so I was lazy
% --- you correct it!}. It is most notaciable on the overview page.
%    \begin{macrocode}
  \newcommand{\thumbfilledoval}{
    \thumbtmp\paperwidth
    \advance\thumbtmp-\textwidth
    \divide\thumbtmp6
    \setlength{\thumbskip}{\thumbtmp}
    \setlength{\thumbheight}{983025sp}
    \typeout{Expect some overfull hbox'es in the overvew page}
    \AtBeginDocument{
      \setcounter{rthumbskip}{\thumbskip}
      \setcounter{lthumbskip}{\thumbtmp}
      \addtocounter{lthumbskip}{\value{thumbwidth}}
    }
    \def\rthumb{%
      \thumbHskip\thumbchap
      \multiply\thumbHskip\value{thumbhskip}
      \begin{picture}(0,0)
        \thicklines
        \put(\value{rthumbskip},-\the\thumbHskip){    
          \circle*{983025}}
        \advance\thumbHskip491513
        \put(\value{rthumbskip},-\the\thumbHskip){    
          \rule{\thumbwidth}{\thumbheight}}
      \end{picture}
    }
    \def\lthumb{%
      \thumbHskip\thumbchap
      \multiply\thumbHskip\value{thumbhskip}
      \begin{picture}(0,0)
        \thicklines
        \put(-\value{rthumbskip},-\the\thumbHskip){    
          \circle*{983025}}
        \advance\thumbHskip491513
        \put(-\value{lthumbskip},-\the\thumbHskip){    
          \rule{\thumbwidth}{\thumbheight}}
      \end{picture}
    }
    \def\ovrthumb{%
      \thumbHskip\thumbchap
      \multiply\thumbHskip\value{thumbhskip}
      \begin{picture}(0,0)
        \thicklines
        \put(\value{rthumbskip},-\the\thumbHskip){    
          \circle*{983025}}
        \advance\thumbHskip491513
        \put(\value{rthumbskip},-\the\thumbHskip){    
          \rule{\thumbwidth}{\thumbheight}}
      \end{picture}
    }
    \def\HOLD{}
  }
%    \end{macrocode}
% \end{macro}
% \begin{macro}{thumboval}
% This makes the ``thumbs'' non--filled ovals.
%
% These thumbs are a bit interresting, in a \TeX{}nical way. The
% starting point of a \Lcs{framebox}, is the lower left corner, and
% for \Lcs{rule}, is the left edge, and for \Lcs{cirle} it is the
% center. However, for \Lcs{oval}, it is the \emph{middle} of the
% oval, so some extra care must be used to put the thumbs on the
% page, when ovals is used. Again, we use the page dimensions to
% ensure the optimal layout.
%
%    \begin{macrocode}
  \newcommand{\thumboval}{
    \thumbtmp\paperwidth
    \advance\thumbtmp-\textwidth
    \divide\thumbtmp6
    \multiply\thumbtmp2
    \setlength{\thumbskip}{\thumbtmp}
    \AtBeginDocument{
      \setcounter{rthumbskip}{\thumbskip}
      \newcounter{thumbovrwidth}
      \setcounter{thumbovrwidth}{\value{overviewskip}}
      \addtocounter{thumbovrwidth}{\value{overviewskip}}
    }
    \def\rthumb{%
      \thumbHskip\thumbchap
      \multiply\thumbHskip\value{thumbhskip}
      \begin{picture}(0,0)
        \thicklines
        \put(\value{rthumbskip},-\the\thumbHskip)
          {\oval(\value{thumbwidth},\value{thumbheight})}
        \put(\value{rthumbskip},-\the\thumbHskip)
          {\usebox{\ovalmark}}
      \end{picture}
    }
    \def\lthumb{%
      \thumbHskip\thumbchap
      \multiply\thumbHskip\value{thumbhskip}
      \begin{picture}(0,0)
        \thicklines
        \put(-\value{rthumbskip},-\the\thumbHskip)
          {\oval(\value{thumbwidth},\value{thumbheight})}
        \put(-\value{rthumbskip},-\the\thumbHskip)
         {\usebox{\ovalmark}}
      \end{picture}
    }
%    \end{macrocode}
% The thumbmarks for overview page are actually a bit interresting,
% since it produces a frame around the text put on the overview page
% for each chapter. This is like the thumbs defined in the
% \texttt{box} style option.
%
%    \begin{macrocode}
    \def\ovrthumb{%
      \hspace{-2em}\oval(\value{thumbovrwidth},\value{thumbheight})
    }
  }
%    \end{macrocode}
% \end{macro}
% Now the actual option declarations.
%    \begin{macrocode}
  \DeclareOption{filledbox}{%
    \thumbfilledbox{}
  }
  \DeclareOption{box}{
    \thumbbox{}
  }
  \DeclareOption{filledoval}{
    \thumbfilledoval{}
  }
  \DeclareOption{oval}{
    \thumboval{}
  }
%    \end{macrocode}
%
% If some strange option is given, then give an error message.
%    \begin{macrocode}
  \DeclareOption*{\PackageWarning{thumb}{Unkown option: `\CurrentOption'}} 
%    \end{macrocode}
% Make the option \texttt{filledbox} default.
%    \begin{macrocode}
  \ExecuteOptions{filledbox}
  \ProcessOptions\relax
%    \end{macrocode}
%
% \subsection{Boxes for headings} 
%
% \begin{macro}{\thumbheader}
% \begin{macro}{\Overviewtitlefont}
% \begin{macro}{\Overviewauthorfont}
% \begin{macro}{\Overviewdatefont}
% First we define some commands, and there defaults, to format the 
% header fonts.
%    \begin{macrocode} 
  \newcommand{\thumbheaderfont}{\small\sl}
  \newcommand{\Overviewtitlefont}{\sf\LARGE}
  \let\Overviewauthorfont\Overviewtitlefont
  \let\Overviewdatefont\Overviewtitlefont
%    \end{macrocode}
% \end{macro}
% \end{macro}
% \end{macro}
% \end{macro}
%
% \begin{macro}{\thumbtitlefont}
% \begin{macro}{\thumbauthorfont}
% \begin{macro}{\thumbdatefont}
% Then we define a new macro to control the font of the title \emph{on
% the titlepage} and nowhere else. This is done, so that the font in
% the headers may be different from the one on the titlepage. However,
% this must be considered a bug, because it means, you have to use the
% macro \Lcs{thumbtitlefont}, if you wish to change the font of the
% title on the titlepage. Otherwise, you can get strange headers!
% Also, you can only use fontmacroes, that normally change the font
% globally, e.g.\ \Lcs{tt}, \Lcs{rm}, \Lcs{bf}, \Lcs{Huge},
% \Lcs{Large}, and \emph{not} the local counterparts \Lcs{texttt},
% \Lcs{textrm}, \Lcs{textbs} etc.. \Lcs{thumbauthorfont} and
% \Lcs{thumbdatefont} work in the same way as \Lcs{thumbtitlefont}
%    \begin{macrocode}
  \newcommand{\thumbtitlefont}{}
  \newcommand{\thumbauthorfont}{}
  \newcommand{\thumbdatefont}{}
%    \end{macrocode}
% \end{macro}
% \end{macro}
% \end{macro}
%
% \begin{macro}{\thumbtitle}
% \begin{macro}{\thumbauthor}
% \begin{macro}{\thumbdate}
% \begin{macro}{\Overviewtitle}
% The next step is to make some saveboxes, and fill them with the
% title of the document.
%    \begin{macrocode}
  \newsavebox{\thumbtitle}
  \newsavebox{\thumbauthor}
  \newsavebox{\thumbdate}
  \newsavebox{\Overviewtitle}
%    \end{macrocode}
% \end{macro}
% \end{macro}
% \end{macro}
% \end{macro}
%
% Then we fill the boxes. this have to be done at the begin of the
% document, since, other styles might change the marks.
%    \begin{macrocode}
  \AtBeginDocument{
    \let\tmptitlefont\thumbtitlefont
    \let\thumbtitlefont\relax
    \let\tmpauthorfont\thumbauthorfont
    \let\thumbauthorfont\relax
    \let\tmpdatefont\thumbdatefont
    \let\thumbdatefont\relax
%    \end{macrocode}
% This is the tricky part. Here we set \Lcs{thumbtitlefont} to be
% void, so that any large font, or otherwise unreasonable font,
% headerwise that is, specified on the titlepage, will not affect the
% header fonts. this is why you \emph{must} use the
% \Lcs{thumbtitlefont} macro to change the bfont on the title font.  
% The \Lcs{thumbtitlefont} is restored to it original use below.
%    \begin{macrocode}
    \let\tmpand\and
    \let\and\par
    \savebox{\thumbtitle}
      {{\thumbheaderfont\@title}}
    \begin{lrbox}{\Overviewtitle}
      \begin{minipage}[c]{0.4\textwidth}
          \Overviewtitlefont\@title\newline
          \Overviewauthorfont\@author\newline
          \Overviewdatefont\@date\newline
        \end{minipage}
    \end{lrbox}
    \let\thumbtitlefont\tmptitlefont
    \let\thumbauthorfont\tmpauthorfont
    \let\thumbdatefont\tmpdatefont
    \let\and\tmpand
%    \end{macrocode}
%
% \begin{macro}{\chaptermark}
% \begin{macro}{\sectionmark}
% \begin{macro}{\subsectionmark}
% After having filled the boxes, we make use of them in the
% headers. If you don't like to have the title in the headers, you
% can always redefine the marks, with \Lcs{markboth\{\}\{\}} and
% \Lcs{markright\{\}\{\}}, though it should be done \emph{after} the
% \Lcs{begin\{document\}}.
%    \begin{macrocode}
    \markboth{\usebox{\thumbtitle}}{\usebox{\thumbtitle}}
    \renewcommand{\chaptermark}[1]{
      \markboth{{\small #1}}{\usebox{\thumbtitle}}
    }
    \renewcommand{\sectionmark}[1]{
      \markboth{{\small #1}}{\usebox{\thumbtitle}}
    }
    \renewcommand{\subsectionmark}[1]{
      \markboth{{\small #1}}{\usebox{\thumbtitle}}
    }
  }
%    \end{macrocode}
% \end{macro}
% \end{macro}
% \end{macro}
%
% \subsection{The Appendix}
%
% \begin{macro}{\appendix}
% The command \Lcs{appendix} is redefined, so that the thumb
% marks come out running, instead of shifting up at the appendix
% \DescribeCount{appendix}
% \DescribeCount{chapter}
% \DescribeCount{section}
%    \begin{macrocode}
  \newif\ifAppendix
  \Appendixfalse
  \newcounter{appendix}
  \setcounter{appendix}{0}
  \renewcommand\appendix{\par
    \setcounter{appendix}{\value{chapter}}
    \setcounter{chapter}{0}%
    \setcounter{section}{0}%
    \renewcommand\chaptername{\appendixname}%
    \renewcommand\thechapter{\Alph{chapter}}
    \ifOverview
      \immediate\write\thumbovr{\noexpand\Appendixtrue}
    \fi
  }
%    \end{macrocode}
% \end{macro}
%
% \subsection{The Lists}
%
% If the option \texttt{minitoc} is set, then the tableofcontents,
% should only contain the parts and and chapters (Like a ``Short Table
% of Contents'')
%    \begin{macrocode}
  \ifx\thumbmini\undefined
  \else
    \RequirePackage{minitoc}
    \setcounter{tocdepth}{0}
    \def\contentsname{Short Table of Contents}%
    \AtBeginDocument{\dominitoc}
%    \end{macrocode}
% Also, since \texttt{minitoc} redefines a lot of the behavior of the
% lists, we have to be able to print the ``List of Figures'', and
% ``List of Tables''. Now since \texttt{tocdepth} is set to 0, normally
% the above lists would \emph{not} be printed, but if we redefine the
% macroes \verb+\l@figure+ and \verb+\l@table+, wich \LaTeX{} use to
% control the beaviour of the lists, so that the first number is less
% then 0, then the lists \emph{do} get printed\footnote{This all sound
% so simple, but it tok me a couple of hours to figure it out.}.
%    \begin{macrocode}
    \renewcommand*\l@figure{\@dottedtocline{-1}{1.5em}{2.3em}}
    \let\l@table\l@figure
  \fi
%    \end{macrocode}
%
% \subsection{Page Styles}
%
% \begin{macro}{\thumbsecnum}
% First we create the macro \Lcs{thumbsecnum}, to get nice headers. It
% test if a counter is 0, if it is, then it, and \emph{all} subsequent
% counters in the section number, should not be printed. Notice, if all
% section numbers are 0, nothing is printed \emph{at all}.
%    \begin{macrocode}
  \newlength{\thumbspace}
  \settowidth{\thumbspace}{9}
  \newcommand{\thumbsecnum}{
    \ifnum\value{chapter}=0
    \else
      \ifnum\value{section}=0
        \thechapter
      \else
        \ifnum\value{subsection}=0
          \thesection
        \else
          \thesubsection
        \fi
      \fi
    \fi
  }
%    \end{macrocode}
% \end{macro}
% The macro \Lcs{thumbspace} is a length, we use when a number was
% supposed to be there, but isn't dor some reason, and so we
% want a empty space. It also used below.
%
% \DescribePage{fancynormal}
% This is he normal page style, that puts the ``thumbs'' on the page,
% and the current section name in the center of the header, and the
% current subsection number on the farmost corner of the header. The
% page number is put in the farmost footer.
%
% The macroes \emph{could} have been made so that the user could
% change them, but instead of doing that, I leave it up to the user,
% to browse the manual of \texttt{fanyhdr} and find out how to do that
%
%    \begin{macrocode}
  \fancypagestyle{thumbnormal}{
    \fancyhf{}
    \renewcommand{\footrulewidth}{\headrulewidth}
    \fancyfoot[LO]{}
    \fancyfoot[CO]{}
    \fancyfoot[RO]{\thepage}
    \fancyfoot[LE]{\thepage}
    \fancyfoot[CE]{}
    \fancyfoot[RE]{}
    \fancyhead[LO]{}
    \fancyhead[CO]{\small\leftmark}
    \fancyhead[RO]{\thumbsecnum\rthumb}
    \fancyhead[LE]{\lthumb\thumbsecnum}
    \fancyhead[CE]{\small\rightmark}
    \fancyhead[RE]{}
  }
  \pagestyle{thumbnormal}
%    \end{macrocode}
%
% \DescribePage{plain}
% This is the page style \LaTeX{} on special pages, such as the first
% of the chapter or part, the tableofcontents, etc. 
%    \begin{macrocode}
  \fancypagestyle{thumbplain}{%
    \fancyhf{}
    \fancyfoot[C]{\thepage}
    \fancyhead[RO]{\rthumb}
    \fancyhead[LE]{\lthumb}
    \renewcommand{\headrulewidth}{0pt}
    \renewcommand{\footrulewidth}{0pt}
  }
  \fancypagestyle{plain}{%
    \fancyhf{}
    \fancyfoot[C]{}
    \fancyhead[R]{}
    \fancyhead[L]{}
    \renewcommand{\headrulewidth}{0pt}
    \renewcommand{\footrulewidth}{0pt}
  }
%    \end{macrocode}
% \DescribePage{fancyoverview}
% This is the page style of the overview page. This page contains a
% ``thumb'' for every chapter of document, plus a line containing the
% chapter number and name. 
%    \begin{macrocode}
  \fancypagestyle{thumboverview}{%
    \fancyhf{}
    \fancyfoot[C]{}
    \fancyfoot[R]{}
    \fancyfoot[L]{}
    \fancyhead[C]{}
    \fancyhead[L]{}
    \fancyhead[R]{\thumboverview}
    \renewcommand{\headrulewidth}{0pt}
    \renewcommand{\footrulewidth}{0pt}
  }
%    \end{macrocode}
%
% \subsection{Commands for the Overview page}
%
% \begin{macro}{\Ovreviewname}
% The first macro, is for deciding what the name of 
% ``Overview'' page should be. This can be refined by saying:
%
% \Lcs{renewcommand\{\bs{}Overviewname\}\{}\textsl{name\_of\_page}\texttt{\}} 
% 
%    \begin{macrocode}
  \newcommand{\Overviewname}{Overview}
%    \end{macrocode}
% \end{macro}
%
% Like above, only it sets the font.
%
% \begin{macro}{\Overviewfont}
%    \begin{macrocode}
  \newcommand{\Overviewfont}{\sf\Large}
%    \end{macrocode}
% \end{macro}
%
% \begin{macro}{\secname}
% This is the command that sets the individual lines on the
% ``Overview'' page. Notice how the previous defined \Lcs{ifAppendix}
% is used here, and the names of the chapters are set to align, by
% inspecting the number before.
%
% First, we put in the thumb, then we put in the text, though the
% numbers need special attention.
%    \begin{macrocode}
  \newcommand{\secname}[3]{%
    \put(0,-\value{thumbline}){\ovrthumb}
    \put(-\value{overviewskip},-\value{thumbline}){%
      {\Overviewfont
%    \end{macrocode}
% If the first number is zero, the we don't want the number on the page.
%    \begin{macrocode}
        \ifnum#1=0
          #2\dotfill\ #3
        \else
%    \end{macrocode}
% Else, the counter is put on the page, according to thier meaning. 
%    \begin{macrocode}
          \setcounter{Overviewnumber}{#1}
          \ifAppendix
            \Alph{Overviewnumber}
          \else
            \arabic{Overviewnumber}
          \fi
%    \end{macrocode}
% Then a check to see if space is need in front of the number, so that
% the numbers will come out right adjusted.
%    \begin{macrocode}
          \ifnum#1>99
            \hspace{\thumbspace}
          \else
            \ifnum#1>9
              \hspace{2\thumbspace}
            \else
              \hspace{3\thumbspace}
            \fi
          \fi
%    \end{macrocode}
% Finally, the text and page number is put on the page. 
%    \begin{macrocode}
          #2\dotfill\ #3
        \fi
      }
    }
%    \end{macrocode}
% We need to increment the \texttt{line} counter, so that the next
% line isn't put on top of this line.  
%    \begin{macrocode}
    \stepcounter{line}
    \addtocounter{thumbline}{\value{thumbhskip}}
  }
%    \end{macrocode}
% \end{macro}
%
% \begin{macro}{\thumboverview}
% This is the command used in the previous defined
% \texttt{thumboverview} page style. It does nothing fancy, expect
% check to see if the file \Lcs{\jobname.ovr} exists. If it does, it
% inputs it. Othewise it outputs a warning.
%
%    \begin{macrocode}
  \newcommand{\thumboverview}{%
    \begin{picture}(0,0)
      \IfFileExists{\jobname.ovr}{%
        \noindent
        \input{\jobname.ovr}
      }
      {
        \PackageWarning{thumb}{I couldn't finde \jobname.ovr}
        \AtEndDocument{
          \typeout{``thumbs'' set, please run LaTeX again}
      }
      }
    \end{picture}
  }
%    \end{macrocode}
% \end{macro}
%
% If the Overview page is not used, the \texttt{thumbhskip} counter
% still needs to be set, so we set it in any case. This is not a
% problem, since the overview page \emph{always} comes after
% \Lcs{begin\{document\}}.  
%    \begin{macrocode}
  \newcounter{thumbhskip}
  \AtBeginDocument{
    \setcounter{thumbhskip}{\value{thumbheight}}
  }
%    \end{macrocode}
% \begin{macro}{\OverviewPage}
% This is the macro to be used in the document. It typesets the
% ``Overview'' page where it is given. It is \emph{always} set on a
% right--handed page. This look best, I think (and makes the coding
% easier). Notice, the empty \Lcs{mbox} at the end is
% nessecary. Oterwise \LaTeX{} think you have asked for 3 or 4 empty
% pages, and will reduce these to 1 or zero respectivly. Figure that
% out! Also the file \textsl{jobname}\texttt{.ovr} is opened and ready
% to be written in. At the end of the document, we close it.
%
% First some counters and a new \textsl{if}, so that we may check in
% the \Lcs{chapter} macro, wether we should write to the file
% \Lcs{jobname.ovr}.
%    \begin{macrocode}
  \newcounter{Overviewnumber}
  \newcounter{line}
  \newcounter{thumbline}
  \newif\ifOverview
  \Overviewfalse
%    \end{macrocode}
% Now to the macro it self. First \Lcs{ifOverview} should test
% \textsl{true}, and the counter \texttt{thumbhskip} is reset.
% Then we open the file \Lcs{jobname.ovr} for writting, and start a
% new righthanded page. The page style is what actually does all the
% work. This is defined earlier.
%    \begin{macrocode}
  \newcommand{\OverviewPage}{%
    \Overviewtrue
    \setcounter{thumbhskip}{0}
    \newwrite\thumbovr  
    \cleardoublepage
    \thispagestyle{thumboverview}
    \ifPreface
    \else
      \newcommand{\Overviewpage}{\thepage}
    \fi
%    \end{macrocode}
% Now, we put a small paragraph containg the titlepage stuff, next to
% the overview index. The \Lcs{vfill}'s is to ensure the page will
% appear non--empty, and create appropiate vertical space.
%    \begin{macrocode}
    \vfill  
    \usebox{\Overviewtitle}
    \vfill
%    \end{document}
% The file \Lcs{jobname.ovr} is closed at the end of the document.
%    \begin{macrocode}
    \AtEndDocument{
      \closeout\thumbovr
      \typeout{(\jobname.ovr)}
    }
%    \end{macrocode}
% Then a new right--handed page is started, and we open the file
% \Lcs{jobname.ovr} for writting.
%    \begin{macrocode}
    \cleardoublepage
    \immediate\openout\thumbovr=\jobname.ovr
%    \end{macrocode}
% If the file \Lcs{jobname.ovr} wasn't found, then we do not know the
% number of chapters in the document, so we guess it to be less then
% 10 (I think thats a reasonable number --- you may differ). Also, we
% ask the user to run \LaTeX{} again, so that the thumbs may come
% out right.
%    \begin{macrocode}
    \newdimen\tempa
    \tempa\textheight
    \ifnum\value{line}>0
      \divide\tempa\value{line}
      \thumbtmp\value{thumbhskip}sp
      \multiply\thumbtmp\value{line}
%    \end{macrocode}
% Next we test to see if there are too many thumbs, or if they are to
% high, if so right an error, and prompt.
%    \begin{macrocode}
      \ifdim\thumbtmp>\textheight
         \PackageError{thumb}{The total height of the thumbs is larger
            then the text}{Go on now, but correct the mistake afterwards}
      \fi
    \else
      \divide\tempa10
      \AtEndDocument{
        \typeout{``thumbs'' not set right, please run LaTeX again}
      }
    \fi
%    \end{macrocode}
% Now we that we have the (approximate) value of \texttt{thumbhskip},
% we output it to the file \Lcs{jobname.ovr}.
%    \begin{macrocode}
    \newcounter{thumbtmp}
    \setcounter{thumbtmp}{\tempa}
    \immediate\write\thumbovr{\noexpand\setcounter{thumbhskip}{\thethumbtmp}}
    \def\ovrout{
      \noexpand\secname{0}{\noexpand\Overviewfont\noexpand\Overviewname}
      {\noexpand\Overviewpage}
    }
    \immediate\write\thumbovr{\ovrout}
  }
%    \end{macrocode}
% \end{macro}
%
% \subsection{Ending}
%
% Finally, the errormessage if \texttt{fancyhdr} wasn't found.
%    \begin{macrocode}
\else
  \PackageWarning{thumb}{style `fancyhdr� not avaliable \return
    style `thumb' will not be used}
\fi
%    \end{macrocode}
% \iffalse
%</package>
% \fi
%
% \Finale
% \PrintIndex 
% \PrintChanges
\endinput
