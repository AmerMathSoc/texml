% \iffalse meta-comment
%
% Copyright (C) 2005-10 by Simon Dreher
%
% This work may be distributed and/or modified under the
% conditions of the LaTeX Project Public License, either
% version 1.3 of this license or (at your option) any later
% version. The latest version of this license is in:
%
%    http://www.latex-project.org/lppl.txt
%
% and version 1.3 or later is part of all distributions of
% LaTeX version 2005/12/01 or later.
%
% This work has the LPPL maintenance status `maintained'.
% The Current Maintainer of this work is Simon Dreher.
% This work consists of all files listed in README.
%
% \fi
%
% \iffalse
%<*driver>
\ProvidesFile{IMTEKda.dtx}
%</driver>
%<class|template>\NeedsTeXFormat{LaTeX2e}[2005/12/01]
%<class>\ProvidesClass{IMTEKda}
%<*class>
   [2010/01/01 v1.7 IMTEK-Diplomarbeitsvorlage]
%</class>
%<*template>
%%    2010/01/01 v1.7 IMTEK-Diplomarbeitsvorlage
%% Template fuer Diplom-, Bachelor- und Masterarbeiten
%% am IMTEK (c) Simon Dreher
%% Verbesserungsvorschlaege bitte an dreher@imtek.de

%%%%%%%%%%%%%%%%%%%%%%%%%%%%%%%%%%%%%%%%%%%%%%%%%%%%%%%%%%
%%%%%%%%%% Bitte vor dem Veraendern umbenennen! %%%%%%%%%%
%%%%%%%%%%%%%%%%%%%%%%%%%%%%%%%%%%%%%%%%%%%%%%%%%%%%%%%%%%
%
%</template>
%
%<*driver>
\documentclass[a4paper]{ltxdoc}
\usepackage[ngerman]{babel}
\addto{\captionsngerman}{\renewcommand*{\glossaryname}{Change History}}
\usepackage{calc}
\usepackage{units}
\usepackage{longtable}
  \setlength{\LTpre}{\lineskip}
  \setlength{\LTpost}{\lineskip}
\usepackage{gensymb,textcomp}
\usepackage{amsmath}
\usepackage[T1]{fontenc}
\usepackage{lmodern}
\usepackage{hypdoc}
\hypersetup{%
  pdfstartview=FitH%
}
\DeclareRobustCommand{\KOMAScript}{\textsf{K\kern.05em O\kern.05em%
      M\kern.05em A\kern.1em-\kern.1em Script}}
\makeatletter
  \def\DescribePackage#1{\leavevmode\@bsphack
    \marginpar{\raggedleft\strut \HD@target \small\sffamily #1\ }%
    \index{#1\actualchar{\protect\sffamily#1} (package)%
      \encapchar hdclindex{\the\c@HD@hypercount}{usage}}%
    \index{packages:\levelchar#1\actualchar{\protect\sffamily#1}\encapchar
      hdclindex{\the\c@HD@hypercount}{usage}}%
    \@esphack\ignorespaces}
  \def\DescribeOption#1{\leavevmode\@bsphack\begingroup\endgroup%
    \marginpar{\raggedleft\strut \HD@target \small\ttfamily #1\ }%
    \index{#1\actualchar{\protect\ttfamily#1} (option)%
      \encapchar hdclindex{\the\c@HD@hypercount}{usage}}%
    \index{options:\levelchar#1\actualchar{\protect\ttfamily#1}\encapchar
      hdclindex{\the\c@HD@hypercount}{usage}}%
    \@esphack\ignorespaces}
  \def\DescribeProgramm#1{\leavevmode\@bsphack\begingroup\endgroup%
    \marginpar{\raggedleft\strut \HD@target \small #1\ }%
    \index{#1\actualchar{\protect\ttfamily#1} (programm)%
      \encapchar hdclindex{\the\c@HD@hypercount}{usage}}%
    \index{programm:\levelchar#1\actualchar{\protect\ttfamily#1}\encapchar
      hdclindex{\the\c@HD@hypercount}{usage}}%
    \@esphack\ignorespaces}
  \def\DescribeElement#1{\leavevmode\@bsphack\begingroup\endgroup%
    \marginpar{\raggedleft\strut \HD@target \small\ttfamily #1\ }%
    \index{#1\actualchar{\protect\ttfamily#1}%
      \encapchar hdclindex{\the\c@HD@hypercount}{usage}}%
    \@esphack\ignorespaces}
  \def\DescribeProgrammMacro#1#2{\leavevmode\@bsphack\begingroup\endgroup%
    \marginpar{\raggedleft\strut \HD@target \small #2\ }%
    \index{#1\actualchar{\protect #2} (programm)%
       \encapchar hdclindex{\the\c@HD@hypercount}{usage}}%
    \index{programm:\levelchar#1\actualchar{\protect #2}\encapchar
       hdclindex{\the\c@HD@hypercount}{usage}}%
    \@esphack\ignorespaces}
\makeatother
\index{environments}
\EnableCrossrefs
\CodelineIndex
\RecordChanges
%\OnlyDescription
\begin{document}
  \DocInput{IMTEKda.dtx}
\end{document}
%</driver>
% \fi
%
% \CheckSum{897}
%
% \CharacterTable
% {Upper-case \A\B\C\D\E\F\G\H\I\J\K\L\M\N\O\P\Q\R\S\T\U\V\W\X\Y\Z
% Lower-case \a\b\c\d\e\f\g\h\i\j\k\l\m\n\o\p\q\r\s\t\u\v\w\x\y\z
% Digits \0\1\2\3\4\5\6\7\8\9
% Exclamation \! Double quote \" Hash (number) \#
% Dollar \$ Percent \% Ampersand \&
% Acute accent \' Left paren \( Right paren \)
% Asterisk \* Plus \+ Comma \,
% Minus \- Point \. Solidus \/
% Colon \: Semicolon \; Less than \<
% Equals \= Greater than \> Question mark \?
% Commercial at \@ Left bracket \[ Backslash \\
% Right bracket \] Circumflex \^ Underscore \_
% Grave accent \` Left brace \{ Vertical bar \|
% Right brace \} Tilde \~}
%
% \changes{v0.9}{2005/09/30}{Initial version}
% \changes{v1.0}{2005/11/30}{Bugfix in \texttt{gerplainnat.bst} (in entry type article)}
% \changes{v1.0}{2005/11/30}{Bugfix in \texttt{diplarb.bib} (in field language)}
% \changes{v1.0}{2005/11/30}{In diplarb.tex package \textsf{showkeys} commented out}
% \changes{v1.0}{2005/11/30}{Options \texttt{cleardoubleplain} and \texttt{cleardoubleempty} described}
% \changes{v1.1}{2006/03/04}{Bibliography extended}
% \changes{v1.1}{2006/03/04}{Bugfix: orthographic error}
% \changes{v1.1}{2006/03/04}{Bugfix: typing error in option \texttt{liststotoc}}
% \changes{v1.1}{2006/03/04}{\textsf{subfigure} changed to \textsf{subfig}}
% \changes{v1.1}{2006/03/04}{Documentation: new packages described}
% \changes{v1.2}{2006/04/22}{Bugfix: orthographic errors}
% \changes{v1.2}{2006/04/22}{English examples added}
% \changes{v1.2}{2006/04/22}{\textsf{babelbib} mentioned}
% \changes{v1.3}{2006/07/13}{pdfbookmarks added}
% \changes{v1.4}{2007/01/03}{New structure of documentation}
% \changes{v1.4}{2007/01/03}{Documentation extended}
% \changes{v1.4}{2007/01/03}{Packed as \texttt{.dtx}}
% \changes{v1.4c}{2007/06/28}{Bugfix: Documentation error in \texttt{IMTEKda.ins} solved}
% \changes{v1.4c}{2007/06/28}{Bugfix: orthographic error}
% \changes{v1.5}{2008/02/10}{Bachelor/Master added}
%
% \GetFileInfo{IMTEKda.dtx}
%
% \DoNotIndex{\#,\",\$,\%,\&,\@,\\,\{,\},\^,\_,\~,\ ,\S,\ss}
% \DoNotIndex{\@ne,\@ehc,\@fnsymbol,\@empty,\@latex@error,\z@,\null}
% \DoNotIndex{\advance,\begingroup,\catcode,\closein,\relax,\rlap,\par}
% \DoNotIndex{\day,\def,\edef,\gdef,\else,\empty,\endgroup,\ifx,\fi,\let}
% \DoNotIndex{\begin,\end,\bfseries,\centering,\Large,\large,\hfill,\vfill}
% \DoNotIndex{\@makefnmark,\@oldmakefnmark,\@mkboth,\@plus,\@thanks,\@title}
% \DoNotIndex{\@titlehead,\@uppertitleback,\@author,\@date,\@dedication}
% \DoNotIndex{\@lowertitleback,\@publishers,\@subject,\addchap,\addto}
% \DoNotIndex{\addtolength,\AfterPackage,\and,\c@footnote,\chapter,\global}
% \DoNotIndex{\cleardoublepage,\clearpage,\CurrentOption,\lineskip,\huge}
% \DoNotIndex{\iflanguage,\linewidth,\lowertitleback,\renewcommand,\newcommand}
% \DoNotIndex{\uppertitleback,\LoadClass,\vfil,\small,\setlength,\setcounter}
% \DoNotIndex{\usepackage,\vskip,\today,\thefootnote,\thispagestyle,\titlefont}
% \DoNotIndex{\newenvironment,\newif,\newlength,\next@tpage,\noexpand,\noindent}
% \DoNotIndex{\PassOptionsToClass,\ProcessOptions,\publishers,\RequirePackage}
% \DoNotIndex{\selectlanguage,\tabcolsep,\textwidth,\thanks}
%
% \title{IMTEK-Diplomarbeitsvorlage\thanks{Dokumentation
% zu \textsf{IMTEKda}~\fileversion, \filedate.}}
% \author{Simon Dreher \\ \texttt{dreher@imtek.de}}
%
% \maketitle
%
% \begin{abstract}\noindent
% Diese Layoutvorlage f"ur Diplom-, Bachelor- und Masterarbeiten am Institut
% f"ur Mikrosystemtechnik (IMTEK) der Uni Freiburg wurde in Abstimmung mit
% mehreren Lehrst"uhlen erstellt, um den Studierenden bei der Erstellung ihrer
% Arbeit eine Hilfestellung zu geben. Das Layout ist als Empfehlung gedacht und
% stellt keine verbindliche Richtlinie dar.
% \end{abstract}
%
% \tableofcontents
%
% \section{Einleitung}
%
% Diese Vorlage basiert auf einer alten Diplomarbeitsvorlage von Jan Lienemann.
% Sie wurde an vielen Stellen aktualisiert und von \textsf{book} als
% Elternklasse auf \textsf{\mbox{scrbook}} umgestellt, um vor allem eine
% leichtere Verwaltung der Schriften und anderer Layoutparameter zu
% gew"ahrleisten. Da Diplomarbeiten h"aufig auch auf Englisch geschrieben
% werden, wurden auch hierf"ur Anpassungen eingef"uhrt. Ein wichtiger Teil, der
% neu mit in die Vorlage aufgenommen wurde, ist eine Sammlung vieler
% \LaTeX-Tipps und "~Tricks sowie empfohlener Pakete f"ur diverse h"aufiger
% auftretende Aufgabenstellungen.
%
% \section{Allgemeines zur Arbeit}
%
% Der Aufbau bzw.\ die Gliederung einer Diplomarbeit h"angt sehr von der
% bearbeiteten Thematik ab. Das Beispiel ist f"ur eine Arbeit gedacht, die einen
% experimentellen Teil enth"alt. Bei theoretischen Arbeiten oder
% Programmierungen kann ein anderer Aufbau der Arbeit als der im Template
% vorgestellte sinnvoll sein. Der Umfang einer Diplomarbeit betr"agt
% "ublicherweise 60 bis 100 Seiten inkl.\ Anhang. Viel mehr als 110 Seiten
% werden ungern gesehen, schlie"slich soll die Arbeit ja auch noch lesbar
% bleiben. Die Sprache ist bei den meisten Lehrst"uhlen Deutsch, oft kann die
% Arbeit aber auch auf Englisch verfasst werden.
%
% Ein wichtiger Hinweis zum \textbf{Ausdrucken} eines mit \LaTeX{} erzeugten
% pdf-Files aus dem Acrobat Reader: Im Drucken-Dialog sollte die Option "`Shrink
% oversized pages to paper size"' (bzw.\ ihr deutsches "Aquivalent) \emph{nicht}
% eingeschaltet sein, sonst stimmen die Schriftgr"o"se und der Satzspiegel nicht
% mehr.
%
% Allgemeine Notationen wie Normen oder Schriftarten sollten im Vorwort kurz
% erkl"art werden, falls sie f"ur ein schnelles Verst"andnis notwendig sind.
% Symbole und Abk"urzungen kommen in eine Nomenklatur, die hinter dem
% Inhaltsverzeichnis steht.
%
% Am Ende der Zusammenfassung m"ussen Stichw"orter zu den wichtigsten
% Themengebieten, in die die Arbeit passt, angegeben werden. Diese Stichw"orter
% sollten ebenso (sowohl auf Deutsch als auch auf Englisch) als pdfkeywords
% angegeben werden, siehe dazu auch Abschnitt~\ref{pdfkeywords}. 
%
% Bei der Verwendung von Gliederungsebenen gibt es folgendes zu beachten:
% \begin{itemize}
% \item Es sollten nicht mehr als 3 Gliederungstiefen nummeriert werden.
% \item Unterkapitel sind nur dann sinnvoll, wenn es auch mehrere
% Untergliederungen gibt. Ein Unterkapitel 2.1.1 sollte somit nur dann
% verwendet werden, wenn es auch 2.1.2 gibt.
% \end{itemize}
%
% \changes{v1.4b}{2007/03/13}{Error in equation referencing corrected}
% Wichtige Gleichungen, die in der Arbeit h"aufiger zitiert werden, sollten eine
% automatische Gleichungsnummer erhalten. Die Nummerierung beginnt
% in jedem Kapitel neu und enth"alt die Kapitelnummer, dieses Standardverhalten
% sollte beibehalten werden. Beim Zitieren von Gleichungen steht die
% Formelnummer nicht in Klammern, falls sie zusammen mit der Bezeichnung
% "`Gleichung"' verwendet wird (also z.\,B.\ Gleichung~\ref{eq:sample}). Wird in
% Herleitungen ein Querverweis auf eine andere Formel gesetzt, so wird hier nur
% die Formelnummer gesetzt, hier allerdings in Klammern. Bei Verwendung des
% Pakets \textsf{amsmath} geht dies am einfachsten, indem zum Referenzieren
% \DescribeMacro{\eqref}
% |\eqref|\marg{label} verwendet wird.
% Ein Beispiel ist
% \begin{align}
% a + b &= c\label{eq:sample}\\
% b + c &= a + 2 b \qquad\text{aus \eqref{eq:sample}}
% \end{align}
%
% \section{Bedienung der Klasse \textsf{IMTEKda}}
% 
% Die Klasse \textsf{IMTEKda} basiert auf der Klasse \textsf{scrbook} aus
% \KOMAScript. Au"ser diesem Paket m"ussen die Pakete \DescribePackage{graphicx}
% \textsf{graphicx}, \DescribePackage{textpos} \textsf{textpos} und
% \DescribePackage{calc} \textsf{calc} sowie ein Paket zur
% Sprachumschaltung (vorzugsweise \textsf{babel}) verf"ugbar sein.
% Da das Paket \textsf{textpos} mit der Option |absolute| f"ur die
% Positionierung des neuen Uni-Logos ben"otigt wird, kann dieses Paket leider
% nicht mehr f"ur die Positionierung relativ zur aktuellen Position genutzt
% werden.
%
% Als Klassenoptionen k"onnen alle Optionen von \textsf{scrbook} verwendet
% werden \cite{scrguide}. Die Optionen \DescribeOption{titlepage} |titlepage|
% und \DescribeOption{notitlepage} |notitlepage| sind jedoch nicht sinnvoll und
% werden daher ignoriert.
% \changes{v1.6}{2009/01/07}{Documentation: class options \texttt{a4paper} and \texttt{pagesize} already loaded in class}
% Ebenso sind die Optionen zum Papierformat \DescribeOption{a4paper} |a4paper|
% und \DescribeOption{pagesize} |pagesize| nicht notwendig, da diese
% automatisch geladen werden.
%
% Da die Arbeit ja irgendwie gebunden wird, muss in den Klassenoptionen
% ein Wert f"ur die Bindekorrektur \DescribeOption{BCOR} |BCOR| angegeben
% werden, der dem Seitenrand in der Mitte entspricht, der in der Bindung
% verschwindet \cite{scrguide}. Der im Template angegebene Wert von
% \unit[10]{mm} ist eine erste grobe Absch"atzung f"ur die "ublichen
% Klebebindungen.
%
% Der Typ der Arbeit wird mit einer der Klassenoptionen  \DescribeOption{diplom}
% |diplom|,  \DescribeOption{bachelor}|bachelor| und  \DescribeOption{master}
% |master| angegeben. Entsprechend wird die Titelei angepasst, d.\,h.\ die
% Bezeichnung der Arbeit und die notwendigen Angaben im offiziellen Vorspann
% werden entsprechend gesetzt.
%
% Seit der Umstellung auf das neue Corporate Design 2010 wird mit der Option
% \DescribeOption{oldcd}|oldcd| auf das bisherige Layout umgestellt.
% \emph{Diese Option ist nur f"ur die Kompatibilit"at zu alten Arbeiten gedacht
% und soll keinesfalls bei neuen Bachelor- oder Masterarbeiten verwendet
% werden!}
%
% \subsection{Titelei}
% 
% Die Titelei wird standardm"a"sig in der Sprache der Arbeit gesetzt (also
% Englisch oder Deutsch), mit der Klassenoption
% \DescribeOption{noenglishpreamble}|noenglishpreamble| kann jedoch auch bei
% einer auf Englisch geschriebenen Arbeit die Titelei auf Deutsch gesetzt
% werden. Der Vollst"andigkeit halber ist auch eine Option
% \DescribeOption{englishpreamble} |englishpreamble| definiert, die eine
% englische Titelei einstellt (auch bei deutschem Text, was allerdings nicht
% sinnvoll ist).
%
% Wie in den Standardklassen werden mit \DescribeMacro{\author}
% |\author|\marg{Name} und \DescribeMacro{\title} |\title|\marg{Titel} Autor und
% Titel festgelegt. Neu definiert wurden\\ \DescribeMacro{\dpoversion}
% |\dpoversion|\marg{DPO} f"ur die g"ultige DPO,\\ \DescribeMacro{\chair}
% |\chair|\marg{Lehrstuhl} f"ur den Lehrstuhl,\\
% \DescribeMacro{\referees} |\referees|\marg{Gutachter} f"ur den Gutachter,\\
% \DescribeMacro{\thesistime} |\thesistime|\marg{Bearbeitungszeitraum} f"ur den
% Bearbeitungszeitraum und \\
% \DescribeMacro{\supervisor} |\supervisor|\marg{Betreuer} f"ur den Betreuer.\\
% Diese m"ussen vor dem
% eigentlichen Setzen der Titelei mit \DescribeMacro{\maketitle} |\maketitle|
% festgelegt sein. Um eine korrekte Bearbeitung von Sonderzeichen in Namen zu
% erreichen, sollten diese Befehle nicht in der Pr"aambel stehen, sondern erst
% nach |\begin{document}|.
%
% Der Befehl \DescribeMacro{\extratitle} |\extratitle| aus \KOMAScript{} ist
% f"ur eine Diplom"~, Bachelor"~ oder Masterarbeit nicht sinnvoll, sodass dieser
% Befehl wirkungslos bleibt. Von den Elementen der Titelseite, die in der Klasse
% \textsf{scrbook} vordefiniert sind \cite{scrguide}, sind
% \DescribeMacro{\titlehead} |\titlehead| und
% \DescribeMacro{\subject} |\subject| bereits definiert und d"urfen nicht neu
% definiert werden. Alle hier nicht genannten Elemente aus \textsf{scrbook}
% k"onnen auch verwendet werden.
% Eine ausf"uhrliche Beschreibung aller M"oglichkeiten und
% Hintergrundinformationen befinden sich im |scrguide.pdf| \cite{scrguide}.
%
% Ein Titelbild wird mit \DescribeMacro{\titlepic} |\titlepic|\marg{Bild}
% gesetzt. Dabei muss f"ur \meta{Bild} der \LaTeX-Code zum Einbinden (oder
% Zeichnen) des Bilds eingesetzt werden, nicht nur ein Dateiname. Die
% Bildbeschreibung wird mit \DescribeMacro{\titlepicdesc}
% |\titlepicdesc|\marg{Bild} deklariert.
%
% \subsection{Zusammenfassung und Verzeichnisse}
%
% Als neue Umgebung wurde \DescribeEnv{abstract} |abstract| enstsprechend der
% |abstract|"=Umgebung der \textsf{article}"=Klasse definiert. Diese Umgebung
% sollte doppelt verwendet werden, um zweisprachige Zusammenfassungen zu
% erhalten. Bei einer der beiden Umgebungen muss dabei die Sprache auf die
% Alternativsprache (Englisch oder Deutsch) umgeschaltet werden. Falls das
% Paket \textsf{hyperref} verwendet wird, sollte \DescribeMacro{\pdfbookmark}
% |\pdfbookmark{\abstractname}{abstract}| vor der ersten |abstract|"=Umgebung
% eingef"ugt werden, um ein PDF"=Lesezeichen zu erstellen.
%
% Nach den Zusammenfassungen sollte mit \DescribeMacro{\tableofcontents}
% |\tableofcontents| das Inhaltsverzeichnis gesetzt werden. Damit auch
% dieses in den PDF"=Lesezeichen auftaucht, muss auch hier wieder
% |\cleardoublepage\pdfbookmark{\contentsname}{toc}| direkt davor eingef"ugt
% werden.
%
% Eine weitere neue Umgebung ist \DescribeEnv{nomenclature} |nomenclature|, in
% der die Nomenklatur gesetzt wird. Sie sollte nach dem Inhaltsverzeichnis
% gesetzt werden.
% Als neue Klassenoption wurde \DescribeOption{nomtotoc} |nomtotoc| definiert,
% die im Inhaltsverzeichnis einen Eintrag f"ur die Nomenklatur erstellt. Da
% diese immer zwischen Inhaltsverzeichnis und Tabellen"~ und
% Abbildungsverzeichnis
% steht, m"ussen diese Verzeichnisse nicht ins Inhaltsverzeichnis eingetragen
% werden. Abbildungs"~ und Tabellenverzeichnis werden mit
% \DescribeMacro{\listoffigures} |\listoffigures| und
% \DescribeMacro{\listoftables} |\listoftables| erstellt.
% Diese beiden Verzeichnisse k"onnen auch mit der Klassenoption
% \DescribeOption{liststotoc} |liststotoc| ins Inhaltsverzeichnis aufgenommen
% werden, was allerdings auch nicht notwendig ist.
%
% \section{Pakete und Befehle}
%
% In diesem Abschnitt habe ich L"osungen f"ur die h"aufigsten Probleme
% zusammengefasst und Tipps gesammelt. Er kann und soll allerdings nicht eine
% gute \LaTeX"=Einf"uhrung ersetzen. Eine ganz knappe Einf"uhrung zum vorneweg
% durchlesen ist \cite{poolmgr}, eine etwas ausf"uhrlichere Anleitung
% (allerdings ohne Formelsatz) ist \cite{beginlatex}. Die absoluten
% Standardwerke, von denen jeder mindestens eines durchgelesen haben sollte,
% sind \cite{lkurz} auf Deutsch, \cite{lshort} als etwas ausf"uhrliche
% englische Variante und \cite{latexauthors} als weiteres englisches
% Benutzerhandbuch. Weitere Literaturtipps sind in der Bibliographie angegeben.
%
% \changes{v1.4a}{2007/02/09}{\texttt{texdoc} explained}
% Die Dokumentation zu den einzelnen Paketen ist jeweils auch zu Rate zu ziehen.
% Man findet sie bei installierten Paketen, indem man auf einer Kommandozeile
% den Befehl |texdoc |\meta{Paketname} eingibt. In wenigen F"allen kann es
% n"otig sein, die Dokumentation manuell in der \TeX"=Installation zu suchen,
% falls sie nicht aus einem Dokument mit dem Paketnamen als Dateinamen besteht.
% Sie liegt dann aber in einem Verzeichnis mit dem Paketnamen.
%
% \subsection{Sprachenwahl}
% \DescribePackage{babel}
% In der Vorlage wird zur Sprachenwahl das \textsf{babel}"=Paket verwendet. Die
% Sprachenauswahl mu"s dabei global als Klassenoption gesetzt werden, damit
% einige sprachenabh"angige Definitionen in der Titelei richtig gesetzt werden.
% \DescribeOption{english} \DescribeOption{ngerman} Die entsprechenden Optionen
% sind |english| und |ngerman|\footnote{\textsf{IMTEKda} akzeptiert "ubrigens
% auch |german|, das auf die alte Rechtschreibung angepasst ist.}, wobei die
% letzte Sprachoption die Dokumentsprache angibt, Alternativsprachen werden
% davor angegeben.
% Falls die Diplomarbeit auf Englisch abgefasst werden soll, m"ussen somit die
% Klassenoptionen |english| und |ngerman| gegen"uber dem deutschen Beispiel im
% Template vertauscht werden. Die deutsche Zusammenfassung muss
% dann statt der englischen in |\begin{otherlanguage}{ngerman}| \dots
% |\end{otherlanguage}| eingeschlossen werden.
%
% Falls das \textsf{babel}"=Paket aus irgendeinem Grund nicht verwendet werden
% kann (wenn m"oglich, sollte allerdings unbedingt \textsf{babel} verwendet
% werden!), kann auch das Paket \textsf{ngerman} verwendet werden; die
% \textsf{babel}"=Anweisungen aus der Vorlage m"ussen dann durch die
% entsprechenden Anweisungen dieses Pakets ersetzt werden \cite{gerdoc}.
%
% \subsection{Schriften}
% \KOMAScript{} setzt die "Uberschriften standardm"a"sig in Sansserif, um
% leichtere "Uberschriften zu erhalten, was empfehlenswert ist. Wer trotzdem das
% Verhalten der Standardklassen wiederherstellen will, kann in die Pr"aambel die
% Zeile\\
% |\setkomafont{sectioning}{\normalfont\normalcolor\bfseries}|\\
% einf"ugen.
%
% Mit \DescribePackage{inputenc} |\usepackage[latin1]{inputenc}| k"onnen
% Umlaute in latin1"=codierten \TeX"=Dateien direkt eingegeben werden. Diese
% Codierung wird "ublicherweise von Texteditoren unter Windows und alten
% Versionen von Linux und MacOS verwendet. Bei aktuellen Linux-Distributionen
% und MacOS~X wird allerdings die UTF\,8-Codierung standardm"a"sig
% benutzt, so dass man hier entweder manuell im Editor die Zeichencodierung
% umstellen oder stattdessen |\usepackage[utf8]{inputenc}| verwenden muss.
%
% Das Paket \DescribePackage{fontenc} |\usepackage[T1]{fontenc}| benutzt die
% neuere Schriftcodierung T1 statt der OT1-Codierung. Damit werden einige Fehler
% bei Umlauten umgangen und es stehen erweiterte Zeichens"atze zur Verf"ugung.
% Da die Standardschrift ComputerModern nur in OT1-Codierung verf"ugbar ist (und
% damit Umlaute, Akzente und manche Sonderzeichen nur unzureichend
% unterst"utzt), schaltet |\usepackage[T1]{fontenc}| auf die EC-Schrift (eine
% erweiterte Version der ComputerModern) um. Diese gibt es aber leider nicht als
% type1-Schrift, d.\,h.\ sie wird als Pixelschrift in der Aufl"osung des
% Ausgabetreibers (z.\,B.\ dvips oder pdf\LaTeX, hier meist \unit[600]{dpi}) in
% die Ausgabedatei eingebunden. Damit ist die Bildschirmdarstellung von PDFs im
% AdobeReader ziemlich schlecht.\footnote{Der AdobeReader~7 scheint mittlerweile
% auch die Pixelschriften akzeptabel darzustellen, so dass diese Einschr"ankung
% weniger wichtig geworden ist.} Au"serdem sollte hier bei Schriftgr"o"sen "uber
% \unit[10]{pt} \DescribePackage{exscale} |\usepackage{exscale}| verwendet
% werden, um einige mathematische Zeichen zu skalieren, die hier fehlen.
%
% \DescribePackage{lmodern} Eine empfehlenswerte Variante der EC-Schrift, die
% als type1-Schrift vorliegt, ist die LatinModern, die mit
% |\usepackage{lmodern}| geladen wird.
%
% \DescribePackage{mathdesign} \DescribePackage{berasans}
% \DescribePackage{beramono}
% Eine weitere empfehlenswerte Schriftkombination ist die URW~Garamond mit der
% Bera als Sansserif"~ und Schreibmaschinenschrift. Man l"adt sie mit\\
% |\usepackage[garamond,sfscaled=false,ttscaled=false]{mathdesign}|\\
% |\usepackage[scaled=0.9]{berasans,beramono}|.
%
% \DescribePackage{mathptmx}\DescribePackage{helvet}\DescribePackage{courier}
% Die h"aufig verwendete Times ist f"ur die relativ breiten Seiten einer
% Diplomarbeit wenig geeignet. Da mittlerweile alle Schriften in PDFs
% eingebunden werden, bringt hier auch ihre Verwendung als Systemschrift
% keinerlei Vorteile mehr.
% Wer trotzdem als Schriftart unbedingt Times, Helvetica und Courier verwenden
% will, sollte sie mit\\
% |\usepackage{mathptmx}|\\
% |\usepackage[scaled=.92]{helvet}|\\
% |\usepackage{courier}|\\
% laden.
%
% \subsection{Literaturzitate und Bibliographie}
%
% F"ur wissenschaftliche Arbeiten ist es g"unstig, die Literaturzitate
% fortlaufend zu nummerieren und so anzugeben, dass die Literatur auch
% wiedergefunden werden kann. Bei dickeren B�chern sollte das entsprechende
% Kapitel oder die Seitenzahl angegeben werden, falls generell nur aus einem
% Kapitel zitiert wird im Literaturverzeichnis, sonst mit dem optionalen
% Argument von |\cite|.
%
% \DescribeProgrammMacro{BibTeX}{\BibTeX}
% F"ur die Bibliographie sollte eine \BibTeX"=Literaturdatenbank verwendet
% werden. Es bietet sich dabei an, die Literaturliste mit einem Programm zu 
% verwalten. Als sehr komfortabel hat sich das freie Javaprogramm
% \DescribeProgramm{JabRef} JabRef \cite{jabref} erwiesen. \BibTeX"=Zitate
% k"onnen damit von Literaturdatenbanken aus dem Internet direkt "ubernommen
% werden, ansonsten ist die Eingabe "uber selbsterkl"arende Eingabemasken
% m"oglich.
%
% \changes{v1.4b}{2007/03/15}{\BibTeX{} syntax explained}
% \changes{v1.6a}{2009/01/27}{\BibTeX{} syntax corrected}
% In der \BibTeX"=Datenbank sollten die Autoren als \meta{Nachname},
% \meta{Vornamen} oder \meta{von} \meta{Nachname}, \meta{Vornamen} oder
% \meta{von} \meta{Nachname}, \meta{Jr}, \meta{Vornamen} eingegeben
% werden. Mehrere Autorennamen werden durch |and| (auch f"ur deutsche Eintr"age)
% getrennt. Dies ist notwendig, damit \BibTeX{} erkennt, dass mehrere Autoren
% angegeben sind, das |and| wird ggf.\ je nach Zitier"~ und Verzeichnisstil
% ersetzt, z.\,B.\ durch ein Komma oder ein deutsches "`und"'.
%
% Die Bibliographie selbst wird mit dem Befehl \DescribeMacro{\bibliography}
% |\bibliography|\marg{Biblio} ausgegeben. \meta{biblio} steht darin f"ur den
% Dateinamen der Bibliographie ohne die Endung |.bib|. Wenn Referenzen aus
% verschieden \BibTeX"=Dateien verwendet werden sollen, kann hier auch eine
% mit Kommata getrennte Liste von Dateien stehen. Nach dem ersten \LaTeX"=Lauf
% muss zur eigentlichen Berechnung der ben"otigten Eintr"age \BibTeX{} als
% Hilfsprogramm aufgerufen werden (mit |bibtex |\meta{dokument}, wobei
% \meta{dokument} durch den Dateinamen der \LaTeX"=Hauptdatei ohne Endung zu
% ersetzen ist). Bei den meisten \LaTeX"=Entwicklungsumgebungen wie
% TechnicCenter (Windows) oder Kile (Linux) kann dies beim Erstellen des
% Dokuments automatisch mit ausgef"uhrt werden. Danach sind weitere ein bis zwei
% \LaTeX"=L"aufe notwendig, um das Verzeichnis korrekt einzubinden.
%
% Generell sollte \DescribePackage{natbib} \label{natbib}
% |\usepackage[comma,numbers,sort&compress]{natbib}| f"ur einen sch"onen und
% richtigen Zitierstil innerhalb des Textes benutzt werden. F"ur den Stil des
% Literaturverzeichnisses ist dann
% \DescribeMacro{\bibliographystyle} |\bibliographystyle{plainnat}| zust"andig.
% Dabei kann |plainnat| auch durch |abbrvnat| oder |unsrtnat| ersetzt
% werden, wenn z.\,B.\ Vornamen und Journalnamen abgek"urzt werden sollen oder
% die Eintr"age unsortiert, d.\,h.\ in der Reihenfolge der Zitate erscheinen
% sollen. \textsf{natbib} stellt zus"atzlich zu |\cite| noch weitere Befehle zur
% Verf"ugung, unter anderem |\citet| und |\citep|, die Zitate mit Autorenangabe
% im Textfluss erlauben. Sie sind in Kapitel~4 der Dokumentation zu
% \textsf{natbib} \cite{natbib} sehr sch"on beschrieben.
%
% \changes{v1.6a}{2009/01/27}{\textsf{natbib}\slash \textsf{babelbib} incompatibilites added}
% Es wird au"serdem empfohlen, das Paket \DescribePackage{babelbib}
% \textsf{babelbib} zu verwenden, um Eintr"age f"ur deutsche Literatur auch auf
% Deutsch zu formatieren -- standardm"a"sig setzt \LaTeX{} die Literaturliste
% unabh"angig von der Dokumentsprache auf Englisch (also auch den automatisch
% erzeugten Text wie and zwischen den Autoren, volume statt Band usw.). Dieses
% Paket benutzt ein \BibTeX-Feld |language|, das als Wert die
% \textsf{babel}"=Sprachbezeichnung f"ur den jeweiligen Literatureintrag hat
% (also |ngerman| oder |english|). 
% Um die Sprachanpassung nutzen zu k"onnen, muss der Stil des
% Literaturverzeichnisses mit |\bibliographystyle{babplain}| statt |plainnat|
% bzw.\ |bababbrv| oder |babunsrt| statt |abbrvnat| oder |unsrtnat| auf die
% angepassten Stile umge"andert werden. Leider werden mit diesen Stilen die
% zus"atzlichen Zitier"=Befehle aus \textsf{natbib} wie |\citet| und |\citep|
% nicht unterst�tzt, sondern nur der normale |\cite|"=Befehl.
%
% \DescribeProgramm{JabRef}
% Um ein Feld in JabRef f"ur die Sprachumschaltung mit \textsf{babelbib}
% einzuf"ugen, muss einmal unter \emph{Options $\rightarrow$ Set up general
% fields} bei \emph{General} am Schlu"s |;language| angef"ugt werden, dann ist
% dieses Feld unter dem Reiter \emph{General} im Editierfenster zu finden.
%
% \changes{v1.6a}{2009/01/27}{\textsf{biblatex} introduced}
% \DescribePackage{biblatex}
% Als Ersatz f"ur \textsf{natbib} und \textsf{babelbib} kann auch das recht
% neue Paket \textsf{biblatex} eingesetzt werden. Es bietet wie \textsf{natbib}
% viele Zitier"=M"oglichkeiten und l"asst sich sehr leicht konfigurieren.
% Die Zitierstile werden dabei durch \LaTeX{}"=Makros bestimmt, was eine
% Anpassung an eigenen W"unsche erheblich vereinfacht.
%
% \subsection{Schreib- und Korrekturhilfen}
%
% \DescribeOption{draft}
% Beim schnelleren Auffinden
% von zu langen Zeilen, die durch zu gro"se Tabellen,
% Bilder und Gleichungen oder durch schwer trennbare W"orter entstehen, hilft
% die Klassenoption |draft|, die diese Zeilenenden dick markiert. Mit ihr werden
% auch statt den Bildern nur Platzhalter eingef"ugt, was die Zeit zum Erstellen
% des Dokuments erheblich verk"urzt.
%
% Damit man Labels von Gleichungen, Bildern usw.\ im Probeausdruck angezeigt
% bekommt und nicht den gesamten Quelltext nach einem Gleichungslabel
% durchsuchen muss, um eine Referenz darauf zu erzeugen, ist das Paket
% \DescribePackage{showkeys} \textsf{showkeys} sehr hilfreich. F"ur die
% Endfassung braucht man nur die Klassenoption |final| angeben, die intern
% verwendeten Labels werden dann nicht mehr gedruckt.
%
% \DescribeMacro{\input} \DescribeMacro{\include} \DescribeMacro{\includeonly}
% Es ist sinnvoll, einzelne Teile der Arbeit in eigenen |.tex|"=Dateinen
% auszulagern. Solche Dateien k"onnen in der Haupt"=Datei mit
% |\input|\marg{Datei} eingef"ugt werden, wobei \meta{Datei} der Dateiname ohne
% Da\-tei\-endung |.tex| ist. Werden gesamte Kapitel in einzelne Dateien
% geschrieben, so sollten diese mit |\include|\marg{Datei} eingef"ugt werden. Im
% Gegensatz zu |\input|\marg{Datei} werden hier jedoch vor und hinter dem
% eingef"ugten Code neue Seiten begonnen, so dass hier ganze Kapitel in der
% Datei stehen sollten. Mit |\includeonly|\marg{Datei1,Datei2,\dots} kann bei so
% eingef"ugten Kapiteln die Ausgabe von ausgew"ahlten Kapiteln erzeugt werden.
% Wenn irgendwann vorher ein \LaTeX{}"=Lauf mit allen Kapiteln gelaufen ist,
% werden sogar Referenzen und Seitenzahlen wie im Gesamtdokument erhalten.
%
% \changes{v1.4e}{2007/12/31}{\textsf{import} mentioned}
% \label{import}
% Werden eingebundene Dateien in Unterordner verteilt und darin wieder
% Einzeldateien mit |\input| eingebunden, k"onnen mit den Befehlen aus dem Paket
% \DescribePackage{import} \textsf{import} auch Pfadangaben relativ zur
% aktuellen Datei statt zur Hauptdatei angegeben werden.
% Zum Erweitern des Suchpfads siehe auch Abschnitt~\ref{inputpath}
%
% \subsection{Text}
%
% \subsubsection{Links und PDF-Informationen}
%
% Um Referenzen, Inhaltsverzeichnis usw.\ in einer PDF-Ausgabe als Links zu
% erhalten, muss als eines der letzten Pakete \DescribePackage{hyperref}
% \textsf{hyperref} geladen werden. Dies funktioniert auch, wenn das PDF "uber
% PS aus einem DVI erzeugt wird.
% \changes{v1.5}{2008/01/21}{Reference on \textsf{hypernat} no longer needed}
%
% \changes{v1.4b}{2007/03/13}{Description of \cs{autoref} added}
% \DescribeMacro{\autoref}
% Zus"atzlich zum Befehl |\ref| ist in \textsf{hyperref} ein Befehl |\autoref|
% definiert. Dieser setzt vor die Nummer des referenzierten Elements seine
% Bezeichnung.
% Um auch hier die Abk"urzungen f"ur Abbildung und Tabelle zu verwenden, muss
% bei Verwendung von \textsf{babel}\\
% \DescribeMacro{\figureautorefname}
% |\addto{\extrasngerman}{\renewcommand*{\figureautorefname}{Abb.}}|\\
% \DescribeMacro{\tableautorefname}
% |\addto{\extrasngerman}{\renewcommand*{\tableautorefname}{Tab.}}|\\
% im Vorspann eingef"ugt werden
%
% \DescribeElement{pdftitle} \DescribeElement{pdfauthor}
% \DescribeElement{pdfsubject}
% Falls \textsf{hyperref} verwendet wird, werden |pdftitle|, |pdfauthor| und
% |pdfsubject| auf die entsprechenden Werte der Titelseite gesetzt und k"onnen
% nicht mehr anders belegt werden. \label{pdfkeywords}
% \DescribeElement{pdfkeywords} |pdfkeywords| k"onnen (und sollen) aber noch
% gesetzt werden.
%
% Werden die durch \textsf{hyperref} erzeugten Links bei Referenzen auf Bilder
% und Tabellen angeklickt, springt der PDF"=Viewer auf die Bild- oder
% Tabellenunterschrift (nicht auf das Bild oder die Tabelle selbst). Dieses
% Verhalten korrigiert das Paket \DescribePackage{hypcap} \textsf{hypcap}, das
% mit der Option |all| geladen werden sollte (also |\usepackage[all]{hypcap}|).
% Es stellt auch Befehle zur Verf"ugung, die ein manuelles Setzen der Anker f"ur
% von Hand platzierte Bilder (siehe Abschnitt~\ref{bilder}) erlauben.
%
% \DescribePackage{hypbmsec}
% Werden in "Uberschriften \LaTeX"=Befehle verwendet, so f"uhrt dies oft zu
% Fehlern oder unlesbaren Eintr"agen in den PDF"=Bookmarks. Hier hilft
% das Paket \textsf{hypbmsec}, indem es zus"atzlich zu dem optionalen Argument
% der Gliederungsbefehle ein weiteres optionales Argument erlaubt, mit dem der
% Text des PDF"=Bookmarks einzeln angegeben werden kann (auch mit direkter
% Eingabe von manchen Sonderzeichen).
%
% \changes{v1.5}{2008/02/21}{\textsf{url} removed, functionality in hyperref}
%
% \subsubsection{Seitenlayout}
%
% \DescribeOption{openany} \DescribeOption{openright}
% F"ur Probeausdrucke kann man getrost die Klassenoption
% |openany| verwenden (spart die leeren Seiten gegen"uber den Kapitelanf"angen
% ein), die endg"ultige Version muss aber ohne diese Option (bzw.\ mit
% |openright|) erstellt werden.
% \DescribeOption{cleardoubleplain} \DescribeOption{cleardoubleempty}
% Wen auf den leeren Seiten die Kolumnentitel st"oren,
% der kann mit den Klassenoptionen |cleardoubleplain| oder |cleardoubleempty|
% (aus \KOMAScript) diese unterdr"ucken. |cleardoubleplain| schaltet auf den
% eingef"ugten Seiten auf den Seitenstil |plain| um, |cleardoubleempty|
% auf den Seitenstil |empty|.
%
% \changes{v1.4f}{2008/01/10}{Section \ref{layoutkorrekturen} added}
% \subsubsection{Manuelle Layoutkorrekturen}
% \label{layoutkorrekturen}
%
% Gelegentlich kann es vorkommen, dass eine Seite um eine Zeile zu kurz ist, so
% dass an einem Kapitelende eine fast leere neue Seite angefangen wird.
% Wenn \emph{keine} anderen "Anderungen mehr gemacht werden, kann \emph{ganz zum
% Schluss} die zu kleine Seite mit \DescribeMacro{\enlargethispage}
% |\enlargethispage{1\baselineskip}| um eine Zeile (d.\,h.\ die L"ange
% |\baselineskip|) vergr"o\ss ert werden.
%
% Ebenso k"onnen manuelle Umbr"uche oder eine Seitenvergr"o\ss erung im
% Inhaltsverzeichnis oder einem der anderen Verzeichnisse notwendig werden. Die
% dazu notwendigen Befehle werden mit \DescribeMacro{\addtocontents}
% |\addtocontents|\marg{Dateiendung}\marg{Befehle} in die Verzeichnisse
% geschrieben.\footnote{Durch das verz"ogerte Schreiben der Hilfsdateien darf
% dieser Befehl nicht als erster Befehl in einer mit \cs{include} eingebundenen
% Datei oder direkt nach dem \cs{include} stehen, sondern muss stattdessen am
% Ende der letzten eingebundenen Datei sein.}
%
% \subsubsection{Anf"uhrungszeichen, Bindestriche etc.}
%
% Allgemeine Tipps zu Schreibweisen und wie man sie mit \LaTeX{} erreicht,
% findet man in \cite{neubauer} und \cite{lexikon}. F"ur die Schreibweisen im
% Deutschen lohnt sich~-- auch wenn man zur Sprachumstellung \textsf{babel}
% verwendet~-- ein Blick in \cite{gerdoc}.
%
% Die wichtigsten Schreibweisen hier im "Uberblick:
% \begin{longtable}[l]{@{}ll@{}}
% |"`| und |"'| & deutsche Anf"uhrungszeichen ("` und "', ohne Leerzeichen)\\
% |``| und |''| & englische Anf"uhrungszeichen (`` und '', ohne Leerzeichen)\\
% |'| & Apostroph (Auslassung von Buchstaben, sparsam verwenden!)\\
% |-| & Trenn"~, Bindestrich (nur im Textmodus definiert!)\\
% |-| & Minus (nur im Mathemodus definiert!)\\
% |--|& Gedankenstrich (mit Leerzeichen); von--bis (ohne Leerzeichen)\\
% |'| & Fu"s, Minute ($'$) (nur Mathemodus, ohne Abstand zur Zahl)\\
% |''| & Zoll, Sekunde ($''$) (nur Mathemodus, ohne Abstand zur Zahl)\\
% |\dots| & Auslassungspunkte
% \end{longtable}
%
% \changes{v1.4b}{2007/03/31}{\textsf{csquotes} mentioned}
% Wer sich um die richtigen Anf"uhrungszeichen keine Gedanken mehr machen will,
% kann auch das Paket \DescribePackage{csquotes} \textsf{csquotes} verwenden.
%
% \subsubsection{Leerstellen, Abs"atze, Zeilenumbr"uche etc.}
%
% Abs"atze im laufenden Text werden mit einer Leerzeile im Quelltext eingegeben.
% Alternativ kann stattdessen auch der Befehl
% \changes{v1.4f}{2008/01/12}{\cs{par} added to index}%
% \makeatletter\leavevmode\@bsphack\marginpar{\raggedleft\cs{par}}%
%              \SpecialUsageIndex{par}\@esphack\makeatother\ignorespaces%
% |\par|
% verwendet werden. Diese Befehle d"urfen auf keinen Fall mit dem Befehl
% \DescribeMacro{\\} |\\| verwechselt werden. Dieser f"ugt einen Zeilenumbruch
% ein, als optionales Argument kann dabei noch ein vertikaler Abstand eingegeben
% werden (also mit |\\|\oarg{Abstand}).
% Zus"atzlicher vertikaler Abstand nach einem Absatz (nur in Sonderf"allen zu
% verwenden!) kann mit \DescribeMacro{\bigskip} |\bigskip| eingef"ugt werden.
%
% Im Text werden verschiedene Leerzeichen h"aufig gebraucht:
% \begin{longtable}[l]%
%   {@{}p{\widthof{\texttt{xx}}}p{\textwidth-\widthof{\texttt{xx}}-2\tabcolsep}@{}}
% \verb*| | & "`normales"' Leerzeichen. Nach einem Punkt wird evtl.\ ein
% gr"o"serer Zwischenraum eingef"ugt, deshalb nach Abk"urzung die folgende
% Variante w"ahlen\\
% \verb*|\ | & "`normales"' Leerzeichen, das auf jeden Fall gesetzt wird (auch
% nach Befehlen und nach einem Punkt); Zeilenumbruch m"oglich\\
% |~| & gesch"utztes Leerzeichen (Zeilenumbruch wird danach unterdr"uckt,
% z.\,B.\ in |Abb.~\ref{fig:bild}|)\\
% |\,| & Spatium (halbes Leereichen), zwischen mehrteiligen Abk"urzungen wie in
% |z.\,B.| sowie als zus"atzlichen Zwischenraum in Formeln
% \end{longtable}
%
% \subsubsection{M"ogliche Trennstellen}
%
% Um der automatischen Silbentrennung zu helfen, ist es gelegentlich n"otig,
% zus"atzliche m"ogliche Trennstellen einzuf"ugen oder die Trennung eines Worts
% zu verhindern. F"ur h"aufig vorkommende W"orter, die nicht oder falsch
% getrennt werden, kann in der Pr"aambel mit \DescribeMacro{\hyphenation}
% |\hyphenation{Mi-kro-sys-tem,Mi-kro-sys-tem-tech-nik}| eine Liste mit
% expliziten Trennmustern hinterlegt werden. Im Text k"onnen mit folgenden
% Befehlen Trennschwierigkeiten beseitigt werden:
% \begin{longtable}[l]{@{}ll@{}}
% \rlap{\texttt{\textbackslash{}mbox}\marg{Wort} \hspace{2\tabcolsep}
% Unterdr"uckung aller Trennungen}\\
% |-| & Bindestrich, der andere Trennungen unterdr"uckt\\
% |"=| & Bindestrich, der andere Trennungen erlaubt:
% |Ruhr"=Universit"at|\\
% |"~| & Bindestrich, an dem nicht getrennt werden darf:
% |Ein"~ und Ausg"ange|\\[.5ex]
% |\-| & Trennm"oglichkeit, die andere Trennungen ausschlie"st:
% |Ur\-instinkt|\\
% |"-| & Trennm"oglichkeit, die andere Trennungen nicht ausschlie"st\\
% |""| & Trennm"oglichkeit, bei der kein Trennstrich ben"otigt wird:
% |(Ein"~)""G"ange|\\[.5ex]
% \texttt{\textquotedbl\textbar} & Aufl"osen einer Ligatur und
% Trennm"oglichkeit: \texttt{Auf\textquotedbl\textbar{}forderung}
% \end{longtable}
% Von diesen Befehlen sind im Englischen leider nur |\mbox|, |-| und |\-|
% erlaubt, alle anderen werden von \textsf{babel}, \textsf{german} oder
% \textsf{ngerman} nur f"ur die deutsche Sprache bereitgestellt.
%
% \subsubsection{Listen}
%
% \DescribePackage{paralist}
% Oft ist es sch"oner, Aufz"ahlungen als Text zu formulieren. Hierf"ur lohnt
% sich das Paket \textsf{paralist}, das au"serdem engere Listen"~ und
% Aufz"ahlungsvarianten bietet.
%
% \subsection{Einheiten, chemische Formeln, Mathematik etc.}
%
% Um verbesserte Mathematik-Umgebungen zu laden, muss das Paket
% \DescribePackage{amsmath} \textsf{amsmath} verwendet werden, mit dem Paket
% \DescribePackage{amssymb} \textsf{amssymb} werden weitere mathematische
% Zeichen geladen. Die Dokumentation zu \textsf{amsmath} ist \cite{amsldoc},
% sie sollte auf jeden Fall gelesen werden.
%
% \DescribeMacro{\text} \DescribePackage{gensymb} \DescribePackage{textcomp}
% Werden in Indizes Beschreibungen abgek"urzt (z.\,B.\ max oder in und out), so
% werden diese als Text (also nicht im Mathemodus) gesetzt. Dies erreicht man am
% einfachsten mit dem Befehl |\text|\marg{Text} aus dem \textsf{amsmath}"=Paket.
% Um einige Sonderzeichen wie \DescribeMacro{\micro} |\micro| (\micro{} in
% aufrechter Schreibweise, f"ur Einheiten), \DescribeMacro{\ohm} |\ohm| (\ohm{}
% f"ur Einheiten), \DescribeMacro{\celsius} |\celsius| 
% (\celsius{}) oder \DescribeMacro{\degree} |\degree| (\degree{}) im Text"~ und
% Mathemodus nutzen zu k"onnen, m"ussen die Pakete \textsf{gensymb} und
% \textsf{textcomp} verwendet werden.
%
% Einheiten sollten mit den Befehlen aus einem der Pakete
% \DescribePackage{units} \textsf{units} oder \DescribePackage{SIunits}
% \textsf{SIunits} gesetzt werden. Diese Pakete sorgen f"ur korrekte Abst"ande
% zwischen Zahl und Einheit und f"ur deren korrekte und einheitliche
% Formatierung, die sonst ziemlich umst"andlich w"aren. Falls \textsf{units}
% verwendet wird, sollte als \micro{} |\micro| aus dem Paket \textsf{gensymb}
% verwendet werden (s.\,o.).
%
% \DescribePackage{bpchem} Um chemische Summenformeln und Reaktionsgleichungen
% zu setzen, sind die Pakete \textsf{bpchem} oder \DescribePackage{chemsym}
% \textsf{chemsym} jeweils in Verbindung mit \DescribePackage{chemarr}
% \textsf{chemarr} oder das Paket \DescribePackage{mhchem} \textsf{mhchem} zu
% empfehlen.
%
% Wer Zahlen mit einem Komma statt einem Punkt als Dezimaltrenner schreiben
% will, sollte das Paket \DescribePackage{icomma} \textsf{icomma} verwenden.
% Dann muss aber im Mathemodus in Listen nach dem Komma jeweils ein Leerzeichen
% stehen.
%
% Eine sehr sch"one Dokumentation zu fast allem, was man im Mathe-Modus machen
% kann, ist \cite{vossmathmode}.
%
% Ein neues, nettes Paket mit abk"urzenden Schreibweisen einiger h"aufiger
% Mathe-Befehle (Klammern, partiellen Ableitungen, Grenzen usw.) ist
% \DescribePackage{commath} \textsf{commath}. Wahrscheinlich wird es noch
% erweitert, d.\,h.\ man sollte "ofters mal nach der aktuellen Version schauen.
%
% \subsection{Bilder und Tabellen sowie Programmcode etc.}
%
% \label{bilder}\DescribeEnv{figure} \DescribeEnv{table}
% Bilder und Tabellen sollten in den Gleitumgebungen |figure| bzw.\ |table|
% gesetzt werden. Eine gute Dokumentation "uber alles zum Thema Grafiken und
% Gleitumgebungen ist \cite{epslatex}.
% \DescribePackage{capt-of} \DescribePackage{nofloat}
% Sollen einzelne Bilder oder Tabellen ausnahmsweise nicht automatisch plaziert
% werden (beispielsweise Tabellen oder technische Zeichnungen im Anhang), so
% bieten die Pakete \textsf{capt-of} und ggf.\ auch
% \textsf{nofloat} Befehle, um nicht gleitende Objekte
% einzuf"ugen und vor allem Bild- oder Tabellenbeschreibungen einzugeben.
% \changes{v1.6}{2008/11/15}{\textsf{float}, \textsf{rotfloat}, \textsf{floatrow} and \textsf{rotating} added}
% \DescribePackage{float}\DescribePackage{rotfloat}\DescribePackage{floatrow}
% Umfangreichere Pakete sind \textsf{float} und \textsf{rotfloat} (Erweitertung
% von \textsf{float}, die auch gedrehte Querformat-Bilder erlaubt, wie mit
% \textsf{rotating}) sowie \textsf{floatrow} als weitere Erweiterung. Mit
% diesen Paketen k�nnen neben Bild- und Tabellenunterschriften von
% nicht-gleitenden Bildern und Tabellen (f�r den Anhang) auch
% Tabellenbeschriftungen global als �ber- oder Unterschriften formatiert
% werden, unabh�ngig von der Position des |\caption|-Befehls.
%
% \DescribePackage{rotating}\DescribeEnv{sidewaysfigure}\DescribeEnv{sidewaystable}
% Breite Bilder und Tabellen, die nur im Querformat
% auf der Seite Platz finden, k�nnen statt in normale |figure|-
% bzw.\ |table|-Umgebungen in |sidewaysfigure| bzw. |sidewaystable| gesetzt
% werden. Diese werden vom \textsf{rotating}-Paket bereitgestellt. Falls
% \textsf{rotfloat} oder \textsf{floatrow} verwendet werden, sind die Befehle
% ebenfalls definiert, \textsf{rotating} darf dann nicht mehr geladen werden.
% Statt der Kombination aus \textsf{float} und \textsf{rotating} muss das Paket
% \textsf{rotfloat} verwendet werden.
%
% \DescribeMacro{\textfraction}\DescribeMacro{\topfraction}
% \DescribeMacro{\bottomfraction}\DescribeMacro{\floatpagefraction}
% Falls \LaTeX{} bei Kapiteln mit vielen Bildern Probleme beim Platzieren der
% einzelnen Floats hat, k"onnen gro"sz"ugigere Vorschriften mit\\
% |\renewcommand{\textfraction}{0.15}|\\
% |\renewcommand{\topfraction}{0.85}|\\
% |\renewcommand{\bottomfraction}{0.70}|\\
% |\renewcommand{\floatpagefraction}{0.66}|\\
%  in der Pr"aambel erreicht werden.
% \DescribePackage{placeins}\DescribeMacro{\FloatBarrier}
% Um zu verhindern, dass ein Bild nach einer bestimmten Stelle
% gesetz wird, kann an der entsprechenden Stelle der Befehl |\FloatBarrier| aus
% dem Paket \textsf{placeins} geschrieben werden.
%
% \DescribeMacro{\caption}
% Beschreibungen zu Abbildungen und Tabellen stehen unter dem Bild und werden
% innerhalb der Gleitumgebung in |\caption|\marg{Beschriftung}. (Manche
% platzieren auch die Tabellenbeschreibungen oberhalb der Tabelle, daf"ur bitte
% die Klassenoption \DescribeOption{tablecaptionabove} |tablecaptionabove|
% verwenden!) Wenn das Wort Abbildung bzw.\ Tabelle am Satzanfang steht, wird es
% ausgeschrieben, innerhalb eines Satzes wird es "ublicherweise abgek"urzt.
% Wer die Bildunterschriften vom Text absetzen will, kann sie mit
% |\addtokomafont{caption}{\footnotesize}| in der Pr"aambel
% in einer kleineren Schriftgr"o"se setzen.
%
% \changes{v1.5}{2008/02/18}{Documentation: Acrobat optimized PDF}
% \DescribeElement{Acrobat: PDF optimiert}
% Wenn viele gro"se Bilder eingebunden werden, wird die Dateigr"o"se des
% erzeugten PDFs leicht riesig. F"ur eine elektronische Fassung der Arbeit kann
% das PDF deutlich kleiner gemacht werden, indem mit dem Adobe Acrobat bei
% "`Speichern unter"' als Dateiformat "`Adobe PDF"=Dateien, optimiert"'
% ausgew"ahlt wird. Unter "`Einstellungen\dots"' k"onnen dann verschiedene
% Optimierungen des fertigen PDFs eingestellt werden, z.\,B.\ ein Downsampling
% aller Bilder auf eine moderate Endauf"|l"osung (ergibt viel kleinere Dateien).
%
% \subsubsection{Bilder}
% Die Dateinamen eingef"ugter Bilder sollten ohne Endung angegeben werden. Dies
% erleichtert die Verwendung von pdf\LaTeX{} und \LaTeX{} mit derselben
% \TeX"=Datei.
% \DescribeMacro{\graphicspath}
% Sollen Bilder in Unterordnern des aktuellen Arbeitsverzeichnisses abgelegt
% werden, k"onnen diese mit |\graphicspath|\marg{Pfadliste} zum Suchpfad
% hinzugef"ugt werden. \meta{Pfadliste} besteht dabei aus einer
% Liste nochmals geklammerter Pfade, also z.\,B.\
% |\graphicspath{{figures/}{fotos/}}|\footnote{Achtung Mac-User: Bei MacOS
% m"ussen die Pfade ggf.\ mit den Mac"=typischen Verzeichnistrennern geschrieben
% werden.}. Besonders angenehm ist dieser Befehl bei Verwendung von GnuPlot mit
% dem Ausgabeterminal |epslatex|, da hier \TeX"=Dateien mit darin eingebundenen
% eps"=Bildern erzeugt werden, die nat"urlich keine Unterordner
% ber"ucksichtigen.
% \changes{v1.4e}{2007/12/31}{\cs{input@path} explained}
% \DescribeMacro{\input@path}
% \changes{v1.5}{2008/02/24}{Use of \cs{input@path} corrected}
% Ordner k"onnen auch zum allgemeinen \TeX"=Suchpfad hinzugef"ugt werden, indem
% \label{inputpath}\\
% |\makeatletter|\\
% |\newcommand*{\input@path}|\marg{Pfadliste}\\
% |\makeatother|\\
% in der Pr"aambel (vor dem Einbinden des \textsf{graphicx}"=Pakets, falls auch
% zum Grafik"=Pfad) eingef"ugt wird. F"ur Pfade relativ zur gerade eingebundenen
% Datei siehe auch Abschnitt~\ref{import}.
%
% \changes{v1.4d}{2007/07/03}{Definition of \cs{setgraphicsbaseline} explained}
% Sollen Bilder neben Text ausgerichtet werden, z.\,B. in einer Tabelle mit
% Beschreibungstext inder einen Spalte und den dazugeh"origen Bildern in der
% anderen, so richtet \LaTeX{} normalerweise die baselines des Textes und des
% Bildes aneinander aus (auch bei der Ausrichtung von parboxen oder minipages
% mit der Positionierungsangabe |t|). Dies kann man umgehen, wenn man im
% Vorspann einen Befehl \DescribeMacro{\setgraphicsbaseline}
% |\setgraphicsbaseline| definiert durch\\
% |\newcommand{\setgraphicsbaseline}[1]{%|\\
% |  \raisebox{-\height+\heightof{M}}[\heightof{M}]%|\\
% |              [\totalheight-\heightof{M}]{#1}%|\\
% |}|\\
% Dabei muss das Paket \DescribePackage{calc} \textsf{calc} geladen sein, um die
% Berechnungen zu erm"oglichen.
% Mit |\setgraphicsbaseline|\marg{Bild} (\meta{Bild} steht hier f"ur den
% \LaTeX"=Code zum Einbinden oder Zeichnen des Bildes) wird dann die baseline
% des Bildes so angepasst, dass die Oberkante des Bildes mit der Oberkante des
% Buchstabens M zusammenf"allt.
%
% \DescribePackage{epstopdf}
% Wird pdf\LaTeX{} verwendet, Vektorgrafiken aber als EPS"=Bilder erzeugt, so
% k"onnen diese auch direkt beim pdf\LaTeX"=Lauf konvertiert werden. Dazu wird
% das Paket \textsf{epstopdf} eingebunden, wobei die Ausf"uhrung des
% Konvertierungsprogramms epstopdf per Kommandozeilenoption oder
% Konfigurationsdatei erlaubt werden muss. N"utzlich ist das Paket z.\,B.\ bei
% Verwendung des Plotprogramms GnuPlot mit dem Ausgabeterminal |epslatex|,
% allerdings m"ussen hier bei ver"anderten Plots die alten pdf"=Dateien
% gel"oscht werden.
%
% \changes{v1.6}{2008/11/12}{Description of obsolete \textsf{ps4pdf} changed to \mbox{\textsf{pst-pdf}}}
% \DescribePackage{psfrag} \DescribePackage{pst-pdf}
% Falls EPS-Grafiken aus anderen Programmen (z.\,B.\ OpenOfficeOrg Draw)
% eingebunden werden sollen, kann Beschriftung innerhalb der Bilder mit dem
% Paket \textsf{psfrag} angepasst werden. Um Postscript bzw.\ EPS auch mit
% pdf\LaTeX{} verwenden zu k"onnen, lohnt sich auch ein Blick auf das Paket
% \mbox{\textsf{pst-pdf}}.
%
% \DescribePackage{subfig}
% Falls mehrere kleine Bilder nebeneinander gesetzt oder verglichen werden
% sollen, ist das Paket \textsf{subfig} sehr empfehlenswert. Eventuell muss
% dabei mit der Paketoption |caption=false| die Formatierung der
% Bildunterschriften wieder der \KOMAScript"=Klasse (die intern von
% \textsf{IMTEKda} aufgerufen wird) "uberlassen werden.
%
% \changes{v1.4d}{2007/06/03}{Description of package \textsf{pdfpages} added}
% \DescribePackage{pdfpages}
% Sollen ganze Seiten aus anderen PDF-Dokumenten eingebunden werden,
% beispielsweise technische Zeichnungen im Anhang, so ist das Paket
% \textsf{pdfpages} zu empfehlen.\footnote{\textsf{pdfpages} funktioniert nur
% mit pdf\LaTeX.}
%
% \changes{v1.4e}{2007/07/17}{Documentation: pdftops etc. added}
% \DescribeProgramm{pdftops}
% Falls PDF-Dateien in EPS-Dateien umgewandelt werden sollen, kann daf"ur das
% Programm pdftops verwendet werden. Man ruft es auf mit\\
% |pdftops -paper match |\meta{Datei.pdf}\\
% \DescribeProgramm{epstopdf}
% Die umgekehrte Konvertierung kann mit |epstopdf |\meta{Datei.eps} erreicht
% werden.
% \changes{v1.5}{2008/02/02}{Documentation: epspdf added}
% \DescribeProgramm{epspdf}
% Konvertierungen in beiden Richtungen k"onnen auch mit dem Programm epspdf
% durchgef"uhrt werden. Es bietet viele Optionen beim Konvertieren wie
% z.\,B.\ das Entfernen wei"ser R"ander oder Konvertierung in Graustufen. Auch
% eine Variante mit grafischer Benutzeroberfl"ache, epspdftk, ist verf"ugbar.
%
% \subsubsection{Tabellen}
%
% \DescribePackage{booktabs} \DescribePackage{longtable}
% F"ur Tabellen innerhalb der Diplomarbeit sollte das Paket \textsf{booktabs}
% verwendet werden. Bei langen Tabellen, die umbrochen werden k"onnen
% (z.\,B.\ in der Nomenklatur), sollte das Paket \textsf{longtable} oder ein
% entsprechendes Paket verwendet werden.
%
% Das Paket \DescribePackage{array} \textsf{array} bietet weitere
% Spaltendefinitionen und bietet M"oglichkeiten, um \LaTeX"=Befehle vor oder
% nach jedem Tabellenfeld in einer bestimmten Spalte einzuf"ugen und so
% gezielt eine Spalte anders zu formatieren.
%
% Blocksatz in Tabellen wirkt oft unsch"on, da die Wortzwischenr"aume zu stark
% gedehnt werden m"ussen (was auch viele Warnungen erzeugt). In schmalen Spalten
% sollte daher linksb"undiger Flattersatz verwendet werden. Da das normale
% |\raggedright| Worttrennungen unterdr"uckt, sollte man hier das Paket
% \DescribePackage{ragged2e} \textsf{ragged2e} verwenden, das neue Kommandos
% f"ur Flattersatz bereitstellt. F"ur linksb"undigen Text ist dies
% \DescribeMacro{\RaggedRight} |\RaggedRight|. Dieser Befehl ist besonders in
% den Spaltenmodifikatoren des \textsf{array}"=Pakets zu empfehlen.
%
% Um Tabellen in Seitenbreite zu erstellen, kann das Paket
% \DescribePackage{tabularx} \textsf{tabularx} verwendet werden. Eine Anpassung
% dieses Pakets f"ur \textsf{longtable} ist \DescribePackage{ltxtable}
% \textsf{ltxtable}.
% \changes{v1.4d}{2007/07/03}{Package \textsf{calc} mentioned}
%
% Um die Breite einer Spalte in einer Tabelle in Seitenbreite
% \label{tabellenbreite}zu berechnen, kann die L"angenangabe auch als
% Berechnung angegeben werden, falls das Paket \DescribePackage{calc}
% \textsf{calc} geladen wurde, also z.\,B.\ mit\\
% |\begin{tabular}[t]{p{0.2\textwidth}p{0.8\textwidth-4\tabcolsep}}|\\
% \dots\\
% |\end{tabular}|\\
% wobei die Abst"ande zwischen den Spalten (je Spalte einer links und rechts)
% hier von der Breite der zweiten Spalte subtrahiert werden.
% Falls vor der Tabelle ein neuer Absatz begonnen hat, muss davor noch der
% Absatzeinzug mit |\noindent| entfernt werden.
%
% \subsubsection{Programmcode und "Ahnliches}
%
% \changes{v1.4a}{2007/02/09}{Description of package \textsf{listings} added}
% Zum Einf"ugen von Programmcode"=Ausz"ugen ist das Paket
% \DescribePackage{listings} \textsf{listings} sehr zu empfehlen. Es bietet
% Umgebungen, um Programmcode verbatim in den Text einzubinden. Dabei wird sogar
% Syntax Highlighting f"ur sehr viele Programmier"~ und Skriptsprachen
% unterst"utzt. Es gibt auch eine Funktion, um Programmcode direkt aus einer
% Quelldatei einzubinden (auch einzelne Teile daraus), so dass der Code nicht
% extra aufgearbeitet und in die \LaTeX"=Datei eingef"ugt werden muss.
%
% \newcommand*{\mybibliography}{
% \begin{thebibliography}{18}
% \bibitem{poolmgr}
% Sascha Frank.
% \newblock \emph{Erste Schritte mit \LaTeX}.
% \newblock Druckbares PDF:
%   \url{http://www.informatik.uni-freiburg.de/~frank/latex/handout-4-auf-1.pdf},
%   Bildschirmversion:
%   \url{http://www.informatik.uni-freiburg.de/~frank/latex/kurs.pdf}.
% \newblock Anm.: Kurze Anleitung der Poolmanager, gut f"ur die allerersten
%   Dokumente (\`a~la "`Hello World"').
% \bibitem{beginlatex}
% Peter Flynn.
% \newblock \emph{A beginner's introduction to typesetting with {\LaTeX}}.
% \newblock \url{http://www.ctan.org/tex-archive/info/beginlatex/beginlatex-3.6.pdf}.
% \newblock Anm.: \LaTeX-Grundlagen mit Anleitung zur Installation unter
%   verschiedenen Betriebssystemen; ohne Formelsatz.
% \bibitem{lkurz}
% Walter Schmidt, J"org Knappen, Hubert Partl, and Irene Hyna.
% \newblock \emph{l2kurz: \LaTeXe-Kurzbeschreibung}.
% \newblock \url{http://www.ctan.org/tex-archive/info/lshort/german/l2kurz2.pdf}.
% \newblock Anm.: Sehr gute Einsteigerlekt"ure f"ur Anf"anger, Pflichtlekt"ure!
% \bibitem{lshort}
% Tobias Oetiker, Hubert Partl, Irene Hyna, and Elisabeth Schlegl.
% \newblock \emph{lshort: The not so short introduction to {\LaTeXe}}.
% \newblock \url{http://www.ctan.org/tex-archive/info/lshort/english/lshort.pdf}.
% \newblock Anm.: Erweiterte, englische Version von \cite{lkurz}.
% \changes{v1.5}{2008/02/24}{{\LaTeXe} for authors mentioned}
% \bibitem{latexauthors}
% \LaTeX3 Project Team.
% \newblock \emph{{\LaTeXe} for authors}, Juli 2001.
% \newblock  \url{http://www.ctan.org/tex-archive/macros/latex/doc/usrguide.pdf}.
% \newblock Anm.: Weitere Standard"=Einsteigerlekt"ure, "ahnlich wie \cite{lkurz}.
% \bibitem{rrzn}
% Thomas~F. Sturm.
% \newblock \emph{{\LaTeX~-- Einf"uhrung in das Textsatzsystem}}.
% \newblock RRZN-Handbuch, 2006.
% \newblock Anm.: Das Handbuch kann beim Rechenzentrum gekauft werden.
%   Empfehlenswerte Einf"uhrung in \LaTeX{}, beschreibt auch ausgew"ahlte,
%   wichtige Zusatzpakete.
% \bibitem{companion}
% Frank Mittelbach, Michel Goossens, and Johannes Bahms.
% \newblock \emph{The \LaTeX{} companion}.
% \newblock Addison-Wesley, 2. Auflage, 2004.
% \newblock Deutsche Ausgabe:
% \newblock Frank Mittelbach, Michel Goossens.
% \newblock \emph{Der \LaTeX"=Begleiter}.
% \newblock Pearson Studium, 2.~"uberarbeitete und erweiterte Auflage, 2005.
% \newblock Anm.: Die \LaTeX-"`Bibel"', auch als "`Bernhardiner"' bekannt;
%   2.~Auflage hat sich sehr gegen"uber der 1.\ verbessert.
% \bibitem{scrguide}
% Markus Kohm und Jens-Uwe Morawski.
% \newblock \emph{{KOMA-Script -- Die Anleitung}}.
% \newblock \url{http://www.ctan.org/tex-archive/macros/latex/contrib/koma-script/scrguide.pdf}
% \newblock Als Buch erh"altlich unter:
% \newblock Markus Kohm und Jens-Uwe Morawski.
% \newblock \emph{{KOMA-Script -- Die Anleitung}}.
% \newblock DANTE~e.\,V., Lehmanns Fachbuchhandlung, 2., verbesserte Auflage, Mai
%   2005.
% \bibitem{epslatex}
% Keith Reckdahl.
% \newblock \emph{Using imported graphics in {\LaTeX} and {pdf\LaTeX}}.
% \newblock \url{http://www.ctan.org/tex-archive/info/epslatex.pdf}.
% \newblock Anm.: Alles zum Thema Bilder und Floats, \emph{das} Nachschlagewerk
% bei Problemen damit.
% \bibitem{natbib}
% Patrick Daly.
% \newblock \emph{Natural sciences citations and references}.
% \newblock \url{http://www.ctan.org/tex-archive/macros/latex/contrib/natbib/natbib.dvi}.
% \bibitem{gerdoc}
% Bernd Raichle.
% \newblock \emph{Kurzbeschreibung \texttt{german.sty} und \texttt{ngerman.sty}}.
% \newblock \url{http://www.ctan.org/tex-archive/language/german/gerdoc.dvi}.
% \bibitem{vossmathmode}
% Herbert Vo"s.
% \newblock \emph{Math mode}.
% \newblock \url{http://www.ctan.org/tex-archive/info/math/voss/mathmode/Mathmode.pdf}.
% \newblock Anm.: Alles zum Thema Formeln und Mathemodus.
% \bibitem{amsldoc}
% American Mathematical Society.
% \newblock \emph{User's guide for the \texttt{amsmath} package}.
% \newblock \url{http://www.ctan.org/tex-archive/macros/latex/required/amslatex/math/amsldoc.pdf}.
% \bibitem{neubauer}
% Marion Neubauer.
% \newblock \emph{Feinheiten bei wissenschaftlichen Publikationen --
%   Mikrotypographie"=Regeln, Teil I und II}.
% \newblock
%   \url{http://www.dante.de/dante/DTK/dtk96_4/dtk96_4_neubauer_feinheiten.html}
%   und
%   \url{http://www.dante.de/dante/DTK/dtk97_1/dtk97_1_neubauer_feinheiten.html}.
% \bibitem{lexikon}
% Eberhard Dilba.
% \newblock \emph{Orthotypographie oder Schreibweisen im Schriftsatz}.
% \newblock \url{http://eberhard-dilba.homepage.t-online.de/pdf-Dateien/Schreibweisen.pdf}.
% \bibitem{jabref}
% {JabRef}.
% \newblock \url{http://jabref.sourceforge.net}.
% \newblock Programm zur Verwaltung von Bibliographiedaten.
% \bibitem{faq}
% Bernd Raichle, Rolf Niepraschk, Thomas Hafner.
% \newblock \emph{DE-TeX-FAQ: Fragen und Antworten (FAQ) "uber das Textsatzsystem
%   \TeX{} und DANTE, Deutschsprachige Anwendervereinigung \TeX{} e.\,V.}
% \newblock \url{http://www.dante.de/faq/de-tex-faq}.
% \newblock Anm.: Vor Nachfragen unbedingt zu lesen!
% \changes{v1.5}{2008/02/16}{VisualFAQ mentioned}
% \bibitem{visualfaq}
% Scott Pakin.
% \newblock \emph{The visual FAQ}.
% \newblock \url{http://www.tex.ac.uk/tex-archive/info/visualFAQ/visualFAQ.pdf}.
% \newblock Anm.: Nettes PDF mit typischen Problemen, Formatierungen usw.\ zum
% Anklicken, f"uhrt auf die englische FAQ auf \url{http://www.tex.ac.uk/faq}.
% \bibitem{ltabu}
% Mark Trettin.
% \newblock \emph{l2tabu: Das \LaTeXe"=S"undenregister oder Veraltete Befehle,
%   Pakete und andere Fehler}.
% \newblock \url{http://www.ctan.org/tex-archive/info/l2tabu/german/l2tabu.pdf}.
% \newblock Anm.: F"ur alle, die schon etwas \LaTeX"=Erfahrung haben oder Tipps
%   bekommen haben; unbedingt durchzulesen!
% \bibitem{texnik}
% Herbert Vo"s.
% \newblock \emph{The {\TeX}nik web site}.
% \newblock \url{http://tug.org/TeXnik/mainFAQ.cgi}.
% \newblock Anm.: Nachschlagewerk und Suchm"oglichkeit bei h"aufigen Problemen.
% \bibitem{ctan}
% Graham Williams und J"urgen Fenn.
% \newblock \emph{The {\TeX} catalogue online}.
% \newblock \url{http://www.ctan.org/tex-archive/help/Catalogue/bytopic.html}.
% \newblock Anm.: Katalog zum Suchen von \LaTeX-Paketen.
% \bibitem{symbole}
% Scott Pakin.
% \newblock \emph{The comprehensive {\LaTeX} symbol list}.
% \newblock \url{http://www.ctan.org/tex-archive/info/symbols/comprehensive/symbols-a4.pdf}.
% \newblock Anm.: "Ubersicht "uber fast alle m"oglichen Symbole und die dazu
%   notwendigen Pakete.
% \end{thebibliography}}
%
% \StopEventually{\mybibliography\PrintIndex\PrintChanges}
%
% \iffalse
%<*class>
% \fi
%
% \section{Implementierung der Klasse}
%
% \subsection{Optionen und Initialisierungen}
%
% Zun"achst werden die Abfragen f"ur die Optionen |nomtotoc| und
% |englishpreamble| definiert und initialisiert sowie eine Abfrage, ob die
% Sprache f"ur die Pr"aambel festgelegt wurde.
%    \begin{macrocode}
\newif\if@nomtotoc\@nomtotocfalse
\newif\if@englishpreamble\@englishpreambletrue
\newif\if@preamblelangdef\@preamblelangdeffalse
%    \end{macrocode}
% Dasselbe f"ur die Sprachumschaltung per Klassenoption auf |german| bzw.\ 
% |ngerman|
%    \begin{macrocode}
\newif\if@germanopt\@germanoptfalse
\newif\if@ngermanopt\@ngermanoptfalse
%    \end{macrocode}
% \begin{macro}{nomtotoc}
% \begin{macro}{noenglishpreamble}
% \begin{macro}{englishpreamble}
% Die Optionen |nomtotoc|, |noenglishpreamble| und (nur der Vollst"andigkeit
% halber) |englishpreamble| sowie die Sprachoptionen |german| und |ngerman|
% werden definiert und schalten die Umschalter auf
% die entsprechenden Werte.
%    \begin{macrocode}
\DeclareOption{nomtotoc}{\@nomtotoctrue}
\DeclareOption{noenglishpreamble}%
  {\@preamblelangdeftrue\@englishpreamblefalse}
\DeclareOption{englishpreamble}%
  {\@preamblelangdeftrue\@englishpreambletrue}
\DeclareOption{german}%
  {\PassOptionsToClass{\CurrentOption}{scrbook}\@germanopttrue}
\DeclareOption{ngerman}%
  {\PassOptionsToClass{\CurrentOption}{scrbook}\@ngermanopttrue}
%    \end{macrocode}
% \end{macro}
% \end{macro}
% \end{macro}
% \changes{v1.5}{2008/01/19}{options |diplom|, |bachelor| and |master| added}
% Abfragen f"ur den Typ der Arbeit: Diplom, Bachelor oder Master sowie
% einer Abfrage, ob eine dieser Optionen gesetzt wurde
%    \begin{macrocode}
\newif\if@diplom\@diplomtrue
\newif\if@bachelor\@bachelorfalse
\newif\if@master\@masterfalse
\newif\ifh@snothesistype\h@snothesistypetrue
%    \end{macrocode}
% \begin{macro}{diplom}
% \begin{macro}{bachelor}
% \changes{v1.5}{2008/02/10}{Bachelor/Master added}
% \begin{macro}{master}
% \changes{v1.5}{2008/02/10}{Bachelor/Master added}
% Definition der Optionen |diplom|, |bachelor| und |master| und Setzen der
% dazugeh"origen Abfragen
%    \begin{macrocode}
\DeclareOption{diplom}%
  {\@diplomtrue\@bachelorfalse\@masterfalse\h@snothesistypefalse}
\DeclareOption{bachelor}%
  {\@diplomfalse\@bachelortrue\@masterfalse\h@snothesistypefalse}
\DeclareOption{master}%
  {\@diplomfalse\@bachelorfalse\@mastertrue\h@snothesistypefalse}
%    \end{macrocode}
% Falls keine dieser Optionen gesetzt ist, wird eine Warnung ausgegeben.
% \end{macro}
% \end{macro}
% \end{macro}
% \changes{v1.7}{2010/01/01}{Option \texttt{oldcd}}
% Noch eine Variable f"ur die Option |oldcd|, die auf das alte Corporate
% Design umschaltet
%    \begin{macrocode}
\newif\if@oldcd\@oldcdfalse
%    \end{macrocode}
% \begin{macro}{oldcd}
% Die Klassenoption |oldcd| schaltet auf das alte (bisherige) Layout um
%    \begin{macrocode}
\DeclareOption{oldcd}%
  {\@oldcdtrue}
%    \end{macrocode}
% \end{macro}
% Alle weiteren Optionen werden weitergegeben und als Basis-Klasse
% \textsf{scrbook} mit der Default-Option |a4paper| geladen. F"ur die
% Einbindung der Logos und Bilder wird \textsf{graphicx} ben"otigt, f"ur die
% Berechnung der Tabellenbreiten in der Titelei \textsf{calc}.
% \changes{v1.4d}{2007/07/03}{Package \textsf{calc} required}
% \changes{v1.6}{2008/11/13}{Load \textsf{graphics} only at \cs{begin\{document\}} to avoid option clashes}
% \changes{v1.6}{2009/01/07}{Load \textsf{scrbook} with option \texttt{pagesize}}
% \changes{v1.7}{2010/01/01}{\textsf{textpos} required}
%    \begin{macrocode}
\DeclareOption*{\PassOptionsToClass{\CurrentOption}{scrbook}}
\ProcessOptions\relax
\ifh@snothesistype
  \@latex@warning@no@line{%
    Eine der Optionen diplom, bachelor oder master
    \MessageBreak muss angegeben werden. Nehme diplom.%
  }%
\fi
\LoadClass[a4paper,pagesize]{scrbook}
\AtBeginDocument{\RequirePackage{graphicx}}
\RequirePackage{calc}
\RequirePackage[absolute]{textpos}
%    \end{macrocode}
%
% \subsection{Definition sprachenabh"angiger Begriffe}
%
% \begin{macro}{\nomname}
% Definition des Nomenklatur-Namens (auf Englisch)
%    \begin{macrocode}
\def\nomname{Nomenclature}
%    \end{macrocode}
% \end{macro}
% F"ur \textsf{babel}: Definition des deutschen Nomenklatur-Namens sowie der
% Kurzfassung von |\figurename| und |\tablename|
%    \begin{macrocode}
\AfterPackage*{babel}{
\iflanguage{german}{\@englishpreamblefalse}%
  {\if@preamblelangdef\else\@englishpreambletrue\fi}
    \addto{\captionsgerman}{\renewcommand*{\figurename}{Abb.}}
    \addto{\captionsgerman}{\renewcommand*{\tablename}{Tab.}}
    \addto{\captionsgerman}{\def\nomname{Nomenklatur}}
\iflanguage{ngerman}{\@englishpreamblefalse}%
  {\if@preamblelangdef\else\@englishpreambletrue\fi}
    \addto{\captionsngerman}{\renewcommand*{\figurename}{Abb.}}
    \addto{\captionsngerman}{\renewcommand*{\tablename}{Tab.}}
    \addto{\captionsngerman}{\def\nomname{Nomenklatur}}
}
%    \end{macrocode}
% Dasselbe f"ur die Pakete \textsf{german} und \textsf{ngerman}
%    \begin{macrocode}
\AfterPackage*{german}{
    \@englishpreamblefalse\@germanopttrue
    \renewcommand*{\figurename}{Abb.}
    \renewcommand*{\tablename}{Tab.}
    \newcommand*{\captionsgermansav}{}
    \let\captionsgermansav\captionsgerman
    \renewcommand*{\captionsgerman}%
        {\captionsgermansav\def\figurename{Abb.}%
        \def\tablename{Tab.}\def\nomname{Nomenklatur}}
}
\AfterPackage*{ngerman}{
    \@englishpreamblefalse\@ngermanopttrue
    \renewcommand*{\figurename}{Abb.}
    \renewcommand*{\tablename}{Tab.}
    \newcommand*{\captionsngermansav}{}
    \let\captionsngermansav\captionsngerman
    \renewcommand*{\captionsngerman}%
        {\captionsngermansav\def\figurename{Abb.}%
        \def\tablename{Tab.}\def\nomname{Nomenklatur}}
}
%    \end{macrocode}
%
% \subsection{Definition eigener Makros und Umgebungen}
%
% \begin{macro}{\titlepic}
% \begin{macro}{\titlepicdesc}
% Definition des Makros |\titlepic| f"ur ein Titelbild sowie |\titlepicdesc|
% f"ur die dazugeh"orige Bildbeschreibung. Ebenso Definition und Initialisierung
% der Abfragen, ob Titelbild und Beschreibung des Titelbilds existieren
%    \begin{macrocode}
\newif\ifh@stitlepic\h@stitlepicfalse
\def\titlepic#1{\gdef\@titlepic{#1}\h@stitlepictrue}
\newif\ifh@stitlepicdesc\h@stitlepicdescfalse
\def\titlepicdesc#1{\gdef\@titlepicdesc{#1}\h@stitlepicdesctrue}
%    \end{macrocode}
% \end{macro}
% \end{macro}
%
% \begin{macro}{\dpoversion}
% \begin{macro}{\chair}
% \begin{macro}{\referees}
% \begin{macro}{\thesistime}
% \begin{macro}{\supervisor}
% Definition der Makros f"ur Pr"ufungsordnung, Lehrstuhl, Gutachter, Betreuer
% und Bearbeitungszeitraum. Wenn eines dieser Makros nicht definiert ist, wird
% eine Fehlermeldung ausgegeben.
%    \begin{macrocode}
\def\dpoversion#1{\gdef\@dpoversion{#1}}
\def\@dpoversion{\@latex@error{No \noexpand\dpoversion given}\@ehc}
\def\chair#1{\gdef\@chair{#1}}
\def\@chair{\@latex@error{No \noexpand\chair given}\@ehc}
\def\referees#1{\gdef\@referees{#1}}
\def\@referees{\@latex@error{No \noexpand\referees given}\@ehc}
\def\supervisor#1{\gdef\@supervisor{#1}}
\def\@supervisor{\@latex@error{No \noexpand\supervisor given}\@ehc}
\def\thesistime#1{\gdef\@thesistime{#1}}
\def\@thesistime{\@latex@error{No \noexpand\thesistime given}\@ehc}
%    \end{macrocode}
% \end{macro}
% \end{macro}
% \end{macro}
% \end{macro}
% \end{macro}
%
% \begin{environment}{abstract}
% Erg"anzung der Abstract-Umgebung (die in \textsf{scrbook} nicht definiert ist)
%    \begin{macrocode}
\newenvironment{abstract}{\addchap*{\abstractname}}{}
%    \end{macrocode}
% \end{environment}
%
% \begin{environment}{nomenclature}
% Definition der Nomenklatur und Anpassen der Kopfzeile. Bei der Option
% |nomtotoc| wird die Nomenklatur ins Inhaltsverzeichnis aufgenommen.
% \changes{v1.4}{2007/01/03}{pdfbookmark added}
% \changes{v1.5}{2008/02/21}{bookmark anchor corrected}
%    \begin{macrocode}
\newenvironment{nomenclature}{%
  \if@nomtotoc
    \addchap{\nomname}
    \@mkboth{\nomname}{\nomname}
  \else
    \if@hyperref
        \if@openright\cleardoublepage\else\clearpage\fi
        \phantomsection\pdfbookmark{\nomname}{nom}%
    \fi
    \chapter*{\nomname}
    \@mkboth{\nomname}{\nomname}
  \fi
}{%
}
%    \end{macrocode}
% \end{environment}
%
% \subsection{Setzen der Titelei}
%
% \changes{v1.4d}{2007/06/02}{\cs{mytablewidth} no more defined}
% \begin{macro}{\titlehead}
% \changes{v1.4c}{2007/04/29}{IMTEK-Logo and Uni-Siegel substituted}
% \changes{v1.6}{2008/05/23}{Error if Logos are not installed}
% \begin{macro}{\subject}
% \changes{v1.4}{2007/01/01}{Bugfix pdfsubject}
% \changes{v1.5}{2008/01/20}{Variants for \texttt{bachelor} and \texttt{master}}
% \begin{macro}{\date}
% Definition der in \textsf{scrbook} vorgesehenen Felder der Titelei, die
% statisch belegt werden. F"ur |\subject| wird intern auch eine unformatierte
% Variante definiert, die f"ur die PDF-Informationen ben"otigt wird.
% \begin{macro}{\LLogo}
% \begin{macro}{\RLogo}
% \begin{macro}{\BLogo}
% Logos werden gesetzt, falls sie vorhanden sind, sonst werden eine
% Fehlermeldung und ein Ersatztext ausgegeben.
%    \begin{macrocode}
\newcommand{\LLogo}{\parbox[b][2.2cm]{0.3\textwidth}{%
  \texttt{figures/IMTEK\_Logo\_Farbe.*}
  von \texttt{http://intern.imtek.de} downloaden}}
\newcommand{\RLogo}{\parbox[b][2.2cm]{0.3\textwidth}{%
  \texttt{figures/Uni\_Siegel.*}
  von \texttt{http://intern.imtek.de} downloaden}}
\newcommand{\BLogo}{\parbox[b][2.2cm]{7.5cm}{%
  \texttt{figures/Uni\_Logo\_E2\_A4\_CMYK.*}
  von \texttt{http://www.uni-freiburg.de/go/cd} downloaden}}
\newcommand{\l@g@error}{%
  \ClassError{IMTEKda}{Logo file(s) not found}
    {One or more of the files\MessageBreak
    \space\space\space\space figures/IMTEK_Logo_Farbe.eps \MessageBreak
    \space\space\space\space figures/IMTEK_Logo_Farbe.pdf \MessageBreak
    \space\space\space\space figures/Uni_Siegel.eps \MessageBreak
    \space\space\space\space figures/Uni_Siegel.pdf \MessageBreak
    \space\space\space\space figures/Uni_Logo_E2_A4_CMYK.eps \MessageBreak
    \space\space\space\space figures/Uni_Logo_E2_A4_CMYK.pdf \MessageBreak
    are not installed properly. \MessageBreak
    Install this subdirectory with the logo files \MessageBreak
    together with the class file, see README.\MessageBreak
    Type <return> to proceed without the logos.}
  }
\IfFileExists{figures/IMTEK_Logo_Farbe.eps}%
  {\IfFileExists{figures/IMTEK_Logo_Farbe.pdf}%
    {\renewcommand{\LLogo}%
      {\includegraphics[height=2.2cm]{figures/IMTEK_Logo_Farbe}}}%
    {\l@g@error}%
  }{\l@g@error}
\if@oldcd
  \IfFileExists{figures/Uni_Siegel.eps}%
    {\IfFileExists{figures/Uni_Siegel.pdf}%
      {\renewcommand{\RLogo}%
        {\includegraphics[height=2.2cm]{figures/Uni_Siegel}}}%
      {\l@g@error}%
    }{\l@g@error}
  \renewcommand{\BLogo}{\null}%
\else
  \IfFileExists{figures/Uni_Logo_E2_A4_CMYK.eps}%
    {\IfFileExists{figures/Uni_Logo_E2_A4_CMYK.pdf}%
      {\renewcommand{\BLogo}%
        {\includegraphics{figures/Uni_Logo_E2_A4_CMYK}}}%
      {\l@g@error}%
    }{\l@g@error}
  \renewcommand{\RLogo}{\null}%
\fi
\titlehead{\LLogo\hfill\RLogo}
\def\s@bject{%
  \if@diplom\if@englishpreamble{Diploma Thesis}\else{Diplomarbeit}\fi%
  \else%
    \if@bachelor\if@englishpreamble{Bachelor's Thesis}\else{Bachelorarbeit}\fi%
    \else
    \if@englishpreamble{Master's Thesis}\else{Masterarbeit}\fi%
    \fi
  \fi}
\subject{\titlefont{\s@bject}}
\date{}
%    \end{macrocode}
% \end{macro}
% \end{macro}
% \end{macro}
% \end{macro}
% \end{macro}
% \end{macro}
%
% \begin{macro}{\maketitle}
% Neudefinition der Titelei. Die Definition der Titelseite wurde aus
% \textsf{scrbook} im wesentlichen "ubernommen, danach allerdings noch zwei
% Seiten mit offiziellen Angaben erg"anzt. Die Abfrage der Option |titlepage|
% wird nicht ber"ucksichtigt.
% \changes{v1.7}{2010/01/01}{new Uni logo}
%    \begin{macrocode}
  \renewcommand*\maketitle[1][1]{\begin{titlepage}%
    \begin{textblock*}{0pt}[0,1](\paperwidth-17pt, \paperheight-36pt)
      \llap{\BLogo}
    \end{textblock*}
%    \end{macrocode}
% Die Titelseite wird gegen"uber \textsf{scrbook} vergr"o"sert
%    \begin{macrocode}
    \enlargethispage{2cm}
    \setcounter{page}{#1}
    \let\footnotesize\small
    \let\footnoterule\relax
    \let\footnote\thanks
    \renewcommand*\thefootnote{\@fnsymbol\c@footnote}%
    \let\@oldmakefnmark\@makefnmark
    \renewcommand*{\@makefnmark}{\rlap\@oldmakefnmark}
%    \end{macrocode}
% Ein Schmutztitel wird nicht vorgesehen
%    \begin{macrocode}
    \ifx\@titlehead\@empty \else
        \noindent\begin{minipage}[t]{\textwidth}
        \@titlehead
        \end{minipage}\par
    \fi
    \null\vfill
    \begin{center}
    \ifx\@subject\@empty \else
        {\Large \@subject \par}
        \vskip 3em
    \fi
    {\titlefont\huge \@title\par}
    \vskip 3em
%    \end{macrocode}
% Das Titelbild wird ggf.\ eingef"ugt
% \changes{v1.4}{2007/01/02}{error message if titlepic without titlepicdesc given}
%    \begin{macrocode}
    \ifh@stitlepic
        \ifh@stitlepicdesc\relax\else
            \@latex@error{No \noexpand\titlepicdesc given}\@ehc
        \fi
        \@titlepic\par
        \vskip 3em
    \fi
    {\Large \lineskip 0.75em
    \begin{tabular}[t]{c}
        \@author
    \end{tabular}\par}
    \vskip 1.5em
    {\Large \@date \par}
    \vskip \z@ \@plus3fill
    {\Large \@publishers \par}
    \vskip 3em
    \end{center}\par
    \@thanks
    \vfill\null
    \if@twoside\next@tpage
        \noindent\begin{minipage}[t]{\textwidth}
        \@uppertitleback
        \end{minipage}\par
        \vfill
        \noindent\begin{minipage}[b]{\textwidth}
        \@lowertitleback
        \end{minipage}
    \fi
%    \end{macrocode}
% Eine neue Seite ohne Kopf und Fu"s wird f"ur Organisatorisches eingef"ugt.
% F"ur die Berechnung der Tabellenbreiten wird das Paket \textsf{calc} ben"otigt.
%    \begin{macrocode}
    \clearpage\thispagestyle{empty}
    \noindent%
%    \end{macrocode}
% Schreiben der englischen Variante
% \changes{v1.5}{2008/01/20}{Variants for \texttt{bachelor} and \texttt{master}}
%    \begin{macrocode}
    \if@englishpreamble
      \noindent
      \begin{tabular}[t]{p{0.24\textwidth}p{0.76\textwidth-4\tabcolsep}}
      &A \if@diplom{diploma}%
      \else{\if@bachelor{bachelor's}\else{master's}\fi}\fi{}
      thesis submitted in partial fulfillment of the
      requirements for the degree of
      \\[1em]
      &\if@diplom{Graduate Engineer of Microsystems Engineering}\else
      {\if@bachelor{Bachelor of Science of Microsystems Engineering}\else
      Master of Science of Microsystems Engineering\fi}\fi
      \\[1em]
      &according to the examination regulations
      at the University of Freiburg for the
      \if@diplom{Diploma}\else{\if@bachelor{Bachelor's degree}%
      \else{Master's degree}\fi}\fi{}
      in Microsystems Engineering of \@dpoversion{}.\\[1em]
      &\@chair\\
      &Department of Microsystems Engineering (IMTEK)\\
      &University of Freiburg\\
      &Freiburg im Breisgau, Germany
      \end{tabular}
      \vfil
      \noindent
      \begin{tabular}[t]{p{0.24\textwidth}p{0.76\textwidth-4\tabcolsep}}
      \bfseries Author&
        \begin{minipage}[t]{0.76\textwidth-4\tabcolsep}%
	  \@author\end{minipage}
      \end{tabular}
      \vfil
      \noindent
      \begin{tabular}[t]{p{0.24\linewidth}p{0.76\textwidth-4\tabcolsep}}
      \bfseries Thesis period&
        \begin{minipage}[t]{0.76\textwidth-4\tabcolsep}%
	  \@thesistime\end{minipage}\\&\\
      \bfseries Referees&
        \begin{minipage}[t]{0.76\textwidth-4\tabcolsep}%
	  \@referees\end{minipage}\\&\\
      \bfseries Supervisor&
        \begin{minipage}[t]{0.76\textwidth-4\tabcolsep}%
	  \@supervisor\end{minipage}\\
      \end{tabular}
      \ifh@stitlepicdesc
        \vfil
        \noindent
        \begin{tabular}[t]{p{0.24\linewidth}p{0.76\textwidth-4\tabcolsep}}
        \bfseries Title page&
          \begin{minipage}[t]{0.76\textwidth-4\tabcolsep}%
	    \@titlepicdesc\end{minipage}
        \end{tabular}
      \fi
      \clearpage\thispagestyle{empty}
      \null\vfill
      \noindent
      \begin{tabular}[t]{p{0.24\linewidth}p{0.76\textwidth-4\tabcolsep}}
        \bfseries\large Declaration&according to
        \if@diplom\S9(5) of the\else\if@bachelor\S22(8) of the\else{the}\fi\fi{}
        Examination Regulations:\\[1em]
        &I hereby confirm to have written the following thesis on my own,
        not having used any other sources or resources than those listed.
        All passages taken over literally or correspondingly from published
        sources have been marked accordingly. Additionally, this thesis has not
        been prepared or submitted for another examination, neither partially
        nor completely.
        \\[1em]
        &Freiburg, \today\\[2cm]
        &\begin{minipage}[t]{0.76\textwidth-4\tabcolsep}%
	  \@author\end{minipage}
      \end{tabular}
%    \end{macrocode}
% Schreiben der deutschen Variante
% \changes{v1.1}{2006/03/04}{spelling error corrected}
% \changes{v1.2}{2006/04/22}{spelling error corrected}
% \changes{v1.5}{2008/01/20}{Variants for \texttt{bachelor} and \texttt{master}}
%    \begin{macrocode}
    \else
      {\if@ngermanopt\selectlanguage{ngerman}%
        \else\if@germanopt\selectlanguage{german}%
        \fi%
      \fi%
      \noindent%
      \begin{tabular}[t]{p{0.24\textwidth}p{0.76\textwidth-4\tabcolsep}}
      &Eingereichte \if@diplom{Diplomarbeit}\else{\if@bachelor{Bachelorarbeit}%
      \else{Masterarbeit}\fi}\fi{} gem\"a\ss{} den Bestimmungen der
      Pr\"ufungsordnung
      der Universit\"at Freiburg f\"ur den
      \if@diplom{Diplomstudiengang}\else{\if@bachelor{Bachelorstudiengang}%
      \else{Masterstudiengang}\fi}\fi{}
      Mikrosystemtechnik vom \@dpoversion\\[1em]
      &\@chair\\
      &Institut f\"ur Mikrosystemtechnik (IMTEK)\\
      &Albert-Ludwigs-Universit\"at Freiburg\\
      &Freiburg im Breisgau
      \end{tabular}
      \vfil
      \noindent
      \begin{tabular}[t]{p{0.24\linewidth}p{0.76\textwidth-4\tabcolsep}}
      \bfseries Autor&
        \begin{minipage}[t]{0.76\textwidth-4\tabcolsep}%
	  \@author\end{minipage}
      \end{tabular}
      \vfil
      \noindent
      \begin{tabular}[t]{p{0.24\linewidth}p{0.76\textwidth-4\tabcolsep}}
      \bfseries Bearbeitungszeit&
        \begin{minipage}[t]{0.76\textwidth-4\tabcolsep}%
	  \@thesistime\end{minipage}\\&\\
      \bfseries Gutachter&
        \begin{minipage}[t]{0.76\textwidth-4\tabcolsep}%
	  \@referees\end{minipage}\\&\\
      \bfseries Betreuer&
        \begin{minipage}[t]{0.76\textwidth-4\tabcolsep}%
	  \@supervisor\end{minipage}\\
      \end{tabular}
      \ifh@stitlepicdesc
        \vfil
        \noindent
        \begin{tabular}[t]{p{0.24\textwidth}p{0.76\textwidth-4\tabcolsep}}
        \bfseries Titelseite&
          \begin{minipage}[t]{0.76\textwidth-4\tabcolsep}%
	    \@titlepicdesc\end{minipage}
        \end{tabular}
      \fi
      \clearpage\thispagestyle{empty}
      \null\vfill
      \noindent
      \begin{tabular}[t]{p{0.24\textwidth}p{0.76\textwidth-4\tabcolsep}}
        \bfseries\large Erkl\"arung&nach
        \if@diplom\S9(5) der Diplompr\"ufungsordnung\else
        {\if@bachelor\S22(8) der Pr\"ufungsordnung%
        \else der Pr\"ufungsordnung\fi}\fi{}:\\[1em]
        &Hiermit erkl\"are ich, dass ich diese Abschlussarbeit
        selbst\"andig verfasst habe, keine anderen als die
        angegebenen Quellen und Hilfsmittel benutzt habe und alle Stellen,
        die w\"ortlich oder sinngem\"a\ss\ aus ver\"offentlichten Schriften
        entnommen wurden, als solche kenntlich gemacht habe. Dar\"uberhinaus
        erkl\"are ich, dass diese Abschlussarbeit nicht, auch nicht
        auszugsweise, bereits f\"ur eine andere Pr\"ufung angefertigt
        wurde.\\[1em]
        &Freiburg, den \today\\[2cm]
        &\begin{minipage}[t]{0.76\textwidth-4\tabcolsep}
	  \@author\end{minipage}
      \end{tabular}%
      }
    \fi
%    \end{macrocode}
% Ggf.\ Einf"ugen einer Widmung
%    \begin{macrocode}
    \ifx\@dedication\@empty \else
        \next@tpage\null\vfill
        {\centering \Large \@dedication \par}
        \vskip \z@ \@plus3fill
        \if@twoside \next@tpage\cleardoublepage \fi
    \fi
    \end{titlepage}
%    \end{macrocode}
% Zur"ucksetzen der Titelei-spezifischen Definitionen
%    \begin{macrocode}
    \setcounter{footnote}{0}%
    \global\let\thanks\relax
    \global\let\maketitle\relax
    \global\let\@thanks\@empty
    \global\let\@author\@empty
    \global\let\@date\@empty
    \global\let\@title\@empty
    \global\let\@titlehead\@empty
    \global\let\@subject\@empty
    \global\let\@publishers\@empty
    \global\let\@uppertitleback\@empty
    \global\let\@lowertitleback\@empty
    \global\let\@dedication\@empty
    \global\let\author\relax
    \global\let\title\relax
    \global\let\extratitle\relax
    \global\let\titlehead\relax
    \global\let\subject\relax
    \global\let\publishers\relax
    \global\let\uppertitleback\relax
    \global\let\lowertitleback\relax
    \global\let\dedication\relax
    \global\let\date\relax
    \global\let\and\relax}
%    \end{macrocode}
% \end{macro}
%
% \subsection{Weitere Voreinstellungen}
%
% \changes{v1.4}{2007/01/03}{\cs{if@hyperref} defined}
% Definition und Initialisierung einer Abfrage, ob \textsf{hyperref} verwendet
% wird
%    \begin{macrocode}
\newif\if@hyperref\@hyperreffalse
%    \end{macrocode}
%
% Definition einiger PDF-Infos bei Verwendung des \textsf{hyperref}-Pakets
%    \begin{macrocode}
\AfterPackage*{hyperref}{%
    \@hyperreftrue
    \newcommand{\org@maketitle}{}%
    \let\org@maketitle\maketitle
    \def\maketitle{%
      \hypersetup{
        pdftitle={\@title},
        pdfauthor={\@author},
        pdfsubject={\s@bject}
        }%
      \org@maketitle
    }
}
%    \end{macrocode}
%
% \iffalse
%</class>
%<*template>
%
% \changes{v1.4b}{2007/03/15}{\textsf{babelbib} commented out}
%
%<<COMMENT

%% Moegliche Optionen: diejenigen der Klasse scrbook ausser titlepage

%% deutsche DA:
\documentclass[diplom,         %% Typ der Arbeit: diplom, bachelor oder master
               12pt,           %% Schriftgroesse
               twoside,        %% zweiseitiges Layout
               BCOR10mm,       %% Bindekorrektur 10 mm
%               liststotoc,nomtotoc,bibtotoc, %% Aufnahme der div. Verzeichnisse
                                              %% ins Inhaltsverzeichnis
%               pointlessnumbers, %% Ueberschriftnummer. ohne angehaengtem Punkt
               english,ngerman, %% Alternativspr. Englisch, Dokumentspr. Deutsch
%               final,          %% Endversion; draft fuer schnelles Kompilieren
               ]{IMTEKda}
%% Englisch mit dt. Vorspann:
% \documentclass[diplom,12pt,twoside,BCOR10mm,pointlessnumbers,ngerman,english,noenglishpreamble]{IMTEKda}

%% Labels anzeigen zur Korrektur
%\usepackage{showkeys} %% Labels verschwinden mit der Klassenoption final

\usepackage{babel}     %% Sprachen-Unterstuetzung
\usepackage{calc}      %% ermoeglicht Rechnen mit Laengen und Zaehlern
\usepackage[T1]{fontenc}
\usepackage[latin1]{inputenc}
%% in aktuellem Linux & MacOS X wird standardmaessig UTF8 kodiert!
% \usepackage[utf8]{inputenc}

\usepackage{amsmath,amssymb} %% zusaetzliche Mathe-Symbole

\usepackage{lmodern} %% type1-taugliche CM-Schrift als Variante zur
                     %% "normalen" EC-Schrift
%% Variante: Schriftumschaltung auf URW Garamond und Bitstream Vera
%\usepackage[garamond,sfscaled=false,ttscaled=false]{mathdesign}
%\usepackage[scaled=0.9]{berasans,beramono}

%% Paket fuer bibtex-Datenbanken
\usepackage[comma,numbers,sort&compress]{natbib}
%\usepackage{babelbib}         %% korrekte Sprache in Bibliographieeintraegen
\bibliographystyle{plainnat}  %% Formatierung Bibliographie ohne babelbib
%\bibliographystyle{babplain}  %% Formatierung Bibliographie mit babelbib

\newcommand{\tabheadfont}[1]{\textbf{#1}} %% Tabellenkopf in Fett
\usepackage{booktabs}  %% Befehle fuer besseres Tabellenlayout
\usepackage{longtable} %% umbrechbare Tabellen
%\usepackage{array}     %% zusaetzliche Spaltenoptionen

%% umfangreiche Pakete fuer Symbole wie \micro, \ohm, \degree, \celsius etc.
\usepackage{textcomp,gensymb}

%\usepackage{SIunits} %% Korrektes Setzen von Einheiten
\usepackage{units}   %% Variante fuer Einheiten

%\usepackage{icomma}  %% Abstandskorrektur fuer , als Dezimaltrenner

%% Hyperlinks im Dokument; muss als eines der letzten Pakete geladen werden
\usepackage[pdfstartview=FitH,      % Oeffnen mit fit width
            breaklinks=true,        % Umbrueche in Links, nur bei pdflatex default
            bookmarksopen=true,     % aufgeklappte Bookmarks
            bookmarksnumbered=true, % Kapitelnummerierung in bookmarks
            pdfprintscaling=None,   % Default-Einstellung zum Drucken: nicht skaliert
            pdfduplex=DuplexFlipLongEdge, % Default-Druck-Einstellung: Duplex
            ]{hyperref} 

%% um keine SANSSERIF Schriften fuer Ueberschriften zu verwenden:
%\setkomafont{sectioning}{\normalfont\normalcolor\bfseries}
%% fuer kleinere Bild- und Tabellenunterschriften:
%\addtokomafont{caption}{\footnotesize}

%% um abgekuerzte Abbildungs- und Tabellenbezeichnung mit \autoref zu erhalten:
%\addto{\extrasngerman}{\renewcommand*{\figureautorefname}{Abb.}}
%\addto{\extrasngerman}{\renewcommand*{\tableautorefname}{Tab.}}

\begin{document}

\author{Max Mustermann}
\title{Diplomarbeitsthema -- hier steht das Thema der Diplomarbeit}
\hypersetup{pdfkeywords={IMTEK, Diplomarbeit, diploma thesis}}

%% Einfuegen eines Titelbilds (optional)
\titlepic{\includegraphics[height=10cm]{figures/bild}}
\titlepicdesc{Dieses Bild zeigt einen xyz-Sensor beim Messen der ABC-Kon"-zentration.}

%% Die jeweils auskommentierte Variante ist bei englischer Praeambel zu verwenden
\dpoversion{20.\,7.~2001} 
%\dpoversion{July 20, 2001}
%\dpoversion{28.\,9.~2000}       %% DPO 2000
%\dpoversion{September 28, 2000} %% DPO 2000
\chair{Lehrstuhl f"ur \dots}
%\chair{Micro-optics}
\referees{Prof.\ Dr.\ \dots, Lehrstuhl f"ur \dots\\
Prof. Dr. \dots, Lehrstuhl f"ur \dots, Universit"at \dots (wenn nicht FR)}
%\referees{Prof.\ \dots of \dots}
\supervisor{Prof.\ Dr.\ \dots, Lehrstuhl f"ur \dots\\
Prof.\ Dr.\ \dots, Lehrstuhl f"ur \dots, Universit"at \dots (wenn nicht FR)}
\thesistime{1.\ Januar 2006 bis 31.\ Mai 2006}
%\thesistime{January~1, 2006 to May~31, 2006}

\frontmatter
\maketitle
\cleardoublepage\phantomsection\pdfbookmark{\abstractname}{abstract} %% fuegt ersten Abstract in die Bookmarks ein
%\begin{otherlanguage}{ngerman}
\begin{abstract}
  Hier werden auf einer halben Seite die Kernaussagen der Diplomarbeit
  auf Deutsch zusammengefasst.
  \bigskip\par
  \textbf{Stichw"orter:} IMTEK, Diplomarbeit
\end{abstract}
%\end{otherlanguage}
\begin{otherlanguage}{english}
\begin{abstract}
  Hier werden auf einer halben Seite die Kernaussagen der Diplomarbeit
  auf Englisch zusammengefasst.
  \bigskip\par
  \textbf{Keywords:} IMTEK, Diploma thesis
\end{abstract}
\end{otherlanguage}

%% fuegt Inhaltsverzeichnis in die Bookmarks ein
\cleardoublepage\phantomsection\pdfbookmark{\contentsname}{toc}
%% setzt Inhaltsverzeichnis
\tableofcontents

\begin{nomenclature}
%% Fuer die Berechnung der Spaltenbreiten muss \usepackage{calc}
%% geladen sein!
\section*{Lateinische Buchstaben}
\noindent
\begin{longtable}[l]{p{0.2\textwidth}p{0.7\textwidth-6\tabcolsep}p{0.1\textwidth}}
  \tabheadfont{Variable}&\tabheadfont{Bedeutung}&\tabheadfont{Einheit}\\\midrule\endhead
  $A$ & Querschnittsfl"ache & $\unit{m^2}$\\
  $c$ & Geschwindigkeit & $\unitfrac{m}{s}$
\end{longtable}
\section*{Griechische Buchstaben}
\begin{longtable}[l]{p{0.2\textwidth}p{0.7\textwidth-6\tabcolsep}p{0.1\textwidth}}
  \tabheadfont{Variable}&\tabheadfont{Bedeutung}&\tabheadfont{Einheit}\\\midrule\endhead
  $\alpha$  & Winkel & $\unit{\degree}$; --\\
  $\varrho$ & Dichte & $\unitfrac{kg}{m^3}$
\end{longtable}
\section*{Indizes}
\begin{longtable}[l]{p{0.2\textwidth}p{0.8\textwidth-4\tabcolsep}}
  \tabheadfont{Index}&\tabheadfont{Bedeutung}\\\midrule\endhead
  m & Meridian\\
  $r$ & Radial
\end{longtable}
\section*{Abk"urzungen}
\begin{longtable}[l]{p{0.2\textwidth}p{0.8\textwidth-4\tabcolsep}}
  \tabheadfont{Abk"urzung}&\tabheadfont{Bedeutung}\\\midrule\endhead
  2D & zweidimensional\\
  3D & dreidimensional\\
  max & maximal
\end{longtable}
\end{nomenclature}

%% die Klassenoption liststotoc uebernimmt das Abbildungs- und Tabellen-
%% verzeichnis in den TOC
%% \listoftables und \listoffigures sollten nur bei genuegender Anzahl Tabellen
%% verwendet werden
\listoffigures
\listoftables

\mainmatter   %% Anfang Hauptteil

\chapter{Einleitung}
Diese Diplomarbeitsvorlage ist eine Neu"uberarbeitung der Vorlage von Jan
Lienemann durch Simon Dreher. Wichtige Hinweise finden sich in der beigef"ugten
Dokumentation \verb|IMTEKda.pdf|, bitte diese sorgf"altig lesen!

\chapter{Grundlagen}
In diesem Kapitel werden die theoretischen Grundlagen erl"autert.

Wichtige Gleichungen, die in der Arbeit h"aufiger zitiert werden,
sollten eine Gleichungsnummer erhalten.
\begin{equation}
  \label{eq:pythagoras}
  a^2+b^2=c^2
\end{equation}
Zum Beispiel wird in Gleichung~\ref{eq:pythagoras} der Satz des Pythagoras
angegeben.

Gerade im Bereich der Grundlagen wird viel Literatur zitiert, z.\,B.\ 
\cite{Menz97}. Falls
mehrere Literaturzitate auf einmal zitiert werden, ist folgendes
z.\,B.\ m"oglich \cite{Horn90,DINEN6232,Menz97,Knuth84}.

\section{Unterkapitel Gliederungsebene 2}
Hier sollte etwas Text stehen.
\subsection{Unterkapitel Gliederungsebene 3}
Noch ein paar Beispiele zu Abbildungen und Tabellen:

Abbildung~\ref{fig:bildplatzhalter} verdeutlicht \dots

Wie die Abb.~\ref{fig:bildplatzhalter} und
Tab.~\ref{tab:tabellenplatzhalter} verdeutlichen \dots

\begin{figure}
  \centering
  \includegraphics[width=0.5\linewidth]{figures/bild}
  \caption{Bildbeschreibung}
  \label{fig:bildplatzhalter}
\end{figure}

Text\dots
\begin{table}
  \centering
  \begin{tabular}{llll}
    \toprule
    $A$-Wert&$B$-Wert&$C$-Wert&$D$-Wert\\
    \midrule
    aaaaaa&bbbbbbb&cccccc&ddddddd\\
    aaaaaa&bbbbbbb&cccccc&ddddddd\\
    \bottomrule
  \end{tabular}
  \caption{Tabellenbeschreibung}
  \label{tab:tabellenplatzhalter}
\end{table}

Text\dots

\chapter{Experimentelle Vorgehensweise}
Text\dots
\chapter{Ergebnisse}
Text\dots
\chapter{Diskussion}
Text\dots
\section{Unterkapitel}
Text\dots
\subsection{Unterkapitel}
Text\dots
\chapter{Zusammenfassung}
Text\dots

\appendix
\chapter{erster Anhang}
Text\dots
\chapter{zweiter Anhang}
Text\dots

%% fuegt Literaturverzeichnis in die Bookmarks ein
\cleardoublepage\phantomsection\pdfbookmark{\bibname}{bib}
\bibliography{diplarb} %% Bibliographie; unbedingt umbenennen!

\chapter*{Danksagung}
Dank\dots

\end{document}
%COMMENT
%</template>
%
%<*bib>
%<<COMMENT
@inbook{Menz97,
        author    = {W. Menz and J. Mohr},
        title     = {Mikrosystemtechnik f"ur Ingenieure},
        publisher = {VCH},
        chapter   = {1--9},
        year      = 1997,
        language  = {ngerman}
        }

@article{Horn90,
        author    = {R. G. Horn},
        title     = {Surface Forces and Their Actions in Ceramic Materials},
        journal   = {Journal of the American Ceramic Society},
        volume    = {73},
        pages     = {1117--1135},
        year      = {1990},
        language  = {english}
        }

@manual{DINEN6232,
        title     = {Pr"ufverfahren f"ur Hochleistungskeramiken: Allgemeine und strukturelle Eigenschaften: Bestimmung von Dichte und Porosit"at},
        organization = {DIN EN 623--2},
        year      = 1991,
        language  = {ngerman}
        }

@book{Lamport94,
        author    = {Leslie Lamport},
        title     = {\LaTeX: A Document Preparation System},
        publisher = {Addison-Wesley},
        year      = 1994,
        language  = {english}
        }

@book{Knuth84,
        author    = {Donald E. Knuth},
        title     = {The \TeX book},
        volume    = {A},
        series    = {Computers and Typesetting},
        publisher = {Addison-Wesley},
        year      = 1984,
        language  = {english}
        }
%COMMENT
%</bib>
% \fi
%
% \Finale
\endinput
