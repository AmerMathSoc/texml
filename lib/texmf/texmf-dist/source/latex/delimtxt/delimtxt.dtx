% \iffalse
%<*drv>
\documentclass{ltxdoc}
\usepackage{doc}
\usepackage[T1]{fontenc}
\usepackage[ansinew]{inputenc}
\CodelineIndex
\parindent0pt
\setcounter{IndexColumns}{2}
\begin{document}
%\OnlyDescription  %Comment out for Desription only
\setlength\hfuzz{15pt}
\sloppy
\DocInput{delimtxt.dtx}
\end{document}
%</drv>
%\fi
% \MakeShortVerb{\|}
% \CheckSum{270}
% Copyright  1999 Bjoern Pedersen
% This program can be redistributed and/or modified under the terms
% of the LaTeX Project Public License Distributed from CTAN
% archives in directory macros/latex/base/lppl.txt; either
% version 1 of the License, or any later version.
%
%
%% This is heavily based on:
%% textmerg.dtx  (with options: `textmerg')
%% 
%% Copyright (C) 1992,1994 Mike Piff, University of Sheffield, England
% \section{Intro}
% This package tries do deal with single character delimited table
% files. It was mainly inspired and is heavily based on Mike Piff's
% textmerg package.
%
% Note for package writers:
% As this package is still under development, the interface is not g
% guaranteed to be stable. Please consider this if you want to use this
% package in your style files 
%
% \section{Description}
% \DescribeMacro{\SetDel}
% The |\SetDel<charactertoken>| set the character used as 
% a delimiter in the input file. The default is \texttt{\char126}.
% If the character does not have |\catcode=12|, you should adjust this
% before setting it as in this example:
% \begin{verbatim}
% {\catcode`\^^I=12
% \SetDel^^I}
% \end{verbatim}
% This would set  del to the <tab>-character 
%
% \DescribeMacro{\Fields}
% The |\Fields| macor takes a list of control sequence, which will be
% assigned during the read in process. 
% Example:
% \begin{verbatim}
% \Fields{\Title\Forenames\Surname\Address\Grade}
% \end{verbatim}
% 
% \DescribeMacro{\DelimRead}
% After defining the |\Fields|, a file is read in with
% |\DelimRead{File}{Template}|,
% where |File| is the filename of the data file, and |Template| is the
% text, in which occurences of the csnames in the |\Fields|-macro
% should be replaced by text from the data file.
%
% \StopEventually
% \section{Code}
% \subsection{Header}
% Announce the file.
%    \begin{macrocode}
%<*delimtxt> 
\def\fileversion{1.02}
\def\filedate{1999/05/03}
\def\filename{delimtxt.dtx}
\def\Copyright{Copyright 1999 Bjoern Pedersen}
\NeedsTeXFormat{LaTeX2e}[1998/06/01]
\ProvidesPackage{delimtxt}[\filedate]
\typeout{Package `delimtxt� <\filedate>.}
\typeout{\Copyright}
%    \end{macrocode}
% \subsection{Utility macros}
%    \begin{macrocode}
\def\glet{\global\let}

%    \end{macrocode}
%\subsection{File Handling}
% This opens a file and reads it line by line into |\InputBuffer|.
%    \begin{macrocode}
\newread\DelimFile

\def\InputFile#1{%
   \openin\DelimFile=#1
   \ifeof\DelimFile
   \errmessage{Empty Delim file}%
   \closein\DelimFile
   \long\def\MakeTemplate##1{%
      \def\Template{}}%
   \else\GetInput
   \fi}
%    \end{macrocode}
%
% Adjust the catcode of the delimiter temporarily, and read one line
% of input.
%    \begin{macrocode}
\def\GetInput{{
 \global\LF@false
 \endlinechar=-1%
 \expandafter\catcode\expandafter`\the\Del=12
   \global\read\DelimFile to\InputBuffer}}
%    \end{macrocode}
%
%  Check, if there is anything left in the Input file. If not, stop
%   Iterating. Empty lines in the file are silently skipped.
%
%    \begin{macrocode}
\def\SeeIfEof{%
   \let\NextLook\relax
   \ifeof\DelimFile
   \else
      \ifx\InputBuffer\empty
         \LookAgain
      \fi
   \fi
   \NextLook}

\def\LookAgain{\GetInput
   \let\NextLook\SeeIfEof}
%    \end{macrocode}


%    \begin{macro}{\ifNonBlank}
%    \begin{macro}{\AllowBlank}
%    \begin{macro}{\DontAllowBlank}
%We can now prepare to read actual fields from the merge file. A conditional
%is used to indicate whether or not the field we are about to read is
%allowed to be blank. We also set up a mechanism for changing its value.
%    \begin{macrocode}
\newif\ifNonBlank \NonBlankfalse
\def\AllowBlank{\global\NonBlankfalse}
\def\DontAllowBlank{\global\NonBlanktrue}
%    \end{macrocode}
%    \end{macro}
%    \end{macro}
%    \end{macro}



% \subsection{Parsing the Input Buffer}
% This is the difficult part of the processing.
% \subsubsection{Helper macros and registers}
% We need some token registers to save the Input, the delimiter, and
% some of the definitions for dynamic parameter lists
%    \begin{macrocode}
\newif\ifLF@
\def\mark{\relax}
\newtoks\InPutField
\newtoks\Del
\newtoks\StripT@k
\newtoks\NextFieldT@k
%    \end{macrocode}
%\subsubsection{Strip mark helper}
% A helper macro to strip of a marker, we placed in the input
% stream. It is put in |\StripT@k| and the actual definiton will take
% place on execution of |\SetDel| as we need to know what the
% delimiter actually is.
%
%    \begin{macrocode}  
\StripT@k={%
{%
\aftergroup\gdef%
\aftergroup\StripMark%
\aftergroup#\aftergroup1%
\expandafter\aftergroup\the\Del%
\aftergroup\mark%
}{\gdef\InputBuffer{#1}}}
%    \end{macrocode}
% \subsubsection{Get the next Field value from the input stream }
% On execution of |\SetDel| this mess will define a macro
% |\GetNextInputField#1<expanded \del>#2\lineend|.
% This will perform somthing similar to the C language |strtok| function.
% This macro gets the contents of the |\InputBuffer| plus an extra
% delimiter, a mark and a lineend marker.
% On Exit |\InputBuffer| is reassigned with one less Field after
% stripping of all markers. 
% If nothing is left, a flag is set and |\InputBuffer| is set empty.
% This flag is currently unused, but could be used for better error
% handling in case of missing fields in the input.
%    \begin{macrocode}

\NextFieldT@k={%
{%begin of aftergroup group
\aftergroup\gdef%
\aftergroup\GetNextInputField%
\aftergroup#\aftergroup1%
\expandafter\aftergroup\the\Del%
\aftergroup#\aftergroup2%
\aftergroup\lineend}%end of aftergroup group
{%
 \if\mark #2% 
    \global\LF@true% 
     \glet\InputBuffer=\empty% 
  \else%
     \global\LF@false% 
     \StripMark#2% 
  \fi%
  \InPutField={#1}%
  \if!#1!% check if Field is empty (Ref: D.Carlise in comp.text.tex)
      \ifNonBlank%
         \MissingField%
         \InPutField={???}%  
      \else%
         \InPutField={#1}%
      \fi%
      \else
      \relax	
  \fi%
}}
%    \end{macrocode}
%
%  This macro sets the Delimiter. As this may be called at any time,
%  we need  to redefine the macros |\GetNextInputField|
%  and |\StripMark|. The definitions have been stored in two token
%  registers, so we have just to execute them. The trickery with
%  |\aftergroup| in the token list enables expansion of |\the\Del| in
%  the macro parameter list.
%
%
%    \begin{macrocode}
\def\SetDel#1{\global\Del={#1}%
\the\StripT@k%
\the\NextFieldT@k%
}
\SetDel|


%    \end{macrocode}
% \subsection{Parsing the fields}
% Here we parse the inout fields as in the textmerg package, but
% getting values from our new parser. Probably, the treatment of
% missing items is not very good( in fact it is completly missing)
% We have to put a |\mark| and |\lineend| in the stream, do detect
% the end of the input line.
%
%    \begin{macrocode}

\def\ReadIn#1{%
     \expandafter\expandafter\expandafter%
     \GetNextInputField%
     \expandafter\InputBuffer\the\Del%
     \mark\lineend%
      \global\edef#1{\the\InPutField}%
}
%    \end{macrocode}
% This is not used yet. 
%    \begin{macrocode}
\def\MissingField{%
  \message{Missing field in file}}
%    \end{macrocode}
%
% Here begins the field parsing, as in the textmerg-package.
%
%    \begin{macrocode}

\newtoks\GlobalFields
%
\def\Fields#1{\GlobalFields{#1}}
%
\def\ParseFields#1{%
  \ifx#1\EndParseFields%
  \let\NextParse\relax%
      \ifLF@% 
        \message{ Line was OK}%
      \else%
        {\message{ There were more items than fields on line
        \the\Iteratecounter.  They will be skipped.}
        \glet\InputBuffer=\empty}%
      \fi%
  \else%
      \let\NextParse\ParseFields%
      \ifx#1+\DontAllowBlank%
      \else%
         \ifx#1-\AllowBlank%
         \else\ReadIn#1%
         \fi%
      \fi%
  \fi\NextParse}%

\let\EndParseFields\ParseFields%
%    \end{macrocode}
%    \begin{macrocode}
\def\ReadFields#1{
\ifeof\DelimFile%
  \else%
\expandafter\ParseFields%
  \the#1\EndParseFields%
\fi}%
%    \end{macrocode}
%\subsection{The iteration code}
%    \begin{macrocode}
\long\def\DelimRead#1#2{\begingroup%
   \InputFile{#1}%
   \def\Fields##1{%
      \ParseFields##1\EndParseFields}%
   \MakeTemplate{#2}\Iterate}%
%    \end{macrocode}
%    \begin{macrocode}
\long\def\MakeTemplate#1{\def\Template{#1}}
%    \end{macrocode}
%    \begin{macrocode}
\countdef\Iteratecounter=1%

\Iteratecounter=0
\def\Iterate{%
   \global\advance\Iteratecounter by1%
   \ReadFields\GlobalFields%
   \Template%
   \SeeIfEof%
   \ifeof\DelimFile%
      \def\NextIteration{%
         \endgroup\closein\DelimFile}%
   \else%     
      \let\NextIteration\Iterate%
   \fi%
   \NextIteration}
\endinput
%    \end{macrocode}
%</delimtxt>
%\Finale
