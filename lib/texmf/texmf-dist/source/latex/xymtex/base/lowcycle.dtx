% \iffalse meta-comment
%% File: lowcycle.dtx
%
%  Copyright 1993,1996,1998,2001,2002,2004,2010 by Shinsaku Fujita
%
%  This file is part of XyMTeX system.
%  -------------------------------------
%
% This file is a successor to:
%
% lowcycle.sty
% %%%%%%%%%%%%%%%%%%%%%%%%%%%%%%%%%%%%%%%%%%%%%%%%%%%%%%%%%%%%%%%%%%%%
% \typeout{XyMTeX for Drawing Chemical Structural Formulas. Version 1.00}
% \typeout{       -- Released December 1, 1993 by Shinsaku Fujita}
% Copyright (C) 1993 by Shinsaku Fujita, all rights reserved.
%
% This file is a part of the macro package ``XyMTeX'' which has been 
% designed for typesetting chemical structural formulas.
%
% This file is to be contained in the ``xymtex'' directory which is 
% an input directory for TeX. It is a LaTeX optional style file and 
% should be used only within LaTeX, because several macros of the file 
% are based on LaTeX commands. 
%
% For the review of XyMTeX, see
%  (1)  Shinsaku Fujita, ``Typesetting structural formulas with the text
%    formatter TeX/LaTeX'', Computers and Chemistry, in press.    
% The following book deals with an application of TeX/LaTeX to 
% preparation of manuscripts of chemical fields:
%  (2)  Shinsaku Fujita, ``LaTeX for Chemists and Biochemists'' 
%    Tokyo Kagaku Dozin, Tokyo (1993) [in Japanese].  
%
% This work may be distributed and/or modified under the
% conditions of the LaTeX Project Public License, either version 1.3
% of this license or (at your option) any later version.
% The latest version of this license is in
%   http://www.latex-project.org/lppl.txt
% and version 1.3 or later is part of all distributions of LaTeX
% version 2005/12/01 or later.
%
% This work has the LPPL maintenance status `maintained'. 
% The Current Maintainer of this work is Shinsaku Fujita.
%
% This work consists of the files lowcycle.dtx and lowcycle.ins
% and the derived file lowcycle.sty.
%
% Please report any bugs, comments, suggestions, etc. to:
%   Shinsaku Fujita, 
%   Shonan Institute of Chemoinformatics and Mathematical Chemistry
%   Kaneko 479-7 Ooimachi, Ashigara-Kami-Gun, Kanagawa 250-0019 Japan
%  (old address)
%   Ashigara Research Laboratories, Fuji Photo Film Co., Ltd., 
%   Minami-Ashigara, Kanagawa-ken, 250-01, Japan.
%  (old address)
%   Department of Chemistry and Materials Technology, 
%   Kyoto Institute of Technology, \\
%   Matsugasaki, Sakyoku, Kyoto, 606 Japan
% %%%%%%%%%%%%%%%%%%%%%%%%%%%%%%%%%%%%%%%%%%%%%%%%%%%%%%%%%%%%%%%%%%%%
% \def\j@urnalname{lowcycle}
% \def\versi@ndate{December 01, 1993}
% \def\versi@nno{ver1.00}
% \def\copyrighth@lder{SF} % Shinsaku Fujita
% %%%%%%%%%%%%%%%%%%%%%%%%%%%%%%%%%%%%%%%%%%%%%%%%%%%%%%%%%%%%%%%%%%%%
% \def\j@urnalname{lowcycle}
% \def\versi@ndate{August 16, 1996}
% \def\versi@nno{ver1.01}
% \def\copyrighth@lder{SF} % Shinsaku Fujita
% %%%%%%%%%%%%%%%%%%%%%%%%%%%%%%%%%%%%%%%%%%%%%%%%%%%%%%%%%%%%%%%%%%%%%
% \def\j@urnalname{lowcycle}
% \def\versi@ndate{October 31, 1998}
% \def\versi@nno{ver1.02}
% \def\copyrighth@lder{SF} % Shinsaku Fujita
% %%%%%%%%%%%%%%%%%%%%%%%%%%%%%%%%%%%%%%%%%%%%%%%%%%%%%%%%%%%%%%%%%%%%%
% \def\j@urnalname{lowcycle}
% \def\versi@ndate{December 25, 1998}
% \def\versi@nno{ver2.00}
% \def\copyrighth@lder{SF} % Shinsaku Fujita
% %%%%%%%%%%%%%%%%%%%%%%%%%%%%%%%%%%%%%%%%%%%%%%%%%%%%%%%%%%%%%%%%%%%%%
% \def\j@urnalname{lowcycle}
% \def\versi@ndate{June 20, 2001}
% \def\versi@nno{ver2.01}
% \def\copyrighth@lder{SF} % Shinsaku Fujita
% %%%%%%%%%%%%%%%%%%%%%%%%%%%%%%%%%%%%%%%%%%%%%%%%%%%%%%%%%%%%%%%%%%%%%
% \def\j@urnalname{lowcycle}
% \def\versi@ndate{April 30, 2002}
% \def\versi@nno{ver3.00}
% \def\copyrighth@lder{SF} % Shinsaku Fujita
% %%%%%%%%%%%%%%%%%%%%%%%%%%%%%%%%%%%%%%%%%%%%%%%%%%%%%%%%%%%%%%%%%%%%%
% \def\j@urnalname{lowcycle}
% \def\versi@ndate{May 30, 2002}
% \def\versi@nno{ver4.00}
% \def\copyrighth@lder{SF} % Shinsaku Fujita
% %%%%%%%%%%%%%%%%%%%%%%%%%%%%%%%%%%%%%%%%%%%%%%%%%%%%%%%%%%%%%%%%%%%%%
% \def\j@urnalname{lowcycle}
% \def\versi@ndate{August 30, 2004}
% \def\versi@nno{ver4.01}
% \def\copyrighth@lder{SF} % Shinsaku Fujita
% %%%%%%%%%%%%%%%%%%%%%%%%%%%%%%%%%%%%%%%%%%%%%%%%%%%%%%%%%%%%%%%%%%%%%
%
% \fi
%
% \CheckSum{175}
%% \CharacterTable
%%  {Upper-case    \A\B\C\D\E\F\G\H\I\J\K\L\M\N\O\P\Q\R\S\T\U\V\W\X\Y\Z
%%   Lower-case    \a\b\c\d\e\f\g\h\i\j\k\l\m\n\o\p\q\r\s\t\u\v\w\x\y\z
%%   Digits        \0\1\2\3\4\5\6\7\8\9
%%   Exclamation   \!     Double quote  \"     Hash (number) \#
%%   Dollar        \$     Percent       \%     Ampersand     \&
%%   Acute accent  \'     Left paren    \(     Right paren   \)
%%   Asterisk      \*     Plus          \+     Comma         \,
%%   Minus         \-     Point         \.     Solidus       \/
%%   Colon         \:     Semicolon     \;     Less than     \<
%%   Equals        \=     Greater than  \>     Question mark \?
%%   Commercial at \@     Left bracket  \[     Backslash     \\
%%   Right bracket \]     Circumflex    \^     Underscore    \_
%%   Grave accent  \`     Left brace    \{     Vertical bar  \|
%%   Right brace   \}     Tilde         \~}
%
% \setcounter{StandardModuleDepth}{1}
%
% \StopEventually{}
% \MakeShortVerb{\|}
%
% \iffalse
% \changes{v1.01}{1996/06/26}{first edition for LaTeX2e}
% \changes{v1.02}{1998/10/31}{revised edition for LaTeX2e}
% \changes{v2.00}{1998/12/25}{enhanced edition for LaTeX2e}
% \changes{v2.01}{2001/6/20}{Size reduction and Clip information}
% \changes{v3.00}{2002/4/30}{sfpicture environment, etc.}
% \changes{v4.00}{2002/05/30}{PostScript output and ShiftPicEnv}
% \changes{v4.01}{2004/08/30}{Minor additions}
% \changes{v5.00}{2010/10/01}{the LaTeX Project Public License}
% \fi
%
% \iffalse
%<*driver>
\NeedsTeXFormat{pLaTeX2e}
% \fi
\ProvidesFile{lowcycle.dtx}[2010/10/01 v5.00 XyMTeX{} package file]
% \iffalse
\documentclass{ltxdoc}
\GetFileInfo{lowcycle.dtx}
%
% %%XyMTeX Logo: Definition 2%%%
\def\UPSILON{\char'7}
\def\XyM{X\kern-.30em\smash{%
\raise.50ex\hbox{\UPSILON}}\kern-.30em{M}}
\def\XyMTeX{\XyM\kern-.1em\TeX}
% %%%%%%%%%%%%%%%%%%%%%%%%%%%%%%
\title{Lower carbocyclic compounds by {\sffamily lowcycle.sty} 
(\fileversion) of \XyMTeX{}}
\author{Shinsaku Fujita \\ 
Shonan Institute of Chemoinformatics and Mathematical Chemistry, \\
Kaneko 479-7 Ooimachi, Ashigara-Kami-Gun, Kanagawa 250-0019 Japan
% % (old address)
% %Department of Chemistry and Materials Technology, \\
% %Kyoto Institute of Technology, \\
% %Matsugasaki, Sakyoku, Kyoto, 606-8585 Japan
% %% (old address)
% %% Ashigara Research Laboratories, 
% %% Fuji Photo Film Co., Ltd., \\ 
% %% Minami-Ashigara, Kanagawa, 250-01 Japan
}
\date{\filedate}
%
\begin{document}
   \maketitle
   \DocInput{lowcycle.dtx}
\end{document}
%</driver>
% \fi
%
% \section{Introduction}\label{lowcycle:intro}
%
% \subsection{Options for {\sffamily docstrip}}
%
% \DeleteShortVerb{\|}
% \begin{center}
% \begin{tabular}{|l|l|}
% \hline
% \emph{option} & \emph{function}\\ \hline
% lowcycle & lowcycle.sty \\
% driver & driver for this dtx file \\
% \hline
% \end{tabular}
% \end{center}
% \MakeShortVerb{\|}
%
% \subsection{Version Information}
%
%    \begin{macrocode}
%<*lowcycle>
\typeout{XyMTeX for Drawing Chemical Structural Formulas. Version 5.00}
\typeout{       -- Released October 01, 2010 by Shinsaku Fujita}
% %%%%%%%%%%%%%%%%%%%%%%%%%%%%%%%%%%%%%%%%%%%%%%%%%%%%%%%%%%%%%%%%%%%%%
\def\j@urnalname{lowcycle}
\def\versi@ndate{October 01, 2010}
\def\versi@nno{ver5.00}
\def\copyrighth@lder{SF} % Shinsaku Fujita
% %%%%%%%%%%%%%%%%%%%%%%%%%%%%%%%%%%%%%%%%%%%%%%%%%%%%%%%%%%%%%%%%%%%%%
\typeout{XyMTeX Macro File `\j@urnalname' (\versi@nno) <\versi@ndate>%
\space[\copyrighth@lder]}
%    \end{macrocode}
%
% \section{List of commands for lowcycle.sty}
%
% \begin{verbatim}
% **********************************
% * lowcycle.sty: list of commands * 
% **********************************
%
%     \cyclopentanev                      \@cyclopentanev
%     \cyclopentanevi                     \@cyclopentanevi
%     \cyclopentaneh                      \@cyclopentaneh
%     \cyclopentanehi                     \@cyclopentanehi
%
%     \cyclobutane                        \@cyclobutane
%
%     \cyclopropane                       \@cyclopropane
%     \cyclopropanei                      \@cyclopropanei
%     \cyclopropaneh                      \@cyclopropaneh
%     \cyclopropanehi                     \@cyclopropanehi
%
%     \indanev                            \@indanev
%     \indanevi                           \@indanevi
%     \indaneh                            \@indaneh
%     \indanehi                           \@indanehi
% \end{verbatim}
%
% \section{Input of basic macros}
%
% To assure the compatibility to \LaTeX{}2.09 (the native mode), 
% the commands added by \LaTeXe{} have not been used in the resulting sty 
% files ({\sf lowcycle.sty} for the present case).  Hence, the combination 
% of |\input| and |\@ifundefined| is used to crossload sty 
% files ({\sf chemstr.sty} etc. for the present case) in place of the 
% |\RequirePackage| command of \LaTeXe{}. 
%
%    \begin{macrocode}
% *************************
% * input of basic macros *
% *************************
\@ifundefined{setsixringv}{\input chemstr.sty\relax}{}
\@ifundefined{threehetero}{\input hetarom.sty\relax}{}
\@ifundefined{sixheteroh}{\input hetaromh.sty\relax}{}
\unitlength=0.1pt
%    \end{macrocode}
%
% \section{Cyclopentane derivatives}
% \subsection{Vertical type}
%
% The macros |\cyclopentanev| and |\cyclopentanevi| have an 
% argument |SUBSLIST| as well as an optional argument |BONDLIST|.  
% The macro |\cyclopentanev| draws a five-membered ring with a 
% flat top bond, while the macro |\cyclopentanevi| draws an 
% inverse ring with a flat bottom bond. 
% \begin{verbatim}
% ******************************
% *  cyclopentane derivatives  *
% *  (vertical type & inverse) *
% ******************************
%
%   \cyclopentanev[BONDLIST]{SUBSLIST}
%   \cyclopentanevi[BONDLIST]{SUBSLIST}
% \end{verbatim}
%
% The |BONDLIST| argument contains one or more 
% characters selected from a to e, each of which indicates the presence of 
% an inner (endcyclic) double bond on the corresponding position. The 
% option `$n+$' ($n=1$--$5$) is used for designating a plus charge on 
% the $n$-carbon. 
% The option `$0+$'is used for designating a plus charge at the center 
% of the five-membered ring. 
% \begin{verbatim}
%     BONDLIST: list of inner double bonds 
%
%           none       :  mother nucleus (fully saturated form)
%           a          :  1,2-double bond
%           b          :  2,3-double bond
%           c          :  4,3-double bond
%           d          :  4,5-double bond
%           e          :  5,1-double bond
%           {n+}       :  plus at the n-carbon atom (n = 1 to 5)
%           {0+}       :  plus at the center
% \end{verbatim}
%
% The |SUBSLIST| argument contains one or more substitution descriptors 
% which are separated from each other by a semicolon.  Each substitution 
% descriptor has a locant number with a bond modifier and a substituent, 
% where these are separated with a double equality symbol. 
% \begin{verbatim}
%
%     SUBSLIST: list of substituents
%
%       for n = 1 to 5 
%
%           nD         :  exocyclic double bond at n-atom
%           n or nS    :  exocyclic single bond at n-atom
%           nA         :  alpha single bond at n-atom
%           nB         :  beta single bond at n-atom
%           nSA        :  alpha single bond at n-atom (boldface)
%           nSB        :  beta single bond at n-atom (dotted line)
%           nSa        :  alpha (not specified) single bond at n-atom
%           nSb        :  beta (not specified) single bond at n-atom
% \end{verbatim}
%
% Several examples are shown as follows.
% \begin{verbatim}
%       e.g. 
%        
%        \cyclopentanev{1==H;2==F}
%        \cyclopentanev[]{1==H;2==F}
%        \cyclopentanev[H]{1==H;2==F}
% \end{verbatim}
%
% \begin{macro}{\cyclopentanev}
% \begin{macro}{\cyclopentanevi}
%    \begin{macrocode}
\def\cyclopentanev{\@ifnextchar[{\@cyclopentanev[@}{\@cyclopentanev[]}}
\def\@cyclopentanev[#1]#2{%
\iforigpt \typeout{command `cyclopentanev' %
                  is based on `fiveheterov'.}\fi%
\fiveheterov[#1]{}{#2}}
%    \end{macrocode}
%
%    \begin{macrocode}
\def\cyclopentanevi{\@ifnextchar[{\@cyclopentanevi[@}{\@cyclopentanevi[]}}
\def\@cyclopentanevi[#1]#2{%
\iforigpt \typeout{command `cyclopentanevi' %
                   is based on `fiveheterovi'.}\fi%
\fiveheterovi[#1]{}{#2}}
%    \end{macrocode}
% \end{macro}
% \end{macro}
%
% \subsection{Horizontal type}
%
% The macros |\cyclopentaneh| and |\cyclopentanehi| have an 
% argument |SUBSLIST| as well as an optional argument |BONDLIST|.  
% The macro |\cyclopentaneh| draws a five-membered ring with a 
% vertical left bond, while the macro |\cyclopentanehi| draws an 
% inverse ring with a vertical right bond. 
% \begin{verbatim}
% ********************************
% *  cyclopentane derivatives    *
% *  (horizontal type & inverse) *
% ********************************
%
%   \cyclopentaneh[BONDLIST]{SUBSLIST}
%   \cyclopentanehi[BONDLIST]{SUBSLIST}
% \end{verbatim}
%
% The |BONDLIST| argument contains one or more 
% characters selected from a to e, each of which indicates the presence of 
% an inner (endcyclic) double bond on the corresponding position. The 
% option `$n+$' ($n=1$--$5$) is used for designating a plus charge on 
% the $n$-carbon. 
% The option `$0+$'is used for designating a plus charge at the center 
% of the five-membered ring. 
% \begin{verbatim}
%  
%     BONDLIST: list of inner double bonds 
%
%           none       :  mother nucleus (fully saturated form)
%           a          :  1,2-double bond
%           b          :  2,3-double bond
%           c          :  4,3-double bond
%           d          :  4,5-double bond
%           e          :  5,1-double bond
%           {n+}       :  plus at the n-carbon atom (n = 1 to 5)
%           {0+}       :  plus at the center
% \end{verbatim}
%
% The |SUBSLIST| argument contains one or more substitution descriptors 
% which are separated from each other by a semicolon.  Each substitution 
% descriptor has a locant number with a bond modifier and a substituent, 
% where these are separated with a double equality symbol. 
% \begin{verbatim}
%
%     SUBSLIST: list of substituents
%
%       for n = 1 to 5 
%
%           nD         :  exocyclic double bond at n-atom
%           n or nS    :  exocyclic single bond at n-atom
%           nA         :  alpha single bond at n-atom
%           nB         :  beta single bond at n-atom
%           nSA        :  alpha single bond at n-atom (boldface)
%           nSB        :  beta single bond at n-atom (dotted line)
%           nSa        :  alpha (not specified) single bond at n-atom
%           nSb        :  beta (not specified) single bond at n-atom
% \end{verbatim}
%
% Several examples are shown as follows.
% \begin{verbatim}
%
%       e.g. 
%        
%        \cyclopentaneh{1==H;2==F}
%        \cyclopentanehi[]{1==H;2==F}
%        \cyclopentanehi[H]{1==H;2==F}
%
% \end{verbatim}
%
% \begin{macro}{\cyclopentaneh}
% \begin{macro}{\cyclopentanehi}
%    \begin{macrocode}
\def\cyclopentaneh{\@ifnextchar[{\@cyclopentaneh[@}{\@cyclopentaneh[]}}
\def\@cyclopentaneh[#1]#2{%
\iforigpt \typeout{command `cyclophentaneh' %
                   is based on `fiveheteroh'.}\fi%
\fiveheteroh[#1]{}{#2}}
%    \end{macrocode}
%
%    \begin{macrocode}
\def\cyclopentanehi{%
  \@ifnextchar[{\@cyclopentanehi[@}{\@cyclopentanehi[]}}
\def\@cyclopentanehi[#1]#2{%
\iforigpt \typeout{command `cyclopentanehi' %
                   is based on `fiveheterohi'.}\fi%
\fiveheterohi[#1]{}{#2}}
%    \end{macrocode}
% \end{macro}
% \end{macro}
%
% \section{Indan derivatives}
% \subsection{Vertical type}
%
% The macros |\indanev| and |\indanevi| have an 
% argument |SUBSLIST| as well as an optional argument |BONDLIST|.  
% The macro |\indanev| draws a five-membered ring with a 
% flat top bond, while the macro |\indanevi| draws an 
% inverse ring with a flat bottom bond. 
% \begin{verbatim}
% ***********************************************************
% * indane derivatives (fused six- and five-membered rings) *
% *  (vertical type & inverse)                              *
% ***********************************************************
%
%   \indanev[BONDLIST]{SUBSLIST}
%   \indanevi[BONDLIST]{SUBSLIST}
% \end{verbatim}
%
% The |BONDLIST| argument contains one or more 
% characters selected from a to j, each of which indicates the presence of 
% an inner (endcyclic) double bond on the corresponding position. 
% The option `A' typesets a six-membered ring with an aromatic circle. 
% The option `$n+$' ($n=1$--$7$) is used for designating a plus charge on 
% the $n$-carbon. 
% \begin{verbatim}
%
%     BONDLIST:  list of bonds 
%
%           none or r  :  aromatic six-membered ring
%           H or []    :  fully saturated form
%           a    :  1,2-double bond      b    :  2,3-double bond
%           c    :  3,3a-double bond     d    :  4,3a-double bond
%           e    :  4,5-double bond      f    :  5,6-double bond
%           g    :  6,7-double bond      h    :  7,7a-double bond
%           i    :  1,7a-double bond     j    :  3a,4a-double bond
%           A    :  aromatic circle
%           {n+}       :  plus at the n-nitrogen atom (n = 1 to 7)
% \end{verbatim}
%
% The |SUBSLIST| argument contains one or more substitution descriptors 
% which are separated from each other by a semicolon.  Each substitution 
% descriptor has a locant number with a bond modifier and a substituent, 
% where these are separated with a double equality symbol. 
% \begin{verbatim}
%
%     SUBSLIST: list of substituents (max 7 substitution positions)
%
%       for n = 1 to 8
%
%           nD         :  exocyclic double bond at n-atom
%           n or nS    :  exocyclic single bond at n-atom
%           nA         :  alpha single bond at n-atom
%           nB         :  beta single bond at n-atom
%           nSA        :  alpha single bond at n-atom (boldface)
%           nSB        :  beta single bond at n-atom (dotted line)
%           nSa        :  alpha (not specified) single bond at n-atom
%           nSb        :  beta (not specified) single bond at n-atom
%
%       for 8 (3a position) and 9 (7a position)
%
%           nD         :  exocyclic double bond at n-atom
%           n or nS    :  exocyclic single bond at n-atom
%           nA         :  alpha single bond at n-atom
%           nB         :  beta single bond at n-atom
% \end{verbatim}
%
% Several examples are shown as follows.
% \begin{verbatim}
%
%       e.g. 
%        
%        \indanev{1==Cl;2==F}
%        \indanev[c]{1==Cl;4==F;2==CH$_{3}$}
%        \indanev[eb]{1D==O;4SA==MeO;4SB==OMe;5==Cl;6==Cl}
%
% \end{verbatim}
%
% \begin{macro}{\indanev}
% \begin{macro}{\indanevi}
%    \begin{macrocode}
\def\indanev{\@ifnextchar[{\@indanev[@}{\@indanev[rb]}}
\def\@indanev[#1]#2{%
\iforigpt \typeout{command `indanev' %
               is based on `nonaheterov'.}\fi%
\nonaheterov[#1]{}{#2}}
%    \end{macrocode}
%
%    \begin{macrocode}
\def\indanevi{\@ifnextchar[{\@indanevi[@}{\@indanevi[rb]}}
\def\@indanevi[#1]#2{%
\iforigpt \typeout{command `indanevi' %
              is based on `nonaheterovi'.}\fi%
\nonaheterovi[#1]{}{#2}}
%    \end{macrocode}
% \end{macro}
% \end{macro}
%
% \subsection{Horizontal type}
%
% The macros |\indaneh| and |\indanehi| have an 
% argument |SUBSLIST| as well as an optional argument |BONDLIST|.  
% The macro |\indaneh| draws a five-membered ring with a 
% vertical left bond, while the macro |\indanehi| draws an 
% inverse ring with a vertical right bond. 
% \begin{verbatim}
% ***********************************************************
% * indane derivatives (fused six- and five-membered rings) *
% *  (horizontal type & inverse)                            *
% ***********************************************************
%
%   \indaneh[BONDLIST]{SUBSLIST}
%   \indanehi[BONDLIST]{SUBSLIST}
% \end{verbatim}
%
% The |BONDLIST| argument contains one or more 
% characters selected from a to j, each of which indicates the presence of 
% an inner (endcyclic) double bond on the corresponding position. 
% The option `A' typesets a six-membered ring with an aromatic circle. 
% The option `$n+$' ($n=1$--$7$) is used for designating a plus charge on 
% the $n$-carbon. 
% \begin{verbatim}
%
%     BONDLIST:  list of bonds 
%
%           none or r  :  aromatic six-membered ring
%           H or []    :  fully saturated form
%           a    :  1,2-double bond      b    :  2,3-double bond
%           c    :  3,3a-double bond     d    :  4,3a-double bond
%           e    :  4,5-double bond      f    :  5,6-double bond
%           g    :  6,7-double bond      h    :  7,7a-double bond
%           i    :  1,7a-double bond     j    :  3a,4a-double bond
%           A    :  aromatic circle
%           {n+}       :  plus at the n-nitrogen atom (n = 1 to 7)
% \end{verbatim}
%
% The |SUBSLIST| argument contains one or more substitution descriptors 
% which are separated from each other by a semicolon.  Each substitution 
% descriptor has a locant number with a bond modifier and a substituent, 
% where these are separated with a double equality symbol. 
% \begin{verbatim}
%
%     SUBSLIST: list of substituents
%
%       for n = 1 to 8
%
%           nD         :  exocyclic double bond at n-atom
%           n or nS    :  exocyclic single bond at n-atom
%           nA         :  alpha single bond at n-atom
%           nB         :  beta single bond at n-atom
%           nSA        :  alpha single bond at n-atom (boldface)
%           nSB        :  beta single bond at n-atom (dotted line)
%           nSa        :  alpha (not specified) single bond at n-atom
%           nSb        :  beta (not specified) single bond at n-atom
%
%       for 8 (3a position) and 9 (7a position)
%
%           nD         :  exocyclic double bond at n-atom
%           n or nS    :  exocyclic single bond at n-atom
%           nA         :  alpha single bond at n-atom
%           nB         :  beta single bond at n-atom
% \end{verbatim}
%
% Several examples are shown as follows.
% \begin{verbatim}
%       e.g. 
%        
%        \indaneh{1==Cl;2==F}
%        \indaneh[c]{1==Cl;4==F;2==CH$_{3}$}
%        \indaneh[eb]{1D==O;4SA==MeO;4SB==OMe;5==Cl;6==Cl}
% \end{verbatim}
%
% \begin{macro}{\indaneh}
% \begin{macro}{\indanehi}
%    \begin{macrocode}
\def\indaneh{\@ifnextchar[{\@indaneh[@}{\@indaneh[rb]}}
\def\@indaneh[#1]#2{%
\iforigpt \typeout{command `indaneh' %
        is based on `nonaheteroh'.}\fi%
\nonaheteroh[#1]{}{#2}}
\def\indanehi{\@ifnextchar[{\@indanehi[@}{\@indanehi[rb]}}
\def\@indanehi[#1]#2{%
\iforigpt \typeout{command `indanehi' %
        is based on `nonaheterohi'.}\fi%
\nonaheterohi[#1]{}{#2}}
%    \end{macrocode}
% \end{macro}
% \end{macro}
%
%
% \section{Cyclobutane derivatives}
%
% The macros |\cyclobutane| has an 
% argument |SUBSLIST| as well as an optional argument |BONDLIST|.  
% This macro is based on the \verb/\fourhetero/ command for drawing 
% four-membered heterocycles. 
% \begin{verbatim}
% ***************************
% * cyclobutane derivatives *
% ***************************
% The following numbering is adopted in this macro. 
%
%          c
%     4  _____  3
%    d  |     |  b
%       |     |
%     1  -----  2<===== the original point
%          a
%
%   \cyclobutane[BONDLIST]{SUBSLIST}
% \end{verbatim}
%
% The |BONDLIST| argument contains one or more 
% characters selected from a to e, each of which indicates the presence of 
% an inner (endcyclic) double bond on the corresponding position. 
% The option `$n+$' ($n=1$--$4$) is used for designating a plus charge on 
% the $n$-carbon. 
% \begin{verbatim}
%  
%     BONDLIST: list of inner double bonds 
%
%           a          :  1,2-double bond
%           b          :  2,3-double bond
%           c          :  4,3-double bond
%           e          :  4,1-double bond
%
%           {n+}       :  plus at the n-nitrogen atom (n = 1 to 4)
% \end{verbatim}
%
% The |SUBSLIST| argument contains one or more substitution descriptors 
% which are separated from each other by a semicolon.  Each substitution 
% descriptor has a locant number with a bond modifier and a substituent, 
% where these are separated with a double equality symbol. 
% \begin{verbatim}
%
%     SUBSLIST: list of substituents (max 5 substitution positions)
%
%       for n = 1 to 4 
%
%           nD         :  exocyclic double bond at n-atom
%           n or nS    :  exocyclic single bond at n-atom
%           nA         :  alpha single bond at n-atom
%           nB         :  beta single bond at n-atom
%           nSA        :  alpha single bond at n-atom (boldface)
%           nSB        :  beta single bond at n-atom (dotted line)
%           nSa        :  alpha (not specified) single bond at n-atom
%           nSb        :  beta (not specified) single bond at n-atom
% \end{verbatim}
%
% Several examples are shown as follows.
% \begin{verbatim}
%       e.g. 
%        
%        \cyclobutane{1==H;2==F}
%        \cyclobutane[c]{1==Cl;4==F;2==CH$_{3}$}
%        \cyclobutane[eb]{1D==O;4SA==MeO;4SB==OMe}
% \end{verbatim}
%
% \begin{macro}{\cyclobutane}
%    \begin{macrocode}
\def\cyclobutane{\@ifnextchar[{\@cyclobutane[@}{\@cyclobutane[]}}
\def\@cyclobutane[#1]#2{%
\iforigpt \typeout{command `cyclobutane' %
      is based on `fourhetero'.}\fi%
\fourhetero[#1]{}{#2}}
%    \end{macrocode}
% \end{macro}
%
% \section{Cyclopropane derivatives}
%
% The macros |\cyclopropane| has an 
% argument |SUBSLIST| as well as an optional argument |BONDLIST|.  
% This macro is based on the \verb/\threehetero/ command for drawing 
% three-membered heterocycles. 
% \begin{verbatim}
% ****************************
% * cyclopropane derivatives *
% *  (vertical type)         *
% ****************************
% The following numbering is adopted in this macro. 
%
%         b
%     3--------2
%    c `     / a
%        `1/ <===== the original point
%
%
%   \cyclopropane[BONDLIST]{SUBSLIST}
% \end{verbatim}
%
% The |BONDLIST| argument contains one or more 
% characters selected from a to c, each of which indicates the presence of 
% an inner (endcyclic) double bond on the corresponding position. 
% The option `A' is used for designating an aromatic character with a 
% circle. 
% The option `$n+$' ($n=1$--$3$) is used for designating a plus charge on 
% the $n$-carbon, while 
% the case $n=4$--$6$) indicates a plus charge outside the ring. 
% The option `0+' draws a plus symbol at the center of the ring. 
% \begin{verbatim}
%  
%     BONDLIST = 
%
%           none       :  saturated 
%           a          :  1,2-double bond
%           b          :  2,3-double bond
%           c          :  3,1-double bond
%           A          :  aromatic circle 
%           {n+}       :  plus at the n-hetero atom (n = 1 to 3)
%                      :  n=4 -- outer plus at 1 position
%                      :  n=5 -- outer plus at 2 position
%                      :  n=6 -- outer plus at 3 position
%           {0+}       :  plus at the center of a cyclopropane ring
% \end{verbatim}
%
% The |SUBSLIST| argument contains one or more substitution descriptors 
% which are separated from each other by a semicolon.  Each substitution 
% descriptor has a locant number with a bond modifier and a substituent, 
% where these are separated with a double equality symbol. 
% \begin{verbatim}
%
%     SUBSLIST: list of substituents (max 3 substitution positions)
%
%       for n = 1 to 3 
%
%           nD         :  exocyclic double bond at n-atom
%           n or nS    :  exocyclic single bond at n-atom
%           nA         :  alpha single bond at n-atom
%           nB         :  beta single bond at n-atom
%           nSA        :  alpha single bond at n-atom (boldface)
%           nSB        :  beta single bond at n-atom (dotted line)
%           nSa        :  alpha (not specified) single bond at n-atom
%           nSb        :  beta (not specified) single bond at n-atom
% \end{verbatim}
%
% Several examples are shown as follows.
% \begin{verbatim}
%
%       e.g. 
%        
%        \cyclopropane{1==N}{1==Cl;2==F}
%
% \end{verbatim}
%
% \begin{macro}{\cyclopropane}
% \begin{macro}{\cyclopropanev}
%    \begin{macrocode}
\def\cyclopropane{\@ifnextchar[{\@cyclopropane[@}{\@cyclopropane[]}}
\def\@cyclopropane[#1]#2{%
\iforigpt \typeout{command `cyclopropane' %
      is based on `threehetero'.}\fi%
\threehetero[#1]{}{#2}}
\let\cyclopropanev=\cyclopropane
%    \end{macrocode}
% \end{macro}
% \end{macro}
%
% The macros |\cyclopropanei| has an 
% argument |SUBSLIST| as well as an optional argument |BONDLIST|.  
% This macro is based on the \verb/\threehetero/ command for drawing 
% three-membered heterocycles. 
% \begin{verbatim}
% ****************************
% * cyclopropane derivatives *
% *  (inverse vertical type) *
% ****************************
% The following numbering is adopted in this macro. 
%
%        /1` <===== the original point
%    c /    ` a
%     3--------2
%         b
%
%   \cyclopropanei[BONDLIST]{SUBSLIST}
% \end{verbatim}
%
% The |BONDLIST| argument contains one or more 
% characters selected from a to c, each of which indicates the presence of 
% an inner (endcyclic) double bond on the corresponding position. 
% The option `A' is used for designating an aromatic character with a 
% circle. 
% The option `$n+$' ($n=1$--$3$) is used for designating a plus charge on 
% the $n$-carbon, while 
% the case $n=4$--$6$) indicates a plus charge outside the ring. 
% The option `0+' draws a plus symbol at the center of the ring. 
% \begin{verbatim}
%  
%     BONDLIST = 
%
%           none       :  saturated 
%           a          :  1,2-double bond
%           b          :  2,3-double bond
%           c          :  3,1-double bond
%           A          :  aromatic circle 
%           {n+}       :  plus at the n-hetero atom (n = 1 to 3)
%                      :  n=4 -- outer plus at 1 position
%                      :  n=5 -- outer plus at 2 position
%                      :  n=6 -- outer plus at 3 position
%           {0+}       :  plus at the center of a cyclopropane ring
% \end{verbatim}
%
% The |SUBSLIST| argument contains one or more substitution descriptors 
% which are separated from each other by a semicolon.  Each substitution 
% descriptor has a locant number with a bond modifier and a substituent, 
% where these are separated with a double equality symbol. 
% \begin{verbatim}
%
%     SUBSLIST: list of substituents (max 3 substitution positions)
%
%       for n = 1 to 3 
%
%           nD         :  exocyclic double bond at n-atom
%           n or nS    :  exocyclic single bond at n-atom
%           nA         :  alpha single bond at n-atom
%           nB         :  beta single bond at n-atom
%           nSA        :  alpha single bond at n-atom (boldface)
%           nSB        :  beta single bond at n-atom (dotted line)
%           nSa        :  alpha (not specified) single bond at n-atom
%           nSb        :  beta (not specified) single bond at n-atom
% \end{verbatim}
%
% Several examples are shown as follows.
% \begin{verbatim}
%
%       e.g. 
%        
%        \cyclopropanei{1==N}{1==Cl;2==F}
%
% \end{verbatim}
%   \changes{v1.02}{1998/10/20}{Newly added command}
%
% \begin{macro}{\cyclopropanei}
% \begin{macro}{\cyclopropenevi}
%    \begin{macrocode}
\def\cyclopropanei{\@ifnextchar[{\@cyclopropanei[@}{\@cyclopropanei[]}}
\def\@cyclopropanei[#1]#2{%
\iforigpt \typeout{command `cyclopropanei' %
      is based on `threeheteroi'.}\fi%
\threeheteroi[#1]{}{#2}}
\let\cyclopropanevi=\cyclopropanei
%    \end{macrocode}
% \end{macro}
% \end{macro}
%
% \begin{macro}{\cyclopropaneh}
%    \begin{macrocode}
\def\cyclopropaneh{\@ifnextchar[{\@cyclopropaneh[@}{\@cyclopropaneh[]}}
\def\@cyclopropaneh[#1]#2{%
\iforigpt \typeout{command `cyclopropaneh' %
      is based on `threeheteroh'.}\fi%
\threeheteroh[#1]{}{#2}}
%    \end{macrocode}
% \end{macro}
%
% \begin{macro}{\cyclopropanehi}
%    \begin{macrocode}
\def\cyclopropanehi{\@ifnextchar[{\@cyclopropanehi[@}{\@cyclopropanehi[]}}
\def\@cyclopropanehi[#1]#2{%
\iforigpt \typeout{command `cyclopropanehi' %
      is based on `threeheterohi'.}\fi%
\threeheterohi[#1]{}{#2}}
%</lowcycle>
%    \end{macrocode}
% \end{macro}
%
% \Finale
%
\endinput
