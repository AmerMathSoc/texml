% \iffalse meta-comment
%
% Extracted from makebase.xml
% makebase.dtx is copyright © 2016 by Peter Flynn <peter@silmaril.ie>
%
% This work may be distributed and/or modified under the
% conditions of the LaTeX Project Public License, either
% version 1.3 of this license or (at your option) any later
% version. The latest version of this license is in:
%
%     http://www.latex-project.org/lppl.txt
%
% and version 1.3 or later is part of all distributions of
% LaTeX version 2005/12/01 or later.
%
% This work has the LPPL maintenance status `maintained'.
% 
% The current maintainer of this work is Peter Flynn <peter@silmaril.ie>
%
% This work consists of the files makebase.dtx and makebase.ins,
% the derived file makebase.sty, and any ancillary files listed
% in the MANIFEST.
%
% \fi
% \iffalse
%<package>\NeedsTeXFormat{LaTeX2e}[2014/09/29]
%<package>\ProvidesPackage{makebase}[2016/05/10 v0.2
%<package> Typeset counters in a different base]
%<*driver>
\RequirePackage{fix-cm}
\PassOptionsToPackage{svgnames}{xcolor}
\PassOptionsToPackage{hyphens}{url}
\documentclass[12pt]{ltxdoc}
%%
%% Packages added automatically
%%
\usepackage{classpack}% included by default. (0)
\usepackage{mflogo}% included by default. (23)
\usepackage[british]{babel}% included by default. (31)
\usepackage{ccaption}% included by default. (34)
\captionnamefont{\bfseries}
\captionstyle{\raggedright}
\usepackage{fancyvrb}% use of 'programlisting' was detected (41)
\usepackage{makeidx}% included by default. (57)
\makeindex
%%
%% Packages specified by author
%%
\let\SavedShow\show
\usepackage[utf8x]{inputenc}[2008/03/30]
\AtBeginDocument{\let\show\SavedShow}
\DeclareUnicodeCharacter{9251}{\textvisiblespace}
\PrerenderUnicode{–}
\PrerenderUnicode{š}
\PrerenderUnicode{ć}
\PrerenderUnicode{Å}
\PrerenderUnicode{ı}
\PrerenderUnicode{É}
\usepackage[T1]{fontenc}
\usepackage{dox}[2010/12/16]
\makeatletter
\doxitem[idxtype=attribute]{Attribute}{CPK@attribute}{attributes}
\makeatother
\makeatletter
\doxitem[idxtype=attributevalue]{AttributeValue}{CPK@attributevalue}{attribute values}
\makeatother
\makeatletter
\doxitem[idxtype=class]{Class}{CPK@class}{classes}
\makeatother
\makeatletter
\doxitem[idxtype=colour]{Colour}{CPK@colour}{colours}
\makeatother
\makeatletter
\doxitem[idxtype=counter]{Counter}{CPK@counter}{counters}
\makeatother
\makeatletter
\doxitem[idxtype=DTD]{DTD}{CPK@dtd}{DTDs/Schemas}
\makeatother
\makeatletter
\doxitem[idxtype=element]{Element}{CPK@element}{element types}
\makeatother
\makeatletter
\doxitem[idxtype=entity]{Entity}{CPK@entity}{entities}
\makeatother
\makeatletter
\doxitem[idxtype=error]{Error}{CPK@error}{errors}
\makeatother
\makeatletter
\doxitem[idxtype=function]{Function}{CPK@function}{functions}
\makeatother
\makeatletter
\doxitem[idxtype=language]{Language}{CPK@language}{languages}
\makeatother
\makeatletter
\doxitem[macrolike,idxtype=length]{Length}{CPK@length}{lengths}
\makeatother
\makeatletter
\doxitem[idxtype=mode]{Mode}{CPK@mode}{modes}
\makeatother
\makeatletter
\doxitem[idxtype=option]{Option}{CPK@option}{options}
\makeatother
\makeatletter
\doxitem[idxtype=package]{Package}{CPK@package}{packages}
\makeatother
\makeatletter
\doxitem[idxtype=template]{Template}{CPK@template}{templates}
\makeatother
\makeatletter
\doxitem[idxtype=typeface]{Typeface}{CPK@typeface}{typefaces}
\makeatother
\makeatletter
\doxitem[idxtype=font]{Font}{CPK@font}{fonts}
\makeatother
\newcommand{\LabelFont}[2][\relax]{\strut
    {\fontencoding\encodingdefault
     \fontfamily{lmtt}\fontseries{lc}#1\selectfont#2}\space}
\makeatletter
\let\CPK@macro\macro\let\CPK@endmacro\endmacro
\makeatother
\makeatletter
\let\CPK@environment\environment\let\CPK@endenvironment\endenvironment
\makeatother
\makeatletter
\def\PrintAttributeName#1{\LabelFont{@#1}}
\makeatother
\def\PrintAttributeValueName#1{\LabelFont{"#1"}}
\def\PrintClassName#1{\LabelFont[\fontfamily{lmss}]{#1}}
\def\PrintColourName#1{\LabelFont[\color{#1}]{#1}}
\def\PrintCounterName#1{\LabelFont{#1}}
\def\PrintDTDName#1{\LabelFont{#1}}
\def\PrintElementName#1{\LabelFont{<#1>}}
\def\PrintEntityName#1{\LabelFont{\&#1;}}
\def\PrintEnvironmentName#1{\LabelFont[\fontfamily{lmss}]{#1}}
\def\PrintErrorName#1{\LabelFont[\color{Red}!]{#1}}
\def\PrintFunctionName#1{\LabelFont[\bfseries\itshape]{#1}}
\def\PrintLanguageName#1{\LabelFont{#1}}
\def\PrintLengthName#1{\LabelFont{#1}}
\def\PrintMacroName#1{\LabelFont{#1}}
\def\PrintModeName#1{\LabelFont[\sffamily]{\textlangle#1\textrangle}}
\def\PrintOptionName#1{\LabelFont[\bfseries]{#1}}
\def\PrintPackageName#1{\LabelFont[\fontfamily{lmss}]{#1}}
\def\PrintTemplateName#1{\LabelFont[\bfseries]{#1}}
\usepackage[textwidth=159mm,
  textheight=229mm,a4paper,left=1in,right=1in,
  textwidth=160mm,textheight=9in]{geometry}
\usepackage{url}
\usepackage{parskip}
\usepackage{varioref}
\vrefwarning
\labelformat{chapter}{Chapter~#1}
\makeatletter
\labelformat{chapter}{\@chapapp~#1}
\makeatother
\labelformat{section}{section~#1}
\labelformat{subsection}{section~#1}
\labelformat{subsubsection}{section~#1}
\labelformat{paragraph}{section~#1}
\labelformat{figure}{Figure~#1}
\labelformat{table}{Table~#1}
\labelformat{item}{item~#1}
\renewcommand{\reftextcurrent}{elsewhere on this
	    page}
\def\reftextafter{on the
	    \reftextvario{next}{following} page}
\usepackage{lmodern}
\usepackage{calc}
\usepackage{fmtcount}
\usepackage{listings}
\lstdefinelanguage{dummy}
	    {morekeywords={dummy}}
	  
\lstdefinelanguage{DocBook}[]{XML}
	    {morekeywords={abstract,address,affiliation,annotation,arg,
	    author,book,chapter,classname,cmdsynopsis,command,
	    constraintdef,contrib,copyright,cover,date,email,emphasis,
	    envar,filename,firstname,footnote,guibutton,guilabel,
	    guimenu,guimenuitem,guisubmenu,holder,info,itemizedlist,
	    listitem,literal,member,option,orderedlist,orgdiv,orgname,
	    package,para,parameter,part,personname,phrase,procedure,
	    productname,programlisting,quote,refsection,remark,
	    constructorsynopsis,methodparan,modifier,funcparams,olink,
	    bibliography,biblioentry,biblioset,subtitle,artpagenums,
	    volumenum,issuenum,DOCTYPE,SYSTEM,xml:id,releaseinfo,
	    replaceable,revdescription,revhistory,revision,sect1,sect2,
	    sect3,sect4,seg,seglistitem,segmentedlist,segtitle,
	    simplelist,step,surname,systemitem,tag,term,title,uri,
	    userinput,variablelist,varlistentry,wordasword,xref,year,
	    xlink:href},}
	  
\makeatletter
\lstdefinelanguage{bash}
	    {morestring=[s]{[]},morekeywords={exit,logout,yes,no,@,
	    password,ssh,URL,cd,dvips,latex,ls,makeindex,man,mkdir,
	    pdflatex,sudo,texconfig,texdoc,updmap,xelatex}} 
	  
\makeatother
\lstdefinelanguage{APA}[]{XML}
	    {morekeywords={TTL}}
	  
\lstdefinelanguage{OOXML}[]{XML}
	    {morekeywords={w:p,w:pPr,w:pStyle,w:rPr,w:rFonts,
	    w:r,w:t,w:lang}}
	  
\lstdefinelanguage{SGML}[]{XML}
	    {morekeywords={sec,ttl}}
	  
\lstdefinelanguage{DTD}[]{XML}
	    {morekeywords={ELEMENT,ENTITY,ATTLIST,CDATA,ID,REQUIRED,
	    IMPLIED,PCDATA}}
	  
\lstdefinelanguage{Runoff}
	    {morekeywords={h1}}
	  
\lstdefinelanguage{GML}
	    {morekeywords={h1}}
	  
\lstdefinelanguage{Scribe}
	    {morekeywords={Heading},morestring=[s]{[]}}
	  
\lstdefinelanguage{RTF}[]{TeX}
	    {moretexcs={rtf,ansi,deff,adeflang,fonttbl,f,froman,fprq,
	    fcharset,f1,fswiss,falt,fnil,colortbl,red,green,blue,
	    stylesheet,s,snext,nowidctlpar,hyphen,hyphlead,hyphtrail,
	    hyphmax,cf,kerning,dbch,af,langfe,afs,alang,loch,fs,
	    pgndec,pard,plain,qc,sb,sa,keepn,b,ab,rtlch,ltrch,par}}

\lstdefinelanguage{TEI}[]{XML}
	    {morekeywords={TEI,TEI.2,teiHeader,fileDesc,sourceDesc,
	    titleStmt,title,author,editor,respStmt,resp,name,
	    editionStmt,edition,text,body,publicationStmt,publisher,
	    div,div1,placeName,lg,l,s,cl,phr,w,list,distinct,p,pb,
	    mls,div2,head,num,val,app,lem,rdg,q,sup,uncl,note,
	    DOCTYPE,SYSTEM,xml:id}}[keywords,comments,strings]
	  
\lstdefinelanguage{XSLT2}[]{XML}
	    {morekeywords={xsl:stylesheet,xsl:transform,
	    xsl:apply-imports,xsl:attribute-set,xsl:decimal-format,
	    xsl:import,xsl:include,xsl:key,xsl:namespace-alias,
	    xsl:output,xsl:param,
	    xsl:preserve-space,xsl:strip-space,xsl:template,
	    xsl:variable,xsl:character-map,xsl:function,
	    xsl:import-schema,xsl:param,xsl:variable,
	    xsl:apply-imports,xsl:apply-templates,xsl:attribute,
	    xsl:call-template,xsl:choose,xsl:comment,xsl:copy,
	    xsl:copy-of,xsl:element,xsl:fallback,xsl:for-each,
	    xsl:if,xsl:message,xsl:number,xsl:otherwise,
	    xsl:processing-instruction,xsl:text,xsl:value-of,
	    xsl:variable,xsl:when,xsl:with-param,xsl:sort,
	    xsl:for-each-group,xsl:next-match,xsl:analyze-string,
	    xsl:namespace,xsl:result-document,xsl:copy,
	    xsl:fallback,xsl:document,xsl:sequence,
	    xsl:matching-substring,xsl:non-matching-substring,
	    xsl:perform-sort,xsl:output-character},
	    alsodigit={-}}
	  
\lstdefinelanguage{LaTeXe}[LaTeX]{TeX}
	    {morekeywords = {selectlanguage,foreignlanguage,
	    textbrokenbar,textlangle,textrangle,subsection,url,
	    chapter,tableofcontents,part,subsubsection,paragraph,
	    subparagraph,maketitle,setlength,listoffigures,
	    listoftables,color,arraybackslash,includegraphics,
	    textcite,parencite,graphicspath,lstinline,
	    DeclareLanguageMapping,textcolor,definecolor,colorbox,
	    fcolorbox}}
	  
\lstdefinelanguage{BIBTeX}{
	    morekeywords = {title,author,edition,publisher,year,
	    address},
	    morestring=[b]",
	    }
	  
\lstdefinelanguage{Email}{
	    morekeywords={From,Subject,To,Date},
	    }
	  
\lstset{defaultdialect=LaTeXe,frame=single,
	    framesep=.5em,backgroundcolor=\color{AliceBlue},
	    rulecolor=\color{LightSteelBlue},framerule=1pt}
	  
\lstloadlanguages{LaTeXe,DocBook,XML,XSLT2,bash}
\lstnewenvironment{listingsdoc}
	    {\lstset{language={[LaTeX]TeX}}}
	    {}
\newcommand\basicdefault[1]{\footnotesize
	    \color{Black}\ttfamily#1}
	  
\lstset{basicstyle=\basicdefault{\spaceskip.5em}}
\lstset{literate=
	    {§}{{\S}}1
	    {©}{{\raisebox{.125ex}{\copyright}\enspace}}1
	    {«}{{\guillemotleft}}1
	    {»}{{\guillemotright}}1
	    {Á}{{\'A}}1
	    {Ä}{{\"A}}1
	    {É}{{\'E}}1
	    {Í}{{\'I}}1
	    {Ó}{{\'O}}1
	    {Ö}{{\"O}}1
	    {Ú}{{\'U}}1
	    {Ü}{{\"U}}1
	    {ß}{{\ss}}2
	    {à}{{\`a}}1
	    {á}{{\'a}}1
	    {ä}{{\"a}}1
	    {é}{{\'e}}1
	    {í}{{\'i}}1
	    {ó}{{\'o}}1
	    {ö}{{\"o}}1
	    {ú}{{\'u}}1
	    {ü}{{\"u}}1
	    {ı}{{\i}}1
	    {—}{{---}}1
	    {’}{{'}}1
	    {…}{{\dots}}1
	    {␣}{{\textvisiblespace}}1,
	    keywordstyle=\color{DarkGreen}\bfseries,
	    identifierstyle=\color{DarkRed},
	    commentstyle=\color{Gray}\upshape,
	    stringstyle=\color{DarkBlue}\upshape,
	    emphstyle=\color{Chocolate}\upshape,
	    showstringspaces=false,
	    columns=fullflexible,
	    keepspaces=true}
\usepackage{graphicx}
\usepackage[svgnames]{xcolor}
\makeatletter
\@ifundefined{T}{%
	    \newcommand{\T}[2]{{\fontencoding{T1}\selectfont#2}}}{}
\makeatother
\usepackage{sectsty}
\allsectionsfont{\sffamily}
\renewcommand*{\descriptionlabel}[1]{\hspace\labelsep
	    \sffamily\bfseries #1}
\usepackage{nicefrac}
\def\textonehalf{\ensuremath{\nicefrac12}}
\usepackage{fancybox}
\usepackage[inline]{enumitem}
\setlist[description]{style=unboxed}
\usepackage{array}
\newcommand{\classorpackage}{package}
\usepackage{classpack}
%%
%% Settings for docstrip and latexdoc 
%%
\EnableCrossrefs
\CodelineIndex
\RecordChanges
\begin{document}
  \DocInput{makebase.dtx}
\end{document}
%</driver>
% \fi
%
% \CheckSum{133}
%
% \CharacterTable
%  {Upper-case    \A\B\C\D\E\F\G\H\I\J\K\L\M\N\O\P\Q\R\S\T\U\V\W\X\Y\Z
%   Lower-case    \a\b\c\d\e\f\g\h\i\j\k\l\m\n\o\p\q\r\s\t\u\v\w\x\y\z
%   Digits        \0\1\2\3\4\5\6\7\8\9
%   Exclamation   \!     Double quote  \"     Hash (number) \#
%   Dollar        \$     Percent       \%     Ampersand     \&
%   Acute accent  \'     Left paren    \(     Right paren   \)
%   Asterisk      \*     Plus          \+     Comma         \,
%   Minus         \-     Point         \.     Solidus       \/
%   Colon         \:     Semicolon     \;     Less than     \<
%   Equals        \=     Greater than  \>     Question mark \?
%   Commercial at \@     Left bracket  \[     Backslash     \\
%   Right bracket \]     Circumflex    \^     Underscore    \_
%   Grave accent  \`     Left brace    \{     Vertical bar  \|
%   Right brace   \}     Tilde         \~}
% 
% \changes{v0.2}{2016/05/10}{Regenerated: Recreated package with new classpack code to create zip file to the CTAN standard.}
% \changes{v0.1}{2016/04/28}{Initial test: Rewritten from an earlier test.}
%
% \GetFileInfo{makebase.dtx}
%
% \DoNotIndex{\@,\@@par,\@beginparpenalty,\@empty}
% \DoNotIndex{\@flushglue,\@gobble,\@input}
% \DoNotIndex{\@makefnmark,\@makeother,\@maketitle}
% \DoNotIndex{\@namedef,\@ne,\@spaces,\@tempa}
% \DoNotIndex{\@tempb,\@tempswafalse,\@tempswatrue}
% \DoNotIndex{\@thanks,\@thefnmark,\@topnum}
% \DoNotIndex{\@@,\@elt,\@forloop,\@fortmp,\@gtempa,\@totalleftmargin}
% \DoNotIndex{\",\/,\@ifundefined,\@nil,\@verbatim,\@vobeyspaces}
% \DoNotIndex{\|,\~,\ ,\active,\advance,\aftergroup,\begingroup,\bgroup}
% \DoNotIndex{\mathcal,\csname,\def,\documentstyle,\dospecials,\edef}
% \DoNotIndex{\egroup}
% \DoNotIndex{\else,\endcsname,\endgroup,\endinput,\endtrivlist}
% \DoNotIndex{\expandafter,\fi,\fnsymbol,\futurelet,\gdef,\global}
% \DoNotIndex{\hbox,\hss,\if,\if@inlabel,\if@tempswa,\if@twocolumn}
% \DoNotIndex{\ifcase}
% \DoNotIndex{\ifcat,\iffalse,\ifx,\ignorespaces,\index,\input,\item}
% \DoNotIndex{\jobname,\kern,\leavevmode,\leftskip,\let,\llap,\lower}
% \DoNotIndex{\m@ne,\next,\newpage,\nobreak,\noexpand,\nonfrenchspacing}
% \DoNotIndex{\obeylines,\or,\protect,\raggedleft,\rightskip,\rm,\sc}
% \DoNotIndex{\setbox,\setcounter,\small,\space,\string,\strut}
% \DoNotIndex{\strutbox}
% \DoNotIndex{\thefootnote,\thispagestyle,\topmargin,\trivlist,\tt}
% \DoNotIndex{\twocolumn,\typeout,\vss,\vtop,\xdef,\z@}
% \DoNotIndex{\,,\@bsphack,\@esphack,\@noligs,\@vobeyspaces,\@xverbatim}
% \DoNotIndex{\`,\catcode,\end,\escapechar,\frenchspacing,\glossary}
% \DoNotIndex{\hangindent,\hfil,\hfill,\hskip,\hspace,\ht,\it,\langle}
% \DoNotIndex{\leaders,\long,\makelabel,\marginpar,\markboth,\mathcode}
% \DoNotIndex{\mathsurround,\mbox,\newcount,\newdimen,\newskip}
% \DoNotIndex{\nopagebreak}
% \DoNotIndex{\parfillskip,\parindent,\parskip,\penalty,\raise,\rangle}
% \DoNotIndex{\section,\setlength,\TeX,\topsep,\underline,\unskip,\verb}
% \DoNotIndex{\vskip,\vspace,\widetilde,\\,\%,\@date,\@defpar}
% \DoNotIndex{\[,\{,\},\]}
% \DoNotIndex{\count@,\ifnum,\loop,\today,\uppercase,\uccode}
% \DoNotIndex{\baselineskip,\begin,\tw@}
% \DoNotIndex{\a,\b,\c,\d,\e,\f,\g,\h,\i,\j,\k,\l,\m,\n,\o,\p,\q}
% \DoNotIndex{\r,\s,\t,\u,\v,\w,\x,\y,\z,\A,\B,\C,\D,\E,\F,\G,\H}
% \DoNotIndex{\I,\J,\K,\L,\M,\N,\O,\P,\Q,\R,\S,\T,\U,\V,\W,\X,\Y,\Z}
% \DoNotIndex{\1,\2,\3,\4,\5,\6,\7,\8,\9,\0}
% \DoNotIndex{\!,\#,\$,\&,\',\(,\),\+,\.,\:,\;,\<,\=,\>,\?,\_}
% \DoNotIndex{\discretionary,\immediate,\makeatletter,\makeatother}
% \DoNotIndex{\meaning,\newenvironment,\par,\relax,\renewenvironment}
% \DoNotIndex{\repeat,\scriptsize,\selectfont,\the,\undefined}
% \DoNotIndex{\arabic,\do,\makeindex,\null,\number,\show,\write,\@ehc}
% \DoNotIndex{\@author,\@ehc,\@ifstar,\@sanitize,\@title,\everypar}
% \DoNotIndex{\if@minipage,\if@restonecol,\ifeof,\ifmmode}
% \DoNotIndex{\lccode,\newtoks,\onecolumn,\openin,\p@,\SelfDocumenting}
% \DoNotIndex{\settowidth,\@resetonecoltrue,\@resetonecolfalse,\bf}
% \DoNotIndex{\clearpage,\closein,\lowercase,\@inlabelfalse}
% \DoNotIndex{\selectfont,\mathcode,\newmathalphabet,\rmdefault}
% \DoNotIndex{\bfdefault,\DeclareRobustCommand}
% \DoNotIndex{\classorpackage}
% \DoNotIndex{\Sil@MB@maxdiv}
% \DoNotIndex{\makebase}
% \DoNotIndex{\divide}
% \DoNotIndex{\multiply}
% \DoNotIndex{\repeat}
% \setcounter{tocdepth}{5}
% \setcounter{secnumdepth}{5}
% \makeatletter
% \def\@@doxdescribe#1#2{\endgroup \ifdox@noprint\else\marginpar{\raggedleft \textcolor{DarkRed}{\@nameuse{PrintDescribe#1}{#2}}}\fi \ifdox@noindex\else\@nameuse{Special#1Index}{#2}\fi \endgroup\@esphack\ignorespaces}
% \makeatother
%
% \def\fileversion{0.2}
% \def\filedate{2016/05/10}
% \title{The  \textsf{makebase} \LaTeXe\ package\thanks{%
% This document corresponds to \textsf{makebase}
% \textit{v.}\ \fileversion $\alpha$, dated \filedate.}
% \\[1em]\Large 
% Typeset counters in a different base}
% \author{Peter Flynn\\\normalsize Silmaril Consultants\\[-.25ex]\normalsize Textual Therapy Division\\\normalsize(\url{peter@silmaril.ie})}
% \maketitle
% \renewcommand{\abstractname}{Summary}\thispagestyle{empty}
% \begin{abstract}
% \parskip=0.5\baselineskip
% \advance\parskip by 0pt plus 2pt
% \parindent=0pt% \noindent
% This package typesets a \LaTeX{} counter such as {\ttfamily{}page} in an arbitrary base (default
% 16). It does not change font or typeface.\par
% The package extends the functionality of the existing
% \textsf{hex} \LaTeX{}2.09 package and provides
% documentation. However, the author is not a mathematician, and
%       suggestions for rewriting the code are welcomed.\par
% Warning: this is alpha software and may contain bugs.
% Please report problems to the author.\par
% \end{abstract}
% \clearpage
% \tableofcontents
% \clearpage
% \section{Description}
% This package was developed for an application which
% typeset smoke proofs of font characters from the Private Use
% areas of Unicode, where the codepoint is a very large number
% usually expressed in hexadecimal, and this was required for
% the page numbers.\par
% The package contains two macros:\par
% \begin{enumerate}
% \item {\ttfamily{}\textbackslash{}Sil@MB@maxdiv} finds the highest
%     power of the selected base which is smaller than the value
%     of the counter and makes it the primary divisor;
% \item {\ttfamily{}\textbackslash{}makebase} divides the counter
%     repeatedly by the divisor found in
%     {\ttfamily{}\textbackslash{}Sil@MB@maxdiv}, reducing it by one power
%     each time, and outputting the quotient as a digit.
% \end{enumerate}
% This has only been tested in page numbering so far (see
% the accompanying file
% {\ttfamily{}makebase-test.tex}, for example:\par
% \iffalse
%<*ignore>
% \fi
\begin{lstlisting}[language={[LaTeX]TeX}]
\setcounter{page}{57635}
\makebase{page}
\end{lstlisting}
% \iffalse
%</ignore>
% \fi
% will produce \texttt{0xE123} (the default base is
%       16). Using\par
% \iffalse
%<*ignore>
% \fi
\begin{lstlisting}[language={[LaTeX]TeX}]
\makebase[8]{page}
\end{lstlisting}
% \iffalse
%</ignore>
% \fi
% produces 0160443.\par
% Warning: this is alpha software and may contain bugs.
% Please report problems to the author.\par
% \StopEventually{\label{endcode}
%   \clearpage
%   \newgeometry{left=3cm}
%   \addcontentsline{toc}{section}{Change History}
%   \label{}
%   \PrintChanges
%   \clearpage
%   \label{codeindex}
%   \addcontentsline{toc}{section}{Index}
%   \PrintIndex}
% \addtolength{\revmarg}{\widthof{\LabelFont{Sil@MB@postnum}}}
% \newgeometry{left=\revmarg}
% \iffalse
%<*package>
% \fi
% \clearpage
% \section{Implementation}\label{implementation}
%\iffalse
%%
%% Packages required
%% 
% \fi
% \subsection{Packages required for the package itself}\label{stypackages}
% \begin{CPK@package}{calc}
% Required for calculations involving lengths or counters,
% such as changes to widths for margin adjustment.
% \iffalse
%% Required for calculations involving lengths or counters, such as changes to widths for margin adjustment.
% \fi
%    \begin{macrocode}
\RequirePackage{calc}
%    \end{macrocode}
%  \end{CPK@package}
% 
% \subsection{Counters}\label{counters}
% \begin{CPK@counter}{Sil@MB@prenum}
% The value of the counter before calculations start.\par
%    \begin{macrocode}
\newcounter{Sil@MB@prenum}
%    \end{macrocode}
% \end{CPK@counter}
% \begin{CPK@counter}{Sil@MB@postnum}
% The value of the counter after calculations.\par
%    \begin{macrocode}
\newcounter{Sil@MB@postnum}
%    \end{macrocode}
% \end{CPK@counter}
% \begin{CPK@counter}{Sil@MB@quot}
% The quotient (number of times the computed divisor
%     goes into the current value of the counter {\ttfamily{}Sil@MB@postnum}.\par
%    \begin{macrocode}
\newcounter{Sil@MB@quot}
%    \end{macrocode}
% \end{CPK@counter}
% \begin{CPK@counter}{Sil@MB@rem}
% The remainder after division.\par
%    \begin{macrocode}
\newcounter{Sil@MB@rem}
%    \end{macrocode}
% \end{CPK@counter}
% \begin{CPK@macro}{\Sil@MB@div}
% The divisor (initially the base).\par
%    \begin{macrocode}
\newcounter{Sil@MB@div}
%    \end{macrocode}
% \end{CPK@macro}
% \subsection{Macros}\label{macros}
% Two macros are defined: one internal for calculating the
%   largest dividor needed to start with, and one public, for
%   implementing the conversion.\par
% \begin{CPK@macro}{\Sil@MB@maxdiv}
% This is the internal (private) macro.\par
%    \begin{macrocode}
\newcommand{\Sil@MB@maxdiv}[2][16]{%
%    \end{macrocode}
% Record the counter value and the base (default
%     16)\par
%    \begin{macrocode}
  \setcounter{Sil@MB@div}{#1}%
  \setcounter{Sil@MB@prenum}{#2}%
%  \message{Testing \theSil@MB@prenum\space for divisibility
%	    by powers of \theSil@MB@div}%
%    \end{macrocode}
% It repeatedly divides the counter by the base using
%     \TeX{}'s {\ttfamily{}\textbackslash{}divide} primitive and then
%     re-multiplies it using {\ttfamily{}\textbackslash{}multiply} to see
%     if the result is zero yet, and if not, increases the base
%     by a power and tries again.\par
%    \begin{macrocode}
  \loop
%    \message{...dividing \theSil@MB@prenum\space by \theSil@MB@div}%
    \divide\c@Sil@MB@prenum by\c@Sil@MB@div
    \multiply\c@Sil@MB@prenum by\c@Sil@MB@div
    \multiply\c@Sil@MB@div by#1
    \ifnum\c@Sil@MB@prenum>0
  \repeat
%    \end{macrocode}
% Because the repeating condition above cannot include
%     anything other than the {\ttfamily{}\textbackslash{}repeat} command,
%     it is necessary to compute the increase of the power
%     \emph{before} the condition.\par
% We therefore need to decrease it on exit to
%     compensate, before decreasing it yet again to the previous
%     successful level, which is the one we want to use in the
%     actual computation as the initial divisor.\par
%    \begin{macrocode}
  \divide\c@Sil@MB@div by#1
  \divide\c@Sil@MB@div by#1
%  \message{Need to start at \theSil@MB@div.}%
}
%    \end{macrocode}
% \end{CPK@macro}
% \begin{CPK@macro}{\makebase}
% This is the user-level macro which performs the
%   conversion. It too defaults to base 16.\par
%    \begin{macrocode}
\newcommand{\makebase}[2][16]{%
%    \end{macrocode}
% The first action is to call the internal macro
%     {\ttfamily{}\textbackslash{}Sil@MB@maxdiv} to compute the initial
%     divisor as above.\par
%    \begin{macrocode}
  \Sil@MB@maxdiv[#1]{#2}%
%    \end{macrocode}
% The prefix \texttt{0x} is output if the base is
%     16; for base 8 the prefix is \texttt{0}. These should
%     probably be developed for other bases with an option to
%     use \TeX{}'s double-quote and single-quote prefixes.\par
%    \begin{macrocode}
  \ifnum#1=16 0x\else\ifnum#1=8 0\fi\fi
%    \end{macrocode}
% Set the initial values for before-and-after
%     calculation.\par
%    \begin{macrocode}
  \setcounter{Sil@MB@prenum}{#2}%
  \setcounter{Sil@MB@postnum}{#2}%
%  \message{Formatting \theSil@MB@prenum\space in base#1 
%           starting at \theSil@MB@div}%
%    \end{macrocode}
% Loop through the calculations. If the current divisor
%     is bigger than the counter, output a zero and decrement
%     the power.\par
%    \begin{macrocode}
  \loop
    \ifnum\c@Sil@MB@div>\c@Sil@MB@prenum
      0\divide\c@Sil@MB@div by#1
%      \message{skipped a power, divisor now \theSil@MB@div}%
    \fi
%    \end{macrocode}
% Perform the division and record the quotient, then
%   re-multiply the value so the remainder can be computed.\par
%    \begin{macrocode}
%    \message{Dividing \theSil@MB@postnum\space by \theSil@MB@div}%
    \divide\c@Sil@MB@postnum by\c@Sil@MB@div
%    \message{...got \theSil@MB@postnum}%
    \setcounter{Sil@MB@quot}{\value{Sil@MB@postnum}}%
%    \message{Multiplying \theSil@MB@postnum\space by \theSil@MB@div}%
    \multiply\c@Sil@MB@postnum by\c@Sil@MB@div
%    \message{...got \theSil@MB@postnum, setting remainder}%
    \setcounter{Sil@MB@rem}{\value{Sil@MB@prenum}-\value{Sil@MB@postnum}}%
%    \message{...got \theSil@MB@rem}%
%    \end{macrocode}
% Output the quotient, allowing for alphabetic
%   hexadecimal digits.\par
%    \begin{macrocode}
%   \par\dots divided \theSil@MB@prenum\ by \theSil@MB@div\ 
%   \theSil@MB@quot\ times (\theSil@MB@postnum) leaving \theSil@MB@rem:
    \ifcase\c@Sil@MB@quot 0\or 1\or 2\or 3\or 4\or 5\or 6\or
           7\or 8\or 9\or A\or B\or C\or D\or E\or F\else !Z\fi
%    \end{macrocode}
% As in the internal macro, make the resetting of the
%     next values for the
%     next loop \emph{before} testing the loop
%     condition. If the remainder is zero, exit.\par
%    \begin{macrocode}
    \setcounter{Sil@MB@prenum}{\value{Sil@MB@rem}}%
    \setcounter{Sil@MB@postnum}{\value{Sil@MB@rem}}%
    \divide\c@Sil@MB@div by#1
    \ifnum\c@Sil@MB@rem>0
  \repeat
}
%    \end{macrocode}
% \end{CPK@macro}
% \iffalse
%</package>
% \fi
% \appendix
% \iffalse
%<*testdoc>
% \fi
% \clearpage
% \section{Test of package}\label{testdoc}
% \iffalse
%% Test of package% \fi
%    \begin{macrocode}
\documentclass{article}
\usepackage{makebase}
\setcounter{page}{"E123}
\renewcommand{\thepage}{\ttfamily\makebase[16]{\value{page}}}
\begin{document}
This is page number \arabic{page}.
\end{document}
%    \end{macrocode}
% \iffalse
%</testdoc>
% \fi
% \newgeometry{left=3cm}
% \clearpage
% \section{The \LaTeX{} Project Public License}\label{LPPL:LPPL}
% \begin{quotation}\small\noindent
% Everyone is allowed to distribute verbatim copies of this
%       license document, but modification of it is not allowed.
% \end{quotation}
% \subsection{Preamble}\label{LPPL:Preamble}
% The \LaTeX{} Project Public License (\textsc{lppl})
%       is the primary license under which the \LaTeX{} kernel and the
%       base \LaTeX{} packages are distributed.\par
% You may use this license for any work of which you hold the
%       copyright and which you wish to distribute.  This license may be
%       particularly suitable if your work is \TeX{}-related (such as a
%       \LaTeX{} package), but it is written in such a way that you can
%       use it even if your work is unrelated to \TeX{}.\par
% The section \emph{Whether and How to Distribute Works under This
%       License}, below, gives instructions, examples, and
%       recommendations for authors who are considering distributing
%       their works under this license.\par
% This license gives conditions under which a work may be
%       distributed and modified, as well as conditions under which
%       modified versions of that work may be distributed.\par
% We, the \LaTeX{3} Project, believe that the conditions below
%       give you the freedom to make and distribute modified versions of
%       your work that conform with whatever technical specifications
%       you wish while maintaining the availability, integrity, and
%       reliability of that work.  If you do not see how to achieve your
%       goal while meeting these conditions, then read the document
%       {\ttfamily{}cfgguide.tex} and {\ttfamily{}modguide.tex} in the base \LaTeX{}
%       distribution for suggestions.\par
% \subsection{Definitions}\label{LPPL:Definitions}
% In this license document the following terms are used:\par
% \begin{description}[style=unboxed]
% \item[Work\thinspace:]Any work being distributed under this License.
% \item[Derived Work\thinspace:]Any work that under any applicable law is derived from
%     the Work.
% \item[Modification\thinspace:]Any procedure that produces a Derived Work under any
%     applicable law~--- for example, the production of a file
%     containing an original file associated with the Work or a
%     significant portion of such a file, either verbatim or
%     with modifications and/or translated into another
%     language.
% \item[Modify\thinspace:]To apply any procedure that produces a Derived Work
%     under any applicable law.
% \item[Distribution\thinspace:]Making copies of the Work available from one person to
%     another, in whole or in part.  Distribution includes (but
%     is not limited to) making any electronic components of the
%     Work accessible by file transfer protocols such as
%     \textsc{ftp} or \textsc{http} or by
%     shared file systems such as Sun's Network File System
%     (\textsc{nfs}).
% \item[Compiled Work\thinspace:]A version of the Work that has been processed into a
%     form where it is directly usable on a computer system.
%     This processing may include using installation facilities
%     provided by the Work, transformations of the Work, copying
%     of components of the Work, or other activities.  Note that
%     modification of any installation facilities provided by
%     the Work constitutes modification of the Work.
% \item[Current Maintainer\thinspace:]A person or persons nominated as such within the Work.
%     If there is no such explicit nomination then it is the
%     `Copyright Holder' under any applicable
%     law.
% \item[Base Interpreter\thinspace:]A program or process that is normally needed for
%     running or interpreting a part or the whole of the
%     Work.\par
% A Base Interpreter may depend on external components
%     but these are not considered part of the Base Interpreter
%     provided that each external component clearly identifies
%     itself whenever it is used interactively.  Unless
%     explicitly specified when applying the license to the
%     Work, the only applicable Base Interpreter is a
%     `\LaTeX{}-Format' or in the case of files
%     belonging to the `\LaTeX{}-format' a program
%     implementing the `\TeX{} language'.
% \end{description}
% \subsection{Conditions on Distribution and Modification}\label{LPPL:Conditions}
% \begin{enumerate}
% \item Activities other than distribution and/or modification
%   of the Work are not covered by this license; they are
%   outside its scope. In particular, the act of running the
%   Work is not restricted and no requirements are made
%   concerning any offers of support for the Work.
% \item \label{LPPL:item:distribute}You may distribute a complete, unmodified copy of the
%   Work as you received it.  Distribution of only part of the
%   Work is considered modification of the Work, and no right to
%   distribute such a Derived Work may be assumed under the
%   terms of this clause.
% \item You may distribute a Compiled Work that has been
%   generated from a complete, unmodified copy of the Work as
%   distributed under Clause~item~\ref{LPPL:item:distribute} above above, as
%   long as that Compiled Work is distributed in such a way that
%   the recipients may install the Compiled Work on their system
%   exactly as it would have been installed if they generated a
%   Compiled Work directly from the Work.
% \item \label{LPPL:item:currmaint}If you are the Current Maintainer of the Work, you may,
%   without restriction, modify the Work, thus creating a
%   Derived Work.  You may also distribute the Derived Work
%   without restriction, including Compiled Works generated from
%   the Derived Work.  Derived Works distributed in this manner
%   by the Current Maintainer are considered to be updated
%   versions of the Work.
% \item If you are not the Current Maintainer of the Work, you
%   may modify your copy of the Work, thus creating a Derived
%   Work based on the Work, and compile this Derived Work, thus
%   creating a Compiled Work based on the Derived Work.
% \item \label{LPPL:item:conditions}If you are not the Current Maintainer of the Work, you
%   may distribute a Derived Work provided the following
%   conditions are met for every component of the Work unless
%   that component clearly states in the copyright notice that
%   it is exempt from that condition.  Only the Current
%   Maintainer is allowed to add such statements of exemption to
%   a component of the Work.
% \begin{enumerate}
% \item If a component of this Derived Work can be a direct
%       replacement for a component of the Work when that
%       component is used with the Base Interpreter, then,
%       wherever this component of the Work identifies itself to
%       the user when used interactively with that Base
%       Interpreter, the replacement component of this Derived
%       Work clearly and unambiguously identifies itself as a
%       modified version of this component to the user when used
%       interactively with that Base Interpreter.
% \item Every component of the Derived Work contains
%       prominent notices detailing the nature of the changes to
%       that component, or a prominent reference to another file
%       that is distributed as part of the Derived Work and that
%       contains a complete and accurate log of the
%       changes.
% \item No information in the Derived Work implies that any
%       persons, including (but not limited to) the authors of
%       the original version of the Work, provide any support,
%       including (but not limited to) the reporting and
%       handling of errors, to recipients of the Derived Work
%       unless those persons have stated explicitly that they do
%       provide such support for the Derived Work.
% \item You distribute at least one of the following with
%       the Derived Work:
% \begin{enumerate}
% \item A complete, unmodified copy of the Work; if your
%   distribution of a modified component is made by
%   offering access to copy the modified component from
%   a designated place, then offering equivalent access
%   to copy the Work from the same or some similar place
%   meets this condition, even though third parties are
%   not compelled to copy the Work along with the
%   modified component;
% \item Information that is sufficient to obtain a
%   complete, unmodified copy of the Work.
% \end{enumerate}
% \end{enumerate}
% \item If you are not the Current Maintainer of the Work, you
%   may distribute a Compiled Work generated from a Derived
%   Work, as long as the Derived Work is distributed to all
%   recipients of the Compiled Work, and as long as the
%   conditions of Clause~item~\ref{LPPL:item:conditions} above, above, are met
%   with regard to the Derived Work.
% \item The conditions above are not intended to prohibit, and
%   hence do not apply to, the modification, by any method, of
%   any component so that it becomes identical to an updated
%   version of that component of the Work as it is distributed
%   by the Current Maintainer under Clause~item~\ref{LPPL:item:currmaint} above, above.
% \item Distribution of the Work or any Derived Work in an
%   alternative format, where the Work or that Derived Work (in
%   whole or in part) is then produced by applying some process
%   to that format, does not relax or nullify any sections of
%   this license as they pertain to the results of applying that
%   process.
% \item % \begin{enumerate}
% \item A Derived Work may be distributed under a different
%       license provided that license itself honors the
%       conditions listed in Clause~item~\ref{LPPL:item:conditions} above above, in
%       regard to the Work, though it does not have to honor the
%       rest of the conditions in this license.
% \item If a Derived Work is distributed under a different
%       license, that Derived Work must provide sufficient
%       documentation as part of itself to allow each recipient
%       of that Derived Work to honor the restrictions in
%       Clause~item~\ref{LPPL:item:conditions} above above, concerning
%       changes from the Work.
% \end{enumerate}
% \item This license places no restrictions on works that are
%   unrelated to the Work, nor does this license place any
%   restrictions on aggregating such works with the Work by any
%   means.
% \item Nothing in this license is intended to, or may be used
%   to, prevent complete compliance by all parties with all
%   applicable laws.
% \end{enumerate}
% \subsection{No Warranty}\label{LPPL:Warranty}
% There is no warranty for the Work.  Except when otherwise
%       stated in writing, the Copyright Holder provides the Work
%       `as is', without warranty of any kind, either
%       expressed or implied, including, but not limited to, the implied
%       warranties of merchantability and fitness for a particular
%       purpose.  The entire risk as to the quality and performance of
%       the Work is with you.  Should the Work prove defective, you
%       assume the cost of all necessary servicing, repair, or
%       correction.\par
% In no event unless required by applicable law or agreed to
%       in writing will The Copyright Holder, or any author named in the
%       components of the Work, or any other party who may distribute
%       and/or modify the Work as permitted above, be liable to you for
%       damages, including any general, special, incidental or
%       consequential damages arising out of any use of the Work or out
%       of inability to use the Work (including, but not limited to,
%       loss of data, data being rendered inaccurate, or losses
%       sustained by anyone as a result of any failure of the Work to
%       operate with any other programs), even if the Copyright Holder
%       or said author or said other party has been advised of the
%       possibility of such damages.\par
% \subsection{Maintenance of The Work}\label{LPPL:Maintenance}
% The Work has the status `author-maintained'
%       if the Copyright Holder explicitly and prominently states near
%       the primary copyright notice in the Work that the Work can only
%       be maintained by the Copyright Holder or simply that it is
%       `author-maintained'.\par
% The Work has the status `maintained' if there
%       is a Current Maintainer who has indicated in the Work that they
%       are willing to receive error reports for the Work (for example,
%       by supplying a valid e-mail address). It is not required for the
%       Current Maintainer to acknowledge or act upon these error
%       reports.\par
% The Work changes from status `maintained' to
%       `unmaintained' if there is no Current Maintainer,
%       or the person stated to be Current Maintainer of the work cannot
%       be reached through the indicated means of communication for a
%       period of six months, and there are no other significant signs
%       of active maintenance.\par
% You can become the Current Maintainer of the Work by
%       agreement with any existing Current Maintainer to take over this
%       role.\par
% If the Work is unmaintained, you can become the Current
%       Maintainer of the Work through the following steps:\par
% \begin{enumerate}
% \item Make a reasonable attempt to trace the Current
%   Maintainer (and the Copyright Holder, if the two differ)
%   through the means of an Internet or similar search.
% \item If this search is successful, then enquire whether the
%   Work is still maintained.
% \begin{enumerate}
% \item If it is being maintained, then ask the Current
%       Maintainer to update their communication data within one
%       month.
% \item \label{LPPL:item:intention}If the search is unsuccessful or no action to resume
%       active maintenance is taken by the Current Maintainer,
%       then announce within the pertinent community your
%       intention to take over maintenance.  (If the Work is a
%       \LaTeX{} work, this could be done, for example, by
%       posting to \url{news:comp.text.tex}.)
% \end{enumerate}
% \item % \begin{enumerate}
% \item If the Current Maintainer is reachable and agrees to
%       pass maintenance of the Work to you, then this takes
%       effect immediately upon announcement.
% \item \label{LPPL:item:announce}If the Current Maintainer is not reachable and the
%       Copyright Holder agrees that maintenance of the Work be
%       passed to you, then this takes effect immediately upon
%       announcement.
% \end{enumerate}
% \item \label{LPPL:item:change}If you make an `intention announcement'
%   as described in~item~\ref{LPPL:item:intention} above above and after three
%   months your intention is challenged neither by the Current
%   Maintainer nor by the Copyright Holder nor by other people,
%   then you may arrange for the Work to be changed so as to
%   name you as the (new) Current Maintainer.
% \item If the previously unreachable Current Maintainer becomes
%   reachable once more within three months of a change
%   completed under the terms of~item~\ref{LPPL:item:announce} above
%   or~item~\ref{LPPL:item:change} above, then that
%   Current
%   Maintainer must become or remain the Current Maintainer upon
%   request provided they then update their communication data
%   within one month.
% \end{enumerate}
% A change in the Current Maintainer does not, of itself,
%       alter the fact that the Work is distributed under the
%       \textsc{lppl} license.\par
% If you become the Current Maintainer of the Work, you should
%       immediately provide, within the Work, a prominent and
%       unambiguous statement of your status as Current Maintainer.  You
%       should also announce your new status to the same pertinent
%       community as in~item~\ref{LPPL:item:intention} above
%       above.\par
% \subsection{Whether and How to Distribute Works under This
%       License}\label{LPPL:Distribute}
% This section contains important instructions, examples, and
%       recommendations for authors who are considering distributing
%       their works under this license.  These authors are addressed as
%       `you' in this section.\par
% \subsubsection{Choosing This License or Another License}\label{LPPL:Choosing}
% If for any part of your work you want or need to use
% \emph{distribution} conditions that differ
% significantly from those in this license, then do not refer to
% this license anywhere in your work but, instead, distribute
% your work under a different license. You may use the text of
% this license as a model for your own license, but your license
% should not refer to the \textsc{lppl} or otherwise
% give the impression that your work is distributed under the
% \textsc{lppl}.\par
% The document {\ttfamily{}modguide.tex} in the base \LaTeX{}
% distribution explains the motivation behind the conditions of
% this license.  It explains, for example, why distributing
% \LaTeX{} under the \textsc{gnu} General Public
% License (\textsc{gpl}) was considered inappropriate.
% Even if your work is unrelated to \LaTeX{}, the discussion in
% {\ttfamily{}modguide.tex} may still be
% relevant, and authors intending to distribute their works
% under any license are encouraged to read it.\par
% \subsubsection{A Recommendation on Modification Without
% Distribution}\label{LPPL:WithoutDistribution}
% It is wise never to modify a component of the Work, even
% for your own personal use, without also meeting the above
% conditions for distributing the modified component.  While you
% might intend that such modifications will never be
% distributed, often this will happen by accident~--- you may
% forget that you have modified that component; or it may not
% occur to you when allowing others to access the modified
% version that you are thus distributing it and violating the
% conditions of this license in ways that could have legal
% implications and, worse, cause problems for the community. It
% is therefore usually in your best interest to keep your copy
% of the Work identical with the public one.  Many works provide
% ways to control the behavior of that work without altering any
% of its licensed components.\par
% \subsubsection{How to Use This License}\label{LPPL:HowTo}
% To use this license, place in each of the components of
% your work both an explicit copyright notice including your
% name and the year the work was authored and/or last
% substantially modified.  Include also a statement that the
% distribution and/or modification of that component is
% constrained by the conditions in this license.\par
% Here is an example of such a notice and statement:\par
% \iffalse
%<*ignore>
% \fi
\begin{lstlisting}[language={[LaTeX]TeX}]
%%% pig.dtx
%%% Copyright 2005 M. Y. Name
%%
%% This work may be distributed and/or modified under the
%% conditions of the LaTeX Project Public License, either version 1.3
%% of this license or (at your option) any later version.
%% The latest version of this license is in
%%   http://www.latex-project.org/lppl.txt
%% and version 1.3 or later is part of all distributions of LaTeX
%% version 2005/12/01 or later.
%%
%% This work has the LPPL maintenance status `maintained'.
%% 
%% The Current Maintainer of this work is M. Y. Name.
%%
%% This work consists of the files pig.dtx and pig.ins
%% and the derived file pig.sty.
\end{lstlisting}
% \iffalse
%</ignore>
% \fi
% Given such a notice and statement in a file, the
% conditions given in this license document would apply, with
% the `Work' referring to the three files
% {\ttfamily{}pig.dtx}, {\ttfamily{}pig.ins}, and {\ttfamily{}pig.sty} (the last being generated
% from {\ttfamily{}pig.dtx} using {\ttfamily{}pig.ins}), the `Base
%   Interpreter' referring to any
% `\LaTeX{}-Format', and both `Copyright
%   Holder' and `Current Maintainer'
% referring to the person
% M.~Y.~Name\index{!}.\par
% If you do not want the Maintenance section of
% \textsc{lppl} to apply to your Work, change
% `maintained' above into
% `author-maintained'. However, we recommend that
% you use `maintained' as the Maintenance
% section was added in order to ensure that your Work remains
% useful to the community even when you can no longer maintain
% and support it yourself.\par
% \subsubsection{Derived Works That Are Not Replacements}\label{LPPL:NotReplacements}
% Several clauses of the \textsc{lppl} specify
% means to provide reliability and stability for the user
% community. They therefore concern themselves with the case
% that a Derived Work is intended to be used as a (compatible or
% incompatible) replacement of the original Work. If this is not
% the case (e.g., if a few lines of code are reused for a
% completely different task), then clauses 6b and 6d shall not
% apply.\par
% \subsubsection{Important Recommendations}\label{LPPL:Recommendations}
% \paragraph{Defining What Constitutes the Work}
% The \textsc{lppl} requires that distributions
%   of the Work contain all the files of the Work.  It is
%   therefore important that you provide a way for the licensee
%   to determine which files constitute the Work.  This could,
%   for example, be achieved by explicitly listing all the files
%   of the Work near the copyright notice of each file or by
%   using a line such as:\par
% \iffalse
%<*ignore>
% \fi
\begin{lstlisting}[language={[LaTeX]TeX}]
%% This work consists of all files listed in manifest.txt.
\end{lstlisting}
% \iffalse
%</ignore>
% \fi
% in that place.  In the absence of an unequivocal list it
%   might be impossible for the licensee to determine what is
%   considered by you to comprise the Work and, in such a case,
%   the licensee would be entitled to make reasonable
%   conjectures as to which files comprise the Work.\par
% \Finale

