% -*- coding: utf-8; time-stamp-format: "%02d-%02m-%:y at %02H:%02M:%02S %Z" -*-
%<*none>
\def\dtxtimestamp {Time-stamp: <30-04-2016 at 20:34:58 CEST>}
%</none>
%<*!readme>
%%
%% Package: footnotehyper
%% Version: 0.9e (2016/04/30)
%% License: LPPL 1.3c
%% Copyright (C) 2016 Jean-Francois Burnol <jfbu at free dot fr>.
%%
%</!readme>
%<*insfile|tex>
\def\pkgname        {footnotehyper}
\def\pkgdate        {2016/04/30}
\def\docdate        {2016/04/30}
\def\pkgversion     {v0.9e}
\def\pkgdescription {hyperref aware footnote.sty (JFB)}
%</insfile|tex>
%<*none>
% Definition of \pkgLicense
\begingroup% cette méthode ne marcherait pas avec caractères en dehors de 32-127
    \long\def\firstofone #1{#1}\catcode1=14\catcode2=0
    \catcode`\%=12\catcode`\_=12\endlinechar13\catcode13=13 ^^A
    \catcode32=13\catcode`\\=12^^Brelax^^A
^^Bfirstofone{^^Bendgroup^^Bdef^^BpkgLicense^^A
{% Package: footnotehyper
% Version: 0.9e (2016/04/30)
% License: LPPL 1.3c
% Copyright (C) 2016 Jean-Francois Burnol <jfbu at free dot fr>.
%
% This Work may be distributed and/or modified under the conditions
% of the LaTeX Project Public License, version 1.3c. This version of
% this license is in:
%
% > <http://www.latex-project.org/lppl/lppl-1-3c.txt>
%
% and the latest version of this license is in:
%
% > <http://www.latex-project.org/lppl.txt>
%
% Version 1.3 or later is part of all distributions of
% LaTeX version 2005/12/01 or later.
%
% The Author of this Work is: Jean-Francois Burnol `<jfbu at free dot fr>`
%
% This Work consists of the main source file footnotehyper.dtx and the
% derived files footnotehyper.sty, footnotehyper.ins, footnotehyper.tex,
% footnotehyper.pdf, footnotehyper.dvi.
}}%
\begingroup\catcode1 0 \catcode`\\ 12 
^^Aiffalse
%</none>
%<*readme>
<!-- -->

    Source:  footnotehyper.dtx (v0.9e 2016/04/30)
    Author:  Jean-Francois Burnol
    Info:    hyperref aware footnote.sty
    License: LPPL 1.3c or later
    Copyright (C) 2016 Jean-Francois Burnol <jfbu at free dot fr>.


ABSTRACT
========

The `footnote` package by Mark Wooding dates back to 1997 and has
not been made `hyperref` compatible. The aim of the present
package is to do that. Its state is what I found sufficiently
well-working on a current LaTeX document of mine.

For those who don't know: the `footnote` package allows via
`\savenotes` to gather footnotes and later release them via
`\spewnotes` (one can also use a `savenotes` environment.) 
Thus footnotes emitted from tabulars or minipages
are not separated from the general document stream of footnotes,
and are printed with the others at bottom of page. This works also
for environments like `framed` (1) from the eponymous package and
avoids the `\footnotemark/\footnotetext` approach, which anyhow is
not immediately `hyperref` compatible. The `footnote` package has
a facility to patch any user-chosen environment to do the
`\savenotes/\spewnotes` automatically.

This package provides no facility for handling footnotes from floating
environments.

Some issues from `footnote.sty` made it incompatible with the
`color` and `xcolor` packages; this is corrected by `footnotehyper`. The
compatibility with `babel-frenchb` is improved, too.

The loading of `hyperref` (either before or after) is mandatory
but left to the user.

(1): in case of multi-page content, the footnotes are delivered in
the last page.


INSTALLATION
============

To extract the package (.sty) and driver (.tex) files from
footnotehyper.dtx:

- if footnotehyper.ins is present:   etex footnotehyper.ins
- without footnotehyper.ins:         etex footnotehyper.dtx
- or run latex or pdflatex directly on footnotehyper.dtx

At least three ways to produce footnotehyper.pdf (method (1) is
preferred):

1. latex footnotehyper.tex (twice), then dvipdfmx
2. pdflatex footnotehyper.dtx (twice)
3. latex footnotehyper.dtx (twice), then dvips, then ps2pdf

Method (1) produces the smallest pdf files.
Options can be set in footnotehyper.tex:

- scrdoc class options (paper size, font size, ...)
- with or without source code,
- with dvipdfmx or with latex+dvips or pdflatex.

Installation:

    footnotehyper.sty    -> TDS:tex/latex/footnotehyper/footnotehyper.sty
    footnotehyper.dtx    -> TDS:source/latex/footnotehyper/footnotehyper.dtx
    footnotehyper.pdf    -> TDS:doc/latex/footnotehyper/footnotehyper.pdf
    README.md            -> TDS:doc/latex/footnotehyper/README.md

The other files may be discarded.


LICENSE
=======

This Work may be distributed and/or modified under the conditions
of the LaTeX Project Public License, version 1.3c. This version of
this license is in:

> <http://www.latex-project.org/lppl/lppl-1-3c.txt>

and the latest version of this license is in:

> <http://www.latex-project.org/lppl.txt>

Version 1.3 or later is part of all distributions of
LaTeX version 2005/12/01 or later.

The Author of this Work is:

- Jean-Francois Burnol `<jfbu at free dot fr>`

This Work consists of the main source file footnotehyper.dtx and the
derived files footnotehyper.sty, footnotehyper.ins, footnotehyper.tex,
footnotehyper.pdf, footnotehyper.dvi.
%</readme>
%<*tex>-------------------------------------------------------------------------
%%
%% run latex twice on this file footnotehyper.tex then dvipdfmx on
%% footnotehyper.dvi to produce the documentation footnotehyper.pdf, with
%% source code included.
%%
\chardef\Withdvipdfmx 1 % replace 1 by 0 for using latex+dvips or pdflatex
\chardef\NoSourceCode 0 % replace 0 by 1 for the doc *without* the source code
\NeedsTeXFormat{LaTeX2e}
\ProvidesFile {\pkgname.tex}[Driver for \pkgname\space documentation]%
\PassOptionsToClass   {a4paper,fontsize=11pt,oneside}{scrdoc}
\PassOptionsToPackage {english}{babel}
\input \pkgname.dtx
%%% Local Variables:
%%% mode: latex
%%% End:
%</tex>-------------------------------------------------------------------------
%<*insfile>---------------------------------------------------------------------
%%
%% Run etex on this file to extract from footnotehyper.dtx:
%%        footnotehyper.sty, footnotehyper.tex, and README.md
%%
%% Refer to README.md for installation instructions, if needed.
%%
\input docstrip.tex
\askforoverwritefalse
\def\pkgpreamble{\defaultpreamble^^J\MetaPrefix^^J%
\string\NeedsTeXFormat{LaTeX2e}^^J%
\string\ProvidesPackage{\pkgname}\perCent^^J%
\space[\pkgdate\space\pkgversion\space\pkgdescription]}%
\generate{\nopreamble\nopostamble
\file{README.md}{\from{\pkgname.dtx}{readme}}%
\usepostamble\defaultpostamble
\file{\pkgname.tex}{\from{\pkgname.dtx}{tex}}%
\usepreamble\pkgpreamble
\file{\pkgname.sty}{\from{\pkgname.dtx}{package}}}%
\catcode32=13\relax% active space
\let =\space%
\Msg{************************************************************************}
\Msg{*}
\Msg{* To finish the installation you have to move the following}
\Msg{* file into a directory searched by TeX:}
\Msg{*}
\Msg{*     \pkgname.sty} 
\Msg{*}
\Msg{* To produce the documentation run latex twice on file \pkgname.tex}
\Msg{* and then run dvipdfmx on file \pkgname.dvi.}
\Msg{*}
\Msg{* Happy TeXing!}
\Msg{*}
\Msg{************************************************************************}
\ifx\numexpr\undefined
\Msg{* warning: to get correct utf-8 encoded README.md}%
\Msg{* do 'etex \pkgname.ins' and not 'tex \pkgname.ins'}%
\Msg{************************************************************************}
\fi
\endbatchfile
%</insfile>---------------------------------------------------------------------
%<*none>------------------------------------------------------------------------
^^Afi^^Aendgroup
%
\chardef\noetex 0
\ifx\numexpr\undefined\chardef\noetex 1 \fi
\ifnum\noetex=1 \chardef\extractfiles 0 % extract files, then stop
\else
    \ifx\ProvidesFile\undefined
      \chardef\extractfiles 0 % etex etc.. on \pkgname.dtx, only file extraction.
    \else % latex/pdflatex on \pkgname.tex or on \pkgname.dtx
      \ifx\Withdvipdfmx\undefined
        % latex/pdflatex run is on \pkgname.dtx
        \chardef\extractfiles 1 % 1 = extract files and typeset manual, 2 = only typeset
        \chardef\Withdvipdfmx 0 % 0 = pdflatex or latex+dvips, 1 = dvipdfmx
        \chardef\NoSourceCode 0 % 0 =  include source code, 1 = do not
        \NeedsTeXFormat {LaTeX2e}%
        \PassOptionsToClass   {a4paper,fontsize=11pt,oneside}{scrdoc}% 
        \PassOptionsToPackage {english}{babel}%
      \else % latex run is on \pkgname.tex
        \chardef\extractfiles 2 % do not extract files, only typeset
      \fi
      \ProvidesFile{\pkgname.dtx}%
        [\pkgname\space source and documentation (\dtxtimestamp)]%
    \fi
\fi
\ifnum\extractfiles<2 % extract files
\def\MessageDeFin{\newlinechar10 \let\Msg\message
\Msg{********************************************************************^^J}%
\Msg{*^^J}%
\Msg{* To finish the installation you have to move the following^^J}%
\Msg{* file into a directory searched by TeX:^^J}%
\Msg{*^^J}%
\Msg{*\space\space\space\space \pkgname.sty^^J}%
\Msg{*^^J}%
\Msg{* To produce the documentation with source code included run latex^^J}%
\Msg{* twice on file \pkgname.tex and then dvipdfmx on \pkgname.dvi^^J}%
\Msg{*^^J}%
\Msg{* Happy TeXing!^^J}%
\Msg{*^^J}%
\Msg{********************************************************************^^J}%
}%
\begingroup
    \input docstrip.tex
    \askforoverwritefalse
    \def\pkgpreamble{\defaultpreamble^^J\MetaPrefix^^J%
    \string\NeedsTeXFormat{LaTeX2e}^^J%
    \string\ProvidesPackage{\pkgname}\perCent^^J%
    \space[\pkgdate\space\pkgversion\space\pkgdescription]}%
    \generate{\nopreamble\nopostamble
    \file{README.md}{\from{\pkgname.dtx}{readme}}%
    \usepostamble\defaultpostamble
    \file{\pkgname.ins}{\from{\pkgname.dtx}{insfile}}%
    \file{\pkgname.tex}{\from{\pkgname.dtx}{tex}}%
    \usepreamble\pkgpreamble
    \file{\pkgname.sty}{\from{\pkgname.dtx}{package}}}%
\endgroup
\fi % end of file extraction (from etex/latex/pdflatex run on \pkgname.dtx)
\ifnum\noetex=1 % warning for README.md
    \expandafter\def\expandafter\MessageDeFin\expandafter
{\MessageDeFin
\Msg{* warning: to get correct utf-8 encoded README.md ^^J}%
\Msg{* do 'etex \pkgname.dtx' and not 'tex \pkgname.dtx' ^^J}%
\Msg{********************************************************************^^J}}%
\fi
\ifnum\extractfiles=0 % tex/etex/xetex/etc on \pkgname.dtx, files extracted, stop
      \MessageDeFin\expandafter\end
\fi
% From this point on, run is necessarily with e-TeX.
% Check if \MessageDeFin got defined, if yes put it at end of run.
\ifdefined\MessageDeFin\AtEndDocument{\MessageDeFin}\fi
%-------------------------------------------------------------------------------
% START OF USER MANUAL TEX SOURCE
\documentclass[abstract]{scrdoc}

\ifnum\NoSourceCode=1 \OnlyDescription\fi

\usepackage{ifpdf}
\ifpdf\chardef\Withdvipdfmx 0 \fi

\makeatletter
\ifnum\Withdvipdfmx=1
   \@for\@tempa:=hyperref,bookmark,graphicx,xcolor,pict2e\do
            {\PassOptionsToPackage{dvipdfmx}\@tempa}
   %
   \PassOptionsToPackage{dvipdfm}{geometry}
   \PassOptionsToPackage{bookmarks=true}{hyperref}
   \PassOptionsToPackage{dvipdfmx-outline-open}{hyperref}
   \PassOptionsToPackage{dvipdfmx-outline-open}{bookmark}
   %
   \def\pgfsysdriver{pgfsys-dvipdfm.def}
\else
   \PassOptionsToPackage{bookmarks=true}{hyperref}
\fi
\makeatother

\usepackage[T1]{fontenc}
\usepackage[utf8]{inputenc}
\usepackage {babel}

\usepackage[hscale=0.72,vscale=0.7]{geometry}

\def\MacroFont{\ttfamily\small\hyphenchar\font45 \baselineskip11pt\relax}

\usepackage{amsmath}
\usepackage{newtxtext, newtxmath}
\usepackage[straightquotes, scaled=0.97]{newtxtt}
\usepackage{xspace}

\usepackage[dvipsnames]{xcolor}
\definecolor{joli}{RGB}{225,95,0}
\definecolor{JOLI}{RGB}{225,95,0}
\definecolor{BLUE}{RGB}{0,0,255}
\colorlet{niceone}{green!35!blue!75}

\usepackage{framed}

\usepackage[pdfencoding=pdfdoc]{hyperref}
\hypersetup{%
linktoc=all,%
breaklinks=true,%
colorlinks,%   
linkcolor=RoyalBlue,%
urlcolor=OliveGreen,%
pdfauthor={Jean-Fran\c cois Burnol},%
pdftitle={The \pkgname\space package},%
pdfsubject={\pkgdescription},%
pdfkeywords={LaTeX, footnotes},%
pdfstartview=FitH,%
pdfpagemode=UseNone}
% added usage of package bookmark 2013/10/10
\usepackage{bookmark} 

\usepackage{\pkgname}

\newcommand\fnh{%
        \texorpdfstring{{\color{joli}\ttfamily\bfseries \pkgname}}{\pkgname}\xspace}

\frenchspacing

\renewcommand\familydefault\sfdefault
\pagestyle{headings}

\begin{document}\thispagestyle{empty}
% the \MacroFont valid at begin document will be the one used in Implementation
% the one defined here will be used by verbatim (not by \verb)
\def\MacroFont{\ttfamily\small}
\rmfamily
\bookmark[named=FirstPage,level=1]{Title page}

\begin{center}
  {\normalfont\LARGE The \fnh  package}\\
\textsc{\small Jean-François Burnol}\par
  \footnotesize \ttfamily 
  jfbu (at) free (dot) fr\par
  Package version: \pkgversion\ (\pkgdate)\par
  From source file \texttt{\pkgname.dtx} of \dtxtimestamp.\par
\end{center}

\MakeShortVerb{\`}%
\begin{abstract}
  The `footnote` package by \textsc{Mark Wooding} dates back to
  1997 and has not been made `hyperref` compatible. The aim of the
  present package is to do that. Its state is what I found
  sufficiently well-working on a current LaTeX document of mine.

  For those who don't know: the `footnote` package allows via
  `\savenotes` to gather footnotes and later release them via
  `\spewnotes` (one can also use a `savenotes` environment.) Thus
  footnotes emitted from tabulars or minipages are not separated
  from the general document stream of footnotes, and are printed
  with the others at bottom of page. This works also for
  environments like `framed`%
%
%\footnote
\space({in case of multi-page content, the footnotes are delivered in the last
page})
%
from the eponymous package and avoids the `\footnotemark/\footnotetext`
approach, which anyhow is not immediately `hyperref` compatible. The
`footnote` package has a facility to patch any user-chosen environment to
do the `\savenotes/\spewnotes` automatically.

Some issues from `footnote.sty` made it incompatible with the
`color` and `xcolor` packages; this is corrected by \fnh. The
compatibility with `babel-frenchb` has been improved, too.

The loading of `hyperref` (either before or after) is mandatory
but left to the user.
\end{abstract}
\DeleteShortVerb{\`}

\section{License}

\begingroup\ttfamily\footnotesize\hyphenchar\font -1
           \parindent0pt 
           \obeyspaces\obeylines %
\pkgLicense\endgroup

\section{Usage}

Do \emph{not} load |footnote|, leave that job to \fnh. You \emph{must} load
|hyperref|.\footnote{Since |0.9e| \fnh deactivates itself
  gracefully if |hyperref| is not loaded, or under |hyperref| option
  |hyperfootnotes=false|.}

Then you can use |\savenotes/\spewnotes| or the equivalent |savenotes|
environment; there is also a |footnote| environment, but its set-up during
\fnh loading is more delicate and it could well fail depending on how
|\@makefntext| has been customized by the class or other packages; a warning
is issued in that case. A functional |footnote| environment allows footnotes
with verbatim material.

\savenotes
\begin{framed}
Please refer to the documentation of the |footnote| package.%
\footnote{\url{http://ctan.org/pkg/footnote}}

Particularly you may check its |\makesavenoteenv| command.%
\footnote{This won't handle floating environments, though.}
\end{framed}
\spewnotes

We can try some normal footnote.\footnote{Here it is.}

\begin{savenotes}
\begin{framed}
\DeleteShortVerb{\|}\MakeShortVerb{\*}%
{\centering
  \begin{tabular}{|c|c|}
\hline
  \strut The above\footnote{Notice that if the present frame
    extended to next page,
    the end of the *savenotes* environment would then (try to) deliver its
    footnotes to that
    page.} was & coded\footnote{Alternatively a savenotes environment
    could have been used.} as:\\\hline
\end{tabular}\par}
\DeleteShortVerb{\*}\MakeShortVerb{\|}%
\begin{verbatim}
\savenotes
\begin{framed}
  Please refer to the documentation of the |footnote| package.%
  \footnote{\url{http://ctan.org/pkg/footnote}}

  Particularly you may check its |\makesavenoteenv| command.%
  \footnote{This won't handle floating environments, though.}
\end{framed}
\spewnotes
\end{verbatim}
and the present frame has \cs{footnote}'s from inside a |tabular| and is
inside a |savenotes| environment.% 
%
% plus maintenant, \small pour le \MacroFont utilisé par verbatim
%\footnote{Well, I end up on next page. Not
% much I can do about this, ask why to |framed.sty| and the output routine I
% guess.}
%
\footnote{Here is an issue which has nothing (as I finally figured out) to do
  with |footnote|, and only indirectly with \LaTeX: if you embed a
  \emph{full-width} minipage (with initial \cs{noindent}) in any environment
  not doing \cs{ignorespacesafterend}, be careful to add a \% after the
  minipage (or after the surrounding environment; or a \cs{par} immediately),
  else the output will have an extra blank line if the source has itself a
  blank line there. I hesitated adding that to
  \cs{spewnotes}/\cs{endsavenotes}. Finally I left as is.} Let's test an
|amsmath| environment with |\intertext|. As
\begin{align}
  E&=mc^2\;,
\intertext{was too easy,\footnote{There is also $E=h\nu$.}, let's
  try:}
  F&=nd^3\;.
\end{align}
\end{framed}
\end{savenotes}

Use of the \texttt{footnote} \emph{environment} with some verbatim material%
\begin{footnote}
  \verb|&$^%\[}$|
\end{footnote}
which was coded as:
\begin{verbatim}
Use of the \texttt{footnote} \emph{environment} with some verbatim material%
\begin{footnote}
  \verb|&$^%\[}$|
\end{footnote}
\end{verbatim}
And one more normal footnote.\footnote{\fnh deactivates itself if
  |hyperfootnotes=false| option to |hyperref| is detected.
  Essentially, it only
fixes then the incompatibilities between |footnote.sty| and
|color/xcolor/babel-frenchb| packages.}

\StopEventually{\end{document}\endinput}
\makeatletter
    \let\check@percent\original@check@percent
\makeatother

%\clearpage
\section{Implementation}

\small

\makeatletter
\noindent
\begingroup
\topsep\MacrocodeTopsep
\trivlist\parskip\z@\item[]
\macro@font
\leftskip\@totalleftmargin  \advance\leftskip\MacroIndent
\rightskip\z@  \parindent\z@  \parfillskip\@flushglue
\global\@newlistfalse \global\@minipagefalse
\ifcodeline@index
  \everypar{\global\advance\c@CodelineNo\@ne
  \llap{\theCodelineNo\ \hskip\@totalleftmargin}}%
\fi
\string\NeedsTeXFormat\string{LaTeX2e\string}\par
\string\ProvidesPackage\string{\pkgname\string}\@percentchar\par
\noindent\space [\pkgdate\space\pkgversion\space\pkgdescription]\par
\nointerlineskip
\global\@inlabelfalse
\endtrivlist
\endgroup
\makeatother

% The catcode hackery next is to avoid to have <*package> to be listed
% in the commented source code...
% (c) 2012/11/19 jf burnol ;-)

\MakePercentIgnore

%
% \catcode`\<=0 \catcode`\>=11 \catcode`\*=11 \catcode`\/=11
% \let</none>\relax
% \def<*package>{\catcode`\<=12 \catcode`\>=12 \catcode`\*=12 \catcode`\/=12}
%
%</none>
%<*package>
% \begin{macro}{no options}
%    \begin{macrocode}
\DeclareOption*{\PackageWarning{footnotehyper}{Option `\CurrentOption' is unknown}}%
\ProcessOptions\relax
%    \end{macrocode}
% \end{macro}
% \begin{macro}{\@makefntext}
% \begin{macro}{\footnote}
% \begin{macro}{\footnotetext}
% We load |footnote| but leave to the user to take care of |hyperref|.
%
% As we need to intercept some re-definitions done by |footnote| we will first
% check if it is already loaded.
%
% If \cs{@makefntext} has been customized and its argument is not visible at
% top level in its meaning, then loading |footnote.sty| will fail with a low
% level \TeX\ error. We save its meaning and replace it by an innocuous one
% for the time being. We will come back to this at begin document.
% 
% We also postpone to at begin document the redefinitions of \cs{footnote} and
% \cs{footnotetext}.
%    \begin{macrocode}
\@ifpackageloaded{footnote}
 {\PackageWarning{footnotehyper}{Please next time do not load footnote,^^J
    but leave that to me, that is much safer.}}
 {\let\FNH@@makefntext\@makefntext\let\@makefntext\@firstofone
  \RequirePackage{footnote}
  \let\@makefntext\FNH@@makefntext
 }%
\let\FNH@fn@footnote    \footnote    
\let\FNH@fn@footnotetext\footnotetext
\let\footnote    \fn@latex@@footnote
\let\footnotetext\fn@latex@@footnotetext
%    \end{macrocode}
% \end{macro}\end{macro}\end{macro}
% There are some |\let|'s done by |footnote.sty| in what appears to be
% premature ways.
%    \begin{macrocode}
\AtBeginDocument {%
    \let\fn@latex@@footnote    \footnote
    \let\fn@latex@@footnotetext\footnotetext
    \let\footnote    \FNH@fn@footnote
    \let\footnotetext\FNH@fn@footnotetext
}%
%    \end{macrocode}
% \begin{macro}{\FNH@inactive}
% We intervene (for real) only if |hyperfootnotes| option of |hyperref| applies.
%    \begin{macrocode}
\AtBeginDocument{\@ifpackageloaded{hyperref}
  {\ifHy@hyperfootnotes
    \let\fn@fntext \FNH@fn@fntext
    \let\spewnotes \FNH@spewnotes
    \let\endsavenotes\spewnotes
    \let\fn@endfntext\FNH@fn@endfntext
   \else
     \FNH@inactive
   \fi }\FNH@inactive 
    \let\endfootnote\fn@endfntext
    \let\endfootnotetext\endfootnote
}%
\def\FNH@fixendfntext\@finalstrut\strutbox\fn@postfntext
    {\@finalstrut\strutbox\fn@postfntext\fn@endnote}%
\def\FNH@inactive {%
    \expandafter\expandafter\expandafter\def
    \expandafter\expandafter\expandafter\fn@endfntext
    \expandafter\expandafter\expandafter
       {\expandafter\FNH@fixendfntext\fn@endfntext}%
    \PackageInfo{footnotehyper}{hyperref package not loaded,^^J
     or hyperfootnotes=false option; I did not activate myself and only^^J
     patched footnote.sty for color/xcolor/babel-frenchb compatibility}}%
%    \end{macrocode}
% \end{macro}
% \begin{macro}{\FNH@fn@fntext}
% Some |amsmath| complications. I change the coding, but same effect as
% original.
%    \begin{macrocode}
\def\FNH@fn@fntext {\ifx\ifmeasuring@\undefined\expandafter\@secondoftwo
                                      \else\expandafter\@firstofone\fi
    {\ifmeasuring@\expandafter\@gobbletwo\fi}%
    \FNH@fn@fntext@i }%
%    \end{macrocode}
% \end{macro}
% \begin{macro}{\FNH@fn@fntext@i}
% Note: original \cs{fn@fntext} had no \cs{long}, which looks wrong.
%
% We do the \cs{ifHy@nesting} test although hyperref's manual
% says ``Allows links to be nested; no drivers currently support this.''
%    \begin{macrocode}
\long\def\FNH@fn@fntext@i #1{\global\setbox\fn@notes\vbox
     {\unvbox\fn@notes
      \fn@startnote
      \@makefntext
       {\rule\z@\footnotesep\ignorespaces
        \ifHy@nesting\expandafter\ltx@firstoftwo
                \else\expandafter\ltx@secondoftwo
        \fi
        {\expandafter\hyper@@anchor\expandafter{\Hy@footnote@currentHref}{#1}}%
        {\Hy@raisedlink
          {\expandafter\hyper@@anchor\expandafter{\Hy@footnote@currentHref}%
          {\relax}}%
         \let\@currentHref\Hy@footnote@currentHref
         \let\@currentlabelname\@empty
         #1}%
      \@finalstrut\strutbox }%
      \fn@endnote }%
}%
%    \end{macrocode}
% \end{macro}
% \begin{macro}{\FNH@spewnotes}
% \begin{macro}{\FNH@endsavenotes}
%   The final touch in our hack is to patch the original \cs{spewnotes} for it
%   to use the original, non-hyperref modified, version of \cs{@footnotetext}.
%   And let's not forget \cs{endsavenotes} (done at begin document).
%    \begin{macrocode}
\def\FNH@spewnotes {\endgroup
    \if@savingnotes\else\ifvoid\fn@notes\else
    \begingroup\let\@makefntext\@empty
               \let\@finalstrut\@gobble
               \let\rule\@gobbletwo
               \H@@footnotetext{\unvbox\fn@notes}%
    \endgroup\fi\fi
}%
%    \end{macrocode}
% \end{macro}\end{macro}
% \begin{macro}{\FNH@fn@endfntext}
% \begin{macro}{\fn@endnote}
% We now take care of |footnote.sty|'s |footnote| environment. The original
% \cs{fn@endfntext} is lacking a \cs{fn@endnote}, and this meant that
% |footnote.sty| was incompatible with |color/xcolor| packages. Also this
% \cs{fn@endnote} was |\let| to |\color@endgroup| which is wrong.
%
% And we need our usual replacement of \cs{@footnotetext} by
% \cs{H@@footnotetext}. And this time (|v0.9d|) let's not forget the hyperlink
% target.
%    \begin{macrocode}
\def\fn@endnote  {\color@endgroup}%
\def\FNH@fn@endfntext{%
    \@finalstrut\strutbox
    \fn@postfntext
    \fn@endnote
    \egroup
    \begingroup
    \let\@makefntext\@empty
    \let\@finalstrut\@gobble
    \let\rule\@gobbletwo
    \H@@footnotetext
    {\ifHy@nesting\expandafter\ltx@firstoftwo
             \else\expandafter\ltx@secondoftwo
        \fi
        {\expandafter\hyper@@anchor
         \expandafter{\Hy@footnote@currentHref}{\unvbox\z@}}%
        {\Hy@raisedlink
          {\expandafter\hyper@@anchor\expandafter{\Hy@footnote@currentHref}%
          {\relax}}%
         \let\@currentHref\Hy@footnote@currentHref
         \let\@currentlabelname\@empty
         \unvbox\z@}%
    }%
    \endgroup
}%
%    \end{macrocode}
% \end{macro}\end{macro}
% \begin{macro}{\@makefntext}
% \begin{macro}{\fn@prefntext}
% \begin{macro}{\fn@postfntext}
%   The |footnote.sty|'s mechanism for setting up a |footnote| environment is
%   doomed to failure if the parameter in |\@makefntext| is not visible at top
%   level in its meaning. This is the case with babel-frenchb at begin
%   document. It is also wrong if the parameter is used multiple times. Let's
%   figure it out; and a special rescue for |footnote+babel+frenchb|... notice
%   that these checks will fail (rather, they are skipped) if |footnote|'s
%   loading was done by the user and not by |footnotehyper|.
%    \begin{macrocode}
\ifx\FNH@@makefntext\undefined\expandafter\@gobble
   \else\expandafter\AtBeginDocument\fi
{%
 \ifx\@makefntextFB\undefined
                   \expandafter\@gobble\else\expandafter\@firstofone\fi
 {\ifFBFrenchFootnotes \let\FNH@@makefntext\@makefntextFB \else 
                       \let\FNH@@makefntext\@makefntextORI\fi}% 
 \expandafter\FNH@check@a\FNH@@makefntext{1.2!3?4,}\FNH@@@1.2!3?4,\FNH@@@\relax
}%
\long\def\FNH@check@a #11.2!3?4,#2\FNH@@@#3%
{%
     \ifx\relax#3\expandafter\@firstoftwo\else\expandafter\@secondoftwo\fi
     \FNH@bad@footnote@env
     {\def\fn@prefntext{#1}\def\fn@postfntext{#2}\FNH@check@b}%
}%
\def\FNH@check@b #1\relax
{%
    \expandafter\expandafter\expandafter\FNH@check@c
    \expandafter\meaning\expandafter\fn@prefntext
                            \meaning\fn@postfntext1.2!3?4,\FNH@check@c\relax
}%
\def\FNH@check@c #11.2!3?4,#2#3\relax 
   {\ifx\FNH@check@c#2\expandafter\@gobble\fi\FNH@bad@footnote@env}%
\def\FNH@bad@footnote@env 
{%
    \PackageWarning{footnotehyper}%
     {The footnote environment from package footnote^^J%
      will be dysfunctional, sorry (not my fault...). You may try to mail
      me^^Jthe preamble and/or only the next lines:}%
    \typeout{\meaning\@makefntext}%
    \let\fn@prefntext\@empty\let\fn@postfntext\@empty  
}%
\endinput
%    \end{macrocode}
% \end{macro}\end{macro}\end{macro}
% Let me say again I have barely tested on a few examples that it
% does at all work! One of the main tests being the present documentation\dots
% \MakePercentComment
\Finale
%%
%% End of file `footnotehyper.dtx'.
