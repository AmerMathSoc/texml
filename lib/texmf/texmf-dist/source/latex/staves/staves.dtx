%% \CharacterTable
%%  {Upper-case    \A\B\C\D\E\F\G\H\I\J\K\L\M\N\O\P\Q\R\S\T\U\V\W\X\Y\Z
%%   Lower-case    \a\b\c\d\e\f\g\h\i\j\k\l\m\n\o\p\q\r\s\t\u\v\w\x\y\z
%%   Digits        \0\1\2\3\4\5\6\7\8\9
%%   Exclamation   \!     Double quote  \"     Hash (number) \#
%%   Dollar        \$     Percent       \%     Ampersand     \&
%%   Acute accent  \'     Left paren    \(     Right paren   \)
%%   Asterisk      \*     Plus          \+     Comma         \,
%%   Minus         \-     Point         \.     Solidus       \/
%%   Colon         \:     Semicolon     \;     Less than     \<
%%   Equals        \=     Greater than  \>     Question mark \?
%%   Commercial at \@     Left bracket  \[     Backslash     \\
%%   Right bracket \]     Circumflex    \^     Underscore    \_
%%   Grave accent  \`     Left brace    \{     Vertical bar  \|
%%   Right brace   \}     Tilde         \~}
%\iffalse
%
% (c) copyright 2006 Apostolos Syropoulos 
% This program can be redistributed and/or modified under the terms
% of the LaTeX Project Public License Distributed from CTAN
% archives in directory macros/latex/base/lppl.txt; either
% version 1.3b of the License, or any later version.
%
% Please report errors or suggestions for improvement to
%
%    Apostolos Syropoulos  (asyropoulos@yahoo.com)
%
%\fi
% \CheckSum{292}
% \iffalse This is a Metacomment
%
%<linearA, >\ProvidesFile{staves.sty}
%
%<linearA, >  [2006/11/06 v1.0 Package `staves.sty']
%
%    \begin{macrocode}
%<*driver>
\documentclass{ltxdoc}
\usepackage{url}
\GetFileInfo{staves.drv}
\begin{document}
   \DocInput{staves.dtx}
\end{document}
%</driver>
%    \end{macrocode}
% \fi
%\StopEventually{}
%\title{The \textsf{staves} package}
%\author{Apostolos Syropoulos\\
%        Xanthi, Greece\\
%        \texttt{asyropoulos@yahoo.com}}
% \date{2006/11/06}
%\maketitle
% \begin{abstract}
% The \textsf{staves} package has been designed to provide an interface to the \textsf{icelandic}
% font, designed by the author of this package. The font contains all ``magical'' staves presented
% on the Web site of the Icelandic Museum of Sorcery and Witchcraft. In addition, the package 
% provides a simple command to typeset Icelandic runes.
%\end{abstract}
%
% \vspace*{20pt}
% \fbox{\begin{minipage}{320pt}
%\textbf{Disclaimer} {\em Sorcery} as a cultural phenomenon is studied by ethnologists. Becuase of
% this, the author believes that his work should be of interest to enthnologists, in particular,
% and people with an interest in (digital) typography, in general. However, the author does not 
% believe that the glyphs of the font do actually have any ``magical'' power. Also, if someone will
% use them to perform any kind of ``magical'' ceremony, the author cannot be hold responsible for 
% any such action and its consequences.
% \end{minipage}}
%\vspace*{20pt}
%\section{Introduction}
%
% The \textsf{staves} package provides a simple interface to the 
% \textsf{icelandic} font designed by this author. The font contain all
% the ``magical'' staves that are described at the Web site of the 
% Icelandic Museum of Sorcery and Witchcraft (see 
% \url{http://www.galdrasyning.is/index.php?option=com_content&task=category&sectionid=5&id=18&Itemid=60} 
% for more details). In addition, the font contains all the Icelandic runes
% as they are presented at the same Web pages (see \url{http://www.galdrasyning.is/index.php?option=com_content&task=view&id=229&Itemid=131} for more details).
% 
% The various staves can be accessed by simple commands that have the following form
% \begin{center}
% \texttt{\char`\\icelandicStave\textit{CCC}}
% \end{center}
% where \texttt{\textit{CCC}} is an uppercase roman numeral. It would be possible to
% give ``real'' names to these glyph access commands, but since these staves do not
% have a name to the best of my language, I opted to use this naming scheme.
%
%\begin{figure}
%\begin{center}
%\begin{tabular}{lc}\hline
% Unicode name                                   & Access character\\ \hline
% \textsc{runic letter ansuz a}                  & a \\
% \textsc{runic letter berkanan beorc rjarkan b} & b \\
% \textsc{runic letter iwaz eoh}                 & c \\
% \textsc{runic letter d}                        & d \\
% \textsc{runic letter e}                        & e \\
% \textsc{runic letter fehu feoh fe f}           & f \\
% \textsc{runic letter gebo gyfu g}              & g \\
% \textsc{runic letter haglaz h}                 & h \\
% \textsc{runic letter isaz is iss i}            & i \\
% \textsc{runic letter thurisaz thurs thorn}     & j \\
% \textsc{runic letter kauna}                    & k \\
% \textsc{runic letter laukaz lagu logr l}       & l \\
% \textsc{runic letter mannaz man m}             & m \\
% \textsc{runic letter naudiz nyd naud n}        & n \\
% \textsc{runic letter othalan ethel o}          & o \\
% \textsc{runic letter pertho peorth p}          & p \\
% \textsc{runic letter ingwaz}                   & q \\
% \textsc{runic letter raido rad reid r}         & r \\
% \textsc{runic letter sigel long-branch-sol s}  & s \\
% \textsc{runic letter tiwaz tir tyr t}          & t \\
% \textsc{runic letter uruz ur u}                & u \\
% \hline
% \end{tabular}
% \end{center}
% \caption{Runic letters supported by the \textsf{staves} package.}\label{unicode}
%\end{figure}
%
% The package defines the command |\runictext|, which typesets its argument using the 
% \textsf{icelandic} font. In Table~\ref{unicode} the reader can see which runic letters
% are supported and to which letters they are associated to.
%
% \section{The Source Code}
% 
% Since we want to have everything in a single file, we will start by announcing the 
% \textsf{icelandic} font to \LaTeX.  
%    \begin{macrocode}
%<*staves>
\DeclareFontSubstitution{U}{icelandic}{m}{n}
\DeclareFontFamily{U}{icelandic}{\hyphenchar\font=-1}
%    \end{macrocode}
% Now, we make sure that all font switching commands will default to the 
% \textsf{icelandic} font:
%    \begin{macrocode} 
\DeclareFontShape{U}{icelandic}{m}{n}{%
   <-> icelandic }{}
\DeclareFontShape{U}{icelandic}{m}{sl}{
   <-> ssub * icelandic/m/n}{}
\DeclareFontShape{U}{icelandic}{m}{it}{
   <-> ssub * icelandic/m/n}{}
\DeclareFontShape{U}{icelandic}{m}{sc}{
   <-> ssub * icelandic/m/n}{}
\DeclareFontShape{U}{icelandic}{bx}{n}{
   <-> ssub * icelandic/m/n}{}
\DeclareFontShape{U}{icelandic}{bx}{sl}{
   <-> ssub * icelandic/m/n}{}
\DeclareFontShape{U}{icelandic}{bx}{it}{
   <-> ssub * icelandic/m/n}{}
\DeclareFontShape{U}{icelandic}{bx}{sc}{
   <-> ssub * icelandic/m/n}{}
%    \end{macrocode}
% Also, we need to define a command that will select the \textsf{icelandic} font:
%    \begin{macrocode}
\def\icelandicFamily{\fontencoding{U}\fontfamily{icelandic}\selectfont}
%    \end{macrocode}
% Next, we define the command that will allow users to typeset icelandic runes:
%    \begin{macrocode}
\newcommand{\runictext}[1]{{\icelandicFamily #1}}
%    \end{macrocode}
% And now we are ready to define the various glyph access commands for the various
% ``magical'' staves.
%    \begin{macrocode}
\def\staveI{{\icelandicFamily\char'0}}
\def\staveII{{\icelandicFamily\char'1}}
\def\staveIII{{\icelandicFamily\char'2}}
\def\staveIV{{\icelandicFamily\char'3}}
\def\staveV{{\icelandicFamily\char'4}}
\def\staveVI{{\icelandicFamily\char'5}}
\def\staveVII{{\icelandicFamily\char'6}}
\def\staveVIII{{\icelandicFamily\char'7}}
\def\staveIX{{\icelandicFamily\char'10}}
\def\staveX{{\icelandicFamily\char'11}}
\def\staveXI{{\icelandicFamily\char'12}}
\def\staveXII{{\icelandicFamily\char'13}}
\def\staveXIII{{\icelandicFamily\char'14}}
\def\staveXIV{{\icelandicFamily\char'15}}
\def\staveXV{{\icelandicFamily\char'16}}
\def\staveXVI{{\icelandicFamily\char'17}}
\def\staveXVII{{\icelandicFamily\char'20}}
\def\staveXVIII{{\icelandicFamily\char'21}}
\def\staveXIX{{\icelandicFamily\char'22}}
\def\staveXX{{\icelandicFamily\char'23}}
\def\staveXXI{{\icelandicFamily\char'24}}
\def\staveXXII{{\icelandicFamily\char'25}}
\def\staveXXIII{{\icelandicFamily\char'26}}
\def\staveXXIV{{\icelandicFamily\char'27}}
\def\staveXXV{{\icelandicFamily\char'30}}
\def\staveXXVI{{\icelandicFamily\char'31}}
\def\staveXXVII{{\icelandicFamily\char'32}}
\def\staveXXVIII{{\icelandicFamily\char'33}}
\def\staveXXIX{{\icelandicFamily\char'34}}
\def\staveXXX{{\icelandicFamily\char'35}}
\def\staveXXXI{{\icelandicFamily\char'36}}
\def\staveXXXII{{\icelandicFamily\char'37}}
\def\staveXXXIII{{\icelandicFamily\char'40}}
\def\staveXXXIV{{\icelandicFamily\char'41}}
\def\staveXXXV{{\icelandicFamily\char'42}}
\def\staveXXXVI{{\icelandicFamily\char'43}}
\def\staveXXXVII{{\icelandicFamily\char'44}}
\def\staveXXXVIII{{\icelandicFamily\char'45}}
\def\staveXXXIX{{\icelandicFamily\char'46}}
\def\staveXL{{\icelandicFamily\char'47}}
\def\staveXLI{{\icelandicFamily\char'50}}
\def\staveXLII{{\icelandicFamily\char'51}}
\def\staveXLIII{{\icelandicFamily\char'52}}
\def\staveXLIV{{\icelandicFamily\char'53}}
\def\staveXLV{{\icelandicFamily\char'54}}
\def\staveXLVI{{\icelandicFamily\char'55}}
\def\staveXLVII{{\icelandicFamily\char'56}}
\def\staveXLVIII{{\icelandicFamily\char'57}}
\def\staveXLIX{{\icelandicFamily\char'60}}
\def\staveL{{\icelandicFamily\char'61}}
\def\staveLI{{\icelandicFamily\char'62}}
\def\staveLII{{\icelandicFamily\char'63}}
\def\staveLIII{{\icelandicFamily\char'64}}
\def\staveLIV{{\icelandicFamily\char'65}}
\def\staveLV{{\icelandicFamily\char'66}}
\def\staveLVI{{\icelandicFamily\char'67}}
\def\staveLVII{{\icelandicFamily\char'70}}
\def\staveLVIII{{\icelandicFamily\char'71}}
\def\staveLIX{{\icelandicFamily\char'72}}
\def\staveLX{{\icelandicFamily\char'73}}
\def\staveLXI{{\icelandicFamily\char'74}}
\def\staveLXII{{\icelandicFamily\char'75}}
\def\staveLXIII{{\icelandicFamily\char'76}}
\def\staveLXIV{{\icelandicFamily\char'77}}
\def\staveLXV{{\icelandicFamily\char'100}}
\def\staveLXVI{{\icelandicFamily\char'101}}
\def\staveLXVII{{\icelandicFamily\char'102}}
\def\staveLXVIII{{\icelandicFamily\char'103}}
%</staves>
%    \end{macrocode}
%
% \Finale
%\section*{Acknowledgements}
% I would like to thank the people of the Icelandic Museum of Sorcery and Witchcraft for their
% encouragement in this project. Also, I would like to thank my Icelandic friends for giving
% me and my son the chance to visit their beautiful country!
