%% \CharacterTable
%%  {Upper-case    \A\B\C\D\E\F\G\H\I\J\K\L\M\N\O\P\Q\R\S\T\U\V\W\X\Y\Z
%%   Lower-case    \a\b\c\d\e\f\g\h\i\j\k\l\m\n\o\p\q\r\s\t\u\v\w\x\y\z
%%   Digits        \0\1\2\3\4\5\6\7\8\9
%%   Exclamation   \!     Double quote  \"     Hash (number) \#
%%   Dollar        \$     Percent       \%     Ampersand     \&
%%   Acute accent  \'     Left paren    \(     Right paren   \)
%%   Asterisk      \*     Plus          \+     Comma         \,
%%   Minus         \-     Point         \.     Solidus       \/
%%   Colon         \:     Semicolon     \;     Less than     \<
%%   Equals        \=     Greater than  \>     Question mark \?
%%   Commercial at \@     Left bracket  \[     Backslash     \\
%%   Right bracket \]     Circumflex    \^     Underscore    \_
%%   Grave accent  \`     Left brace    \{     Vertical bar  \|
%%   Right brace   \}     Tilde         \~}
%\iffalse
%
% (c) Copyright  2003 Apostolos Syropoulos
% This program can be redistributed and/or modified under the terms
% of the LaTeX Project Public License Distributed from CTAN
% archives in directory macros/latex/base/lppl.txt; either
% version 1 of the License, or any later version.
%
% Please report errors or suggestions for improvement to
%
%    Apostolos Syropoulos
%    366, 28th October Str.
%    GR-671 00 Xanthi, GREECE
%    apostolo@ocean1.ee.duth.gr or apostolo@obelix.ee.duth.gr
%
%\fi
% \CheckSum{859}
% \iffalse This is a Metacomment
%
%<epiolmec, >\ProvidesFile{epiolmec.sty}
%<epiolmec, >  [2003/11/05 v1.0 Package `epiolmec.sty']
%
%    \begin{macrocode}
%<*driver>
\documentclass{ltxdoc}
\usepackage{url}
\GetFileInfo{epiolmec.drv}
\begin{document}
   \DocInput{epiolmec.dtx}
\end{document}
%</driver>
%    \end{macrocode}
% \fi
%
% \title{The \textsf{epiolmec} package}
% \author{Apostolos Syropoulos\\366, 28th October Str.\\
% GR-671 00 Xanthi, HELLAS\\
% Email:\texttt{apostolo@obelix.ee.duth.gr}}
% \date{2003/11/05}
% \maketitle
%
%\MakeShortVerb{\|}
%\StopEventually{}
% %%%%%%%%%%%%%%%%%%%%%%%%%%%%%%%%%
% \section{Introduction}
% %%%%%%%%%%%%%%%%%%%%%%%%%%%%%%%%%
%
% The \textsf{epiolmec} package defines the necessary \LaTeX\ interface to
% the Epi-Olmec font, which consists of all known glyphs of the Epi-Olmec
% script. The Epi-Olmec script is an ancient Mesoamerican logosyllabic 
% script, which has been recently deciphered by Terrence Kaufman and John 
% Justeson. A complete description of the script can be found in 
% {\em Epi-Olmec Hieroglyphic Writing and Texts} (by Kaufman and Justeson)
% available from \url{http://www.albany.edu/anthro/maldp/EOTEXTS.pdf}.
% In addition, the reader is refereed to the article {\em Replicating Archaic 
% Documents: A Typographic Challenge} (which was presented by the author of 
% this package at the Euro\TeX2003 conference in Brest, France and will appear
% in TUGboat) for more information regarding the Epi-Olmec script and the 
% Epi-Olmec font.
%
% Although, it is a common practice to have different files for the
% font encoding, the glyph access commands and the various support commands,
% we opted to put everything in just one file. Initially, we describe the
% source code of the package, then we provide some usage information and
% we conclude with some notes concerning our future plans.
%
% %%%%%%%%%%%%%%%%%%%%%%%%%%%%%%%%%
% \section{The Source Code}
% %%%%%%%%%%%%%%%%%%%%%%%%%%%%%%%%%
% 
% The first thing we need to define is a new local font encoding. The
% following code is not much of a font encoding, nevertheless it is
% required to have these minimum declarations in order to use the
% ``font encoding.''
%    \begin{macrocode}
%<*epiolmec>
\DeclareFontEncoding{LEO}{}{}
\DeclareFontSubstitution{LEO}{cmr}{m}{n}
\DeclareFontFamily{LEO}{cmr}{\hyphenchar\font=-1}
%    \end{macrocode}
% Clearly, it makes no sense to have any series other than normal. So all
% of the following definitions default to the first case.
%    \begin{macrocode}
\DeclareFontShape{LEO}{cmr}{m}{n}{%
   <-> EpiOlmec }{}

\DeclareFontShape{LEO}{cmr}{m}{sc}{%
   <-> ssub * EO/m/n}{}

\DeclareFontShape{LEO}{cmr}{m}{sl}{%
   <-> ssub * cmr/m/n}{}

\DeclareFontShape{LEO}{cmr}{m}{it}{%
   <-> ssub * cmr/m/n}{}

\DeclareFontShape{LEO}{cmr}{bx}{n}{%
   <-> ssub * cmr/m/n}{}

\DeclareFontShape{LEO}{cmr}{bx}{sc}{%
   <-> ssub * cmr/m/n}{}

\DeclareFontShape{LEO}{cmr}{bx}{sl}{%
   <-> ssub * cmr/m/n}{}

\DeclareFontShape{LEO}{cmr}{bx}{it}{%
   <-> ssub * cmr/m/n}{}
%    \end{macrocode}
% Let us now proceed with the definition of the various glyph access
% commands. First, we define the commands that can be used to access
% the various digits. Note that there is written evidence that the
% Epi-Olmec used the digit zero. However, since their numbering system
% is identical to the numbering system of the Maya people, we have
% ``borrowed'' the most common form of the Maya zero digit.
%    \begin{macrocode}
\DeclareTextSymbol{\EOi}{LEO}{'00}
\DeclareTextSymbolDefault{\EOi}{LEO}

\DeclareTextSymbol{\EOii}{LEO}{'01}
\DeclareTextSymbolDefault{\EOii}{LEO}

\DeclareTextSymbol{\EOiii}{LEO}{'02}
\DeclareTextSymbolDefault{\EOiii}{LEO}

\DeclareTextSymbol{\EOiv}{LEO}{'03}
\DeclareTextSymbolDefault{\EOiv}{LEO}

\DeclareTextSymbol{\EOv}{LEO}{'04}
\DeclareTextSymbolDefault{\EOv}{LEO}

\DeclareTextSymbol{\EOvi}{LEO}{'05}
\DeclareTextSymbolDefault{\EOvi}{LEO}

\DeclareTextSymbol{\EOvii}{LEO}{'06}
\DeclareTextSymbolDefault{\EOvii}{LEO}

\DeclareTextSymbol{\EOviii}{LEO}{'07}
\DeclareTextSymbolDefault{\EOviii}{LEO}

\DeclareTextSymbol{\EOix}{LEO}{'10}
\DeclareTextSymbolDefault{\EOix}{LEO}

\DeclareTextSymbol{\EOx}{LEO}{'11}
\DeclareTextSymbolDefault{\EOx}{LEO}

\DeclareTextSymbol{\EOxi}{LEO}{'12}
\DeclareTextSymbolDefault{\EOxi}{LEO}

\DeclareTextSymbol{\EOxii}{LEO}{'13}
\DeclareTextSymbolDefault{\EOxii}{LEO}

\DeclareTextSymbol{\EOxiii}{LEO}{'14}
\DeclareTextSymbolDefault{\EOxiii}{LEO}

\DeclareTextSymbol{\EOxiv}{LEO}{'15}
\DeclareTextSymbolDefault{\EOxiv}{LEO}

\DeclareTextSymbol{\EOxv}{LEO}{'16}
\DeclareTextSymbolDefault{\EOxv}{LEO}

\DeclareTextSymbol{\EOxvi}{LEO}{'17}
\DeclareTextSymbolDefault{\EOxvi}{LEO}

\DeclareTextSymbol{\EOxvii}{LEO}{'20}
\DeclareTextSymbolDefault{\EOxvii}{LEO}

\DeclareTextSymbol{\EOxviii}{LEO}{'21}
\DeclareTextSymbolDefault{\EOxviii}{LEO}

\DeclareTextSymbol{\EOxix}{LEO}{'22}
\DeclareTextSymbolDefault{\EOxix}{LEO}

\DeclareTextSymbol{\EOxx}{LEO}{'23}
\DeclareTextSymbolDefault{\EOxx}{LEO}

\DeclareTextSymbol{\EOzero}{LEO}{'24}
\DeclareTextSymbolDefault{\EOzero}{LEO}
%    \end{macrocode}
%
% The commands that follow can be used to access the glyphs that are shown
% in Figure~4 on page~5 of the text by Kaufman and Justeson. Somehow, we
% can claim that this is the basic set of glyphs of the Epi-Olmec script.
%
%    \begin{macrocode}
\DeclareTextSymbol{\EOpi}{LEO}{'60}
\DeclareTextSymbolDefault{\EOpi}{LEO}

\DeclareTextSymbol{\EOpe}{LEO}{'61}
\DeclareTextSymbolDefault{\EOpe}{LEO}

\DeclareTextSymbol{\EOpuu}{LEO}{'62}
\DeclareTextSymbolDefault{\EOpuu}{LEO}

\DeclareTextSymbol{\EOpa}{LEO}{'63}
\DeclareTextSymbolDefault{\EOpa}{LEO}

\DeclareTextSymbol{\EOvarpa}{LEO}{'64}
\DeclareTextSymbolDefault{\EOvarpa}{LEO}

\DeclareTextSymbol{\EOpu}{LEO}{'65}
\DeclareTextSymbolDefault{\EOpu}{LEO}

\DeclareTextSymbol{\EOpo}{LEO}{'66}
\DeclareTextSymbolDefault{\EOpo}{LEO}

\DeclareTextSymbol{\EOti}{LEO}{'67}
\DeclareTextSymbolDefault{\EOti}{LEO}

\DeclareTextSymbol{\EOte}{LEO}{'70}
\DeclareTextSymbolDefault{\EOte}{LEO}

\DeclareTextSymbol{\EOtuu}{LEO}{'71}
\DeclareTextSymbolDefault{\EOtuu}{LEO}

\DeclareTextSymbol{\EOta}{LEO}{'72}
\DeclareTextSymbolDefault{\EOta}{LEO}

\DeclareTextSymbol{\EOtu}{LEO}{'73}
\DeclareTextSymbolDefault{\EOtu}{LEO}

\DeclareTextSymbol{\EOto}{LEO}{'74}
\DeclareTextSymbolDefault{\EOto}{LEO}

\DeclareTextSymbol{\EOtzi}{LEO}{'75}
\DeclareTextSymbolDefault{\EOtzi}{LEO}

\DeclareTextSymbol{\EOtze}{LEO}{'76}
\DeclareTextSymbolDefault{\EOtze}{LEO}

\DeclareTextSymbol{\EOtzuu}{LEO}{'77}
\DeclareTextSymbolDefault{\EOtzuu}{LEO}

\DeclareTextSymbol{\EOtza}{LEO}{'100}
\DeclareTextSymbolDefault{\EOtza}{LEO}

\DeclareTextSymbol{\EOvartza}{LEO}{'101}
\DeclareTextSymbolDefault{\EOvartza}{LEO}

\DeclareTextSymbol{\EOtzu}{LEO}{'102}
\DeclareTextSymbolDefault{\EOtzu}{LEO}

\DeclareTextSymbol{\EOki}{LEO}{'103}
\DeclareTextSymbolDefault{\EOki}{LEO}

\DeclareTextSymbol{\EOke}{LEO}{'104}
\DeclareTextSymbolDefault{\EOke}{LEO}

\DeclareTextSymbol{\EOkuu}{LEO}{'105}
\DeclareTextSymbolDefault{\EOkuu}{LEO}

\DeclareTextSymbol{\EOvarkuu}{LEO}{'106}
\DeclareTextSymbolDefault{\EOvarkuu}{LEO}

\DeclareTextSymbol{\EOku}{LEO}{'107}
\DeclareTextSymbolDefault{\EOku}{LEO}

\DeclareTextSymbol{\EOko}{LEO}{'110}
\DeclareTextSymbolDefault{\EOko}{LEO}

\DeclareTextSymbol{\EOSi}{LEO}{'111}
\DeclareTextSymbolDefault{\EOSi}{LEO}

\DeclareTextSymbol{\EOvarSi}{LEO}{'112}
\DeclareTextSymbolDefault{\EOvarSi}{LEO}

\DeclareTextSymbol{\EOSuu}{LEO}{'113}
\DeclareTextSymbolDefault{\EOSuu}{LEO}

\DeclareTextSymbol{\EOSa}{LEO}{'114}
\DeclareTextSymbolDefault{\EOSa}{LEO}

\DeclareTextSymbol{\EOSu}{LEO}{'115}
\DeclareTextSymbolDefault{\EOSu}{LEO}

\DeclareTextSymbol{\EOSo}{LEO}{'116}
\DeclareTextSymbolDefault{\EOSo}{LEO}

\DeclareTextSymbol{\EOsi}{LEO}{'117}
\DeclareTextSymbolDefault{\EOsi}{LEO}

\DeclareTextSymbol{\EOvarsi}{LEO}{'120}
\DeclareTextSymbolDefault{\EOvarsi}{LEO}

\DeclareTextSymbol{\EOsuu}{LEO}{'121}
\DeclareTextSymbolDefault{\EOsuu}{LEO}

\DeclareTextSymbol{\EOsa}{LEO}{'122}
\DeclareTextSymbolDefault{\EOsa}{LEO}

\DeclareTextSymbol{\EOsu}{LEO}{'123}
\DeclareTextSymbolDefault{\EOsu}{LEO}

\DeclareTextSymbol{\EOji}{LEO}{'124}
\DeclareTextSymbolDefault{\EOji}{LEO}

\DeclareTextSymbol{\EOje}{LEO}{'125}
\DeclareTextSymbolDefault{\EOje}{LEO}

\DeclareTextSymbol{\EOja}{LEO}{'126}
\DeclareTextSymbolDefault{\EOja}{LEO}

\DeclareTextSymbol{\EOvarja}{LEO}{'127}
\DeclareTextSymbolDefault{\EOvarja}{LEO}

\DeclareTextSymbol{\EOju}{LEO}{'130}
\DeclareTextSymbolDefault{\EOju}{LEO}

\DeclareTextSymbol{\EOjo}{LEO}{'131}
\DeclareTextSymbolDefault{\EOjo}{LEO}

\DeclareTextSymbol{\EOmi}{LEO}{'132}
\DeclareTextSymbolDefault{\EOmi}{LEO}

\DeclareTextSymbol{\EOme}{LEO}{'133}
\DeclareTextSymbolDefault{\EOme}{LEO}

\DeclareTextSymbol{\EOmuu}{LEO}{'134}
\DeclareTextSymbolDefault{\EOmuu}{LEO}

\DeclareTextSymbol{\EOma}{LEO}{'135}
\DeclareTextSymbolDefault{\EOma}{LEO}

\DeclareTextSymbol{\EOni}{LEO}{'136}
\DeclareTextSymbolDefault{\EOni}{LEO}

\DeclareTextSymbol{\EOvarni}{LEO}{'137}
\DeclareTextSymbolDefault{\EOvarni}{LEO}

\DeclareTextSymbol{\EOne}{LEO}{'140}
\DeclareTextSymbolDefault{\EOne}{LEO}

\DeclareTextSymbol{\EOnuu}{LEO}{'141}
\DeclareTextSymbolDefault{\EOnuu}{LEO}

\DeclareTextSymbol{\EOna}{LEO}{'142}
\DeclareTextSymbolDefault{\EOna}{LEO}

\DeclareTextSymbol{\EOnu}{LEO}{'143}
\DeclareTextSymbolDefault{\EOnu}{LEO}

\DeclareTextSymbol{\EOwi}{LEO}{'144}
\DeclareTextSymbolDefault{\EOwi}{LEO}

\DeclareTextSymbol{\EOwe}{LEO}{'145}
\DeclareTextSymbolDefault{\EOwe}{LEO}

\DeclareTextSymbol{\EOwuu}{LEO}{'146}
\DeclareTextSymbolDefault{\EOwuu}{LEO}

\DeclareTextSymbol{\EOvarwuu}{LEO}{'147}
\DeclareTextSymbolDefault{\EOvarwuu}{LEO}

\DeclareTextSymbol{\EOwa}{LEO}{'150}
\DeclareTextSymbolDefault{\EOwa}{LEO}

\DeclareTextSymbol{\EOwo}{LEO}{'151}
\DeclareTextSymbolDefault{\EOwo}{LEO}

\DeclareTextSymbol{\EOye}{LEO}{'152}
\DeclareTextSymbolDefault{\EOye}{LEO}

\DeclareTextSymbol{\EOyuu}{LEO}{'153}
\DeclareTextSymbolDefault{\EOyuu}{LEO}

\DeclareTextSymbol{\EOya}{LEO}{'154}
\DeclareTextSymbolDefault{\EOya}{LEO}
%    \end{macrocode}
%
% The commands that follow can be used to access glyphs that are unusual or 
% alternative forms of common glyphs. 
%
%    \begin{macrocode}
\DeclareTextSymbol{\EOSpan}{LEO}{'54}
\DeclareTextSymbolDefault{\EOSpan}{LEO}

\DeclareTextSymbol{\EOJI}{LEO}{'55}
\DeclareTextSymbolDefault{\EOJI}{LEO}

\DeclareTextSymbol{\EOvarji}{LEO}{'56}
\DeclareTextSymbolDefault{\EOvarji}{LEO}

\DeclareTextSymbol{\EOvarki}{LEO}{'57}
\DeclareTextSymbolDefault{\EOvarki}{LEO}

\DeclareTextSymbol{\EOkak}{LEO}{'155}
\DeclareTextSymbolDefault{\EOkak}{LEO}

\DeclareTextSymbol{\EOpak}{LEO}{'156}
\DeclareTextSymbolDefault{\EOpak}{LEO}

\DeclareTextSymbol{\EOpuuk}{LEO}{'157}
\DeclareTextSymbolDefault{\EOpuuk}{LEO}

\DeclareTextSymbol{\EOyaj}{LEO}{'160}
\DeclareTextSymbolDefault{\EOyaj}{LEO}

\DeclareTextSymbol{\EOScorpius}{LEO}{'161}
\DeclareTextSymbolDefault{\EOScorpius}{LEO}

\DeclareTextSymbol{\EODealWith}{LEO}{'162}
\DeclareTextSymbolDefault{\EODealWith}{LEO}

\DeclareTextSymbol{\EOYear}{LEO}{'163}
\DeclareTextSymbolDefault{\EOYear}{LEO}

\DeclareTextSymbol{\EOBeardMask}{LEO}{'164}
\DeclareTextSymbolDefault{\EOBeardMask}{LEO}

\DeclareTextSymbol{\EOBlood}{LEO}{'165}
\DeclareTextSymbolDefault{\EOBlood}{LEO}

\DeclareTextSymbol{\EOBundle}{LEO}{'166}
\DeclareTextSymbolDefault{\EOBundle}{LEO}

\DeclareTextSymbol{\EOChop}{LEO}{'167}
\DeclareTextSymbolDefault{\EOChop}{LEO}

\DeclareTextSymbol{\EOCloth}{LEO}{'170}
\DeclareTextSymbolDefault{\EOCloth}{LEO}

\DeclareTextSymbol{\EOSaw}{LEO}{'171}
\DeclareTextSymbolDefault{\EOSaw}{LEO}

\DeclareTextSymbol{\EOGuise}{LEO}{'172}
\DeclareTextSymbolDefault{\EOGuise}{LEO}

\DeclareTextSymbol{\EOofficerI}{LEO}{'173}
\DeclareTextSymbolDefault{\EOofficerI}{LEO}

\DeclareTextSymbol{\EOofficerII}{LEO}{'174}
\DeclareTextSymbolDefault{\EOofficerII}{LEO}

\DeclareTextSymbol{\EOofficerIII}{LEO}{'175}
\DeclareTextSymbolDefault{\EOofficerIII}{LEO}

\DeclareTextSymbol{\EOofficerIV}{LEO}{'176}
\DeclareTextSymbolDefault{\EOofficerIV}{LEO}

\DeclareTextSymbol{\EOKing}{LEO}{'200}
\DeclareTextSymbolDefault{\EOKing}{LEO}

\DeclareTextSymbol{\EOloinCloth}{LEO}{'201}
\DeclareTextSymbolDefault{\EOloinCloth}{LEO}

\DeclareTextSymbol{\EOlongLipII}{LEO}{'202}
\DeclareTextSymbolDefault{\EOlongLipII}{LEO}

\DeclareTextSymbol{\EOLose}{LEO}{'203}
\DeclareTextSymbolDefault{\EOLose}{LEO}

\DeclareTextSymbol{\EOmexNew}{LEO}{'204}
\DeclareTextSymbolDefault{\EOmexNew}{LEO}

\DeclareTextSymbol{\EOMiddle}{LEO}{'205}
\DeclareTextSymbolDefault{\EOMiddle}{LEO}

\DeclareTextSymbol{\EOPlant}{LEO}{'206}
\DeclareTextSymbolDefault{\EOPlant}{LEO}

\DeclareTextSymbol{\EOPlay}{LEO}{'207}
\DeclareTextSymbolDefault{\EOPlay}{LEO}

\DeclareTextSymbol{\EOPrince}{LEO}{'210}
\DeclareTextSymbolDefault{\EOPrince}{LEO}

\DeclareTextSymbol{\EOSky}{LEO}{'211}
\DeclareTextSymbolDefault{\EOSky}{LEO}

\DeclareTextSymbol{\EOskyPillar}{LEO}{'212}
\DeclareTextSymbolDefault{\EOskyPillar}{LEO}

\DeclareTextSymbol{\EOSprinkle}{LEO}{'213}
\DeclareTextSymbolDefault{\EOSprinkle}{LEO}

\DeclareTextSymbol{\EOstarWarrior}{LEO}{'214}
\DeclareTextSymbolDefault{\EOstarWarrior}{LEO}

\DeclareTextSymbol{\EOTitleII}{LEO}{'215}
\DeclareTextSymbolDefault{\EOTitleII}{LEO}

\DeclareTextSymbol{\EOtuki}{LEO}{'216}
\DeclareTextSymbolDefault{\EOtuki}{LEO}

\DeclareTextSymbol{\EOtzetze}{LEO}{'217}
\DeclareTextSymbolDefault{\EOtzetze}{LEO}

\DeclareTextSymbol{\EOChronI}{LEO}{'220}
\DeclareTextSymbolDefault{\EOChronI}{LEO}

\DeclareTextSymbol{\EOPatron}{LEO}{'221}
\DeclareTextSymbolDefault{\EOPatron}{LEO}

\DeclareTextSymbol{\EOandThen}{LEO}{'222}
\DeclareTextSymbolDefault{\EOandThen}{LEO}

\DeclareTextSymbol{\EOAppear}{LEO}{'223}
\DeclareTextSymbolDefault{\EOAppear}{LEO}

\DeclareTextSymbol{\EODeer}{LEO}{'224}
\DeclareTextSymbolDefault{\EODeer}{LEO}

\DeclareTextSymbol{\EOeat}{LEO}{'225}
\DeclareTextSymbolDefault{\EOeat}{LEO}

\DeclareTextSymbol{\EOPatronII}{LEO}{'226}
\DeclareTextSymbolDefault{\EOPatronII}{LEO}

\DeclareTextSymbol{\EOPierce}{LEO}{'227}
\DeclareTextSymbolDefault{\EOPierce}{LEO}

\DeclareTextSymbol{\EOkij}{LEO}{'230}
\DeclareTextSymbolDefault{\EOkij}{LEO}

\DeclareTextSymbol{\EOstar}{LEO}{'231}
\DeclareTextSymbolDefault{\EOstar}{LEO}

\DeclareTextSymbol{\EOsnake}{LEO}{'232}
\DeclareTextSymbolDefault{\EOsnake}{LEO}

\DeclareTextSymbol{\EOtime}{LEO}{'233}
\DeclareTextSymbolDefault{\EOtime}{LEO}

\DeclareTextSymbol{\EOtukpa}{LEO}{'234}
\DeclareTextSymbolDefault{\EOtukpa}{LEO}

\DeclareTextSymbol{\EOflint}{LEO}{'235}
\DeclareTextSymbolDefault{\EOflint}{LEO}

\DeclareTextSymbol{\EOafter}{LEO}{'236}
\DeclareTextSymbolDefault{\EOafter}{LEO}

\DeclareTextSymbol{\EOvarBeardMask}{LEO}{'237}
\DeclareTextSymbolDefault{\EOvarBeardMask}{LEO}

\DeclareTextSymbol{\EOBedeck}{LEO}{'240}
\DeclareTextSymbolDefault{\EOBedeck}{LEO}

\DeclareTextSymbol{\EObrace}{LEO}{'241}
\DeclareTextSymbolDefault{\EObrace}{LEO}

\DeclareTextSymbol{\EOflower}{LEO}{'242}
\DeclareTextSymbolDefault{\EOflower}{LEO}

\DeclareTextSymbol{\EOGod}{LEO}{'243}
\DeclareTextSymbolDefault{\EOGod}{LEO}

\DeclareTextSymbol{\EOMountain}{LEO}{'244}
\DeclareTextSymbolDefault{\EOMountain}{LEO}

\DeclareTextSymbol{\EOgovernor}{LEO}{'245}
\DeclareTextSymbolDefault{\EOgovernor}{LEO}

\DeclareTextSymbol{\EOHallow}{LEO}{'246}
\DeclareTextSymbolDefault{\EOHallow}{LEO}

\DeclareTextSymbol{\EOjaguar}{LEO}{'247}
\DeclareTextSymbolDefault{\EOjaguar}{LEO}

\DeclareTextSymbol{\EOSini}{LEO}{'250}
\DeclareTextSymbolDefault{\EOSini}{LEO}

\DeclareTextSymbol{\EOknottedCloth}{LEO}{'251}
\DeclareTextSymbolDefault{\EOknottedCloth}{LEO}

\DeclareTextSymbol{\EOknottedClothStraps}{LEO}{'252}
\DeclareTextSymbolDefault{\EOknottedClothStraps}{LEO}

\DeclareTextSymbol{\EOLord}{LEO}{'253}
\DeclareTextSymbolDefault{\EOLord}{LEO}

\DeclareTextSymbol{\EOmacaw}{LEO}{'254}
\DeclareTextSymbolDefault{\EOmacaw}{LEO}

\DeclareTextSymbol{\EOmonster}{LEO}{'255}
\DeclareTextSymbolDefault{\EOmonster}{LEO}

\DeclareTextSymbol{\EOmacawI}{LEO}{'256}
\DeclareTextSymbolDefault{\EOmacawI}{LEO}

\DeclareTextSymbol{\EOskyAnimal}{LEO}{'257}
\DeclareTextSymbolDefault{\EOskyAnimal}{LEO}

\DeclareTextSymbol{\EOnow}{LEO}{'260}
\DeclareTextSymbolDefault{\EOnow}{LEO}

\DeclareTextSymbol{\EOTitleIV}{LEO}{'261}
\DeclareTextSymbolDefault{\EOTitleIV}{LEO}

\DeclareTextSymbol{\EOpenis}{LEO}{'262}
\DeclareTextSymbolDefault{\EOpenis}{LEO}

\DeclareTextSymbol{\EOpriest}{LEO}{'263}
\DeclareTextSymbolDefault{\EOpriest}{LEO}

\DeclareTextSymbol{\EOstep}{LEO}{'264}
\DeclareTextSymbolDefault{\EOstep}{LEO}

\DeclareTextSymbol{\EOsing}{LEO}{'265}
\DeclareTextSymbolDefault{\EOsing}{LEO}

\DeclareTextSymbol{\EOskin}{LEO}{'266}
\DeclareTextSymbolDefault{\EOskin}{LEO}

\DeclareTextSymbol{\EOStarWarrior}{LEO}{'267}
\DeclareTextSymbolDefault{\EOStarWarrior}{LEO}

\DeclareTextSymbol{\EOsun}{LEO}{'270}
\DeclareTextSymbolDefault{\EOsun}{LEO}

\DeclareTextSymbol{\EOthrone}{LEO}{'271}
\DeclareTextSymbolDefault{\EOthrone}{LEO}

\DeclareTextSymbol{\EOTime}{LEO}{'272}
\DeclareTextSymbolDefault{\EOTime}{LEO}

\DeclareTextSymbol{\EOHallow}{LEO}{'273}
\DeclareTextSymbolDefault{\EOHallow}{LEO}

\DeclareTextSymbol{\EOTitle}{LEO}{'274}
\DeclareTextSymbolDefault{\EOTitle}{LEO}

\DeclareTextSymbol{\EOturtle}{LEO}{'275}
\DeclareTextSymbolDefault{\EOturtle}{LEO}

\DeclareTextSymbol{\EOundef}{LEO}{'276}
\DeclareTextSymbolDefault{\EOundef}{LEO}

\DeclareTextSymbol{\EOGoUp}{LEO}{'277}
\DeclareTextSymbolDefault{\EOGoUp}{LEO}

\DeclareTextSymbol{\EOLetBlood}{LEO}{'300}
\DeclareTextSymbolDefault{\EOLetBlood}{LEO}

\DeclareTextSymbol{\EORain}{LEO}{'301}
\DeclareTextSymbolDefault{\EORain}{LEO}

\DeclareTextSymbol{\EOset}{LEO}{'302}
\DeclareTextSymbolDefault{\EOset}{LEO}

\DeclareTextSymbol{\EOvarYear}{LEO}{'303}
\DeclareTextSymbolDefault{\EOvarYear}{LEO}

\DeclareTextSymbol{\EOFold}{LEO}{'304}
\DeclareTextSymbolDefault{\EOFold}{LEO}

\DeclareTextSymbol{\EOsacrifice}{LEO}{'305}
\DeclareTextSymbolDefault{\EOsacrifice}{LEO}

\DeclareTextSymbol{\EObuilding}{LEO}{'306}
\DeclareTextSymbolDefault{\EObuilding}{LEO}
%    \end{macrocode}
% As we mentioned above, the Epi-Olmec people used the same numbering system  
% as the Maya people did. Their numbering system was a vigesimal system and
% the digits were written in a top-down fashion. Thus, we need a macro
% that will typeset numbers in this fashion when it is used with \LaTeX\
% (actually $\epsilon$-\LaTeX). In addition, we need a macro that will
% just output the vigesimal digits. Such a macro could be used with
% $\Lambda$ with the |LTL| text and paragraph directions. To recapitulate,
% we need to define two macros that will basically typeset vigesimal numbers
% in either horizontal or vertical mode.
%
% For the various calculations that are performed, we need at least three
% counter variables. The fourth is needed for the macro that typesets the
% vigesimal numbers vertically and its usage is explained below. 
%    \begin{macrocode}
\newcount\EO@n
\newcount\EO@m
\newcount\EO@k
\newcount\EO@l
%    \end{macrocode}
% Although we do not foresee any complicated usage of these macros, we opted
% to design them in such a way that they can have as argument either a
% counter or a number. For this reason, we need to make sure the argument
% is stored to one of our counter variables. Next, we pass the counter
% to an auxiliary macro, described below. 
%    \begin{macrocode}
\def\vigesimal#1{%
       \EO@n#1\relax
       \@vigesimal{\EO@n}}
%    \end{macrocode}
% The macro |\@vigesimal| is just an auxiliary macro that is used to
% set the correct value to the counter |\EO@n|. 
%    \begin{macrocode}
\def\@vigesimal#1{%
       \EO@n\expandafter=\the#1\relax
       \@vig}
%    \end{macrocode}
% The following macro is based on the macro |\hex|, whose definition
% can be found on page~219 of the \TeX{book}. Therefore, the reader should
% consult his/her copy of the \TeX{book} for more information regarding
% the functionality of the following macro. 
%    \begin{macrocode}
\def\@vig{{\EO@m=\EO@n \divide\EO@n by20
          \ifnum\EO@n>0 \@vig\fi \EO@k=\EO@n
          \multiply\EO@k by-20
          \advance\EO@m by \EO@k \@vigdigit}}
%    \end{macrocode}
% The macro that follows is used to get the name of the digit. Note
% that all digits greater than one can be accessed with a command of
% the form |\EOzz|, where |zz| is the Roman numeral that corresponds 
% the current value of counter |\EO@m|.
%    \begin{macrocode}
\def\@vigdigit{\ifnum\EO@m=0\EOzero%
               \else \csname EO\@roman{\EO@m}\endcsname\fi}
%    \end{macrocode}
% The macro |\StackedVigesimal| takes the same argument as the macro 
% |\vigesimal| and operates similarly. However, this macro should be used
% when we want the ``stacked'' version of the number (i.e., the 
% typographically correct version).
%    \begin{macrocode}
\def\StackedVigesimal#1{%
       \EO@n#1\relax
       \s@vigesimal{\EO@n}}
%    \end{macrocode}
% The macro |\s@vigesimal| assigns to counter |\EO@n| the value of the 
% number or counter that the user has supplied. In order to typeset the
% number in a top-down fashion, we use a list structure, which is initially
% set to empty. The number will be typeset inside a vertical box that has
% a rather strange baseline skip. The macro |\s@vig| constructs the list
% of digits, which, in turn, will be typeset inside a horizontal
% alignment command.   
%    \begin{macrocode}
\def\s@vigesimal#1{%
       \EO@n\expandafter=\the#1\relax
       \global\let\epi@lmecDigits\empty
       \vbox{\baselineskip=-1000\p@\lineskip=3\p@
       \let\\=\cr%
       \s@vig%
       \halign{\hfil##\hfil\cr \epi@lmecDigits\cr}}}
%    \end{macrocode}
% Now let us see how we construct the list that contains the various digits.
% First we need some auxiliary macros that can be used to append elements
% to a list. The two macros that follow have been designed after two macros
% that are part of the PiC\TeX\ distribution. The second argument of the first
% macro is the list and the first argument is the element that will be 
% appended to the list. Note that the list is actually augmented by calling
% the second macro. This macro creates a new global macro that expands to
% the list element and the list itself separated by the |\\| token.
%    \begin{macrocode}
\def\@rightappend#1\t@#2{\expandafter\@@rightappend#2\t@{#1}#2}
\def\@@rightappend#1\t@#2#3{\gdef#3{#1\\{#2}}}
%    \end{macrocode}
% The macro |\s@vig| is the crux of the package! This macro builds the
% list that contains the Epi-Olmec digits of the vigesimal representation
% of the number. Each digit is ``stored'' in a global macro the name of which
% has the form \texttt{\char`\\EO@d\textit{i}}, where \texttt{\textit{i}}
% is a roman numeral that corresponds to the current value of the counter
% |\EO@l|. Note that this counter is increased each time we execute this
% macro. Each of the \texttt{\char`\\EO@d\textit{i}} macros is constructed
% in a rather peculiar way: not only its expansion is constructed 
% ``on-the-fly'' but its name too! Clearly, if the reader is not a somehow
% advanced \TeX\ programmer, he should not make any attempt to study the
% code that follows as it is particularly complex, but not tricky at all.
%    \begin{macrocode}
\def\s@vig{{\EO@m=\EO@n% 
            \divide\EO@n by20
            \ifnum\EO@n>0\s@vig\fi% 
            \EO@k=\EO@n\relax
            \multiply\EO@k by-20\relax
            \advance\EO@m by \EO@k\relax
            \global\advance\EO@l by \@ne%
            \expandafter\xdef\csname EO@d\@roman{\EO@l}\endcsname{%
                 \ifnum\EO@m=0\noexpand\noexpand\EOzero%
                 \else\expandafter\noexpand%
                 \expandafter\csname EO\@roman{\EO@m}\endcsname\fi}
            \expandafter\@rightappend\csname EO@d\@roman{\EO@l}\endcsname
                     \t@\epi@lmecDigits}}
%</epiolmec>
%    \end{macrocode}
%
% %%%%%%%%%%%%%%%%%%%%%%%%%%%%%%%%%
% \section{Usage Examples}
% %%%%%%%%%%%%%%%%%%%%%%%%%%%%%%%%%
% If one wishes to use the package with $\Lambda$, then the easiest way to 
% correctly typeset Epi-Olmec text is to have a minipage and set 
% accordingly the text and paragraph directions: 
% \begin{center}
%   \verb|\begin{minipage}{80pt} |\\
%   \verb| \textdir LTL\pardir LTL|\\
%   \verb|\EOku \EOji \EOkuu     |\\
%   \verb|\EOtze\\ \EOstep       |\\
%   \verb|\end{minipage}         |\\
% \end{center}
% With \LaTeX\ one can typeset Epi-Olmec text using a construct like the
% following one:
%  \begin{center}
%     \verb| \begin{minipage}{80pt}|\\
%     \verb|  \begin{multicols}{3}|\\
%     \verb|    \EOku\\ \EOji\\   |\\
%     \verb|    \EOtze\\ \EOstep  |\\
%     \verb|  \end{multicols}     |\\
%     \verb|\end{minipage}        |\\
%  \end{center}
% Note that one cannot correctly typeset numbers using thse constructs. One
% has to redefine the macro |\@vigdigit| so that it appends the |\\| token
% to each digit that it computes.
%\Finale