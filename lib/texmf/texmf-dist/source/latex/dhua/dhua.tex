\ProvidesFile{dhua.tex}[2011/09/19 make dhua.pdf about dhua.sty]
\title{\textsf{dhua.sty}\\---\\German Abbreviations 
       Using Thin Space\thanks{This
       document describes version
       \textcolor{blue}{\UseVersionOf{\jobname.sty}}
       of \textsf{\jobname.sty} as of \UseDateOf{\jobname.sty}.}}
% \listfiles
{ \RequirePackage{makedoc} \ProcessLineMessage{}
  \MakeJobDoc{16}
  {\SectionLevelTwoParseInput}  }
\documentclass[fleqn]{article}      %% TODO paper dimensions!?
\ProvidesFile{makedoc.cfg}[{2013/03/25 documentation settings}] 
%%
\author{Uwe L\"uck\thanks{%
        \url{http://contact-ednotes.sty.de.vu}}}
%%
%% 'hyperref':
\RequirePackage{ifpdf}
\usepackage[%
  \ifpdf
%     bookmarks=false,                  %% 2010/12/22
%     bookmarksnumbered,
    bookmarksopen,                      %% 2011/01/24!?
    bookmarksopenlevel=2,               %% 2011/01/23
%     pdfpagemode=UseNone,
%     pdfstartpage=10,
    pdfstartview=FitH,                  %% 2012/11/26 again
%     pdfstartview=0 0 100,             %% 2011/08/22
%     pdfstartview={XYZ null null 1},   %% 2011/08/25
%     pdfstartview={XYZ null null null},%% 2011/08/25
%     pdfstartview={XYZ null null .5},    %% 2011/08/26
%     pdffitwindow=true,          %% 2011/08/22
    citebordercolor={ .6 1    .6},
    filebordercolor={1    .6 1},
    linkbordercolor={1    .9  .7},
     urlbordercolor={ .7 1   1},   %% playing 2011/01/24
  \else
    draft
  \fi
]{hyperref}
\hypersetup{% 
    pdfauthor={Uwe L\374ck}% 
}
%% metadata, |\MDkeywords{<text>}|, |\MDkeywordsstring|:
%% %% 2011/08/22:
\makeatletter
  \newcommand*{\MDkeywords}[1]{%
    \gdef\MDkeywordsstring{#1}%
    \hypersetup{pdfkeywords=\MDkeywordsstring}%% TODO!?
  }
  \@onlypreamble\MDkeywords
%% |\MDaddtoabstract{<par-head>}|, `:' added:
  \newcommand*{\MDaddtoabstract}[1]{%           %% 2012/05/10
    \par\smallskip\noindent
    \strong{#1:}\quad\ignorespaces}
%% \pagebreak[2]
%% |\printMDkeywords|:
  \newcommand*{\printMDkeywords}{%
    \MDaddtoabstract{Keywords}%
    \MDkeywordsstring 
%     \global\let\MDkeywordsstring\relax    %% `%' 2012/11/12
  }
%% The previous definitions mainly are useful with a variant 
%% |\begin{MDabstract}| of \LaTeX's `{abstract}' environment:
  \newenvironment{MDabstract}
                 {\abstract\noindent
                  \hspace{1sp}%% for niceverb
                  \ignorespaces}
                 {\@ifundefined{MDkeywordsstring}%
                               {}%
                               {\printMDkeywords}%
                  \global\let\MDabstract\relax    %% 2012/11/12
                  \global\let\endMDabstract\relax %% 2012/11/12
                  \endabstract}
%% |\[MD]docnewline| 2012/11/12 from `readprov.tex':
  \newcommand*{\MDdocnewline}{\leavevmode\@normalcr[\topsep]}
%% <- `\leavevmode' for use with `\paragraph'.
%%    Sometimes needs to be preceded by a space.
%% 
%% |\MDfinaldatechecks[<tex-script>]| with \ctanpkgref{filedate}:
  \newcommand*{\MDfinaldatechecks}[1][fdatechk]{%
    \AtEndDocument{%
%       \clearpage %% 2013/03/25 no avail -- with `filedate'!
      \def\@pkgextension{sty}%
      \def\NeedsTeXFormat##1[##2]{}%
      \noNiceVerb                       %% 2013/03/22
      \input{#1}%
    }}
  \@onlypreamble\MDfinaldatechecks
\makeatother
%% Use other packages:
\RequirePackage{niceverb}[2011/01/24] 
\RequirePackage{readprov}               %% 2010/12/08
\RequirePackage{hypertoc}               %% 2011/01/23
\RequirePackage{texlinks}               %% 2011/01/24
\RequirePackage{relsize}                %% 2011/06/27
\RequirePackage{color}                  %% 2011/08/06
\RequirePackage{lmodern}                %% 2012/10/29
\RequirePackage{filedate}               %% 2012/11/12
\RequirePackage{filesdo}                %% 2013/03/22 
%% \pagebreak[3]
%% Logical markup:\qquad  |\strong{<chars>}|, |\meta{<chars>}|, 
%% |\acro{<chars>}|, |\pkg{<chars>}|, 
%% |\code{<chars>}|, |\file{<chars>}|:{\sloppy\par}
\makeatletter
  \def\do#1#2{\@ifdefinable#1{\let#1#2}}%% 2012/07/13
  \do\strong\textbf \do\file\texttt \do\acro\textsmaller 
  %% <- wrong tests before 2012/07/13
  \do\meta\textit   \do \pkg\textsf \do\code\texttt
  \ifpdf
    \pdfstringdefDisableCommands{%
        \let\acro\textrm 
        \let\file\textrm                            %% 2011/11/09
        \let\code\textrm                            %% 2011/11/20
        \let\pkg \textrm                            %% 2012/03/23
    }
  \fi
  %% TODO 2011/07/22 -> `htlogml.sty'
\makeatother
%% |\qtdcode{<text>}|: 2012/10/24:
    \newcommand*{\qtdcode}[1]{`\code{#1}'} 
%% |\pkgtitle{<package-name>}{<caption>}| 
\newcommand*{\pkgtitle}[2]{%            %% 2012/07/13
    \global\let\pkgtitle\relax
    \pkg{\huge #1}\\---\\#2\thanks{This 
       document describes version 
       \textcolor{blue}{\UseVersionOf{\jobname.sty}} 
       of \textsf{\jobname.sty} as of \UseDateOf{\jobname.sty}.}}
%% TODO: %% |\TODO| bad with `mdoccorr.cfg'
\newcommand*{\TODO}{\textcolor{blue}{\acro{TODO}}}  %% 2012/11/06
%% `\MDsampleinput[{<file>}' was added 2012/11/06. 
%% Problems with `myfilist.tex' were due to 'parskip.sty'
%% there. On 2012/11/12, we change the former simple macro to a 
%% much more complex
%% |\MDsamplecodeinput[<add-hfuss>]{<file>}| 
\newcommand*{\MDsamplecodeinput}[2][]{%
    \begingroup
        \vskip\bigskipamount \hrule
        \nobreak\vskip-\parskip 
%         \nobreak\vskip\medskipamount
%% Previous mistake (same below) due to manual change 
%% of `\topsep' in the file `myfilist.tex' (2012/11/30).
        \ifx\\#1\\\else
            \hfuzz=\textwidth \advance\hfuzz#1\relax
        \fi
        \noNiceVerb \verbatiminput{#2}%
%         \nobreak\vskip\medskipamount 
        \hrule \vskip-\parskip 
        \bigskip %%% \bigbreak
%% `\bigbreak' made much larger space in `myfilist.tex'.
    \endgroup
}
%% |\ctanpkgdref{<pkg-id>}| adds the printed link to 
%% `ctan.org/pkg' as a footnote. There is a little space 
%% for coloured link borders:
\newcommand*{\ctanpkgdref}[1]{%
    \ctanpkgref{#1}\,\urlfoot{CtanPkgRef}{#1}}
\errorcontextlines=4
\pagestyle{headings}

\endinput 
                 %% shared formatting settings
\providecommand*\pkg{\pkgnamefmt}   %% TODO should be in makedoc.cfg
%% 2011/09/09:
\AddQuotes                          %% niceverb
\hypersetup{%
    pdftitle=dhua.sty for German abbreviations using thin space,
    pdfsubject=documenting dhua.sty
}
\MDkeywords{German typography; 
            web typography, language support, macro programming}
\usepackage{dhua}
% \makeatletter
\newenvironment*{german}
                {\par
                 \it\frenchspacing\DontAddQuotes
                 \let\qtd\deqtd}
%                 {\par\@endpefalse}    %% strange: in vain
                {\par}
% \makeatother
% \usepackage[T1]{fontenc}    %% TODO just for ...!?
% \newcommand*{\deqtd}[1]{\guillemotleft#1\guillemotright}
\let\EnToday\today
\usepackage{german} \mdqoff \let\today\EnToday
\renewcommand*{\contentsname}{Contents} 
\renewcommand*{\abstractname}{Abstract/Zusammenfassung}
\newcommand*{\deqtd}[1]{\glqq#1\grqq}
\newcommand*{\HTML}{\acro{HTML}}
\sloppy 
\begin{document}
\maketitle
\begin{MDabstract}\DontAddQuotes
'dhua.sty' provides commands for German phrase abbreviations 
such as \qtd{\dh}\ that are recommended to use a 
\Wikienref{thin space}---set-up commands `\newdhua' and 
`\newtwopartdhua' %%% --- %% rm. 2011/09/16
as well as commands for single cases (e.g., `\zB' for \qtd{\zB}, 
to save you from typing `z.\,B.').
% %% 2011/09/15:
% Moreover, there are package options for (i)~web typography 
% and (ii)~automatical inclusion of `\xspace'.
Package options are intended to support generating \acro{PDF}
and \HTML\ from the same source, maybe automatically using 
`\xspace'.

\begin{german}                                  %% reworded 2011/09/16
  Das Paket 'dhua' bietet Befehle f\"ur sog.\ 
  mehrgliedrige Abk\"urzungen, f\"ur die
  \wikideref{Schmales Lehrzeichen}{schmale Leerzeichen}
  (\wikideref{Festabstand}{Festabst\"ande}) empfohlen werden.
  In die englische Paketdokumentation sind deutsche Hinweise 
  (kursiv) eingestreut.
\end{german}
\end{MDabstract}
\tableofcontents

%   \newpage
\section{Installing, Calling, Usage}
The file 'dhua.sty' is provided ready, installation only requires
putting it somewhere where \TeX\ finds it
(which may need updating the filename data
 base).\urlfoot{ukfaqref}{inst-wlcf}           %% corr. 2011/02/08

%% extended 2011/01/14:
Below the `\documentclass' line(s) and above `\begin{document}',
you load 'dhua.sty' (as usually) by
\[`\usepackage{dhua}'\]
or by 
\[`\usepackage[<option(s)>]{dhua}'\]
with the option(s) <option(s)> described in Section~\ref{sec:opt} 
(`[web]', `[xspace]'). A few macros for single abbreviations 
are described in Section~\ref{sec:single}, 
the macros `\newdhua' and `\newtwopartdhua' for defining such 
abbreviation macros are described in Section~\ref{sec:setup}.

\begin{german}
  Unten werden (i)~Paketoptionen `[web]' und `[xspace]', 
  (ii)~die Makros `\newdhua' und `\newtwopartdhua' 
  f\"ur die Definition einzelner Ab\-k\"ur\-zungs\-makros 
  sowie (iii)~einzelne vordefinierte Abk\"urzungsmakros beschrieben.
\end{german}

\section{Package File Header (Legalize)}
\ProvidesFile{dhua.tex}[2011/09/19 make dhua.pdf about dhua.sty]
\title{\textsf{dhua.sty}\\---\\German Abbreviations 
       Using Thin Space\thanks{This
       document describes version
       \textcolor{blue}{\UseVersionOf{\jobname.sty}}
       of \textsf{\jobname.sty} as of \UseDateOf{\jobname.sty}.}}
% \listfiles
{ \RequirePackage{makedoc} \ProcessLineMessage{}
  \MakeJobDoc{16}
  {\SectionLevelTwoParseInput}  }
\documentclass[fleqn]{article}      %% TODO paper dimensions!?
\ProvidesFile{makedoc.cfg}[{2013/03/25 documentation settings}] 
%%
\author{Uwe L\"uck\thanks{%
        \url{http://contact-ednotes.sty.de.vu}}}
%%
%% 'hyperref':
\RequirePackage{ifpdf}
\usepackage[%
  \ifpdf
%     bookmarks=false,                  %% 2010/12/22
%     bookmarksnumbered,
    bookmarksopen,                      %% 2011/01/24!?
    bookmarksopenlevel=2,               %% 2011/01/23
%     pdfpagemode=UseNone,
%     pdfstartpage=10,
    pdfstartview=FitH,                  %% 2012/11/26 again
%     pdfstartview=0 0 100,             %% 2011/08/22
%     pdfstartview={XYZ null null 1},   %% 2011/08/25
%     pdfstartview={XYZ null null null},%% 2011/08/25
%     pdfstartview={XYZ null null .5},    %% 2011/08/26
%     pdffitwindow=true,          %% 2011/08/22
    citebordercolor={ .6 1    .6},
    filebordercolor={1    .6 1},
    linkbordercolor={1    .9  .7},
     urlbordercolor={ .7 1   1},   %% playing 2011/01/24
  \else
    draft
  \fi
]{hyperref}
\hypersetup{% 
    pdfauthor={Uwe L\374ck}% 
}
%% metadata, |\MDkeywords{<text>}|, |\MDkeywordsstring|:
%% %% 2011/08/22:
\makeatletter
  \newcommand*{\MDkeywords}[1]{%
    \gdef\MDkeywordsstring{#1}%
    \hypersetup{pdfkeywords=\MDkeywordsstring}%% TODO!?
  }
  \@onlypreamble\MDkeywords
%% |\MDaddtoabstract{<par-head>}|, `:' added:
  \newcommand*{\MDaddtoabstract}[1]{%           %% 2012/05/10
    \par\smallskip\noindent
    \strong{#1:}\quad\ignorespaces}
%% \pagebreak[2]
%% |\printMDkeywords|:
  \newcommand*{\printMDkeywords}{%
    \MDaddtoabstract{Keywords}%
    \MDkeywordsstring 
%     \global\let\MDkeywordsstring\relax    %% `%' 2012/11/12
  }
%% The previous definitions mainly are useful with a variant 
%% |\begin{MDabstract}| of \LaTeX's `{abstract}' environment:
  \newenvironment{MDabstract}
                 {\abstract\noindent
                  \hspace{1sp}%% for niceverb
                  \ignorespaces}
                 {\@ifundefined{MDkeywordsstring}%
                               {}%
                               {\printMDkeywords}%
                  \global\let\MDabstract\relax    %% 2012/11/12
                  \global\let\endMDabstract\relax %% 2012/11/12
                  \endabstract}
%% |\[MD]docnewline| 2012/11/12 from `readprov.tex':
  \newcommand*{\MDdocnewline}{\leavevmode\@normalcr[\topsep]}
%% <- `\leavevmode' for use with `\paragraph'.
%%    Sometimes needs to be preceded by a space.
%% 
%% |\MDfinaldatechecks[<tex-script>]| with \ctanpkgref{filedate}:
  \newcommand*{\MDfinaldatechecks}[1][fdatechk]{%
    \AtEndDocument{%
%       \clearpage %% 2013/03/25 no avail -- with `filedate'!
      \def\@pkgextension{sty}%
      \def\NeedsTeXFormat##1[##2]{}%
      \noNiceVerb                       %% 2013/03/22
      \input{#1}%
    }}
  \@onlypreamble\MDfinaldatechecks
\makeatother
%% Use other packages:
\RequirePackage{niceverb}[2011/01/24] 
\RequirePackage{readprov}               %% 2010/12/08
\RequirePackage{hypertoc}               %% 2011/01/23
\RequirePackage{texlinks}               %% 2011/01/24
\RequirePackage{relsize}                %% 2011/06/27
\RequirePackage{color}                  %% 2011/08/06
\RequirePackage{lmodern}                %% 2012/10/29
\RequirePackage{filedate}               %% 2012/11/12
\RequirePackage{filesdo}                %% 2013/03/22 
%% \pagebreak[3]
%% Logical markup:\qquad  |\strong{<chars>}|, |\meta{<chars>}|, 
%% |\acro{<chars>}|, |\pkg{<chars>}|, 
%% |\code{<chars>}|, |\file{<chars>}|:{\sloppy\par}
\makeatletter
  \def\do#1#2{\@ifdefinable#1{\let#1#2}}%% 2012/07/13
  \do\strong\textbf \do\file\texttt \do\acro\textsmaller 
  %% <- wrong tests before 2012/07/13
  \do\meta\textit   \do \pkg\textsf \do\code\texttt
  \ifpdf
    \pdfstringdefDisableCommands{%
        \let\acro\textrm 
        \let\file\textrm                            %% 2011/11/09
        \let\code\textrm                            %% 2011/11/20
        \let\pkg \textrm                            %% 2012/03/23
    }
  \fi
  %% TODO 2011/07/22 -> `htlogml.sty'
\makeatother
%% |\qtdcode{<text>}|: 2012/10/24:
    \newcommand*{\qtdcode}[1]{`\code{#1}'} 
%% |\pkgtitle{<package-name>}{<caption>}| 
\newcommand*{\pkgtitle}[2]{%            %% 2012/07/13
    \global\let\pkgtitle\relax
    \pkg{\huge #1}\\---\\#2\thanks{This 
       document describes version 
       \textcolor{blue}{\UseVersionOf{\jobname.sty}} 
       of \textsf{\jobname.sty} as of \UseDateOf{\jobname.sty}.}}
%% TODO: %% |\TODO| bad with `mdoccorr.cfg'
\newcommand*{\TODO}{\textcolor{blue}{\acro{TODO}}}  %% 2012/11/06
%% `\MDsampleinput[{<file>}' was added 2012/11/06. 
%% Problems with `myfilist.tex' were due to 'parskip.sty'
%% there. On 2012/11/12, we change the former simple macro to a 
%% much more complex
%% |\MDsamplecodeinput[<add-hfuss>]{<file>}| 
\newcommand*{\MDsamplecodeinput}[2][]{%
    \begingroup
        \vskip\bigskipamount \hrule
        \nobreak\vskip-\parskip 
%         \nobreak\vskip\medskipamount
%% Previous mistake (same below) due to manual change 
%% of `\topsep' in the file `myfilist.tex' (2012/11/30).
        \ifx\\#1\\\else
            \hfuzz=\textwidth \advance\hfuzz#1\relax
        \fi
        \noNiceVerb \verbatiminput{#2}%
%         \nobreak\vskip\medskipamount 
        \hrule \vskip-\parskip 
        \bigskip %%% \bigbreak
%% `\bigbreak' made much larger space in `myfilist.tex'.
    \endgroup
}
%% |\ctanpkgdref{<pkg-id>}| adds the printed link to 
%% `ctan.org/pkg' as a footnote. There is a little space 
%% for coloured link borders:
\newcommand*{\ctanpkgdref}[1]{%
    \ctanpkgref{#1}\,\urlfoot{CtanPkgRef}{#1}}
\errorcontextlines=4
\pagestyle{headings}

\endinput 
                 %% shared formatting settings
\providecommand*\pkg{\pkgnamefmt}   %% TODO should be in makedoc.cfg
%% 2011/09/09:
\AddQuotes                          %% niceverb
\hypersetup{%
    pdftitle=dhua.sty for German abbreviations using thin space,
    pdfsubject=documenting dhua.sty
}
\MDkeywords{German typography; 
            web typography, language support, macro programming}
\usepackage{dhua}
% \makeatletter
\newenvironment*{german}
                {\par
                 \it\frenchspacing\DontAddQuotes
                 \let\qtd\deqtd}
%                 {\par\@endpefalse}    %% strange: in vain
                {\par}
% \makeatother
% \usepackage[T1]{fontenc}    %% TODO just for ...!?
% \newcommand*{\deqtd}[1]{\guillemotleft#1\guillemotright}
\let\EnToday\today
\usepackage{german} \mdqoff \let\today\EnToday
\renewcommand*{\contentsname}{Contents} 
\renewcommand*{\abstractname}{Abstract/Zusammenfassung}
\newcommand*{\deqtd}[1]{\glqq#1\grqq}
\newcommand*{\HTML}{\acro{HTML}}
\sloppy 
\begin{document}
\maketitle
\begin{MDabstract}\DontAddQuotes
'dhua.sty' provides commands for German phrase abbreviations 
such as \qtd{\dh}\ that are recommended to use a 
\Wikienref{thin space}---set-up commands `\newdhua' and 
`\newtwopartdhua' %%% --- %% rm. 2011/09/16
as well as commands for single cases (e.g., `\zB' for \qtd{\zB}, 
to save you from typing `z.\,B.').
% %% 2011/09/15:
% Moreover, there are package options for (i)~web typography 
% and (ii)~automatical inclusion of `\xspace'.
Package options are intended to support generating \acro{PDF}
and \HTML\ from the same source, maybe automatically using 
`\xspace'.

\begin{german}                                  %% reworded 2011/09/16
  Das Paket 'dhua' bietet Befehle f\"ur sog.\ 
  mehrgliedrige Abk\"urzungen, f\"ur die
  \wikideref{Schmales Lehrzeichen}{schmale Leerzeichen}
  (\wikideref{Festabstand}{Festabst\"ande}) empfohlen werden.
  In die englische Paketdokumentation sind deutsche Hinweise 
  (kursiv) eingestreut.
\end{german}
\end{MDabstract}
\tableofcontents

%   \newpage
\section{Installing, Calling, Usage}
The file 'dhua.sty' is provided ready, installation only requires
putting it somewhere where \TeX\ finds it
(which may need updating the filename data
 base).\urlfoot{ukfaqref}{inst-wlcf}           %% corr. 2011/02/08

%% extended 2011/01/14:
Below the `\documentclass' line(s) and above `\begin{document}',
you load 'dhua.sty' (as usually) by
\[`\usepackage{dhua}'\]
or by 
\[`\usepackage[<option(s)>]{dhua}'\]
with the option(s) <option(s)> described in Section~\ref{sec:opt} 
(`[web]', `[xspace]'). A few macros for single abbreviations 
are described in Section~\ref{sec:single}, 
the macros `\newdhua' and `\newtwopartdhua' for defining such 
abbreviation macros are described in Section~\ref{sec:setup}.

\begin{german}
  Unten werden (i)~Paketoptionen `[web]' und `[xspace]', 
  (ii)~die Makros `\newdhua' und `\newtwopartdhua' 
  f\"ur die Definition einzelner Ab\-k\"ur\-zungs\-makros 
  sowie (iii)~einzelne vordefinierte Abk\"urzungsmakros beschrieben.
\end{german}

\section{Package File Header (Legalize)}
\ProvidesFile{dhua.tex}[2011/09/19 make dhua.pdf about dhua.sty]
\title{\textsf{dhua.sty}\\---\\German Abbreviations 
       Using Thin Space\thanks{This
       document describes version
       \textcolor{blue}{\UseVersionOf{\jobname.sty}}
       of \textsf{\jobname.sty} as of \UseDateOf{\jobname.sty}.}}
% \listfiles
{ \RequirePackage{makedoc} \ProcessLineMessage{}
  \MakeJobDoc{16}
  {\SectionLevelTwoParseInput}  }
\documentclass[fleqn]{article}      %% TODO paper dimensions!?
\ProvidesFile{makedoc.cfg}[{2013/03/25 documentation settings}] 
%%
\author{Uwe L\"uck\thanks{%
        \url{http://contact-ednotes.sty.de.vu}}}
%%
%% 'hyperref':
\RequirePackage{ifpdf}
\usepackage[%
  \ifpdf
%     bookmarks=false,                  %% 2010/12/22
%     bookmarksnumbered,
    bookmarksopen,                      %% 2011/01/24!?
    bookmarksopenlevel=2,               %% 2011/01/23
%     pdfpagemode=UseNone,
%     pdfstartpage=10,
    pdfstartview=FitH,                  %% 2012/11/26 again
%     pdfstartview=0 0 100,             %% 2011/08/22
%     pdfstartview={XYZ null null 1},   %% 2011/08/25
%     pdfstartview={XYZ null null null},%% 2011/08/25
%     pdfstartview={XYZ null null .5},    %% 2011/08/26
%     pdffitwindow=true,          %% 2011/08/22
    citebordercolor={ .6 1    .6},
    filebordercolor={1    .6 1},
    linkbordercolor={1    .9  .7},
     urlbordercolor={ .7 1   1},   %% playing 2011/01/24
  \else
    draft
  \fi
]{hyperref}
\hypersetup{% 
    pdfauthor={Uwe L\374ck}% 
}
%% metadata, |\MDkeywords{<text>}|, |\MDkeywordsstring|:
%% %% 2011/08/22:
\makeatletter
  \newcommand*{\MDkeywords}[1]{%
    \gdef\MDkeywordsstring{#1}%
    \hypersetup{pdfkeywords=\MDkeywordsstring}%% TODO!?
  }
  \@onlypreamble\MDkeywords
%% |\MDaddtoabstract{<par-head>}|, `:' added:
  \newcommand*{\MDaddtoabstract}[1]{%           %% 2012/05/10
    \par\smallskip\noindent
    \strong{#1:}\quad\ignorespaces}
%% \pagebreak[2]
%% |\printMDkeywords|:
  \newcommand*{\printMDkeywords}{%
    \MDaddtoabstract{Keywords}%
    \MDkeywordsstring 
%     \global\let\MDkeywordsstring\relax    %% `%' 2012/11/12
  }
%% The previous definitions mainly are useful with a variant 
%% |\begin{MDabstract}| of \LaTeX's `{abstract}' environment:
  \newenvironment{MDabstract}
                 {\abstract\noindent
                  \hspace{1sp}%% for niceverb
                  \ignorespaces}
                 {\@ifundefined{MDkeywordsstring}%
                               {}%
                               {\printMDkeywords}%
                  \global\let\MDabstract\relax    %% 2012/11/12
                  \global\let\endMDabstract\relax %% 2012/11/12
                  \endabstract}
%% |\[MD]docnewline| 2012/11/12 from `readprov.tex':
  \newcommand*{\MDdocnewline}{\leavevmode\@normalcr[\topsep]}
%% <- `\leavevmode' for use with `\paragraph'.
%%    Sometimes needs to be preceded by a space.
%% 
%% |\MDfinaldatechecks[<tex-script>]| with \ctanpkgref{filedate}:
  \newcommand*{\MDfinaldatechecks}[1][fdatechk]{%
    \AtEndDocument{%
%       \clearpage %% 2013/03/25 no avail -- with `filedate'!
      \def\@pkgextension{sty}%
      \def\NeedsTeXFormat##1[##2]{}%
      \noNiceVerb                       %% 2013/03/22
      \input{#1}%
    }}
  \@onlypreamble\MDfinaldatechecks
\makeatother
%% Use other packages:
\RequirePackage{niceverb}[2011/01/24] 
\RequirePackage{readprov}               %% 2010/12/08
\RequirePackage{hypertoc}               %% 2011/01/23
\RequirePackage{texlinks}               %% 2011/01/24
\RequirePackage{relsize}                %% 2011/06/27
\RequirePackage{color}                  %% 2011/08/06
\RequirePackage{lmodern}                %% 2012/10/29
\RequirePackage{filedate}               %% 2012/11/12
\RequirePackage{filesdo}                %% 2013/03/22 
%% \pagebreak[3]
%% Logical markup:\qquad  |\strong{<chars>}|, |\meta{<chars>}|, 
%% |\acro{<chars>}|, |\pkg{<chars>}|, 
%% |\code{<chars>}|, |\file{<chars>}|:{\sloppy\par}
\makeatletter
  \def\do#1#2{\@ifdefinable#1{\let#1#2}}%% 2012/07/13
  \do\strong\textbf \do\file\texttt \do\acro\textsmaller 
  %% <- wrong tests before 2012/07/13
  \do\meta\textit   \do \pkg\textsf \do\code\texttt
  \ifpdf
    \pdfstringdefDisableCommands{%
        \let\acro\textrm 
        \let\file\textrm                            %% 2011/11/09
        \let\code\textrm                            %% 2011/11/20
        \let\pkg \textrm                            %% 2012/03/23
    }
  \fi
  %% TODO 2011/07/22 -> `htlogml.sty'
\makeatother
%% |\qtdcode{<text>}|: 2012/10/24:
    \newcommand*{\qtdcode}[1]{`\code{#1}'} 
%% |\pkgtitle{<package-name>}{<caption>}| 
\newcommand*{\pkgtitle}[2]{%            %% 2012/07/13
    \global\let\pkgtitle\relax
    \pkg{\huge #1}\\---\\#2\thanks{This 
       document describes version 
       \textcolor{blue}{\UseVersionOf{\jobname.sty}} 
       of \textsf{\jobname.sty} as of \UseDateOf{\jobname.sty}.}}
%% TODO: %% |\TODO| bad with `mdoccorr.cfg'
\newcommand*{\TODO}{\textcolor{blue}{\acro{TODO}}}  %% 2012/11/06
%% `\MDsampleinput[{<file>}' was added 2012/11/06. 
%% Problems with `myfilist.tex' were due to 'parskip.sty'
%% there. On 2012/11/12, we change the former simple macro to a 
%% much more complex
%% |\MDsamplecodeinput[<add-hfuss>]{<file>}| 
\newcommand*{\MDsamplecodeinput}[2][]{%
    \begingroup
        \vskip\bigskipamount \hrule
        \nobreak\vskip-\parskip 
%         \nobreak\vskip\medskipamount
%% Previous mistake (same below) due to manual change 
%% of `\topsep' in the file `myfilist.tex' (2012/11/30).
        \ifx\\#1\\\else
            \hfuzz=\textwidth \advance\hfuzz#1\relax
        \fi
        \noNiceVerb \verbatiminput{#2}%
%         \nobreak\vskip\medskipamount 
        \hrule \vskip-\parskip 
        \bigskip %%% \bigbreak
%% `\bigbreak' made much larger space in `myfilist.tex'.
    \endgroup
}
%% |\ctanpkgdref{<pkg-id>}| adds the printed link to 
%% `ctan.org/pkg' as a footnote. There is a little space 
%% for coloured link borders:
\newcommand*{\ctanpkgdref}[1]{%
    \ctanpkgref{#1}\,\urlfoot{CtanPkgRef}{#1}}
\errorcontextlines=4
\pagestyle{headings}

\endinput 
                 %% shared formatting settings
\providecommand*\pkg{\pkgnamefmt}   %% TODO should be in makedoc.cfg
%% 2011/09/09:
\AddQuotes                          %% niceverb
\hypersetup{%
    pdftitle=dhua.sty for German abbreviations using thin space,
    pdfsubject=documenting dhua.sty
}
\MDkeywords{German typography; 
            web typography, language support, macro programming}
\usepackage{dhua}
% \makeatletter
\newenvironment*{german}
                {\par
                 \it\frenchspacing\DontAddQuotes
                 \let\qtd\deqtd}
%                 {\par\@endpefalse}    %% strange: in vain
                {\par}
% \makeatother
% \usepackage[T1]{fontenc}    %% TODO just for ...!?
% \newcommand*{\deqtd}[1]{\guillemotleft#1\guillemotright}
\let\EnToday\today
\usepackage{german} \mdqoff \let\today\EnToday
\renewcommand*{\contentsname}{Contents} 
\renewcommand*{\abstractname}{Abstract/Zusammenfassung}
\newcommand*{\deqtd}[1]{\glqq#1\grqq}
\newcommand*{\HTML}{\acro{HTML}}
\sloppy 
\begin{document}
\maketitle
\begin{MDabstract}\DontAddQuotes
'dhua.sty' provides commands for German phrase abbreviations 
such as \qtd{\dh}\ that are recommended to use a 
\Wikienref{thin space}---set-up commands `\newdhua' and 
`\newtwopartdhua' %%% --- %% rm. 2011/09/16
as well as commands for single cases (e.g., `\zB' for \qtd{\zB}, 
to save you from typing `z.\,B.').
% %% 2011/09/15:
% Moreover, there are package options for (i)~web typography 
% and (ii)~automatical inclusion of `\xspace'.
Package options are intended to support generating \acro{PDF}
and \HTML\ from the same source, maybe automatically using 
`\xspace'.

\begin{german}                                  %% reworded 2011/09/16
  Das Paket 'dhua' bietet Befehle f\"ur sog.\ 
  mehrgliedrige Abk\"urzungen, f\"ur die
  \wikideref{Schmales Lehrzeichen}{schmale Leerzeichen}
  (\wikideref{Festabstand}{Festabst\"ande}) empfohlen werden.
  In die englische Paketdokumentation sind deutsche Hinweise 
  (kursiv) eingestreut.
\end{german}
\end{MDabstract}
\tableofcontents

%   \newpage
\section{Installing, Calling, Usage}
The file 'dhua.sty' is provided ready, installation only requires
putting it somewhere where \TeX\ finds it
(which may need updating the filename data
 base).\urlfoot{ukfaqref}{inst-wlcf}           %% corr. 2011/02/08

%% extended 2011/01/14:
Below the `\documentclass' line(s) and above `\begin{document}',
you load 'dhua.sty' (as usually) by
\[`\usepackage{dhua}'\]
or by 
\[`\usepackage[<option(s)>]{dhua}'\]
with the option(s) <option(s)> described in Section~\ref{sec:opt} 
(`[web]', `[xspace]'). A few macros for single abbreviations 
are described in Section~\ref{sec:single}, 
the macros `\newdhua' and `\newtwopartdhua' for defining such 
abbreviation macros are described in Section~\ref{sec:setup}.

\begin{german}
  Unten werden (i)~Paketoptionen `[web]' und `[xspace]', 
  (ii)~die Makros `\newdhua' und `\newtwopartdhua' 
  f\"ur die Definition einzelner Ab\-k\"ur\-zungs\-makros 
  sowie (iii)~einzelne vordefinierte Abk\"urzungsmakros beschrieben.
\end{german}

\section{Package File Header (Legalize)}
\ProvidesFile{dhua.tex}[2011/09/19 make dhua.pdf about dhua.sty]
\title{\textsf{dhua.sty}\\---\\German Abbreviations 
       Using Thin Space\thanks{This
       document describes version
       \textcolor{blue}{\UseVersionOf{\jobname.sty}}
       of \textsf{\jobname.sty} as of \UseDateOf{\jobname.sty}.}}
% \listfiles
{ \RequirePackage{makedoc} \ProcessLineMessage{}
  \MakeJobDoc{16}
  {\SectionLevelTwoParseInput}  }
\documentclass[fleqn]{article}      %% TODO paper dimensions!?
\input{makedoc.cfg}                 %% shared formatting settings
\providecommand*\pkg{\pkgnamefmt}   %% TODO should be in makedoc.cfg
%% 2011/09/09:
\AddQuotes                          %% niceverb
\hypersetup{%
    pdftitle=dhua.sty for German abbreviations using thin space,
    pdfsubject=documenting dhua.sty
}
\MDkeywords{German typography; 
            web typography, language support, macro programming}
\usepackage{dhua}
% \makeatletter
\newenvironment*{german}
                {\par
                 \it\frenchspacing\DontAddQuotes
                 \let\qtd\deqtd}
%                 {\par\@endpefalse}    %% strange: in vain
                {\par}
% \makeatother
% \usepackage[T1]{fontenc}    %% TODO just for ...!?
% \newcommand*{\deqtd}[1]{\guillemotleft#1\guillemotright}
\let\EnToday\today
\usepackage{german} \mdqoff \let\today\EnToday
\renewcommand*{\contentsname}{Contents} 
\renewcommand*{\abstractname}{Abstract/Zusammenfassung}
\newcommand*{\deqtd}[1]{\glqq#1\grqq}
\newcommand*{\HTML}{\acro{HTML}}
\sloppy 
\begin{document}
\maketitle
\begin{MDabstract}\DontAddQuotes
'dhua.sty' provides commands for German phrase abbreviations 
such as \qtd{\dh}\ that are recommended to use a 
\Wikienref{thin space}---set-up commands `\newdhua' and 
`\newtwopartdhua' %%% --- %% rm. 2011/09/16
as well as commands for single cases (e.g., `\zB' for \qtd{\zB}, 
to save you from typing `z.\,B.').
% %% 2011/09/15:
% Moreover, there are package options for (i)~web typography 
% and (ii)~automatical inclusion of `\xspace'.
Package options are intended to support generating \acro{PDF}
and \HTML\ from the same source, maybe automatically using 
`\xspace'.

\begin{german}                                  %% reworded 2011/09/16
  Das Paket 'dhua' bietet Befehle f\"ur sog.\ 
  mehrgliedrige Abk\"urzungen, f\"ur die
  \wikideref{Schmales Lehrzeichen}{schmale Leerzeichen}
  (\wikideref{Festabstand}{Festabst\"ande}) empfohlen werden.
  In die englische Paketdokumentation sind deutsche Hinweise 
  (kursiv) eingestreut.
\end{german}
\end{MDabstract}
\tableofcontents

%   \newpage
\section{Installing, Calling, Usage}
The file 'dhua.sty' is provided ready, installation only requires
putting it somewhere where \TeX\ finds it
(which may need updating the filename data
 base).\urlfoot{ukfaqref}{inst-wlcf}           %% corr. 2011/02/08

%% extended 2011/01/14:
Below the `\documentclass' line(s) and above `\begin{document}',
you load 'dhua.sty' (as usually) by
\[`\usepackage{dhua}'\]
or by 
\[`\usepackage[<option(s)>]{dhua}'\]
with the option(s) <option(s)> described in Section~\ref{sec:opt} 
(`[web]', `[xspace]'). A few macros for single abbreviations 
are described in Section~\ref{sec:single}, 
the macros `\newdhua' and `\newtwopartdhua' for defining such 
abbreviation macros are described in Section~\ref{sec:setup}.

\begin{german}
  Unten werden (i)~Paketoptionen `[web]' und `[xspace]', 
  (ii)~die Makros `\newdhua' und `\newtwopartdhua' 
  f\"ur die Definition einzelner Ab\-k\"ur\-zungs\-makros 
  sowie (iii)~einzelne vordefinierte Abk\"urzungsmakros beschrieben.
\end{german}

\section{Package File Header (Legalize)}
\input{dhua.doc}
\section{Colophon}                      %% was `Coda' 2011/09/16
              %% rm. \enlargethispage{3\baselineskip} 2011/09/19
The English part of the documentation exemplifies a new 
(2011/09/09) function of 'niceverb.sty' v0.44: automatically 
enclose inline \TeX\ code in single quotation marks after 
`\AddQuotes'. 
I needed especially much time for this because group nesting 
spans several documentation pages. 

% On my Atari~ST, there must be another 'dhua.sty'. 
% I~guess it was quite worthless. 
% I think it is only about one year ago that I~became aware 
% of the Duden recommendation about thin unbreakable spaces. 
% I consider this older package 'dhua'~v0.\dots
% The present version number \qtd{v1.1} is an analogue 
% to my usual starting version number \qtd{v0.1}.
 
I spent much time with a special environment `{german}'
for the present purpose: the indent of the following 
paragraph was missing---until I added an empty 
documentation line. (Same with standard `{sloppypar}' 
environment, I don't understand it, tried 
`\@endpefalse' in vain.) 
%% modified 2011/09/16:
I don't like \ctanpkgref{babel}~\dots

%% 2011/09/16:
The German parts use 'niceverb''s `\DontAddQuotes' 
because of a different frequency of \TeX\ code. 
Even in the English parts I considered the single 
quotation marks bad and avoided them using \LaTeX's `\verb'.

And my terms \qtd{phrase abbreviation} and \qtd{abbreviation macro} 
may be bad, please help me~\dots

\end{document}

VERSION HISTORY

2011/09/14  for v0.1    renaming from `dhusw'
2011/09/15  for v0.1a   options in abstract
2011/09/16              more keywords, different \Provides..., 
                        Coda/Colophon extended
2011/09/19  for v0.11   Colophon on next page


\section{Colophon}                      %% was `Coda' 2011/09/16
              %% rm. \enlargethispage{3\baselineskip} 2011/09/19
The English part of the documentation exemplifies a new 
(2011/09/09) function of 'niceverb.sty' v0.44: automatically 
enclose inline \TeX\ code in single quotation marks after 
`\AddQuotes'. 
I needed especially much time for this because group nesting 
spans several documentation pages. 

% On my Atari~ST, there must be another 'dhua.sty'. 
% I~guess it was quite worthless. 
% I think it is only about one year ago that I~became aware 
% of the Duden recommendation about thin unbreakable spaces. 
% I consider this older package 'dhua'~v0.\dots
% The present version number \qtd{v1.1} is an analogue 
% to my usual starting version number \qtd{v0.1}.
 
I spent much time with a special environment `{german}'
for the present purpose: the indent of the following 
paragraph was missing---until I added an empty 
documentation line. (Same with standard `{sloppypar}' 
environment, I don't understand it, tried 
`\@endpefalse' in vain.) 
%% modified 2011/09/16:
I don't like \ctanpkgref{babel}~\dots

%% 2011/09/16:
The German parts use 'niceverb''s `\DontAddQuotes' 
because of a different frequency of \TeX\ code. 
Even in the English parts I considered the single 
quotation marks bad and avoided them using \LaTeX's `\verb'.

And my terms \qtd{phrase abbreviation} and \qtd{abbreviation macro} 
may be bad, please help me~\dots

\end{document}

VERSION HISTORY

2011/09/14  for v0.1    renaming from `dhusw'
2011/09/15  for v0.1a   options in abstract
2011/09/16              more keywords, different \Provides..., 
                        Coda/Colophon extended
2011/09/19  for v0.11   Colophon on next page


\section{Colophon}                      %% was `Coda' 2011/09/16
              %% rm. \enlargethispage{3\baselineskip} 2011/09/19
The English part of the documentation exemplifies a new 
(2011/09/09) function of 'niceverb.sty' v0.44: automatically 
enclose inline \TeX\ code in single quotation marks after 
`\AddQuotes'. 
I needed especially much time for this because group nesting 
spans several documentation pages. 

% On my Atari~ST, there must be another 'dhua.sty'. 
% I~guess it was quite worthless. 
% I think it is only about one year ago that I~became aware 
% of the Duden recommendation about thin unbreakable spaces. 
% I consider this older package 'dhua'~v0.\dots
% The present version number \qtd{v1.1} is an analogue 
% to my usual starting version number \qtd{v0.1}.
 
I spent much time with a special environment `{german}'
for the present purpose: the indent of the following 
paragraph was missing---until I added an empty 
documentation line. (Same with standard `{sloppypar}' 
environment, I don't understand it, tried 
`\@endpefalse' in vain.) 
%% modified 2011/09/16:
I don't like \ctanpkgref{babel}~\dots

%% 2011/09/16:
The German parts use 'niceverb''s `\DontAddQuotes' 
because of a different frequency of \TeX\ code. 
Even in the English parts I considered the single 
quotation marks bad and avoided them using \LaTeX's `\verb'.

And my terms \qtd{phrase abbreviation} and \qtd{abbreviation macro} 
may be bad, please help me~\dots

\end{document}

VERSION HISTORY

2011/09/14  for v0.1    renaming from `dhusw'
2011/09/15  for v0.1a   options in abstract
2011/09/16              more keywords, different \Provides..., 
                        Coda/Colophon extended
2011/09/19  for v0.11   Colophon on next page


\section{Colophon}                      %% was `Coda' 2011/09/16
              %% rm. \enlargethispage{3\baselineskip} 2011/09/19
The English part of the documentation exemplifies a new 
(2011/09/09) function of 'niceverb.sty' v0.44: automatically 
enclose inline \TeX\ code in single quotation marks after 
`\AddQuotes'. 
I needed especially much time for this because group nesting 
spans several documentation pages. 

% On my Atari~ST, there must be another 'dhua.sty'. 
% I~guess it was quite worthless. 
% I think it is only about one year ago that I~became aware 
% of the Duden recommendation about thin unbreakable spaces. 
% I consider this older package 'dhua'~v0.\dots
% The present version number \qtd{v1.1} is an analogue 
% to my usual starting version number \qtd{v0.1}.
 
I spent much time with a special environment `{german}'
for the present purpose: the indent of the following 
paragraph was missing---until I added an empty 
documentation line. (Same with standard `{sloppypar}' 
environment, I don't understand it, tried 
`\@endpefalse' in vain.) 
%% modified 2011/09/16:
I don't like \ctanpkgref{babel}~\dots

%% 2011/09/16:
The German parts use 'niceverb''s `\DontAddQuotes' 
because of a different frequency of \TeX\ code. 
Even in the English parts I considered the single 
quotation marks bad and avoided them using \LaTeX's `\verb'.

And my terms \qtd{phrase abbreviation} and \qtd{abbreviation macro} 
may be bad, please help me~\dots

\end{document}

VERSION HISTORY

2011/09/14  for v0.1    renaming from `dhusw'
2011/09/15  for v0.1a   options in abstract
2011/09/16              more keywords, different \Provides..., 
                        Coda/Colophon extended
2011/09/19  for v0.11   Colophon on next page

