% \iffalse
%
% refconfig.dtx
% Copyright (C) 2002--2010 Danie Els
%
% -------------------------------------------------------------------
%                     The refstyle package
%                 for the formatting of references
% -------------------------------------------------------------------
% This work may be distributed and/or modified under the conditions
% of the LaTeX Project Public License, either version 1.3c of this
% license or (at your option) any later version. The latest version
% of this license is in
%      http://www.latex-project.org/lppl.txt
% and version 1.3c or later is part of all distributions of LaTeX
% version 2005/12/01 or later.
%
% This work is "maintained" (as per LPPL maintenance status)
% by Danie Els (dnjels@sun.ac.za).
%
% This package consists of the files: refstyle.dtx
%                                     refconfig.dtx
%                                     refstyle.ins
%              and the derived files: refstyle.sty
%                                     refstyle.cfg
% -------------------------------------------------------------------
%
%
%<*driver>
\documentclass[a4paper]{ltxdoc}
    \EnableCrossrefs
    \CodelineIndex
    \RecordChanges
\usepackage{color}
\usepackage{framed}
    \setlength{\FrameSep}{1\fboxsep}
    \definecolor{shadecolor}{gray}{0.9}
    \newenvironment{RSframed}{\shaded}{\endshaded}
\usepackage{xspace}
\usepackage{varioref}
\usepackage{refstyle}

%-- macrocode env changes ------------------------------

\setlength{\MacroIndent}{1.5em}

%-- Indented environments ------------------------------

\newlength{\tdima}
\newsavebox{\tboxa}
\newlength{\mytab}
\setlength{\mytab}{2\parindent}

\newenvironment{IndentPara}[1][1]
    {\list{}{\setlength{\leftmargin}{#1\mytab}%
             \setlength{\labelwidth}{0pt}%
             \setlength{\labelsep}{0pt}%
             \setlength{\itemindent}{\parindent}%
             \setlength{\listparindent}{\parindent}%
             }\item[]}%
    {\endlist}
\newenvironment{Ipara}[1][\small]%
    {\begin{IndentPara}\noindent#1\ignorespaces}%
    {\end{IndentPara}}
\newenvironment{Itabb}[1][\small]%
    {\begin{IndentPara}#1\ignorespaces\begin{tabbing}\ignorespaces}%
    {\end{tabbing}\end{IndentPara}}

%-- Additional documenting commands --------------------

\makeatletter
\def\meta@font@select{\normalfont\slshape}
\makeatother

\newcommand*{\file}[1]{\texttt{#1}}
\newcommand*{\pkg}[1]{\textsf{#1}}
\newcommand*{\env}[1]{\texttt{#1}}
\newcommand*{\opt}[1]{\texttt{\slshape#1}}
\newcommand*{\RS}{\pkg{refstyle}\xspace}

\newcommand*{\ArTab}{\>\quad$\rightarrow$\quad}

%-------------------------------------------------------

\setlength\hfuzz{15pt}
\hbadness=7000
\begin{document}
    \DocInput{refconfig.dtx}
\end{document}

%</driver>
% \fi
%
% \CheckSum{1098}
%
% \CharacterTable
%  {Upper-case    \A\B\C\D\E\F\G\H\I\J\K\L\M\N\O\P\Q\R\S\T\U\V\W\X\Y\Z
%   Lower-case    \a\b\c\d\e\f\g\h\i\j\k\l\m\n\o\p\q\r\s\t\u\v\w\x\y\z
%   Digits        \0\1\2\3\4\5\6\7\8\9
%   Exclamation   \!     Double quote  \"     Hash (number) \#
%   Dollar        \$     Percent       \%     Ampersand     \&
%   Acute accent  \'     Left paren    \(     Right paren   \)
%   Asterisk      \*     Plus          \+     Comma         \,
%   Minus         \-     Point         \.     Solidus       \/
%   Colon         \:     Semicolon     \;     Less than     \<
%   Equals        \=     Greater than  \>     Question mark \?
%   Commercial at \@     Left bracket  \[     Backslash     \\
%   Right bracket \]     Circumflex    \^     Underscore    \_
%   Grave accent  \`     Left brace    \{     Vertical bar  \|
%   Right brace   \}     Tilde         \~}
%
% \changes{v0.0}{2002/04/01}{Initial version}
% \changes{v0.2}{2003/07/30}{Updated version}
% \changes{v0.3}{2006/09/07}{Documentation update}
% \changes{v0.4}{2010/10/21}{Add \cmd{\RSlsttwotxt} for lists}
% \changes{v0.5}{2010/11/02}{Bug fixes}
%
%   \DoNotIndex{
%   \",\@chapapp,\@ifpackageloaded,\AtBeginDocument,
%   \begingroup,\def,\edef,\else,\endgroup,
%   \fi,\ifx, \let, \newcommand,\newref,
%   \providecommand,\ProvidesFile,\ref,\relax,
%   \renewcommand,\RS@ifundefined,\RS@tmpa,\RS@tmpb,
%   \S,\space,\ss,\string,\textup,\unrestored@protected@xdef}
%
%^^A-- Titling --------------------------------------------
% \GetFileInfo{refstyle.cfg}
% \title{\vskip -2em
%        Configuration file for \RS\ package\thanks{
%                    This file has version number
%                    \fileversion, last revised
%                    \filedate.}}
% \author{Danie Els\\[1ex]
%         \small Department of Mechanical and Mechatroncs Engineering\\
%         \small University of Stellenbosch, South Africa.\\
%    \normalsize e-mail: \texttt{dnjels@sun.ac.za}}
% \date{\filedate}
% \maketitle
%
% \tableofcontents
% \clearpage
%
%
% \section{The \RS\ configuration file}
%
%  \subsubsection*{Reference templates}
%    The default configuration file, \file{refstyle.cfg}, makes the
%    following reference declarations:
% \begin{Ipara}
%  |\newref{part}{...}| \quad$\rightarrow$ Parts\\
%  |\newref{chap}{...}| \quad$\rightarrow$ Chapters and Appendices\\
%  |\newref{sec}{...} | \quad$\rightarrow$ Sections\\
%  |\newref{eq}{...}  | \quad$\rightarrow$ Equations\\
%  |\newref{fig}{...} | \quad$\rightarrow$ Figures\\
%  |\newref{tab}{...} | \quad$\rightarrow$ Tables\\
%  |\newref{fn}{...}  | \quad$\rightarrow$ Footnotes
% \end{Ipara}
%  If there is a need for more reference types, this standard
%  list can be expanded.
%
%  \subsubsection*{Language definitions}
%  The \RS\ package also provides the following language definitions
%  options for references, when loaded together with \pkg{babel}:
%  \begin{itemize}
%  \item afrikaans
%  \item danish
%  \item english, USenglish, american, canadian, UKenglish, british
%  \item french
%  \item german, ngerman, austrian, naustrian
%  \item italian
%  \item norwegian, nynorsk, norsk, bokmal\footnote{Norwegian Bokm{\aa}l
%           is not defined in \pkg{babel} yet.}
%  \item portuges, portuguese, brazilian, brazil
%  \item swedish
%  \end{itemize}
%
%  \noindent See \secref[vref]{Thm} for an example of how to link your
%  own definitions to an exsiting language setup.
% \StopEventually{\PrintChanges}
% \clearpage
% \section{Implementation}
%    \begin{macrocode}
%<*cfg>
%    \end{macrocode}
% \subsection{Identification}
%    \begin{RSframed}
%    \begin{macrocode}
\ProvidesFile{refstyle.cfg}[2010/11/02\space
                            0.5\space
                            Configuration file for refstyle (DNJ Els)]
%    \end{macrocode}
%    \end{RSframed}
%
% \subsection{Language Definitions for \RS\ Package}
%
%    \subsubsection{English}
%
%    \begin{macro}{\RSenglish,\RSukenglish}
%    Support provided by the author.
%    \begin{RSframed}
%    \begin{macrocode}
%%-- ENGLISH ------------------------------------
%    \end{macrocode}
%    \begin{macrocode}
\newcommand\RSukenglish{%
    \def\RSrngtxt{\space to~}%........... Range:     figures 5 to 6
    \def\RSlsttwotxt{\space and~}%....... List two:  figures 5 and 6
    \def\RSlsttxt{, and~}%............... List more: figures 5, 6, and 7
    \def\RSparttxt{Part~}%............... Part     lowercase singular
    \def\RSpartstxt{Parts~}%.............          lowercase plural
    \def\RSParttxt{Part~}%...............          uppercase singular (sentence start)
    \def\RSPartstxt{Parts~}%.............          uppercase plural   (sentence start)
    \def\RSappendixname{appendix~}%...... Appendix lowercase singular
    \def\RSappendicesname{appendices~}%..          lowercase plural
    \def\RSAppendixname{Appendix~}%......          uppercase singular (sentence start)
    \def\RSAppendicesname{Appendices~}%..          uppercase plural   (sentence start)
    \def\RSchaptername{chapter~}%........ Chapter  lowercase singular
    \def\RSchaptersname{chapters~}%......          lowercase plural
    \def\RSChaptername{Chapter~}%........          uppercase singular (sentence start)
    \def\RSChaptersname{Chapters~}%......          uppercase plural   (sentence start)
    \def\RSsectxt{section~}%............. Section  lowercase singular
    \def\RSsecstxt{sections~}%...........          lowercase plural
    \def\RSSectxt{Section~}%.............          uppercase singular (sentence start)
    \def\RSSecstxt{Sections~}%...........          uppercase plural   (sentence start)
    \def\RSeqtxt{equation~}%............. Equation lowercase singular
    \def\RSeqstxt{equations~}%...........          lowercase plural
    \def\RSEqtxt{Equation~}%.............          uppercase singular (sentence start)
    \def\RSEqstxt{Equations~}%...........          uppercase plural   (sentence start)
    \def\RSfigtxt{figure~}%.............. Figure   lowercase singular
    \def\RSfigstxt{figures~}%............          lowercase plural
    \def\RSFigtxt{Figure~}%..............          uppercase singular (sentence start)
    \def\RSFigstxt{Figures~}%............          uppercase plural   (sentence start)
    \def\RStabtxt{table~}%............... Table    lowercase singular
    \def\RStabstxt{tables~}%.............          lowercase plural
    \def\RSTabtxt{Table~}%...............          uppercase singular (sentence start)
    \def\RSTabstxt{Tables~}%.............          uppercase plural   (sentence start)
    \def\RSfootntxt{footnote~}%.......... Footnote lowercase singular
    \def\RSfootnstxt{footnotes~}%........          lowercase plural
    \def\RSFootntxt{Footnote~}%..........          uppercase singular (sentence start)
    \def\RSFootnstxt{Footnotes~}%........          uppercase plural   (sentence start)
}
%    \end{macrocode}
%    \begin{macrocode}
\newcommand\RSenglish{%
    \def\RSrngtxt{\space to~}%
    \def\RSlsttwotxt{\space and~}%
    \def\RSlsttxt{\space and~}%
    \def\RSparttxt{Part~}%
    \def\RSpartstxt{Parts~}%
    \def\RSParttxt{Part~}%
    \def\RSPartstxt{Parts~}%
    \def\RSappendixname{appendix~}%
    \def\RSappendicesname{appendices~}%
    \def\RSAppendixname{Appendix~}%
    \def\RSAppendicesname{Appendices~}%
    \def\RSchaptername{chapter~}%
    \def\RSchaptersname{chapters~}%
    \def\RSChaptername{Chapter~}%
    \def\RSChaptersname{Chapters~}%
    \def\RSsectxt{section~}%
    \def\RSsecstxt{sections~}%
    \def\RSSectxt{Section~}%
    \def\RSSecstxt{Sections~}%
    \def\RSeqtxt{equation~}%
    \def\RSeqstxt{equations~}%
    \def\RSEqtxt{Equation~}%
    \def\RSEqstxt{Equations~}%
    \def\RSfigtxt{figure~}%
    \def\RSfigstxt{figures~}%
    \def\RSFigtxt{Figure~}%
    \def\RSFigstxt{Figures~}%
    \def\RStabtxt{table~}%
    \def\RStabstxt{tables~}%
    \def\RSTabtxt{Table~}%
    \def\RSTabstxt{Tables~}%
    \def\RSfootntxt{footnote~}%
    \def\RSfootnstxt{footnotes~}%
    \def\RSFootntxt{Footnote~}%
    \def\RSFootnstxt{Footnotes~}%
}
%    \end{macrocode}
%    \end{RSframed}
%    \end{macro}
%
%    \noindent Make English the default.
%    \begin{RSframed}
%    \begin{macrocode}
\RSenglish% Default
%    \end{macrocode}
%    \end{RSframed}
%
%    \noindent Options
%    \begin{RSframed}
%    \begin{macrocode}
\DeclareLangOpt{english}{\RSenglish}
\DeclareLangOpt{USenglish}{\RSenglish}
\DeclareLangOpt{american}{\RSenglish}
\DeclareLangOpt{canadian}{\RSenglish}
\DeclareLangOpt{UKenglish}{\RSukenglish}
\DeclareLangOpt{british}{\RSukenglish}
%    \end{macrocode}
%    \end{RSframed}
%
%
%    \subsubsection{Afrikaans}
%
%    \begin{macro}{\RSafrikaans}
%    Support provided by the author.
%    \begin{RSframed}
%    \begin{macrocode}
%%-- AFRIKAANS ----------------------------------
%    \end{macrocode}
%    \begin{macrocode}
\newcommand\RSafrikaans{%
    \def\RSrngtxt{\space tot~}%
    \def\RSlsttwotxt{\space en~}%
    \def\RSlsttxt{\space en~}%
    \def\RSparttxt{Deel~}%
    \def\RSpartstxt{Dele~}%
    \def\RSParttxt{Deel~}%
    \def\RSPartstxt{Dele~}%
    \def\RSappendixname{bylae~}%
    \def\RSappendicesname{bylaes~}%
    \def\RSAppendixname{Bylae~}%
    \def\RSAppendicesname{Bylaes~}%
    \def\RSchaptername{hoofstuk~}%
    \def\RSchaptersname{hoofstukke~}%
    \def\RSChaptername{Hoofstuk~}%
    \def\RSChaptersname{Hoofstukke~}%
    \def\RSsectxt{afdeling~}%
    \def\RSsecstxt{afdelings~}%
    \def\RSSectxt{Afdeling~}%
    \def\RSSecstxt{Afdelings~}%
    \def\RSeqtxt{vergelyking~}%
    \def\RSeqstxt{vergelykings~}%
    \def\RSEqtxt{Vergelyking~}%
    \def\RSEqstxt{Vergelyking~}%
    \def\RSfigtxt{figuur~}%
    \def\RSfigstxt{figure~}%
    \def\RSFigtxt{Figuur~}%
    \def\RSFigstxt{Figure~}%
    \def\RStabtxt{tabel~}%
    \def\RStabstxt{tabelle~}%
    \def\RSTabtxt{Tabel~}%
    \def\RSTabstxt{Tabelle~}%
    \def\RSfootntxt{footnota~}%
    \def\RSfootnstxt{footnotas~}%
    \def\RSFootntxt{Footnota~}%
    \def\RSFootnstxt{Footnotas~}%
}
%    \end{macrocode}
%    \end{RSframed}
%    \end{macro}
%
%    \noindent Options
%    \begin{RSframed}
%    \begin{macrocode}
\DeclareLangOpt{afrikaans}{\RSafrikaans}
%    \end{macrocode}
%    \end{RSframed}
%
%
%    \subsubsection{Danish}
%
%    \begin{macro}{\RSdanish}
%    \changes{v0.2}{2003/07/30}{Add Danish language support}
%    Thomas Widmann: \texttt{<\,thomas.widmann AT harpercollins DOT co DOT uk>}
%    provided the definitions for Danish.
%    \begin{RSframed}
%    \begin{macrocode}
%%-- DANISH -------------------------------------
%    \end{macrocode}
%    \begin{macrocode}
\newcommand\RSdanish{%
    \def\RSrngtxt{\space til~}%
    \def\RSlsttwotxt{\space og~}%
    \def\RSlsttxt{\space og~}%
    \def\RSparttxt{del~}%
    \def\RSpartstxt{del~}%
    \def\RSParttxt{Del~}%
    \def\RSPartstxt{Del~}%
    \def\RSappendixname{bilag~}%      % 'appendiks' is also possible,
    \def\RSappendicesname{bilag~}%    % but 'bilag' is used in Babel
    \def\RSAppendixname{Bilag~}%
    \def\RSAppendicesname{Bilag~}%
    \def\RSchaptername{kapitel~}%
    \def\RSchaptersname{kapitel~}%
    \def\RSChaptername{Kapitel~}%
    \def\RSChaptersname{Kapitel~}%
    \def\RSsectxt{afsnit~}%
    \def\RSsecstxt{afsnit~}%
    \def\RSSectxt{Afsnit~}%
    \def\RSSecstxt{Afsnit~}%
    \def\RSeqtxt{ligning~}%
    \def\RSeqstxt{ligning~}%
    \def\RSEqtxt{Ligning~}%
    \def\RSEqstxt{Ligning~}%
    \def\RSfigtxt{figur~}%
    \def\RSfigstxt{figur~}%
    \def\RSFigtxt{Figur~}%
    \def\RSFigstxt{Figur~}%
    \def\RStabtxt{tabel~}%
    \def\RStabstxt{tablel~}%
    \def\RSTabtxt{Tabel~}%
    \def\RSTabstxt{Tabel~}%
    \def\RSfootntxt{fodnote~}%
    \def\RSfootnstxt{fodnote~}%
    \def\RSFootntxt{Fodnote~}%
    \def\RSFootnstxt{Fodnote~}%
}
%    \end{macrocode}
%    \end{RSframed}
%    \end{macro}
%
%    \noindent Options
%    \begin{RSframed}
%    \begin{macrocode}
\DeclareLangOpt{danish}{\RSdanish}
%    \end{macrocode}
%    \end{RSframed}
%
%
%    \subsubsection{French}
%
%    \begin{macro}{\RSfrench}
%    \changes{v0.4}{2010/10/21}{Add French language support}
%    Jean-Pierre Chr\'etien \texttt{<\,jeanpierre.chretien AT free DOT fr>}
%    provided the definitions for French.
%    \begin{RSframed}
%    \begin{macrocode}
%%-- FRENCH -------------------------------------
%    \end{macrocode}
%    \begin{macrocode}
\newcommand\RSfrench{%
    \def\RSrngtxt{\space \`{a}~}%
    \def\RSlsttwotxt{\space et~}%
    \def\RSlsttxt{\space et~}%
    \def\RSparttxt{partie~}%
    \def\RSpartstxt{parties~}%
    \def\RSParttxt{La partie~}%
    \def\RSPartstxt{Les parties~}%
    \def\RSappendixname{appendice~}%
    \def\RSappendicesname{appendices~}%
    \def\RSAppendixname{L'appendice~}%
    \def\RSAppendicesname{Les appendices~}%
    \def\RSchaptername{chapitre~}%
    \def\RSchaptersname{chapitres~}%
    \def\RSChaptername{Le chapitre~}%
    \def\RSChaptersname{Les chapitres~}%
    \def\RSsectxt{section~}%
    \def\RSsecstxt{sections~}%
    \def\RSSectxt{La section~}%
    \def\RSSecstxt{Les sections~}%
    \def\RSeqtxt{\'{e}quation~}%
    \def\RSeqstxt{\'{e}quations~}%
    \def\RSEqtxt{L'\'{e}quation~}%
    \def\RSEqstxt{Les \'{e}quations~}%
    \def\RSfigtxt{figure~}%
    \def\RSfigstxt{figures~}%
    \def\RSFigtxt{La figure~}%
    \def\RSFigstxt{Les figures~}%
    \def\RStabtxt{tableau~}%
    \def\RStabstxt{tableaux~}%
    \def\RSTabtxt{Le tableau~}%
    \def\RSTabstxt{Les tableaux~}%
    \def\RSfootntxt{note~}%
    \def\RSfootnstxt{notes~}%
    \def\RSFootntxt{La note~}%
    \def\RSFootnstxt{Les notes~}%
}
%    \end{macrocode}
%    \end{RSframed}
%    \end{macro}
%
%    \noindent Options
%    \begin{RSframed}
%    \begin{macrocode}
\DeclareLangOpt{french}{\RSfrench}
%    \end{macrocode}
%    \end{RSframed}
%
%
%
%    \subsubsection{German}
%
%    \begin{macro}{\RSgerman}
%    \changes{v0.2}{2003/07/30}{Add German language support}
%    Harald Harders: \texttt{<\,h.harders AT tu-bs DOT de>}
%    provided the definitions for German.
%    \begin{RSframed}
%    \begin{macrocode}
%%-- GERMAN -------------------------------------
%    \end{macrocode}
%    \begin{macrocode}
\newcommand\RSgerman{%
    \def\RSrngtxt{\space bis~}%
    \def\RSlsttwotxt{\space und~}%
    \def\RSlsttxt{\space und~}%
    \def\RSparttxt{Teil~}%
    \def\RSpartstxt{Teile~}%
    \def\RSParttxt{Teil~}%
    \def\RSPartstxt{Teile~}%
    \def\RSappendixname{Anhang~}%
    \def\RSappendicesname{Anh\"{a}nge~}%
    \def\RSAppendixname{Anhang~}%
    \def\RSAppendicesname{Anh\"{a}nge~}%
    \def\RSchaptername{Kapitel~}%
    \def\RSchaptersname{Kapitel~}%
    \def\RSChaptername{Kapitel~}%
    \def\RSChaptersname{Kapitel~}%
    \def\RSsectxt{Abschnitt~}%
    \def\RSsecstxt{Abschnitt~}%
    \def\RSSectxt{Abschnitt~}%
    \def\RSSecstxt{Abschnitt~}%
    \def\RSeqtxt{Gleichung~}%
    \def\RSeqstxt{Gleichungen~}%
    \def\RSEqtxt{Gleichung~}%
    \def\RSEqstxt{Gleichungen~}%
    \def\RSfigtxt{Abbildung~}%
    \def\RSfigstxt{Abbildung~}%
    \def\RSFigtxt{Abbildung~}%
    \def\RSFigstxt{Abbildung~}%
    \def\RStabtxt{Tabelle~}%
    \def\RStabstxt{Tabellen~}%
    \def\RSTabtxt{Tabelle~}%
    \def\RSTabstxt{Tabellen~}%
    \def\RSfootntxt{Fu\ss note~}%
    \def\RSfootnstxt{Fu\ss noten~}%
    \def\RSFootntxt{Fu\ss note~}%
    \def\RSFootnstxt{Fu\ss noten~}%
}
%    \end{macrocode}
%    \end{RSframed}
%    \end{macro}
%
%    \noindent Options
%    \begin{RSframed}
%    \begin{macrocode}
\DeclareLangOpt{german}{\RSgerman}
\DeclareLangOpt{ngerman}{\RSgerman}
\DeclareLangOpt{austrian}{\RSgerman}
\DeclareLangOpt{naustrian}{\RSgerman}
%    \end{macrocode}
%    \end{RSframed}
%
%
%    \subsubsection{Italian}
%
%    \begin{macro}{\RSitalian}
%    \changes{v0.4}{2010/10/21}{Add Italian language support}
%    Nicola Lunghi: \texttt{<\,nick83ola AT gmail DOT com>}
%    provided the definitions for Italian.
%    \begin{RSframed}
%    \begin{macrocode}
%%-- ITALIAN ------------------------------------
%    \end{macrocode}
%    \begin{macrocode}
\newcommand\RSitalian{%
    \def\RSrngtxt{--}%
    \def\RSlsttwotxt{\space e~}%
    \def\RSlsttxt{\space e~}%
    \def\RSparttxt{Parte~}%
    \def\RSpartstxt{Parti~}%
    \def\RSParttxt{Parte~}%
    \def\RSPartstxt{Parti~}%
    \def\RSappendixname{l'appendice~}%
    \def\RSappendicesname{le appendici~}%
    \def\RSAppendixname{l'Appendice~}%
    \def\RSAppendicesname{le Appendici~}%
    \def\RSchaptername{il capitolo~}%
    \def\RSchaptersname{i capitoli~}%
    \def\RSChaptername{il Capitolo~}%
    \def\RSChaptersname{i Capitoli~}%
    \def\RSsectxt{la sezione~}%
    \def\RSsecstxt{le sezioni~}%
    \def\RSSectxt{la Sezione~}%
    \def\RSSecstxt{la Sezioni~}%
    \def\RSeqtxt{l'equazione~}%
    \def\RSeqstxt{le equazioni~}%
    \def\RSEqtxt{l'Equazione~}%
    \def\RSEqstxt{le Equazioni~}%
    \def\RSfigtxt{la figura~}%
    \def\RSfigstxt{le figure~}%
    \def\RSFigtxt{la Figura~}%
    \def\RSFigstxt{le Figure~}%
    \def\RStabtxt{la tabella~}%
    \def\RStabstxt{le tabelle~}%
    \def\RSTabtxt{la Tabella~}%
    \def\RSTabstxt{le Tabelle~}%
    \def\RSfootntxt{la nota~}%
    \def\RSfootnstxt{le note~}%
    \def\RSFootntxt{la Nota~}%
    \def\RSFootnstxt{le Note~}%
}
%    \end{macrocode}
%    \end{RSframed}
%    \end{macro}
%
%    \noindent Options
%    \begin{RSframed}
%    \begin{macrocode}
\DeclareLangOpt{italian}{\RSitalian}
%    \end{macrocode}
%    \end{RSframed}
%
%
%
%    \subsubsection{Norwegian}
%
%    \begin{macro}{\RSnorwegian}
%    \changes{v0.3}{2006/09/07}{Add Norwegian language support}
%    Karl Ove Hufthammer: \texttt{<\,karloh AT mi DOT uib DOT no>}
%    provided the definitions for Norwegian.
%
%    Norway has two official languages Norwegian Nynorsk and
%    Norwegian Bokm{\aa}l (commonly 'nynorsk' and 'norsk' in LaTeX),
%    but the \pkg{refstyle} definitions are identically in both variants.
%    \begin{RSframed}
%    \begin{macrocode}
%%-- NORWEGIAN -------------------------------------
%    \end{macrocode}
%    \begin{macrocode}
\newcommand\RSnorwegian{%
    \def\RSrngtxt{\space til~}%
    \def\RSlsttwotxt{\space og~}%
    \def\RSlsttxt{\space og~}%
    \def\RSparttxt{del~}%
    \def\RSpartstxt{del~}%
    \def\RSParttxt{Del~}%
    \def\RSPartstxt{Del~}%
    \def\RSappendixname{tillegg~}%
    \def\RSappendicesname{tillegg~}%
    \def\RSAppendixname{Tillegg~}%
    \def\RSAppendicesname{Tillegg~}%
    \def\RSchaptername{kapittel~}%
    \def\RSchaptersname{kapittel~}%
    \def\RSChaptername{Kapittel~}%
    \def\RSChaptersname{Kapittel~}%
    \def\RSsectxt{avsnitt~}%
    \def\RSsecstxt{avsnitt~}%
    \def\RSSectxt{Avsnitt~}%
    \def\RSSecstxt{Avsnitt~}%
    \def\RSeqtxt{formel~}%
    \def\RSeqstxt{formel~}%
    \def\RSEqtxt{Formel~}%
    \def\RSEqstxt{Formel~}%
    \def\RSfigtxt{figur~}%
    \def\RSfigstxt{figur~}%
    \def\RSFigtxt{Figur~}%
    \def\RSFigstxt{Figur~}%
    \def\RStabtxt{tabell~}%
    \def\RStabstxt{tabell~}%
    \def\RSTabtxt{Tabell~}%
    \def\RSTabstxt{Tabell~}%
    \def\RSfootntxt{fotnote~}%
    \def\RSfootnstxt{fotnote~}%
    \def\RSFootntxt{Fotnote~}%
    \def\RSFootnstxt{Fotnote~}%
}
%    \end{macrocode}
%    \end{RSframed}
%    \end{macro}
%
%    \noindent Options
%    \begin{RSframed}
%    \begin{macrocode}
\DeclareLangOpt{norwegian}{\RSnorwegian}
\DeclareLangOpt{nynorsk}{\RSnorwegian}
\DeclareLangOpt{bokmal}{\RSnorwegian}% Not in babel yet
\DeclareLangOpt{norsk}{\RSnorwegian}
%    \end{macrocode}
%    \end{RSframed}
%
%
%    \subsubsection{Portuguese/Brazilian}
%
%    \begin{macro}{\RSportuguese,\RSbrazilian}
%    \changes{v0.4}{2010/10/21}{Add Portuguese language support}
%    Bernhard Enders and Flavio Costa \texttt{<\,flaviocosta AT yahoo DOT com DOT br>}
%    provided the definitions for Portuguese.
%    \begin{RSframed}
%    \begin{macrocode}
%%-- PORTUGUESE -----------------------------------
%    \end{macrocode}
%    \begin{macrocode}
\newcommand\RSportuguese{%
    \def\RSrngtxt{\space a~}%
    \def\RSlsttwotxt{\space e~}%
    \def\RSlsttxt{\space e~}%
    \def\RSparttxt{Parte~}%
    \def\RSpartstxt{Partes~}%
    \def\RSParttxt{Parte~}%
    \def\RSPartstxt{Partes~}%
    \def\RSappendixname{ap\^{e}ndice~}%
    \def\RSappendicesname{ap\^{e}ndices~}%
    \def\RSAppendixname{Ap\^{e}ndice~}%
    \def\RSAppendicesname{Ap\^{e}ndices~}%
    \def\RSchaptername{cap\'{i}tulo~}%
    \def\RSchaptersname{cap\'{i}tulos~}%
    \def\RSChaptername{Cap\'{i}tulo~}%
    \def\RSChaptersname{Cap\'{i}tulos~}%
    \def\RSsectxt{se\c{c}\~{a}o~}%
    \def\RSsecstxt{se\c{c}\~{o}es~}%
    \def\RSSectxt{Se\c{c}\~{a}o~}%
    \def\RSSecstxt{Se\c{c}\~{o}es~}%
    \def\RSeqtxt{equa\c{c}\~{a}o~}%
    \def\RSeqstxt{equa\c{c}\~{o}es~}%
    \def\RSEqtxt{Equa\c{c}\~{a}o~}%
    \def\RSEqstxt{Equa\c{c}\~{o}es~}%
    \def\RSfigtxt{figura~}%
    \def\RSfigstxt{figuras~}%
    \def\RSFigtxt{Figura~}%
    \def\RSFigstxt{Figuras~}%
    \def\RStabtxt{tabela~}%
    \def\RStabstxt{tabelas~}%
    \def\RSTabtxt{Tabela~}%
    \def\RSTabstxt{Tabelas~}%
    \def\RSfootntxt{nota de rodap\'{e}~}%
    \def\RSfootnstxt{notas de rodap\'{e}~}%
    \def\RSFootntxt{Nota de rodap\'{e}~}%
    \def\RSFootnstxt{Notas de rodap\'{e}~}%
}
%    \end{macrocode}
%    \begin{macrocode}
\newcommand\RSbrazilian{%
    \def\RSrngtxt{\space a~}%
    \def\RSlsttwotxt{\space e~}%
    \def\RSlsttxt{\space e~}%
    \def\RSparttxt{Parte~}%
    \def\RSpartstxt{Partes~}%
    \def\RSParttxt{Parte~}%
    \def\RSPartstxt{Partes~}%
    \def\RSappendixname{ap\^{e}ndice~}%
    \def\RSappendicesname{ap\^{e}ndices~}%
    \def\RSAppendixname{Ap\^{e}ndice~}%
    \def\RSAppendicesname{Ap\^{e}ndices~}%
    \def\RSchaptername{cap\'{i}tulo~}%
    \def\RSchaptersname{cap\'{i}tulos~}%
    \def\RSChaptername{Cap\'{i}tulo~}%
    \def\RSChaptersname{Cap\'{i}tulos~}%
    \def\RSsectxt{sec\c{c}\~{a}o~}%<---------------
    \def\RSsecstxt{sec\c{c}\~{o}es~}%<-------------
    \def\RSSectxt{Sec\c{c}\~{a}o~}%<---------------
    \def\RSSecstxt{Sec\c{c}\~{o}es~}%<-------------
    \def\RSeqtxt{equa\c{c}\~{a}o~}%
    \def\RSeqstxt{equa\c{c}\~{o}es~}%
    \def\RSEqtxt{Equa\c{c}\~{a}o~}%
    \def\RSEqstxt{Equa\c{c}\~{o}es~}%
    \def\RSfigtxt{figura~}%
    \def\RSfigstxt{figuras~}%
    \def\RSFigtxt{Figura~}%
    \def\RSFigstxt{Figuras~}%
    \def\RStabtxt{tabela~}%
    \def\RStabstxt{tabelas~}%
    \def\RSTabtxt{Tabela~}%
    \def\RSTabstxt{Tabelas~}%
    \def\RSfootntxt{nota de rodap\'{e}~}%
    \def\RSfootnstxt{notas de rodap\'{e}~}%
    \def\RSFootntxt{Nota de rodap\'{e}~}%
    \def\RSFootnstxt{Notas de rodap\'{e}~}%
}
%    \end{macrocode}
%    \end{RSframed}
%    \end{macro}
%
%    \noindent Options
%    \begin{RSframed}
%    \begin{macrocode}
\DeclareLangOpt{portuges}{\RSportuguese}
\DeclareLangOpt{portuguese}{\RSportuguese}
\DeclareLangOpt{brazilian}{\RSbrazilian}
\DeclareLangOpt{brazil}{\RSbrazilian}
%    \end{macrocode}
%    \end{RSframed}
%
%
%
%    \subsubsection{Swedish}
%
%    \begin{macro}{\RSswedish}
%    \changes{v0.2}{2003/07/30}{Add Swedish language support}
%     Bj\"{o}rn Thors \texttt{<\,bjorn.thors AT alfvenlab DOT kth DOT se>}
%    provided the definitions for Swedish.
%    \begin{RSframed}
%    \begin{macrocode}
%%-- SWEDISH ------------------------------------
%    \end{macrocode}
%    \begin{macrocode}
\newcommand\RSswedish{%
    \def\RSrngtxt{\space till~}%
    \def\RSlsttwotxt{\space och~}%
    \def\RSlsttxt{\space och~}%
    \def\RSparttxt{del~}%
    \def\RSpartstxt{del~}%
    \def\RSParttxt{Del~}%
    \def\RSPartstxt{Del~}%
    \def\RSappendixname{appendix~}%
    \def\RSappendicesname{appendix~}%
    \def\RSAppendixname{Appendix~}%
    \def\RSAppendicesname{Appendix~}%
    \def\RSchaptername{kapitel~}%
    \def\RSchaptersname{kapitel~}%
    \def\RSChaptername{Kapitel~}%
    \def\RSChaptersname{Kapitel~}%
    \def\RSsectxt{sektion~}%
    \def\RSsecstxt{sektion~}%
    \def\RSSectxt{Sektion~}%
    \def\RSSecstxt{Sektion~}%
    \def\RSeqtxt{ekvation~}%
    \def\RSeqstxt{ekvation~}%
    \def\RSEqtxt{Ekvation~}%
    \def\RSEqstxt{Ekvation~}%
    \def\RSfigtxt{figur~}%
    \def\RSfigstxt{figur~}%
    \def\RSFigtxt{Figur~}%
    \def\RSFigstxt{Figur~}%
    \def\RStabtxt{tabell~}%
    \def\RStabstxt{tabell~}%
    \def\RSTabtxt{Tabell~}%
    \def\RSTabstxt{Tabell~}%
    \def\RSfootntxt{fotnot~}%
    \def\RSfootnstxt{fotnot~}%
    \def\RSFootntxt{Fotnot~}%
    \def\RSFootnstxt{Fotnot~}%
}
%    \end{macrocode}
%    \end{RSframed}
%    \end{macro}
%    \noindent Options
%    \begin{RSframed}
%    \begin{macrocode}
\DeclareLangOpt{swedish}{\RSswedish}
%    \end{macrocode}
%    \end{RSframed}
%
%
%
%
% \subsection{Templates for the \RS Package}
%
%    The user can also define his or her own templates and put it
%    in a separate configuration file to ensure uniformity
%    of reference formats in your documents.
%
% \subsubsection{Parts}
%
%    References to parts are usually straight forward, except that the
%    document division ``Part'' must be distinguished from the normal
%    usage of the word ``part'' in a sentence. A personal preference
%    is to use small caps.
%
%    \begin{RSframed}
%    \begin{macrocode}
%%-- TEMPLATE FOR PARTS -------------------------
%    \end{macrocode}
%    \begin{macrocode}
   \newref{part}{%
      name    = \RSparttxt,
      names   = \RSpartstxt,
      Name    = \RSPparttxt,
      Names   = \RSPartstxt,
      rngtxt  = \RSrngtxt,
      lsttxt  = \RSlsttxt}
%    \end{macrocode}
%    \end{RSframed}
%
%
%
%    \subsubsection{Chapters and Appendices}
%
%    A major problem with a reference to a chapter is that at the time
%    when the reference label, |\label|\marg{chap-label}, is created,
%    it is unknown whether it would eventually ends up in the main
%    matter or the appendix part of a book or report.
%
%    A simple solution is to use the prefix to the \meta{chapter}
%    counter, |\p@chapter|, to write the definition of |\@chapapp| to
%    the auxiliary file (|.aux|) together with information of the
%    label. The command |\@chapapp| expands to either |\chaptername| or
%    |\appendixname| depending on whether the reference label is
%    defined in the main matter or after the |\appendix| was called.
%    (Note that |\@chapapp| is not defined in the AMS book class.)
%
%    \begin{Ipara}
%    |\renewcommand*{\p@chapter}{\string\chname{\@chapapp}}|\\
%    |\newcommand*{\chname}[1]{}%<-make it harmless         |\\
%    |\newcommand*{\chapref}[1]{{\renewcommand*{\chname}[1]{##1~}\ref{#1}}}|
%    \end{Ipara}
%
%    \noindent The reference to a chapter, |\chapref|\marg{chap-label},
%    will then be prefixed with |\chaptername| or |\appendixname| and
%    it does not matter where it was defined or where it was called.
%
%    To utilize the full functionality of the \RS package for
%    references to chapters and appendices, a complex form of
%    \cmd{\newref} needs to be implemented.
%    The following is a template for a book or report classes.
%
%
%    \begin{RSframed}
%    \begin{macrocode}
%%-- TEMPLATE FOR CHAPTERS & APPENDIXES ---------
%    \end{macrocode}
%    Prevent crashes for non-book type classes.
%    \begin{macrocode}
\providecommand*{\p@chapter}{}
%    \end{macrocode}
%    Add \cmd{\@chapapp} to \cmd{\p@chapter} for writeout to auxiliary file.
%    If an AMS book class is loaded, then \cmd{\chaptername} must
%    be used.
%    \begin{macrocode}
\AtBeginDocument{%
  \RS@ifundefined{chapter}{}{%
    \RS@ifundefined{@chapapp}%
      {\renewcommand*{\p@chapter}{\string\chpname{\chaptername}}}%AMS
      {\renewcommand*{\p@chapter}{\string\chpname{\@chapapp}}}%   Normal
    }%
  }
%    \end{macrocode}
%    Define \cmd{\chpname} to gobble its contents outside the
%    reference commands.
%    \begin{macrocode}
\newcommand*{\chpname}[1]{}
%    \end{macrocode}
%    Define the \cmd{\RSchpname} to typeset all the different
%    perturbations of chapter and appendix names.
%    Use the conditionals to switch between the
%    different options.
%
%    \begin{macrocode}
\newcommand*{\RS@chpname}[1]{%
   \ifRSnameon
      \edef\RS@tmpa{#1}%
      \edef\RS@tmpb{\appendixname}%
      \ifx\RS@tmpa\RS@tmpb\relax%
         \ifRSplural
            \ifRScapname \RSAppendicesname \else \RSappendicesname \fi
         \else
            \ifRScapname \RSAppendixname   \else \RSappendixname   \fi
         \fi
      \else
         \ifRSplural
            \ifRScapname \RSChaptersname   \else \RSchaptersname   \fi
         \else
            \ifRScapname \RSChaptername    \else \RSchaptername    \fi
         \fi
      \fi
   \fi}
%    \end{macrocode}
%    Define the \cmd{\newref} for chapters.
%    \begin{macrocode}
   \newref{chap}{%
      refcmd    = {{\let\chpname=\RS@chpname\ref{#1}}},
      rngtxt    = \RSrngtxt,
      lsttwotxt = \RSlsttwotxt,
      lsttxt    = \RSlsttxt}
%    \end{macrocode}
%    \end{RSframed}
%
%
%    \subsubsection{Sections and paragraphs}\seclabel{secTempl}
%
%    A template for references to sections is given below. References
%    to paragraphs are similar and is left as an exercise to the users.
%
%    \begin{RSframed}
%    \begin{macrocode}
%%-- TEMPLATE FOR SECTIONS ----------------------
%    \end{macrocode}
%    \begin{macrocode}
   \newref{sec}{%
      name      = \RSsectxt,
      names     = \RSsecstxt,
      Name      = \RSSectxt,
      Names     = \RSSecstxt,
      refcmd    = {\S\ref{#1}},
      rngtxt    = \RSrngtxt,
      lsttwotxt = \RSlsttwotxt,
      lsttxt    = \RSlsttxt}
%    \end{macrocode}
%    \end{RSframed}
%
%    \subsubsection{Equations}\seclabel{EqTempl}
%
%    The equation number in references to equations are traditionally
%    written in an upright text, irrespective of the surrounding text.
%
%    \begin{RSframed}
%    \begin{macrocode}
%%-- TEMPLATE FOR EQUATIONS ---------------------
%    \end{macrocode}
%    The standard equation ref format if \pkg{amsmath.sty} is not loaded.
%    \begin{macrocode}
\newcommand*{\RSeqrefform}[1]{\textup{(\ref{#1})}}
%    \end{macrocode}
%    If \pkg{amsmath.sty} is loaded, store \cmd{\eqref}
%    and then undefine it before the template is created.
%    \begin{macrocode}
\@ifpackageloaded{amsmath}%
   {\let\AMSeqref\eqref
    \let\eqref\relax}%
   {}
%    \end{macrocode}
%    \begin{macrocode}
\newref{eq}{%
   name      = \RSeqtxt,
   names     = \RSeqstxt,
   Name      = \RSEqtxt,
   Names     = \RSEqstxt,
   refcmd    = \RSeqrefform{#1},
   rngtxt    = \RSrngtxt,
   lsttwotxt = \RSlsttwotxt,
   lsttxt    = \RSlsttxt}
%    \end{macrocode}
%    Make way for \pkg{amsmath.sty} definitions
%    \begin{macrocode}
\let\RSeqref\eqref
\let\eqref\relax
%    \end{macrocode}
%    Some footwork to bring the AMS definition of
%    \cmd{\eqref} back if \pkg{amsmath.sty} is loaded
%    afterwards.
%    \begin{macrocode}
\AtBeginDocument{%
   \@ifpackageloaded{amsmath}%
      {\RS@ifundefined{AMSeqref}{\let\AMSeqref\eqref}{}%
       \let\RSeqrefform\AMSeqref}%
         {}%
     \let\eqref=\RSeqref
   }
%    \end{macrocode}
%    \end{RSframed}
%
%
% \subsubsection{Figures and Tables}\seclabel{tabtemp}
%    References to figures and tables are usually
%    straight forward.
%    \begin{RSframed}
%    \begin{macrocode}
%%-- TEMPLATE FOR FIGURES -----------------------
%    \end{macrocode}
%    \begin{macrocode}
   \newref{fig}{%
      name      = \RSfigtxt,
      names     = \RSfigstxt,
      Name      = \RSFigtxt,
      Names     = \RSFigstxt,
      rngtxt    = \RSrngtxt,
      lsttwotxt = \RSlsttwotxt,
      lsttxt    = \RSlsttxt}
%    \end{macrocode}
%    \end{RSframed}
%
%    \begin{RSframed}
%    \begin{macrocode}
%%-- TEMPLATE FOR TABLES ------------------------
%    \end{macrocode}
%    \begin{macrocode}
   \newref{tab}{%
      name      = \RStabtxt,
      names     = \RStabstxt,
      Name      = \RSTabtxt,
      Names     = \RSTabstxt,
      rngtxt    = \RSrngtxt,
      lsttwotxt = \RSlsttwotxt,
      lsttxt    = \RSlsttxt}
%    \end{macrocode}
%    \end{RSframed}
%
%    \pagebreak[2]
%    \subsubsection{Footnotes}\seclabel{fnTempl}
%
%    A reference to a footnote differs from other references in that
%    it is only defined inside the footnote definition itself:
%    \begin{Ipara}
%       |\footnote{This is a footnote with ...  \label{fn:xx}}|
%    \end{Ipara}
%    It can then be referred to with |\ref{fn:xx}|.
%
%    A useful application of the \RS package is for references to
%    footnotes, were you need a duplicate footnote mark that refers to
%    a previously defined footnote.\footnote{^^A
%         This is a footnote with \cmd{\fnlabel}\texttt{\{xx\}} ...\fnlabel{xx}}
%    You can use the stared form of the reference
%    command to format the reference as a footnote mark, while the
%    reference commands without a star behave as normal.
%    \begin{Itabb}
%    \hspace*{2cm}\=\kill
%       |\fnref*{xx}| \ArTab  \fnref*{xx}\\
%       |\fnref{xx}|  \ArTab  \fnref{xx}\\
%    \end{Itabb}
%
%    \begin{RSframed}
%    \begin{macrocode}
%%-- TEMPLATE FOR FOOTNOTES ---------------------
%    \end{macrocode}
%    Define the \cmd{\RSfnmark} to reproduce the footnote mark.
%    Use the |\ifRSstar| conditional to switch between a superscripted
%    and a normal reference.
%    \begin{macrocode}
   \newcommand{\RSfnmark}[1]{%
      \begingroup
        \unrestored@protected@xdef\@thefnmark{\ref{#1}}%
      \endgroup
      \@footnotemark}
%    \end{macrocode}
%    \begin{macrocode}
   \newref{fn}{%
      name      = \RSfootntxt,
      names     = \RSfootnstxt,
      Name      = \RSFootntxt,
      Names     = \RSFootnstxt,
      refcmd    = {\ifRSstar\RSfnmark{#1}\else(\ref{#1})\fi},
      rngtxt    = \RSrngtxt,
      lsttwotxt = \RSlsttwotxt,
      lsttxt    = \RSlsttxt}
%    \end{macrocode}
%    \end{RSframed}
%
%    \subsubsection{Enumerated lists}
%
%    A reference to an item in an enumerated list can be obtained
%    by placing the labelling command after the \cmd{\item} command
%    inside the list. The second level numbering of the \cmd{\ref}
%    label in the standard \LaTeX{} is: 2a, 2b, etc.
%
%    If you want to change the reference labels to 2(a), 2(b),
%    etc., without effecting the display in the \env{enumerate}
%    environment, you can make the following redefinition:
%    \begin{Ipara}
%        |\makeatletter|\\
%        |\renewcommand{\p@enumii}{\expandafter\p@@enumii}|\\
%        |\newcommand{\p@@enumii}[1]{\theenumi(#1)}|\\
%        |\makeatother|
%    \end{Ipara}
%    The contents of the \cmd{\@currentlable} is then
%    \begin{Ipara}
%    |{\expandafter\p@@enumii\theenumii}| ~$\rightarrow$~
%        |{\theenumi(\theenumii)}|
%    \end{Ipara}
%
%    Writing a template for enumerated lists is left to the user.
%
%
%    \subsubsection{Theorems, lemmas, etc.}\seclabel{Thm}
%
%    There exists many perturbations to the theorem environment such
%    as, Theorem, Lemma, Exercise, etc., and it is left to the user as
%    an exercise to construct his or her own templates.
%
%    As an example to add reference for theorems to an exiting setup:
%
%    \begin{Ipara}
%    |\usepackage{refstyle}  |\\
%    |\newref{thm}{          |\\
%    |   name      = {theorem~}, |\\
%    |   names     = {theorems~},|\\
%    |   Name      = {Theorem~}, |\\
%    |   Names     = {Theorems~},|\\
%    |   rngtxt    = \RSrngtxt,|\\
%    |   lsttwotxt = \RSlsttxt,|\\
%    |   lsttxt    = \RSlsttxt}|
%    \end{Ipara}
%
%    If you want to add it to a specific language setup with \pkg{babel} then
%    you can do the following, for example for a dual language document with
%    English and Norwegian:
%    \begin{Ipara}
%    |\documentclass[english, norwegian]{article}% NB: Languages as global opts!  |\\
%    |\usepackage{babel}     %Use global languages           |\\
%    |\usepackage{varioref}  %Use global languages           |\\
%    |\usepackage{refstyle}  %Use global languages           |\\
%    |\RSaddto{\RSnorwegian}{%          |\\
%    |   \def\RSthmtxt{teorem~}%        |\\
%    |   \def\RSthmstxt{teorem~}%       |\\
%    |   \def\RSThmtxt{Teorem~}%        |\\
%    |   \def\RSThmstxt{Teorem~}}       |\\
%    |\RSaddto{\RSenglish}{%            |\\
%    |   \def\RSthmtxt{theorem~}%       |\\
%    |   \def\RSthmstxt{theorems~}%     |\\
%    |   \def\RSThmtxt{Theorem~}%       |\\
%    |   \def\RSThmstxt{Theorems~}}     |\\
%    |\newref{thm}{                     |\\
%    |   name      = \RSthmtxt,         |\\
%    |   names     = \RSthmstxt,        |\\
%    |   Name      = \RSThmtxt,         |\\
%    |   Names     = \RSThmstxt,        |\\
%    |   rngtxt    = \RSrngtxt,         |\\
%    |   lsttwotxt = \RSlsttxt,         |\\
%    |   lsttxt    = \RSlsttwotxt}      |
%    \end{Ipara}
%
%    \noindent All \pkg{babel} language changes will then activate
%    the proper code setup. In this example the \pkg{refstyle}
%    mechanism is used where the code is appended to the |\RSenglish|
%    and  |\RSnorwegian| container functions. These functions are added
%    by the package
%    to the \pkg{babel} containers |\extrasenglish| and |\extrasnorwegian|,
%    that is called every time a language change occurs.
%    Please see the contents of \pkg{refstyle.cfg} for the names of the
%    \pkg{refstyle} containers associated with the different languages.
%
%    Note that the extended definitions
%    can also be added directly to the \pkg{babel} containers without using
%    \pkg{refstyle}'s mechanism.
%
%    \begin{macrocode}
%</cfg>
%    \end{macrocode}
%
% \Finale
\endinput
