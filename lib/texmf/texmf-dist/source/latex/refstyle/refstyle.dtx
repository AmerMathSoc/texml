% \iffalse
%
% refstyle.dtx
% Copyright (C) 2002--2010 Danie Els
%
% -------------------------------------------------------------------
%                     The refstyle package
%                 for the formatting of references
% -------------------------------------------------------------------
% This work may be distributed and/or modified under the conditions
% of the LaTeX Project Public License, either version 1.3c of this
% license or (at your option) any later version. The latest version
% of this license is in
%      http://www.latex-project.org/lppl.txt
% and version 1.3c or later is part of all distributions of LaTeX
% version 2005/12/01 or later.
%
% This work is "maintained" (as per LPPL maintenance status)
% by Danie Els (dnjels@sun.ac.za).
%
% This package consists of the files: refstyle.dtx
%                                     refconfig.dtx
%                                     refstyle.ins
%              and the derived files: refstyle.sty
%                                     refstyle.cfg
% -------------------------------------------------------------------
%
%<*pkg>
\newcommand*{\RefstyleFileDate}{2010/11/02}
\newcommand*{\RefstyleFileVersion}{v0.5}
%</pkg>
%
%
%<*driver>
\documentclass[a4paper]{ltxdoc}
\let\RefstyleFileDate\relax
\let\RefstyleFileVersion\relax
\usepackage{amsmath}
\usepackage{amsfonts}
   \DeclareMathSymbol{\square}     {\mathord}{AMSa}{"03}
   \DeclareMathSymbol{\blacksquare}{\mathord}{AMSa}{"04}
\usepackage{calc}
\usepackage{rotating}
\usepackage[rounded]{syntax}
\usepackage{varioref}
\usepackage{refstyle}
\usepackage{color}
    \definecolor{hrefcol}{rgb}{0.00,0.00,1.00}
\usepackage[pdfpagelabels]{hyperref}
\hypersetup{breaklinks=true,
            linktocpage,
            pdfstartview=FitH,
            colorlinks        = true,
            linkcolor         = hrefcol,
            urlcolor          = hrefcol,
            menucolor         = hrefcol,
            filecolor         = hrefcol,
            citecolor         = hrefcol,
            linkbordercolor = {0.8 0.8 0.8},
            urlbordercolor  = {0.8 0.8 0.8},
            menubordercolor = {0.8 0.8 0.8},
            filebordercolor = {0.8 0.8 0.8},
            runbordercolor  = {0.8 0.8 0.8},
            citebordercolor = {0.8 0.8 0.8},
            pdftitle={refstyle},
            pdfauthor={Danie Els}}

%---- macrocode env changes -------------------------------

\setlength{\MacroIndent}{1.5em}

%---- Scratches -------------------------------------------

\newlength{\tdima}
\newsavebox{\tboxa}

%---- Spacing ---------------------------------------------

\newlength{\myskip}
\setlength{\myskip}{\topsep}
    \addtolength{\myskip}{\parskip}
    \addtolength{\myskip}{\partopsep}
\newcommand*{\blankline}{\par\addvspace{\myskip}}
\newlength{\mytab}
\setlength{\mytab}{2\parindent}
\newcommand{\tab}{\hspace*{\mytab}}

%---- Indented environments -------------------------------

\newenvironment{IndentPara}[1][1]
    {\list{}{\setlength{\leftmargin}{#1\mytab}%
             \setlength{\labelwidth}{0pt}%
             \setlength{\labelsep}{0pt}%
             \setlength{\itemindent}{\parindent}%
             \setlength{\listparindent}{\parindent}%
             }\item[]}%
    {\endlist}
\newenvironment{IndentParaX}[1][1]
    {\list{}{\setlength{\leftmargin}{#1\mytab}%
             \setlength{\topsep}{0pt}%
             \setlength{\partopsep}{0pt}%
             \setlength{\labelwidth}{0pt}%
             \setlength{\labelsep}{0pt}%
             \setlength{\itemindent}{\parindent}%
             \setlength{\listparindent}{\parindent}%
             }\item[]}%
   {\endlist}
\newcommand\bNP{\begin{IndentParaX}}
\newcommand\eNP{\end{IndentParaX}}
\newenvironment{Ipara}[1][\small]%
    {\begin{IndentPara}\noindent#1\ignorespaces}%
    {\end{IndentPara}}
\newenvironment{Itabb}[1][\small]%
    {\begin{IndentPara}#1\ignorespaces\begin{tabbing}\ignorespaces}%
    {\end{tabbing}\end{IndentPara}}

%---- Ruled & colored minipage ----------------------------

\definecolor{shadecolor}{gray}{0.9}

\newenvironment{fminipage}[1][\linewidth]
    {\setlength{\tdima}{#1-2\fboxsep-2\fboxrule}%
     \begin{lrbox}{\tboxa}\begin{minipage}{\tdima}}%
    {\end{minipage}\end{lrbox}\noindent\fbox{\usebox{\tboxa}}}
\newenvironment{cminipage}[1][\linewidth]
    {\setlength{\tdima}{#1-2\fboxsep-2\fboxrule}%
     \begin{lrbox}{\tboxa}\begin{minipage}{\tdima}}%
    {\end{minipage}\end{lrbox}%
     \noindent\colorbox{shadecolor}{\usebox{\tboxa}}}

%---- Additional documenting commands ---------------------

\def\lcbrack{\symbol{`\{}}
\def\rcbrack{\symbol{`\}}}
\makeatletter
\def\meta@font@select{\normalfont\slshape}
\makeatother
\DeclareRobustCommand{\cmeta}[1]{%<- \<xxx>
    \texttt{\bslash}\meta{#1}}
\DeclareRobustCommand{\cmarg}[1]{%<- {\<xxx>}
    \texttt{\lcbrack}\cmeta{#1}\texttt{\rcbrack}}
\DeclareRobustCommand{\coarg}[1]{%<- [\<xxx>]
    \texttt{[}\cmeta{#1}\texttt{]}}
\DeclareRobustCommand{\argcmd}[2]{%<- \<xxx>yyy
    \cmeta{#1}\texttt{#2}}
\newcommand{\rmeta}[3][]{%
    \texttt{\lcbrack#1}\meta{#2}\texttt{#3\rcbrack}}
\newcommand{\refmeta}[3][]{\cmd{\brref}\rmeta[#1]{#2}{#3}}

\newcommand*{\file}[1]{\texttt{#1}}
\newcommand*{\pkg}[1]{\textsf{#1}}
\newcommand*{\env}[1]{\texttt{#1}}
\newcommand*{\opt}[1]{\texorpdfstring{\normalfont\texttt{\slshape#1}}{#1}}
\newcommand*{\RS}{\pkg{refstyle}}
\newcommand*{\RScfg}{\file{refstyle.cfg}}
\newcommand*{\rtp}{type}
\newcommand*{\Rtp}{Type}
\newcommand*{\okeylst}{\oarg{key\_lst}}
\newcommand*{\mkeylst}{\marg{key\_lst}}

%---- Special eq tag command ------------------------------

\def\spestag#1{\stepcounter{equation}%
    \gdef\tagextra##1{%
        \makebox[0pt][l]{\small\quad%
            \cmd{\eqlabel}\texttt{\lcbrack##1\rcbrack}}%
        \gdef\tagextra####1{}}%
    \tag{\theequation\tagextra{#1}}\eqlabel{#1}}

%---- other stuff -----------------------------------------

\newcommand*{\ArTab}{\>\quad$\rightarrow$\quad}

%----------------------------------------------------------

\EnableCrossrefs
\CodelineIndex
\RecordChanges
\setlength\hfuzz{15pt}
\hbadness=7000
\begin{document}
  \DocInput{refstyle.dtx}
\end{document}

%</driver>
% \fi
%
% \CheckSum{552}
%
% \CharacterTable
%  {Upper-case    \A\B\C\D\E\F\G\H\I\J\K\L\M\N\O\P\Q\R\S\T\U\V\W\X\Y\Z
%   Lower-case    \a\b\c\d\e\f\g\h\i\j\k\l\m\n\o\p\q\r\s\t\u\v\w\x\y\z
%   Digits        \0\1\2\3\4\5\6\7\8\9
%   Exclamation   \!     Double quote  \"     Hash (number) \#
%   Dollar        \$     Percent       \%     Ampersand     \&
%   Acute accent  \'     Left paren    \(     Right paren   \)
%   Asterisk      \*     Plus          \+     Comma         \,
%   Minus         \-     Point         \.     Solidus       \/
%   Colon         \:     Semicolon     \;     Less than     \<
%   Equals        \=     Greater than  \>     Question mark \?
%   Commercial at \@     Left bracket  \[     Backslash     \\
%   Right bracket \]     Circumflex    \^     Underscore    \_
%   Grave accent  \`     Left brace    \{     Vertical bar  \|
%   Right brace   \}     Tilde         \~}
%
% \changes{v0.0}{2002/12/01}{Initial version}
% \changes{v0.1}{2003/04/01}{First stable version}
% \changes{v0.2}{2003/07/30}{First updated version}
% \changes{v0.3}{2006/09/09}{Documentation update}
% \changes{v0.4}{2010/10/21}{Documentation update}
% \changes{v0.4}{2010/10/21}{Add \texttt{nokeyprefix} option}
% \changes{v0.5}{2010/11/02}{Documentation update}
%
%   \DoNotIndex{
% \@backslashchar, \@carcube, \@ehc, \@empty, \@firstoftwo,
% \@ifnextchar, \@ifpackageloaded, \@ifstar, \@ifundefined, \@nil,
% \@qend, \@qrelax, \@secondoftwo, \@tempa, \@undefined,
% \AtEndOfPackage, \begingroup,  \csname, \CurrentOption,
% \DeclareOption, \def, \edef, \else, \endcsname, \endgroup,
% \expandafter, \fi,  \gdef, \iffalse, \ifx,  \InputIfFileExists,
% \let, \long, \lowercase, \MakeLowercase, \mbox,  \MessageBreak,
% \NeedsTeXFormat, \newcommand, \newif, \noexpand, \PackageError, \PackageInfo,
% \PassOptionsToPackage, \RS@pkgname,  \ProcessOptions, \protect,
% \providecommand, \ProvidesPackage, \relax, \renewcommand,
% \RequirePackage, \reserved@a,  \reserved@b,  \RS@tmpa,
% \RS@tmpb,  \S, \RS@setkeys, \space,  \string, \textsc, \textup,
% \unrestored@protected@xdef, \unskip,  \uppercase,
% \zap@space}
%
%^^A==== Titling ==========================================
% \GetFileInfo{refstyle.sty}
% \title{\vspace*{-1cm}
%        The \RS\ package\thanks{This file has version number
%                                \fileversion, last revised
%                                \filedate.}}
% \author{Danie Els\\[1ex]
%         \small Department of Mechanical and Mechatronics Engineering\\
%         \small University of Stellenbosch, South Africa.\\
%    \normalsize e-mail: \texttt{\href{mailto:dnjels@sun.ac.za}%
%                                          {dnjels@sun.ac.za}}}
% \date{\filedate}
% \maketitle
% \begin{center}
% \textbf{Overview of the \RS\ package}
% \end{center}
%    \noindent
%    When writing complex documents, often a large number of commands
%    for different type of references are defined, for example:
% \begin{Ipara}
%    |\newcommand*{\eqref}[1]{eqn~(\ref{#1})}|\\
%    |\newcommand*{\Eqref}[1]{Equation~(\ref{#1})}|
% \end{Ipara}
%    The \RS\ package was developed to automate this process. The
%    package provides a user interface to define sets of reference
%    and label commands for each referable object such as an equation
%    or a table, etc. When you declare, for example, a set of
%    reference commands to an equation:
% \begin{Ipara}
%    \cmd{\newref}|{eq}|\mkeylst\
% \end{Ipara}
%    \noindent
%    a series of commands of the format \cmd{\eq...} and \cmd{\Eq...}
%    are produced. The configuration options are set with a list of
%    key-values. Prefixes, inserts and other options for all the
%    different perturbations such as capitalized first letters,
%    singular and plural from, etc.\ can be defined.  The
%    configuration can be changed temporarily with an optional list
%    of key-values when the commands are used.  A direct interface to
%    the \pkg{varioref} package is also provided.
%    This enables compact reference formats:
% \begin{Itabb}
%       \hspace{14em}\=\kill
%       |\eqref{e1} ...|            \ArTab\eqref{e1} ...\\
%       |\Eqref[vref]{e1} ...|      \ArTab\Eqref[vref]{e1} ...\\
%       |\eqref[s]{e1} and ...|     \ArTab\eqref[s]{e1} and ...\\
%       |\eqref[name=eq.~]{e1} ...| \ArTab\eqref[name=eq.~]{e1} ...
% \end{Itabb}
%    A range or a list of references can also be referred to in a
%    consistent way.
% \begin{Itabb}
%       \hspace{14em}\=\kill
%       |\eqref{e1,e2,e3} ...|          \ArTab\eqref{e1,e2,e3} ...\\
%       |\eqrangeref[vref]{e1}{e3} ...| \ArTab\eqrangeref[vref]{e1}{e3} ...
% \end{Itabb}
%
%    Templates for the different reference types and different
%    languages can be loaded with a configuring file.
%
%    The package is aimed at large projects, enabling a consistent
%    way of producing references throughout a project. Enough
%    flexibility is provided to make local changes to a single
%    reference.  For large projects such as a series of books or a
%    multi volume thesis, written as freestanding documents, a
%    facility is provided to interface to the \pkg{xr} package for
%    external document references.
%
%    \medskip
%
%    \begin{center}
%    \begin{fminipage}[.6\textwidth]
%    See also \pkg{refconfig.pdf} for setup and examples.
%    \end{fminipage}
%    \end{center}

% \clearpage
%
% \tableofcontents
% \clearpage
%
%
% \section{Loading the \RS\ Package}
%
% \subsection{Document preamble}
%
%    The \RS\ package  is loaded in the preamble of the document:
%    \begin{enumerate}
%    \item[(a)] With a config file with \pkg{babel} language support
%           \begin{Ipara}
%              |\usepackage{varioref}[2001/09/04]%|~
%                   ${\leftarrow}$ To use the \opt{vref} option\\
%              |\usepackage{refstyle}|
%           \end{Ipara}
%           When the package is loaded, it first searchers for a local user defined
%           configuration file \file{refstyle.def}. If it is not available it
%           looks for the global default config file \RScfg\ provided with this package.
%           These configuration files are loaded before the package options are precess.
%           It can therefore contain option commands using globally defined language
%           options and can interface to \pkg{babel} for language changes. The macro
%           \begin{Ipara}
%           |\DeclareLangOpt{|\meta{language}|}{|\meta{definitions}|}|
%           \end{Ipara}
%           is provided to declare the package option \meta{language} and add
%           \meta{definitions} to the |\extras|\meta{language} token for \pkg{babel}.
%           It can only be used inside \RScfg\ or \file{refstyle.def}.
%
%    \item[(b)]
%         Or with an existing configuration with language support and addition
%         of your own extensions
%    \begin{Ipara}
%    |\documentclass[norwegian]{article}|\\
%    |\usepackage{babel}                |\\
%    |\usepackage{refstyle}             |\\
%    |\RSaddto{\RSnorwegian}{%          |\\
%    |   \def\RSthmtxt{teorem~}%        |\\
%    |   \def\RSthmstxt{teorem~}%       |\\
%    |   \def\RSThmtxt{Teorem~}%        |\\
%    |   \def\RSThmstxt{Teorem~}}       |\\
%    |\newref{thm}{                     |\\
%    |   name      = \RSthmtxt,         |\\
%    |   names     = \RSthmstxt,        |\\
%    |   Name      = \RSThmtxt,         |\\
%    |   Names     = \RSThmstxt,        |\\
%    |   rngtxt    = \RSrngtxt,         |\\
%    |   lsttxt    = \RSlsttxt,         |\\
%    |   lsttwotxt = \RSlsttwotxt}      |
%    \end{Ipara}
%
%    \item[(c)]
%          Or without any configuration file, but by declaring your own
%          reference commands in the preamble.
%           \begin{Itabb}
%              \cmd{\usepackage}|[noconfig]{refstyle}|\\
%              \qquad\cmd{\newref}\marg{reftype$_{\,1}$}\mkeylst \\
%              \qquad\cmd{\newref}\marg{reftype$_{\,2}$}\mkeylst\\
%              \tab$\vdots$
%           \end{Itabb}
%
%     \item[(d)]
%           Or with your own configuration file (without babel language support) for a specific project:
%           \begin{Itabb}
%              |\usepackage[noconfig]{refstyle}|\\
%              |\input{thisproject.ref}|
%           \end{Itabb}
%
%    \end{enumerate}
%
% \subsection{Package options}
%
% \begin{description}
%    \item[\opt{noconfig}] Neither of the user supplied \file{refstyle.def} config
%          file or \file{refstyle.cfg} the config file supplied with this package
%          are loaded. The user must declare his or her own definitions and templates.
%          Note that these definitions are not passed on to \pkg{babel}'s language
%          changing mechanisms.
%
%    \item[\opt{nokeyprefix}] Custom labeling commands \argcmd{\rtp}{label} are generated
%          by \pkg{refstyle} for every \meta{type} definition. It defaults to
%          \begin{Itabb}
%          \hspace{5em}\=hspace{5em}\=\kill
%          \argcmd{\rtp}{label}|{abc}|
%             \>\>~ $\rightarrow$ \cmd{\label}\rmeta{type\,:\,}{abc}
%          \end{Itabb}
%          With the \opt{nokeyprefix} option, the prefix is turned of and the
%          labeling commands defaults to
%          \begin{Itabb}
%          \hspace{5em}\=hspace{5em}\=\kill
%          \argcmd{\rtp}{label}|{abc}|
%             \>\>~ $\rightarrow$ \cmd{\label}|{abc}|
%          \end{Itabb}
%          This is useful for old documents with existing labeling commands or
%          where the user prefer not to use the  \argcmd{\rtp}{label} commands.
%
%
% \end{description}
%
%
%
%
% \subsection{Companion packages}
%
%    The \RS\ packages is intended for large projects. It is therefore
%    important that it works together with, or has direct interfaces
%    to the following packages:
% \begin{description}
%    \item[\pkg{varioref}:\footnotemark]
%         \footnotetext{\pkg{varioref} v1.3c, 2001/09/04 or later,
%                      because the starred versions of the commands
%                      are used.}
%         Produce sophisticated page and page range references.
%
%    \item[\pkg{hyperref}:\footnotemark]
%         \footnotetext{\pkg{hyperref} v6.72r, 2002/05/27 or later,
%                       where a bug for interference with
%                       \pkg{varioref} was fixed.}
%         To establish hyper links between the references and the
%         labels.
%
%    \item[\pkg{xr, xr-hyper}:]
%         To establish references to external documents.
%
%    \item[\pkg{showkeys}:]
%         To show all the labels and references. This is very useful
%         to find labels in large documents.
% \end{description}
%
% \clearpage
% \section{User Interface}
%
%    The \RS\ package has one configuring command, \DescribeMacro{\newref}
%    \cmd{\newref}, that internally creates a series of label and
%    reference commands:
%    \blankline
%
% \begin{cminipage}[\linewidth]
%    \newcommand*{\crsi}{\smash{\rule[0pt]{.4pt}{1.25em}\rule[.5ex]{1.5em}{.4pt}}}^^A
%    \newcommand*{\crsl}{\smash{\rule[.5ex]{.4pt}{1.5em}\rule[.5ex]{1.5em}{.4pt}}}^^A
%    \begin{Itabb}
%     \hspace*{2em}\=\hspace{2em}\=\hspace{10em}\=\kill
%     \cmd{\newref}\marg{\rtp}\mkeylst\tab\mbox{}\\[.75ex]
%     \>\crsi \>\argcmd{{\rtp}}{key}       \> Reference argument prefix      \\[.75ex]
%     \>\crsl \>\argcmd{{\rtp}}{label}     \> Custom label                   \\[.75ex]
%     \>\crsl \>\argcmd{{\rtp}}{ref}       \> In text reference              \\[.75ex]
%     \>\crsl \>\argcmd{{\Rtp}}{ref}       \> Start of sentence reference    \\[.75ex]
%     \>\crsl \>\argcmd{{\rtp}}{rangeref}  \> In text range reference        \\[.75ex]
%     \>\crsl \>\argcmd{{\Rtp}}{rangeref}  \> Start of sentence range reference\quad\mbox{}\\[.75ex]
%     \>\crsl \>\argcmd{{\rtp}}{pageref}   \> Page reference
%    \end{Itabb}
% \end{cminipage}
%    \blankline
%
%    \noindent
%    All the \cmeta{\rtp}... commands, excluding \argcmd{{\rtp}}{key}
%    and \argcmd{{\rtp}}{label} are robust.
%    All the options for the referencing commands are set with a
%    key-value list. \Tabref[vref=far]{refopt} gives a full list of all the
%    key-values and defaults.
%
%    The \RS\ package do not redefine any internal \LaTeX\ commands and
%    depends only on the \cmd{\label}, \cmd{\ref}, \cmd{\pageref} and
%    the \pkg{varioref} commands.
%    The internally defined commands do not overwrite any
%    existing command with the same name, and an error results if a command
%    already exists. The exception is commands declared with a previous
%    \cmd{\newref} call, can be redefined by calling \cmd{\newref} again
%    with a new set of parameter.
%    If the \pkg{amsmath} package is loaded, and you define
%    \cmd{\newref}|{eq}| for references to equations, you need to
%    undefine \cmd{\eqref} before issuing \cmd{\newref} by
%  \begin{Ipara}
%    \cmd{\let}\cmd{\eqref}|=|\cmd{\relax}
%  \end{Ipara}
%
%
%    \section{Command Descriptions}
%
%    The structure of the label and reference commands is given by the
%    syntax diagrams that follows. Examples are included for references
%    to equations, defined according to the template in
%    \RScfg\  with the
%    \cmd{\newref}|{eq}|\mkeylst. See  also \secref{keyvals} for
%    explanations of the key-values.
%
%    \subsection{The reference key}
%
%    \begin{cminipage}[\linewidth]
%    \begin{syntdiag}
%    \tok{\argcmd{\rtp}{key}}
%    \end{syntdiag}
%    \end{cminipage}
%    \blankline
%    \noindent
%    The \argcmd{\rtp}{key} command returns the prefix added to the
%    argument of the label and the reference commands, for example:
%
%          \begin{Itabb}
%          \hspace{10em}\=hspace{5em}\=\kill
%          \cmd{\newref}\marg{\rtp}|{}|
%              \>\>\argcmd{\rtp}{key} $\rightarrow$ \meta{type\,:\,}\\
%          \cmd{\newref}\marg{\rtp}|{key=xxx-}|
%              \>\>\argcmd{\rtp}{key} $\rightarrow$ |xxx-|
%          \end{Itabb}
%
%    \noindent
%    The \argcmd{\rtp}{key} command is not a general command, but was
%    provided only as a link to the standard \LaTeX{} \cmd{\ref} and
%    \cmd{\pageref} commands:
%
%          \begin{Itabb}
%          \hspace{10em}\=hspace{5em}\=\kill
%          \cmd{\label}|{|\argcmd{\rtp}{key}| abc}|
%             \>\>~ $\rightarrow$ \cmd{\label}\rmeta{type\,:\,}{abc}\\
%          \cmd{\ref}|{|\argcmd{\rtp}{key}| abc}|
%             \>\>~ $\rightarrow$ \cmd{\ref}\rmeta{type\,:\,}{abc}
%          \end{Itabb}
%    Note if the the \opt{nokeyprefix} option is active then the \argcmd{\rtp}{key}
%    is empty.
%
%    \blankline
%     \noindent
%    \textbf{\slshape Examples:}  equations
%    with \cmd{\newref}|{eq}|\mkeylst:
%
%          \begin{Itabb}
%          \hspace*{18em}\=\kill
%            |\eqkey|                         \> \eqkey                         \\
%            |in eq.~\ref{\eqkey e1} ...|     \>in eq.~\ref{\eqkey e1} ...      \\
%            |on page \pageref{\eqkey e1} ...|\>on page \pageref{\eqkey e1} ... \\
%            |\vpageref*{\eqkey e1} ... |     \>\vpageref*{\eqkey e1} ...
%          \end{Itabb}
%
%    \subsection{Reference label}
%
%    \begin{cminipage}[\linewidth]
%    \begin{syntdiag}
%    \tok{\argcmd{\rtp}{label}"{"<lbl>"}"}
%    \end{syntdiag}
%    \end{cminipage}
%    \blankline
%
%    \noindent
%    The command \argcmd{\rtp}{label} prefix the reference
%    string in the \cmd{\label} with the string \meta{type\,:\,}, or its
%    redefinition with the \opt{key} option.
%
%          \begin{Itabb}
%          \hspace{10em}\=hspace{5em}\=\kill
%          \argcmd{\rtp}{label}|{abc}|
%             \>\>~ $\rightarrow$ \cmd{\label}\rmeta{type\,:\,}{abc}
%          \end{Itabb}
%         \blankline
%
%     \noindent
%     \textbf{\slshape Examples:}
%
% \begingroup
%           \begin{Ipara}
%           Let $z = x+\mathrm{i} y$ and $\alpha = \beta + \mathrm{i}
%           \gamma$, with $\mathrm{i}^2 = -1$, then
%        \begin{align}
%          &\mathrm{e}^z = \mathrm{e}^x \left(\cos y + \mathrm{i} \sin y \right)
%          \spestag{e1}\\
%          &z^\alpha = \mathrm{e}^{\alpha \ln z}
%          \spestag{e2}\\
%          \intertext{\Eqref{e1,e2} lead to the  following interesting results:}
%          &\mathrm{e}^{\mathrm{i} \pi} + 1 = 0 \spestag{e3}\\
%          &\mathrm{i}^\mathrm{i} = \mathrm{e}^{-\pi/2}\spestag{e4}
%          \end{align}
%          \end{Ipara}
% \endgroup
%
%
%    \subsection{Reference commands}
%
%    \begin{cminipage}[1.05\textwidth]
%    \begin{syntdiag}
%    \tok{\argcmd{\rtp}{ref}}
%      \begin{stack}\\ \tok{$\ast$}  \end{stack}
%      \begin{stack}\\ \tok{\okeylst}  \end{stack}
%      "{"
%       \begin{rep} \tok{<lbl>} \\ ~,~ \end{rep}
%      "}"
%    \end{syntdiag}
%    \end{cminipage}
%    \smallskip
%
%    \begin{cminipage}[1.05\textwidth]
%    \begin{syntdiag}
%    \tok{\argcmd{\Rtp}{ref}}
%      \begin{stack}\\ \tok{$\ast$}  \end{stack}
%      \begin{stack}\\ \tok{\okeylst}  \end{stack}
%      \tok{"{"}
%       \begin{rep} \tok{<lbl>} \\ ~,~ \end{rep}
%      \tok{"}"}
%    \end{syntdiag}
%    \end{cminipage}
%    \blankline
%    \noindent
%    The |*| optional form of the \argcmd{\rtp}{ref} and
%    \argcmd{\Rtp}{ref} commands eliminates the
%    name prefix.  The |[s]| optional key-value argument is for the
%    plural form of the name prefix.
%    \blankline
%
%       \noindent
%       \textbf{\slshape Examples:}
%          \begin{Itabb}
%          \hspace*{21em}\=\kill
%          |in \eqref{e1} ...|                \>in \eqref{e1} ...                \\
%          |in \eqref[s]{e1}--\eqref*{e4} ...|\>in \eqref[s]{e1}--\eqref*{e4} ...\\
%          |in \eqref[name=eq.~]{e2} ...|     \>in \eqref[name=eq.~]{e2} ...
%          \end{Itabb}
%    \pagebreak[1]
%    \noindent
%    A list of references can be used:
%          \begin{Itabb}
%          \hspace*{21em}\=\kill
%          |in \eqref{e1,e2} ...|             \>in \eqref{e1,e2} ...             \\
%          |in \eqref{e1,e2,e3} ...|          \>in \eqref{e1,e2,e3} ...         \\
%          |in \eqref[lsttxt={, and~}]{e1,e2,e3} ...|          \>in \eqref[lsttxt={, and~}]{e1,e2,e3} ...
%          \end{Itabb}
%    \pagebreak[2]
%    The reference to the page can be included with the \opt{vref} or \opt{vref=far}
%    options that activates the \pkg{varioref} reference.
%          \begin{Itabb}
%          \hspace*{18em}\=\kill
%          |in \eqref[vref]{e1} ...|          \>in \eqref[vref]{e1} ...\\
%          |in \eqref[vref=far]{e1} ...|      \>in \eqref[vref=far]{e1} ...
%          \end{Itabb}
%    The \argcmd{\Rtp}{ref} command is identical to the
%    \argcmd{\rtp}{ref} command except that it uses the \opt{Name} and
%    \opt{Names} key-value options.
%          \begin{Itabb}
%          \hspace*{18em}\=\kill
%          |\Eqref{e1} is ...|                \>\Eqref{e1} is ...                \\
%          |\Eqref{e1,e2} are ...|            \>\Eqref{e1,e2} are ...            \\
%          |\Eqref[lsttxt={\ or~}]{e1,e2,e3}| \>\Eqref[lsttxt={\ or~}]{e1,e2,e3}
%          \end{Itabb}
%    \noindent
%    References to external documents can be added with the \opt{xr}
%    option. Please read the documentation of the \pkg{xr} package.
%
%          \begin{Itabb}
%          \hspace{10em}\=hspace{5em}\=\kill
%          \cmd{\eqref}|[|\opt{xr}|=A-]{xyz}|   \> \>$\rightarrow$~
%                   |equation~\ref{A-eq:xyz}|
%          \end{Itabb}
%
%
%    \subsection{Range reference commands}
%
%    \begin{cminipage}[\linewidth]
%    \begin{syntdiag}
%    \tok{\argcmd{\rtp}{rangeref}}
%      \begin{stack}\\ \tok{$\ast$}  \end{stack}
%      \begin{stack}\\ \tok{\okeylst}  \end{stack}
%      \tok{"{"<lbl>$_1$"}{"<lbl>$_2$"}"}
%    \end{syntdiag}
%    \end{cminipage}
%    \smallskip
%
%    \begin{cminipage}[\linewidth]
%    \begin{syntdiag}
%    \tok{\argcmd{\Rtp}{rangeref}}
%      \begin{stack}\\ \tok{$\ast$}  \end{stack}
%      \begin{stack}\\ \tok{\okeylst}  \end{stack}
%      \tok{"{"<lbl>$_1$"}{"<lbl>$_2$"}"}
%    \end{syntdiag}
%    \end{cminipage}
%    \blankline
%
%    \noindent
%    The \argcmd{\rtp}{rangeref} and \argcmd{\Rtp}{rangeref} commands
%    return a range of references and take two arguments. The |*|
%    optional form again eliminates the name prefix.
%     \blankline
%
%       \noindent
%       \textbf{\slshape Examples:}
%          \begin{Itabb}
%          \hspace*{19em}\=\kill
%          |in \eqrangeref{e1}{e4} ...|          \>in \eqrangeref{e1}{e4} ...       \\
%          |... and \eqrangeref*{e1}{e4} ...|    \>... and \eqrangeref*{e1}{e4} ... \\
%          |\Eqrangeref{e1}{e4} are ...|         \>\Eqrangeref{e1}{e4} are ...      \\
%          |\Eqrangeref[vref,rngtxt=--]{e1}{e4}| \>\Eqrangeref[rngtxt=--,vref]{e1}{e4}
%          \end{Itabb}
%
%
%    \subsection{Page reference command}\seclabel{PRefCmds}
%
%    \begin{cminipage}
%    \begin{syntdiag}
%    \tok{\argcmd{\rtp}{pageref}}
%      \begin{stack}\\ \tok{\okeylst}  \end{stack}
%      \tok{"{"<lbl>"}"}
%    \end{syntdiag}
%    \end{cminipage}
%    \blankline
%    \noindent
%    The \argcmd{\rtp}{pageref} commands returns the page number of a
%    reference.
%     \blankline
%
%       \noindent
%       \textbf{\slshape Examples:}
%          \begin{Itabb}
%          \hspace*{18em}\=\kill
%          |it is on page \eqpageref{e1} ...| \>it is on page \eqpageref{e1} ...\\
%          |it is \eqpageref[vref]{e1} ...|   \>it is \eqpageref[vref]{e1} ...
%          \end{Itabb}
%
%
%^^A=================================================================
%    \begin{table}
%      \newcommand*{\Q}{\ensuremath{\blacksquare}}^^A
%      \newcommand*{\X}{\ensuremath{\square}}^^A
%      \setlength{\fboxsep}{1em}^^A
%      \centering
%      \caption[The available options and key-value parameters]^^A
%              {The available options and key-value parameters for
%               the label and reference commands of the \RS\ package.}
%      \tablabel{refopt}
%      \blankline
%    \begin{cminipage}[.9\textwidth]
%    \renewcommand{\thempfootnote}{\fnsymbol{mpfootnote}}
%    \renewcommand{\footnoterule}{}
%    \begin{tabular*}{\linewidth}{@{\extracolsep{\fill}}llccccccc@{}}
%     & & \multicolumn{7}{c}{Commands\footnote{^^A
%             The commands are obtained by calling the configuring command
%             \cmd{\newref} and setting the default key-values. The
%             active key-values can be changed temporarily inside the
%             commands.}} \\
%     \cline{3-9}
%     Parameter & Default &
%     \begin{turn}{90}\argcmd{{\rtp}}{key}~      \end{turn}&
%     \begin{turn}{90}\argcmd{{\rtp}}{label}~    \end{turn}&
%     \begin{turn}{90}\argcmd{{\rtp}}{ref}~      \end{turn}&
%     \begin{turn}{90}\argcmd{{\Rtp}}{ref}~      \end{turn}&
%     \begin{turn}{90}\argcmd{{\rtp}}{rangeref}~ \end{turn}&
%     \begin{turn}{90}\argcmd{{\Rtp}}{rangeref}~ \end{turn}&
%     \begin{turn}{90}\argcmd{{\rtp}}{pageref}~  \end{turn}\\
%     \hline
%     \noalign{\smallskip}
%     ~$\ast$              &                  &\X &\X &\Q &\Q &\Q &\Q &\X\\
%     \okeylst             &                  &\X &\X &\Q &\Q &\Q &\Q &\Q\\[1ex]
%     \opt{key}         &|={|\meta{\rtp:}|},| &\X &\X &\Q &\Q &\Q &\Q &\Q\\
%     \opt{s}           & |={true}|\footnote{^^A
%          Options defaults to \opt{true} but is initialized
%          as \opt{false}. The \argcmd{{\rtp}}{ref}\texttt{[s]}\marg{lbl}
%          command results in the plural: \opt{names}.
%          The same principle is also valid for the \opt{vref}
%          option.\fnlabel{b}}|,|
%                                             &\X &\X &\Q &\Q &\X &\X &\X\\
%     \opt{vref}        &|={true}|\fnref*{b}|,|
%                                             &\X &\X &\Q &\Q &\Q &\Q &\Q\\
%     \opt{xr}          &|={},|               &\X &\X &\Q &\Q &\Q &\Q &\Q\\[1ex]
%     \opt{name}        &|={},|               &\X &\X &\Q &\X &\X &\X &\X\\
%     \opt{names}\footnote{^^A
%           Option depends on the selection of the \texttt{s=true/false}
%           optional key-value for singular or plural.\fnlabel{c}}
%                       &|={},|               &\X &\X &\Q &\X &\Q &\X &\X\\
%     \opt{Name}        &|={},|               &\X &\X &\X &\Q &\X &\X &\X\\
%     \opt{Names}\fnref*{c}
%                       &|={},|               &\X &\X &\X &\Q &\X &\Q &\X\\
%     \opt{lsttwotxt}   &|={\space and~},|    &\X &\X &\Q &\Q &\X &\X &\X\\
%     \opt{lsttxt}      &|={\space and~},|    &\X &\X &\Q &\Q &\X &\X &\X\\
%     \opt{rngtxt}      &|={\space to~},|     &\X &\X &\X &\X &\Q &\Q &\X\\[1ex]
%     \opt{refcmd}      &|=\ref{#1},|         &\X &\X &\Q &\Q &\Q &\Q &\Q\\
%
%     \hline
%     \noalign{\smallskip}
%
%     \footnotesize \Q\ -- \makebox[0pt][l]{\footnotesize Available}\\
%     \footnotesize \X\ -- \makebox[0pt][l]{Not available}
%     \end{tabular*}
%     \end{cminipage}
%    \end{table}
%^^A=================================================================
%
%
%
%    \clearpage
%    \section{\pkg{Keyval} Optional Arguments}\seclabel{keyvals}
%
%    Al the options for the referencing commands are set with a key-value
%    list. \Tabref[vref]{refopt} gives a full list of all the key-values and
%    defaults. The options can also be changed locally with the
%    \okeylst\ optional arguments.
%
%
%    \subsection{Identifier: \opt{key}}
%
%    The \opt{key} key-value is the prefix to the reference label
%    of the \cmd{\ref} and \cmd{\pageref} commands.
%    The default is \opt{key=}\meta{type\,:\,}.
%    With the default \RScfg:
%
%    \begin{Itabb}
%    \tab\=\hspace{3.5cm}\=\hspace{2em}\=\kill
%    \cmd{\newref}|{tab}{...}|\\[1ex]
%       \>|\tablabel{abc}|
%                 \>~ $\rightarrow$ |\label{tab:abc}|\\
%       \>|\tabref{abc}|
%                 \>~ $\rightarrow$ |table~\ref{tab:abc}|\\[1ex]
%    \cmd{\newref}|{tab}{key=xxx-,...}|\\[1ex]
%       \>|\tablabel{abc}|
%                 \>~ $\rightarrow$ |\label{xxx-abc}|\\
%       \>|\tabref{abc}|
%                 \>~ $\rightarrow$ |table~\ref{xxx-abc}|
%    \end{Itabb}
%    For existing documents containing labels such as |\label{tab:xx}|,
%    you can use the \opt{nokeyprefix} option which defines \opt{key}|={}|. The
%    normal \cmd{\label} command can then be used and the reference
%    commands defaults to
%    \begin{Itabb}
%    \tab\=\hspace{3.5cm}\=\hspace{2em}\=\kill
%       \>\cmd{\label}|{tab:xx}| \\
%       \>|\tabref{tab:xx}|  \>~ $\rightarrow$ |table~\ref{tab:xx}|
%    \end{Itabb}
%
%
%    \subsection{Plural form: \opt{s}}
%
%    The \opt{s} conditional option (\texttt{true/false}) switches the
%    singular/plural form of the reference on and off. The default is
%    \opt{s}\texttt{=true}, but it is initialized to \texttt{false}.
%    \begin{Itabb}
%    \tab\=\hspace{3.5cm}\=\hspace{2em}\=\kill
%       |\tabref{abc}|
%            \>\>~ $\rightarrow$ |table~\ref{tab:abc}|\\
%       |\tabref[s]{abc}|
%             \>\>~ $\rightarrow$ |tables~\ref{tab:abc}|
%    \end{Itabb}
%
%
%    \subsection{Extended reference: \opt{vref}}
%
%    The \opt{vref} conditional option (\texttt{true/far/false}) switches the
%    \pkg{varioref} page referencing on and off. The default is
%    \opt{vref}\texttt{=true}, but it is initialized to \texttt{false}.
%
%    \begin{Itabb}
%    \tab\=\hspace{3.5cm}\=\hspace{2em}\=\kill
%       |\tabref{abc}|
%            \>\>~ $\rightarrow$ |table~\ref{tab:abc}|\\
%       |\tabref[vref]{abc}|
%             \>\>~ $\rightarrow$ |table~\ref{tab:abc}|
%                  |\vpageref[\unskip]{tab:abc}|\\
%       |\tabref[vref=far]{abc}|^^A
%             \>\>~ $\rightarrow$ |table~\ref{tab:abc}|
%                  |\reftextfaraway{tab:abc}|
%    \end{Itabb}
%
%
%    \subsection{External interfaces: \opt{xr}}
%
%    The \opt{xr} option is for references to external documents.
%    It inserts a prefix in the reference label, compatible with the
%    \pkg{xr} package. The default is \opt{xr}|={}|.
%
%    External document can be defined in the preamble with the
%    \pkg{xr} or \pkg{xr-hyper} packages:
%    \begin{Ipara}
%      \cmd{\usepackage}|{xr}|\\
%      \cmd{\externaldocument}\oarg{xr\_key}\marg{filename}
%    \end{Ipara}
%
%    \noindent
%    If, for example, an external document defined with
%    \meta{xr\_key}\texttt{=\{A-\}}, uses an identical setup (e.g.\ the same
%    \RScfg), then it can be referenced with
%
%    \begin{Itabb}
%    \tab\=\hspace{3.5cm}\=\hspace{2em}\=\kill
%       |\tabref[xr=A-]{abc}|
%            \>\>~ $\rightarrow$ |table~\ref{A-tab:abc}|
%    \end{Itabb}
%    \noindent or otherwise
%    \begin{Itabb}
%    \tab\=\hspace{3.5cm}\=\hspace{2em}\=\kill
%     |\tabref[xr=A-,key=]{abc}|
%           \>\>~ $\rightarrow$ |table~\ref{A-abc}|
%    \end{Itabb}
%
%
%    \subsection{Language parameters: \opt{name}, \opt{names},
%    \opt{Name}, \opt{Names}, \opt{rngtxt}, \opt{lsttwotxt}, \opt{lsttxt}}
%    \seclabel{langopts}
%
%    This key-values contain the text prefixes and insertions.
%    Every house style or user has his or her own preference for naming the reference types,
%    therefore are there no defaults provided.
%
%    \begin{Itabb}
%    \hspace*{1.5cm}\=\hspace{7cm}\=\kill
%    \opt{name} \> ---~Inside sentence reference prefix (singular)
%                \> default=|{}|\\
%    \opt{names}\> ---~Inside sentence reference prefix (plural)
%                \> default=|{}|\\
%    \opt{Name} \> ---~First word reference prefix (singular)
%                \> default=|{}|\\
%    \opt{Names}\> ---~First word reference prefix (plural)
%                \> default=|{}|\\
%    \opt{rngtxt}\> ---~Range of references
%                \> default=|{\ to~}|\\
%    \opt{lsttwotxt}\> ---~List of references (two)
%                \> default=|{\ and~}|\\
%    \opt{lsttxt}\> ---~List of references (more than two)
%                \> default=|{\ and~}|
%    \end{Itabb}
%    \noindent
%    Good typographic style manuals recommend the minimum use of
%    capital letters and punctuation that breaks the flow of a sentence
%    or paragraph. For abbreviations, Bringhurst\cite{Bring1996}
%    recommends the Oxford house style: Use a period only when the word
%    stops prematurely. The period is omitted if the abbreviation
%    begins with the first letter and end with the last. As an example
%    for equations, use eq.~(1) or eqn~(1).
%    A good guideline is not to abbreviate any reference type names.
%    If a sentence starts with a
%    reference then the type name must always be written in full.
%    A typical example for references to a table is:
%    \begin{Ipara}
%    \opt{name}|  ={table~},   |\opt{names}| ={tables~},|\\
%    \opt{Name}|  ={Table~},   |\opt{Names}| ={Tables~},|\\
%    \opt{rngtxt}|={\ to~},    |\opt{lsttxt}|={, and~},   |\opt{lsttwotxt}|={\ and~},|
%    \end{Ipara}
%    Note the hardspace after the text. It is needed to keep the text
%    and the reference together on the same line.
%
%    The \RS\ configuration file can be setup to interface with \pkg{babel}
%    for different languages or for automatic language changes inside a
%    document. The language specific key-values can be added to the
%    \pkg{babel} hook |\extras|\meta{language}. The command
%    \DescribeMacro{\DeclareLangOpt}\cmd{\DeclareLangOpt}\footnote{^^A
%        Only for use in \RScfg\ the default config file}
%    is provided to supply a \meta{language} option to the package
%    and to add the option contents to |\extras|\meta{language}. The
%    default config file contains the following lines for equations:
% \begin{Ipara}
%    |\newcommand\RSenglish{%      |\\
%    |   \def\RSeqtxt{equation~}%  |\\
%    |   \def\RSeqstxt{equations~}%|\\
%    |   \def\RSEqtxt{Equation~}%  |\\
%    |   \def\RSEqstxt{Equations~}%|\\
%    |   :                         |\\
%    |   }                         |\\
%    |\DeclareLangOpt{english}{\RSenglish}|
% \end{Ipara}
% \DescribeMacro{\RSaddto}or manually with the \cmd{\RSaddto} command
% \begin{Ipara}
%    |\RSaddto{\extrasenglish}{\RSenglish}|
% \end{Ipara}
%    The key-value options for language specific options
%    are then set as:
% \begin{Ipara}
%    |\newref{eq}{%        |\\
%    |   name  = \RSeqtxt, |\\
%    |   names = \RSeqstxt,|\\
%    |   Name  = \RSEqtxt, |\\
%    |   Names = \RSEqstxt,|\\
%    |   :                 |\\
%    |   }                 |
% \end{Ipara}
%
%    \noindent\LaTeX{}/\pkg{babel} provides some language specific
%    names that can be utilized.
%
%    \begin{Ipara}
%       |\chaptername       \appendixname|\\
%       |\figurename        \tablename|\\
%       |\partname          \pagename|
%    \end{Ipara}
%
%       To setup a multilingual document with \pkg{babel},
%    \emph{always} make the language options global so that other
%    language compliant packages can detect it.
%    A typical setup for an Afrikaans/English document would be:
%    {\makeatletter\def\meta@font@select{\normalfont\ttfamily\slshape}
%    \begin{Ipara}
%    |\documentclass[UKenglish,afrikaans,|\meta{options}|]{|\meta{LaTeX\_class}|}|\\
%    |\usepackage[T1]{fontenc}%..|\texttt{\slshape hyphenation of words with accents}\\
%    |\usepackage{babel}%........|\texttt{\slshape language def's}\\
%    |\usepackage{varioref}%.....|\texttt{\slshape for \opt{vref} option}\\
%    |\usepackage{refstyle}|
%    \end{Ipara}}
%
%
% \subsection{Reference formatting command: \opt{refcmd}}
%
%    The \opt{refcmd} key-value holds the contents of the internal
%    command that formats the reference. The |#1| parameter
%    passed to the command is the full reference label.
%    For example:
%    \begin{Itabb}
%    \hspace{10em}\=hspace{5em}\=\kill
%    \opt{refcmd}|=(\ref{#1})| \> $\rightarrow$~
%                          |(\ref{|\meta{label}|})|
%    \end{Itabb}
%
%    External commands can be employed. For example, to make references
%    to equations identical to the \AmS\ \cmd{\eqref} command:
%    \begin{Itabb}
%        \opt{refcmd}|={\textup{\tagform@{\ref{#1}}}}%|~~It needs \file{amsmath.sty}
%    \end{Itabb}
%
%    The \opt{refcmd} can be used in conjuntion with the |\ifRSstar|,
%    |\ifRSnameon|, |\ifRSplural| and |\ifRScapname| internal
%    conditional variables to format the reference. As an example for a
%    reference to a footnote, where a duplicate footnote mark is needed,
%    can the \opt{refcmd} be configured so that the starred form of the
%    reference command produce a superscripted duplicate mark:
%    \begin{Itabb}
%       |\newcommand{\RSfnmark}[1]{%|\\
%       |   \begingroup             |\\
%       |      \unrestored@protected@xdef\@thefnmark{#1}%|\\
%       |   \endgroup               |\\
%       |   \@footnotemark}         |\\[1ex]
%       \opt{refcmd}|={\ifRSstar\RSfnmark{\ref{#1}}\else(\ref{#1})\fi}|
%    \end{Itabb}
%    \noindent The second footnote mark, \fnref*[name]{b}, in
%    \tabref[vref]{refopt}, was obtained in this way with the reference
%    |\fnref*{b}|. See \RScfg\  for another
%    example for references to chapters and appendices.
%
%    The \pkg{nameref} package can easily be incorporated if you need elaborate
%    references which include the section or chapter name:
%    \begin{Ipara}
%       |\Secref[vref, refcmd={\S\ref{#1}, `\nameref{#1}'}]{PRefCmds}|
%    \end{Ipara}
%    gives
%    \begin{Ipara}
%       \Secref[vref, refcmd={\S\ref{#1}, `\nameref{#1}'}]{PRefCmds}
%    \end{Ipara}
%
% \section{Default configuration files}
%
%    The \pkg{refstyle} package first searches for the \file{refstyle.def}
%    configuration file, and if it not found, then it uses the default
%    configuration file \RScfg\ that is supplied with the package.
%
%    The default configuration file, \RScfg, makes a
%    number of default reference declarations and provides language
%    definitions for the language parameters. See the file
%    \file{refconfig.pdf} for the documentation.
%    \medskip
%
%    \begin{center}
%    \begin{fminipage}[.55\textwidth]
%    \slshape Any user is welcome to customize the
%             local copy of the \RScfg\ file or copy
%             the relevant contents to your own \file{refstyle.def}
%             configuration file.
%    \end{fminipage}
%    \end{center}
%
% \begin{thebibliography}{99}
%    \bibitem{Bring1996}
%      {Bringhurst, R.} (1996), \emph{The elements of typographic style},
%            Hartley \& Marks Publishers, Point Roberts, WA, USA and Vancouver,
%            BC, Canada, second edn.
% \end{thebibliography}
%
% \StopEventually{\PrintChanges\PrintIndex}
% \clearpage
%
% \section{Implementation: \pkg{refstyle.sty}}
%
% \subsection{Identification}
%    \begin{macrocode}
%<*pkg>
\NeedsTeXFormat{LaTeX2e}[1999/12/01]
\ProvidesPackage{refstyle}[\RefstyleFileDate\space
                           \RefstyleFileVersion\space
                           Reference formatting (DNJ Els)]
\newcommand*{\RS@pkgname}{refstyle}
%    \end{macrocode}
%
%
%    \subsection{External packages}
%
%    Load all the external packages.
%    \begin{macrocode}
\RequirePackage{keyval}
%    \end{macrocode}
%
%    \begin{macro}{\RS@setkeys}
%    \changes{v0.3}{2006/09/09}{original \cmd{\setkeys} copy}
%    Note if \pkg{xkeyval} is loaded, it redefines \pkg{keyval}'s
%    macros. To fix this bug, we need the original \cmd{\setkeys}
%    command.
%    \begin{macrocode}
\def\RS@setkeys#1#2{%
   \def\KV@prefix{KV@#1@}%
   \let\@tempc\relax
   \KV@do#2,\relax,}
%    \end{macrocode}
%    \end{macro}
%
%    \noindent A small bug-fix for \pkg{showkeys}. Will be removed
%    after release of new version.
%
%    \begin{macrocode}
\@ifundefined{vref@space}{\let\vref@space\space}{}
%    \end{macrocode}
%
%    \begin{macro}{\@safe@activestrue}
%    \changes{v0.5}{2010/11/02}{Add \cmd{\@safe@activesfalse}}
%    \begin{macro}{\@safe@activesfalse}
%    \changes{v0.5}{2010/11/02}{Add \cmd{\@safe@activestrue}}
%    If \pkg{babel} is not loaded, make the following
%    commands inactive.
%    \begin{macrocode}
\providecommand*{\@safe@activestrue}{}
\providecommand*{\@safe@activesfalse}{}
%    \end{macrocode}
%    \end{macro}
%    \end{macro}
%
%    \subsection{Utility commands}
%
%    \begin{macro}{\RS@namelet}
%    \begin{macro}{\RS@nameuse}
%    \begin{macro}{\RS@namedef}
%    \begin{macro}{\RS@robustnamedef}
%    \noindent
%    The following is a list of commands that take a variable \marg{name}
%    as argument. This enables on-the-fly definitions of user commands.
%    \begin{Itabb}
%    Usage:~\=\cmd{\RS@robustnamedef}\marg{name}~\=\kill
%    Usage:
%     \>\cmd{\RS@namelet}\marg{name}
%          \> $\rightarrow$ \cmd{\let}\cmeta{name}\\
%     \>\cmd{\RS@nameuse}\marg{name}
%         \> $\rightarrow$ \cmeta{name}\\
%     \>\cmd{\RS@namedef}\marg{name}
%         \> $\rightarrow$ \cmd{\def}\cmeta{name}\\
%     \>\cmd{\RS@robustnamedef}\marg{name}
%         \> $\rightarrow$
%             \cmd{\def}\cmeta{name}|{\protect|\cmeta{name\textvisiblespace}|}|^^A
%               \cmd{\def}\cmeta{name\textvisiblespace}
%    \end{Itabb}
%    \begin{macrocode}
\newcommand*{\RS@namelet}[1]{\expandafter\let\csname #1\endcsname}
\newcommand*{\RS@nameuse}[1]{\csname #1\endcsname}
\newcommand*{\RS@namedef}[1]{\expandafter\def\csname #1\endcsname}
\newcommand*{\RS@robustnamedef}[1]{%
   \expandafter\edef\csname #1\endcsname{%
      \noexpand\protect\RS@nameuse{#1 }}%
   \RS@namedef{#1 }}
%    \end{macrocode}
%    \end{macro}
%    \end{macro}
%    \end{macro}
%    \end{macro}
%
%    \begin{macro}{\RS@ifundefined}
%    This is an improved definition\footnote{Posted by Markus Kohm
%    on c.t.t. 2002/11/11} for the \LaTeX{} kernel command
%    \cmd{\@ifundefined} that do not leave an undefined command
%    defined as \cmd{\relax} after the test.
%
%    The usage is:
%    \cmd{\RS@ifundefined}\marg{name}\marg{true}\marg{false} executes
%    the contents of \meta{true} if \cmeta{name} is not defined and
%    \meta{false} if defined.
%    \begin{macrocode}
\def\RS@ifundefined#1{%
   \begingroup\expandafter\expandafter\expandafter\endgroup
   \expandafter\ifx\csname#1\endcsname\relax
       \expandafter\@firstoftwo
   \else
       \expandafter\@secondoftwo
   \fi}
%    \end{macrocode}
%    \end{macro}
%
%    \begin{macro}{\RS@removedef}
%    The command removes the definition of a command, including
%    robust definitions.
%    \begin{macrocode}
\newcommand*{\RS@removedef}[1]{%
   \RS@namelet{#1}\@undefined%
   \RS@ifundefined{#1 }{}{\RS@namelet{#1 }\@undefined}}
%    \end{macrocode}
%    \end{macro}
%
%    \begin{macro}{\RS@testednamedef}
%    \begin{macro}{\RS@testedrobustnamedef}
%    These command are identical to \cmd{\RS@namedef} and
%    \cmd{\RS@robustnamedef}, but only define the \cmeta{name}
%    command if it is legal. Otherwise an error message
%    is written to the log file and the program is terminated.
%    \begin{macrocode}
\newcommand*{\RS@testednamedef}[1]{%
   \RS@ifnamedefinable{#1}\RS@namedef{#1}}
%    \end{macrocode}
%    \begin{macrocode}
\newcommand*{\RS@testedrobustnamedef}[1]{%
   \RS@ifnamedefinable{#1}\RS@robustnamedef{#1}}
%    \end{macrocode}
%    \end{macro}
%    \end{macro}
%
%    \begin{macro}{\RS@ifnamedefinable}
%    A modified version of the \LaTeX{} kernel command
%    (from ltdefns.dtx).
%    \begin{macrocode}
\long\def\RS@ifnamedefinable #1{%
    \edef\reserved@a{#1}%
    \RS@ifundefined\reserved@a
       {\edef\reserved@b{\expandafter\@carcube \reserved@a xxx\@nil}%
       \ifx \reserved@b\@qend \RS@notdefinable\else
          \ifx \reserved@a\@qrelax \RS@notdefinable\else
            \PackageInfo{\RS@pkgname}{\@backslashchar\reserved@a\space created}%
          \fi
       \fi}%
    \RS@notdefinable}
%    \end{macrocode}
%    \end{macro}
%
%    \begin{macro}{\RS@notdefinable}
%    The error message when an illegal definition is attempted.
%    \begin{macrocode}
\gdef\RS@notdefinable{%
   \PackageError{\RS@pkgname}{%
      Command \@backslashchar\reserved@a\space
      already defined.\MessageBreak
      Or name \@backslashchar\@qend... illegal.\MessageBreak
      It can not be redefined by the \@backslashchar newref%
      }{%
      If \@backslashchar\reserved@a\space is not important\MessageBreak
      then \protect\let\@backslashchar\reserved@a%
      =\protect\relax,\MessageBreak
      else use a different \@backslashchar newref.}%
      }
%    \end{macrocode}
%    \end{macro}
%
%    \begin{macro}{\RS@setbool}
%    The command\footnote{Taken from the \pkg{ifthen} package.}
%    \cmd{\RS@setbool}\marg{conditional}\marg{true/false}
%    sets the \meta{conditional} to |true| or |false|.
%    \begin{Itabb}
%    Usage:~\=
%      \cmd{\RS@setbool}|{RSplural}{false}|\quad\=$\rightarrow$\quad
%          |\RSpluralfalse|\\
%    \>\cmd{\RS@setbool}|{RSplural}{true}|\quad\>$\rightarrow$\quad
%          |\RSpluraltrue|
%    \end{Itabb}
%
%    \begin{macrocode}
\newcommand*{\RS@setbool}[2]{%
  \lowercase{\def\@tempa{#2}}%
  \@ifundefined{@tempswa\@tempa}%
    {\PackageError{\RS@pkgname}%
       {You can only set the option to `true' or `false'}\@ehc}%
    {\csname#1\@tempa\endcsname}}
%    \end{macrocode}
%    \end{macro}
%
%
% \subsection{First character case changes}
%
%    \begin{macro}{\RS@firstcap}
%    This macro\footnote{Posted by Dan Luecking on c.t.t.} change the
%    first character of a string to uppercase and returns the result
%    in \cmd{\RS@cap}.
%    \begin{Ipara}
%    Usage:~|\RS@fistcap xxxx\@nil| then |\RS@cap| $\rightarrow$ |Xxxx|
%    \end{Ipara}
%    \begin{macrocode}
\def\RS@firstcap#1#2\@nil{%
  \iffalse{\fi
    \uppercase{\edef\RS@cap{\iffalse}\fi#1}#2}}%
%    \end{macrocode}
%    \end{macro}
%
%
%
%  \subsection{Reference building commands}
%
%    \begin{macro}{\ifRS@keyactive}
%    \changes{v0.4}{2010/10/21}{Add \opt{nokeyprefix} to turn of the key prefix}
%    We need to peek into the options list for the \opt{nokeyprefix} option
%    before the options are processed to find out if the key prefix must be
%    included in the |\|\meta{key}|label| command.  Make \opt{nokeyprefix}
%    not used  afterwards.
%    \begin{macrocode}
\newif\ifRS@keyactive
\@ifpackagewith{\@currname}{nokeyprefix}%
    {\RS@keyactivefalse}%
    {\RS@keyactivetrue}
\DeclareOption{nokeyprefix}{\OptionNotUsed}
%    \end{macrocode}
%    \end{macro}
%
%
%    \begin{macro}{\ifRSstar}
%    \begin{macro}{\ifRSnameon}
%    \begin{macro}{\ifRScapname}
%    \begin{macro}{\ifRSplural}
%    \begin{macro}{\ifRSlsttwo}
%    \changes{v0.4}{2010/10/21}{Add \cmd{\ifRSlsttwo} test for list containing only two parameter}
%    The |\if| conditional values that are set by the reference commands.
%    These values can be accessed by user defined key-values.
%    \begin{macrocode}
\newif\ifRSstar\RSstarfalse
\newif\ifRSnameon\RSnameontrue
\newif\ifRScapname\RScapnamefalse
\newif\ifRSplural\RSpluralfalse
\newif\ifRSlsttwo\RSlsttwofalse
%    \end{macrocode}
%    \end{macro}
%    \end{macro}
%    \end{macro}
%    \end{macro}
%    \end{macro}
%
%
%    \begin{macro}{\newref}
%    The main user interface for template setup. It take the |#1| or
%    \meta{key} parameter and make it lowercase before passing it on to
%    \cmd{\RS@newref}.
%
%
%    \begin{macrocode}
\newcommand*{\newref}[1]{%
   \lowercase{\def\RS@tempa{#1}}%
   \expandafter\RS@newref\expandafter{\RS@tempa}}
%    \end{macrocode}
%    \end{macro}
%
%    \begin{macro}{\RS@newref}
%    \changes{v0.4}{2010/10/21}{Ad \texttt{lsttwotxt} option}
%    This command configures a new template.
%    \begin{macrocode}
\newcommand*{\RS@newref}[2]{%
%    \end{macrocode}
%    \bNP[2.0]\noindent Clears an existing template before defining a new one.\eNP
%    \begin{macrocode}
   \RS@clearref{#1}%
%    \end{macrocode}
%    \bNP[2.0]\noindent Create |\ifRS@|\meta{key}|vref| conditional \eNP
%    \begin{macrocode}
%   \expandafter\newif\csname ifRS@#1vref\endcsname%
%    \end{macrocode}
%    \bNP[2.0]\noindent Creates a series of key-values for every template
%           that stores the setup for the specific template.\eNP
%    \begin{macrocode}
    \ifRS@keyactive
        \define@key{RS@#1}{key}[#1:]{\RS@namedef{RS@#1@key}{##1}}%
    \else
        \define@key{RS@#1}{key}[]{\RS@namedef{RS@#1@key}{##1}}%
    \fi
    \define@key{RS@#1}{s}[true]{\RS@setbool{RSplural}{##1}}%
    \define@key{RS@#1}{name}[]{\RS@namedef{RS@#1@name}{##1}}%
    \define@key{RS@#1}{names}[]{\RS@namedef{RS@#1@names}{##1}}%
    \define@key{RS@#1}{Name}[]{\RS@namedef{RS@#1@Name}{##1}}%
    \define@key{RS@#1}{Names}[]{\RS@namedef{RS@#1@Names}{##1}}%
    \define@key{RS@#1}{rngtxt}[\space to~]{\RS@namedef{RS@#1@rngtxt}{##1}}%
    \define@key{RS@#1}{lsttwotxt}[\space and~]{\RS@namedef{RS@#1@lsttwotxt}{##1}}%
    \define@key{RS@#1}{lsttxt}[\space and~]{\RS@namedef{RS@#1@lsttxt}{##1}}%
    \define@key{RS@#1}{refcmd}[\ref{####1}]{\RS@namedef{RS@#1@rcmd}####1{##1}}%
    \define@key{RS@#1}{xr}[]{\RS@namedef{RS@#1@xr}{##1}}%
    \define@key{RS@#1}{vref}[true]{\RS@namedef{RS@#1vref}{##1}}%
%    \end{macrocode}
%    \bNP[2.0]\noindent Set default key-value parameters.\eNP
%    \begin{macrocode}
    \RS@setkeys{RS@#1}{key,
                       s=false,
                       name,names,Name,Names,
                       rngtxt,lsttwotxt,lsttxt,
                       refcmd,
                       xr,
                       vref=false}%
%    \end{macrocode}
%    \bNP[2.0]\noindent Set key-values according to user definitions.\eNP
%    \begin{macrocode}
    \RS@setkeys{RS@#1}{#2}%
%    \end{macrocode}
%    \bNP[2.0]\noindent Build the reference commands.\eNP
%    \begin{macrocode}
    \RS@buildref{#1}%
    }
%    \end{macrocode}
%    \end{macro}
%
%    \begin{macro}{\RS@clearref}
%    Clear a reference template for redefinition. It check
%    if the template already exists and clear it if it does.
%
%    \begin{macrocode}
\newcommand*{\RS@clearref}[1]{%
   \RS@ifundefined{RS@#1@template}
      {\RS@namedef{RS@#1@template}{#1}%
       \PackageInfo{\RS@pkgname}%
                   {New reference template \protect\newref{#1}}{}}
      {\PackageInfo{\RS@pkgname}%
                   {Reference template \protect\newref{#1} redefined}{}
       \RS@firstcap#1\@nil
       \RS@removedef{#1key}%
       \RS@removedef{#1label}%
       \RS@removedef{#1ref}%
       \RS@removedef{\RS@cap ref}%
       \RS@removedef{#1rangeref}%
       \RS@removedef{\RS@cap rangeref}%
       \RS@removedef{#1pageref}%
      }%
   }
%    \end{macrocode}
%    \end{macro}
%
%    \begin{macro}{\RS@buildref}
%    Build the reference commands. See \tabref{refopt} for the list of
%    commands. The \cmd{\RS@buildref}\marg{key} build commands to
%    call \cmd{\RS@cmd}\marg{cmd}\marg{key}, for example:
%    \begin{Ipara}
%    \cmeta{key}|ref| \quad$\rightarrow$\quad
%          |{\RScapnamefalse\RS@cmd{ref}|\marg{key}|}|
%    \end{Ipara}
% \changes{v0.2}{2003/07/30}{Remove robust def of \argcmd{{\rtp}}{label}}
%    \begin{macrocode}
\newcommand*{\RS@buildref}[1]{%
    \RS@firstcap#1\@nil
    \RS@testednamedef{#1key}{\RS@nameuse{RS@#1@key}}
    \RS@testednamedef{#1label}##1{\label{\RS@nameuse{RS@#1@key}##1}}
    \RS@testedrobustnamedef{#1ref}{\RScapnamefalse\RS@cmd{ref}{#1}}
    \RS@testedrobustnamedef{\RS@cap ref}{\RScapnametrue\RS@cmd{ref}{#1}}
    \RS@testedrobustnamedef{#1rangeref}{\RScapnamefalse\RS@cmd{rangeref}{#1}}
    \RS@testedrobustnamedef{\RS@cap rangeref}{\RScapnametrue\RS@cmd{rangeref}{#1}}
    \RS@testedrobustnamedef{#1pageref}{\RScapnamefalse\RS@cmd{pageref}{#1}}
    }
%    \end{macrocode}
%    \end{macro}
%
%    \begin{macro}{\RS@cmd}
%    \begin{macro}{\RS@@cmd}
%    The command \cmd{\RS@cmd}\marg{cmd}\marg{key} calls the final
%    reference formatting commands.
%    It checks for the starred form and set the conditionals
%    |\ifRSstar| and |\ifRSnameon| accordingly.
%    It also extracts the optional key-value list.
%    \begin{Itabb}
%    \hspace*{4.5cm}\=\kill
%    \cmd{\RS@cmd}|{ref}|\marg{key} \>$\rightarrow$\quad
%        |\RS@ref|\marg{key}\okeylst\\
%    \cmd{\RS@cmd}|{rangeref}|\marg{key} \>$\rightarrow$\quad
%        |\RS@rangeref|\marg{key}\okeylst\\
%    \cmd{\RS@cmd}|{pageref}|\marg{key} \>$\rightarrow$\quad
%        |\RS@pageref|\marg{key}\okeylst
%    \end{Itabb}
%    \begin{macrocode}
\newcommand*{\RS@cmd}[2]{%
    \@ifstar{\RSstartrue\RSnameonfalse\RS@@cmd{#1}{#2}}%
            {\RSstarfalse\RSnameontrue\RS@@cmd{#1}{#2}}}
%    \end{macrocode}
%    \begin{macrocode}
\newcommand*{\RS@@cmd}[2]{%
   \@ifnextchar[%
      {\RS@nameuse{RS@#1}{#2}}%
      {\RS@nameuse{RS@#1}{#2}[]}}
%    \end{macrocode}
%    \end{macro}
%    \end{macro}
%
%
%    \subsection{Reference formatting commands}
%
%    \begin{macro}{\RS@ref}
%      \changes{v0.5}{2010/11/02}{Add \cmd{\@safe@activestrue} to avoid active chars is ref list}
%    \begin{macro}{\RS@@ref}
%    \begin{macro}{\RS@@@ref}
%      \changes{v0.4}{2010/10/21}{Add \cmd{\ifRSlsttwo} test for list containing only two parameter}
%    The command \cmd{\RS@ref}\marg{key}\okeylst\marg{label\_lst}
%    typeset the references to the comma-separated reference
%    label list according to the configuration for \meta{key}.
%
%    First of all, remove all spaces for the reference label list.
%    \begin{macrocode}
\def\RS@ref#1[#2]#3{%
    \begingroup
        \RS@setkeys{RS@#1}{#2}%
        \@safe@activestrue%
        \edef\RS@tmpa{\zap@space#3 \@empty}%
        \@safe@activesfalse%
        \edef\RS@tmpa{\noexpand\RS@@ref{#1} \RS@tmpa,\relax\noexpand\@eolst}%
        \RS@tmpa%
    \endgroup}
%    \end{macrocode}
%    \end{macro}
%     Check if there is a single or multiple references in the
%     reference label list. If a single reference label then use the
%     form set by the \opt{s} key-value. If multiple reference labels
%     the use the plural form of the name prefix.\footnote{
%     The list of reference commands came from the \pkg{typedref} package.}
%
%    \begin{macrocode}
\def\RS@@ref#1 #2,#3\@eolst{%
   \ifx\relax#3\relax
      \RS@makename{#1}%
      \RS@makeref{#1}{#2}%
      \RS@makevpageref{#1}{#2}%
   \else
      \RSpluraltrue%
      \RS@makename{#1}%
      \RS@makeref{#1}{#2}%
      \RS@makevpageref{#1}{#2}%
      \RSnameonfalse%
      \RSlsttwotrue%
      \RS@@@ref{#1} #3\@eolst%
   \fi}
%    \end{macrocode}
%    \end{macro}
%    For more than one reference in the reference list, typeset the
%    rest of the references.
%    \begin{macrocode}
\def\RS@@@ref#1 #2,#3\@eolst{%
    \ifx\relax#3\relax
        \ifRSlsttwo
            \RS@nameuse{RS@#1@lsttwotxt}%
        \else
            \RS@nameuse{RS@#1@lsttxt}%
        \fi
        \RS@makeref{#1}{#2}%
        \RS@makevpageref{#1}{#2}%
    \else
        \RSlsttwofalse%
        \unskip,\space%
        \RS@makeref{#1}{#2}%
        \RS@makevpageref{#1}{#2}%
        \RS@@@ref{#1} #3\@eolst%
   \fi}
%    \end{macrocode}
%    \end{macro}
%
%    \begin{macro}{\RS@rangeref}
%    The command \cmd{\RS@rangeref}\marg{key}\okeylst\marg{lbl1}\marg{lbl2}
%    typeset the references as a range.
%    \begin{macrocode}
\def\RS@rangeref#1[#2]#3#4{%
   \begingroup
      \RS@setkeys{RS@#1}{#2}%
      \RSpluraltrue%
      \RS@makename{#1}%
      \RS@makeref{#1}{#3}%
      \RS@nameuse{RS@#1@rngtxt}%
      \RSnameonfalse%
      \RS@makeref{#1}{#4}%
      \RS@makevpagerefrange{#1}{#3}{#4}%
   \endgroup}
%    \end{macrocode}
%    \end{macro}
%
%    \begin{macro}{\RS@pageref}
%    The command \cmd{\RS@pageref}\marg{key}\okeylst\marg{lbl} type
%    the page where \marg{lbl} was defined.
%    \begin{macrocode}
\def\RS@pageref#1[#2]#3{%
   \begingroup%
      \RS@setkeys{RS@#1}{#2}%
      \RS@ifvref{#1}%
         {\mbox{}\vpageref*{\RS@lbl{#1}{#3}}}%
         {\reftextfaraway{\RS@lbl{#1}{#3}}}%
         {\pageref{\RS@lbl{#1}{#3}}}%
   \endgroup}
%    \end{macrocode}
%    \end{macro}
%
% \changes{v0.2}{2003/07/30}{Remove redundant \cmd{\RS@label}}
%
%    \begin{macrocode}
\newcommand*{\RS@true}{true}
\newcommand*{\RS@false}{false}
\newcommand*{\RS@far}{far}
%    \end{macrocode}
%
%    \begin{macro}{\RS@ifvref}
%    The command \cmd{\RS@ifvref}\marg{key}\marg{true}\marg{faraway}\marg{false}
%    executes the contents of \meta{true} if the \opt{vref} option for
%    the \meta{key} reference type is true and \meta{false} otherwise.
%    \begin{macrocode}
\newcommand{\RS@ifvref}[4]{%
   \edef\RS@tempa{\RS@nameuse{RS@#1vref}}%
   \ifx\RS@tempa\RS@true\relax
      #2%
   \else\ifx\RS@tempa\RS@far\relax
      #3%
   \else\ifx\RS@tempa\RS@false\relax
      #4%
   \else
      \PackageError{\RS@pkgname}%
         {You can only set the vref option to `true', `far' or `false'}\@ehc
   \fi\fi\fi}
%    \end{macrocode}
%    \end{macro}
%
%    \begin{macro}{\RS@makename}
%    The command \cmd{\RS@makename}\marg{key} build the prefix
%    to the reference commands.
%    \begin{macrocode}
\newcommand{\RS@makename}[1]{%
   \ifRSstar\else\ifRSnameon
      \ifRSplural
         \ifRScapname
            \RS@nameuse{RS@#1@Names}%
         \else
            \RS@nameuse{RS@#1@names}%
         \fi
      \else
         \ifRScapname
            \RS@nameuse{RS@#1@Name}%
         \else
            \RS@nameuse{RS@#1@name}%
         \fi
      \fi
   \fi\fi
   }
%    \end{macrocode}
%    \end{macro}
%
%    \begin{macro}{\RS@lbl}
%    This command  builds the full label string for the \cmd{\ref}
%    command.
%    \begin{Ipara}
%    \cmd{\RS@lbl}\marg{key}\marg{label} \quad$\rightarrow$\quad
%      |{|\meta{xr_key}\meta{key}\meta{label}|}|
%    \end{Ipara}
%    \begin{macrocode}
\newcommand*{\RS@lbl}[2]{%
   \RS@nameuse{RS@#1@xr}\RS@nameuse{RS@#1@key}#2%
   }
%    \end{macrocode}
%    \end{macro}
%
%    \begin{macro}{\RS@makeref}
%    The command \cmd{\RS@makeref}\marg{key}\marg{label} formats
%    the \cmd{\ref} output
%    \begin{Ipara}
%    \cmd{\RS@makeref}\marg{key}\marg{label} \quad$\rightarrow$\quad
%      \cmeta{rcmd}|{\ref{|\meta{xr_key}\meta{key}\meta{label}|}}|
%    \end{Ipara}
%    \begin{macrocode}
\newcommand{\RS@makeref}[2]{%
   \RS@nameuse{RS@#1@rcmd}{\RS@lbl{#1}{#2}}%
   }
%    \end{macrocode}
%    \end{macro}
%
%    \begin{macro}{\RS@makevpageref}
%    The command \cmd{\RS@makevpageref}\marg{key}\marg{label} adds the
%    \pkg{varioref} page reference if the \opt{vref} option is true.
%    \begin{macrocode}
\newcommand{\RS@makevpageref}[2]{%
   \RS@ifvref{#1}%
             {\vpageref[\unskip]{\RS@lbl{#1}{#2}}}%
             { \reftextfaraway{\RS@lbl{#1}{#2}}}%
             {}%
   }
%    \end{macrocode}
%    \end{macro}
%
%    \begin{macro}{\RS@makevpagerefrange}
%    The command \cmd{\RS@makevpagerefrange}\marg{key}\marg{lbl1}\marg{lbl1}
%    adds the \pkg{varioref} page range reference if the \opt{vref}
%    option is true.
%    \begin{macrocode}
\newcommand{\RS@makevpagerefrange}[3]{%
   \RS@ifvref{#1}%
             {\space\vpagerefrange[\unskip]{\RS@lbl{#1}{#2}}{\RS@lbl{#1}{#3}}}%
             {\space\vpagerefrange[\unskip]{\RS@lbl{#1}{#2}}{\RS@lbl{#1}{#3}}}%
             {}%
   }
%    \end{macrocode}
%    \end{macro}
%
%
%  \subsection{\pkg{varioref} command predefinitions}
%
%    \begin{macrocode}
\AtBeginDocument{%
  \providecommand{\vpageref}{%
     \PackageError{\RS@pkgname}%
        {The vref option used, but varioref.sty not loaded.}%
        {Load varioref.sty}}
  \providecommand{\reftextfaraway}{%
     \PackageError{\RS@pkgname}%
        {The vref=far option used, but varioref.sty not loaded.}%
        {Load varioref.sty}}
  \providecommand{\vpagerefrange}{%
     \PackageError{\RS@pkgname}%
        {The vref option used, but varioref.sty not loaded.}%
        {Load varioref.sty}}
  }
%    \end{macrocode}
%
%
%  \subsection{Support for language option inclusions in config file}
%
%    \begin{macro}{\RSaddto}
% \changes{v0.2}{2003/07/30}{Rename \cmd{\RS@addto} to \cmd{\RSaddto}}
%    Command from the \pkg{varioref} package is used to add
%    language definitions to the |\extras|\meta{language} token
%    for \pkg{babel}.
%    \begin{macrocode}
\def\RSaddto#1#2{%
  #2%
  \@temptokena{#2}%
  \ifx#1\relax
    \let#1\@empty
  \fi
  \ifx#1\undefined
     \edef#1{\the\@temptokena}%
  \else
     \toks@\expandafter{#1}%
     \edef#1{\the\toks@\the\@temptokena}%
  \fi
  \@temptokena{}\toks@\@temptokena}
%    \end{macrocode}
%    \end{macro}
%
%    \begin{macro}{\DeclareLangOpt}
%    \changes{v0.5}{2010/11/02}{Add \cmd{\extras}\meta{language} to beginning of document.}
%    Command to declare a language option and add
%    language definitions to the |\extras|\meta{language} token
%    for \pkg{babel}.
%    \begin{macrocode}
\def\DeclareLangOpt#1#2{%
    \edef\RS@tempa{\expandafter\@gobble\string#2}%
    \RS@ifundefined{\RS@tempa}%
        {\PackageError{\RS@pkgname}{%
            Unknown definitions \@backslashchar\RS@tempa\MessageBreak
            for language option `#1'}{}}%
        {\DeclareOption{#1}{%
            \AtBeginDocument{\expandafter\RSaddto\csname extras#1\endcsname #2}}}%
  }
%    \end{macrocode}
%    \end{macro}
%
%
%
% \subsection{Configuration files}
%
%    \begin{macro}{\RS@cfgfile,\RS@reffile}
%    \changes{v0.4}{2010/10/21}{Ad alternative definition file \cmd{refstyle.def}}
%    Define the config file and alternate definition file names.
%    \begin{macrocode}
\newcommand*{\RS@cfgfile}{refstyle.cfg}
\newcommand*{\RS@reffile}{refstyle.def}
%    \end{macrocode}
%    \end{macro}
%
%    We need to peek into the options list before the
%    options are processed to find out if the config
%    file is to be loaded or not. The config file can
%    contain options and must be loaded before
%    \cmd{\ProcessOptions}. Make \opt{noconfig} not used
%    afterwards.
%
%    We first test for a local config file \file{refstyle.def}
%    and then for the global config file \file{refstyle.cfg}.
%    \begin{macrocode}
\@ifpackagewith{\@currname}{noconfig}%
    {\PackageInfo{\RS@pkgname}{No config file loaded}}%
    {\InputIfFileExists{\RS@reffile}%
        {\PackageInfo{\RS@pkgname}{Local config file \RS@reffile\space used}}
        {\InputIfFileExists{\RS@cfgfile}%
            {\PackageInfo{\RS@pkgname}{Config file \RS@cfgfile\space used}}%
            {\PackageInfo{\RS@pkgname}{No config file found}}}}
%    \end{macrocode}
%
%    \begin{macrocode}
\DeclareOption{noconfig}{\OptionNotUsed}%
%    \end{macrocode}
%
%    Process the options, including options in config file.
%    \begin{macrocode}
\ProcessOptions*\relax
%    \end{macrocode}
%
%    The end of this package.
%    \begin{macrocode}
%</pkg>
%    \end{macrocode}
%
% \Finale
\endinput
