% \iffalse
%<*system>
\begingroup
\input docstrip.tex
\keepsilent
\preamble
________________________________________
A package to draw ECG.
Copyright (C) 2016 Marco Scavino and Ezio Aimé
All rights reserved
License information appended
\endpreamble
\postamble
Copyright 2016

Distributable under the LaTeX Project Public License,
version 1.3c or higher (your choice). The latest version of
this license is at: http://www.latex-project.org/lppl.txt
This work is "author-maintained"
This work consists of this file ECG.dtx,
and the derived files ECG.sty, archivio.tex and ECG.pdf.
\endpostamble
\askforoverwritefalse
%
\generate{
	\file{ecgdraw.sty}{\from{ecgdraw.dtx}{package}}
	\file{archivio.tex}{\from{ecgdraw.dtx}{archivio}}}
%
\def\tmpa{plain}
\ifx\tmpa\fmtname\endgroup\@xp\bye\fi
\endgroup
%</system>
% \fi
%
% \iffalse
%<*package>
\NeedsTeXFormat{LaTeX2e}[2009/01/01]
\ProvidesPackage{ecgdraw}[2016/06/29 v.1.0 Draw ECG]
%</package>
%<*driver>
\documentclass[dvipsnames, a4paper]{ltxdoc}
\usepackage[utf8]{inputenc}
\usepackage[T1]{fontenc}
\usepackage{lmodern,textcomp}
\usepackage[english]{babel}
\usepackage{layaureo}
\usepackage{ecgdraw}
\usepackage[colorlinks,linkcolor=MidnightBlue]{hyperref}
\usepackage{listings}
\usepackage{hyperref}
\hypersetup{linkcolor=red}
\def\TikZ{\textsf{TikZ}}
\def\env#1{\texttt{#1}}
\def\pkg#1{\textsf{#1}}
\RecordChanges
\CodelineIndex\EnableCrossrefs
\begin{document}
\GetFileInfo{ecgdraw.sty}
\title{\textsf{ecgdraw} package\thanks{Version \fileversion; last revision \filedate.}}
\author{Marco Scavino\thanks{e-mail: \texttt{scavino dot marco93 at gmail dot com}} and Ezio Aimé\thanks{e-mail: \texttt{ezio dot aime at fastwebnet dot it}}}
\maketitle
\DocInput{ecgdraw.dtx}
\end{document}
%</driver>
% \fi
%
%
% \changes{v0.1}{2015/09/04}{Initial version}
% \changes{v0.1}{2015/12/05}{New keys |ECG title| and |ECG title align|}
%
% \begin{abstract}
% This package was born to create fake elettrocardiograms, thanks to \TikZ package and \LaTeX3 bundle.
% \end{abstract}
% \tableofcontents
% \section{Introduction}
% To work correctly \textsf{ecgdraw} package are needed
% \begin{itemize}
% 	\item \TikZ package,
% 	\item \LaTeX3 bundle.
% \end{itemize}
%
%
% \section{Use}
%
% \begin{macro}{ecg}
% The package defined |ecg| environment with a optional argument \oarg{options}
% \begin{center}
% |\begin{ecg}| \oarg{options}
% ECG path
% |\end{ecg}|
% \end{center}
% where \meta{options} are \pkg{TikZ} keys. Inside the environment it's possible to draw a ECG thanks the |\ECG| macro
% \begin{center}
% |\ECG| \oarg{TikZ options} \parg{vertical position} \marg{ECG waves}
% \end{center}
% \end{macro}
% The macro have an optional argument \oarg{options}, that accepts Tikz options, an optional argument delimeted by brace \parg{vertical position}, vertical position of the path, and a mandatory argument \marg{ECG waves} which contain the list of ECG waves abbreviation.
%
% Each abbreviation is made of different part:
% \begin{center}
% |\ECG {| \oarg{options} \marg{wave name} \meta{other} |}|
% \end{center}
% \meta{options} aregiven to the single wave, \meta{wave name} is the abbreviation of the wave, while \meta{other} depends on the types of wave.
%
% \subsection{Waves}
% Different wave types are possible:
% \begin{macro}{p}
% |p| wave needs \meta{polarity} (allowed value |p|, |n|), wave height \meta{tenths of millivolts} (between 0.1-0.3 mV) and time \meta{milliseconds}.
% \begin{center}
% |p| \meta{polarity} |0| \meta{tenths of millivolts} \meta{milliseconds}
% \end{center}
% Bifasich wave with |d| and |b| polarity is needed a second wave
% \begin{center}
% |p| \meta{polarity} \meta{first tenths millivolts} \meta{second tenths millivolts} \meta{milliseconds}
% \end{center}
% \end{macro}
% \begin{macro}{q,r,s}
% Waves for QRS complex. Thei take as first argument wave height in millivolts and as second argument the duration in milliseconds.
% \begin{center}
% |q/r/s| \marg{wave height Q/R/S} \meta{milliseconds}
% \end{center}
% \end{macro}
% \begin{macro}{i}
% Isoelettric wave, take only one argument, which is time in \meta{milliseconds}.
%\begin{center}
%	|i| \meta{milliseconds}
%\end{center}
% \end{macro}
% \begin{macro}{t}
% First argument is \meta{polarity}, positive |p| or negative |n|, second argument is \meta{tenth of milliVolts}, as optional argument a correction if wave isn't simmetrical and last argument \meta{milliseconds}.
%\begin{center}
%	|t| \meta{polarity} \meta{tenth of milliVolts} \oarg{correction} \meta{milliseconds}
%\end{center}
% \end{macro}
% \begin{macro}{!}
% Allow to use a wave defined through |\newECG| macro.
%\begin{center}
%	|!| \meta{wave name}
%\end{center}
% \end{macro}
% \begin{macro}{?}
% Insert a label left to the path. Optional argument (default value |1 cm|) set horizontal shift.
%\begin{center}
%	|?| \oarg{horizontal shift} \meta{text}
%\end{center}
% \end{macro}
% \subsection{Options}
% \subsubsection{Grid}
% |ecg| environment accept different options to modify grid dimension.
% \begin{macro}{grid top}
% Accept a dimension as value. Grid is elarged toward top of the set value.
% \end{macro}
% \begin{macro}{grid bottom}
% Similar to |grid top|, but grid is enlarged toward bottom.
% \end{macro}
% \begin{macro}{grid left}
% Similar to |grid top|, but grid is enlarged toward left.
% \end{macro}
% \begin{macro}{grid right}
% Similar to |grid top|, but grid is enlarged toward right.
% \end{macro}
% \begin{macro}{grid border}
% Set bottom, top, left and right with the same \meta{value}.
% \end{macro}
% \subsection{Break ECG path}
% Sometimes ECG are too much wide and cannot fit the textwidth. So it's possible to allow \LaTeX{} to break ECG using the |breaklines| key.
% \begin{macro}{breaklines}\begin{macro}{breakindent}
% This key allow automate wrap ECG pattern. New line has an indent of |breakindent| value (default 1~cm).
% \end{macro}\end{macro}
% \subsection{ECG title}
% \begin{macro}{ECG title}\begin{macro}{ECG title align}
% I's possible to insert a ECG title by |ECG title| and set title align by |ECG title align| key (value |righ|, |left|, |center|).
% \end{macro}\end{macro}
% \subsection{Wave database}
% |\newECG| macro add a custom wave
% \begin{center}
% |\newECG| \marg{wave name} \marg{wave code}
% \end{center}
% It's possible to call \meta{wave name} inside |\ECG| using the key |!|.
% \StopEventually{}
%
% \iffalse
% \section{ECG implementation}
%
%<*package>
% Prima carichiamo i pacchetti \TikZ{} e \textsf{xparse}.
%    \begin{macrocode}
\RequirePackage{tikz}
\RequirePackage{xparse}

\usetikzlibrary{calc}
%    \end{macrocode}

% Questo è necessario per definire un nuovo livello. \pkg{pgf} può gestire più livelli, e questo risulta comodo per sovrappore oggetti collocandoli su livelli diversi. Nel caso definisco il livello |sk@back| e lo colloco dietro a quello principale (qui indicato da |main|).
%    \begin{macrocode}
\pgfdeclarelayer{sk@back}
\pgfsetlayers{sk@back,main}
\ExplSyntaxOn
%    \end{macrocode}

% Definisco alcune chiavi da passare all'ambiente \env{ecg}. Nello specifico permettono di gestire quanto la griglia sottostante sborda rispetto ai tracciati. grid border imposta lo stesso valore a tutti i lati (di default .2~cm) mentre le altre chiavi impostano lo spessore del rispettivo lato. Il booleano |\sk_ECG_breaklines_bool| serve per gestire i tracciati, abilitando o disabilitando la cesura automatica a fine riga.
%    \begin{macrocode}
\bool_new:N \sk_ECG_breaklines_bool
\bool_new:N \sk_ECG_breaklines_arrow_bool

\tikzset {
  breakindent / .store ~ in = \sk_ECG_breakindent_tl ,
  breakindent = 1cm,
  breaklines / .is ~ choice,
  breaklines / true / .code = \bool_set_true:N \sk_ECG_breaklines_bool ,
  breaklines / false / .code = \bool_set_false:N \sk_ECG_breaklines_bool ,
  breaklines / arrow / .code =
  	\bool_set_true:N \sk_ECG_breaklines_bool 
  	\bool_set_true:N \sk_ECG_breaklines_arrow_bool ,
  breaklines / no ~ indent /.code =
  	\bool_set_true:N \sk_ECG_breaklines_bool
  	\tl_set:Nn \sk_ECG_breakindent_tl { 0 } ,
  breaklines / .default = true,
  breaklines = false,
  ECG ~ title / .store ~ in = \sk_ECG_title_tl,
  ECG ~ title  = {},
  ECG ~ title ~ align / .is~choice ,
  ECG ~ title ~ align / center / .code = \tl_set:Nn \sk_ECG_title_align_tl { north } \tl_set:Nn \sk_ECG_title_anchor_tl { south } ,
  ECG ~ title ~ align / right / .code = \tl_set:Nn \sk_ECG_title_align_tl { north ~ east } \tl_set:Nn \sk_ECG_title_anchor_tl { south ~ east } ,
  ECG ~ title ~ align / left / .code = \tl_set:Nn \sk_ECG_title_align_tl { north ~ west } \tl_set:Nn \sk_ECG_title_anchor_tl { south ~ west } ,
  ECG ~ title ~ align = left ,
  grid ~ border / .code =
    \tl_set:Nn \sk_ECG_gridtp_tl { #1 }
    \tl_set:Nn \sk_ECG_gridlf_tl { #1 }
    \tl_set:Nn \sk_ECG_gridrg_tl { #1 }
    \tl_set:Nn \sk_ECG_gridbt_tl { #1 },
  grid ~ top / .store ~ in = \sk_ECG_gridtp_tl,
  grid ~ bottom / .store ~ in = \sk_ECG_gridbt_tl,
  grid ~ left / .store ~ in = \sk_ECG_gridlf_tl,
  grid ~ right / .store ~ in = \sk_ECG_gridrg_tl,
  grid ~ border = .2cm,
}
%    \end{macrocode}

% Nuovo booleano che permette di comprendere se siamo alla prima onda o no. Di base impostato su falso.
%    \begin{macrocode}
\bool_new:N \sk_ECG_start_bool
%    \end{macrocode}

% Viene definito il comando |\ECG| che prende due argomenti: uno opzionale |O{}| (che di default non ha nulla) e uno obbligatorio m. L'opzionale rappresenta le opzioni di \TikZ passate poi al comando |\draw[]| (come i comandi per il colore). L'obbligatorio invece è la lista di onde che va poi processata
%    \begin{macrocode}
\DeclareDocumentCommand \ECG { O{} m }
  {
  \begin{tikzpicture}
  \coordinate(sk_end) at (0,0);
  \sk_ECG:nn { #1 } { #2 }
  \end{tikzpicture}
  }
%    \end{macrocode}

% Sempre analogo del comando |\ECG|, ma pensato per funzionare dentro all'ambiente ecg. Questa versioni infatti non apre ne chiude l'ambiente tikzpicture. Inoltre, oltre ai due precedenti comandi, si introduce il comando facoltativo racchiuso tra parentesi tonde, che permette di collocare il traccaito all'altezza voluta. Il sistema con |\sk_ECG_yshift_dim| toglie per ogni comando |\ECG| 2.5~cm in modo che il successivo comando (se non viene specificato un valore tra parentesi) sia collocato sotto il precedente.
%    \begin{macrocode}
\cs_set_nopar:Npn \sk_ECG_env:nnn #1 #2 #3
{
  \tl_if_empty:nTF { #2 }
      { \coordinate(sk_end) at (0,\sk_ECG_yshift_dim);  }
      { \coordinate(sk_end) at (#2); }
  \sk_ECG:nn { #1 } { #3 }
  \dim_set:Nn \sk_ECG_yshift_dim { \sk_ECG_yshift_dim - 2.5cm }
}
%    \end{macrocode}

% Definisco la dimensione per lo spostamento in verticale dei vari tracciati. Inoltre definisco due contatori per la numerazione di tutte le onde, uno per il tracciato e uno per l'onda.
%    \begin{macrocode}
\dim_new:N \sk_ECG_yshift_dim
\dim_new:N \sk_ECG_x_dim
\int_new:N \sk_ECG_row_int
\int_new:N \sk_ECG_wave_int
%    \end{macrocode}

% Ambiente \env{ecg}. Il suo scopo è permettere di disegnare più tracciati, e collocarci dietro una griglia millimetrata. Inanzitutto azzero il valore dell'altezza per partire nuovamente da zero (non sarebbe necessario, ma per scrupolo è meglio azzerare comunque). Imposto poi che il comando |\ECG| accetti un argomento opzionale tra parentesi quadre, uno opzionale tra tonde e uno obbligatorio. Passo tutto a |\sk_ECG_env:nnn|. Apro l'ambiente \env{tikzpicture}. La versione asteriscata non utilizza l'ambiente \env{center}.
%    \begin{macrocode}
\DeclareDocumentEnvironment { ecg } { O{} }
{
  \int_zero:N \sk_ECG_row_int
  \int_zero:N \sk_ECG_wave_int
  \dim_zero:N \sk_ECG_yshift_dim
  \DeclareDocumentCommand \ECG { O{} D(){} m }
  {
    \dim_zero:N \sk_ECG_x_dim
    \int_incr:N \sk_ECG_row_int
    \sk_ECG_env:nnn { ##1 } { ##2 } { ##3 }
  }
  \begin{center}
  \begin{tikzpicture}[#1]
}
%    \end{macrocode}

% L'ambiente \env{pgfonlayer} permette di selezionare su quale livello porre la nostra griglia millimetrata. Affinché si adatti automaticamente ai tracciati sfrutto un nodo di \pkg{TikZ} speciale: current bounding box, le cui dimensioni sono proprio le attuali dimensioni del disegno \pkg{TikZ}. Sfrutto quindi due ancore del nodo |current bounding box|: quella in altro a destra (ancora |north east|) e in basso a sinistra (ancora |south west|). Le chiavi |xshift| e |yshift| permettono di traslare il punto dei valori passati (quelli impostati con le varie chiavi |border|). In tal modo sono definiti due nodi |sk_start| e |sk_stop| che definiscono i vertici in alto a destra e in basso a sinistra della griglia, che sarà semplicemente disegnata con grid e i settaggi desiderati (colore e step). Il tutto, grazie all'ambiente \env{pfgonlayer}, posto automaticamente sullo sfondo.
%    \begin{macrocode}
{
  \tl_if_empty:NF \sk_ECG_title_tl
      { \node at (current ~ bounding ~ box.\sk_ECG_title_align_tl) [inner ~ sep=\c_zero_dim,anchor=\sk_ECG_title_anchor_tl] {\sk_ECG_title_tl}; }
  \begin{pgfonlayer}{sk@back}
    \coordinate(grid_start) at ([xshift=-\sk_ECG_gridlf_tl,
      yshift=-\sk_ECG_gridbt_tl] current ~ bounding ~ box. south ~ west);
    \coordinate(grid_stop) at ([xshift=\sk_ECG_gridrg_tl,
      yshift=\sk_ECG_gridtp_tl] current ~ bounding ~ box.north ~ east);
    \draw[color=red!20,step=0.1cm,very ~ thin]
      (grid_start) grid (grid_stop);
    \draw [color=red!50,step=.5cm](grid_start)grid(grid_stop);
    \end{pgfonlayer}
  \end{tikzpicture}
  \end{center}
}

% L'ambiente \env{ecg*} è definito in modo simile a \env{ecg}, ma non allinea al centro il grafico.
%    \begin{macrocode}
\DeclareDocumentEnvironment { ecg* } { O{} }
{
  \int_zero:N \sk:ECG_row_int
  \int_zero:N \sk:ECG_wave_int
  \dim_zero:N \sk_ECG_yshift_dim
  \DeclareDocumentCommand \ECG { O{} D(){} m }
  {
    \dim_zero:N \sk_ECG_x_dim
    \int_incr:N \sk_ECG_row_int
    \sk_ECG_env:nnn { ##1 } { ##2 } { ##3 }
  }
  \begin{tikzpicture}[#1]
}
{
  \tl_if_empty:NF \sk_ECG_title_tl
    { \node at (current ~ bounding ~ box.\sk_ECG_title_align_tl) {\sk_ECG_title_tl}; }
  \begin{pgfonlayer}{sk@back}
    \coordinate(grid_start) at ([xshift=-\sk_ECG_gridlf_tl,
      yshift=-\sk_ECG_gridbt_tl] current ~ bounding ~ box. south ~ west);
    \coordinate(grid_stop) at ([xshift=\sk_ECG_gridrg_tl,
      yshift=\sk_ECG_gridtp_tl] current ~ bounding ~ box.north ~ east);
    \draw[color=red!20,step=0.1cm,very ~ thin]
      (grid_start) grid (grid_stop);
    \draw [color=red!50,step=.5cm](grid_start)grid(grid_stop);
    \end{pgfonlayer}
  \end{tikzpicture}
}
%    \end{macrocode}

% Definisce una nuova sequence. È un costrutto speciale di \LaTeX3 che permette di gestire dei dati.
%    \begin{macrocode}
\seq_new:N \sk_ECG_seq
%    \end{macrocode}

% Dopo aver svuotato la token listi delle onde e la sequence (e settato a vero il booleano di onda iniziale) si suddivide la lista di onde rispetto alle virgole (il secondo argomento di |\seq_set_split:Nnn|) e si salva nella sequence. Il comando |\seq_map_inline:Nn| passa poi ogni singolo elemento della sequence al suo argomento come |##1| (l'argomento |#1| sono le opzioni \env{TikZ}).
%    \begin{macrocode}
\cs_set_nopar:Npn \sk_ECG:nn #1 #2
{
  \tl_clear:N \sk_ECG_onda_tl
  \seq_clear:N \sk_ECG_seq
  \bool_set_true:N \sk_ECG_start_bool
  \seq_set_split:Nnn \sk_ECG_seq { , } { #2 }
  \seq_map_inline:Nn \sk_ECG_seq { \sk_ECG_use:w ##1 \q_nil { #1 } }
}
%    \end{macrocode}

% Macro che processa ogni singola onda: prende 4 argomenti.
% \begin{enumerate}
% \item Il primo opzionale sono eventuali opzioni per una singola onda (pensato principalemente per colorare una singola onda, magari per evidenziarla). Viene passato insieme a |#4| (che rappresenta le opzioni globali del tracciato), alle opzioni di |\draw|, che è salvato nella token list |\sk_ECG_onda_tl|. Il comando |\draw| è impostato con le chiavi osservabili. Lo shift pone il punto zero nell'ultimo node |sk_end|. Al termine, prima di espandere la token list (e rendere effettivo il disegno \pkg{TikZ}), si aggiorna la posizione del nodo |sk_end| e si chiude |\draw| con il punto e virgola necessario.
% \item Il secondo è la prima lettera della sigla. Viene passato al comando |\str_case:nn| che lo confronta con tutte quelle presenti e usa quella che coincide con il codice corrente.
% \item Il terzo è un argomento delimitato, ossia anche se non sono presenti tonde, il comando raccoglie tutto ciò che trova fino al delimitatore (in questo caso il mark |\q_nil|). Permette di scrivere indipendentemente costrutti come |!{nome}| o |!nome| e ottenere il medesimo risultato. Il comando viene poi passato alle macro interne selezionate dalla prima lettera dell'onda.
% \end{enumerate}
%    \begin{macrocode}
\DeclareDocumentCommand \sk_ECG_use:w { O{} m u\q_nil m }
{
  \int_incr:N \sk_ECG_wave_int
  \tl_set:Nn \sk_ECG_onda_tl
    { \draw[thick,rounded ~ corners=0.25mm,line ~ cap=round, shift=(sk_end),
    #4,#1] }
  \str_case:nn { #2 }
    {
      { ! } { \tl_put_right:Nx \sk_ECG_onda_tl
        {
          \cs_if_exist_use:cF { sk_ECG_ #3 } { (0,0) }
        } }
      { ? } { \sk_ECG_onda_label:w #3 \q_nil }
      { i } { \sk_ECG_onda_iso:w #3 \q_nil }
      { p } { \sk_ECG_onda_p:nnnw #3 \q_nil }
      { q } { \sk_ECG_onda_q:nw #3 \q_nil }
      { r } { \sk_ECG_onda_r:nw #3 \q_nil }
      { s } { \sk_ECG_onda_s:nw #3 \q_nil }
      { t } { \sk_ECG_onda_t:nw #3 \q_nil }
      { T } { \tl_put_right:Nx \sk_ECG_onda_tl
        { (0,0) -- (.1cm,0) -- (.1cm,#3*1cm) -- (.5cm,#3*1cm) -- (.5,0)
        -- (1,0) } }
      { * } { \int_decr:N \sk_ECG_wave_int \sk_ECG_onda_ripeti:nw #3 \q_nil { #4 }  \tracingmacros=0}
    }
  \tl_put_right:Nn \sk_ECG_onda_tl { coordinate (sk_end)
    let \p1=(current ~ path ~ bounding ~ box.south ~ west),
    \p2=(current ~ path ~ bounding ~ box.north ~ east)
    in ~
    node (ecg-\int_use:N \sk_ECG_row_int-\int_use:N \sk_ECG_wave_int)
       [at=(current ~ path ~ bounding ~ box),text ~ width=\x2-\x1,text ~ height=\y2-\y1]{}
   ; }
  
  \str_if_eq:nnF { #2 } { * } { \sk_ECG_onda_tl }
  \bool_if:NT \sk_ECG_breaklines_bool
    { \pgfgetlastxy{\sk_ECG_lastx_tl}{\sk_ECG_lasty_tl}
  \dim_set:Nn \l_tmpa_dim
    { \textwidth - \sk_ECG_gridlf_tl - \sk_ECG_gridrg_tl - \sk_ECG_lastx_tl }
  \dim_compare:nT { \l_tmpa_dim < 1.5cm }
    {
      \dim_zero:N \sk_ECG_x_dim
      \dim_set:Nn \sk_ECG_yshift_dim { \sk_ECG_yshift_dim - 2.5cm }
      \coordinate(sk_end) at (\sk_ECG_breakindent_tl,\sk_ECG_yshift_dim);
      \bool_if:NT \sk_ECG_breaklines_arrow_bool {
      \draw[thick,<-,rounded ~ corners]([xshift=-.25cm]sk_end) -| ++ (-.5cm,1.5cm); }
    } }
  \bool_if:NT \sk_ECG_start_bool { \bool_set_false:N \sk_ECG_start_bool }
}
%    \end{macrocode}

% Nel caso in cui il primo elemento sia |*|, vengono poi cercati due argomenti: il primo è il numero di ripetizioni, il secondo la lista di onde. Il comando |\prg_replicate:nn| si occupa di ripetere per |#1| volte il comando |\sk_ECG:nn{#3}{#2}|, dove |#3| sono le opioni globali del tracciato, e |#2| la lista di onde da ripetere
%    \begin{macrocode}
\cs_set_nopar:Npn \sk_ECG_onda_ripeti:nw #1 #2 \q_nil #3
{
\group_begin:
  \prg_replicate:nn { #1 } { \sk_ECG:nn { #3 } { #2 } }
\group_end:
}
%    \end{macrocode}

% Se il primo elemento è |?|, si cerca un argomento opzionale (di default posto a .5~cm) e poi uno delimitato (il testo dell'etichetta). L'etichetta viene quindi salvata nella token list e posta al punto |(#1,0)| e l'allineamento del nodo deciso in base al booleano |\sk_ECG_start_bool|: se infatti è vero (siamo all'inizio del tracciato) l'etichetta è allineata a sinistra, altrimenti a destra.
%
%|#2| Decide quale derivazione usare in base alla sigla e nell'eventualità non sia prevista, viene stampato come appare. L'utilizzo dei nodi |sk_deriv| serve per posizionare correttamente il tracciato successivo e impedire la sovrapposizione di questo all'etichetta. Inoltre esso permette lo spostamento automatico del tracciato  (tiene conto infatti della larghezza dell'etichetta).
%    \begin{macrocode}
\DeclareDocumentCommand \sk_ECG_onda_label:w { O{.5cm} u\q_nil }
{
  \tl_put_right:Nx \sk_ECG_onda_tl { (#1,0) node [
  \bool_if:NTF \sk_ECG_start_bool
   { left }
   { right }
  ,name=sk_deriv] 
  { \str_case:nnF { #2 }
  {
   { d1 } { I }
   { d2 } { II }
   { d3 } { III }
   { vr } { aVR }
   { vl } { aVL }
   { vf } { aVF }
   { v1 } { V1 }
   { v2 } { V2 }
   { v3 } { V3 }
   { v4 } { V4 }
   { v5 } { V5 }
   { v6 } { V6 }
  } { #2 } } ; \exp_not:N \path(sk_deriv.east) }
}
%    \end{macrocode}

% Macro associata alla lettera |i|. L'argomento |#1| viene controllato (se uguale a |k| viene sostituito da |1000|) e poi usato per la lunghezza del tratto.
%    \begin{macrocode}
\cs_set_nopar:Npn \sk_ECG_onda_iso:w #1 \q_nil
{
  \tl_put_right:Nx \sk_ECG_onda_tl
  { (0,0) -- ( 0.025 mm *
    \str_if_eq:nnTF { k } { #1 }
      { 1000 }
      { #1 }
   ,0) }
}
%    \end{macrocode}

% Onda Q, il primo argomento è l'altezza, il secondo la durata. Sono moltiplicati per i fattori correttivi affinché corrispondano con i millisecondi e i millimetri.
%    \begin{macrocode}
\cs_set_nopar:Npn \sk_ECG_onda_q:nw #1 #2 \q_nil
{
  \tl_put_right:Nn \sk_ECG_onda_tl {
  (0,0)--(#1*.0125cm,-#2*.1cm)--(#1*.025cm,0) }
}
%    \end{macrocode}

% Onda R: come l'onda Q.
%    \begin{macrocode}
\cs_set_nopar:Npn \sk_ECG_onda_r:nw #1 #2 \q_nil
{
  \tl_put_right:Nn \sk_ECG_onda_tl
    { (0,0) -- (#1*.0125cm,#2*.1) -- (#1*.025cm,0) }
}
%    \end{macrocode}

% Onda S: come l'onda Q.
%    \begin{macrocode}
\cs_set_nopar:Npn \sk_ECG_onda_s:nw #1 #2 \q_nil
{
  \tl_put_right:Nn \sk_ECG_onda_tl
    { (0,0)--(#1*0.0125cm-0.005cm,-#2*.1cm)--(#1*0.025cm,0) }
}
%    \end{macrocode}

% Onda T. |#1| indica se è positiva o negativa (dobbiamo aggiungere la difasica). |#2| sono i millisecondi e |#3| i millimetri.
%    \begin{macrocode}
\cs_set_nopar:Npn \sk_ECG_onda_t:nw #1 #2 #3\q_nil
{
  \str_case:nn { #1 }
  {
    { p }
    { \tl_put_right:Nn \sk_ECG_onda_tl
      { (0,0) .. controls (#2*.0625cm,.05cm) and (#2*.075cm,#3*.065cm)
      .. (#2*.112cm,#3*.1cm) -- (#2*.138cm,#3*.09cm) .. controls
      (#2*.175cm,#3*.065cm) and (#2*.188cm,.05) .. (#2*.25cm,0) }
    }
    { n }
    { \tl_put_right:Nn \sk_ECG_onda_tl
      { (0,0) .. controls (#2*.0625cm,-.05cm) and (#2*.075cm,-#3*.065cm) ..
      (#2*.112cm,-#3*.1cm) -- (#2*.138cm,-#3*.09cm) ..
      controls (#2*.175cm,-#3*.065cm) and (#2*.188cm,-.05) .. (#2*.25cm,0) }
    }
  }
}
%    \end{macrocode}

% Onda P. In modo simile all'onda T, |#1| distingue tra le varie |p|, |n|, |d| e |b|. |#2| e |#3| sono le altezze dei due picchi, |#4| la durata in millisecondi.
%    \begin{macrocode}
\cs_set_nopar:Npn \sk_ECG_onda_p:nnnw #1 #2 #3 #4 \q_nil
{
  \tl_put_right:Nx \sk_ECG_onda_tl { \str_case:nn { #1 }
  {
    { p } { (0,0) -- (.0005*#4,#2#3*.06) -- (.0011*#4,#2#3*.1) --
    (.002*#4,#2#3*.06) -- (.0025*#4,.0) }
    { n } { (0,0) -- (.0005*#4,-#2#3*.06) -- (.0011*#4,-#2#3*.1) --
    (.002*#4,-#2#3*.06) -- (.0025*#4,.0) }
    { d } { (0,0) -- (.01cm,.005cm)
    --  (#4*0.3333-.01cm,#2*.1cm-.008cm)
    -- (#4*0.3333+.01cm,#2*.1cm-.008cm)
    --  (#4*0.6667-.01cm,-#3*.1cm+.008cm)
    -- (#4*0.6667+.01cm,-#3*.1cm+.008cm)
    -- (#4-.01cm,.005cm)
    -- (#4,0) }
    { b } { (0,0) -- (.01cm,.005cm)
    --  (#4*1/3-.01cm,#2*0.045cm+.105cm)
    -- (#4*1/3+.01cm,#2*0.045cm+.105cm)
    -- (#4*1/2-.01cm,.1cm)
    -- (#4*1/2+.01cm,.1cm)
    -- (#4*2/3-.01cm,#3*0.045cm+.105cm)
    -- (#4*2/3+.01cm,#3*0.045cm+.105cm)
    -- (#4-.01cm,.005cm)
    -- (#4,0) }
  } }
}
%    \end{macrocode}

% Il comando |\msg_new:nnn| definisce un messaggio da stampare nel log. Nel caso specifico serve per avertire che la sigla scelta è già utilizzata per un altra onda.
%    \begin{macrocode}
\msg_new:nnn { ECG } { wave ~ defined }
  { wave ~ "#1" ~ already ~ defined. ~ Change ~ name. }
%    \end{macrocode}

% Definisco una nuova onda con sigla |#1| e codice |#3| (il |#2| è codice opzionale, lo messo nel caso volessimo aggiungere funzionalità a questo comando). Nel caso di onda già definita stampa un avvertimento e non rinomina la precendete.
%    \begin{macrocode}
\cs_set_nopar:Npn \sk_ECG_new:nn #1 #2 #3
{
  \cs_if_exist:cTF { sk_ECG_ #1 }
    { \msg_warning:nnn { ECG } { wave ~ defined } { #1 } }
    { \cs_set_nopar:cpn { sk_ECG_ #1 } { #3 } }
}
%    \end{macrocode}

% Interfaccia con l'utente per definire nuove onde nel database. |#1| è obbligatorio e contiene il nome dell'onda, |#2| opzionale (come scritto sopra in previsione di futuri sviluppi) e |#3| è il codice \pkg{TikZ}.
%    \begin{macrocode}
\DeclareDocumentCommand \nuovoECG { m O{} m }
{
  \sk_ECG_new:nn { #1 } { #2 } { #3 }
}
%    \end{macrocode}

% Messaggio nel caso di archivio non trovato
%    \begin{macrocode}
\msg_new:nnn { ECG } { fnotf }
  { File ~ “#1” ~ not ~ found. ~ Please ~ check ~ your ~ installation.}
%    \end{macrocode}

% Include l'achivio delle onde se presente, altrimente avverte l'utente che il file è assente.
%    \begin{macrocode}
\file_if_exist:nTF { archivio }
  { \file_input:n { archivio } }
  { \msg_warning:nnn { ECG } { fnotf } { archivio } }

\ExplSyntaxOff
%    \end{macrocode}
%
%</package>
%
%
% \section{Archivio delle onde}
%
%<*archivio>
%
%    \begin{macrocode}
\ProvidesFile{archivio}[2015/08/31 v.1.0 Draw ECG]
%    \end{macrocode}
% Etichette delle derivazioni
%    \begin{macrocode}
\nuovoECG{d1}{(1cm,0)node[left]         {I}} 
\nuovoECG{d2}{(1cm,0)node[left]{II}} 
\nuovoECG{d3}{(1cm,0)node[left]{III}} 
\nuovoECG{vr}{(1cm,0)node[left]{aVR}} 
\nuovoECG{vl}{(1cm,0)node[left]{aVL}} 
\nuovoECG{vf}{(1cm,0)node[left]{aVF}} 
\nuovoECG{v1}{(1cm,0)node[left]{V1}} 
\nuovoECG{v2}{(1cm,0)node[left]{V2}} 
\nuovoECG{v3}{(1cm,0)node[left]{V3}} 
\nuovoECG{v4}{(1cm,0)node[left]{V4}} 
\nuovoECG{v5}{(1cm,0)node[left]{V5}} 
\nuovoECG{v6}{(1cm,0)node[left]{V6}} 
%    \end{macrocode}

% Comando per la tara.
%    \begin{macrocode}
\nuovoECG{tara}{(0,0)--(.1,0)--(.1,1)--(.5,1)--(.5,0)--(1,0)}
%    \end{macrocode}

% Onde P positive (pp) di ampiezza da 0.1 a 0.3~mVolt x 60, 80 e 100 msec
%    \begin{macrocode}
\nuovoECG{pp0160}{(0,0)--(.04,.06)--(.085,.1)--(.12,.06)--(.15,.0)}
\nuovoECG{pp0260}{(0,0)--(.02,.1)--(.085,.2)--(.13,.1)--(.15,.0)}
\nuovoECG{pp0360}{(0,0)--(.04,.2)--(.085,.3)--(.12,.2)--(.15,.0)}
\nuovoECG{pp0180}{(0,0)--(.04,.06)--(.09,.1)--(.14,.06)--(.2,.0)}
\nuovoECG{pp0280}{(0,0)--(.02,.1)--(.09,.2)--(.15,.1)--(.2,.0)}
\nuovoECG{pp0380}{(0,0)--(.04,.2)--(.09,.3)--(.14,.2)--(.2,.0)}
\nuovoECG{pp0199}{(0,0)--(.1,.08)--(.15,.1)--(.2,.08)--(.25,.0)}
\nuovoECG{pp0299}{(0,0)--(.1,.16)--(.15,.2)--(.2,.16)--(.25,.0)}
\nuovoECG{pp0399}{(0,0)--(.1,.2)--(.15,.3)--(.2,.2)--(.25,.0)}
%    \end{macrocode}

% onde P negative (pn) di ampiezza da 0.1 a 0.3 mVolt x 60, 80 e 100 msec
%    \begin{macrocode}
\nuovoECG{pn0160}{(0,0)--(.04,-.06)--(.085,-.1)--(.12,-.06)--(.15,.0)}
\nuovoECG{pn0260}{(0,0)--(.02,-.1)--(.085,-.2)--(.13,-.1)--(.15,.0)}
\nuovoECG{pn0360}{(0,0)--(.04,-.2)--(.085,-.3)--(.12,-.2)--(.15,.0)}
\nuovoECG{pn0180}{(0,0)--(.04,-.06)--(.09,-.1)--(.14,-.06)--(.2,.0)}
\nuovoECG{pn0280}{(0,0)--(.02,-.1)--(.09,-.2)--(.15,-.1)--(.2,.0)}
\nuovoECG{pn0380}{(0,0)--(.04,-.2)--(.09,-.3)--(.14,-.2)--(.2,.0)}
\nuovoECG{pn0199}{(0,0)--(.1,-.08)--(.15,-.1)--(.2,-.08)--(.25,.0)}
\nuovoECG{pn0299}{(0,0)--(.1,-.16)--(.15,-.2)--(.2,-.16)--(.25,.0)}
\nuovoECG{pn0399}{(0,0)--(.1,-.2)--(.15,-.3)--(.2,-.2)--(.25,.0)}
%    \end{macrocode}

% onda P difasica (pd) e bifida (pb) da 100 msec
%    \begin{macrocode}
\nuovoECG{pd1199}{(0,0)--(0.01,0.005)--(0.09,0.095)--(0.11,0.095)
 --(0.19,-0.095)--(0.21,-0.095)--(0.29,-0.005)--(0.3,0)}
\nuovoECG{pb2199}{(0,0)--(0.01,0.005)--(0.09,0.195)--(0.11,0.195)--(0.14,0.1)
 --(0.16,0.1)--(0.19,0.15)--(0.21,0.15)--(0.29,0.005)-- (0.3,0)}
\nuovoECG{pb1299}{(0,0)--(0.01,0.005)--(0.09,0.15)--(0.11,0.15)--(0.14,0.1)
 --(0.16,0.1)--(0.19,0.195)--(0.21,0.195)--(0.29,0.005)-- (0.3,0)}
%    \end{macrocode}

% Onda Q da 10 e 20 msec con profondità rispettivamente da 1 a 3~mV e da 1 a 5~mm.
%    \begin{macrocode}
\nuovoECG{q21}{(0,0)--(.025,-.1)--(.05,0)}
\nuovoECG{q22}{(0,0)--(.025,-.2)--(.05,0)}
\nuovoECG{q23}{(0,0)--(.025,-.3)--(.05,0)}

\nuovoECG{q41}{(0,0)--(.05,-.1)--(.1,0)}
\nuovoECG{q42}{(0,0)--(.05,-.2)--(.1,0)}
\nuovoECG{q43}{(0,0)--(.05,-.3)--(.1,0)}
\nuovoECG{q44}{(0,0)--(.05,-.4)--(.1,0)}
\nuovoECG{q45}{(0,0)--(.05,-.5)--(.1,0)}
%    \end{macrocode}

% onda R da 30 msec alta da 1 a 5 mm
%    \begin{macrocode}
\nuovoECG{r31}{(0,0)--(.035,.1)--(.075,0)}
\nuovoECG{r32}{(0,0)--(.035,.2)--(.075,0)}
\nuovoECG{r33}{(0,0)--(.035,.3)--(.075,0)}
\nuovoECG{r34}{(0,0)--(.035,.4)--(.075,0)}
\nuovoECG{r35}{(0,0)--(.035,.5)--(.075,0)}
%    \end{macrocode}

% onda R da 40 msec con altezze da 2 a 10 mm in step di 2 mm
%    \begin{macrocode}
\nuovoECG{r42}{(0,0)--(.05,.2)--(.1,0)}
\nuovoECG{r44}{(0,0)--(.05,.4)--(.1,0)}
\nuovoECG{r46}{(0,0)--(.05,.6)--(.1,0)}
\nuovoECG{r48}{(0,0)--(.05,.8)--(.1,0)}
\nuovoECG{r100}{(0,0)--(.05,1)--(.1,0)}
%    \end{macrocode}

% onda R da 40 msec con altezze da 12 a 32 mm in step di 4 mm
%    \begin{macrocode}
\nuovoECG{r412}{(0,0)--(.05,1.2)--(.1,0)}
\nuovoECG{r416}{(0,0)--(.05,1.6)--(.1,0)}
\nuovoECG{r420}{(0,0)--(.05,2)--(.1,0)}
\nuovoECG{r424}{(0,0)--(.05,2.4)--(.1,0)}
\nuovoECG{r428}{(0,0)--(.05,2.8)--(.1,0)}
\nuovoECG{r432}{(0,0)--(.05,3.2)--(.1,0)}
%    \end{macrocode}

% onda R da 50 msec con altezze da 10 a 45 mm in step di 5 mm
%    \begin{macrocode}
\nuovoECG{r510}{(0,0)--(.0625,1)--(.125,0)}
\nuovoECG{r515}{(0,0)--(.0625,1.5)--(.125,0)}
\nuovoECG{r520}{(0,0)--(.0625,2)--(.125,0)}
\nuovoECG{r525}{(0,0)--(.0625,2.5)--(.125,0)}
\nuovoECG{r530}{(0,0)--(.0625,3)--(.125,0)}
\nuovoECG{r535}{(0,0)--(.0625,3.5)--(.125,0)}
\nuovoECG{r540}{(0,0)--(.0625,4)--(.125,0)}
\nuovoECG{r545}{(0,0)--(.0625,4.5)--(.125,0)}
%    \end{macrocode}

% onda S da 30 msec profonda da 1 a 5 mm
%    \begin{macrocode}
\nuovoECG{s31}{(0,0)--(.035,-.1)--(.075,0)}
\nuovoECG{s32}{(0,0)--(.035,-.2)--(.075,0)}
\nuovoECG{s33}{(0,0)--(.035,-.3)--(.075,0)}
\nuovoECG{s34}{(0,0)--(.035,-.4)--(.075,0)}
\nuovoECG{s35}{(0,0)--(.035,-.5)--(.075,0)}
%    \end{macrocode}

% onda S da 40 msec con profondità da 2 a 10 mm in step di 2 mm
%    \begin{macrocode}
\nuovoECG{s42}{(0,0)--(.05,-.2)--(.1,0)}
\nuovoECG{s44}{(0,0)--(.05,-.4)--(.1,0)}
\nuovoECG{s46}{(0,0)--(.05,-.6)--(.1,0)}
\nuovoECG{s48}{(0,0)--(.05,-.8)--(.1,0)}
\nuovoECG{s410}{(0,0)--(.05,-1)--(.1,0)}
%    \end{macrocode}

% onda S da 40 msec con profondità da 12 a 32 mm in step di 4 mm
%    \begin{macrocode}
\nuovoECG{s412}{(0,0)--(.05,-1.2)--(.1,0)}
\nuovoECG{s416}{(0,0)--(.05,-1.6)--(.1,0)}
\nuovoECG{s420}{(0,0)--(.05,-2)--(.1,0)}
\nuovoECG{s424}{(0,0)--(.05,-2.4)--(.1,0)}
\nuovoECG{s428}{(0,0)--(.05,-2.8)--(.1,0)}
\nuovoECG{s432}{(0,0)--(.05,-3.2)--(.1,0)}
%    \end{macrocode}

% onda S da 50 msec con profondità da 10 a 45 mm in step di 5 mm
%    \begin{macrocode}
\nuovoECG{s510}{(0,0)--(.0625,-1)--(.125,0)}
\nuovoECG{s515}{(0,0)--(.0625,-1.5)--(.125,0)}
\nuovoECG{s520}{(0,0)--(.0625,-2)--(.125,0)}
\nuovoECG{s525}{(0,0)--(.0625,-2.5)--(.125,0)}
\nuovoECG{s530}{(0,0)--(.0625,-3)--(.125,0)}
\nuovoECG{s535}{(0,0)--(.0625,-3.5)--(.125,0)}
\nuovoECG{s540}{(0,0)--(.0625,-4)--(.125,0)}
\nuovoECG{s545}{(0,0)--(.0625,-4.5)--(.125,0)}
%    \end{macrocode}

% onda T positiva da 400 msec con altezze da 1 a 11 mm - OK
%    \begin{macrocode}
\nuovoECG{tp2}{(0,0)--(.2,.05)--(.3,.15)--(.45,.2)--(.55,.2)--(.7,.15)--(.8,.05)--(1,0)}   
\nuovoECG{tp3}{(0,0)--(.2,.085)--(.3,.18)--(.45,.3)--(.55,.3)--(.7,.18)--(.8,.085)--(1,0)}   
\nuovoECG{tp4}{(0,0)--(.1,.03)-- (.2,.12)--(.3,.25)--(.45,.4)--(.55,.4)--(.7,.25)--(.8,.12)--(.9,.06)--(1,0)}  
\nuovoECG{tp5}{(0,0)--(.1,.03)--(.2,.13)--(.3,.29)--(.4,.47)--(.45,.5)--(.55,.5)--(.6,.47)--(.7,.29)--(.8,.13)--(.9,.03)--(1,0)}  
\nuovoECG{tp6}{(0,0)--(.1,.04)--(.2,.19)--(.3,.36)--(.4,.53)--(.45,.6)--(.55,.6)--(.6,.53)--(.7,.36)--(.8,.19)--(.9,.04)--(1,0)}  
\nuovoECG{tp7}{(0,0)--(.1,.04)--(.2,.2)--(.3,.39)--(.4,.58)--(.45,.7)--(.55,.7)--(.6,.58)--(.7,.39)--(.8,.2)--(.9,.04)--(1,0)}  
\nuovoECG{tp8}{(0,0)--(.1,.04)--(.2,.21)--(.3,.45)--(.4,.67)--(.45,.8)--(.55,.8)--(.6,.67)--(.7,.45)--(.8,.21)--(.9,.04)--(1,0)}  
\nuovoECG{tp9}{(0,0)--(.1,.05)--(.2,.4)--(.3,.62)--(.4,.85)--(.45,.9)--(.55,.85)--(.7,.62)--(.8,.4)--(.9,.05)--(1,0)}
\nuovoECG{tp10}{(0,0)--(.1,.05)--(.2,.42)--(.3,.66)--(.4,.9)--(.45,1)--(.55,1)--(.6,.9)--(.7,.66)--(.8,.42)--(.9,.05)--(1,0)}  
\nuovoECG{tp11}{(0,0)--(.1,.05)--(.2,.48)--(.3,.73)--(.4,.97)--(.45,1.1)--(.55,1.1)--(.6,.97)--(.7,.73)--(.8,.48)--(.9,.05)--(1,0)}  

% onda T negativa da 400 msec con profondità da 1 a 11 mm - OK
\nuovoECG{tn2}{(0,0)--(.2,-.05)--(.3,-.15)--(.45,-.2)--(.55,-.2)--(.7,-.15)--(.8,-.05)--(1,0)}   
\nuovoECG{tn3}{(0,0)--(.2,-.085)--(.3,-.18)--(.45,-.3)--(.55,-.3)--(.7,-.18)--(.8,-.085)--(1,0)}   
\nuovoECG{tn4}{(0,0)--(.1,-.03)-- (.2,-.12)--(.3,-.25)--(.45,-.4)--(.55,-.4)--(.7,-.25)--(.8,-.12)--(.9,-.06)--(1,0)}
\nuovoECG{tn5}{(0,0)--(.1,-.03)--(.2,-.13)--(.3,-.29)--(.4,-.47)--(.45,-.5)--(.55,-.5)--(.6,-.47)--(.7,-.29)--(.8,-.13)--(.9,-.03)--(1,0)}  
\nuovoECG{tn6}{(0,0)--(.1,-.04)--(.2,-.19)--(.3,-.36)--(.4,-.53)--(.45,-.6)--(.55,-.6)--(.6,-.53)--(.7,-.36)--(.8,-.19)--(.9,-.04)--(1,0)}  
\nuovoECG{tn7}{(0,0)--(.1,-.04)--(.2,-.2)--(.3,-.39)--(.4,-.58)--(.45,-.7)--(.55,-.7)--(.6,-.58)--(.7,-.39)--(.8,-.2)--(.9,-.04)--(1,0)}  
\nuovoECG{tn8}{(0,0)--(.1,-.04)--(.2,-.21)--(.3,-.45)--(.4,-.67)--(.45,-.8)--(.55,-.8)--(.6,-.67)--(.7,-.45)--(.8,-.21)--(.9,-.04)--(1,0)}  
\nuovoECG{tn9}{(0,0)--(.1,-.05)--(.2,-.4)--(.3,-.62)--(.4,-.85)--(.45,-.9)--(.55,-.85)--(.7,-.62)--(.8,-.4)--(.9,-.05)--(1,0)}   
\nuovoECG{tn10}{(0,0)--(.1,-.05)--(.2,-.42)--(.3,-.66)--(.4,-.9)--(.45,-1)--(.55,-1)--(.6,-.9)--(.7,-.66)--(.8,-.42)--(.9,-.05)--(1,0)}  
\nuovoECG{tn11}{(0,0)--(.1,-.05)--(.2,-.48)--(.3,-.73)--(.4,-.97)--(.45,-1.1)--(.55,-1.1)--(.6,-.97)--(.7,-.73)--(.8,-.48)--(.9,-.05)--(1,0)}  
%    \end{macrocode}

% Morfologie normali (P-QRS-T) - tipologia normale (n) in classe (n) per 12 derivazioni (D1-3, aVR aVL aVF, V1-6)
%    \begin{macrocode}
\nuovoECG{nnd1}{(0,0)--(.1,.08)--(.18,0)--(.3,0)--(.4,-.1)--(.42,.8)--(.48,.1)--(.55,0)--(.8,0)--(.9,.1)--(1,.2)--(1.1,.3)--(1.2,.2)--(1.3,0)--(1.8,0)}
\nuovoECG{nnd2}{(0,0)--(.05,.05)--(.08,.1)--(.15,.05)--(.2,0)--(.3,0)--(.35,-.1)--(.4,.9)--(.5,-.02)--(.6,0)--(.9,0)--(1,.1)--(1.1,.2)--(1.15,.3)--(1.25,.2)--(1.3,.05)--(1.8,0)}
\nuovoECG{nnd3}{(0,0)--(0.08,0.05)--(0.15,0.02)--(0.3,0)--(0.4,0.08)--(0.43,0.21)--(0.48,-0.1)--(0.52,0)--(1.05,0)--(1.1,0.07)--(1.2,0)--(1.8,0)}
\nuovoECG{nnvr}{(0,0)--(0.08,-0.05)--(0.12,-0.14)--(0.2,0)--(0.3,0)--(0.4,0.01)--(0.43,0.08)--(0.48,-0.9)--(0.52,-0.1)--(0.8,0)--(0.9,-0.1)--(1,-0.2)--(1.1,-0.3)--(1.15,-0.3)--(1.25,-0.1)--(1.3,0)--(1.5,0)}
\nuovoECG{nnvl}{(0,0)--(0.08,-0.05)--(0.16,0.1)--(0.21,0)--(0.4,0)--(0.41,-0.01)--(0.45,0.38)--(0.55,0.05)--(0.6,0)--(0.8,0)--(0.9,0.05)--(1,0.1)--(1.1,0.18)--(1.2,0.05)--(1.3,0)--(1.5,0)}
\nuovoECG{nnvf}{(0,0)--(0.1,0.05)--(0.17,-0.05)--(0.2,0)--(0.4,0)--(0.46,0.6)--(0.5,-0.1)--(0.55,0)--(.7,.02)--(.8,.03)--(0.9,.04)--(1.0,0.5)--(1.1,0.1)--(1.15,0.15)--(1.2,0.1)--(1.3,0)--(1.5,0)}

\nuovoECG{nnv1}{(0,0)--(0.08,0)--(0.1,0.05)--(0.17,0.1)--(0.2,-0.05)--(0.24,0)--(0.4,0)--(0.42,0.2)--(0.5,-1.1)--(0.6,0)--(0.7,0)--(0.8,0.08)--(0.9,0.1)--(1,0)--(1.1,-0.08)--(1.18,-0.1)--(1.2,-0.04)--(1.3,0)--(1.5,0)}
\nuovoECG{nnv2}{(0,0)--(.05,.05)--(.15,.2)--(.2,0)--(.4,0)--(.45,.3)--(.5,-.55)--(.6,0)--(.7,.08)--(.8,.12)--(.9,.2)--(1,.35)--(1.1,.5)--(1.2,.4)--(1.3,.05)--(1.5,0)}
\nuovoECG{nnv3}{(0,0)--(0.1,0)--(0.15,0.1)--(0.2,0)--(0.4,0)--(0.5,1)--(0.55,-0.1)--(0.6,0)--(0.7,0)--(0.8,0.05)--(0.9,0.1)--(1,0.3)--(1.1,0.6)--(1.15,0.7)--(1.2,0.4)--(1.3,0.1)--(1.4,0)--(1.6,0)}
\nuovoECG{nnv4}{(0,0)--(0.1,0)--(0.14,0.1)--(0.2,0)--(0.4,0)--(0.41,-0.07)--(0.5,1.35)--(0.55,-0.03)--(0.6,0)--(0.8,0)--(0.9,0.1)--(1,0.2)--(1.1,0.4)--(1.15,0.55)--(1.2,0.4)--(1.3,0.1)--(1.5,0)--(1.6,0)}
\nuovoECG{nnv5}{(0,0)--(0.1,0)--(0.15,0.1)--(0.2,0)--(0.4,0)--(0.45,-0.1)--(0.5,1.2)--(0.55,-0.05)--(0.6,0)--(0.8,0)--(0.9,0.05)--(1,0.1)--(1.1,0.3)--(1.2,0.4)--(1.3,0.02)--(1.4,0)--(1.6,0)}
\nuovoECG{nnv6}{(0,0)--(0.1,0)--(0.16,0.09)--(0.2,0)--(0.4,0)--(0.44,-0.09)--(0.5,0.9)--(0.55,0.01)--(0.6,0)--(0.9,0)--(1,0.1)--(1.1,0.2)--(1.18,0.34)--(1.2,0.25)--(1.3,0.1)--(1.4,0)--(1.6,0)}
%    \end{macrocode}

% Morfologie di un ritmo sinusale normale (n) bradicarico (b) (d1-d2-d3-vr-vl-vf-v1-v2-v3-v4-v5-v6)
%    \begin{macrocode}
\nuovoECG{nbd1}{(0,0)--(0.1,0.05)--(0.18,0)--(0.2,0.1)--(0.25,0)--(0.4,0)--(0.45,-0.1)--(0.5,0.9)--(0.6,-0.15)--(0.65,0)--(0.9,0)--(1,0.05)--(1.1,0.1)--(1.2,0.2)--(1.3,0.35)--(1.4,0.4)--(1.5,0.1)--(1.6,0)--(1.8,0)}
\nuovoECG{nbd2}{(0,0)--(0.1,0.08)--(0.2,0.12)--(0.3,0)--(0.5,0)--(0.6,0.02)--(0.7,0.75)--(0.8,-0.2)--(0.9,0)--(1.1,0)--(1.2,0.05)--(1.3,0.08)--(1.4,0.3)--(1.45,0.35)--(1.5,0.3)--(1.6,0.05)--(1.7,0)--(1.8,0)}
\nuovoECG{nbd3}{(0,0)--(0.05,0)--(0.1,0.05)--(0.15,0.1)--(0.2,0.05)--(0.3,0)--(0.4,0)--(0.5,0)--(0.6,0.1)--(0.7,-0.7)--(0.8,0)--(1.3,0)--(1.4,-0.08)--(1.5,0)--(1.6,0)--(1.8,0)}
\nuovoECG{nbvr}{(0,0)--(0.1,-0.05)--(0.2,-0.1)--(0.3,0)--(0.5,0)--(0.6,0.02)--(0.65,-0.8)--(0.7,0.2)--(0.8,0)--(1,0)--(1.1,-0.05)--(1.2,-0.1)--(1.3,-0.2)--(1.4,-0.3)--(1.5,-0.2)--(1.6,-0.1)--(1.7,0)--(1.8,0)}
\nuovoECG{nbvl}{(0,0)--(0.09,-0.05)--(0.16,-0.05)--(0.26,0.05)--(0.4,0)--(0.5,0)--(0.57,-0.1)--(0.65,0.8)--(0.7,-0.02)--(0.75,0)--(1.1,0)--(1.2,0.03)--(1.3,0.1)--(1.4,0.2)--(1.5,0.1)--(1.6,0.02)--(1.7,0)--(1.8,0)}
\nuovoECG{nbvf}{(0,0)--(0.09,0.1)--(0.16,0.12)--(0.2,0.02)--(0.3,0)--(0.5,0)--(0.6,0.1)--(0.65,0.3)--(0.7,-0.1)--(0.75,0)--(0.8,0.01)--(0.9,0.02)--(1,0.03)--(1.1,0.04)--(1.2,0.06)--(1.3,0.12)--(1.4,0.16)--(1.45,0.2)--(1.5,0.15)--(1.6,0.06)--(1.7,0.01)--(1.8,0)}

\nuovoECG{nbv1}{(0,0)--(0.1,0)--(0.2,-0.1)--(0.3,-0.04)--(0.4,0)--(0.5,0)--(0.58,0.12)--(0.66,-0.7)--(0.72,0.02)--(0.8,0)--(1.2,0)--(1.3,-0.04)--(1.4,-0.12)--(1.5,-0.15)--(1.6,0)--(1.8,0)}
\nuovoECG{nbv2}{(0,0)--(0.1,0)--(0.15,0.04)--(0.2,-0.04)--(0.3,0)--(0.5,0)--(0.58,0.75)--(0.65,-0.45)--(0.7,-0.2)--(0.75,0)--(1,0.01)--(1.1,0.1)--(1.2,0.2)--(1.3,0.3)--(1.4,0.45)--(1.5,0.3)--(1.6,0.08)--(1.7,0)--(1.8,0)}
\nuovoECG{nbv3}{(0,0)--(0.1,0.02)--(0.2,0.11)--(0.3,0)--(0.5,0)--(0.58,1.4)--(0.67,-0.6)--(0.7,-0.3)--(0.8,0.1)--(0.9,0.15)--(1.0,0.2)--(1.1,0.25)--(1.2,0.35)--(1.3,0.6)--(1.4,0.9)--(1.5,0.5)--(1.6,0.1)--(1.7,0)--(1.8,0)}
\nuovoECG{nbv4}{(0,0)--(0.1,0.03)--(0.16,0.1)--(0.2,0)--(0.3,0.05)--(0.4,0)--(0.5,0)--(0.58,2.2)--(0.68,-0.4)--(0.74,0)--(0.8,0.03)--(0.9,0.05)--(1,0.1)--(1.1,0.15)--(1.2,0.22)--(1.3,0.4)--(1.4,0.8)--(1.45,1)--(1.5,0.5)--(1.6,0.1)--(1.7,0.02)--(1.8,0)--(1.9,0)}
\nuovoECG{nbv5}{(0,0)--(0.1,0)--(0.2,0.03)--(0.3,0.08)--(0.4,0.02)--(0.5,0)--(0.6,0)--(0.63,-0.1)--(0.7,2.3)--(0.8,-0.3)--(0.85,0)--(0.9,0)--(1.0,0.02)--(1.1,0.08)--(1.2,0.1)--(1.3,0.2)--(1.4,0.4)--(1.5,0.8)--(1.6,0.5)--(1.7,0.1)--(1.8,0.05)--(1.9,0)--(2,0)}
\nuovoECG{nbv6}{(0,0)--(0.1,0.05)--(0.2,0.1)--(0.3,0.04)--(0.4,0)--(0.5,0)--(0.55,-0.12)--(0.6,1.9)--(0.7,-0.09)--(0.75,0)--(0.9,0)--(1,0.05)--(1.1,0.1)--(1.2,0.15)--(1.3,0.4)--(1.4,0.7)--(1.5,0.25)--(1.6,0.07)--(1.7,0)--(1.8,0)}
%    \end{macrocode}

% Morfologia di una extrasistole (e) ventricolare (v) nelle 12 derivazioni (d1-d2-d3-vr-vl-vf-v1-v2-v3-v4-v5-v6)
%    \begin{macrocode}
\nuovoECG{evd1}{(0,0)--(.1,0)--(.2,.15)--(.3,.1)--(.4,.44)--(.5,0)--(.7,0)--(.8,-.1)--(.9,0)--(1.0,.1)--(1.1,.15)--(1.2,-.05)--(1.3,-.02)--(1.4,0)--(1.5,0)}
\nuovoECG{evd2}{(0,0)--(.1,.05)--(.2,.5)--(.3,.9)--(.4,1.4)--(.5,.5)--(.6,-.15)--(.7,-.25)--(.8,-.35)--(.9,-.5)--(1.0,-.62)--(1.1,-.43)--(1.2,-.1)--(1.3,.04)--(1.4,0)--(1.5,0)}
\nuovoECG{evd3}{(0,0)--(.1,.05)--(.2,.1)--(.3,.7)--(.4,1.1)--(.45,1.35)--(.5,0)--(.6,-.2)--(.7,-.2)--(.8,-.3)--(.9,-.5)--(1,-.6)--(1.1,-.5)--(1.2,-.2)--(1.3,.1)--(1.4,.05)--(1.5,0)--(1.6,0)}
\nuovoECG{evvr}{(0,0)--(.1,0)--(.2,.05)--(.3,-.05)--(.4,-.05)--(.5,-.2)--(.6,-.5)--(.68,-.9)--(.7,-.7)--(.8,0)--(.9,.09)--(1,.12)--(1.1,.18)--(1.2,.2)--(1.3,.2)--(1.4,.16)--(1.5,.1)--(1.6,0)--(1.7,0)}
\nuovoECG{evvl}{(0,0)--(.1,0)--(.2,0)--(.3,.09)--(.4,-.3)--(.5,-.65)--(.6,0)--(.7,.05)--(.8,.1)--(.9,.2)--(1,.3)--(1.1,.4)--(1.15,.4)--(1.2,.3)--(1.3,0)--(1.4,-.1)--(1.5,0)--(1.6,0)}
\nuovoECG{evvf}{(0,0)--(0.1,.05)--(.2,.07)--(.3,.4)--(.4,.8)--(.5,1.32)--(.6,-.05)--(.7,-.18)--(.8,-.21)--(.9,-.3)--(1,-.5)--(1.1,-.67)--(1.2,-.05)--(1.3,-.01)--(1.4,.05)--(1.5,0)--(1.6,0)}

\nuovoECG{evv1}{(0,0)--(.2,0)--(.3,.02)--(.4,.4)--(.45,1)--(.5,.3)--(.55,-.2)--(.6,-.15)--(.7,-.25)--(.8,-.25)--(.9,-.3)--(1,-.4)--(1.1,-.5)--(1.2,-.3)--(1.3,0)--(1.6,0)}
\nuovoECG{evv2}{(0,0)--(.2,0)--(.3,.3)--(.4,1.6)--(.47,2.5)--(.5,1.7)--(.54,-.3)--(.6,-.25)--(.7,-.3)--(.8,-.35)--(.9,-.4)--(1,-.52)--(1.1,-.6)--(1.2,-.4)--(1.3,-.15)--(1.4,-.1)--(1.5,-.02)--(1.6,0)}
\nuovoECG{evv3}{(0,0)--(.2,0)--(.3,.7)--(.4,1.7)--(.46,2.8)--(.5,2.3)--(.6,0)--(.7,-.35)--(.8,-.4)--(.9,-.6)--(1,-.8)--(1.1,-1.05)--(1.2,-.8)--(1.3,-.3)--(1.4,-.05)--(1.5,0)--(1.6,0)}
\nuovoECG{evv4}{(0,0)--(.2,0)--(.3,.6)--(.4,1.7)--(.5,2.25)--(.6,-.2)--(.7,-.3)--(.8,-.45)--(.9,-.6)--(1,-.87)--(1.1,-1)--(1.2,-.6)--(1.3,-.2)--(1.4,-.1)--(1.5,-.04)--(1.55,0)--(1.6,0)}
\nuovoECG{evv5}{(0,0)--(.2,0)--(.3,.5)--(.4,1.5)--(.45,1.75)--(.5,1.1)--(.6,-.2)--(.7,-.28)--(.8,-.4)--(.9,-.5)--(1,-.7)--(1.07,-.8)--(1.1,-.7)--(1.2,-.4)--(1.3,-.2)--(1.4,-.1)--(1.5,0)--(1.6,0)}
\nuovoECG{evv6}{(0,0)--(.2,0)--(.3,.45)--(.4,1.3)--(.45,1.4)--(.5,.7)--(.6,-.1)--(.7,-.2)--(.8,-.25)--(.9,-.4)--(1,-.5)--(1.1,-.4)--(1.2,-.2)--(1.3,-.05)--(1.4,0)--(1.6,0)}
%    \end{macrocode}

% morfologia di un caso giovanile di WPW con evidenza di onda delta; si utilizza "w" (per Wolf) e "b" (per bradicardia)
%    \begin{macrocode}
\nuovoECG{wbd1}{(0,0)--(.08,.07)--(.1,.04)--(.18,0)--(.2,-.02)--(.25,-.02)--(.26,0)--(.28,.22)--(.3,.15)--(.32,-.1)--(.4,0)--(.42,.02)--(.6,.05)--(.7,.12)--(.8,.19)--(.85,.05)--(.9,.02)--(1,.01)--(1.2,0)--(1.5,0)}
\nuovoECG{wbd2}{(0,0)--(.05,-.02)--(.15,.1)--(.25,-.02)--(.3,.1)--(.35,1.7)--(.4,0)--(.45,.1)--(.5,.05)--(.6,.1)--(.7,.15)--(.8,.3)--(.85,.37)--(.9,.2)--(1,.07)--(1.1,.07)--(1.2,.04)--(1.3,.02)--(1.4,0)--(1.5,0)}
\nuovoECG{wbd3}{(0,0)--(.1,.07)--(.12,.09)--(.2,0)--(.22,-.05)--(.3,.2)--(.34,1.55)--(.4,.15)--(.5,.05)--(.6,.07)--(.7,.09)--(.8,.15)--(.85,.22)--(.9,.15)--(1,.05)--(1.1,.05)--(1.2,.04)--(1.3,.02)--(1.4,0)--(1.5,0)}
\nuovoECG{wbvr}{(0,0)--(.1,0)--(.2,-.08)--(.3,.02)--(.35,0)--(.41,-.95)--(.46,0)--(.5,0)--(.6,-.07)--(.7,-.08)--(.8,-.11)--(.9,-.27)--(1,-.05)--(1.1,-.03)--(1.2,-.02)--(1.3,-.01)--(1.4,0)--(1.5,0)}
\nuovoECG{wbvl}{(0,0)--(.1,-.03)--(.15,-.05)--(.2,-.02)--(.27,0)--(.3,-.02)--(.37,-.7)--(.42,-.1)--(.5,-.05)--(.55,-.02)--(.6,-.02)--(.7,-.02)--(.8,.02)--(.9,-.05)--(1,-.03)--(1.1,0)--(1.5,0)}
\nuovoECG{wbvf}{(0,0)--(.1,.05)--(.15,.1)--(.22,-.02)--(.3,.15)--(.35,1.65)--(.4,.06)--(.5,.075)--(.6,.1)--(.7,.12)--(.75,.14)--(.8,.2)--(.85,.27)--(.9,.15)--(.95,.07)--(1,.05)--(1.1,.05)--(1.2,.04)--(1.3,.01)--(1.4,0)--(1.5,0)}

\nuovoECG{wbv1}{(0,0)--(.075,.05)--(.15,0)--(.24,0)--(.32,.4)--(.38,-.37)--(.4,-.1)--(.5,0)--(.6,0)--(.7,-.03)--(.8,-.1)--(.9,.07)--(1,0)--(1.5,0)}
\nuovoECG{wbv2}{(0,0)--(.05,.05)--(.2,0)--(.3,1.4)--(.35,-.95)--(.4,-.1)--(.5,.05)--(.6,.07)--(.7,.1)--(.8,.15)--(.85,.3)--(.9,.2)--(1,.06)--(1.1,.05)--(1.2,.04)--(1.3,.03)--(1.4,.01)--(1.5,0)}
\nuovoECG{wbv3}{(0,0)--(.12,.07)--(.19,-.01)--(.25,.1)--(.33,2.1)--(.37,-.21)--(.42,.05)--(.5,.06)--(.6,.1)--(.72,.1)--(.8,.04)--(.87,.17)--(.98,.1)--(1.1,.08)--(1.2,.06)--(1.3,.04)--(1.4,.01)--(1.5,0)}
\nuovoECG{wbv4}{(0,0)--(.1,.05)--(.18,-.05)--(.25,.1)--(.3,1.7)--(.33,0)--(.37,.05)--(.5,.075)--(.6,.12)--(.7,.2)--(.81,.4)--(.9,.2)--(.95,.1)--(1.05,.1)--(1.1,.07)--(1.2,.04)--(1.3,.02)--(1.4,.01)--(1.5,0)}
\nuovoECG{wbv5}{(0,0)--(.06,.05)--(.12,-.02)--(.2,0)--(.26,1.75)--(.3,.03)--(.4,.05)--(.5,.08)--(.6,.15)--(.7,.33)--(.76,.43)--(.8,.3)--(.9,.07)--(1,.06)--(1.1,.05)--(1.2,.02)--(1.3,.01)--(1.4,0)--(1.5,0)}
\nuovoECG{wbv6}{(0,0)--(.07,.05)--(.15,0)--(.23,-.03)--(.27,1.06)--(.3,.04)--(.4,.05)--(.5,.08)--(.6,.12)--(.7,.25)--(.76,.39)--(.8,.3)--(.9,.04)--(1,.04)--(1.1,.03)--(1.2,.02)--(1.3,.01)--(1.4,0)--(1.5,0)}
%    \end{macrocode}

% morfologia di un caso giovanile di WPW durante accesso di TPSV; si utilizza "w" (per Wolf) e "t" (per tachicardia) 
%    \begin{macrocode}
\nuovoECG{wtd1}{(0,0)--(.05,.77)--(.08,-.46)--(.1,-.2)--(.15,-.13)--(.2,-.08)--(.25,.2)--(.29,.24)--(.35,.1)--(.4,0)}
\nuovoECG{wtd2}{(0,0)--(.05,1.2)--(.08,-.15)--(.1,-.15)--(.15,-.1)--(.2,-.15)--(.25,.1)--(.28,.18)--(.35,.05)--(.4,0)}
\nuovoECG{wtd3}{(0,0)--(.04,.88)--(.06,.12)--(.1,.07)--(.15,.04)--(.2,-.075)--(.22,-.08)--(.25,-.04)--(.3,.08)--(.33,.1)--(.35,.04)--(.4,0)}
\nuovoECG{wtvr}{(0,0)--(.05,-1.03)--(.08,.25)--(.12,.2)--(.15,.2)--(.2,.15)--(.25,0)--(.28,-.16)--(.35,-.05)--(.4,0)}
\nuovoECG{wtvl}{(0,0)--(.05,.22)--(.08,-.52)--(.1,-.08)--(.2,-.05)--(.28,.13)--(.3,.1)--(.38,0)--(.4,0)}
\nuovoECG{wtvf}{(0,0)--(.04,1.16)--(.07,-.09)--(.15,-.1)--(.18,-.12)--(.28,.13)--(.33,.13)--(.38,.01)--(.4,0)}

\nuovoECG{wtv1}{(0,0)--(.04,-1.17)--(.07,-.1)--(.1,-.1)--(.2,-.2)--(.27,-.4)--(.3,-.25)--(.38,-.1)--(.4,0)}
\nuovoECG{wtv2}{(0,0)--(.06,1.18)--(.08,-1.3)--(.14,-.25)--(.21,-.2)--(.28,0)--(.33,.06)--(.38,.01)--(.4,0)}
\nuovoECG{wtv3}{(0,0)--(.04,1.68)--(.06,-1.34)--(.11,-.5)--(.13,-.34)--(.2,-.3)--(.29,.24)--(.31,.2)--(.38,.01)--(.4,0)}
\nuovoECG{wtv4}{(0,0)--(.04,1.65)--(.07,-.88)--(.1,-.4)--(.2,-.33)--(.28,.35)--(.3,.3)--(.38,.01)--(.4,0)}
\nuovoECG{wtv5}{(0,0)--(.04,1.97)--(.06,-.47)--(.1,-.3)--(.15,-.27)--(.2,-.18)--(.27,.33)--(.3,.3)--(.38,.01)--(.4,0)}
\nuovoECG{wtv6}{(0,0)--(.05,1.53)--(.07,-.25)--(.13,-.2)--(.19,-.2)--(.21,-.1)--(.28,.2)--(.3,.2)--(.37,.02)--(.4,0)}
%    \end{macrocode}

% morfologia di un caso giovanile di PR corto senza preeccitazione e con BBD minore (b per blocco di branca, b per bradicardico)
%    \begin{macrocode}
\nuovoECG{bbd1}{(0,0)--(.1,.05)--(.13,-.02)--(.2,0)--(.25,-.05)--(.3,.5)--(.35,-.35)--(.4,-.05)--(.5,.02)--(.6,.05)--(.7,.1)--(.8,.2)--(.9,.3)--(.95,.35)--(1,.3)--(1.1,.08)--(1.2,0)--(3,0)}
\nuovoECG{bbd2}{(0,0)--(.05,.1)--(.1,.11)--(.17,0)--(.23,0)--(.27,-.1)--(.33,1.57)--(.43,-.25)--(.5,0)--(.6,.05)--(.7,.08)--(.8,.12)--(.9,.23)--(1,.37)--(1.05,.4)--(1.1,.3)--(1.2,.05)--(1.3,0)--(3,0)}
\nuovoECG{bbd3}{(0,0)--(.04,0)--(.1,.1)--(.15,0)--(.25,0)--(.3,-.07)--(.38,1.1)--(.42,-.07)--(.5,0)--(.7,0)--(.8,-.03)--(.9,-.05)--(1,0)--(1.1,.05)--(1.15,.1)--(1.2,.04)--(1.3,0)--(3,0)}
\nuovoECG{bbvr}{(0,0)--(.1,-.06)--(.15,-.1)--(.25,0)--(.3,.04)--(.35,.08)--(.41,-1.02)--(.47,.26)--(.55,0)--(.6,0)--(.7,-.05)--(.8,-.1)--(.9,-.18)--(1,-.3)--(1.08,-.38)--(1.2,-.15)--(1.3,0)--(3,0)}
\nuovoECG{bbvl}{(0,0)--(.1,-.05)--(.2,.05)--(.22,0)--(.32,0)--(.33,.1)--(.4,-.32)--(.42,-.21)--(.44,-.28)--(.5,0)--(.6,0)--(.7,.04)--(.8,.08)--(.9,.13)--(1,.19)--(1.1,.1)--(1.2,0)--(3,0)}
\nuovoECG{bbvf}{(0,0)--(.1,.1)--(.15,.1)--(.2,0)--(.3,0)--(.35,-.1)--(.4,1.35)--(.5,-.12)--(.55,0)--(.7,.03)--(.8,.05)--(.9,.07)--(1,.15)--(1.1,.22)--(1.2,.1)--(1.3,0)--(3,0)}

\nuovoECG{bbv1}{(0,0)--(.1,.02)--(.12,-.1)--(.2,-.04)--(.3,0)--(.32,.27)--(.38,-1.46)--(.45,.25)--(.5,.05)--(.6,0)--(.7,0)--(.8,-.05)--(.9,-.11)--(1,-.24)--(1.05,-.27)--(1.1,-.2)--(1.2,-.05)--(1.3,0)--(3,0)}
\nuovoECG{bbv2}{(0,0)--(.1,.06)--(.15,-.04)--(.2,0)--(.3,0)--(.36,.46)--(.41,-1.95)--(.52,0)--(.6,.13)--(.7,.17)--(.8,.23)--(.9,.32)--(1,.4)--(1.1,.33)--(1.2,.15)--(1.3,.07)--(1.4,.09)--(1.5,.12)--(1.6,.1)--(1.7,.05)--(1.8,0)--(3,0)}
\nuovoECG{bbv3}{(0,0)--(.1,.07)--(.2,0)--(.3,0)--(.35,.45)--(.41,-.8)--(.45,-.5)--(.47,-.7)--(.5,0)--(.55,.15)--(.6,.2)--(.7,.25)--(.8,.33)--(.9,.5)--(1,.65)--(1.1,.52)--(1.2,.23)--(1.3,.11)--(1.4,.12)--(1.5,.13)--(1.6,.09)--(1.7,.06)--(1.8,0)--(3,0)}
\nuovoECG{bbv4}{(0,0)--(.1,.07)--(.2,0)--(.3,0)--(.4,1.5)--(.45,-.35)--(.5,0)--(.55,.12)--(.6,.15)--(.7,.22)--(.8,.32)--(.9,.5)--(1,.7)--(1.08,.75)--(1.1,.65)--(1.2,.3)--(1.3,.13)--(1.4,.15)--(1.5,.15)--(1.6,.09)--(1.7,.05)--(1.8,.02)--(1.9,0)--(3,0)}
\nuovoECG{bbv5}{(0,0)--(.1,.06)--(.2,0)--(.26,0)--(.3,-.1)--(.37,1.92)--(.41,-.21)--(.5,.07)--(.6,.1)--(.7,.18)--(.8,.3)--(.9,.5)--(1,.75)--(1.1,.5)--(1.2,.15)--(1.3,.1)--(1.4,.1)--(1.5,.1)--(1.6,.07)--(1.7,.02)--(1.8,0)--(3,0)}
\nuovoECG{bbv6}{(0,0)--(.1,.1)--(.2,0)--(.3,0)--(.32,-.1)--(.41,1.6)--(.45,-.13)--(.5,.05)--(.6,.07)--(.7,.1)--(.8,.15)--(.9,.3)--(1,.45)--(1.08,.58)--(1.2,.2)--(1.3,.06)--(1.4,.04)--(1.5,.04)--(1.6,.03)--(1.7,.02)--(1.8,0)--(3,0)}
%    \end{macrocode}

% altro tracciato con morfologia normale
%    \begin{macrocode}
\nuovoECG{n2d1}{(0,0)--(.08,.07)--(.2,0)--(.3,0)--(.35,-.04)--(.4,-.1)--(.45,1)--(.5,-.17)--(.55,.02)--(.6,.02)--(.7,.03)--(.8,.06)--(.9,.15)--(1,.32)--(1.07,.37)--(1.1,.3)--(1.2,.08)--(1.3,0)--(2,0)}
\nuovoECG{n2d2}{(0,0)--(.1,.07)--(.2,.05)--(.3,0)--(.4,0)--(.45,-.08)--(.53,.83)--(.59,-.06)--(.65,.06)--(.7,.06)--(.8,.08)--(.9,.12)--(1,.2)--(1.14,.35)--(1.2,.25)--(1.3,.06)--(1.4,-.04)--(1.5,.02)--(2,0)}
\nuovoECG{n2d3}{(0,0)--(.07,.04)--(.14,-.01)--(.2,-.01)--(.3,0)--(.45,0)--(.52,-.25)--(.55,.2)--(.6,.04)--(.7,.02)--(.8,.03)--(.9,0)--(1,-.03)--(1.1,-.04)--(1.2,0)--(2,0)}
\nuovoECG{n2vr}{(0,0)--(.1,-.06)--(.15,-.08)--(.2,-.05)--(.3,0)--(.42,0)--(.45,.1)--(.52,-.85)--(.58,.13)--(.62,-.03)--(.7,-.05)--(.8,-.07)--(.9,-.1)--(1,-.2)--(1.12,-.36)--(1.2,-.2)--(1.3,-.05)--(1.4,0)--(1.6,-.03)--(2,0)}
\nuovoECG{n2vl}{(0,0)--(.1,.05)--(.2,.03)--(.39,0)--(.44,.58)--(.5,-.15)--(.55,0)--(.7,.02)--(.8,.04)--(.9,.1)--(1,.2)--(1.05,.23)--(1.1,.18)--(1.2,.03)--(1.3,.01)--(1.5,0)--(2,0)}
\nuovoECG{n2vf}{(0,0)--(.1,.05)--(.2,.01)--(.3,0)--(.42,0)--(.46,-.06)--(.54,.33)--(.55,.19)--(.57,.21)--(.6,-.02)--(.7,.02)--(.8,.01)--(.9,.04)--(1,.06)--(1.1,.12)--(1.15,.15)--(1.25,.1)--(1.3,.03)--(1.5,0)--(2,0)}

\nuovoECG{n2v1}{(0,0)--(.1,.05)--(.17,-.02)--(.2,0)--(.4,.01)--(.46,.24)--(.52,-.71)--(.63,0)--(.7,0)--(.8,.01)--(.9,-.03)--(1,-.06)--(1.1,-.13)--(1.15,-.13)--(1.2,-.05)--(1.3,0)--(2,0)}
\nuovoECG{n2v2}{(0,0)--(.13,.07)--(.16,0)--(.3,-.02)--(.4,0)--(.45,.35)--(.55,-1.13)--(.63,0)--(.7,.05)--(.8,.08)--(.9,.12)--(1,.22)--(1.05,.25)--(1.12,.2)--(1.2,.12)--(1.3,.04)--(1.4,0)--(1.5,0)--(1.7,.02)--(2,0)}
\nuovoECG{n2v3}{(0,0)--(.1,.08)--(.2,.03)--(.3,0)--(.42,0)--(.44,.3)--(.5,.4)--(.55,-1.05)--(.63,0)--(.7,.07)--(.8,.1)--(.9,.15)--(1,.25)--(1.1,.33)--(1.2,.2)--(1.3,.05)--(1.4,.03)--(1.5,.02)--(1.7,.01)--(2,0)}
\nuovoECG{n2v4}{(0,0)--(.12,.06)--(.22,0)--(.3,-.01)--(.42,0)--(.5,.83)--(.55,-.81)--(.62,.03)--(.7,.05)--(.8,.1)--(.9,.15)--(1,.25)--(1.1,.37)--(1.2,.2)--(1.3,.04)--(1.4,.02)--(1.6,.02)--(1.8,.01)--(2,0)}
\nuovoECG{n2v5}{(0,0)--(.1,.04)--(.2,.06)--(.3,0)--(.46,.03)--(.5,-.06)--(.56,1.13)--(.6,-.5)--(.67,.04)--(.8,.09)--(.9,.13)--(1,.2)--(1.1,.35)--(1.15,.42)--(1.2,.38)--(1.3,.15)--(1.4,.05)--(1.5,.04)--(1.7,.06)--(1.9,0)--(2,0)}
\nuovoECG{n2v6}{(0,0)--(.15,.06)--(.3,0)--(.42,.02)--(.45,-.11)--(.52,1.15)--(.6,-.2)--(.65,.05)--(.7,.06)--(.8,.08)--(.9,.11)--(1,.21)--(1.1,.38)--(1.15,.38)--(1.2,.3)--(1.3,.07)--(1.4,0)--(1.5,.01)--(1.7,.01)--(1.9,0)--(2,0)}
%    \end{macrocode}

%
% \paragraph{Costruzioni speciali}
%    \begin{macrocode}
\nuovoECG{pbav1}{(0,0)--(.08,.15)--(.2,0)--(.8,0)}
\nuovoECG{pbavlw}{(0,0)--(.1,.08)--(.15,.15)--(.2,.05)--(.3,0)}
\nuovoECG{qrsbav1}{(0,0)--(.05,0)--(.1,.08)--(.15,.12)--(.2,-.1)--(.25,-1.1)
 --(.3,0)--(.35,.7)--(.4,.3)--(.5,.2)--(.7,.1)--(1,.07)--(1.2,0)}
\nuovoECG{tpbav1}{(0,0)--(0.7,0)}
\nuovoECG{qrslw}{(0,0)--(0.05,0.9)--(0.12,-0.08)--(0.14,0)--(0.3,0)
 --(0.38,0.05)--(0.43,0.1)--(0.7,0.3)--(0.8,0.4)--(0.9,0.2)--(1.1,0)}
%    \end{macrocode}

% Un battito normale (di un tracciato in FA) seguito da BEV (fatv) che poi degenera in TV (tvtv)
%    \begin{macrocode}
\nuovoECG{fatv}{(0,0)--(.1,0)--(.2,.02)--(.3,.05)--(.35,.05)--(.4,.7)
 --(.42,1.1)--(.5,.05)--(.6,.0)--(.7,.05)--(.8,.08)--(.9,.1)--(1.0,.2)
 --(1.1,.3)--(1.2,.15)--(1.3,0)--(2,0)--(2.1,.08)--(2.2,.65)--(2.3,1.9)
 --(2.33,2.1)--(2.4,1.2)--(2.5,-.1)--(2.6,-.3)--(2.7,-.4)--(2.8,-.55)
 --(2.9,-.8)--(3,-.6)--(3.15,0)}
\nuovoECG{tvtv}{(0,0)--(.1,.08)--(.2,.65)--(.3,1.9)--(.33,2.1)--(0.4,1.2)
 --(.5,-.1)--(.6,-.3)--(.7,-.4)--(.8,-.55)--(.9,-.8)--(1.0,-.6)--(1.15,0)}
%    \end{macrocode}
%
%</archivio>
%\fi
%% \Finale
%
% \endinput
