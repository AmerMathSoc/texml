% \iffalse meta-comment
% vim: textwidth=75
%<*internal>
\iffalse
%</internal>
%<*readme>
|
---------:| ---------------------------------------------------------------
proofread:| Commands for inserting annotations
   Author:| Wybo Dekker
   E-mail:| wybo@dekkerdocumenten.nl
  License:| Released under the LaTeX Project Public License v1.3c or later
      See:| http://www.latex-project.org/lppl.txt

Short description:
The proofread package defines a few LaTeX commands that are useful
when you proofread a latex document. These allow you to easily highlight
text and add comments in the margin. Vim escape sequences are provided
for inserting these LaTeX commands in the source. The package is based
on code for a text highlighting command that was published by Antal S-Z
in http://tex.stackexchange.com/questions/5959.
The main file, proofread.dtx, is self-extracting, so you can generate the
style file by compiling proofread.dtx with pdflatex.

%</readme>
%<*internal>
\fi
\def\nameofplainTeX{plain}
\ifx\fmtname\nameofplainTeX\else
  \expandafter\begingroup
\fi
%</internal>
%<*install>
\input docstrip.tex
\keepsilent
\askforoverwritefalse
\preamble
---------:| ---------------------------------------------------------------
proofread:| Commands for inserting annotations
   Author:| Wybo Dekker
   E-mail:| wybo@dekkerdocumenten.nl
  License:| Released under the LaTeX Project Public License v1.3c or later
      See:| http://www.latex-project.org/lppl.txt

\endpreamble
\postamble

Copyright (C) 2015 by Wybo Dekker <wybo@dekkerdocumenten.nl>

This work may be distributed and/or modified under the
conditions of the LaTeX Project Public License (LPPL), either
version 1.3c of this license or (at your option) any later
version.  The latest version of this license is in the file:

http://www.latex-project.org/lppl.txt

This work is "maintained" (as per LPPL maintenance status) by
Wybo Dekker.

This work consists of the file proofread.dtx and a Makefile.
Running "make" generates the derived files README, proofread.pdf and
proofread.sty.
Running "make inst" installs the files in the user's TeX tree.
Running "make install" installs the files in the local TeX tree.

\endpostamble

\usedir{tex/latex/proofread}
\generate{
  \file{\jobname.sty}{\from{\jobname.dtx}{package}}
}
%</install>
%<install>\endbatchfile
%<*internal>
\usedir{source/latex/proofread}
\generate{
  \file{\jobname.ins}{\from{\jobname.dtx}{install}}
}
\nopreamble\nopostamble
\usedir{doc/latex/proofread}
\generate{
  \file{README.txt}{\from{\jobname.dtx}{readme}}
}
\ifx\fmtname\nameofplainTeX
  \expandafter\endbatchfile
\else
  \expandafter\endgroup
\fi
%</internal>
% \fi
%
% \iffalse
%<*driver>
\ProvidesFile{proofread.dtx}
%</driver>
%<package>\NeedsTeXFormat{LaTeX2e}[1999/12/01]
%<package>\ProvidesPackage{proofread}
%<*package>
    [2015/12/07 v1.01 Commands for inserting annotations]
%</package>
%<*driver>
\documentclass{ltxdoc}
\usepackage[a4paper,margin=25mm,left=50mm,nohead]{geometry}
\usepackage[numbered]{hypdoc}
\usepackage{microtype}
\usepackage{\jobname}
\EnableCrossrefs
\CodelineIndex
\RecordChanges
\parindent0pt\parskip2ex
\makeatletter
\def\com#1{\relax}
\begin{document}
  \DocInput{\jobname.dtx}
\end{document}
%</driver>
% \fi
%
% \GetFileInfo{\jobname.dtx}
% \DoNotIndex{\newcommand, \DeclareRobustCommand, \RequirePackage, \add,
%   \advance, \begin, \bgroup, \coordinate, \def, \del, \discretionary,
%   \egroup, \end, \endinput, \endpgfextra, \fi, \fill, \footnotesize,
%   \global, \hbox, \hilite, \ifdef, \ifdim, \marginpar, \marginparmargin,
%   \marginparpush, \mbox, \newbox, \newcount, \newdimen, \p, \path,
%   \pgfextra, \rep, \sbox, \st, \the, \tikz, \tikzset, \usebox,
%   \usetikzlibrary, \y, \yel
% }
%
%\title{\textsf{proofread} --- Commands for inserting annotations\thanks{This file
%   describes version \fileversion, last revised \filedate.}
%}
%\author{Wybo Dekker\thanks{E-mail: wybo@dekkerdocumenten.nl}}
%\date{Released \filedate}
%
%\maketitle
%
%\changes{v1.00}{2015/09/06}{First public release}
%\changes{v1.01}{2015/12/07}{running counter was sometimes not advanced}
%
% \begin{abstract}\noindent
% The |proofread| package defines a few \LaTeX\ commands that are useful
% when you proofread a latex document. These allow you to easily highlight
% text and add comments in the margin. Vim escape sequences are provided
% for inserting these \LaTeX\ commands in the source. The package is based
% on code for a text highlighting command that was published by Antal S-Z
% in \href{http://tex.stackexchange.com/questions/5959}{StackExchange}.
% \end{abstract}
%
% \section{Usage}
% The commands described below display a highlighted phrase in your
% compiled document and place a comment in the margin, prefixed with a
% counter, which is indicated with \textit{n} in the following. This
% counter is useful in the communication with the author of the document.
%
% \DescribeMacro{\del}
% |\del{phrase}| displays \del{phrase} and places \textit{n}:Delete in the
% margin. In the |vim| editor, you can generate this code by selecting the
% phrase and typing |<escape>d| if you add the following line to your
% |.vimrc| file:\footnote{If you made sequences or commands for your own
% editor, please inform the author; he will include those in this
% document.}
%
% |map <Esc>d s\del{}<Esc>hp|
%
% After typing this escape sequence, you will be in normal mode, behind the
% closing brace.
% 
% \DescribeMacro{\yel}
% |\yel[comment]{phrase}| displays \yel{phrase} and places
% \textit{n}:comment in the margin. In the |vim| editor, you can generate
% this code by selecting the phrase and typing |<escape>y| if you add the
% following line to your |.vimrc| file:
% 
% |map <Esc>y s\yel[]{}<Esc>PF[li|
%
% After typing this escape sequence, you will be in insert mode between the
% square bracket pair, ready to insert the \textit{comment}.
%
% \DescribeMacro{\add}
% |\add{phrase}| displays \add{phrase} and places \textit{n}:Add in the
% margin. In the |vim| editor, you can generate this code by typing
% |<escape>a| if you add the following line to your |.vimrc| file:
% 
% |map <Esc>a a\add{}<Esc>i|
%
% After typing this escape sequence, you will be in insert mode between the
% braces pair, ready to type what should be added.
%
% \DescribeMacro{\rep}
% |\rep{phrase}{replacement}| displays \rep{phrase}{replacement} and places
% \textit{n}:was:phrase in the margin. In the |vim| editor, you can
% generate this code by selecting the phrase and typing |<escape>r| if you
% add the following line to your |.vimrc| file:
%
% |map <Esc>r s\rep{}{}<Esc>2F{plla|
%
%  After typing this escape sequence, you will be in normal mode between
%  the second pair of braces, ready to edit the new content.
%
% \DescribeMacro{\com}
% |\com{comment}| is used by |\del|, |\yel|, |\add|, and |\rep| to place
% \textit{n}:comment in the margin. You can use it to place comment in the
% margin without text highlighting. In the |vim| editor, you can insert the 
% command by typing |<escape>c| if you  add the following line to your
% |.vimrc| file:
% 
% |map <Esc>c a\com{}<Esc>i|
%
% After typing this escape sequence, you will be in insert mode between the
% braces pair, ready to type your comment.
%
% \DescribeMacro{\hilite}
% |\hilite[options]{phrase}| is the command on which the above commands are
% based. It was published by Antal S-Z in
% \href{http://tex.stackexchange.com/questions/5959}{StackExchange}.
% It highlights the phrase with the default colour, yellow, using the
% default fill opacity, 0.25; but using the options, you can change this.
% For example,
% \hilite[fill=blue,draw=yellow,opacity=.5,line width=3pt]{this phrase}
% was highlighted with blue, with a 3pt width line in yellow around it,
% with the command:
%
% |\hilite[fill=blue,draw=yellow,opacity=.5,line width=3pt}|
%
% Both the fill color and the draw color get 50\% opacity, but you can set
% each individually with the |fill opacity| and |draw opacity| options.
% See the documentation of the tikz package for more options. 
%
%\StopEventually{^^A
%  \PrintChanges
%  \PrintIndex
%}
%
% \section{Implementation}
%
%    \begin{macrocode}
%<*package>
%    \end{macrocode}

% The following code for a text highlighting command (here renamed to
% |\hilite| was published by Antal S-Z in
% \href{http://tex.stackexchange.com/questions/5959}{StackExchange}.
%    \begin{macrocode}
\RequirePackage{soul}
\RequirePackage{tikz}
\RequirePackage{etoolbox}
\usetikzlibrary{calc}
\usetikzlibrary{decorations.pathmorphing}

\newcommand{\PR@defhiliter}[3][]{%
  \tikzset{every hiliter/.style={color=#2, fill opacity=#3, #1}}%
}

\PR@defhiliter{yellow}{.25}

\newcommand{\PR@hilite@Dohilite}{
  \fill [ decoration = {random steps, amplitude=1pt, segment length=15pt}
        , outer sep = -15pt, inner sep = 0pt, decorate
        , every hiliter, this hiliter ]
        ($(begin hilite)+(0,8pt)$) rectangle ($(end hilite)+(0,-3pt)$) ;
}

\newcommand{\PR@hilite@Beginhilite}{
  \coordinate (begin hilite) at (0,0) ;
}

\newcommand{\PR@hilite@Endhilite}{
  \coordinate (end hilite) at (0,0) ;
}

\newdimen\PR@hilite@previous
\newdimen\PR@hilite@current
%    \end{macrocode}

% \begin{macro}{\hilite}
%    \begin{macrocode}
\DeclareRobustCommand*\hilite[1][]{%
  \tikzset{this hiliter/.style={#1}}%
  \SOUL@setup
  %
  \def\SOUL@preamble{%
    \begin{tikzpicture}[overlay, remember picture]
      \PR@hilite@Beginhilite
      \PR@hilite@Endhilite
    \end{tikzpicture}%
  }%
  %
  \def\SOUL@postamble{%
    \begin{tikzpicture}[overlay, remember picture]
      \PR@hilite@Endhilite
      \PR@hilite@Dohilite
    \end{tikzpicture}%
  }%
  %
  \def\SOUL@everyhyphen{%
    \discretionary{%
      \SOUL@setkern\SOUL@hyphkern
      \SOUL@sethyphenchar
      \tikz[overlay, remember picture] \PR@hilite@Endhilite ;%
    }{%
    }{%
      \SOUL@setkern\SOUL@charkern
    }%
  }%
  %
  \def\SOUL@everyexhyphen##1{%
    \SOUL@setkern\SOUL@hyphkern
    \hbox{##1}%
    \discretionary{%
      \tikz[overlay, remember picture] \PR@hilite@Endhilite ;%
    }{%
    }{%
      \SOUL@setkern\SOUL@charkern
    }%
  }%
  %
  \def\SOUL@everysyllable{%
    \begin{tikzpicture}[overlay, remember picture]
      \path let \p0 = (begin hilite), \p1 = (0,0) in \pgfextra
        \global\PR@hilite@previous=\y0
        \global\PR@hilite@current =\y1
      \endpgfextra (0,0) ;
      \ifdim\PR@hilite@current < \PR@hilite@previous
        \PR@hilite@Dohilite
        \PR@hilite@Beginhilite
      \fi
    \end{tikzpicture}%
    \the\SOUL@syllable
    \tikz[overlay, remember picture] \PR@hilite@Endhilite ;%
  }%
  \SOUL@
}
%    \end{macrocode}
% \end{macro}
% Reduce minimum vertical space between margin paragraphs; if the memoir
% class is active, use the outer margin:
%    \begin{macrocode}
\AtEndPreamble{\marginparpush2pt}
\ifdef{\marginparmargin}{\marginparmargin{outer}}{}
%    \end{macrocode}
% We need a save box and a counter for prefixing the margin paragraphs:
%    \begin{macrocode}
\newbox\PR@soulbox
\newcount\PR@markerno\PR@markerno=1
%    \end{macrocode}
% \begin{macro}{\com}
% Make a margin paragraph, in footnote fontsize, prefixed with the counter
% plus a colon:
%    \begin{macrocode}
\newcommand{\com}[1]{%
  \marginpar{%
    \footnotesize%
    \the\PR@markerno:#1%
  }%
  \global\advance\PR@markerno1%
}
%    \end{macrocode}
% \end{macro}
% \begin{macro}{\del}
%    \begin{macrocode}
\newcommand{\del}[1]{%
  \com{delete}%
  \sbox\PR@soulbox{\st{#1}}%
  \hilite[red]{{\usebox\PR@soulbox}}%
  \bgroup\egroup%
}
%    \end{macrocode}
% \end{macro}
% \begin{macro}{\yel}
%    \begin{macrocode}
\newcommand{\yel}[2][\mbox{}]{%
  \com{#1}%
  \hilite[yellow]{#2}%
  \bgroup\egroup%
}
%    \end{macrocode}
% \end{macro}
% \begin{macro}{\add}
%    \begin{macrocode}
\newcommand{\add}[1]{%
  \com{add}%
  \hilite[green,draw=blue]{#1}%
  \bgroup\egroup%
}
%    \end{macrocode}
% \end{macro}
% \begin{macro}{\rep}
%    \begin{macrocode}
\newcommand{\rep}[2]{%
  \com{was:#1}%
  \hilite[blue]{#2}%
  \bgroup\egroup%
}
\endinput
%</package>
%    \end{macrocode}
% \end{macro}
%\Finale
