% \iffalse meta-comment
% Time-stamp: <2011-04-15 10:09:34 yvon>
% Copyright (C) 2011 by Yvon Henel 
% dit "le TeXnicien de surface" 
% mail: <le.texnicien.de.surface@wanadoo.fr>
% ------------------------------------------------------------------
% 
% This file may be distributed and/or modified under the
% conditions of the LaTeX Project Public License, either version 1.2
% of this license or (at your option) any later version.
% The latest version of this license is in:
%
%    http://www.latex-project.org/lppl.txt
%
% and version 1.2 or later is part of all distributions of LaTeX 
% version 1999/12/01 or later.
% \fi
%
% \iffalse
%<*package>
\NeedsTeXFormat{LaTeX2e}[1999/12/01]
\def\PackageName{nonumonpart}  
\def\fileversion{v1}
\def\filedate{2011/04/15}
\def\docdate{2011/04/15}
\def\fileinfo{nonumonpart.sty by Le TeXnicien de surface}
\ProvidesPackage{nonumonpart}
   [\filedate\space\fileversion\space\fileinfo] 
%</package>
% \fi
% \iffalse
%<*driver>
\documentclass[a4paper]{ltxdoc}
\usepackage[latin9]{inputenc}
\usepackage[T1]{fontenc}  
\usepackage{nonumonpart}[2011/04/15]
\usepackage{nonumonpart-code}
\usepackage{xspace}
\usepackage{lmodern}
\usepackage[french,english]{babel}
\usepackage{hypdoc}
\usepackage{hologo}
% commentez la ligne suivante pour avoir un document avec le code
% \OnlyDescription
% comment out the preceding line to obtain the full code 
\EnableCrossrefs         
\CodelineIndex
\RecordChanges
\begin{document}
  \DocInput{nonumonpart.dtx}
\end{document}
%</driver>
% \fi
%
% \CheckSum{22}
% 
% \CharacterTable
%  {Upper-case    \A\B\C\D\E\F\G\H\I\J\K\L\M\N\O\P\Q\R\S\T\U\V\W\X\Y\Z
%   Lower-case    \a\b\c\d\e\f\g\h\i\j\k\l\m\n\o\p\q\r\s\t\u\v\w\x\y\z
%   Digits        \0\1\2\3\4\5\6\7\8\9
%   Exclamation   \!     Double quote  \"     Hash (number) \#
%   Dollar        \$     Percent       \%     Ampersand     \&
%   Acute accent  \'     Left paren    \(     Right paren   \)
%   Asterisk      \*     Plus          \+     Comma         \,
%   Minus         \-     Point         \.     Solidus       \/
%   Colon         \:     Semicolon     \;     Less than     \<
%   Equals        \=     Greater than  \>     Question mark \?
%   Commercial at \@     Left bracket  \[     Backslash     \\
%   Right bracket \]     Circumflex    \^     Underscore    \_
%   Grave accent  \`     Left brace    \{     Vertical bar  \|
%   Right brace   \}     Tilde         \~}
%
%
% \changes{v1}{2011/04/15}{first public version.}
%
% \DoNotIndex{\newcommand,\newenvironment,^^A
%   \i,\ae,\oe,\DeclareOption,\string,\PackageInfo,^^A
%   \ExecuteOptions, \providecommand, \newif,^^A  
%   \DeclareOption, \PackageError, \CurrentOption, \space,^^A  
%   \MessageBreak, \ExecuteOptions, \ProcessOptions,^^A  
%   \fi, \if, \or, \ifnum,\ifcase, \def, \expandafter,^^A
%   \RequirePackage, \@ctrerr, \ProcessOptions,\relax,^^A
%   \@ifundefined, \AtBeginDocument,\let,\renewcommand,^^A
%   \@gobble, \thispagestyle, \ifcsname, \part,^^A
%   \newpage, \chapter, \pagestyle, \cleardoublepage,^^A
%   \text, \textup \csname, \endcsname, \value, \else, \ensuremath}
%
% \GetFileInfo{nonumonpart.sty}
% \title{The package \textsf{nonumonpart}\thanks{This document corresponds 
%        to the file \textsf{nonumonpart}~\fileversion, dated
%        \filedate.}\\ \EXPLICATION}
% \author{Le \TeX nicien de surface\thanks{%
%                 \texttt{le.texnicien.de.surface@wanadoo.fr}}}  
%
% \maketitle
% 
% \begin{abstract}
%   This package is just a \LaTeX\ wrapper around the not so well
%   known answer to the FAQ:
%   ``\href{http://www.tex.ac.uk/cgi-bin/texfaq2html?label=nopageno}{How
%     to get rid of page numbers}''.
% 
%   The English documentation for the user is available in
%   \texttt{\PackageName-en.pdf}. 
% \end{abstract}
% \begin{otherlanguage}{frenchb}
%   \begin{abstract}
%     Cette extension n'est qu'une enveloppe \LaTeX ienne pour la
%     solution pas vraiment bien connue � la question souvent pos�e:
%     \og comment se d�barrasser du num�ro de page sur la page de titre
%     d'une partie?\fg.
% 
%     La documentation en fran�ais pour l'utilisateur est 
%     \texttt{\PackageName-fr.pdf}.  
%   \end{abstract}
% \end{otherlanguage}
% 
% \tableofcontents
% \StopEventually{}
% \section{The Code}
% \iffalse
%<*package>
% \fi
%
% To begin with, I would like to emphasize that I am \emph{not}
% the author of the code which ensures the disappearance of the page
% number on the title page of a part (command \cs{part} of
% \emph{classical} classes).
% 
%\subsection{The Options}
%
% There are no options.
% 
%\subsection{The Commands}
%
% And no end user command either. 
% 
% Nonetheless if the command \cs{part} is defined \cs{@endpart} is
% redefined. Just in case, the original definition of \cs{@endpart}
% is saved in \cs{S@VE@@endpart}.
% 
% To test the existence of \cs{part}, we use a \hologo{eTeX}
% primitive.
% \begin{macro}{\@endpart}
%   This is the macro which defines the page style on the title page
%   of a part. The code is taken from UK-FAQ. I've just put some test
%   around and delayed it at the beginning of the document, just in
%   case, once again.
%    \begin{macrocode}
\AtBeginDocument{
\ifcsname part\endcsname
  \let\S@VE@@endpart\@endpart
  \def\@endpart{\thispagestyle{empty}\S@VE@@endpart}
  \PackageInfo{nonumonpart}{%
    The command \string\@endpart\space is redefined\@gobble}
\else
  \PackageInfo{nonumonpart}{%
    As the command \string\part\space is not known I do nothing\@gobble}
\fi}
%    \end{macrocode}
% \end{macro}
% 
% And that's all for the code.
% \iffalse
%</package>
% \fi
% \Finale \PrintChanges\PrintIndex
% \iffalse
%<*doc-sty>
\ProvidesPackage{nonumonpart-tools}
\RequirePackage[a4paper,%
                margin=3.5cm,%
                marginparwidth=0.5cm]{geometry}
\RequirePackage{fancyvrb}
\AtBeginDocument{\DefineShortVerb{\|}}
\newcommand{\TO}{\textemdash\ \ignorespaces}
\newcommand{\TF}{\unskip\ \textemdash\xspace}
\newcommand{\CAD}{c.-\`a-d.\xspace}
\DeclareRobustCommand\meta[1]{%
     \ensuremath\langle
     \ifmmode \expandafter \nfss@text \fi
     {%
      \meta@font@select
      \edef\meta@hyphen@restore
        {\hyphenchar\the\font\the\hyphenchar\font}%
      \hyphenchar\font\m@ne
      \language\l@nohyphenation
      #1\/%
      \meta@hyphen@restore
     }\ensuremath\rangle}
\def\meta@font@select{\itshape}
\def\GetFileInfo#1{%
  \def\filename{#1}%
  \def\@tempb##1 ##2 ##3\relax##4\relax{%
    \def\filedate{##1}%
    \def\fileversion{##2}%
    \def\fileinfo{##3}}%
  \edef\@tempa{\csname ver@#1\endcsname}%
  \expandafter\@tempb\@tempa\relax? ? \relax\relax}
\DeclareRobustCommand\cs[1]{\texttt{\char`\\#1}}
\providecommand\marg[1]{%
  {\ttfamily\char`\{}\meta{#1}{\ttfamily\char`\}}}
\providecommand\oarg[1]{%
  {\ttfamily[}\meta{#1}{\ttfamily]}}
\providecommand\parg[1]{%
  {\ttfamily(}\meta{#1}{\ttfamily)}}
\newcommand\PrintDescribeMacro[1]{\cs{#1}}
\def\DescribeMacro{\leavevmode\@bsphack
   \begingroup\Describe@Macro}
\def\Describe@Macro#1{\endgroup
  \marginpar{\raggedleft\PrintDescribeMacro{#1}}%
  \index{#1@\PrintDescribeMacro{#1}}\@esphack\ignorespaces} 
\newcommand\DescribeEnv[1]{\@bsphack 
  \marginpar{\raggedleft\texttt{#1}}%
  \index{#1@\texttt{#1} (environnement)}%
  \index{environnement!\texttt{#1}}\@esphack\ignorespaces} 
\reversemarginpar
%</doc-sty>
% \fi
% \iffalse
%<package-sty>\ProvidesPackage{nonumonpart-code}
%<*sty-common>
\newcommand\BOP{\discretionary{}{}{}}
\newcommand\Option[1]{\textsc{#1}}
\newcommand\Pkg[1]{\textsf{#1}}
\newcommand\Sourire{\texttt{\string;-)}}
\newcommand\DescribeOption[1]{\@bsphack 
  \marginpar{\raggedleft\textsc{#1}}%
  \index{#1 (option)}%
  \index{option!\Option{#1}}\@esphack\ignorespaces} 
\newcommand\EXPLICATION{%
  \textsf{nonumonpart} = 
  \textsf{no} (page) \textsf{num}ber \textsf{on} \textsf{part} 
  title page} 
%</sty-common>
% \fi
% \iffalse
%<*doc>
\documentclass[a4paper]{article}
\usepackage{xspace,array,microtype}
\usepackage{nonumonpart}
\usepackage[latin1]{inputenc}
\usepackage[T1]{fontenc}
%<fr>\usepackage[english,frenchb]{babel}
%<en>\usepackage[frenchb,english]{babel}
\usepackage{nonumonpart-tools}
\usepackage{makeidx}
\usepackage[colorlinks=true,
            linkcolor=blue,
            urlcolor=blue,
            citecolor=blue]{hyperref}

\newcommand\Filet{%
  \par\vspace*{0.5\baselineskip}
  \par\noindent\hrulefill
  \par\vspace*{0.5\baselineskip}\par}

\makeindex
\begin{document}
\GetFileInfo{nonumonpart.sty}
%<*fr>
\title{Documentation fran�aise de l'extension
  \textsf{nonumonpart}\thanks{%
    Ce document correspond au fichier \textsf{nonumonpart}~\fileversion,
    du \filedate.}\\[12pt] \large \EXPLICATION}
%</fr>
%<*en>
\title{The English documentation of the package
  \textsf{nonumonpart}\thanks{This document corresponds to the file
    \textsf{nonumonpart}~\fileversion, dated \filedate.}\\[12pt]
  \large \EXPLICATION}
%</en>
\author{Le \TeX nicien de surface\\ 
        \href{mailto:le.texnicien.de.surface@wanadoo.fr}% 
        {le.texnicien.de.surface@wanadoo.fr}}
\date{\docdate}

\maketitle

\begin{abstract} 
%<*fr>
Cette extension n'est qu'une enveloppe \LaTeX ienne pour la solution
pas vraiment bien connue � la question souvent pos�e: \og comment se
d�barrasser du num�ro de page sur la page de titre d'une partie? \fg,
voir \href{http://www.tex.ac.uk/cgi-bin/texfaq2html?label=nopageno}{How
  to get rid of page numbers}.

Je, le \TeX nicien de Surface, ne suis l'auteur \emph{que} de l'extension
et, en aucun cas, du code lui-m�me, code fourni de longue date par la
\href{http://www.tex.ac.uk/cgi-bin/texfaq2html?introduction=yes}{FAQ de
  UK-TUG} et redonn�, de ci, de l�, sur les divers forums d�di�s � \TeX{}
et \LaTeX.

  \Filet

  \begin{otherlanguage}{english}\sffamily{}
    This is the French documentation for user of package
    \Pkg{\PackageName}. 
    The English documentation should be available as
    \texttt{\PackageName-en.pdf}.
  \end{otherlanguage}
%</fr> 
%<*en> 
This package is just a \LaTeX\ wrapper around the not so well known answer
to the FAQ:
``\href{http://www.tex.ac.uk/cgi-bin/texfaq2html?label=nopageno}{How to get
  rid of page numbers}''.

I, the \TeX nicien de Surface, am the author of the wrapping only and
\emph{not} of the code itsel which has been, for many years, available on
the \href{http://www.tex.ac.uk/cgi-bin/texfaq2html?introduction=yes}{FAQ of
  UK-TUG} and given, from time to time, on different newgroups dedicated to
\TeX\ and \LaTeX.

  \Filet

  \begin{otherlanguage}{french}\sffamily{} 
    Ceci est la documentation anglaise pour l'utilisateur de l'extension
    \Pkg{\PackageName}.  La documentation fran�aise devrait �tre disponible
    sous le nom de \texttt{\PackageName-fr.pdf}.
  \end{otherlanguage}
%</en>
\Filet
\end{abstract}
% \tableofcontents

%<*fr>
Comme signal� ci-dessus, ceci n'est ni nouveau ni rare. Le code est
connu depuis longtemps et trouvable facilement sur diverses FAQs et
archives de forums \TeX niques. Cependant, la question m'ayant �t�
pos�e r�cemment dans une formation � \LaTeX, j'ai pens� qu'en �crivant
cette petite extension je me faciliterais la vie. Il est, en effet,
plus facile de dire \og chargez \emph{cette} extension\fg plut�t que
\og allez sur le net, cherchez la bonne FAQ, trouvez la bonne question
et suivez le mode d'emploi\fg.
%</fr>
%<*en>
As stated above, this is neither new nor rare. The code has been known
for a long time and easily available on different FAQs and in
different archives of \TeX nical newsgroups. However, the question has
been recently asked to me during a session of training in \LaTeX\ and
I have thought it would help me if I wrote this tiny package. It is,
actually, easier to say ``load \emph{this} package'' than ``surf the
net, find the good FAQ, find the good question, follow the how-to''.
%</en>

%<*fr>
Cette extension n'�met aucun avertissement \TO \emph{package
  warning}\TF mais fournit une information \TO \emph{package
  info}.
%</fr>
%<*en>
This package gives no \emph{warning} but one \emph{info}.
%</en>

%<*fr>
J'esp�re que cette extension rendra quelques services. S'il �tait
possible de la rendre encore plus mince, j'esp�re �galement que l'on
voudra bien m'indiquer o� je peux apporter quelque am�lioration.
%</fr>
%<*en>
I hope this package will be of service. If it appears that it could be
possible to make even slimmer, I also hope that one would like to tell
me where I could improve it. 

Last but not least, do not hesitate to tell me where, and how, I have
offended the English language by sheer clumsiness for I certainly do
not intend any offence. \Sourire
%</en>

%<fr>\section{Utilisation}
%<en>\section{Usage}

%<*fr>
Cette extension n'a pas d'option et ne fournit aucune commande pour
l'utilisateur. Elle ne fait que red�finir \TO le code est tir� de la
page
\href{http://www.tex.ac.uk/cgi-bin/texfaq2html?label=nopageno}{How to
  get rid of page numbers} de la FAQ de UK-TUG\TF la commande interne
\cs{@endpart} de telle sorte que la page de titre d'une partie \TO
obtenue avec la commande \cs{part}\TF soit dot�e du style de page
\texttt{empty}.
%</fr>
%<*en>
This package has no option and provides no end user macro. It
redefines \TO the code is taken from the page
\href{http://www.tex.ac.uk/cgi-bin/texfaq2html?label=nopageno}{How to
  get rid of page numbers} of UK-TUG FAQ\TF the internal macro
\cs{@endpart} in order to ensures that the title page of a part \TO
obtained with the commande \cs{part}\TF has page style \texttt{empty}.
%</en>

%<*fr>
Il suffit donc de charger cette extension avec l'habituel
%</fr>
%<*en>
Suffices to load this package with the usual
%</en>
|\usepackage{nonumonpart}|. 

%<*fr>

%</fr>
%<*en>

%</en>
\end{document}
%</doc> 
% \fi
% \endinput
% Local Variables: 
% coding: iso-8859-15
% mode: doctex
% TeX-master: t
% End: 
