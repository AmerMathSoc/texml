% \iffalse meta-comment
% emarks : 2011/03/26 v1.0 - e-TeX named marks registers (FC)
%
% This work may be distributed and/or modified under the
% conditions of the LaTeX Project Public License, either
% version 1.3 of this license or (at your option) any later
% version. The latest version of this license is in
%    http://www.latex-project.org/lppl.txt
%
% This work consists of the main source file emarks.dtx
% and the derived files:
%       emarks.sty, emarks.ins, emarks.drv
% end               emarks.pdf
%
% Unpacking:
%    (a) Without emarks.ins:            etex emarks.dtx
%    (b) If emarks.ins is present:      etex emarks.ins
%    (c) If you insist on using LaTeX
%           latex \let\install=y% \iffalse meta-comment
% emarks : 2011/03/26 v1.0 - e-TeX named marks registers (FC)
%
% This work may be distributed and/or modified under the
% conditions of the LaTeX Project Public License, either
% version 1.3 of this license or (at your option) any later
% version. The latest version of this license is in
%    http://www.latex-project.org/lppl.txt
%
% This work consists of the main source file emarks.dtx
% and the derived files:
%       emarks.sty, emarks.ins, emarks.drv
% end               emarks.pdf
%
% Unpacking:
%    (a) Without emarks.ins:            etex emarks.dtx
%    (b) If emarks.ins is present:      etex emarks.ins
%    (c) If you insist on using LaTeX
%           latex \let\install=y% \iffalse meta-comment
% emarks : 2011/03/26 v1.0 - e-TeX named marks registers (FC)
%
% This work may be distributed and/or modified under the
% conditions of the LaTeX Project Public License, either
% version 1.3 of this license or (at your option) any later
% version. The latest version of this license is in
%    http://www.latex-project.org/lppl.txt
%
% This work consists of the main source file emarks.dtx
% and the derived files:
%       emarks.sty, emarks.ins, emarks.drv
% end               emarks.pdf
%
% Unpacking:
%    (a) Without emarks.ins:            etex emarks.dtx
%    (b) If emarks.ins is present:      etex emarks.ins
%    (c) If you insist on using LaTeX
%           latex \let\install=y% \iffalse meta-comment
% emarks : 2011/03/26 v1.0 - e-TeX named marks registers (FC)
%
% This work may be distributed and/or modified under the
% conditions of the LaTeX Project Public License, either
% version 1.3 of this license or (at your option) any later
% version. The latest version of this license is in
%    http://www.latex-project.org/lppl.txt
%
% This work consists of the main source file emarks.dtx
% and the derived files:
%       emarks.sty, emarks.ins, emarks.drv
% end               emarks.pdf
%
% Unpacking:
%    (a) Without emarks.ins:            etex emarks.dtx
%    (b) If emarks.ins is present:      etex emarks.ins
%    (c) If you insist on using LaTeX
%           latex \let\install=y\input{emarks.dtx}
%        (quote the arguments according to the demands of your shell)
%
% Documentation:                        pdflatex emarks.dtx
% Copyright (C) 2011 by FC <florent.chervet @t free.fr>
%<*ignore>
\begingroup
  \def\x{LaTeX2e}%
\expandafter\endgroup
\ifcase 0\ifx\install y1\fi\expandafter
         \ifx\csname processbatchFile\endcsname\relax\else1\fi
         \ifx\fmtname\x\else 1\fi\relax
\else\csname fi\endcsname
%</ignore>
%<*install>
\input docstrip.tex
\Msg{************************************************************************}
\Msg{* Installation                                                         *}
\Msg{* Package emarks: 2011/03/26 v1.0 - e-TeX named marks registers (FC)   *}
\Msg{************************************************************************}

\keepsilent
\askforoverwritefalse

\let\MetaPrefix\relax
\preamble

This is a generated file.

emarks : 2011/03/26 v1.0 - e-TeX named marks registers (FC)

This work may be distributed and/or modified under the
conditions of the LaTeX Project Public License, either
version 1.3 of this license or (at your option) any later
version. The latest version of this license is in
   http://www.latex-project.org/lppl.txt

This work consists of the main source file emarks.dtx
and the derived files:
                emarks.sty, emarks.ins, emarks.drv,
        and:                emarks.pdf

emarks : 2011/03/26 v1.0 - e-TeX named marks registers (FC)
Copyright (C) 2011 by FC <florent.chervet @t free.fr>

\endpreamble
\let\MetaPrefix\DoubleperCent

\generate{%
   \file{emarks.ins}{\from{emarks.dtx}{install}}%
   \file{emarks.sty}{\from{emarks.dtx}{package}}%
}

\askforoverwritefalse
\generate{%
   \file{emarks.drv}{\from{emarks.dtx}{driver}}%
}

\obeyspaces
\Msg{************************************************************************}
\Msg{*                                                                      *}
\Msg{* To finish the installation you have to move the following            *}
\Msg{* file into a directory searched by TeX:                               *}
\Msg{*                                                                      *}
\Msg{*                        emarks.sty                                    *}
\Msg{*                                                                      *}
\Msg{* To produce the documentation run the file `emarks.dtx'               *}
\Msg{*                     through pdfLaTeX.                                *}
\Msg{*                                                                      *}
\Msg{************************************************************************}
\endbatchfile
%</install>
%<*ignore>
\fi
%</ignore>
%<*driver>
\def\thisinfo {e-TeX named marks registers (FC)}
\def\thisversion {1.0}
\PassOptionsToPackage {full}{tabu}
\RequirePackage [\detokenize{��},hyperlistings]{fcltxdoc}
\AtBeginDocument{\embedfile{README}}
%%\CheckDates{interfaces=2011/02/12,tabu=2011/02/25}
\documentclass[a4paper,11pt,twoside,american,latin1,T1]{ltxdoc}     \usetikz{full}
\usepackage [latin1]{inputenc}
\usepackage [T1]{fontenc}
\usepackage {numprint}
\usepackage {pdfcomment}
\usepackage {ragged2e}   % general tools
\usepackage {arial,bbding,relsize,moresize,manfnt,pifont,upgreek} % fonts
\csname endofdump\endcsname
\usepackage {emarks}
\RequirePackage [full]{tabu}
\usepackage {geometry}
\AtBeginDocument {\let\setkeys \kvsetkeys }
\let\microtypeYN=n
\ifx y\microtypeYN                                                          %
   \usepackage[expansion=all,stretch=20,shrink=60]{microtype}\fi            % font (microtype)
\CodelineNumbered\lastlinefit999
\lstset{backgroundcolor=\color{LightYellow},
texcsstyle=\color{blue},
moretexcs=[1]{
    lstdefinestyle,
    lstinputlisting,lstset,tikzlabel,tikzrefXY,
    color,
    geometry,lasthline,firsthline,
    cmidrule,toprule,bottomrule,tabusetup*,tabusetup,
    everyrow,tabulinestyle,tabureset,savetabu,usetabu,preamble,
    taburulecolor,taburowcolors},
keywordstyle=[3]{\color{black}\bfseries},
morekeywords=[3]{&},
keywordstyle=[4]{\color{red}\bfseries},
morekeywords=[4]{\linegoal,$},
keywordstyle=[5]{\color{blue}\bfseries},
keywordstyle=[6]{\color{green}\bfseries},
keywordstyle=[7]{\color{yellow}\bfseries},
%extendedchars={true},
alsoletter={&},alsoletter={*},alsoletter={$},
morekeywords=[5]{blue},
morekeywords=[6]{green},
morekeywords=[7]{yellow},
}
\hypersetup {%
  pdfauthor=Florent CHERVET,
  pdfkeywords={TeX, LaTeX, e-TeX, marks, firstmarks, botmarks, topmarks, package },
}
\geometry {top=0mm,headheight=8mm,includehead,reversemarginpar,asymmetric,headsep=3mm,bottom=14mm,footskip=5mm,inner=35mm,outer=20mm }
\begin{document}
   \DocInput{\jobname.dtx}
\end{document}
%</driver>
% \fi
%
% \CheckSum{219}
% \CharacterTable
%  {Upper-case    \A\B\C\D\E\F\G\H\I\J\K\L\M\N\O\P\Q\R\S\T\U\V\W\X\Y\Z
%   Lower-case    \a\b\c\d\e\f\g\h\i\j\k\l\m\n\o\p\q\r\s\t\u\v\w\x\y\z
%   Digits        \0\1\2\3\4\5\6\7\8\9
%   Exclamation   \!     Double quote  \"     Hash (number) \#
%   Dollar        \$     Percent       \%     Ampersand     \&
%   Acute accent  \'     Left paren    \(     Right paren   \)
%   Asterisk      \*     Plus          \+     Comma         \,
%   Minus         \-     Point         \.     Solidus       \/
%   Colon         \:     Semicolon     \;     Less than     \<
%   Equals        \=     Greater than  \>     Question mark \?
%   Commercial at \@     Left bracket  \[     Backslash     \\
%   Right bracket \]     Circumflex    \^     Underscore    \_
%   Grave accent  \`     Left brace    \{     Vertical bar  \|
%   Right brace   \}     Tilde         \~}
%
% \DoNotIndex{\globcount,\globdimen,\if,\fi,\else,\def,\the,\gdef,\global,\relax}
% \makeatletter
% \def\ThisInfo {\ifdefvoid\lsstyle {\scalebox{1.35}[1]{\eTeX}\stretchwith\,{\ named\ marks\ registers}}
%                                                                  {\lsstyle \eTeX{} named marks registers }}
% \def\ttdefault{lmvtt}         \colorlet{pkgcolor}{LimeGreen!50!Black}
% \parindent\z@\parskip.4\baselineskip\topsep\parskip\partopsep\z@
% \newrobustcmd*\FC {{\leavevmode\color{copper}\usefont{T1}{fts}xn FC}}
% \colorlet{linkcolor}{DarkSlateBlue}   \colorlet{csrefcolor}{linkcolor}^^ARoyalBlue!70!Indigo!50!Black}
% \definecolor{macrocode}{rgb}{0.08,0.00,0.15}
% \providerobustcmd*\csred{\cs[\colorlet{csrefcolor}{red}]}
% \def\MacroFont{\ttfamily\bfseries }
% \def\macro@font {\def\Cr@scale{.87}\changefont{fam=pcrs,siz=10pt,ser=m,color=macrocode,spread=1}\let\AltMacroFont\macro@font }
% \AtBeginEnvironment {declcs}{\tabusetup* {font=\bfseries,everyrow=\rowbackground{fill=Snow!220,cell fading=-+{70}{0}{30}},line style=\heavyrulewidth,linesep=1mm}}
% \AtBeginEnvironment {lstlisting}{\Needspace{7\baselineskip}}
% \tikzAtEveryShipout {\ifnum \value{page}>\@ne  \fill [fill=Lime,very nearly transparent] (0,0) rectangle (\paperwidth,-\headheight);\fi }
%  \sectionformat\section[hang]{
%       left=\declmarginwidth,
%       font=\bfseries\Large,
%       bookmark={color=pkgcolor},
%       bottom=\smallskipamount,top=\medskipamount,
%  }
%  \sectionformat\subsection{
%       font=\large\bfseries,
%       bookmark={color=MidnightBlue},
%  }
% \pagesetup [corpus]{norule,
%       font=\footnotesize,
%       head/left=\noindent\raise1.5mm\hbox{\thispackage},
%       head/font+=\mdseries\sffamily,
%       head/right=\noindent\raise1.5mm\hbox{\ThisInfo},
%       foot/left/font=\scriptsize\color{gray!80},
%       foot/left=\vbox to\baselineskip{\vss{{\rotatebox[origin=l]{90}{\thispackage\,[rev.\thisversion]\,\CopyRight2011\,\lower.4ex\hbox{\pkgcolor\NibRight}\,\FC}}}},
%       offset=15mm,
%       left/offset+=15mm,
%       foot/right=\oldstylenums{\arabic{page}}/\oldstylenums{\pageref{LastPage}},
%  }
% \pagesetup [plain]{%
%     norules,font=\scriptsize,
%     offset=15mm,
%     left/offset+=15mm,
%     foot/font=\scriptsize\color[gray]{.55},
%     foot/right=\oldstylenums{\arabic{page}}/\oldstylenums{\pageref{LastPage}},
%     foot/left=\vbox to\baselineskip{\vss{{\rotatebox[origin=l]{90}{\thispackage\,[rev.\thisversion]\,\CopyRight2011\,\lower.4ex\hbox{\pkgcolor\NibRight}\,\FC\quad \xemail{florent.chervet at free.fr}}}}},
% }
% \bookmarksetup{openlevel=3}
%
% \newrobustcmd\IMPLEMENTATION{\bigskip
%       \par \tabubox { X[c] }{ \background {cell fading=fuzzy ring 15 percent,fill=Silver,semitransparent}
%               \tabubox {  *2X[c]  } {\includegraphics [width=20mm,keepaspectratio]{emarks-fingerprint.png}
%                   & \indent\tikz{\fill [decorate,decoration={footprints,foot of=felis silvestris}] (\paperwidth-25mm,-32mm) circle (+8mm);}
%       \\[^2mm _8mm]}
%       \\ \LARGE\thispackage }
%       \clearpage
%       \bookmarksetup {bold*,openlevel=1}
%       \sectionformat \section{bookmark={color=black}} \sectionformat \subsection{bookmark={color=gray}}
%       \section(Implementation)[\lsstyle\textsc{\bfseries Implementation}]{\larger\lsstyle\textsc{\bfseries Implementation}}\label{sec:implementation}\parindent1em
%  }
%
% \tikzAtFirstShipout{\fill [decorate,decoration={footprints,foot of=felis silvestris}] (\paperwidth-25mm,-32mm) circle (+5mm);
%                     \node[anchor=south] at (+41mm,-40mm) {\includegraphics [width=15mm,keepaspectratio] {emarks-fingerprint.png}};}
%
% \title  {\vspace*{-28pt}\Huge\bfseries \CTANhref[emarks]{\pkgcolor emarks}\Footnotemark{*}
%          \tabusetup* {linesep=3mm ,font=\large\changefont{fam=txr}}
%   \tabubox { X[c] }{ \ThisInfo \\
%                       \small\FC   \\
%                       \small\mdseries\thisdate~--~\hyperref[\thisversion]{version \thisversion }
%                    }\vspace*{-12pt}}
% \author {}
% \date   {}
% \makeatother
%
% \maketitle            
%
% \pdftikzcomment [author=interfaces-pdfcomment] {TikZ decorations.footprint library}
%   [semithick,color=pkgcolor,->>] ($(current page.north east)-(7mm,7mm)$) -- ++ (-1.5,-1.5);
%
% \bookmark[bold,view=FitH 0,named=FirstPage,color=pkgcolor]{emarks}
% 
% \Footnotetext{\rlap{*}\kern2em}{\parindent0pt\noindent
% This documentation is produced with the \textt{DocStrip} utility.\par
% \begin{tabu}{X[-3]X[-1] >\ttfamily X}
% \smex To get the package,                   &run:          & etex \thisfile.dtx                  \\
% \smex To get the documentation              &run (thrice): & pdflatex \thisfile.dtx               \\
% \leavevmode\hphantom\smex To get the index, &run:          & makeindex -s gind.ist \thisfile.idx
% \end{tabu}\par
% The \xext{dtx}* is embedded into this \xext{pdf}* thank to \Xpackage{embedfile} by H. Oberdiek.}
%
% \vspace*{-16mm}
% \begin{Abstract}[\leftmargin=6mm\parindent0pt\listparindent0pt\parskip\smallskipamount\parsep\smallskipamount]
%
% \shorttabubox { X }{\eTeX{} defines \numprint {32768} marks registers while \TeX{} provided only one \emph!\\ \cbackground {fill=pkgcolor,cell fading=*,nearly transparent}-}
%
% So small, this package provides commands to access \eTeX{} marks registers by their name rather than by their number.
% This makes the use of them far more comfortable than ``old \LaTeX{}'' tricks with \cs\markright, \cs\markboth \etc.
%
% \thispackage requires \eTeX{} and the generic package \CTANhref[etex-pkg]{\xfile{etex.sty}} for allocation.
%
%  Presently designed to be loaded by \textt\LaTeX{}, a \textt{plain \TeX} version might be provided later...
%
% \end{Abstract}
%
% \tocsetup{
%  before+=\hypersetup {linkcolor=black},
%  section/skip=4pt plus2pt minus2pt,
%  subsection/skip=0pt plus2pt minus2pt,
%  section/dotsep=,
%  subsection/dotsep=,
%  subsection/pagenumbers=off,
%  dotsep=1.5mu,
%  title/top = 12pt,
%  dot=\hbox{$\scriptscriptstyle\cdotp$},
%  title={\pkgcolor\leaders\vrule height3.4pt depth-3pt\hfill\null}\quad Contents of \textsfbf{\pkgcolor emarks}\quad{\pkgcolor\leaders\vrule height3.4pt depth-3pt\hfill\null},
%  title/bottom=4pt,
%  multicols/beforeend=\aftergroup\tocrule,
%  columns=2,columns/rule color=pkgcolor,no columns rule,
%  }
%  \def\tocrule{\leavevmode{\pkgcolor\hrule}}
%
% \tableofcontents  \pagestyle{corpus}
%
% \listofsetup {lol}{pagestyle=,
%       title/font=\large\bfseries,
%       title={\pkgcolor\leaders\vrule height3.4pt depth-3pt\hfill\null}\quad List of the listings / examples\quad{\pkgcolor\leaders\vrule height3.4pt depth-3pt\hfill\null},
%       after=\tocrule,
%       twocolumns=false,
%       label=lol,bookmark={text=Listings},
%       lstlisting/dotsep=,
%       lstlisting/pagenumbers=off,
%       lstlisting/leaders=,
%       lstlisting/font=\hfil,
%       lstlisting/number/width=0pt,
% }
%
% \addcontentsline{lol}{lstlisting}{To be done !}
%
% \listoflstlisting
%
% \bookmarksetup{bold*}
%
% \section{The \hologo{eTeX} marks registers}
% \label{userinterface}
%
%
% \begin{declcs}[{ | *2X[-1] | }] \marksthe\M{named-mark}\M{content} & \csanchor\marksthecs\M{named-mark}\M{cs-name} \\
%                        \cs\marksthe*\M{named-mark}\M{content}      &       \cs\marksthecs*\M{named-mark}\M{cs-name}
% \end{declcs}\declcsbookmark\marksthe\declcsbookmark\marksthecs
%
%
% \tabusetup* { extra sep = .5\parskip }
% \declmargin\begin{tabu*}to\dimexpr\linewidth-\declmarginwidth {X[-1]X @{} }
%   \cs\marksthe\M*{section}\M{content} & Marks the \meta{content} into the named mark register \meta{section} in the same way as the \eTeX{} primitive \cs\marks:
%                                         in particular the \meta{content} is immediately expanded. \par
%                                         If the mark register does not exist, it is created (or allocated) with \cs\newmarks (in \xfile{etex.sty}).
%  \\
%  \cs\marksthe*\M*{section}\M{content} & does the same but the \meta{content} is not expanded. The current values of counters, \cs\thesection \etc. will be wrong:
%                                         they will expand to the value they have at the time the mark register is read, not at the time of \cs\marksthe*.\par
%                                         Yet \cs\marksthe* is useful to mark a title only like in�
%                                         \shorttabubox* {X}{ \cs\def\cs\sectionmark \#1\M*{\cs\marksthe*\M*{section}\M{\#1}}}�
%                                         or to control the expansion (the \meta{content} can be expanded before marking in a way and with the protections desired by the user).
% \end{tabu*}
%
% Similarly \cs\marksthecs\M{subsubsection}\M{cs-name} marks the content of \cs{cs-name} by the mean of the named mark
% register \meta{subsubsection}. \meta{cs-name} is really the \emph{name of the control sequence} and not the control sequence itself:
% it does not start with \textttbf{\csname @backslashchar\endcsname}.�
% If \cs{cs-name} is empty the mark is empty, but if it is undefined or \cs\relax: nothing is marked: at reading time, the mark register never expands
% to \cs\undefined nor to \cs\relax.
%
% The syntax follows \eTeX{} \cs\marks primitive (a token-like syntax): braces are mandatory around the \M{content} to be marked, even if it is made of one single token.
%
%   \begin{declcs}[{ | *2X[-1] |}]\thefirstmarks&\M{named-mark}\textsuperscript{\textsc{expandable}} \\
%                \csanchor\thebotmarks  &\M{named-mark}\textsuperscript{\textsc{expandable}} \\
%                \csanchor\thetopmarks  &\M{named-mark}\textsuperscript{\textsc{expandable}}
% \end{declcs}\declcsbookmark\thefirstmarks\declcsbookmark\thebotmarks\declcsbookmark\thetopmarks
%
% Those commands are expandable in exactly one step of expansion. If the \meta{named-mark} mark register does not exists,
% the expansion is null (\ie nothing is done nor printed).
%
% \tabusetup* {colsep=3pt}
% \begin{tabu*}{ @{} X[-1] X @{} }
% \cs\thefirstmarks\M{chapter}  &expands to the content of the first invocation of \cs\marksthe\M{chapter}
%   on the current page if \cs\marksthe\M*{chapter} was used on the current page,
%   or the last invocation of \cs\marksthe\M*{chapter} if no marks occured on the current page.
% \\
%   \hfill\small \TeX nically this is &\cs\firstmarks \cs\marks@chapter
% \\[1mm]
% \cs\thebotmarks\M{chapter}    &expands to the content of the last invocation of \cs\marksthe\M{chapter} (the most recent \cs\marks). \\
%   \hfill\small \TeX nically this is &\cs\botmarks \cs\marks@chapter
% \\[1mm]
% \cs\thetopmarks\M{chapter}    &expands to the content of \cs\botmarks at the time \TeX{} shipped out the last page. \\
%  \hfill\small \TeX nically this is  &\cs\topmarks\cs\marks@chapter
% \end{tabu*}
%
% \def\OR{\stform|}
% \begin{declcs}[{ | X | }]\getthemarks  \cs\firstmarks \OR \cs\botmarks \OR \cs\topmarks \M{named-mark} \M*{\cs{control-sequence}} \\
%      \csanchor\getthefirstmarks \M{named-mark} \M*{\cs{control-sequence}} \\
%      \csanchor\getthebotmarks \M{named-mark} \M*{\cs{control-sequence}} \\
%      \csanchor\getthetopmarks \M{named-mark} \M*{\cs{control-sequence}}
% \end{declcs}\declcsbookmark\getthemarks\declcsbookmark[rellevel=2]\getthefirstmarks\declcsbookmark[rellevel=2]\getthebotmarks\declcsbookmark[rellevel=2]\getthetopmarks
%
% \cs\thefirstmarks, \cs\thebotmarks and \cs\thetopmarks expand the content of the mark. To get it in a macro
% \cs\getthemarks can be used: \cs{control-sequence} is defined as a parameterless macro whose replacement text is
% the content of the given mark register.
%
% If the \meta{named-mark} mark register does not exist, the meaning of\, \cs{control-sequence}\, is
% \textt{undefined}.
%
% \begin{declcs}\ifmarksvoid \M*{\cs\firstmarks}\M*{named-mark}\M{true}\M{false} \\
%            \cs\ifmarksvoid \M*{\cs\botmarks}\M*{named-mark}\M{true}\M{false} \\
%            \cs\ifmarksvoid \M*{\cs\topmarks}\M*{named-mark}\M{true}\M{false} \\
% \end{declcs}\declcsbookmark\ifmaksvoid
%
% \cs\ifmarksvoid expands the \M{true} part if either: ^^A\loggingall
% \begin{itemize}^^A[topsep=0pt,itemsep=0pt]
% \item The requested mark register is empty,
% \item The requested mark register is \cs\undefined,
% \item The requested mark register is \cs\relax,
% \item The \meta{named-mark} mark register does not exist.
% \end{itemize}
%
%
%
% \begin{declcs}\ifmarksequal\M*{\cs\firstmarks}\M*{\cs\topmarks}\M*{named-mark}\M{true}\M{false}   \\
%            \cs\ifmarksequal\M*{\cs\firstmarks}\M*{\cs\botmarks}\M*{named-mark}\M{true}\M{false}
% \end{declcs}\declcsbookmark\ifmarksequal
%
% Pretty often we want to compare the botmarks against the firstmarks or the topmarks, to adapt the header and/or footer
% in case those marks are equal or different, \ie in case the page contains a new section title or not:
%
% \cs\ifmarksequal expands the code in the \M{true} or the \M{false} part if the extraction of the marks are equal
% (in the sense of \cs\ifx) or different.
%
% If any of the marks register\, \cs{marks@\meta{named-mark}}\, does not exist the \M{false} part is expanded.
%
% If marks are used both at \cs\sectionmark \textbf{and at} \cs\sectionbreak then the following assertions are true:�
% \begin{shorttabu}{ !\textbullet  l  @{\,=\,}  l  !{$\Leftrightarrow$} l  }
% \cs\firstmarks &\cs\botmarks &there is at most one section title on the current page; \\
% \cs\topmarks   &\cs\botmarks &there is no section title on the current page;          \\
% \cs\firstmarks &\cs\topmarks &the last section title continues on the current page.
% \end{shorttabu}
%
%
% \begin{declcs}\showthemarks \M{named-mark}
% \end{declcs}\declcsbookmark\showthemarks
%
% \cs\showthemarks is for debugging purpose: it prints a message in the \xext{log}* and the ``standard error'' with
% the contents of the marks \cs\firstmarks, \cs\botmarks and \cs\topmarks for the \meta{named-mark} register given.
% Then it executes \cs\show on the extracted content of \cs\firstmarks in order to stop compilation at that point:
% the console displays the contents of \cs\firstmarks, \cs\botmarks and \cs\topmarks.
%
% ^^A\loggingall
% ^^A\showthemarks{section}
%
% \StopEventually{ }
%
% \IMPLEMENTATION
%
% \subsection*{Identification}       \makeatletter
%
% The package namespace is \cs\em@rks
%
%    \begin{macrocode}
%<*package>
\NeedsTeXFormat{LaTeX2e}[2005/12/01]
\ProvidesPackage{emarks}
         [2011/03/26 v1.0 - e-TeX named marks registers (FC)]
\RequirePackage {etex}
%    \end{macrocode}
%
%    \begin{macro}{\emarks@newmarks}
%
%    allocates a new marks register if it does not exists.
%
%    \begin{macrocode}
\def\emarks@newmarks #1{\PackageInfo {emarks}{New marks register `#1'}%
                        \newmarks #1% \newmarks is global !!
}% \emarks@newmarks
%    \end{macrocode}
%    \end{macro}
%
%    \begin{macro}{\marksthe}
%    \begin{macro}{\marksthecs}
%
%   \noindent\shorttabubox* { X } { \lstinline! \marksthe   { named-mark }{ general text } ! \\
%                          \lstinline! \marksthe*  { named-mark }{ general text } ! \\
%                          \lstinline! \marksthe   { named-mark }{ named control sequence } ! \\
%                          \lstinline! \marksthecs*{ named-mark }{ named control sequence } ! \\
%                        }
%
%    \begin{macrocode}
\protected\def\marksthe   {\emarks@setmarks {}}
\protected\def\marksthecs {\emarks@setmarks {\toks@\expandafter{\csname\the\toks@\endcsname}}}
\def\emarks@setmarks #1{\begingroup \@ifstar {\emarks@ {#1}\def  }
                                             {\emarks@ {#1}\edef }%
}% \emarks@setmarks
\def\emarks@ #1#2#3{\def\@tempa
      {#1#2\@tempa {\the\toks@ }\expandafter\emarks@marks \csname marks@#3\endcsname }%
                                                \afterassignment \@tempa \toks@ =
}% \emarks@
\def\emarks@marks #1{\ifx \relax#1\emarks@newmarks #1\fi \marks #1{\@tempa }\endgroup }
%    \end{macrocode}
%    \end{macro}
%    \end{macro}
%
%    \begin{macro}{\thefirstmarks}
%    \begin{macro}{\thebotmarks}
%    \begin{macro}{\thetopmarks}
%
%   \cs\thefirstmarks extract the \cs\firstmarks from a named mark register.
%
%   The macros are purely expandable in exactly one step of expansion.
%
%    \begin{macrocode}
\newcommand*\thefirstmarks {\romannumeral \emarks@themarks \firstmarks  }
\newcommand*\thebotmarks   {\romannumeral \emarks@themarks \botmarks    }
\newcommand*\thetopmarks   {\romannumeral \emarks@themarks \topmarks    }
\def\emarks@themarks #1#2{\expandafter \ifx
    \csname\ifcsname marks@#2\endcsname marks@#2\else relax\fi\endcsname\relax
            \expandafter \z@
    \else   \expandafter \z@ #1\csname marks@#2\expandafter \endcsname  \fi
}% \emarks@themarks
%    \end{macrocode}
%    \end{macro}
%    \end{macro}
%    \end{macro}
%
%    \begin{macro}{\getthemarks}
%    \begin{macro}{\getthefirstmarks}
%    \begin{macro}{\getthebotmarks}
%    \begin{macro}{\getthetopmarks}
%
%    Extract the marks and store in a parameterless macro.
%
%    \begin{macrocode}
\protected\def\getthemarks #1#2#3{\ifcsname marks@#2\endcsname
          \expandafter \def \expandafter #3\expandafter {#1\csname marks@#2\endcsname}%
    \else                           \let #3=\@undefined    \fi
}% \getthemarks
\protected\def\getthefirstmarks {\getthemarks   \firstmarks }
\protected\def\getthebotmarks   {\getthemarks   \botmarks   }
\protected\def\getthetopmarks   {\getthemarks   \topmarks   }
%    \end{macrocode}
%    \end{macro}
%    \end{macro}
%    \end{macro}
%    \end{macro}
%
%    \begin{macro}{\ifmarksvoid}
%
%   Test if a marks is defined, not empty and not \cs\relax.
%
%    \begin{macrocode}
\protected\def\ifmarksvoid #1#2{\begingroup \getthemarks {#1}{#2}\x
    \ifodd \ifdefined\x \ifx \x\relax 0 \fi \ifx \x\@empty 0 \fi \else 0 \fi
           1 \endgroup\expandafter\@secondoftwo
    \else    \endgroup\expandafter\@firstoftwo      \fi
}% \ifmarksvoid
%    \end{macrocode}
%    \end{macro}
%
%    \begin{macro}{\ifmarksequal}
%
%   Test with \cs\ifx if two marks are equal:�
%   \shorttabubox* {X} { \lstinline ! \ifmarksequal \firstmarks \botmarks { named-mark } ! }
%
%
%    \begin{macrocode}
\protected\def\ifmarksequal #1#2#3{\begingroup \getthemarks{#1}{#3}\x \getthemarks{#2}{#3}\y
        \expandafter \endgroup \ifodd \ifdefined\x \ifdefined\y \ifx \x\y 0 \fi\fi\fi
                                      1 \expandafter\@secondoftwo
                               \else    \expandafter\@firstoftwo     \fi
}% \ifmarksequal
%    \end{macrocode}
%    \end{macro}
%
%    \begin{macro}{\showthemarks}
%
%    Shows the contents of the marks registers
%
%    \begin{macrocode}
\protected\def\showthemarks #1{\begingroup  \emarks@showthemarks 0{#1}\firstmarks
                                            \emarks@showthemarks 2{#1}\botmarks
                                            \emarks@showthemarks 4{#1}\topmarks
    \message{firstmarks "#1": \the\toks0^^J%
             botmarks   "#1": \the\toks2^^J%
             topmarks   "#1": \the\toks4^^J}\show\@tempa
    \endgroup
}% \showthemarks
\def\emarks@showthemarks #1#2#3{\getthemarks #3{#2}\@tempa \toks #1 = \ifdefined\@tempa
    \expandafter\ifx \noexpand\@tempa\@tempa {}\else \expandafter {\@tempa }\fi
                                                                      \else {}\fi
}% \emarks@showthemarks
%    \end{macrocode}
%    \end{macro}
%
%    \begin{macrocode}
%</package>
%    \end{macrocode}
%
%
% \begin{History}
%   \sectionformat\subsection{font=\normalsize\pkgcolor,bottom=0pt,top=\smallskipamount }\makeatletter
%
%   \begin{Version}{2011/03/26}{1.0}
%   \item First version. \\
%   \end{Version}
%
% \end{History}
%
% \begin{thebibliography}{9}
%
% \bibitem{etex} The \xpackage{etex} package by Peter Breitenlohner \\
%       \getpackageinfo{etex} \\
%       \CTANhref[etex-pkg]{\nolinkurl{CTAN:help/Catalogue/entries/etex-pkg.html}}
%
% \end{thebibliography}
%
% \clearpage
% \PrintIndex
%
% \Finale
%        (quote the arguments according to the demands of your shell)
%
% Documentation:                        pdflatex emarks.dtx
% Copyright (C) 2011 by FC <florent.chervet @t free.fr>
%<*ignore>
\begingroup
  \def\x{LaTeX2e}%
\expandafter\endgroup
\ifcase 0\ifx\install y1\fi\expandafter
         \ifx\csname processbatchFile\endcsname\relax\else1\fi
         \ifx\fmtname\x\else 1\fi\relax
\else\csname fi\endcsname
%</ignore>
%<*install>
\input docstrip.tex
\Msg{************************************************************************}
\Msg{* Installation                                                         *}
\Msg{* Package emarks: 2011/03/26 v1.0 - e-TeX named marks registers (FC)   *}
\Msg{************************************************************************}

\keepsilent
\askforoverwritefalse

\let\MetaPrefix\relax
\preamble

This is a generated file.

emarks : 2011/03/26 v1.0 - e-TeX named marks registers (FC)

This work may be distributed and/or modified under the
conditions of the LaTeX Project Public License, either
version 1.3 of this license or (at your option) any later
version. The latest version of this license is in
   http://www.latex-project.org/lppl.txt

This work consists of the main source file emarks.dtx
and the derived files:
                emarks.sty, emarks.ins, emarks.drv,
        and:                emarks.pdf

emarks : 2011/03/26 v1.0 - e-TeX named marks registers (FC)
Copyright (C) 2011 by FC <florent.chervet @t free.fr>

\endpreamble
\let\MetaPrefix\DoubleperCent

\generate{%
   \file{emarks.ins}{\from{emarks.dtx}{install}}%
   \file{emarks.sty}{\from{emarks.dtx}{package}}%
}

\askforoverwritefalse
\generate{%
   \file{emarks.drv}{\from{emarks.dtx}{driver}}%
}

\obeyspaces
\Msg{************************************************************************}
\Msg{*                                                                      *}
\Msg{* To finish the installation you have to move the following            *}
\Msg{* file into a directory searched by TeX:                               *}
\Msg{*                                                                      *}
\Msg{*                        emarks.sty                                    *}
\Msg{*                                                                      *}
\Msg{* To produce the documentation run the file `emarks.dtx'               *}
\Msg{*                     through pdfLaTeX.                                *}
\Msg{*                                                                      *}
\Msg{************************************************************************}
\endbatchfile
%</install>
%<*ignore>
\fi
%</ignore>
%<*driver>
\def\thisinfo {e-TeX named marks registers (FC)}
\def\thisversion {1.0}
\PassOptionsToPackage {full}{tabu}
\RequirePackage [\detokenize{��},hyperlistings]{fcltxdoc}
\AtBeginDocument{\embedfile{README}}
%%\CheckDates{interfaces=2011/02/12,tabu=2011/02/25}
\documentclass[a4paper,11pt,twoside,american,latin1,T1]{ltxdoc}     \usetikz{full}
\usepackage [latin1]{inputenc}
\usepackage [T1]{fontenc}
\usepackage {numprint}
\usepackage {pdfcomment}
\usepackage {ragged2e}   % general tools
\usepackage {arial,bbding,relsize,moresize,manfnt,pifont,upgreek} % fonts
\csname endofdump\endcsname
\usepackage {emarks}
\RequirePackage [full]{tabu}
\usepackage {geometry}
\AtBeginDocument {\let\setkeys \kvsetkeys }
\let\microtypeYN=n
\ifx y\microtypeYN                                                          %
   \usepackage[expansion=all,stretch=20,shrink=60]{microtype}\fi            % font (microtype)
\CodelineNumbered\lastlinefit999
\lstset{backgroundcolor=\color{LightYellow},
texcsstyle=\color{blue},
moretexcs=[1]{
    lstdefinestyle,
    lstinputlisting,lstset,tikzlabel,tikzrefXY,
    color,
    geometry,lasthline,firsthline,
    cmidrule,toprule,bottomrule,tabusetup*,tabusetup,
    everyrow,tabulinestyle,tabureset,savetabu,usetabu,preamble,
    taburulecolor,taburowcolors},
keywordstyle=[3]{\color{black}\bfseries},
morekeywords=[3]{&},
keywordstyle=[4]{\color{red}\bfseries},
morekeywords=[4]{\linegoal,$},
keywordstyle=[5]{\color{blue}\bfseries},
keywordstyle=[6]{\color{green}\bfseries},
keywordstyle=[7]{\color{yellow}\bfseries},
%extendedchars={true},
alsoletter={&},alsoletter={*},alsoletter={$},
morekeywords=[5]{blue},
morekeywords=[6]{green},
morekeywords=[7]{yellow},
}
\hypersetup {%
  pdfauthor=Florent CHERVET,
  pdfkeywords={TeX, LaTeX, e-TeX, marks, firstmarks, botmarks, topmarks, package },
}
\geometry {top=0mm,headheight=8mm,includehead,reversemarginpar,asymmetric,headsep=3mm,bottom=14mm,footskip=5mm,inner=35mm,outer=20mm }
\begin{document}
   \DocInput{\jobname.dtx}
\end{document}
%</driver>
% \fi
%
% \CheckSum{219}
% \CharacterTable
%  {Upper-case    \A\B\C\D\E\F\G\H\I\J\K\L\M\N\O\P\Q\R\S\T\U\V\W\X\Y\Z
%   Lower-case    \a\b\c\d\e\f\g\h\i\j\k\l\m\n\o\p\q\r\s\t\u\v\w\x\y\z
%   Digits        \0\1\2\3\4\5\6\7\8\9
%   Exclamation   \!     Double quote  \"     Hash (number) \#
%   Dollar        \$     Percent       \%     Ampersand     \&
%   Acute accent  \'     Left paren    \(     Right paren   \)
%   Asterisk      \*     Plus          \+     Comma         \,
%   Minus         \-     Point         \.     Solidus       \/
%   Colon         \:     Semicolon     \;     Less than     \<
%   Equals        \=     Greater than  \>     Question mark \?
%   Commercial at \@     Left bracket  \[     Backslash     \\
%   Right bracket \]     Circumflex    \^     Underscore    \_
%   Grave accent  \`     Left brace    \{     Vertical bar  \|
%   Right brace   \}     Tilde         \~}
%
% \DoNotIndex{\globcount,\globdimen,\if,\fi,\else,\def,\the,\gdef,\global,\relax}
% \makeatletter
% \def\ThisInfo {\ifdefvoid\lsstyle {\scalebox{1.35}[1]{\eTeX}\stretchwith\,{\ named\ marks\ registers}}
%                                                                  {\lsstyle \eTeX{} named marks registers }}
% \def\ttdefault{lmvtt}         \colorlet{pkgcolor}{LimeGreen!50!Black}
% \parindent\z@\parskip.4\baselineskip\topsep\parskip\partopsep\z@
% \newrobustcmd*\FC {{\leavevmode\color{copper}\usefont{T1}{fts}xn FC}}
% \colorlet{linkcolor}{DarkSlateBlue}   \colorlet{csrefcolor}{linkcolor}^^ARoyalBlue!70!Indigo!50!Black}
% \definecolor{macrocode}{rgb}{0.08,0.00,0.15}
% \providerobustcmd*\csred{\cs[\colorlet{csrefcolor}{red}]}
% \def\MacroFont{\ttfamily\bfseries }
% \def\macro@font {\def\Cr@scale{.87}\changefont{fam=pcrs,siz=10pt,ser=m,color=macrocode,spread=1}\let\AltMacroFont\macro@font }
% \AtBeginEnvironment {declcs}{\tabusetup* {font=\bfseries,everyrow=\rowbackground{fill=Snow!220,cell fading=-+{70}{0}{30}},line style=\heavyrulewidth,linesep=1mm}}
% \AtBeginEnvironment {lstlisting}{\Needspace{7\baselineskip}}
% \tikzAtEveryShipout {\ifnum \value{page}>\@ne  \fill [fill=Lime,very nearly transparent] (0,0) rectangle (\paperwidth,-\headheight);\fi }
%  \sectionformat\section[hang]{
%       left=\declmarginwidth,
%       font=\bfseries\Large,
%       bookmark={color=pkgcolor},
%       bottom=\smallskipamount,top=\medskipamount,
%  }
%  \sectionformat\subsection{
%       font=\large\bfseries,
%       bookmark={color=MidnightBlue},
%  }
% \pagesetup [corpus]{norule,
%       font=\footnotesize,
%       head/left=\noindent\raise1.5mm\hbox{\thispackage},
%       head/font+=\mdseries\sffamily,
%       head/right=\noindent\raise1.5mm\hbox{\ThisInfo},
%       foot/left/font=\scriptsize\color{gray!80},
%       foot/left=\vbox to\baselineskip{\vss{{\rotatebox[origin=l]{90}{\thispackage\,[rev.\thisversion]\,\CopyRight2011\,\lower.4ex\hbox{\pkgcolor\NibRight}\,\FC}}}},
%       offset=15mm,
%       left/offset+=15mm,
%       foot/right=\oldstylenums{\arabic{page}}/\oldstylenums{\pageref{LastPage}},
%  }
% \pagesetup [plain]{%
%     norules,font=\scriptsize,
%     offset=15mm,
%     left/offset+=15mm,
%     foot/font=\scriptsize\color[gray]{.55},
%     foot/right=\oldstylenums{\arabic{page}}/\oldstylenums{\pageref{LastPage}},
%     foot/left=\vbox to\baselineskip{\vss{{\rotatebox[origin=l]{90}{\thispackage\,[rev.\thisversion]\,\CopyRight2011\,\lower.4ex\hbox{\pkgcolor\NibRight}\,\FC\quad \xemail{florent.chervet at free.fr}}}}},
% }
% \bookmarksetup{openlevel=3}
%
% \newrobustcmd\IMPLEMENTATION{\bigskip
%       \par \tabubox { X[c] }{ \background {cell fading=fuzzy ring 15 percent,fill=Silver,semitransparent}
%               \tabubox {  *2X[c]  } {\includegraphics [width=20mm,keepaspectratio]{emarks-fingerprint.png}
%                   & \indent\tikz{\fill [decorate,decoration={footprints,foot of=felis silvestris}] (\paperwidth-25mm,-32mm) circle (+8mm);}
%       \\[^2mm _8mm]}
%       \\ \LARGE\thispackage }
%       \clearpage
%       \bookmarksetup {bold*,openlevel=1}
%       \sectionformat \section{bookmark={color=black}} \sectionformat \subsection{bookmark={color=gray}}
%       \section(Implementation)[\lsstyle\textsc{\bfseries Implementation}]{\larger\lsstyle\textsc{\bfseries Implementation}}\label{sec:implementation}\parindent1em
%  }
%
% \tikzAtFirstShipout{\fill [decorate,decoration={footprints,foot of=felis silvestris}] (\paperwidth-25mm,-32mm) circle (+5mm);
%                     \node[anchor=south] at (+41mm,-40mm) {\includegraphics [width=15mm,keepaspectratio] {emarks-fingerprint.png}};}
%
% \title  {\vspace*{-28pt}\Huge\bfseries \CTANhref[emarks]{\pkgcolor emarks}\Footnotemark{*}
%          \tabusetup* {linesep=3mm ,font=\large\changefont{fam=txr}}
%   \tabubox { X[c] }{ \ThisInfo \\
%                       \small\FC   \\
%                       \small\mdseries\thisdate~--~\hyperref[\thisversion]{version \thisversion }
%                    }\vspace*{-12pt}}
% \author {}
% \date   {}
% \makeatother
%
% \maketitle            
%
% \pdftikzcomment [author=interfaces-pdfcomment] {TikZ decorations.footprint library}
%   [semithick,color=pkgcolor,->>] ($(current page.north east)-(7mm,7mm)$) -- ++ (-1.5,-1.5);
%
% \bookmark[bold,view=FitH 0,named=FirstPage,color=pkgcolor]{emarks}
% 
% \Footnotetext{\rlap{*}\kern2em}{\parindent0pt\noindent
% This documentation is produced with the \textt{DocStrip} utility.\par
% \begin{tabu}{X[-3]X[-1] >\ttfamily X}
% \smex To get the package,                   &run:          & etex \thisfile.dtx                  \\
% \smex To get the documentation              &run (thrice): & pdflatex \thisfile.dtx               \\
% \leavevmode\hphantom\smex To get the index, &run:          & makeindex -s gind.ist \thisfile.idx
% \end{tabu}\par
% The \xext{dtx}* is embedded into this \xext{pdf}* thank to \Xpackage{embedfile} by H. Oberdiek.}
%
% \vspace*{-16mm}
% \begin{Abstract}[\leftmargin=6mm\parindent0pt\listparindent0pt\parskip\smallskipamount\parsep\smallskipamount]
%
% \shorttabubox { X }{\eTeX{} defines \numprint {32768} marks registers while \TeX{} provided only one \emph!\\ \cbackground {fill=pkgcolor,cell fading=*,nearly transparent}-}
%
% So small, this package provides commands to access \eTeX{} marks registers by their name rather than by their number.
% This makes the use of them far more comfortable than ``old \LaTeX{}'' tricks with \cs\markright, \cs\markboth \etc.
%
% \thispackage requires \eTeX{} and the generic package \CTANhref[etex-pkg]{\xfile{etex.sty}} for allocation.
%
%  Presently designed to be loaded by \textt\LaTeX{}, a \textt{plain \TeX} version might be provided later...
%
% \end{Abstract}
%
% \tocsetup{
%  before+=\hypersetup {linkcolor=black},
%  section/skip=4pt plus2pt minus2pt,
%  subsection/skip=0pt plus2pt minus2pt,
%  section/dotsep=,
%  subsection/dotsep=,
%  subsection/pagenumbers=off,
%  dotsep=1.5mu,
%  title/top = 12pt,
%  dot=\hbox{$\scriptscriptstyle\cdotp$},
%  title={\pkgcolor\leaders\vrule height3.4pt depth-3pt\hfill\null}\quad Contents of \textsfbf{\pkgcolor emarks}\quad{\pkgcolor\leaders\vrule height3.4pt depth-3pt\hfill\null},
%  title/bottom=4pt,
%  multicols/beforeend=\aftergroup\tocrule,
%  columns=2,columns/rule color=pkgcolor,no columns rule,
%  }
%  \def\tocrule{\leavevmode{\pkgcolor\hrule}}
%
% \tableofcontents  \pagestyle{corpus}
%
% \listofsetup {lol}{pagestyle=,
%       title/font=\large\bfseries,
%       title={\pkgcolor\leaders\vrule height3.4pt depth-3pt\hfill\null}\quad List of the listings / examples\quad{\pkgcolor\leaders\vrule height3.4pt depth-3pt\hfill\null},
%       after=\tocrule,
%       twocolumns=false,
%       label=lol,bookmark={text=Listings},
%       lstlisting/dotsep=,
%       lstlisting/pagenumbers=off,
%       lstlisting/leaders=,
%       lstlisting/font=\hfil,
%       lstlisting/number/width=0pt,
% }
%
% \addcontentsline{lol}{lstlisting}{To be done !}
%
% \listoflstlisting
%
% \bookmarksetup{bold*}
%
% \section{The \hologo{eTeX} marks registers}
% \label{userinterface}
%
%
% \begin{declcs}[{ | *2X[-1] | }] \marksthe\M{named-mark}\M{content} & \csanchor\marksthecs\M{named-mark}\M{cs-name} \\
%                        \cs\marksthe*\M{named-mark}\M{content}      &       \cs\marksthecs*\M{named-mark}\M{cs-name}
% \end{declcs}\declcsbookmark\marksthe\declcsbookmark\marksthecs
%
%
% \tabusetup* { extra sep = .5\parskip }
% \declmargin\begin{tabu*}to\dimexpr\linewidth-\declmarginwidth {X[-1]X @{} }
%   \cs\marksthe\M*{section}\M{content} & Marks the \meta{content} into the named mark register \meta{section} in the same way as the \eTeX{} primitive \cs\marks:
%                                         in particular the \meta{content} is immediately expanded. \par
%                                         If the mark register does not exist, it is created (or allocated) with \cs\newmarks (in \xfile{etex.sty}).
%  \\
%  \cs\marksthe*\M*{section}\M{content} & does the same but the \meta{content} is not expanded. The current values of counters, \cs\thesection \etc. will be wrong:
%                                         they will expand to the value they have at the time the mark register is read, not at the time of \cs\marksthe*.\par
%                                         Yet \cs\marksthe* is useful to mark a title only like in�
%                                         \shorttabubox* {X}{ \cs\def\cs\sectionmark \#1\M*{\cs\marksthe*\M*{section}\M{\#1}}}�
%                                         or to control the expansion (the \meta{content} can be expanded before marking in a way and with the protections desired by the user).
% \end{tabu*}
%
% Similarly \cs\marksthecs\M{subsubsection}\M{cs-name} marks the content of \cs{cs-name} by the mean of the named mark
% register \meta{subsubsection}. \meta{cs-name} is really the \emph{name of the control sequence} and not the control sequence itself:
% it does not start with \textttbf{\csname @backslashchar\endcsname}.�
% If \cs{cs-name} is empty the mark is empty, but if it is undefined or \cs\relax: nothing is marked: at reading time, the mark register never expands
% to \cs\undefined nor to \cs\relax.
%
% The syntax follows \eTeX{} \cs\marks primitive (a token-like syntax): braces are mandatory around the \M{content} to be marked, even if it is made of one single token.
%
%   \begin{declcs}[{ | *2X[-1] |}]\thefirstmarks&\M{named-mark}\textsuperscript{\textsc{expandable}} \\
%                \csanchor\thebotmarks  &\M{named-mark}\textsuperscript{\textsc{expandable}} \\
%                \csanchor\thetopmarks  &\M{named-mark}\textsuperscript{\textsc{expandable}}
% \end{declcs}\declcsbookmark\thefirstmarks\declcsbookmark\thebotmarks\declcsbookmark\thetopmarks
%
% Those commands are expandable in exactly one step of expansion. If the \meta{named-mark} mark register does not exists,
% the expansion is null (\ie nothing is done nor printed).
%
% \tabusetup* {colsep=3pt}
% \begin{tabu*}{ @{} X[-1] X @{} }
% \cs\thefirstmarks\M{chapter}  &expands to the content of the first invocation of \cs\marksthe\M{chapter}
%   on the current page if \cs\marksthe\M*{chapter} was used on the current page,
%   or the last invocation of \cs\marksthe\M*{chapter} if no marks occured on the current page.
% \\
%   \hfill\small \TeX nically this is &\cs\firstmarks \cs\marks@chapter
% \\[1mm]
% \cs\thebotmarks\M{chapter}    &expands to the content of the last invocation of \cs\marksthe\M{chapter} (the most recent \cs\marks). \\
%   \hfill\small \TeX nically this is &\cs\botmarks \cs\marks@chapter
% \\[1mm]
% \cs\thetopmarks\M{chapter}    &expands to the content of \cs\botmarks at the time \TeX{} shipped out the last page. \\
%  \hfill\small \TeX nically this is  &\cs\topmarks\cs\marks@chapter
% \end{tabu*}
%
% \def\OR{\stform|}
% \begin{declcs}[{ | X | }]\getthemarks  \cs\firstmarks \OR \cs\botmarks \OR \cs\topmarks \M{named-mark} \M*{\cs{control-sequence}} \\
%      \csanchor\getthefirstmarks \M{named-mark} \M*{\cs{control-sequence}} \\
%      \csanchor\getthebotmarks \M{named-mark} \M*{\cs{control-sequence}} \\
%      \csanchor\getthetopmarks \M{named-mark} \M*{\cs{control-sequence}}
% \end{declcs}\declcsbookmark\getthemarks\declcsbookmark[rellevel=2]\getthefirstmarks\declcsbookmark[rellevel=2]\getthebotmarks\declcsbookmark[rellevel=2]\getthetopmarks
%
% \cs\thefirstmarks, \cs\thebotmarks and \cs\thetopmarks expand the content of the mark. To get it in a macro
% \cs\getthemarks can be used: \cs{control-sequence} is defined as a parameterless macro whose replacement text is
% the content of the given mark register.
%
% If the \meta{named-mark} mark register does not exist, the meaning of\, \cs{control-sequence}\, is
% \textt{undefined}.
%
% \begin{declcs}\ifmarksvoid \M*{\cs\firstmarks}\M*{named-mark}\M{true}\M{false} \\
%            \cs\ifmarksvoid \M*{\cs\botmarks}\M*{named-mark}\M{true}\M{false} \\
%            \cs\ifmarksvoid \M*{\cs\topmarks}\M*{named-mark}\M{true}\M{false} \\
% \end{declcs}\declcsbookmark\ifmaksvoid
%
% \cs\ifmarksvoid expands the \M{true} part if either: ^^A\loggingall
% \begin{itemize}^^A[topsep=0pt,itemsep=0pt]
% \item The requested mark register is empty,
% \item The requested mark register is \cs\undefined,
% \item The requested mark register is \cs\relax,
% \item The \meta{named-mark} mark register does not exist.
% \end{itemize}
%
%
%
% \begin{declcs}\ifmarksequal\M*{\cs\firstmarks}\M*{\cs\topmarks}\M*{named-mark}\M{true}\M{false}   \\
%            \cs\ifmarksequal\M*{\cs\firstmarks}\M*{\cs\botmarks}\M*{named-mark}\M{true}\M{false}
% \end{declcs}\declcsbookmark\ifmarksequal
%
% Pretty often we want to compare the botmarks against the firstmarks or the topmarks, to adapt the header and/or footer
% in case those marks are equal or different, \ie in case the page contains a new section title or not:
%
% \cs\ifmarksequal expands the code in the \M{true} or the \M{false} part if the extraction of the marks are equal
% (in the sense of \cs\ifx) or different.
%
% If any of the marks register\, \cs{marks@\meta{named-mark}}\, does not exist the \M{false} part is expanded.
%
% If marks are used both at \cs\sectionmark \textbf{and at} \cs\sectionbreak then the following assertions are true:�
% \begin{shorttabu}{ !\textbullet  l  @{\,=\,}  l  !{$\Leftrightarrow$} l  }
% \cs\firstmarks &\cs\botmarks &there is at most one section title on the current page; \\
% \cs\topmarks   &\cs\botmarks &there is no section title on the current page;          \\
% \cs\firstmarks &\cs\topmarks &the last section title continues on the current page.
% \end{shorttabu}
%
%
% \begin{declcs}\showthemarks \M{named-mark}
% \end{declcs}\declcsbookmark\showthemarks
%
% \cs\showthemarks is for debugging purpose: it prints a message in the \xext{log}* and the ``standard error'' with
% the contents of the marks \cs\firstmarks, \cs\botmarks and \cs\topmarks for the \meta{named-mark} register given.
% Then it executes \cs\show on the extracted content of \cs\firstmarks in order to stop compilation at that point:
% the console displays the contents of \cs\firstmarks, \cs\botmarks and \cs\topmarks.
%
% ^^A\loggingall
% ^^A\showthemarks{section}
%
% \StopEventually{ }
%
% \IMPLEMENTATION
%
% \subsection*{Identification}       \makeatletter
%
% The package namespace is \cs\em@rks
%
%    \begin{macrocode}
%<*package>
\NeedsTeXFormat{LaTeX2e}[2005/12/01]
\ProvidesPackage{emarks}
         [2011/03/26 v1.0 - e-TeX named marks registers (FC)]
\RequirePackage {etex}
%    \end{macrocode}
%
%    \begin{macro}{\emarks@newmarks}
%
%    allocates a new marks register if it does not exists.
%
%    \begin{macrocode}
\def\emarks@newmarks #1{\PackageInfo {emarks}{New marks register `#1'}%
                        \newmarks #1% \newmarks is global !!
}% \emarks@newmarks
%    \end{macrocode}
%    \end{macro}
%
%    \begin{macro}{\marksthe}
%    \begin{macro}{\marksthecs}
%
%   \noindent\shorttabubox* { X } { \lstinline! \marksthe   { named-mark }{ general text } ! \\
%                          \lstinline! \marksthe*  { named-mark }{ general text } ! \\
%                          \lstinline! \marksthe   { named-mark }{ named control sequence } ! \\
%                          \lstinline! \marksthecs*{ named-mark }{ named control sequence } ! \\
%                        }
%
%    \begin{macrocode}
\protected\def\marksthe   {\emarks@setmarks {}}
\protected\def\marksthecs {\emarks@setmarks {\toks@\expandafter{\csname\the\toks@\endcsname}}}
\def\emarks@setmarks #1{\begingroup \@ifstar {\emarks@ {#1}\def  }
                                             {\emarks@ {#1}\edef }%
}% \emarks@setmarks
\def\emarks@ #1#2#3{\def\@tempa
      {#1#2\@tempa {\the\toks@ }\expandafter\emarks@marks \csname marks@#3\endcsname }%
                                                \afterassignment \@tempa \toks@ =
}% \emarks@
\def\emarks@marks #1{\ifx \relax#1\emarks@newmarks #1\fi \marks #1{\@tempa }\endgroup }
%    \end{macrocode}
%    \end{macro}
%    \end{macro}
%
%    \begin{macro}{\thefirstmarks}
%    \begin{macro}{\thebotmarks}
%    \begin{macro}{\thetopmarks}
%
%   \cs\thefirstmarks extract the \cs\firstmarks from a named mark register.
%
%   The macros are purely expandable in exactly one step of expansion.
%
%    \begin{macrocode}
\newcommand*\thefirstmarks {\romannumeral \emarks@themarks \firstmarks  }
\newcommand*\thebotmarks   {\romannumeral \emarks@themarks \botmarks    }
\newcommand*\thetopmarks   {\romannumeral \emarks@themarks \topmarks    }
\def\emarks@themarks #1#2{\expandafter \ifx
    \csname\ifcsname marks@#2\endcsname marks@#2\else relax\fi\endcsname\relax
            \expandafter \z@
    \else   \expandafter \z@ #1\csname marks@#2\expandafter \endcsname  \fi
}% \emarks@themarks
%    \end{macrocode}
%    \end{macro}
%    \end{macro}
%    \end{macro}
%
%    \begin{macro}{\getthemarks}
%    \begin{macro}{\getthefirstmarks}
%    \begin{macro}{\getthebotmarks}
%    \begin{macro}{\getthetopmarks}
%
%    Extract the marks and store in a parameterless macro.
%
%    \begin{macrocode}
\protected\def\getthemarks #1#2#3{\ifcsname marks@#2\endcsname
          \expandafter \def \expandafter #3\expandafter {#1\csname marks@#2\endcsname}%
    \else                           \let #3=\@undefined    \fi
}% \getthemarks
\protected\def\getthefirstmarks {\getthemarks   \firstmarks }
\protected\def\getthebotmarks   {\getthemarks   \botmarks   }
\protected\def\getthetopmarks   {\getthemarks   \topmarks   }
%    \end{macrocode}
%    \end{macro}
%    \end{macro}
%    \end{macro}
%    \end{macro}
%
%    \begin{macro}{\ifmarksvoid}
%
%   Test if a marks is defined, not empty and not \cs\relax.
%
%    \begin{macrocode}
\protected\def\ifmarksvoid #1#2{\begingroup \getthemarks {#1}{#2}\x
    \ifodd \ifdefined\x \ifx \x\relax 0 \fi \ifx \x\@empty 0 \fi \else 0 \fi
           1 \endgroup\expandafter\@secondoftwo
    \else    \endgroup\expandafter\@firstoftwo      \fi
}% \ifmarksvoid
%    \end{macrocode}
%    \end{macro}
%
%    \begin{macro}{\ifmarksequal}
%
%   Test with \cs\ifx if two marks are equal:�
%   \shorttabubox* {X} { \lstinline ! \ifmarksequal \firstmarks \botmarks { named-mark } ! }
%
%
%    \begin{macrocode}
\protected\def\ifmarksequal #1#2#3{\begingroup \getthemarks{#1}{#3}\x \getthemarks{#2}{#3}\y
        \expandafter \endgroup \ifodd \ifdefined\x \ifdefined\y \ifx \x\y 0 \fi\fi\fi
                                      1 \expandafter\@secondoftwo
                               \else    \expandafter\@firstoftwo     \fi
}% \ifmarksequal
%    \end{macrocode}
%    \end{macro}
%
%    \begin{macro}{\showthemarks}
%
%    Shows the contents of the marks registers
%
%    \begin{macrocode}
\protected\def\showthemarks #1{\begingroup  \emarks@showthemarks 0{#1}\firstmarks
                                            \emarks@showthemarks 2{#1}\botmarks
                                            \emarks@showthemarks 4{#1}\topmarks
    \message{firstmarks "#1": \the\toks0^^J%
             botmarks   "#1": \the\toks2^^J%
             topmarks   "#1": \the\toks4^^J}\show\@tempa
    \endgroup
}% \showthemarks
\def\emarks@showthemarks #1#2#3{\getthemarks #3{#2}\@tempa \toks #1 = \ifdefined\@tempa
    \expandafter\ifx \noexpand\@tempa\@tempa {}\else \expandafter {\@tempa }\fi
                                                                      \else {}\fi
}% \emarks@showthemarks
%    \end{macrocode}
%    \end{macro}
%
%    \begin{macrocode}
%</package>
%    \end{macrocode}
%
%
% \begin{History}
%   \sectionformat\subsection{font=\normalsize\pkgcolor,bottom=0pt,top=\smallskipamount }\makeatletter
%
%   \begin{Version}{2011/03/26}{1.0}
%   \item First version. \\
%   \end{Version}
%
% \end{History}
%
% \begin{thebibliography}{9}
%
% \bibitem{etex} The \xpackage{etex} package by Peter Breitenlohner \\
%       \getpackageinfo{etex} \\
%       \CTANhref[etex-pkg]{\nolinkurl{CTAN:help/Catalogue/entries/etex-pkg.html}}
%
% \end{thebibliography}
%
% \clearpage
% \PrintIndex
%
% \Finale
%        (quote the arguments according to the demands of your shell)
%
% Documentation:                        pdflatex emarks.dtx
% Copyright (C) 2011 by FC <florent.chervet @t free.fr>
%<*ignore>
\begingroup
  \def\x{LaTeX2e}%
\expandafter\endgroup
\ifcase 0\ifx\install y1\fi\expandafter
         \ifx\csname processbatchFile\endcsname\relax\else1\fi
         \ifx\fmtname\x\else 1\fi\relax
\else\csname fi\endcsname
%</ignore>
%<*install>
\input docstrip.tex
\Msg{************************************************************************}
\Msg{* Installation                                                         *}
\Msg{* Package emarks: 2011/03/26 v1.0 - e-TeX named marks registers (FC)   *}
\Msg{************************************************************************}

\keepsilent
\askforoverwritefalse

\let\MetaPrefix\relax
\preamble

This is a generated file.

emarks : 2011/03/26 v1.0 - e-TeX named marks registers (FC)

This work may be distributed and/or modified under the
conditions of the LaTeX Project Public License, either
version 1.3 of this license or (at your option) any later
version. The latest version of this license is in
   http://www.latex-project.org/lppl.txt

This work consists of the main source file emarks.dtx
and the derived files:
                emarks.sty, emarks.ins, emarks.drv,
        and:                emarks.pdf

emarks : 2011/03/26 v1.0 - e-TeX named marks registers (FC)
Copyright (C) 2011 by FC <florent.chervet @t free.fr>

\endpreamble
\let\MetaPrefix\DoubleperCent

\generate{%
   \file{emarks.ins}{\from{emarks.dtx}{install}}%
   \file{emarks.sty}{\from{emarks.dtx}{package}}%
}

\askforoverwritefalse
\generate{%
   \file{emarks.drv}{\from{emarks.dtx}{driver}}%
}

\obeyspaces
\Msg{************************************************************************}
\Msg{*                                                                      *}
\Msg{* To finish the installation you have to move the following            *}
\Msg{* file into a directory searched by TeX:                               *}
\Msg{*                                                                      *}
\Msg{*                        emarks.sty                                    *}
\Msg{*                                                                      *}
\Msg{* To produce the documentation run the file `emarks.dtx'               *}
\Msg{*                     through pdfLaTeX.                                *}
\Msg{*                                                                      *}
\Msg{************************************************************************}
\endbatchfile
%</install>
%<*ignore>
\fi
%</ignore>
%<*driver>
\def\thisinfo {e-TeX named marks registers (FC)}
\def\thisversion {1.0}
\PassOptionsToPackage {full}{tabu}
\RequirePackage [\detokenize{��},hyperlistings]{fcltxdoc}
\AtBeginDocument{\embedfile{README}}
%%\CheckDates{interfaces=2011/02/12,tabu=2011/02/25}
\documentclass[a4paper,11pt,twoside,american,latin1,T1]{ltxdoc}     \usetikz{full}
\usepackage [latin1]{inputenc}
\usepackage [T1]{fontenc}
\usepackage {numprint}
\usepackage {pdfcomment}
\usepackage {ragged2e}   % general tools
\usepackage {arial,bbding,relsize,moresize,manfnt,pifont,upgreek} % fonts
\csname endofdump\endcsname
\usepackage {emarks}
\RequirePackage [full]{tabu}
\usepackage {geometry}
\AtBeginDocument {\let\setkeys \kvsetkeys }
\let\microtypeYN=n
\ifx y\microtypeYN                                                          %
   \usepackage[expansion=all,stretch=20,shrink=60]{microtype}\fi            % font (microtype)
\CodelineNumbered\lastlinefit999
\lstset{backgroundcolor=\color{LightYellow},
texcsstyle=\color{blue},
moretexcs=[1]{
    lstdefinestyle,
    lstinputlisting,lstset,tikzlabel,tikzrefXY,
    color,
    geometry,lasthline,firsthline,
    cmidrule,toprule,bottomrule,tabusetup*,tabusetup,
    everyrow,tabulinestyle,tabureset,savetabu,usetabu,preamble,
    taburulecolor,taburowcolors},
keywordstyle=[3]{\color{black}\bfseries},
morekeywords=[3]{&},
keywordstyle=[4]{\color{red}\bfseries},
morekeywords=[4]{\linegoal,$},
keywordstyle=[5]{\color{blue}\bfseries},
keywordstyle=[6]{\color{green}\bfseries},
keywordstyle=[7]{\color{yellow}\bfseries},
%extendedchars={true},
alsoletter={&},alsoletter={*},alsoletter={$},
morekeywords=[5]{blue},
morekeywords=[6]{green},
morekeywords=[7]{yellow},
}
\hypersetup {%
  pdfauthor=Florent CHERVET,
  pdfkeywords={TeX, LaTeX, e-TeX, marks, firstmarks, botmarks, topmarks, package },
}
\geometry {top=0mm,headheight=8mm,includehead,reversemarginpar,asymmetric,headsep=3mm,bottom=14mm,footskip=5mm,inner=35mm,outer=20mm }
\begin{document}
   \DocInput{\jobname.dtx}
\end{document}
%</driver>
% \fi
%
% \CheckSum{219}
% \CharacterTable
%  {Upper-case    \A\B\C\D\E\F\G\H\I\J\K\L\M\N\O\P\Q\R\S\T\U\V\W\X\Y\Z
%   Lower-case    \a\b\c\d\e\f\g\h\i\j\k\l\m\n\o\p\q\r\s\t\u\v\w\x\y\z
%   Digits        \0\1\2\3\4\5\6\7\8\9
%   Exclamation   \!     Double quote  \"     Hash (number) \#
%   Dollar        \$     Percent       \%     Ampersand     \&
%   Acute accent  \'     Left paren    \(     Right paren   \)
%   Asterisk      \*     Plus          \+     Comma         \,
%   Minus         \-     Point         \.     Solidus       \/
%   Colon         \:     Semicolon     \;     Less than     \<
%   Equals        \=     Greater than  \>     Question mark \?
%   Commercial at \@     Left bracket  \[     Backslash     \\
%   Right bracket \]     Circumflex    \^     Underscore    \_
%   Grave accent  \`     Left brace    \{     Vertical bar  \|
%   Right brace   \}     Tilde         \~}
%
% \DoNotIndex{\globcount,\globdimen,\if,\fi,\else,\def,\the,\gdef,\global,\relax}
% \makeatletter
% \def\ThisInfo {\ifdefvoid\lsstyle {\scalebox{1.35}[1]{\eTeX}\stretchwith\,{\ named\ marks\ registers}}
%                                                                  {\lsstyle \eTeX{} named marks registers }}
% \def\ttdefault{lmvtt}         \colorlet{pkgcolor}{LimeGreen!50!Black}
% \parindent\z@\parskip.4\baselineskip\topsep\parskip\partopsep\z@
% \newrobustcmd*\FC {{\leavevmode\color{copper}\usefont{T1}{fts}xn FC}}
% \colorlet{linkcolor}{DarkSlateBlue}   \colorlet{csrefcolor}{linkcolor}^^ARoyalBlue!70!Indigo!50!Black}
% \definecolor{macrocode}{rgb}{0.08,0.00,0.15}
% \providerobustcmd*\csred{\cs[\colorlet{csrefcolor}{red}]}
% \def\MacroFont{\ttfamily\bfseries }
% \def\macro@font {\def\Cr@scale{.87}\changefont{fam=pcrs,siz=10pt,ser=m,color=macrocode,spread=1}\let\AltMacroFont\macro@font }
% \AtBeginEnvironment {declcs}{\tabusetup* {font=\bfseries,everyrow=\rowbackground{fill=Snow!220,cell fading=-+{70}{0}{30}},line style=\heavyrulewidth,linesep=1mm}}
% \AtBeginEnvironment {lstlisting}{\Needspace{7\baselineskip}}
% \tikzAtEveryShipout {\ifnum \value{page}>\@ne  \fill [fill=Lime,very nearly transparent] (0,0) rectangle (\paperwidth,-\headheight);\fi }
%  \sectionformat\section[hang]{
%       left=\declmarginwidth,
%       font=\bfseries\Large,
%       bookmark={color=pkgcolor},
%       bottom=\smallskipamount,top=\medskipamount,
%  }
%  \sectionformat\subsection{
%       font=\large\bfseries,
%       bookmark={color=MidnightBlue},
%  }
% \pagesetup [corpus]{norule,
%       font=\footnotesize,
%       head/left=\noindent\raise1.5mm\hbox{\thispackage},
%       head/font+=\mdseries\sffamily,
%       head/right=\noindent\raise1.5mm\hbox{\ThisInfo},
%       foot/left/font=\scriptsize\color{gray!80},
%       foot/left=\vbox to\baselineskip{\vss{{\rotatebox[origin=l]{90}{\thispackage\,[rev.\thisversion]\,\CopyRight2011\,\lower.4ex\hbox{\pkgcolor\NibRight}\,\FC}}}},
%       offset=15mm,
%       left/offset+=15mm,
%       foot/right=\oldstylenums{\arabic{page}}/\oldstylenums{\pageref{LastPage}},
%  }
% \pagesetup [plain]{%
%     norules,font=\scriptsize,
%     offset=15mm,
%     left/offset+=15mm,
%     foot/font=\scriptsize\color[gray]{.55},
%     foot/right=\oldstylenums{\arabic{page}}/\oldstylenums{\pageref{LastPage}},
%     foot/left=\vbox to\baselineskip{\vss{{\rotatebox[origin=l]{90}{\thispackage\,[rev.\thisversion]\,\CopyRight2011\,\lower.4ex\hbox{\pkgcolor\NibRight}\,\FC\quad \xemail{florent.chervet at free.fr}}}}},
% }
% \bookmarksetup{openlevel=3}
%
% \newrobustcmd\IMPLEMENTATION{\bigskip
%       \par \tabubox { X[c] }{ \background {cell fading=fuzzy ring 15 percent,fill=Silver,semitransparent}
%               \tabubox {  *2X[c]  } {\includegraphics [width=20mm,keepaspectratio]{emarks-fingerprint.png}
%                   & \indent\tikz{\fill [decorate,decoration={footprints,foot of=felis silvestris}] (\paperwidth-25mm,-32mm) circle (+8mm);}
%       \\[^2mm _8mm]}
%       \\ \LARGE\thispackage }
%       \clearpage
%       \bookmarksetup {bold*,openlevel=1}
%       \sectionformat \section{bookmark={color=black}} \sectionformat \subsection{bookmark={color=gray}}
%       \section(Implementation)[\lsstyle\textsc{\bfseries Implementation}]{\larger\lsstyle\textsc{\bfseries Implementation}}\label{sec:implementation}\parindent1em
%  }
%
% \tikzAtFirstShipout{\fill [decorate,decoration={footprints,foot of=felis silvestris}] (\paperwidth-25mm,-32mm) circle (+5mm);
%                     \node[anchor=south] at (+41mm,-40mm) {\includegraphics [width=15mm,keepaspectratio] {emarks-fingerprint.png}};}
%
% \title  {\vspace*{-28pt}\Huge\bfseries \CTANhref[emarks]{\pkgcolor emarks}\Footnotemark{*}
%          \tabusetup* {linesep=3mm ,font=\large\changefont{fam=txr}}
%   \tabubox { X[c] }{ \ThisInfo \\
%                       \small\FC   \\
%                       \small\mdseries\thisdate~--~\hyperref[\thisversion]{version \thisversion }
%                    }\vspace*{-12pt}}
% \author {}
% \date   {}
% \makeatother
%
% \maketitle            
%
% \pdftikzcomment [author=interfaces-pdfcomment] {TikZ decorations.footprint library}
%   [semithick,color=pkgcolor,->>] ($(current page.north east)-(7mm,7mm)$) -- ++ (-1.5,-1.5);
%
% \bookmark[bold,view=FitH 0,named=FirstPage,color=pkgcolor]{emarks}
% 
% \Footnotetext{\rlap{*}\kern2em}{\parindent0pt\noindent
% This documentation is produced with the \textt{DocStrip} utility.\par
% \begin{tabu}{X[-3]X[-1] >\ttfamily X}
% \smex To get the package,                   &run:          & etex \thisfile.dtx                  \\
% \smex To get the documentation              &run (thrice): & pdflatex \thisfile.dtx               \\
% \leavevmode\hphantom\smex To get the index, &run:          & makeindex -s gind.ist \thisfile.idx
% \end{tabu}\par
% The \xext{dtx}* is embedded into this \xext{pdf}* thank to \Xpackage{embedfile} by H. Oberdiek.}
%
% \vspace*{-16mm}
% \begin{Abstract}[\leftmargin=6mm\parindent0pt\listparindent0pt\parskip\smallskipamount\parsep\smallskipamount]
%
% \shorttabubox { X }{\eTeX{} defines \numprint {32768} marks registers while \TeX{} provided only one \emph!\\ \cbackground {fill=pkgcolor,cell fading=*,nearly transparent}-}
%
% So small, this package provides commands to access \eTeX{} marks registers by their name rather than by their number.
% This makes the use of them far more comfortable than ``old \LaTeX{}'' tricks with \cs\markright, \cs\markboth \etc.
%
% \thispackage requires \eTeX{} and the generic package \CTANhref[etex-pkg]{\xfile{etex.sty}} for allocation.
%
%  Presently designed to be loaded by \textt\LaTeX{}, a \textt{plain \TeX} version might be provided later...
%
% \end{Abstract}
%
% \tocsetup{
%  before+=\hypersetup {linkcolor=black},
%  section/skip=4pt plus2pt minus2pt,
%  subsection/skip=0pt plus2pt minus2pt,
%  section/dotsep=,
%  subsection/dotsep=,
%  subsection/pagenumbers=off,
%  dotsep=1.5mu,
%  title/top = 12pt,
%  dot=\hbox{$\scriptscriptstyle\cdotp$},
%  title={\pkgcolor\leaders\vrule height3.4pt depth-3pt\hfill\null}\quad Contents of \textsfbf{\pkgcolor emarks}\quad{\pkgcolor\leaders\vrule height3.4pt depth-3pt\hfill\null},
%  title/bottom=4pt,
%  multicols/beforeend=\aftergroup\tocrule,
%  columns=2,columns/rule color=pkgcolor,no columns rule,
%  }
%  \def\tocrule{\leavevmode{\pkgcolor\hrule}}
%
% \tableofcontents  \pagestyle{corpus}
%
% \listofsetup {lol}{pagestyle=,
%       title/font=\large\bfseries,
%       title={\pkgcolor\leaders\vrule height3.4pt depth-3pt\hfill\null}\quad List of the listings / examples\quad{\pkgcolor\leaders\vrule height3.4pt depth-3pt\hfill\null},
%       after=\tocrule,
%       twocolumns=false,
%       label=lol,bookmark={text=Listings},
%       lstlisting/dotsep=,
%       lstlisting/pagenumbers=off,
%       lstlisting/leaders=,
%       lstlisting/font=\hfil,
%       lstlisting/number/width=0pt,
% }
%
% \addcontentsline{lol}{lstlisting}{To be done !}
%
% \listoflstlisting
%
% \bookmarksetup{bold*}
%
% \section{The \hologo{eTeX} marks registers}
% \label{userinterface}
%
%
% \begin{declcs}[{ | *2X[-1] | }] \marksthe\M{named-mark}\M{content} & \csanchor\marksthecs\M{named-mark}\M{cs-name} \\
%                        \cs\marksthe*\M{named-mark}\M{content}      &       \cs\marksthecs*\M{named-mark}\M{cs-name}
% \end{declcs}\declcsbookmark\marksthe\declcsbookmark\marksthecs
%
%
% \tabusetup* { extra sep = .5\parskip }
% \declmargin\begin{tabu*}to\dimexpr\linewidth-\declmarginwidth {X[-1]X @{} }
%   \cs\marksthe\M*{section}\M{content} & Marks the \meta{content} into the named mark register \meta{section} in the same way as the \eTeX{} primitive \cs\marks:
%                                         in particular the \meta{content} is immediately expanded. \par
%                                         If the mark register does not exist, it is created (or allocated) with \cs\newmarks (in \xfile{etex.sty}).
%  \\
%  \cs\marksthe*\M*{section}\M{content} & does the same but the \meta{content} is not expanded. The current values of counters, \cs\thesection \etc. will be wrong:
%                                         they will expand to the value they have at the time the mark register is read, not at the time of \cs\marksthe*.\par
%                                         Yet \cs\marksthe* is useful to mark a title only like in�
%                                         \shorttabubox* {X}{ \cs\def\cs\sectionmark \#1\M*{\cs\marksthe*\M*{section}\M{\#1}}}�
%                                         or to control the expansion (the \meta{content} can be expanded before marking in a way and with the protections desired by the user).
% \end{tabu*}
%
% Similarly \cs\marksthecs\M{subsubsection}\M{cs-name} marks the content of \cs{cs-name} by the mean of the named mark
% register \meta{subsubsection}. \meta{cs-name} is really the \emph{name of the control sequence} and not the control sequence itself:
% it does not start with \textttbf{\csname @backslashchar\endcsname}.�
% If \cs{cs-name} is empty the mark is empty, but if it is undefined or \cs\relax: nothing is marked: at reading time, the mark register never expands
% to \cs\undefined nor to \cs\relax.
%
% The syntax follows \eTeX{} \cs\marks primitive (a token-like syntax): braces are mandatory around the \M{content} to be marked, even if it is made of one single token.
%
%   \begin{declcs}[{ | *2X[-1] |}]\thefirstmarks&\M{named-mark}\textsuperscript{\textsc{expandable}} \\
%                \csanchor\thebotmarks  &\M{named-mark}\textsuperscript{\textsc{expandable}} \\
%                \csanchor\thetopmarks  &\M{named-mark}\textsuperscript{\textsc{expandable}}
% \end{declcs}\declcsbookmark\thefirstmarks\declcsbookmark\thebotmarks\declcsbookmark\thetopmarks
%
% Those commands are expandable in exactly one step of expansion. If the \meta{named-mark} mark register does not exists,
% the expansion is null (\ie nothing is done nor printed).
%
% \tabusetup* {colsep=3pt}
% \begin{tabu*}{ @{} X[-1] X @{} }
% \cs\thefirstmarks\M{chapter}  &expands to the content of the first invocation of \cs\marksthe\M{chapter}
%   on the current page if \cs\marksthe\M*{chapter} was used on the current page,
%   or the last invocation of \cs\marksthe\M*{chapter} if no marks occured on the current page.
% \\
%   \hfill\small \TeX nically this is &\cs\firstmarks \cs\marks@chapter
% \\[1mm]
% \cs\thebotmarks\M{chapter}    &expands to the content of the last invocation of \cs\marksthe\M{chapter} (the most recent \cs\marks). \\
%   \hfill\small \TeX nically this is &\cs\botmarks \cs\marks@chapter
% \\[1mm]
% \cs\thetopmarks\M{chapter}    &expands to the content of \cs\botmarks at the time \TeX{} shipped out the last page. \\
%  \hfill\small \TeX nically this is  &\cs\topmarks\cs\marks@chapter
% \end{tabu*}
%
% \def\OR{\stform|}
% \begin{declcs}[{ | X | }]\getthemarks  \cs\firstmarks \OR \cs\botmarks \OR \cs\topmarks \M{named-mark} \M*{\cs{control-sequence}} \\
%      \csanchor\getthefirstmarks \M{named-mark} \M*{\cs{control-sequence}} \\
%      \csanchor\getthebotmarks \M{named-mark} \M*{\cs{control-sequence}} \\
%      \csanchor\getthetopmarks \M{named-mark} \M*{\cs{control-sequence}}
% \end{declcs}\declcsbookmark\getthemarks\declcsbookmark[rellevel=2]\getthefirstmarks\declcsbookmark[rellevel=2]\getthebotmarks\declcsbookmark[rellevel=2]\getthetopmarks
%
% \cs\thefirstmarks, \cs\thebotmarks and \cs\thetopmarks expand the content of the mark. To get it in a macro
% \cs\getthemarks can be used: \cs{control-sequence} is defined as a parameterless macro whose replacement text is
% the content of the given mark register.
%
% If the \meta{named-mark} mark register does not exist, the meaning of\, \cs{control-sequence}\, is
% \textt{undefined}.
%
% \begin{declcs}\ifmarksvoid \M*{\cs\firstmarks}\M*{named-mark}\M{true}\M{false} \\
%            \cs\ifmarksvoid \M*{\cs\botmarks}\M*{named-mark}\M{true}\M{false} \\
%            \cs\ifmarksvoid \M*{\cs\topmarks}\M*{named-mark}\M{true}\M{false} \\
% \end{declcs}\declcsbookmark\ifmaksvoid
%
% \cs\ifmarksvoid expands the \M{true} part if either: ^^A\loggingall
% \begin{itemize}^^A[topsep=0pt,itemsep=0pt]
% \item The requested mark register is empty,
% \item The requested mark register is \cs\undefined,
% \item The requested mark register is \cs\relax,
% \item The \meta{named-mark} mark register does not exist.
% \end{itemize}
%
%
%
% \begin{declcs}\ifmarksequal\M*{\cs\firstmarks}\M*{\cs\topmarks}\M*{named-mark}\M{true}\M{false}   \\
%            \cs\ifmarksequal\M*{\cs\firstmarks}\M*{\cs\botmarks}\M*{named-mark}\M{true}\M{false}
% \end{declcs}\declcsbookmark\ifmarksequal
%
% Pretty often we want to compare the botmarks against the firstmarks or the topmarks, to adapt the header and/or footer
% in case those marks are equal or different, \ie in case the page contains a new section title or not:
%
% \cs\ifmarksequal expands the code in the \M{true} or the \M{false} part if the extraction of the marks are equal
% (in the sense of \cs\ifx) or different.
%
% If any of the marks register\, \cs{marks@\meta{named-mark}}\, does not exist the \M{false} part is expanded.
%
% If marks are used both at \cs\sectionmark \textbf{and at} \cs\sectionbreak then the following assertions are true:�
% \begin{shorttabu}{ !\textbullet  l  @{\,=\,}  l  !{$\Leftrightarrow$} l  }
% \cs\firstmarks &\cs\botmarks &there is at most one section title on the current page; \\
% \cs\topmarks   &\cs\botmarks &there is no section title on the current page;          \\
% \cs\firstmarks &\cs\topmarks &the last section title continues on the current page.
% \end{shorttabu}
%
%
% \begin{declcs}\showthemarks \M{named-mark}
% \end{declcs}\declcsbookmark\showthemarks
%
% \cs\showthemarks is for debugging purpose: it prints a message in the \xext{log}* and the ``standard error'' with
% the contents of the marks \cs\firstmarks, \cs\botmarks and \cs\topmarks for the \meta{named-mark} register given.
% Then it executes \cs\show on the extracted content of \cs\firstmarks in order to stop compilation at that point:
% the console displays the contents of \cs\firstmarks, \cs\botmarks and \cs\topmarks.
%
% ^^A\loggingall
% ^^A\showthemarks{section}
%
% \StopEventually{ }
%
% \IMPLEMENTATION
%
% \subsection*{Identification}       \makeatletter
%
% The package namespace is \cs\em@rks
%
%    \begin{macrocode}
%<*package>
\NeedsTeXFormat{LaTeX2e}[2005/12/01]
\ProvidesPackage{emarks}
         [2011/03/26 v1.0 - e-TeX named marks registers (FC)]
\RequirePackage {etex}
%    \end{macrocode}
%
%    \begin{macro}{\emarks@newmarks}
%
%    allocates a new marks register if it does not exists.
%
%    \begin{macrocode}
\def\emarks@newmarks #1{\PackageInfo {emarks}{New marks register `#1'}%
                        \newmarks #1% \newmarks is global !!
}% \emarks@newmarks
%    \end{macrocode}
%    \end{macro}
%
%    \begin{macro}{\marksthe}
%    \begin{macro}{\marksthecs}
%
%   \noindent\shorttabubox* { X } { \lstinline! \marksthe   { named-mark }{ general text } ! \\
%                          \lstinline! \marksthe*  { named-mark }{ general text } ! \\
%                          \lstinline! \marksthe   { named-mark }{ named control sequence } ! \\
%                          \lstinline! \marksthecs*{ named-mark }{ named control sequence } ! \\
%                        }
%
%    \begin{macrocode}
\protected\def\marksthe   {\emarks@setmarks {}}
\protected\def\marksthecs {\emarks@setmarks {\toks@\expandafter{\csname\the\toks@\endcsname}}}
\def\emarks@setmarks #1{\begingroup \@ifstar {\emarks@ {#1}\def  }
                                             {\emarks@ {#1}\edef }%
}% \emarks@setmarks
\def\emarks@ #1#2#3{\def\@tempa
      {#1#2\@tempa {\the\toks@ }\expandafter\emarks@marks \csname marks@#3\endcsname }%
                                                \afterassignment \@tempa \toks@ =
}% \emarks@
\def\emarks@marks #1{\ifx \relax#1\emarks@newmarks #1\fi \marks #1{\@tempa }\endgroup }
%    \end{macrocode}
%    \end{macro}
%    \end{macro}
%
%    \begin{macro}{\thefirstmarks}
%    \begin{macro}{\thebotmarks}
%    \begin{macro}{\thetopmarks}
%
%   \cs\thefirstmarks extract the \cs\firstmarks from a named mark register.
%
%   The macros are purely expandable in exactly one step of expansion.
%
%    \begin{macrocode}
\newcommand*\thefirstmarks {\romannumeral \emarks@themarks \firstmarks  }
\newcommand*\thebotmarks   {\romannumeral \emarks@themarks \botmarks    }
\newcommand*\thetopmarks   {\romannumeral \emarks@themarks \topmarks    }
\def\emarks@themarks #1#2{\expandafter \ifx
    \csname\ifcsname marks@#2\endcsname marks@#2\else relax\fi\endcsname\relax
            \expandafter \z@
    \else   \expandafter \z@ #1\csname marks@#2\expandafter \endcsname  \fi
}% \emarks@themarks
%    \end{macrocode}
%    \end{macro}
%    \end{macro}
%    \end{macro}
%
%    \begin{macro}{\getthemarks}
%    \begin{macro}{\getthefirstmarks}
%    \begin{macro}{\getthebotmarks}
%    \begin{macro}{\getthetopmarks}
%
%    Extract the marks and store in a parameterless macro.
%
%    \begin{macrocode}
\protected\def\getthemarks #1#2#3{\ifcsname marks@#2\endcsname
          \expandafter \def \expandafter #3\expandafter {#1\csname marks@#2\endcsname}%
    \else                           \let #3=\@undefined    \fi
}% \getthemarks
\protected\def\getthefirstmarks {\getthemarks   \firstmarks }
\protected\def\getthebotmarks   {\getthemarks   \botmarks   }
\protected\def\getthetopmarks   {\getthemarks   \topmarks   }
%    \end{macrocode}
%    \end{macro}
%    \end{macro}
%    \end{macro}
%    \end{macro}
%
%    \begin{macro}{\ifmarksvoid}
%
%   Test if a marks is defined, not empty and not \cs\relax.
%
%    \begin{macrocode}
\protected\def\ifmarksvoid #1#2{\begingroup \getthemarks {#1}{#2}\x
    \ifodd \ifdefined\x \ifx \x\relax 0 \fi \ifx \x\@empty 0 \fi \else 0 \fi
           1 \endgroup\expandafter\@secondoftwo
    \else    \endgroup\expandafter\@firstoftwo      \fi
}% \ifmarksvoid
%    \end{macrocode}
%    \end{macro}
%
%    \begin{macro}{\ifmarksequal}
%
%   Test with \cs\ifx if two marks are equal:�
%   \shorttabubox* {X} { \lstinline ! \ifmarksequal \firstmarks \botmarks { named-mark } ! }
%
%
%    \begin{macrocode}
\protected\def\ifmarksequal #1#2#3{\begingroup \getthemarks{#1}{#3}\x \getthemarks{#2}{#3}\y
        \expandafter \endgroup \ifodd \ifdefined\x \ifdefined\y \ifx \x\y 0 \fi\fi\fi
                                      1 \expandafter\@secondoftwo
                               \else    \expandafter\@firstoftwo     \fi
}% \ifmarksequal
%    \end{macrocode}
%    \end{macro}
%
%    \begin{macro}{\showthemarks}
%
%    Shows the contents of the marks registers
%
%    \begin{macrocode}
\protected\def\showthemarks #1{\begingroup  \emarks@showthemarks 0{#1}\firstmarks
                                            \emarks@showthemarks 2{#1}\botmarks
                                            \emarks@showthemarks 4{#1}\topmarks
    \message{firstmarks "#1": \the\toks0^^J%
             botmarks   "#1": \the\toks2^^J%
             topmarks   "#1": \the\toks4^^J}\show\@tempa
    \endgroup
}% \showthemarks
\def\emarks@showthemarks #1#2#3{\getthemarks #3{#2}\@tempa \toks #1 = \ifdefined\@tempa
    \expandafter\ifx \noexpand\@tempa\@tempa {}\else \expandafter {\@tempa }\fi
                                                                      \else {}\fi
}% \emarks@showthemarks
%    \end{macrocode}
%    \end{macro}
%
%    \begin{macrocode}
%</package>
%    \end{macrocode}
%
%
% \begin{History}
%   \sectionformat\subsection{font=\normalsize\pkgcolor,bottom=0pt,top=\smallskipamount }\makeatletter
%
%   \begin{Version}{2011/03/26}{1.0}
%   \item First version. \\
%   \end{Version}
%
% \end{History}
%
% \begin{thebibliography}{9}
%
% \bibitem{etex} The \xpackage{etex} package by Peter Breitenlohner \\
%       \getpackageinfo{etex} \\
%       \CTANhref[etex-pkg]{\nolinkurl{CTAN:help/Catalogue/entries/etex-pkg.html}}
%
% \end{thebibliography}
%
% \clearpage
% \PrintIndex
%
% \Finale
%        (quote the arguments according to the demands of your shell)
%
% Documentation:                        pdflatex emarks.dtx
% Copyright (C) 2011 by FC <florent.chervet @t free.fr>
%<*ignore>
\begingroup
  \def\x{LaTeX2e}%
\expandafter\endgroup
\ifcase 0\ifx\install y1\fi\expandafter
         \ifx\csname processbatchFile\endcsname\relax\else1\fi
         \ifx\fmtname\x\else 1\fi\relax
\else\csname fi\endcsname
%</ignore>
%<*install>
\input docstrip.tex
\Msg{************************************************************************}
\Msg{* Installation                                                         *}
\Msg{* Package emarks: 2011/03/26 v1.0 - e-TeX named marks registers (FC)   *}
\Msg{************************************************************************}

\keepsilent
\askforoverwritefalse

\let\MetaPrefix\relax
\preamble

This is a generated file.

emarks : 2011/03/26 v1.0 - e-TeX named marks registers (FC)

This work may be distributed and/or modified under the
conditions of the LaTeX Project Public License, either
version 1.3 of this license or (at your option) any later
version. The latest version of this license is in
   http://www.latex-project.org/lppl.txt

This work consists of the main source file emarks.dtx
and the derived files:
                emarks.sty, emarks.ins, emarks.drv,
        and:                emarks.pdf

emarks : 2011/03/26 v1.0 - e-TeX named marks registers (FC)
Copyright (C) 2011 by FC <florent.chervet @t free.fr>

\endpreamble
\let\MetaPrefix\DoubleperCent

\generate{%
   \file{emarks.ins}{\from{emarks.dtx}{install}}%
   \file{emarks.sty}{\from{emarks.dtx}{package}}%
}

\askforoverwritefalse
\generate{%
   \file{emarks.drv}{\from{emarks.dtx}{driver}}%
}

\obeyspaces
\Msg{************************************************************************}
\Msg{*                                                                      *}
\Msg{* To finish the installation you have to move the following            *}
\Msg{* file into a directory searched by TeX:                               *}
\Msg{*                                                                      *}
\Msg{*                        emarks.sty                                    *}
\Msg{*                                                                      *}
\Msg{* To produce the documentation run the file `emarks.dtx'               *}
\Msg{*                     through pdfLaTeX.                                *}
\Msg{*                                                                      *}
\Msg{************************************************************************}
\endbatchfile
%</install>
%<*ignore>
\fi
%</ignore>
%<*driver>
\def\thisinfo {e-TeX named marks registers (FC)}
\def\thisversion {1.0}
\PassOptionsToPackage {full}{tabu}
\RequirePackage [\detokenize{��},hyperlistings]{fcltxdoc}
\AtBeginDocument{\embedfile{README}}
%%\CheckDates{interfaces=2011/02/12,tabu=2011/02/25}
\documentclass[a4paper,11pt,twoside,american,latin1,T1]{ltxdoc}     \usetikz{full}
\usepackage [latin1]{inputenc}
\usepackage [T1]{fontenc}
\usepackage {numprint}
\usepackage {pdfcomment}
\usepackage {ragged2e}   % general tools
\usepackage {arial,bbding,relsize,moresize,manfnt,pifont,upgreek} % fonts
\csname endofdump\endcsname
\usepackage {emarks}
\RequirePackage [full]{tabu}
\usepackage {geometry}
\AtBeginDocument {\let\setkeys \kvsetkeys }
\let\microtypeYN=n
\ifx y\microtypeYN                                                          %
   \usepackage[expansion=all,stretch=20,shrink=60]{microtype}\fi            % font (microtype)
\CodelineNumbered\lastlinefit999
\lstset{backgroundcolor=\color{LightYellow},
texcsstyle=\color{blue},
moretexcs=[1]{
    lstdefinestyle,
    lstinputlisting,lstset,tikzlabel,tikzrefXY,
    color,
    geometry,lasthline,firsthline,
    cmidrule,toprule,bottomrule,tabusetup*,tabusetup,
    everyrow,tabulinestyle,tabureset,savetabu,usetabu,preamble,
    taburulecolor,taburowcolors},
keywordstyle=[3]{\color{black}\bfseries},
morekeywords=[3]{&},
keywordstyle=[4]{\color{red}\bfseries},
morekeywords=[4]{\linegoal,$},
keywordstyle=[5]{\color{blue}\bfseries},
keywordstyle=[6]{\color{green}\bfseries},
keywordstyle=[7]{\color{yellow}\bfseries},
%extendedchars={true},
alsoletter={&},alsoletter={*},alsoletter={$},
morekeywords=[5]{blue},
morekeywords=[6]{green},
morekeywords=[7]{yellow},
}
\hypersetup {%
  pdfauthor=Florent CHERVET,
  pdfkeywords={TeX, LaTeX, e-TeX, marks, firstmarks, botmarks, topmarks, package },
}
\geometry {top=0mm,headheight=8mm,includehead,reversemarginpar,asymmetric,headsep=3mm,bottom=14mm,footskip=5mm,inner=35mm,outer=20mm }
\begin{document}
   \DocInput{\jobname.dtx}
\end{document}
%</driver>
% \fi
%
% \CheckSum{219}
% \CharacterTable
%  {Upper-case    \A\B\C\D\E\F\G\H\I\J\K\L\M\N\O\P\Q\R\S\T\U\V\W\X\Y\Z
%   Lower-case    \a\b\c\d\e\f\g\h\i\j\k\l\m\n\o\p\q\r\s\t\u\v\w\x\y\z
%   Digits        \0\1\2\3\4\5\6\7\8\9
%   Exclamation   \!     Double quote  \"     Hash (number) \#
%   Dollar        \$     Percent       \%     Ampersand     \&
%   Acute accent  \'     Left paren    \(     Right paren   \)
%   Asterisk      \*     Plus          \+     Comma         \,
%   Minus         \-     Point         \.     Solidus       \/
%   Colon         \:     Semicolon     \;     Less than     \<
%   Equals        \=     Greater than  \>     Question mark \?
%   Commercial at \@     Left bracket  \[     Backslash     \\
%   Right bracket \]     Circumflex    \^     Underscore    \_
%   Grave accent  \`     Left brace    \{     Vertical bar  \|
%   Right brace   \}     Tilde         \~}
%
% \DoNotIndex{\globcount,\globdimen,\if,\fi,\else,\def,\the,\gdef,\global,\relax}
% \makeatletter
% \def\ThisInfo {\ifdefvoid\lsstyle {\scalebox{1.35}[1]{\eTeX}\stretchwith\,{\ named\ marks\ registers}}
%                                                                  {\lsstyle \eTeX{} named marks registers }}
% \def\ttdefault{lmvtt}         \colorlet{pkgcolor}{LimeGreen!50!Black}
% \parindent\z@\parskip.4\baselineskip\topsep\parskip\partopsep\z@
% \newrobustcmd*\FC {{\leavevmode\color{copper}\usefont{T1}{fts}xn FC}}
% \colorlet{linkcolor}{DarkSlateBlue}   \colorlet{csrefcolor}{linkcolor}^^ARoyalBlue!70!Indigo!50!Black}
% \definecolor{macrocode}{rgb}{0.08,0.00,0.15}
% \providerobustcmd*\csred{\cs[\colorlet{csrefcolor}{red}]}
% \def\MacroFont{\ttfamily\bfseries }
% \def\macro@font {\def\Cr@scale{.87}\changefont{fam=pcrs,siz=10pt,ser=m,color=macrocode,spread=1}\let\AltMacroFont\macro@font }
% \AtBeginEnvironment {declcs}{\tabusetup* {font=\bfseries,everyrow=\rowbackground{fill=Snow!220,cell fading=-+{70}{0}{30}},line style=\heavyrulewidth,linesep=1mm}}
% \AtBeginEnvironment {lstlisting}{\Needspace{7\baselineskip}}
% \tikzAtEveryShipout {\ifnum \value{page}>\@ne  \fill [fill=Lime,very nearly transparent] (0,0) rectangle (\paperwidth,-\headheight);\fi }
%  \sectionformat\section[hang]{
%       left=\declmarginwidth,
%       font=\bfseries\Large,
%       bookmark={color=pkgcolor},
%       bottom=\smallskipamount,top=\medskipamount,
%  }
%  \sectionformat\subsection{
%       font=\large\bfseries,
%       bookmark={color=MidnightBlue},
%  }
% \pagesetup [corpus]{norule,
%       font=\footnotesize,
%       head/left=\noindent\raise1.5mm\hbox{\thispackage},
%       head/font+=\mdseries\sffamily,
%       head/right=\noindent\raise1.5mm\hbox{\ThisInfo},
%       foot/left/font=\scriptsize\color{gray!80},
%       foot/left=\vbox to\baselineskip{\vss{{\rotatebox[origin=l]{90}{\thispackage\,[rev.\thisversion]\,\CopyRight2011\,\lower.4ex\hbox{\pkgcolor\NibRight}\,\FC}}}},
%       offset=15mm,
%       left/offset+=15mm,
%       foot/right=\oldstylenums{\arabic{page}}/\oldstylenums{\pageref{LastPage}},
%  }
% \pagesetup [plain]{%
%     norules,font=\scriptsize,
%     offset=15mm,
%     left/offset+=15mm,
%     foot/font=\scriptsize\color[gray]{.55},
%     foot/right=\oldstylenums{\arabic{page}}/\oldstylenums{\pageref{LastPage}},
%     foot/left=\vbox to\baselineskip{\vss{{\rotatebox[origin=l]{90}{\thispackage\,[rev.\thisversion]\,\CopyRight2011\,\lower.4ex\hbox{\pkgcolor\NibRight}\,\FC\quad \xemail{florent.chervet at free.fr}}}}},
% }
% \bookmarksetup{openlevel=3}
%
% \newrobustcmd\IMPLEMENTATION{\bigskip
%       \par \tabubox { X[c] }{ \background {cell fading=fuzzy ring 15 percent,fill=Silver,semitransparent}
%               \tabubox {  *2X[c]  } {\includegraphics [width=20mm,keepaspectratio]{emarks-fingerprint.png}
%                   & \indent\tikz{\fill [decorate,decoration={footprints,foot of=felis silvestris}] (\paperwidth-25mm,-32mm) circle (+8mm);}
%       \\[^2mm _8mm]}
%       \\ \LARGE\thispackage }
%       \clearpage
%       \bookmarksetup {bold*,openlevel=1}
%       \sectionformat \section{bookmark={color=black}} \sectionformat \subsection{bookmark={color=gray}}
%       \section(Implementation)[\lsstyle\textsc{\bfseries Implementation}]{\larger\lsstyle\textsc{\bfseries Implementation}}\label{sec:implementation}\parindent1em
%  }
%
% \tikzAtFirstShipout{\fill [decorate,decoration={footprints,foot of=felis silvestris}] (\paperwidth-25mm,-32mm) circle (+5mm);
%                     \node[anchor=south] at (+41mm,-40mm) {\includegraphics [width=15mm,keepaspectratio] {emarks-fingerprint.png}};}
%
% \title  {\vspace*{-28pt}\Huge\bfseries \CTANhref[emarks]{\pkgcolor emarks}\Footnotemark{*}
%          \tabusetup* {linesep=3mm ,font=\large\changefont{fam=txr}}
%   \tabubox { X[c] }{ \ThisInfo \\
%                       \small\FC   \\
%                       \small\mdseries\thisdate~--~\hyperref[\thisversion]{version \thisversion }
%                    }\vspace*{-12pt}}
% \author {}
% \date   {}
% \makeatother
%
% \maketitle            
%
% \pdftikzcomment [author=interfaces-pdfcomment] {TikZ decorations.footprint library}
%   [semithick,color=pkgcolor,->>] ($(current page.north east)-(7mm,7mm)$) -- ++ (-1.5,-1.5);
%
% \bookmark[bold,view=FitH 0,named=FirstPage,color=pkgcolor]{emarks}
% 
% \Footnotetext{\rlap{*}\kern2em}{\parindent0pt\noindent
% This documentation is produced with the \textt{DocStrip} utility.\par
% \begin{tabu}{X[-3]X[-1] >\ttfamily X}
% \smex To get the package,                   &run:          & etex \thisfile.dtx                  \\
% \smex To get the documentation              &run (thrice): & pdflatex \thisfile.dtx               \\
% \leavevmode\hphantom\smex To get the index, &run:          & makeindex -s gind.ist \thisfile.idx
% \end{tabu}\par
% The \xext{dtx}* is embedded into this \xext{pdf}* thank to \Xpackage{embedfile} by H. Oberdiek.}
%
% \vspace*{-16mm}
% \begin{Abstract}[\leftmargin=6mm\parindent0pt\listparindent0pt\parskip\smallskipamount\parsep\smallskipamount]
%
% \shorttabubox { X }{\eTeX{} defines \numprint {32768} marks registers while \TeX{} provided only one \emph!\\ \cbackground {fill=pkgcolor,cell fading=*,nearly transparent}-}
%
% So small, this package provides commands to access \eTeX{} marks registers by their name rather than by their number.
% This makes the use of them far more comfortable than ``old \LaTeX{}'' tricks with \cs\markright, \cs\markboth \etc.
%
% \thispackage requires \eTeX{} and the generic package \CTANhref[etex-pkg]{\xfile{etex.sty}} for allocation.
%
%  Presently designed to be loaded by \textt\LaTeX{}, a \textt{plain \TeX} version might be provided later...
%
% \end{Abstract}
%
% \tocsetup{
%  before+=\hypersetup {linkcolor=black},
%  section/skip=4pt plus2pt minus2pt,
%  subsection/skip=0pt plus2pt minus2pt,
%  section/dotsep=,
%  subsection/dotsep=,
%  subsection/pagenumbers=off,
%  dotsep=1.5mu,
%  title/top = 12pt,
%  dot=\hbox{$\scriptscriptstyle\cdotp$},
%  title={\pkgcolor\leaders\vrule height3.4pt depth-3pt\hfill\null}\quad Contents of \textsfbf{\pkgcolor emarks}\quad{\pkgcolor\leaders\vrule height3.4pt depth-3pt\hfill\null},
%  title/bottom=4pt,
%  multicols/beforeend=\aftergroup\tocrule,
%  columns=2,columns/rule color=pkgcolor,no columns rule,
%  }
%  \def\tocrule{\leavevmode{\pkgcolor\hrule}}
%
% \tableofcontents  \pagestyle{corpus}
%
% \listofsetup {lol}{pagestyle=,
%       title/font=\large\bfseries,
%       title={\pkgcolor\leaders\vrule height3.4pt depth-3pt\hfill\null}\quad List of the listings / examples\quad{\pkgcolor\leaders\vrule height3.4pt depth-3pt\hfill\null},
%       after=\tocrule,
%       twocolumns=false,
%       label=lol,bookmark={text=Listings},
%       lstlisting/dotsep=,
%       lstlisting/pagenumbers=off,
%       lstlisting/leaders=,
%       lstlisting/font=\hfil,
%       lstlisting/number/width=0pt,
% }
%
% \addcontentsline{lol}{lstlisting}{To be done !}
%
% \listoflstlisting
%
% \bookmarksetup{bold*}
%
% \section{The \hologo{eTeX} marks registers}
% \label{userinterface}
%
%
% \begin{declcs}[{ | *2X[-1] | }] \marksthe\M{named-mark}\M{content} & \csanchor\marksthecs\M{named-mark}\M{cs-name} \\
%                        \cs\marksthe*\M{named-mark}\M{content}      &       \cs\marksthecs*\M{named-mark}\M{cs-name}
% \end{declcs}\declcsbookmark\marksthe\declcsbookmark\marksthecs
%
%
% \tabusetup* { extra sep = .5\parskip }
% \declmargin\begin{tabu*}to\dimexpr\linewidth-\declmarginwidth {X[-1]X @{} }
%   \cs\marksthe\M*{section}\M{content} & Marks the \meta{content} into the named mark register \meta{section} in the same way as the \eTeX{} primitive \cs\marks:
%                                         in particular the \meta{content} is immediately expanded. \par
%                                         If the mark register does not exist, it is created (or allocated) with \cs\newmarks (in \xfile{etex.sty}).
%  \\
%  \cs\marksthe*\M*{section}\M{content} & does the same but the \meta{content} is not expanded. The current values of counters, \cs\thesection \etc. will be wrong:
%                                         they will expand to the value they have at the time the mark register is read, not at the time of \cs\marksthe*.\par
%                                         Yet \cs\marksthe* is useful to mark a title only like in�
%                                         \shorttabubox* {X}{ \cs\def\cs\sectionmark \#1\M*{\cs\marksthe*\M*{section}\M{\#1}}}�
%                                         or to control the expansion (the \meta{content} can be expanded before marking in a way and with the protections desired by the user).
% \end{tabu*}
%
% Similarly \cs\marksthecs\M{subsubsection}\M{cs-name} marks the content of \cs{cs-name} by the mean of the named mark
% register \meta{subsubsection}. \meta{cs-name} is really the \emph{name of the control sequence} and not the control sequence itself:
% it does not start with \textttbf{\csname @backslashchar\endcsname}.�
% If \cs{cs-name} is empty the mark is empty, but if it is undefined or \cs\relax: nothing is marked: at reading time, the mark register never expands
% to \cs\undefined nor to \cs\relax.
%
% The syntax follows \eTeX{} \cs\marks primitive (a token-like syntax): braces are mandatory around the \M{content} to be marked, even if it is made of one single token.
%
%   \begin{declcs}[{ | *2X[-1] |}]\thefirstmarks&\M{named-mark}\textsuperscript{\textsc{expandable}} \\
%                \csanchor\thebotmarks  &\M{named-mark}\textsuperscript{\textsc{expandable}} \\
%                \csanchor\thetopmarks  &\M{named-mark}\textsuperscript{\textsc{expandable}}
% \end{declcs}\declcsbookmark\thefirstmarks\declcsbookmark\thebotmarks\declcsbookmark\thetopmarks
%
% Those commands are expandable in exactly one step of expansion. If the \meta{named-mark} mark register does not exists,
% the expansion is null (\ie nothing is done nor printed).
%
% \tabusetup* {colsep=3pt}
% \begin{tabu*}{ @{} X[-1] X @{} }
% \cs\thefirstmarks\M{chapter}  &expands to the content of the first invocation of \cs\marksthe\M{chapter}
%   on the current page if \cs\marksthe\M*{chapter} was used on the current page,
%   or the last invocation of \cs\marksthe\M*{chapter} if no marks occured on the current page.
% \\
%   \hfill\small \TeX nically this is &\cs\firstmarks \cs\marks@chapter
% \\[1mm]
% \cs\thebotmarks\M{chapter}    &expands to the content of the last invocation of \cs\marksthe\M{chapter} (the most recent \cs\marks). \\
%   \hfill\small \TeX nically this is &\cs\botmarks \cs\marks@chapter
% \\[1mm]
% \cs\thetopmarks\M{chapter}    &expands to the content of \cs\botmarks at the time \TeX{} shipped out the last page. \\
%  \hfill\small \TeX nically this is  &\cs\topmarks\cs\marks@chapter
% \end{tabu*}
%
% \def\OR{\stform|}
% \begin{declcs}[{ | X | }]\getthemarks  \cs\firstmarks \OR \cs\botmarks \OR \cs\topmarks \M{named-mark} \M*{\cs{control-sequence}} \\
%      \csanchor\getthefirstmarks \M{named-mark} \M*{\cs{control-sequence}} \\
%      \csanchor\getthebotmarks \M{named-mark} \M*{\cs{control-sequence}} \\
%      \csanchor\getthetopmarks \M{named-mark} \M*{\cs{control-sequence}}
% \end{declcs}\declcsbookmark\getthemarks\declcsbookmark[rellevel=2]\getthefirstmarks\declcsbookmark[rellevel=2]\getthebotmarks\declcsbookmark[rellevel=2]\getthetopmarks
%
% \cs\thefirstmarks, \cs\thebotmarks and \cs\thetopmarks expand the content of the mark. To get it in a macro
% \cs\getthemarks can be used: \cs{control-sequence} is defined as a parameterless macro whose replacement text is
% the content of the given mark register.
%
% If the \meta{named-mark} mark register does not exist, the meaning of\, \cs{control-sequence}\, is
% \textt{undefined}.
%
% \begin{declcs}\ifmarksvoid \M*{\cs\firstmarks}\M*{named-mark}\M{true}\M{false} \\
%            \cs\ifmarksvoid \M*{\cs\botmarks}\M*{named-mark}\M{true}\M{false} \\
%            \cs\ifmarksvoid \M*{\cs\topmarks}\M*{named-mark}\M{true}\M{false} \\
% \end{declcs}\declcsbookmark\ifmaksvoid
%
% \cs\ifmarksvoid expands the \M{true} part if either: ^^A\loggingall
% \begin{itemize}^^A[topsep=0pt,itemsep=0pt]
% \item The requested mark register is empty,
% \item The requested mark register is \cs\undefined,
% \item The requested mark register is \cs\relax,
% \item The \meta{named-mark} mark register does not exist.
% \end{itemize}
%
%
%
% \begin{declcs}\ifmarksequal\M*{\cs\firstmarks}\M*{\cs\topmarks}\M*{named-mark}\M{true}\M{false}   \\
%            \cs\ifmarksequal\M*{\cs\firstmarks}\M*{\cs\botmarks}\M*{named-mark}\M{true}\M{false}
% \end{declcs}\declcsbookmark\ifmarksequal
%
% Pretty often we want to compare the botmarks against the firstmarks or the topmarks, to adapt the header and/or footer
% in case those marks are equal or different, \ie in case the page contains a new section title or not:
%
% \cs\ifmarksequal expands the code in the \M{true} or the \M{false} part if the extraction of the marks are equal
% (in the sense of \cs\ifx) or different.
%
% If any of the marks register\, \cs{marks@\meta{named-mark}}\, does not exist the \M{false} part is expanded.
%
% If marks are used both at \cs\sectionmark \textbf{and at} \cs\sectionbreak then the following assertions are true:�
% \begin{shorttabu}{ !\textbullet  l  @{\,=\,}  l  !{$\Leftrightarrow$} l  }
% \cs\firstmarks &\cs\botmarks &there is at most one section title on the current page; \\
% \cs\topmarks   &\cs\botmarks &there is no section title on the current page;          \\
% \cs\firstmarks &\cs\topmarks &the last section title continues on the current page.
% \end{shorttabu}
%
%
% \begin{declcs}\showthemarks \M{named-mark}
% \end{declcs}\declcsbookmark\showthemarks
%
% \cs\showthemarks is for debugging purpose: it prints a message in the \xext{log}* and the ``standard error'' with
% the contents of the marks \cs\firstmarks, \cs\botmarks and \cs\topmarks for the \meta{named-mark} register given.
% Then it executes \cs\show on the extracted content of \cs\firstmarks in order to stop compilation at that point:
% the console displays the contents of \cs\firstmarks, \cs\botmarks and \cs\topmarks.
%
% ^^A\loggingall
% ^^A\showthemarks{section}
%
% \StopEventually{ }
%
% \IMPLEMENTATION
%
% \subsection*{Identification}       \makeatletter
%
% The package namespace is \cs\em@rks
%
%    \begin{macrocode}
%<*package>
\NeedsTeXFormat{LaTeX2e}[2005/12/01]
\ProvidesPackage{emarks}
         [2011/03/26 v1.0 - e-TeX named marks registers (FC)]
\RequirePackage {etex}
%    \end{macrocode}
%
%    \begin{macro}{\emarks@newmarks}
%
%    allocates a new marks register if it does not exists.
%
%    \begin{macrocode}
\def\emarks@newmarks #1{\PackageInfo {emarks}{New marks register `#1'}%
                        \newmarks #1% \newmarks is global !!
}% \emarks@newmarks
%    \end{macrocode}
%    \end{macro}
%
%    \begin{macro}{\marksthe}
%    \begin{macro}{\marksthecs}
%
%   \noindent\shorttabubox* { X } { \lstinline! \marksthe   { named-mark }{ general text } ! \\
%                          \lstinline! \marksthe*  { named-mark }{ general text } ! \\
%                          \lstinline! \marksthe   { named-mark }{ named control sequence } ! \\
%                          \lstinline! \marksthecs*{ named-mark }{ named control sequence } ! \\
%                        }
%
%    \begin{macrocode}
\protected\def\marksthe   {\emarks@setmarks {}}
\protected\def\marksthecs {\emarks@setmarks {\toks@\expandafter{\csname\the\toks@\endcsname}}}
\def\emarks@setmarks #1{\begingroup \@ifstar {\emarks@ {#1}\def  }
                                             {\emarks@ {#1}\edef }%
}% \emarks@setmarks
\def\emarks@ #1#2#3{\def\@tempa
      {#1#2\@tempa {\the\toks@ }\expandafter\emarks@marks \csname marks@#3\endcsname }%
                                                \afterassignment \@tempa \toks@ =
}% \emarks@
\def\emarks@marks #1{\ifx \relax#1\emarks@newmarks #1\fi \marks #1{\@tempa }\endgroup }
%    \end{macrocode}
%    \end{macro}
%    \end{macro}
%
%    \begin{macro}{\thefirstmarks}
%    \begin{macro}{\thebotmarks}
%    \begin{macro}{\thetopmarks}
%
%   \cs\thefirstmarks extract the \cs\firstmarks from a named mark register.
%
%   The macros are purely expandable in exactly one step of expansion.
%
%    \begin{macrocode}
\newcommand*\thefirstmarks {\romannumeral \emarks@themarks \firstmarks  }
\newcommand*\thebotmarks   {\romannumeral \emarks@themarks \botmarks    }
\newcommand*\thetopmarks   {\romannumeral \emarks@themarks \topmarks    }
\def\emarks@themarks #1#2{\expandafter \ifx
    \csname\ifcsname marks@#2\endcsname marks@#2\else relax\fi\endcsname\relax
            \expandafter \z@
    \else   \expandafter \z@ #1\csname marks@#2\expandafter \endcsname  \fi
}% \emarks@themarks
%    \end{macrocode}
%    \end{macro}
%    \end{macro}
%    \end{macro}
%
%    \begin{macro}{\getthemarks}
%    \begin{macro}{\getthefirstmarks}
%    \begin{macro}{\getthebotmarks}
%    \begin{macro}{\getthetopmarks}
%
%    Extract the marks and store in a parameterless macro.
%
%    \begin{macrocode}
\protected\def\getthemarks #1#2#3{\ifcsname marks@#2\endcsname
          \expandafter \def \expandafter #3\expandafter {#1\csname marks@#2\endcsname}%
    \else                           \let #3=\@undefined    \fi
}% \getthemarks
\protected\def\getthefirstmarks {\getthemarks   \firstmarks }
\protected\def\getthebotmarks   {\getthemarks   \botmarks   }
\protected\def\getthetopmarks   {\getthemarks   \topmarks   }
%    \end{macrocode}
%    \end{macro}
%    \end{macro}
%    \end{macro}
%    \end{macro}
%
%    \begin{macro}{\ifmarksvoid}
%
%   Test if a marks is defined, not empty and not \cs\relax.
%
%    \begin{macrocode}
\protected\def\ifmarksvoid #1#2{\begingroup \getthemarks {#1}{#2}\x
    \ifodd \ifdefined\x \ifx \x\relax 0 \fi \ifx \x\@empty 0 \fi \else 0 \fi
           1 \endgroup\expandafter\@secondoftwo
    \else    \endgroup\expandafter\@firstoftwo      \fi
}% \ifmarksvoid
%    \end{macrocode}
%    \end{macro}
%
%    \begin{macro}{\ifmarksequal}
%
%   Test with \cs\ifx if two marks are equal:�
%   \shorttabubox* {X} { \lstinline ! \ifmarksequal \firstmarks \botmarks { named-mark } ! }
%
%
%    \begin{macrocode}
\protected\def\ifmarksequal #1#2#3{\begingroup \getthemarks{#1}{#3}\x \getthemarks{#2}{#3}\y
        \expandafter \endgroup \ifodd \ifdefined\x \ifdefined\y \ifx \x\y 0 \fi\fi\fi
                                      1 \expandafter\@secondoftwo
                               \else    \expandafter\@firstoftwo     \fi
}% \ifmarksequal
%    \end{macrocode}
%    \end{macro}
%
%    \begin{macro}{\showthemarks}
%
%    Shows the contents of the marks registers
%
%    \begin{macrocode}
\protected\def\showthemarks #1{\begingroup  \emarks@showthemarks 0{#1}\firstmarks
                                            \emarks@showthemarks 2{#1}\botmarks
                                            \emarks@showthemarks 4{#1}\topmarks
    \message{firstmarks "#1": \the\toks0^^J%
             botmarks   "#1": \the\toks2^^J%
             topmarks   "#1": \the\toks4^^J}\show\@tempa
    \endgroup
}% \showthemarks
\def\emarks@showthemarks #1#2#3{\getthemarks #3{#2}\@tempa \toks #1 = \ifdefined\@tempa
    \expandafter\ifx \noexpand\@tempa\@tempa {}\else \expandafter {\@tempa }\fi
                                                                      \else {}\fi
}% \emarks@showthemarks
%    \end{macrocode}
%    \end{macro}
%
%    \begin{macrocode}
%</package>
%    \end{macrocode}
%
%
% \begin{History}
%   \sectionformat\subsection{font=\normalsize\pkgcolor,bottom=0pt,top=\smallskipamount }\makeatletter
%
%   \begin{Version}{2011/03/26}{1.0}
%   \item First version. \\
%   \end{Version}
%
% \end{History}
%
% \begin{thebibliography}{9}
%
% \bibitem{etex} The \xpackage{etex} package by Peter Breitenlohner \\
%       \getpackageinfo{etex} \\
%       \CTANhref[etex-pkg]{\nolinkurl{CTAN:help/Catalogue/entries/etex-pkg.html}}
%
% \end{thebibliography}
%
% \clearpage
% \PrintIndex
%
% \Finale