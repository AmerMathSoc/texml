\ProvidesFile{finstrut.tex}[2011/03/14 documenting finstrut.sty (UL)]
\title{\textsf{finstrut.sty}\\---\\Final Strut Allowing Vertical 
       Mode---The \texttt{\cs{@finalstrut}} Encyclopedia and 
       Toolkit\thanks{This document describes version 
       \textcolor{blue}{\UseVersionOf{finstrut.sty}} 
       of \textsf{finstrut.sty} as of \UseDateOf{finstrut.sty}. 
       This upload is a birthday package for Donald Arseneau---he 
       will be so happy!}}
% \listfiles                                          %% 2010/12/22
{ \RequirePackage{makedoc}[2010/12/20] \ProcessLineMessage{} 
  \MakeJobDoc{18}{\SectionLevelTwoParseInput} 
}
\documentclass{article}%% TODO paper dimensions!?
\ProvidesFile{makedoc.cfg}[{2013/03/25 documentation settings}] 
%%
\author{Uwe L\"uck\thanks{%
        \url{http://contact-ednotes.sty.de.vu}}}
%%
%% 'hyperref':
\RequirePackage{ifpdf}
\usepackage[%
  \ifpdf
%     bookmarks=false,                  %% 2010/12/22
%     bookmarksnumbered,
    bookmarksopen,                      %% 2011/01/24!?
    bookmarksopenlevel=2,               %% 2011/01/23
%     pdfpagemode=UseNone,
%     pdfstartpage=10,
    pdfstartview=FitH,                  %% 2012/11/26 again
%     pdfstartview=0 0 100,             %% 2011/08/22
%     pdfstartview={XYZ null null 1},   %% 2011/08/25
%     pdfstartview={XYZ null null null},%% 2011/08/25
%     pdfstartview={XYZ null null .5},    %% 2011/08/26
%     pdffitwindow=true,          %% 2011/08/22
    citebordercolor={ .6 1    .6},
    filebordercolor={1    .6 1},
    linkbordercolor={1    .9  .7},
     urlbordercolor={ .7 1   1},   %% playing 2011/01/24
  \else
    draft
  \fi
]{hyperref}
\hypersetup{% 
    pdfauthor={Uwe L\374ck}% 
}
%% metadata, |\MDkeywords{<text>}|, |\MDkeywordsstring|:
%% %% 2011/08/22:
\makeatletter
  \newcommand*{\MDkeywords}[1]{%
    \gdef\MDkeywordsstring{#1}%
    \hypersetup{pdfkeywords=\MDkeywordsstring}%% TODO!?
  }
  \@onlypreamble\MDkeywords
%% |\MDaddtoabstract{<par-head>}|, `:' added:
  \newcommand*{\MDaddtoabstract}[1]{%           %% 2012/05/10
    \par\smallskip\noindent
    \strong{#1:}\quad\ignorespaces}
%% \pagebreak[2]
%% |\printMDkeywords|:
  \newcommand*{\printMDkeywords}{%
    \MDaddtoabstract{Keywords}%
    \MDkeywordsstring 
%     \global\let\MDkeywordsstring\relax    %% `%' 2012/11/12
  }
%% The previous definitions mainly are useful with a variant 
%% |\begin{MDabstract}| of \LaTeX's `{abstract}' environment:
  \newenvironment{MDabstract}
                 {\abstract\noindent
                  \hspace{1sp}%% for niceverb
                  \ignorespaces}
                 {\@ifundefined{MDkeywordsstring}%
                               {}%
                               {\printMDkeywords}%
                  \global\let\MDabstract\relax    %% 2012/11/12
                  \global\let\endMDabstract\relax %% 2012/11/12
                  \endabstract}
%% |\[MD]docnewline| 2012/11/12 from `readprov.tex':
  \newcommand*{\MDdocnewline}{\leavevmode\@normalcr[\topsep]}
%% <- `\leavevmode' for use with `\paragraph'.
%%    Sometimes needs to be preceded by a space.
%% 
%% |\MDfinaldatechecks[<tex-script>]| with \ctanpkgref{filedate}:
  \newcommand*{\MDfinaldatechecks}[1][fdatechk]{%
    \AtEndDocument{%
%       \clearpage %% 2013/03/25 no avail -- with `filedate'!
      \def\@pkgextension{sty}%
      \def\NeedsTeXFormat##1[##2]{}%
      \noNiceVerb                       %% 2013/03/22
      \input{#1}%
    }}
  \@onlypreamble\MDfinaldatechecks
\makeatother
%% Use other packages:
\RequirePackage{niceverb}[2011/01/24] 
\RequirePackage{readprov}               %% 2010/12/08
\RequirePackage{hypertoc}               %% 2011/01/23
\RequirePackage{texlinks}               %% 2011/01/24
\RequirePackage{relsize}                %% 2011/06/27
\RequirePackage{color}                  %% 2011/08/06
\RequirePackage{lmodern}                %% 2012/10/29
\RequirePackage{filedate}               %% 2012/11/12
\RequirePackage{filesdo}                %% 2013/03/22 
%% \pagebreak[3]
%% Logical markup:\qquad  |\strong{<chars>}|, |\meta{<chars>}|, 
%% |\acro{<chars>}|, |\pkg{<chars>}|, 
%% |\code{<chars>}|, |\file{<chars>}|:{\sloppy\par}
\makeatletter
  \def\do#1#2{\@ifdefinable#1{\let#1#2}}%% 2012/07/13
  \do\strong\textbf \do\file\texttt \do\acro\textsmaller 
  %% <- wrong tests before 2012/07/13
  \do\meta\textit   \do \pkg\textsf \do\code\texttt
  \ifpdf
    \pdfstringdefDisableCommands{%
        \let\acro\textrm 
        \let\file\textrm                            %% 2011/11/09
        \let\code\textrm                            %% 2011/11/20
        \let\pkg \textrm                            %% 2012/03/23
    }
  \fi
  %% TODO 2011/07/22 -> `htlogml.sty'
\makeatother
%% |\qtdcode{<text>}|: 2012/10/24:
    \newcommand*{\qtdcode}[1]{`\code{#1}'} 
%% |\pkgtitle{<package-name>}{<caption>}| 
\newcommand*{\pkgtitle}[2]{%            %% 2012/07/13
    \global\let\pkgtitle\relax
    \pkg{\huge #1}\\---\\#2\thanks{This 
       document describes version 
       \textcolor{blue}{\UseVersionOf{\jobname.sty}} 
       of \textsf{\jobname.sty} as of \UseDateOf{\jobname.sty}.}}
%% TODO: %% |\TODO| bad with `mdoccorr.cfg'
\newcommand*{\TODO}{\textcolor{blue}{\acro{TODO}}}  %% 2012/11/06
%% `\MDsampleinput[{<file>}' was added 2012/11/06. 
%% Problems with `myfilist.tex' were due to 'parskip.sty'
%% there. On 2012/11/12, we change the former simple macro to a 
%% much more complex
%% |\MDsamplecodeinput[<add-hfuss>]{<file>}| 
\newcommand*{\MDsamplecodeinput}[2][]{%
    \begingroup
        \vskip\bigskipamount \hrule
        \nobreak\vskip-\parskip 
%         \nobreak\vskip\medskipamount
%% Previous mistake (same below) due to manual change 
%% of `\topsep' in the file `myfilist.tex' (2012/11/30).
        \ifx\\#1\\\else
            \hfuzz=\textwidth \advance\hfuzz#1\relax
        \fi
        \noNiceVerb \verbatiminput{#2}%
%         \nobreak\vskip\medskipamount 
        \hrule \vskip-\parskip 
        \bigskip %%% \bigbreak
%% `\bigbreak' made much larger space in `myfilist.tex'.
    \endgroup
}
%% |\ctanpkgdref{<pkg-id>}| adds the printed link to 
%% `ctan.org/pkg' as a footnote. There is a little space 
%% for coloured link borders:
\newcommand*{\ctanpkgdref}[1]{%
    \ctanpkgref{#1}\,\urlfoot{CtanPkgRef}{#1}}
\errorcontextlines=4
\pagestyle{headings}

\endinput 
 %% shared formatting settings
\newcommand*{\lt}{<} \newcommand*{\gt}{>}           %% 2010/12/22
\providecommand*{\strong}{\textbf}                  %% 2010/12/15
\ReadPackageInfos{finstrut}
\usepackage{color}
\makeatletter \@beginparpenalty\@lowpenalty \makeatother
\sloppy
\begin{document}
% \author{Uwe L\"uck\thanks{\urlhttpref{http://contact-ednotes.sty.de.vu}} 
%         \and 
%         Donald Arseneau}
\maketitle
\begin{abstract}\noindent
\LaTeX\ internally inserts `\@finalstrut<strutbox>'
at the end of footnotes or `p' (paragraph) `tabular' cells.
When the user's entry ends on a `\par' token---which may 
be issued by some more-general-purpose user macro 
such as the `\lipsum' command from the \ctanpkgref{lipsum}
package\urlfoot{CtanPkgRef}{lipsum} for dummy text---this 
produces a usually unwanted extra, empty line.
'finstrut.sty' changes `\@finalstrut' in order to avoid 
this effect.
  \par
The first version of the package just had this minimum purpose. 
But then Donald Arseneau has convinced me to consider 
a wide range of cases and to replace my code by his proposals.
v0.5 again adds an idea of my own ones:     %% 2011/03/14:
instead of testing for `hmode' vs.~`vmode', 
`vmode' may be \emph{forced} by a `\par'---saves tokens!
In order not to break code that some package writers may have 
used so far, choices by package options are offered.
v0.4 offered choice among 6 definitions, with v0.5 they are 9---!
(Not 9!.)
  \par\smallskip\noindent
\strong{Keywords:}\quad footnotes; tables, dummy text, 
macro programming
\end{abstract}
\tableofcontents

%   \newpage 
\section{Remark on 'lipsum'}
Almost at the same moment that 'finstrut' v0.1 was installed 
(2011-02-09), a new version v1.1    %% v0.4: v1.1
of 'lipsum' was uploaded that offers a package option 
`[nopar]'                           %% v0.4
to avoid a final `\par'.

\section{Installing and Calling}
The package file 'finstrut.sty' is provided ready, 
installation only requires putting it somewhere where \TeX\ finds it 
(which may need updating the filename data 
 base).\urlfoot{ukfaqref}{inst-wlcf}

Below the `\documentclass' line(s) and above `\begin{document}', 
you load 'finstrut.sty' (as usually) by 
\[`\usepackage{finstrut}'\qquad \mbox{or by}\qquad 
  `\usepackage[<options>]{finstrut}'\]---<options> described below.

\section{The Package File}
\subsection{Header (Legalize)}
\ProvidesFile{finstrut.tex}[2011/03/14 documenting finstrut.sty (UL)]
\title{\textsf{finstrut.sty}\\---\\Final Strut Allowing Vertical 
       Mode---The \texttt{\cs{@finalstrut}} Encyclopedia and 
       Toolkit\thanks{This document describes version 
       \textcolor{blue}{\UseVersionOf{finstrut.sty}} 
       of \textsf{finstrut.sty} as of \UseDateOf{finstrut.sty}. 
       This upload is a birthday package for Donald Arseneau---he 
       will be so happy!}}
% \listfiles                                          %% 2010/12/22
{ \RequirePackage{makedoc}[2010/12/20] \ProcessLineMessage{} 
  \MakeJobDoc{18}{\SectionLevelTwoParseInput} 
}
\documentclass{article}%% TODO paper dimensions!?
\ProvidesFile{makedoc.cfg}[{2013/03/25 documentation settings}] 
%%
\author{Uwe L\"uck\thanks{%
        \url{http://contact-ednotes.sty.de.vu}}}
%%
%% 'hyperref':
\RequirePackage{ifpdf}
\usepackage[%
  \ifpdf
%     bookmarks=false,                  %% 2010/12/22
%     bookmarksnumbered,
    bookmarksopen,                      %% 2011/01/24!?
    bookmarksopenlevel=2,               %% 2011/01/23
%     pdfpagemode=UseNone,
%     pdfstartpage=10,
    pdfstartview=FitH,                  %% 2012/11/26 again
%     pdfstartview=0 0 100,             %% 2011/08/22
%     pdfstartview={XYZ null null 1},   %% 2011/08/25
%     pdfstartview={XYZ null null null},%% 2011/08/25
%     pdfstartview={XYZ null null .5},    %% 2011/08/26
%     pdffitwindow=true,          %% 2011/08/22
    citebordercolor={ .6 1    .6},
    filebordercolor={1    .6 1},
    linkbordercolor={1    .9  .7},
     urlbordercolor={ .7 1   1},   %% playing 2011/01/24
  \else
    draft
  \fi
]{hyperref}
\hypersetup{% 
    pdfauthor={Uwe L\374ck}% 
}
%% metadata, |\MDkeywords{<text>}|, |\MDkeywordsstring|:
%% %% 2011/08/22:
\makeatletter
  \newcommand*{\MDkeywords}[1]{%
    \gdef\MDkeywordsstring{#1}%
    \hypersetup{pdfkeywords=\MDkeywordsstring}%% TODO!?
  }
  \@onlypreamble\MDkeywords
%% |\MDaddtoabstract{<par-head>}|, `:' added:
  \newcommand*{\MDaddtoabstract}[1]{%           %% 2012/05/10
    \par\smallskip\noindent
    \strong{#1:}\quad\ignorespaces}
%% \pagebreak[2]
%% |\printMDkeywords|:
  \newcommand*{\printMDkeywords}{%
    \MDaddtoabstract{Keywords}%
    \MDkeywordsstring 
%     \global\let\MDkeywordsstring\relax    %% `%' 2012/11/12
  }
%% The previous definitions mainly are useful with a variant 
%% |\begin{MDabstract}| of \LaTeX's `{abstract}' environment:
  \newenvironment{MDabstract}
                 {\abstract\noindent
                  \hspace{1sp}%% for niceverb
                  \ignorespaces}
                 {\@ifundefined{MDkeywordsstring}%
                               {}%
                               {\printMDkeywords}%
                  \global\let\MDabstract\relax    %% 2012/11/12
                  \global\let\endMDabstract\relax %% 2012/11/12
                  \endabstract}
%% |\[MD]docnewline| 2012/11/12 from `readprov.tex':
  \newcommand*{\MDdocnewline}{\leavevmode\@normalcr[\topsep]}
%% <- `\leavevmode' for use with `\paragraph'.
%%    Sometimes needs to be preceded by a space.
%% 
%% |\MDfinaldatechecks[<tex-script>]| with \ctanpkgref{filedate}:
  \newcommand*{\MDfinaldatechecks}[1][fdatechk]{%
    \AtEndDocument{%
%       \clearpage %% 2013/03/25 no avail -- with `filedate'!
      \def\@pkgextension{sty}%
      \def\NeedsTeXFormat##1[##2]{}%
      \noNiceVerb                       %% 2013/03/22
      \input{#1}%
    }}
  \@onlypreamble\MDfinaldatechecks
\makeatother
%% Use other packages:
\RequirePackage{niceverb}[2011/01/24] 
\RequirePackage{readprov}               %% 2010/12/08
\RequirePackage{hypertoc}               %% 2011/01/23
\RequirePackage{texlinks}               %% 2011/01/24
\RequirePackage{relsize}                %% 2011/06/27
\RequirePackage{color}                  %% 2011/08/06
\RequirePackage{lmodern}                %% 2012/10/29
\RequirePackage{filedate}               %% 2012/11/12
\RequirePackage{filesdo}                %% 2013/03/22 
%% \pagebreak[3]
%% Logical markup:\qquad  |\strong{<chars>}|, |\meta{<chars>}|, 
%% |\acro{<chars>}|, |\pkg{<chars>}|, 
%% |\code{<chars>}|, |\file{<chars>}|:{\sloppy\par}
\makeatletter
  \def\do#1#2{\@ifdefinable#1{\let#1#2}}%% 2012/07/13
  \do\strong\textbf \do\file\texttt \do\acro\textsmaller 
  %% <- wrong tests before 2012/07/13
  \do\meta\textit   \do \pkg\textsf \do\code\texttt
  \ifpdf
    \pdfstringdefDisableCommands{%
        \let\acro\textrm 
        \let\file\textrm                            %% 2011/11/09
        \let\code\textrm                            %% 2011/11/20
        \let\pkg \textrm                            %% 2012/03/23
    }
  \fi
  %% TODO 2011/07/22 -> `htlogml.sty'
\makeatother
%% |\qtdcode{<text>}|: 2012/10/24:
    \newcommand*{\qtdcode}[1]{`\code{#1}'} 
%% |\pkgtitle{<package-name>}{<caption>}| 
\newcommand*{\pkgtitle}[2]{%            %% 2012/07/13
    \global\let\pkgtitle\relax
    \pkg{\huge #1}\\---\\#2\thanks{This 
       document describes version 
       \textcolor{blue}{\UseVersionOf{\jobname.sty}} 
       of \textsf{\jobname.sty} as of \UseDateOf{\jobname.sty}.}}
%% TODO: %% |\TODO| bad with `mdoccorr.cfg'
\newcommand*{\TODO}{\textcolor{blue}{\acro{TODO}}}  %% 2012/11/06
%% `\MDsampleinput[{<file>}' was added 2012/11/06. 
%% Problems with `myfilist.tex' were due to 'parskip.sty'
%% there. On 2012/11/12, we change the former simple macro to a 
%% much more complex
%% |\MDsamplecodeinput[<add-hfuss>]{<file>}| 
\newcommand*{\MDsamplecodeinput}[2][]{%
    \begingroup
        \vskip\bigskipamount \hrule
        \nobreak\vskip-\parskip 
%         \nobreak\vskip\medskipamount
%% Previous mistake (same below) due to manual change 
%% of `\topsep' in the file `myfilist.tex' (2012/11/30).
        \ifx\\#1\\\else
            \hfuzz=\textwidth \advance\hfuzz#1\relax
        \fi
        \noNiceVerb \verbatiminput{#2}%
%         \nobreak\vskip\medskipamount 
        \hrule \vskip-\parskip 
        \bigskip %%% \bigbreak
%% `\bigbreak' made much larger space in `myfilist.tex'.
    \endgroup
}
%% |\ctanpkgdref{<pkg-id>}| adds the printed link to 
%% `ctan.org/pkg' as a footnote. There is a little space 
%% for coloured link borders:
\newcommand*{\ctanpkgdref}[1]{%
    \ctanpkgref{#1}\,\urlfoot{CtanPkgRef}{#1}}
\errorcontextlines=4
\pagestyle{headings}

\endinput 
 %% shared formatting settings
\newcommand*{\lt}{<} \newcommand*{\gt}{>}           %% 2010/12/22
\providecommand*{\strong}{\textbf}                  %% 2010/12/15
\ReadPackageInfos{finstrut}
\usepackage{color}
\makeatletter \@beginparpenalty\@lowpenalty \makeatother
\sloppy
\begin{document}
% \author{Uwe L\"uck\thanks{\urlhttpref{http://contact-ednotes.sty.de.vu}} 
%         \and 
%         Donald Arseneau}
\maketitle
\begin{abstract}\noindent
\LaTeX\ internally inserts `\@finalstrut<strutbox>'
at the end of footnotes or `p' (paragraph) `tabular' cells.
When the user's entry ends on a `\par' token---which may 
be issued by some more-general-purpose user macro 
such as the `\lipsum' command from the \ctanpkgref{lipsum}
package\urlfoot{CtanPkgRef}{lipsum} for dummy text---this 
produces a usually unwanted extra, empty line.
'finstrut.sty' changes `\@finalstrut' in order to avoid 
this effect.
  \par
The first version of the package just had this minimum purpose. 
But then Donald Arseneau has convinced me to consider 
a wide range of cases and to replace my code by his proposals.
v0.5 again adds an idea of my own ones:     %% 2011/03/14:
instead of testing for `hmode' vs.~`vmode', 
`vmode' may be \emph{forced} by a `\par'---saves tokens!
In order not to break code that some package writers may have 
used so far, choices by package options are offered.
v0.4 offered choice among 6 definitions, with v0.5 they are 9---!
(Not 9!.)
  \par\smallskip\noindent
\strong{Keywords:}\quad footnotes; tables, dummy text, 
macro programming
\end{abstract}
\tableofcontents

%   \newpage 
\section{Remark on 'lipsum'}
Almost at the same moment that 'finstrut' v0.1 was installed 
(2011-02-09), a new version v1.1    %% v0.4: v1.1
of 'lipsum' was uploaded that offers a package option 
`[nopar]'                           %% v0.4
to avoid a final `\par'.

\section{Installing and Calling}
The package file 'finstrut.sty' is provided ready, 
installation only requires putting it somewhere where \TeX\ finds it 
(which may need updating the filename data 
 base).\urlfoot{ukfaqref}{inst-wlcf}

Below the `\documentclass' line(s) and above `\begin{document}', 
you load 'finstrut.sty' (as usually) by 
\[`\usepackage{finstrut}'\qquad \mbox{or by}\qquad 
  `\usepackage[<options>]{finstrut}'\]---<options> described below.

\section{The Package File}
\subsection{Header (Legalize)}
\ProvidesFile{finstrut.tex}[2011/03/14 documenting finstrut.sty (UL)]
\title{\textsf{finstrut.sty}\\---\\Final Strut Allowing Vertical 
       Mode---The \texttt{\cs{@finalstrut}} Encyclopedia and 
       Toolkit\thanks{This document describes version 
       \textcolor{blue}{\UseVersionOf{finstrut.sty}} 
       of \textsf{finstrut.sty} as of \UseDateOf{finstrut.sty}. 
       This upload is a birthday package for Donald Arseneau---he 
       will be so happy!}}
% \listfiles                                          %% 2010/12/22
{ \RequirePackage{makedoc}[2010/12/20] \ProcessLineMessage{} 
  \MakeJobDoc{18}{\SectionLevelTwoParseInput} 
}
\documentclass{article}%% TODO paper dimensions!?
\ProvidesFile{makedoc.cfg}[{2013/03/25 documentation settings}] 
%%
\author{Uwe L\"uck\thanks{%
        \url{http://contact-ednotes.sty.de.vu}}}
%%
%% 'hyperref':
\RequirePackage{ifpdf}
\usepackage[%
  \ifpdf
%     bookmarks=false,                  %% 2010/12/22
%     bookmarksnumbered,
    bookmarksopen,                      %% 2011/01/24!?
    bookmarksopenlevel=2,               %% 2011/01/23
%     pdfpagemode=UseNone,
%     pdfstartpage=10,
    pdfstartview=FitH,                  %% 2012/11/26 again
%     pdfstartview=0 0 100,             %% 2011/08/22
%     pdfstartview={XYZ null null 1},   %% 2011/08/25
%     pdfstartview={XYZ null null null},%% 2011/08/25
%     pdfstartview={XYZ null null .5},    %% 2011/08/26
%     pdffitwindow=true,          %% 2011/08/22
    citebordercolor={ .6 1    .6},
    filebordercolor={1    .6 1},
    linkbordercolor={1    .9  .7},
     urlbordercolor={ .7 1   1},   %% playing 2011/01/24
  \else
    draft
  \fi
]{hyperref}
\hypersetup{% 
    pdfauthor={Uwe L\374ck}% 
}
%% metadata, |\MDkeywords{<text>}|, |\MDkeywordsstring|:
%% %% 2011/08/22:
\makeatletter
  \newcommand*{\MDkeywords}[1]{%
    \gdef\MDkeywordsstring{#1}%
    \hypersetup{pdfkeywords=\MDkeywordsstring}%% TODO!?
  }
  \@onlypreamble\MDkeywords
%% |\MDaddtoabstract{<par-head>}|, `:' added:
  \newcommand*{\MDaddtoabstract}[1]{%           %% 2012/05/10
    \par\smallskip\noindent
    \strong{#1:}\quad\ignorespaces}
%% \pagebreak[2]
%% |\printMDkeywords|:
  \newcommand*{\printMDkeywords}{%
    \MDaddtoabstract{Keywords}%
    \MDkeywordsstring 
%     \global\let\MDkeywordsstring\relax    %% `%' 2012/11/12
  }
%% The previous definitions mainly are useful with a variant 
%% |\begin{MDabstract}| of \LaTeX's `{abstract}' environment:
  \newenvironment{MDabstract}
                 {\abstract\noindent
                  \hspace{1sp}%% for niceverb
                  \ignorespaces}
                 {\@ifundefined{MDkeywordsstring}%
                               {}%
                               {\printMDkeywords}%
                  \global\let\MDabstract\relax    %% 2012/11/12
                  \global\let\endMDabstract\relax %% 2012/11/12
                  \endabstract}
%% |\[MD]docnewline| 2012/11/12 from `readprov.tex':
  \newcommand*{\MDdocnewline}{\leavevmode\@normalcr[\topsep]}
%% <- `\leavevmode' for use with `\paragraph'.
%%    Sometimes needs to be preceded by a space.
%% 
%% |\MDfinaldatechecks[<tex-script>]| with \ctanpkgref{filedate}:
  \newcommand*{\MDfinaldatechecks}[1][fdatechk]{%
    \AtEndDocument{%
%       \clearpage %% 2013/03/25 no avail -- with `filedate'!
      \def\@pkgextension{sty}%
      \def\NeedsTeXFormat##1[##2]{}%
      \noNiceVerb                       %% 2013/03/22
      \input{#1}%
    }}
  \@onlypreamble\MDfinaldatechecks
\makeatother
%% Use other packages:
\RequirePackage{niceverb}[2011/01/24] 
\RequirePackage{readprov}               %% 2010/12/08
\RequirePackage{hypertoc}               %% 2011/01/23
\RequirePackage{texlinks}               %% 2011/01/24
\RequirePackage{relsize}                %% 2011/06/27
\RequirePackage{color}                  %% 2011/08/06
\RequirePackage{lmodern}                %% 2012/10/29
\RequirePackage{filedate}               %% 2012/11/12
\RequirePackage{filesdo}                %% 2013/03/22 
%% \pagebreak[3]
%% Logical markup:\qquad  |\strong{<chars>}|, |\meta{<chars>}|, 
%% |\acro{<chars>}|, |\pkg{<chars>}|, 
%% |\code{<chars>}|, |\file{<chars>}|:{\sloppy\par}
\makeatletter
  \def\do#1#2{\@ifdefinable#1{\let#1#2}}%% 2012/07/13
  \do\strong\textbf \do\file\texttt \do\acro\textsmaller 
  %% <- wrong tests before 2012/07/13
  \do\meta\textit   \do \pkg\textsf \do\code\texttt
  \ifpdf
    \pdfstringdefDisableCommands{%
        \let\acro\textrm 
        \let\file\textrm                            %% 2011/11/09
        \let\code\textrm                            %% 2011/11/20
        \let\pkg \textrm                            %% 2012/03/23
    }
  \fi
  %% TODO 2011/07/22 -> `htlogml.sty'
\makeatother
%% |\qtdcode{<text>}|: 2012/10/24:
    \newcommand*{\qtdcode}[1]{`\code{#1}'} 
%% |\pkgtitle{<package-name>}{<caption>}| 
\newcommand*{\pkgtitle}[2]{%            %% 2012/07/13
    \global\let\pkgtitle\relax
    \pkg{\huge #1}\\---\\#2\thanks{This 
       document describes version 
       \textcolor{blue}{\UseVersionOf{\jobname.sty}} 
       of \textsf{\jobname.sty} as of \UseDateOf{\jobname.sty}.}}
%% TODO: %% |\TODO| bad with `mdoccorr.cfg'
\newcommand*{\TODO}{\textcolor{blue}{\acro{TODO}}}  %% 2012/11/06
%% `\MDsampleinput[{<file>}' was added 2012/11/06. 
%% Problems with `myfilist.tex' were due to 'parskip.sty'
%% there. On 2012/11/12, we change the former simple macro to a 
%% much more complex
%% |\MDsamplecodeinput[<add-hfuss>]{<file>}| 
\newcommand*{\MDsamplecodeinput}[2][]{%
    \begingroup
        \vskip\bigskipamount \hrule
        \nobreak\vskip-\parskip 
%         \nobreak\vskip\medskipamount
%% Previous mistake (same below) due to manual change 
%% of `\topsep' in the file `myfilist.tex' (2012/11/30).
        \ifx\\#1\\\else
            \hfuzz=\textwidth \advance\hfuzz#1\relax
        \fi
        \noNiceVerb \verbatiminput{#2}%
%         \nobreak\vskip\medskipamount 
        \hrule \vskip-\parskip 
        \bigskip %%% \bigbreak
%% `\bigbreak' made much larger space in `myfilist.tex'.
    \endgroup
}
%% |\ctanpkgdref{<pkg-id>}| adds the printed link to 
%% `ctan.org/pkg' as a footnote. There is a little space 
%% for coloured link borders:
\newcommand*{\ctanpkgdref}[1]{%
    \ctanpkgref{#1}\,\urlfoot{CtanPkgRef}{#1}}
\errorcontextlines=4
\pagestyle{headings}

\endinput 
 %% shared formatting settings
\newcommand*{\lt}{<} \newcommand*{\gt}{>}           %% 2010/12/22
\providecommand*{\strong}{\textbf}                  %% 2010/12/15
\ReadPackageInfos{finstrut}
\usepackage{color}
\makeatletter \@beginparpenalty\@lowpenalty \makeatother
\sloppy
\begin{document}
% \author{Uwe L\"uck\thanks{\urlhttpref{http://contact-ednotes.sty.de.vu}} 
%         \and 
%         Donald Arseneau}
\maketitle
\begin{abstract}\noindent
\LaTeX\ internally inserts `\@finalstrut<strutbox>'
at the end of footnotes or `p' (paragraph) `tabular' cells.
When the user's entry ends on a `\par' token---which may 
be issued by some more-general-purpose user macro 
such as the `\lipsum' command from the \ctanpkgref{lipsum}
package\urlfoot{CtanPkgRef}{lipsum} for dummy text---this 
produces a usually unwanted extra, empty line.
'finstrut.sty' changes `\@finalstrut' in order to avoid 
this effect.
  \par
The first version of the package just had this minimum purpose. 
But then Donald Arseneau has convinced me to consider 
a wide range of cases and to replace my code by his proposals.
v0.5 again adds an idea of my own ones:     %% 2011/03/14:
instead of testing for `hmode' vs.~`vmode', 
`vmode' may be \emph{forced} by a `\par'---saves tokens!
In order not to break code that some package writers may have 
used so far, choices by package options are offered.
v0.4 offered choice among 6 definitions, with v0.5 they are 9---!
(Not 9!.)
  \par\smallskip\noindent
\strong{Keywords:}\quad footnotes; tables, dummy text, 
macro programming
\end{abstract}
\tableofcontents

%   \newpage 
\section{Remark on 'lipsum'}
Almost at the same moment that 'finstrut' v0.1 was installed 
(2011-02-09), a new version v1.1    %% v0.4: v1.1
of 'lipsum' was uploaded that offers a package option 
`[nopar]'                           %% v0.4
to avoid a final `\par'.

\section{Installing and Calling}
The package file 'finstrut.sty' is provided ready, 
installation only requires putting it somewhere where \TeX\ finds it 
(which may need updating the filename data 
 base).\urlfoot{ukfaqref}{inst-wlcf}

Below the `\documentclass' line(s) and above `\begin{document}', 
you load 'finstrut.sty' (as usually) by 
\[`\usepackage{finstrut}'\qquad \mbox{or by}\qquad 
  `\usepackage[<options>]{finstrut}'\]---<options> described below.

\section{The Package File}
\subsection{Header (Legalize)}
\ProvidesFile{finstrut.tex}[2011/03/14 documenting finstrut.sty (UL)]
\title{\textsf{finstrut.sty}\\---\\Final Strut Allowing Vertical 
       Mode---The \texttt{\cs{@finalstrut}} Encyclopedia and 
       Toolkit\thanks{This document describes version 
       \textcolor{blue}{\UseVersionOf{finstrut.sty}} 
       of \textsf{finstrut.sty} as of \UseDateOf{finstrut.sty}. 
       This upload is a birthday package for Donald Arseneau---he 
       will be so happy!}}
% \listfiles                                          %% 2010/12/22
{ \RequirePackage{makedoc}[2010/12/20] \ProcessLineMessage{} 
  \MakeJobDoc{18}{\SectionLevelTwoParseInput} 
}
\documentclass{article}%% TODO paper dimensions!?
\input{makedoc.cfg} %% shared formatting settings
\newcommand*{\lt}{<} \newcommand*{\gt}{>}           %% 2010/12/22
\providecommand*{\strong}{\textbf}                  %% 2010/12/15
\ReadPackageInfos{finstrut}
\usepackage{color}
\makeatletter \@beginparpenalty\@lowpenalty \makeatother
\sloppy
\begin{document}
% \author{Uwe L\"uck\thanks{\urlhttpref{http://contact-ednotes.sty.de.vu}} 
%         \and 
%         Donald Arseneau}
\maketitle
\begin{abstract}\noindent
\LaTeX\ internally inserts `\@finalstrut<strutbox>'
at the end of footnotes or `p' (paragraph) `tabular' cells.
When the user's entry ends on a `\par' token---which may 
be issued by some more-general-purpose user macro 
such as the `\lipsum' command from the \ctanpkgref{lipsum}
package\urlfoot{CtanPkgRef}{lipsum} for dummy text---this 
produces a usually unwanted extra, empty line.
'finstrut.sty' changes `\@finalstrut' in order to avoid 
this effect.
  \par
The first version of the package just had this minimum purpose. 
But then Donald Arseneau has convinced me to consider 
a wide range of cases and to replace my code by his proposals.
v0.5 again adds an idea of my own ones:     %% 2011/03/14:
instead of testing for `hmode' vs.~`vmode', 
`vmode' may be \emph{forced} by a `\par'---saves tokens!
In order not to break code that some package writers may have 
used so far, choices by package options are offered.
v0.4 offered choice among 6 definitions, with v0.5 they are 9---!
(Not 9!.)
  \par\smallskip\noindent
\strong{Keywords:}\quad footnotes; tables, dummy text, 
macro programming
\end{abstract}
\tableofcontents

%   \newpage 
\section{Remark on 'lipsum'}
Almost at the same moment that 'finstrut' v0.1 was installed 
(2011-02-09), a new version v1.1    %% v0.4: v1.1
of 'lipsum' was uploaded that offers a package option 
`[nopar]'                           %% v0.4
to avoid a final `\par'.

\section{Installing and Calling}
The package file 'finstrut.sty' is provided ready, 
installation only requires putting it somewhere where \TeX\ finds it 
(which may need updating the filename data 
 base).\urlfoot{ukfaqref}{inst-wlcf}

Below the `\documentclass' line(s) and above `\begin{document}', 
you load 'finstrut.sty' (as usually) by 
\[`\usepackage{finstrut}'\qquad \mbox{or by}\qquad 
  `\usepackage[<options>]{finstrut}'\]---<options> described below.

\section{The Package File}
\subsection{Header (Legalize)}
\input{finstrut.doc}
\end{document}

VERSION HISTORY

2011/02/09      cut out from `fnlineno.sty' [and corr. date+script]
2011/02/09c     remark on `lipsum' upload same day
2011/02/11      \subsection
2011/02/12      for v0.4
2011/02/16      for v0.5, abstract mentions Donald Arseneau
2011/03/14      "encyclopedia and toolkit", abstract extended

\end{document}

VERSION HISTORY

2011/02/09      cut out from `fnlineno.sty' [and corr. date+script]
2011/02/09c     remark on `lipsum' upload same day
2011/02/11      \subsection
2011/02/12      for v0.4
2011/02/16      for v0.5, abstract mentions Donald Arseneau
2011/03/14      "encyclopedia and toolkit", abstract extended

\end{document}

VERSION HISTORY

2011/02/09      cut out from `fnlineno.sty' [and corr. date+script]
2011/02/09c     remark on `lipsum' upload same day
2011/02/11      \subsection
2011/02/12      for v0.4
2011/02/16      for v0.5, abstract mentions Donald Arseneau
2011/03/14      "encyclopedia and toolkit", abstract extended

\end{document}

VERSION HISTORY

2011/02/09      cut out from `fnlineno.sty' [and corr. date+script]
2011/02/09c     remark on `lipsum' upload same day
2011/02/11      \subsection
2011/02/12      for v0.4
2011/02/16      for v0.5, abstract mentions Donald Arseneau
2011/03/14      "encyclopedia and toolkit", abstract extended
