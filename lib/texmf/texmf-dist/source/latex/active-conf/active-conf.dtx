% \iffalse meta-comment
%
% Copyright 2005 by Will Robertson <wspr81@gmail.com>
% 
% Distributable under the LaTeX Project Public License,
% version 1.3a or higher (your choice). The latest version of
% this license is at: http://www.latex-project.org/lppl.txt
%
% This work is "maintained" (as per LPPL maintenance status) 
% by Will Robertson.
%
% This work contains a verbatim copy of the topcapt package, 
% by Robin Fairbairns, which also has an LPPL license.
% 
% This work consists of the file  active-conf.dtx
%           and the derived files active-conf.cls and active-conf.pdf.
%
%<*driver>
\ProvidesFile{active-conf.dtx}
%</driver>
%<classfile>\NeedsTeXFormat{LaTeX2e}
%<!2006>\ProvidesClass{active-conf}
%<classfile&2006>\ProvidesClass{active-conf-2006}
%<*classfile>
  [2006/02/08 v0.3a Class file for the Active conference]
%</classfile>
%
%<*driver>
\documentclass{ltxdoc}
\makeatletter
\EnableCrossrefs
\CodelineIndex
\RecordChanges
\usepackage{amsmath,amssymb,array,calc,graphicx,hyperref,ifpdf,ifthen,varioref}
\usepackage[T1]{fontenc}
\usepackage{lmodern}
\usepackage[osf]{mathpazo}
\ifpdf\IfFileExists{microtype.sty}{\usepackage[protrusion=true,expansion=false]{microtype}}{}\fi
\linespread{1.069}      % Palatino needs more space between lines
\frenchspacing         % Remove ugly extra space after punctuation
\hypersetup{colorlinks, breaklinks, linktocpage,
  linkcolor=red, citecolor=red, filecolor=blue, urlcolor=blue}
\setlength\extrarowheight{2pt} % Make tables bearable
% Crude referencing commands:
\newcommand\tabref[1]{Table~\vref{tab:#1}}
\newcommand\figref[1]{Figure~\vref{fig:#1}}
\newcommand\secref[1]{Section~\vref{sec:#1}}
% Various labelling commands:
\newcommand*\name[1]{\textsc{#1}}
\newcommand*\pkg[2][]{\relax
  \edef\@tempa{#1}\relax
  \ifx\@tempa\@empty
    \textsf{#2}\else
    \href{#1}{\textsf{#2}}\fi}
\newcommand*\env[1]{\textsf{#1}}
\let\latin\textit
\def\eg{\latin{e.g.}}
\def\ie{\latin{i.e.}}
\def\etc{\@ifnextchar.{\latin{etc}}{\latin{etc.}\@}}
% Better float parameters: (from the TeX FAQ)
\renewcommand{\topfraction}{.85}
\renewcommand{\bottomfraction}{.7}
\renewcommand{\textfraction}{.15}
\renewcommand{\floatpagefraction}{.66}
\renewcommand{\dbltopfraction}{.66}
\renewcommand{\dblfloatpagefraction}{.66}
\setcounter{topnumber}{9}
\setcounter{bottomnumber}{9}
\setcounter{totalnumber}{20}
\setcounter{dbltopnumber}{9}
% Section heading customisation:
\renewcommand\section{\@startsection {section}{1}{\z@}%
                                   {-3ex \@plus -1ex \@minus -.2ex}%
                                   {2ex \@plus 0.2ex}%
                                   {\centering\normalsize\scshape}}
\renewcommand\subsection{\@startsection{subsection}{2}{\z@}%
                                     {-2.5ex\@plus -1ex \@minus -.2ex}%
                                     {1.5ex \@plus 0.2ex}%
                                     {\centering\normalsize\itshape}}
\renewcommand\subsubsection{\@startsection{subsubsection}{3}{\z@}%
                                     {-2ex\@plus -1ex \@minus -.2ex}%
                                     {1.5ex \@plus 0.2ex}%
                                     {\centering\normalfont\normalsize}}
\renewcommand\paragraph{\@startsection{paragraph}{4}{\z@}%
                                    {3.25ex \@plus1ex \@minus.2ex}%
                                    {-1em}%
                                    {\normalfont\normalsize\itshape}}
\let\subparagraph\undefined
\def\@maketitle{%
  \newpage
  {\centering
   {\large\@title\par}\vskip1em
   \textsc\@author\par\vskip1em
   \@date\par}\vskip2em}
% Abstract customisation:
\renewenvironment{abstract}{%
  \begin{trivlist}\item[]
    \setlength\leftskip{0.15\textwidth}
    \setlength\rightskip{0.15\textwidth}
    \small\textit{Abstract}\quad}{\end{trivlist}}                                      
% TOC customisation: Make it two-column to save space;
%   Remove leaders in the TOC, replace with \quad.
\setcounter{tocdepth}{2}
\renewcommand\tableofcontents{%
  \section*{\contentsname}
  \begin{trivlist}\item[]
    \begin{multicols}{2}
      \setlength\parskip{0pt}
      \small
      \@starttoc{toc}%
    \end{multicols}
  \end{trivlist}}
\renewcommand*\l@section[2]{%
  \ifnum \c@tocdepth >\z@
    \addpenalty\@secpenalty
    \addvspace{1.0em \@plus\p@}%
    \setlength\@tempdima{1.5em}%
    \begingroup
      \raggedright
      \parindent \z@ 
      \rightskip \z@
      \parfillskip \@flushglue
      \leavevmode
      \advance\leftskip\@tempdima
      \hskip -\leftskip
      #1\quad\nobreak#2\hfil\par
    \endgroup
  \fi}
\def\@dottedtocline#1#2#3#4#5{% 
  \ifnum #1>\c@tocdepth \else 
    {\leftskip #2\relax \rightskip \@tocrmarg \parfillskip \@flushglue
     \parindent #2\relax\@afterindenttrue 
     \interlinepenalty\@M 
     \leavevmode 
     \raggedright
     \@tempdima #3\relax 
     \advance\leftskip \@tempdima \null\nobreak\hskip -\leftskip 
     {#4}\quad\nobreak#5\hfil\par}% 
  \fi}
% Better footnotes:
\let\old@makefntext\@makefntext
\renewcommand\@makefntext[1]{%
  \vspace{2pt}%
  \setlength\parindent{-1.8em}%
  \setlength\leftskip{1.8em}%
  \makebox[1.8em][l]{\normalfont\small\@thefnmark.}#1}
% LOGOS, tuned for Palatino:
\def\TeX{T\kern-.15em\lower.5ex\hbox{E}\kern-.07em X\spacefactor1000\relax}
\begin{document}
  \GetFileInfo{active-conf.dtx}
  \DocInput{active-conf.dtx}
\end{document}
%</driver>
%
% \fi
%
% \DoNotIndex{\@tempa,\@tempb,\@temptokena}
% \DoNotIndex{\begin,\begingroup,\bgroup}
% \DoNotIndex{\def}
% \DoNotIndex{\edef,\egroup,\end,\endgroup,\else,\expandafter}
% \DoNotIndex{\fi}
% \DoNotIndex{\if,\ifnum,\let}
% \DoNotIndex{\relax,\RequirePackage}
% \DoNotIndex{\the,\then}
%
% \changes{v0.2}{2006/02/02}{Style changed to match \textsc{active} 2004. Lots of little changes for this.}
% \changes{v0.3}{2006/02/07}{Combined class implemented for 2006 alone.}
% \changes{v0.3a}{2006/02/08}{Added PDF metadata hooks.}
%
% \makeatletter
% \CheckSum{787}
% \title{The \pkg{active-conf} class}
% \author{Will Robertson}
% \date{\filedate \qquad \fileversion}
%
% \maketitle             ^^A% BEGIN!
%
% \begin{abstract}\noindent
% This document describes the \LaTeX\ class \pkg{active-conf}, which is intended for use initially to typeset papers that are to be published in the proceedings of the Active 2006 conference, which is to be hosted by the School of Mechanical Engineering, University of Adelaide. Its suitability for subsequent conferences is expected.
% \end{abstract}
%
% \tableofcontents
%
% \section{About this document}
%
% This document contains the usage and implementation of the \pkg{active-conf} class. 
% Users will almost certainly be interested in the former. This \textsc{pdf} contains
% hyperlinks within it to aid navigation (these are typeset in {\color{red}red}), and 
% hyperlinks to internet sites to help find further information (these are typeset in 
% {\color{blue}blue}).
%
% The source of this document, \texttt{active-conf.dtx}, when run through \LaTeX, 
% will produce both the PDF documentation (the file you are currently reading) 
% \emph{and} the class file used to typeset your articles.
%
% \section{Description and usage}
% This section describes how to use the class.
% Please refer to the example document for context.
%
% \subsection{Active 2006 class}
% The Active 2006 class, \textsf{active-conf-2006}, is a variation on the 
% \textsf{active-conf} class that incorporates the specific customisation
% for that year's conference. Future conferences may use the base class with
% customisation as described below; conversely, 2006 users need not take notice of
% the first paragraph of the following section.
%
% On confirmation of abstract acceptance, each paper is assigned a number.
% Input this in the document as follows: (for, say, the \texttt{088} paper number)
% \begin{verbatim}
%   \papernumber{088}
% \end{verbatim}
%
% \subsection{Document preamble}
% \DescribeMacro{\documentclass}
% Use this class with standard \LaTeX\ parlance: |\documentclass{active-conf}|.
% One class option is possible: the name or year of the conference. It is the responsibility
% of the conference organisers to distribute a file for the conference of their
% year corresponding to this option. To use an example, if the class option is 
% \texttt{[2006]} (\ie, |\documentclass[2006]{active-conf}|)
% the contents of the file \texttt{active-header-2006.tex} will be used to 
% construct a header on the first page of the article.\footnote{If \texttt{active-header-2006.tex}
% didn't exist, an error would occur when processing the document.}
%
% The document will be set up to use A4 paper with 3\,cm margins all 'round. 
% The body text font is 12\,pt/15.25\,pt Times.\relax
% \footnote{That is, the nominal text font size is 12\,pt, 
%  and the distance between each subsequent baseline of paragraph 
%  text is 15.25\,pt.} 
% The sans serif font is Helvetica\relax
% \footnote{Scaled to match Times' x-height 
%   (\ie, the lowercase letters are the same height in both alphabets)} 
% and the fixed width font (or typewriter font) is Courier.
%
% The following packages are loaded so that you don't have to: 
%  \pkg[http://tug.ctan.org/tex-archive/macros/latex/required/amslatex/math/]{amsmath}, 
%  \pkg[http://tug.ctan.org/tex-archive/macros/latex/required/amslatex/math/]{amssymb}, 
%  \pkg[http://www.ctan.org/tex-archive/macros/latex/required/tools/]{array}, 
%  \pkg[http://tug.ctan.org/tex-archive/macros/latex/required/tools/]{bm}, 
%  \pkg[http://tug.ctan.org/tex-archive/macros/latex/required/graphics/]{graphicx}.\relax
% \footnote{Look for file \texttt{amsldoc.pdf} for \pkg{amsmath} \& \pkg{amssymb}
%   documentation; file \texttt{grfguide.pdf} for \pkg{graphicx} documentation.} 
% Other packages are required for the class but they aren't of particular interest 
% for the purposes of the author; refer to the Implementation (\secref{packages}) 
% for more information. Extra packages may also be loaded if desired, provided that 
% they do not change the layout or text fonts used in the document.
%
% \subsection{Frontmatter metadata}
% \DescribeMacro{\title}
% \DescribeMacro{\author}
% \DescribeMacro{\email}
% \DescribeMacro{\address}
% \DescribeMacro{\note}
% This class provides added procedures to typeset extra information in the frontmatter 
% of the article. This information must be specified before \cmd\maketitle. \cmd\title\
% remains the same, but \cmd\author\ is changed and \cmd\email, \cmd\address, and \cmd\note\
% are all completely new. 
%
% When typeset, the information is divided into `address blocks', each of which may 
% contain multiple authors. True to its description, each block holds a single address 
% that is shared by each author contained in that block.
%
% \DescribeMacro{\maketitle}
% Once the metadata has been specified, the \cmd\maketitle\ command is used
% to create the title block containing this information. To be illustrative, 
% an example best demonstrates the use of the new frontmatter commands. 
% See \figref{fm} for a typical input and its typeset appearance 
% (albeit with different fonts). 
%
% \def\exinput{\noindent\textit{Input:}}
% \def\exoutput{\noindent\textit{Output:}}
% \setlength\fboxsep{1em}
% \begin{figure}
% \exinput
% \begin{verbatim}
% \author{A.\,B.~C------} \email{abc@university}
% \author{D.\,E.~F------}
% \address{University \\ City, State \\ Country}
%
% \author{G.\,H.~I------} \email{ghi@company}
% \address{Company \\ City, State \\ Country}
%
% \note{The authors would like to thank the ABC research grant.}
% \maketitle
% \end{verbatim}
% \vspace{-\baselineskip}
% \exoutput
% \begin{trivlist}\item[]
% \noindent\hspace{-\fboxsep}%
% \fbox{\vbox{\noindent
%   \begin{minipage}[t]{\textwidth}
%     \let\@makefntext\old@makefntext
%     \long\def\@makecaption#1#2{%
%       \vskip\abovecaptionskip
%       \sbox\@tempboxa{{\itshape #1}~--~#2}%
%       \ifdim \wd\@tempboxa >\hsize
%         {\itshape #1}~--~#2\par
%       \else
%         \global\@minipagefalse
%         \hb@xt@\hsize{\hfil\box\@tempboxa\hfil}%
%       \fi
%       \vskip\belowcaptionskip}
%
%     \centerline{\itshape Top of the page\dots}
%     \vspace{2ex}
%     \begin{minipage}[t]{140pt}%
%       \centering
%       A.\,B.~C------\rlap{\textsuperscript a},~\,D.\,E.~F------\par
%       {\linespread{0.9}\small University \\ City, State \\ Country\par}
%     \end{minipage}\hfill
%     \begin{minipage}[t]{140pt}%
%       \centering
%       G.\,H.~I------\rlap{\textsuperscript b}\par
%       {\linespread{0.9}\small Company \\ City, State \\ Country\par}
%     \end{minipage}\hfill\null
%
%     \vspace{2ex}
%     \centerline{\itshape\dots the rest of the page\dots}
%     \vspace{2ex}
%
%     {\def\@makefntext#1{\parindent 1em\noindent#1}
%      \footnotetext[0]{The authors would like to thank the ABC research grant.}
%      \footnotetext[0]{~}}
%     \def\thempfootnote{\@alph\c@mpfootnote}
%     \footnotetext[1]{Email address: \texttt{abc@university}}
%     \footnotetext[2]{Email address: \texttt{ghi@company}}
%   \end{minipage}}}
% \end{trivlist}
% \caption{Example of the frontmatter in the \pkg{active-conf} class.}
% \label{fig:fm}
% \end{figure}
%
% To summarise:
% \begin{description}\setlength\parskip{0pt}
% \item[\cmd\title]   Specified as usual.
% \item[\cmd\author]  Separates groups of authors by their respective \cmd\address.
% \item[\cmd\email]   Adds a footnote containing the email address 
%                       of the previous author.
% \item[\cmd\address] Creates a block containing authors previously specified.
% \item[\cmd\note]    Adds a note (used for thanks', etc.) 
%                       as an unlabelled footnote.\relax
% \footnote{As many notes as desired may be used; each is typeset on a separate line.}
% \end{description}
%
% \DescribeMacro{\showaffiliations}
% If the command \cmd\showaffiliations\ is placed before the author declarations,
% each author will reference the address to which they are associated.
% An optional argument to \cmd\author\ can be used to customise which addresses
% are referenced as affiliations; for example |\author[1,3]{A.\,N.~Author}| designates
% an affiliation for this author with the first and third addresses.
%
%
% \DescribeMacro{\pdfkeywords}
% Finally, optional, comma-separated keywords may be added to the paper with
% the command in the margin:
% \begin{verbatim}
%   \pdfkeywords{Active noise control, Virtual microphones}
% \end{verbatim}
%
% \subsection{Floats: figures and tables}
% \DescribeEnv{figure}
% \DescribeEnv{table}
% Several enhancements have been added to make things easier when using figures and 
% tables. Both are centred on the page, so no explicit commands for doing so are 
% required. Secondly, the default float placement parameter is \texttt{[htbp]},\relax
%\footnote{That is, floats will be placed at the position of their definition 
%   if possible; otherwise they will be placed at the top or bottom of a 
%   subsequent text page or on a page consisting of only floats in the last resort.} 
% so the optional argument generally won't be required for either figures or tables.
%
% \DescribeMacro{\caption}
% The \cmd\caption\ command used to label figures and tables is now contextually 
% sensitive to adjust its vertical spacing depending on which float it is being
% used within. This ensures appropriate spacing for the following rules: 
% captions must be placed \emph{below} figures and \emph{above} tables.
%
% A typical example of the \env{figure} environment follows:
% \begin{verbatim}
% \begin{figure}
%   \includegraphics{figurename}
%   \caption{Figure caption.}
%   \label{fig:label}
% \end{figure}
% \end{verbatim}
% A typeset example of how the \env{table} environment should be used is shown in 
% \figref{tab}. When displaying tabulated data, avoid vertical lines and use horizontal rules 
% only to divide data from its label or to separate multiple sets of data. 
% The more lines used in the table, the more distracting it is from the data being displayed.
% See the \href{http://tug.ctan.org/tex-archive/macros/latex/contrib/booktabs/}{booktabs} 
% package documentation for more information on effective table design.
% \begin{figure}
% \exinput
% \begin{verbatim}
% \begin{table}
%   \caption{This is an example of how you should tabulate data.}
%   \label{tab:label}
%   \begin{tabular}{@{} cccc @{}}
%     \hrule
%     One & Two & Three & Four \\
%     \hrule
%     1 & 4 & 7 & 10 \\ 2 & 5 & 8 & 11 \\ 3 & 6 & 9 & 12 \\
%     \hrule
%   \end{tabular}
% \end{table}
% \end{verbatim}
% \vspace{-\baselineskip}
% \exoutput
% \begin{trivlist}\item[]
% \noindent\hspace{-\fboxsep}%
% \fbox{\vbox{\noindent
%   \begin{minipage}[t]{\textwidth}\centering
%   \def\@captype{table}
%   \long\def\@makecaption#1#2{%
%     \vskip\abovecaptionskip
%     \sbox\@tempboxa{\small #1:~#2}%
%     \ifdim \wd\@tempboxa >\hsize
%       {\small #1:~#2\par}%
%     \else
%       \global\@minipagefalse
%       \hb@xt@\hsize{\hfil\box\@tempboxa\hfil}%
%     \fi
%     \vskip\belowcaptionskip}
%
%   \caption{This is an example of how you should tabulate data.}\vspace{1ex}
%   \label{tab:label}
%   \begin{tabular}{@{} cccc @{}}
%     \hline
%     One & Two & Three & Four \\
%     \hline
%     1 & 4 & 7 & 10 \\ 2 & 5 & 8 & 11 \\ 3 & 6 & 9 & 12 \\
%     \hline
%   \end{tabular}
% \end{minipage}}}
% \end{trivlist}
% \caption{Example of tabular data.}
% \label{fig:tab}
% \end{figure}
%
% \newpage
% \subsection{Referencing}
% \DescribeMacro{\figref}
% \DescribeMacro{\tabref}
% \DescribeMacro{\eqref}
% \DescribeMacro{\secref}
% The class defines the following four commands for referring to document elements: 
% \cmd\figref, \cmd\tabref, \cmd\eqref, and \cmd\secref. They are used in a similar way
% to the \cmd\ref\ command, but print the name of the reference type before the 
% numbered label. For example,
% \begin{quote}
% |\figref{label}| $\quad\longrightarrow\quad$ |Figure~\ref{fig:label}|
% \end{quote}
% and similarly for the others (respectively, `Table', `Equation', and `Section').
%
% Should you wish to use a more sophisticated referencing package 
% (the author's recommendation is 
%  \pkg[http://tug.ctan.org/tex-archive/macros/latex/contrib/refstyle/]{refstyle}),
% these control sequences may need to be undefined before the package is loaded to 
% prevent conflicts. This may done with, \eg, \cmd\let\cmd\figref\cmd\undefined, and so on.
%
% \DescribeMacro{\cite}
% The \cmd\cite\ command is redefined to remove any preceding space and replace it with non-breaking space; in other words, its output will never follow a line break. If you desire more advanced behaviour with this command, use the following in your preamble:
% \begin{verbatim}
%   \usepackage[sort&compress,numbers]{natbib}
%   \usepackage{hypernat}
% \end{verbatim}
% The \pkg[http://www.tug.org/tex-archive/macros/latex/contrib/misc/hypernat.sty]{hypernat} package is not necessarily standard, so you may need to obtain it manually.
%
% \StopEventually{}
%
% \clearpage
% \section{Implementation}
%
% This section contains the commented source code of this package. It is not
% intended to be useful or interesting to the majority of users of the class.
%
% This class has been written somewhat hastily to produce an easy-to-use 
% \LaTeX\ style file for authors writing articles for the conference Active 2006. 
% As such, it is not up to the standard of a `professional' class file although it 
% might contain some (not particularly and not many) novel ideas; there are probably 
% be lots of limitations/bugs since it has not been extensively tested. 
% Use at your own risk!
%
%
%
% \subsection{Option processing}\label{sec:classoption}
% Take one option for the class: the name or year of the conference.
% The file \texttt{YYYY.tex} is then used to construct the titlepage header.
% We use \cmd\active@option@warn\ to give an error if the user tries to use 
% more than one class option.
%<*!2006>
%    \begin{macrocode}
\let\active@option@warn\relax
\DeclareOption*{%
  \active@option@warn
  \def\active@option@warn{%
    \ClassError{active-conf}
      {The only class option possible is the name or year of the conference}
      {\MessageBreak 
       This class takes one option: the name or year of the conference.\MessageBreak
       The file active-header-YYYY.tex is then used to create the titlepage header.\MessageBreak}}
  \let\active@option\CurrentOption}
\ProcessOptions
%    \end{macrocode}
%</!2006>
%<*classfile>
% \subsection{Class and package loading}\label{sec:packages}
% Base everything off the eponymous \pkg{article} class. Set up the fonts\relax
% \footnote{Note that we need to call \cmd\normalfont\ after selecting the new fonts
%   and before selecting the new encoding in order to ensure that \texttt{T1} `CM' 
%   fonts aren't loaded, which can cause an error in some distributions.}, 
% and load a bunch of packages first to set up the document properties and second 
% for the convenience of the user.
%    \begin{macrocode}
\LoadClass[12pt]{article}
\renewcommand\rmdefault{ptm}
\renewcommand\ttdefault{pcr}
\RequirePackage[scaled=0.87]{helvet}
\normalfont
\RequirePackage[T1]{fontenc}
\RequirePackage{textcomp}
\RequirePackage{calc,ifthen,url}
\RequirePackage[a4paper,margin=3cm,ignoreall]{geometry}
\RequirePackage{amsmath,amssymb,array,bm,graphicx,hyperref}
\hypersetup{
  colorlinks,
  linkcolor=black,
  anchorcolor=black,
  citecolor=black,
  filecolor=black,
  menucolor=black,
  pagecolor=black,
  urlcolor=black,
  bookmarks,
  pdfstartview=FitH,
  pdfpagelayout=SinglePage
  }
%    \end{macrocode}
% \paragraph{Fonts}
% Because we need to be able to have Word users copy us, we need to use fonts that they can use too. Palatino is no good because Windows~2000 users will not have it (and I can't be bothered researching to find out if Book~Antiqua or whatever will do the job sufficiently); everyone has Georgia, but that's tricky to get maths going with it and unusual to find in a LaTeX installation. So we're stuck with the age-old and none too pleasant Times. Ah well. 
%
% Note that I \emph{have not} taken the liberty of using the Times maths fonts\footnote{Either via the \pkg{mathptm} or \pkg{mathptmx} packages.} as well, since Computer Modern maths does the job quite nicely (and moreover, contains bold Greek symbols~--~how can there be no bold maths in the Times maths fonts?).
%
% Finally, get rid of extra space after punctuation (it's old-fashioned) and increase the leading between the lines; we need this due to such long lines with so many characters in each. We also want no page numbers, since numbers will be added after all the papers are collated into the proceedings.
%    \begin{macrocode}
\frenchspacing
\linespread{1.0517}
\pagestyle{empty}
\setlength\parindent{1.5em}
%    \end{macrocode}
% \subsection{Formatting specification}
% \begin{macro}{\section}\begin{macro}{\subsection}
% \begin{macro}{\subsubsection}\begin{macro}{\paragraph}
% \begin{macro}{\subparagraph}
% Not too large a font for the titles, make it more unobstrusive.
% Nobody ever uses \cmd\subparagraph, so let's remove it.
%    \begin{macrocode}
\setcounter{secnumdepth}{3}
\renewcommand\section{\@startsection{section}{1}{\z@}%
                                   {-\baselineskip}%
                                   {1pt plus 1pt}%
                                   {\sffamily\bfseries\MakeUppercase}}
\renewcommand\subsection{\@startsection{subsection}{2}{\z@}%
                                     {-\baselineskip}%
                                     {1pt plus 1pt}%
                                     {\sffamily\bfseries}}
\renewcommand\subsubsection{\@startsection{subsubsection}{3}{\z@}%
                                    {-\baselineskip}%
                                    {1pt plus 1pt}%
                                    {\sffamily\bfseries}}
\renewcommand\paragraph{\@startsection{paragraph}{4}{\z@}%
                                    {\baselineskip}%
                                    {-\baselineskip}%
                                    {\sffamily\bfseries}}
\let\subparagraph\undefined
%    \end{macrocode}
% \end{macro}\end{macro}\end{macro}\end{macro}\end{macro}
%
% \begin{environment}{itemize}
% \begin{environment}{enumerate}
% Decrease the amount of vertical space between items in the \env{itemize} and \env{enumerate} environments. Renew the \LaTeX-defined ones in order to adjust all necessary bits and pieces.
%    \begin{macrocode}
\def\list@spacing{%
  \parsep    4pt
  \itemsep   0pt
  \topsep    6pt
  \partopsep 0pt}
\def\enumerate{% 
  \ifnum \@enumdepth > \thr@@\@toodeep\else 
    \advance\@enumdepth\@ne 
    \edef\@enumctr{enum\romannumeral\the\@enumdepth}% 
      \expandafter 
      \list 
        \csname label\@enumctr\endcsname 
        {\usecounter\@enumctr\def\makelabel##1{\hss\llap{##1}}%
         \list@spacing}% 
  \fi} 
\let\endenumerate\endlist  
\def\itemize{% 
  \ifnum \@itemdepth > \thr@@\@toodeep\else 
    \advance\@itemdepth\@ne 
    \edef\@itemitem{labelitem\romannumeral\the\@itemdepth}% 
    \expandafter 
    \list 
      \csname\@itemitem\endcsname 
      {\def\makelabel##1{\hss\llap{##1}}%
       \list@spacing}% 
  \fi} 
\let\enditemize\endlist
%    \end{macrocode}
% \end{environment}
% \end{environment}
% \begin{environment}{itemise}
% Provide an environment with the correct spelling of `itemize'.
%    \begin{macrocode}
\let\itemise\itemize     
\let\enditemise\enditemize   
%    \end{macrocode}
% \end{environment}
%
% \begin{macro}{\descriptionlabel}
% Change the description label to italics instead of bold.
%    \begin{macrocode}
\renewcommand*\descriptionlabel[1]{\hspace\labelsep
                                \normalfont\itshape #1}
%    \end{macrocode}
% \end{macro}
%
% \begin{environment}{abstract}
% The most important part is removing the indent that exists in \pkg{article}!
%    \begin{macrocode}
\def\abstract@error{%
  \ClassError{active-conf}{Not allowed in the abstract.}
    {The abstract must not contain maths, references, or footnotes.}}
\renewenvironment{abstract}
  {\vspace{\baselineskip}\fontsize{10}{11}\selectfont
   {\centering\sffamily\bfseries ABSTRACT\par}
   \setlength\leftskip{\leftskip+1.5cm}%
   \setlength\rightskip{\rightskip+1.5cm}%
   \noindent\ignorespaces}
  {\par}
%    \end{macrocode}
% \end{environment}
% \begin{macro}{\figref}
% \begin{macro}{\tabref}
% \begin{macro}{\secref}
% These are hard-coded because I'm lazy and there's no good solution to this problem without telling people to use \textsf{refstyle}. Which is not widely distributed (as it's new) so users would have to install it themselves. Which is bad.
%    \begin{macrocode}
\def\eqref#1{Equation~\ref{eq:#1}}
\def\figref#1{Figure~\ref{fig:#1}}
\def\tabref#1{Table~\ref{tab:#1}}
\def\secref#1{Section~\ref{sec:#1}}
%    \end{macrocode}
% \end{macro}\end{macro}\end{macro}
% \begin{macro}{\fps@figure}
% \begin{macro}{\fps@table}
% Make the default float placement \texttt{[htbp]}; users will always do it themselves anyway\dots
%    \begin{macrocode}
\def\fps@figure{htbp}
\def\fps@table{htbp}
%    \end{macrocode}
% \end{macro}
% \end{macro}
% \begin{environment}{figure}
% \begin{environment}{table}
% Add \cmd\centering\ to the \env{figure} and \env{table} environments.
% This requires a trick: \cmd{\fps@...} must be expanded, so we can't just
% pass through |#1|. Instead, put everything, expanding all except the 
% \cmd\@float\ command, in a temporary macro, and then use that to produce the float.
%    \begin{macrocode}
\renewenvironment{figure}[1][\fps@figure]
                 {\edef\@tempa{\noexpand\@float{figure}[#1]}
                  \@tempa\centering}
                 {\end@float}
\renewenvironment{table}[1][\fps@table]
                 {\edef\@tempa{\noexpand\@float{table}[#1]}
                  \@tempa\centering}
                 {\end@float}
%    \end{macrocode}
% \end{environment}\end{environment}
% With the \pkg{array} package, add more height to the table rows so that horizontal
% rules don't look ugly. But only if the \pkg{booktabs} package isn't loaded, since it
% performs similar operations itself.
%    \begin{macrocode}
\AtBeginDocument{%
  \@ifpackageloaded{booktabs}{}{\setlength\extrarowheight{2pt}}}
%    \end{macrocode}
%    \begin{macrocode}
% Better float parameters: (from the TeX FAQ)
\renewcommand{\topfraction}{.85}
\renewcommand{\bottomfraction}{.7}
\renewcommand{\textfraction}{.15}
\renewcommand{\floatpagefraction}{.66}
\renewcommand{\dbltopfraction}{.66}
\renewcommand{\dblfloatpagefraction}{.66}
\setcounter{topnumber}{9}
\setcounter{bottomnumber}{9}
\setcounter{totalnumber}{20}
\setcounter{dbltopnumber}{9}
%    \end{macrocode}
%
% \begin{macro}{\@makecaption}
% Adjust the formatting of float captions: smaller font size.
%    \begin{macrocode}
\long\def\@makecaption#1#2{%
  \vskip\abovecaptionskip
  \sbox\@tempboxa{\small #1:~#2}%
  \ifdim \wd\@tempboxa > \hsize
    \begingroup
      \setlength\leftskip{\leftskip+1.5cm}%
      \setlength\rightskip{\rightskip+1.5cm}%
      \small #1:~#2\par\endgroup
  \else
    \global\@minipagefalse
    \hb@xt@\hsize{\hfil\box\@tempboxa\hfil}%
  \fi
  \vskip\belowcaptionskip}
%    \end{macrocode}
% \end{macro}
%
% We wish to describe tables with captions \emph{above} the table. 
% This is not an uncommon style requirement. 
%
% We can't simply use \LaTeX's \cmd\caption\ and put it before a |\begin{tabular}| since the spacing around the caption is defined assuming that it'll be placed below whatever it's describing. So, we use \name{Robin Fairbairn}'s \pkg{topcapt} package (or rather, a plagiarised verbatim copy of it~--~v1.1a to be precise) to define a new command \cmd\topcaption\ (and an auxiliary macro \cmd\@topcaption) that has `opposite' spacing to a regular \cmd\caption\ for the purposes of using it with tabular material.
%
% \begin{macro}{\topcaption}
% Quoth Robin: {\slshape \cmd\topcaption\ simply advances the caption count, sorts out the
% arguments and passes on to \cmd{\@topcaption}.}
%
%    \begin{macrocode}
\newcommand\topcaption{%
  \refstepcounter\@captype
  \@dblarg{\@topcaption\@captype}} 
%    \end{macrocode}
% \end{macro}
% \begin{macro}{\@topcaption}
% Robin: {\slshape \cmd{\@topcaption} swaps above and below caption skips and the relays its arguments to \cmd{\@caption}.}
%    \begin{macrocode}
\def\@topcaption#1[#2]#3{%
  \@tempskipa\abovecaptionskip
  \abovecaptionskip\belowcaptionskip
  \belowcaptionskip\@tempskipa
  \@caption{#1}[{#2}]{#3}%
  \@tempskipa\abovecaptionskip
  \abovecaptionskip\belowcaptionskip
  \belowcaptionskip\@tempskipa}
%    \end{macrocode}
% \end{macro}
%
% Now we want to use \cmd\topcaption\ by default when we're inside a \env{table} environment. We can do this by examining the control sequence \cs{@captype}, which contains the string of the floating environment in which it is called. By wrapping \cmd\caption\ in a macro that normally just calls the original \cmd\caption\ definition unless \cs{@captype} contains \texttt{table}, in which case \cmd\topcaption\ is used, we can make this process transparent to the user.
%
% Of course, the author needs to know in what context to use \cmd\caption, but that's a problem to be tackled another day.
%    \begin{macrocode}
\let\caption@orig\caption
\renewcommand\caption{%
  \let\@tempa\caption@orig
  \def\@tempb{table}%
  \ifx\@captype\@tempb
    \let\@tempa\topcaption
  \fi
  \@tempa}
%    \end{macrocode}
%
% \begin{macro}{\appendix}
% Use this command and all \cmd\section\ commands now print appendices.
%    \begin{macrocode}
\def\appendix{\par
  \c@section\z@\c@subsection\z@
  \gdef\thesection{Appendix \@Alph\c@section}}
%    \end{macrocode}
% \changes{v0.1a}{2006/01/30}{Added `Appendix' before the section number in an Appendix.}
% \end{macro}
%
% \subsection{Frontmatter}
% \begin{macro}{\maketitle}
% This is changed somewhat from the default classes. Instead of making \cmd\footnote\ create \cmd\thanks's, it is defined to gobble its argument and issue a warning. I hope nobody wants a footnote in their frontmatter. They shouldn't.
%
% The reason for doing this is because the mechanism that \cmd\author\ and \cmd\email\ uses to create email address footnotes is rather fragile because I hacked it together rather quickly and don't take other notes into account when they're counting footnote numbers.
%    \begin{macrocode}
\renewcommand\maketitle{%
  \begingroup
    \def\footnote##1{\ClassWarning{active-conf}
      {Footnotes aren't allowed in address text blocks}}
    \def\thefootnote{\@alph\c@footnote}%
    \newpage
    \global\@topnum\z@   % Prevents figures from going at top of page.
    \centering
%    \end{macrocode}
% Put the conference details above the title. (Only if \cmd\active@option\ has been
% defined by the one and only class option, obviously. See \secref{classoption}.)
%    \begin{macrocode}
%<!2006>    \@ifundefined{active@option}{}
      {\vspace*{-24pt}% hard coded because I'm lazy
       \begin{minipage}{\textwidth}
         \sffamily\small
%    \end{macrocode}
%</classfile>
%<*2006>
%    \begin{macrocode}
\vspace*{-2cm}
\centering
\includegraphics{header-logo-2006}
%    \end{macrocode}
%</2006>
%<*!2006>
%    \begin{macrocode}
         \IfFileExists{active-header-\active@option.tex}
           {\input active-header-\active@option.tex}
           {\ClassError{active-conf}
              {Titlepage header file active-header-\active@option.tex not found}
              {\MessageBreak 
               The file active-header-YYYY.tex is required to produce the titlepage header,\MessageBreak
               where YYYY is the name or year of the conference. \MessageBreak
               Please contact the conference organisers for further information.
               \MessageBreak}}
%    \end{macrocode}
%</!2006>
%<*classfile>
%    \begin{macrocode}
       \end{minipage}\vspace{1ex}}
    \begin{center}
      \sffamily\bfseries\fontsize{16}{17}\selectfont\@title
    \end{center}
    \vspace{-2.5ex}%
%    \end{macrocode}
% \paragraph{Notes} These are input by the author with \cmd\note, multiple allowed, and typeset as footnotes without \cmd\footnotemark s. After the last, put some blank space to separate from the other footnotes and redefine \cmd\note\ to gobble its input and produce a warning. Afterwards, redefine footnote formatting to behave as normal.
%    \begin{macrocode}
    \begingroup
      \long\def\@makefntext##1{\parindent 1em\noindent##1}%
      \@note
      \setcounter{footnote}{0}
      \global\def\note##1{%
        \ClassWarning{active-conf}
          {\protect\note\space is only allowed before \protect\maketitle}
          {Please input all notes before the document begins.}}
    \endgroup
%    \end{macrocode}
% \paragraph{Addresses} These are author/address blocks inside the \cmd\address@list\ macro that at present exist inside boxes all of equal width. 
%
% First, it is necessary to ensure that any footnote marks for referring to the email address of an author doesn't increase the vertical height of the box for that address block, since then the baselines of adjacent blocks will no longer be aligned.
%
% The boxes that contain the address blocks are typset with infinite stretchable space\footnotemark\ on either side of each line (the \cmd\leftskip\ and the \cmd\rightskip). There is more infinite stretchable space in between each address block (see the end of the definition of \cmd\address), which serves to equispace each block horizontally.
% \footnotetext{\cs{@flushglue} defined by \LaTeX\ as \texttt{0pt plus 1fil}}
%    \begin{macrocode}
    \def\@makefnmark{\smash{\textsuperscript{\@thefnmark}}}
    {\parindent\z@ 
     \leftskip\@flushglue
     \rightskip\@flushglue
     \parfillskip\z@
     \address@list\par}
    \after@maketitle
  \endgroup
  \hypersetup{pdfauthor={\pdf@authors},pdftitle={\@title}}}
\let\after@maketitle\@empty  
%    \end{macrocode}
% \end{macro}
%
%  \begin{macro}{\pdfkeywords}
%    \begin{macrocode}
\newcommand\pdfkeywords[1]{\hypersetup{pdfkeywords={#1}}}
%    \end{macrocode}
%  \end{macro}
%
%  \begin{macro}{\papernumber}
%<*2006>
%    \begin{macrocode}
\newcommand\papernumber[1]{%
  \hypersetup{pdfsubject={a06\_#1.pdf}}%
  \AtEndDocument{%
    \typeout{ ^^J
             *************************************************^^J
             *************************************************^^J
             **^^J
             ** Please submit your manuscript as "a06_#1.pdf"^^J
             **^^J
             *************************************************^^J
             *************************************************}}}
%    \end{macrocode}
%</2006>
%  \end{macro}
%
% 
% \begin{macro}{\note}
% This command is used to typeset notes about the article at the footer of the opening page. Intended to be used for thanks, acknowledgements, and various miscellany about the article or authors that doesn't fit anywhere else.
%    \begin{macrocode}
\def\note#1{\g@addto@macro\@note{\footnotetext[0]{#1}}} 
\let\@note\@empty
\let\thanks\note
%    \end{macrocode}
% \end{macro}
%
% \begin{macro}{\author@init}
% \begin{macro}{\author@list}
% \begin{macro}{\author}
% \cmd{\author@init} is the top-level macro that creates a `fresh' definition of \cmd\author\ and initialises the \cmd{\author@list} macro. \cmd\author\ simply populates \cmd{\author@list} with a list of authors, separated by the macro \cmd{\author@sep}. The definition is set up to redefine itself the first time it is called so that \cmd{\author@sep} is only inserted \emph{after} this first time.
%
% \cmd{\author@list} is used as the first line in every address block, so once \cmd\address\ is called, \cmd{\author@init} is called again for the next list of authors that happen to work at a different address.
%    \begin{macrocode}
\def\author@init{%
  \def\@@author##1{%
    \g@addto@macro\author@list{##1}%
    \def\@@author####1{\g@addto@macro\author@list{\author@sep ####1}}}%
  \let\author@list\@empty}
\author@init
\renewcommand\author[2][\c@affiliation]{%
  \ifx\pdf@authors\@empty\else
    \g@addto@macro\pdf@authors{, }%
  \fi
  \g@addto@macro\pdf@authors{#2}%
  \g@addto@macro\author@list{\mbox\bgroup}%
  \@@author{#2}%
  \if@showaff
    \@for\@@index :=#1\do{%
      \expandafter\g@addto@macro
      \expandafter\author@list
      \expandafter{%
      \expandafter\place@affiliation
      \expandafter{%
                  \@@index}}%
    }
  \fi
  \g@addto@macro\author@list{\egroup}}
\newcommand\place@affiliation[1]{\kern1pt\textsuperscript{\@fnsymbol#1}}  
\let\pdf@authors\@empty
%    \end{macrocode}
% \end{macro}
% \end{macro}
% \end{macro}
%
% \begin{macro}{\email}
% This macro is intended to be used immediately after an \cmd\author\ declaration, and it simply appends a footnote to the current author detailing their email address. Because we aren't evaluating these things until the end, we regrettably need to spend some effect to replicate the effect that \cmd\footnotemark\ has on \cs{c@footnote}. This could almost certainly be more elegant.
%    \begin{macrocode}
\def\email#1{%
  \g@addto@macro\author@list{\kern1pt\footnotemark}%
  \g@addto@macro\after@maketitle{%
    \stepcounter{footnote}%
    \footnotetext[\the\c@footnote]{Email address: \url{#1}}}}
\g@addto@macro\after@maketitle{\setcounter{footnote}{0}}
%    \end{macrocode}
% \end{macro}
%
% \begin{macro}{\address@list}
% This is the macro used to hold all of the address blocks. Some of its contents is \emph{unexpanded} until \cmd\maketitle, notably the width of the minipages used to typeset the blocks.
%    \begin{macrocode}
\let\address@list\@empty
%    \end{macrocode}
% \end{macro}
% \begin{macro}{\@@authorhook}
% \begin{macro}{\@@addresshook}
% And these are the macros used to format the text in the address blocks. Its default definition is to typeset everything in sans serif and the address in a small font, but you can do more complex things with it if you wish; it's enclosed in a group so don't worry about having to confine state. Unfortunately, it's \emph{not} set up to take an argument, state-changing arguments must be used (\eg, \cmd\sffamily, \cmd\itshape, \cmd\small).
%    \begin{macrocode}
\providecommand\@@authorhook{\sffamily}
\providecommand\@@addresshook{\sffamily\fontsize{10}{11}\selectfont}
%    \end{macrocode}
% \end{macro} \end{macro}
% \begin{macro}{\address}
% This macro is used after any number of \cmd\author\ declarations. It takes the list of authors and typesets them in a box above the specified address. Everything is measured and later put into boxes of equal width so that spacing with several address blocks looks okay.
%
% The trick is to use one of \TeX's vertical boxes, and populate it with restricted-mode horizontal boxes---this makes the \cmd\vbox\ behave ``as expected'' and stretch to exactly the width it requires to typeset everything. The downside to this method is that restricted-mode \cmd\hbox's are required. What does this mean? `Normal' things like paragraph breaks and literal newlines aren't allowed, since we're typesetting in one long horizontal box.
%
% Obviously, people will want to write multi-line addresses, so we can get around the horiz.\ box problem by defining \cmd{\\} to end the current \cmd\hbox\ and start another. The following verbatim sketches the idea\dots
% \begin{verbatim}
% \address{abc \\ def \\ ghi} => \vbox{ ... \hbox{abc \\ def \\ ghi} }
%                          \\ => }\hbox{
%    \hbox{abc \\ def \\ ghi} => \hbox{abc }\hbox{ def }\hbox{ ghi}
% \end{verbatim}
% This leaves out the details, like absorbing the leading space we don't want, and re-applying the address-block formatting hook.
% Finally, at the end of the address, we need to initialise the various author macros so that a fresh lot of authors can be defined for their own subsequent address block.
%    \begin{macrocode}
\def\address#1{%
  \begingroup
    \let\footnotemark\relax
    \def\\{\egroup\hbox\bgroup\@@addresshook\ignorespaces}
    \sbox\tempbox{%
      \vbox{%
        \hbox{\strut\@@authorhook\author@list}
        \hbox{\@@addresshook #1}}}
    \settowidth\templength{\usebox\tempbox}
    \ifthenelse{\lengthtest{\templength>0.49\linewidth}}{\global\boxwidth\linewidth}{%
      \ifthenelse{\lengthtest{\templength>\boxwidth}}{\global\boxwidth\templength}{}}%
    \expandafter\make@address@box\expandafter{\author@list}{#1}
  \endgroup
  \author@init}
\newlength\boxwidth
\newlength\templength
\newbox\tempbox
%    \end{macrocode}
% \end{macro}
% \begin{macro}{\make@address@box}
% This macro is broken out for easy of supplying an expanded \cmd{\author@list} to the middle of a \cmd{\g@addto@macro} declaration. Note well that \cmd\boxwidth\ isn't evaluated until \cmd{\address@list} is expanded in \cmd\maketitle.
%    \begin{macrocode}
\newcommand\make@address@box[2]{%
  \g@addto@macro\address@list{%
    \begin{minipage}[t]{\boxwidth+10pt}%
      \vspace*{\medskipamount}%
      \centering
      {\@@authorhook#1\par\vspace{3pt}}
      {\linespread{0.9}%
       \@@addresshook
       \if@showaff
         \makebox[0pt][r]{\textsuperscript{\@fnsymbol\c@affiliation}}%
       \fi
       \ignorespaces#2\par}
    \end{minipage}%
    \stepcounter{affiliation}%
    \hskip\@flushglue}}
%    \end{macrocode}
% \changes{v0.1a}{2006/01/30}{Vertical space fiddling in the frontmatter.}
% \end{macro}
%
%    \begin{macrocode}
\newcounter{affiliation}
\stepcounter{affiliation}
\g@addto@macro\after@maketitle{\setcounter{affiliation}{1}}
\newif\if@showaff   
\newcommand\showaffiliations{\@showafftrue} 
%    \end{macrocode}
%
% Don't look at the following definition! Yuck!
%    \begin{macrocode}
\def\author@sep{,~\,}
%    \end{macrocode}
% \subsection{Maths and misc.}
% \begin{macro}{\unit}
% \begin{macro}{\num}
% The \cmd\unit\ and \cmd\num\ commands are defined for typesetting units
% and numbers with units, respectively. In the latter case, a thin space is
% used between the number and its unit, which is good typographical practise.
%    \begin{macrocode}
\newcommand\unit[1]{\ensuremath{\mathrm{#1}}}
\newcommand\num[2]{\ensuremath{\text{#1}\,\mathrm{#2}}}
%    \end{macrocode}
% \end{macro}
% \end{macro}
% \begin{macro}{\cite}
% Removes preceding space and replaces it with a non-breaking space.
%    \begin{macrocode}
\AtBeginDocument{%
  \let\cite@orig\cite
  \def\cite{\ifnum\lastskip>0\relax\unskip~\fi\cite@orig}}
%    \end{macrocode}
% \changes{v0.1a}{2006/01/30}{Implemented non-breaking behaviour.}
% \end{macro}
%
% The following is taken from my very own \pkg{fontspec} package, and is
% used to change \cmd\mathrm\ to Times Roman without destroying those aspects
% of default Computer Modern maths that assume that \cmd\rmdefault\ is \texttt{cmr}.
%    \begin{macrocode}
\DeclareSymbolFont{legacymaths}{OT1}{cmr}{m}{n}
\SetSymbolFont{legacymaths}{bold}{OT1}{cmr}{bx}{n}
\DeclareMathAccent{\acute}   {\mathalpha}{legacymaths}{19}
\DeclareMathAccent{\grave}   {\mathalpha}{legacymaths}{18}
\DeclareMathAccent{\ddot}    {\mathalpha}{legacymaths}{127}
\DeclareMathAccent{\tilde}   {\mathalpha}{legacymaths}{126}
\DeclareMathAccent{\bar}     {\mathalpha}{legacymaths}{22}
\DeclareMathAccent{\breve}   {\mathalpha}{legacymaths}{21}
\DeclareMathAccent{\check}   {\mathalpha}{legacymaths}{20}
\DeclareMathAccent{\hat}     {\mathalpha}{legacymaths}{94}
\DeclareMathAccent{\dot}     {\mathalpha}{legacymaths}{95}  
\DeclareMathAccent{\mathring}{\mathalpha}{legacymaths}{23}
\DeclareMathSymbol{!}{\mathclose}{legacymaths}{33}
\DeclareMathSymbol{:}{\mathrel}  {legacymaths}{58}
\DeclareMathSymbol{;}{\mathpunct}{legacymaths}{59}
\DeclareMathSymbol{?}{\mathclose}{legacymaths}{63}
\DeclareMathSymbol{0}{\mathalpha}{legacymaths}{`0}
\DeclareMathSymbol{1}{\mathalpha}{legacymaths}{`1}
\DeclareMathSymbol{2}{\mathalpha}{legacymaths}{`2}
\DeclareMathSymbol{3}{\mathalpha}{legacymaths}{`3}
\DeclareMathSymbol{4}{\mathalpha}{legacymaths}{`4}
\DeclareMathSymbol{5}{\mathalpha}{legacymaths}{`5}
\DeclareMathSymbol{6}{\mathalpha}{legacymaths}{`6}
\DeclareMathSymbol{7}{\mathalpha}{legacymaths}{`7}
\DeclareMathSymbol{8}{\mathalpha}{legacymaths}{`8}
\DeclareMathSymbol{9}{\mathalpha}{legacymaths}{`9}
\DeclareMathSymbol{\Gamma}{\mathalpha}{legacymaths}{0}
\DeclareMathSymbol{\Delta}{\mathalpha}{legacymaths}{1}
\DeclareMathSymbol{\Theta}{\mathalpha}{legacymaths}{2}
\DeclareMathSymbol{\Lambda}{\mathalpha}{legacymaths}{3}
\DeclareMathSymbol{\Xi}{\mathalpha}{legacymaths}{4}
\DeclareMathSymbol{\Pi}{\mathalpha}{legacymaths}{5}
\DeclareMathSymbol{\Sigma}{\mathalpha}{legacymaths}{6}
\DeclareMathSymbol{\Upsilon}{\mathalpha}{legacymaths}{7}
\DeclareMathSymbol{\Phi}{\mathalpha}{legacymaths}{8}
\DeclareMathSymbol{\Psi}{\mathalpha}{legacymaths}{9}
\DeclareMathSymbol{\Omega}{\mathalpha}{legacymaths}{10}
\DeclareMathSymbol{+}{\mathbin}{legacymaths}{43}
\DeclareMathSymbol{=}{\mathrel}{legacymaths}{61}
\DeclareMathDelimiter{(}{\mathopen} {legacymaths}{40}{largesymbols}{0}
\DeclareMathDelimiter{)}{\mathclose}{legacymaths}{41}{largesymbols}{1}
\DeclareMathDelimiter{[}{\mathopen} {legacymaths}{91}{largesymbols}{2}
\DeclareMathDelimiter{]}{\mathclose}{legacymaths}{93}{largesymbols}{3}
\DeclareMathDelimiter{/}{\mathord}{legacymaths}{47}{largesymbols}{14}
\DeclareMathSymbol{\mathdollar}{\mathord}{legacymaths}{36}
\DeclareSymbolFont{operators}\encodingdefault\rmdefault\mddefault\updefault
\SetSymbolFont{operators}{normal}\encodingdefault\rmdefault\mddefault\updefault
\SetMathAlphabet\mathrm{normal}\encodingdefault\rmdefault\mddefault\updefault
\SetMathAlphabet\mathit{normal}\encodingdefault\rmdefault\mddefault\itdefault
\SetMathAlphabet\mathbf{normal}\encodingdefault\rmdefault\bfdefault\updefault
\SetMathAlphabet\mathsf{normal}\encodingdefault\sfdefault\mddefault\updefault
\SetMathAlphabet\mathtt{normal}\encodingdefault\ttdefault\mddefault\updefault
\SetSymbolFont{operators}{bold}\encodingdefault\rmdefault\bfdefault\updefault
\SetMathAlphabet\mathrm{bold}\encodingdefault\rmdefault\bfdefault\updefault
\SetMathAlphabet\mathit{bold}\encodingdefault\rmdefault\bfdefault\itdefault
\SetMathAlphabet\mathsf{bold}\encodingdefault\sfdefault\bfdefault\updefault
\SetMathAlphabet\mathtt{bold}\encodingdefault\ttdefault\bfdefault\updefault
%    \end{macrocode}
% The end! Thanks for coming.
%    \begin{macrocode}
%</classfile>
%    \end{macrocode}
%
% \clearpage
% \begingroup
% \renewenvironment{theglossary}
%   {\small\list{}{}
%      \item\relax
%      \glossary@prologue\GlossaryParms 
%      \let\item\@idxitem \ignorespaces 
%      \def\pfill{\hspace*{\fill}}}
%   {\endlist}
% \PrintChanges
% \endgroup
%
% \setcounter{IndexColumns}{2}
% \PrintIndex
%
% \Finale
%
% \typeout{************************************************}
% \typeout{*}
% \typeout{* To finish the installation, move the following}
% \typeout{* file into a directory searched by TeX:}
% \typeout{*}
% \typeout{* \space\space\space active-conf.cls}
% \typeout{*}
% \typeout{************************************************}
%
\endinput
  