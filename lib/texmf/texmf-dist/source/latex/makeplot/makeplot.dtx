% \iffalse (meta-comment)
%
% Doc-Source file to use with LaTeX2e
%
% Copyright 2005-2009 by Jose-Emilio Vila-Forcen (jemilio@gmail.com)
%                   
% All rights reserved.
%
% This work may be distributed and/or modified under the conditions of
% the LaTeX Project Public License, either version 1.3 of this license
% or (at your option) any later version. The latest version of the
% license is in
%
%     http://www.latex-project.org/lppl.txt
%
% and version 1.3 or later is part of all distributions of LaTeX
% version 2003/12/01 or later.
%
% This work has the LPPL maintenance status "maintained".
% The Current Maintainer of this work is Tobias Oetiker (oetiker@ee.ethz.ch).
%
% I N S T A L L A T I O N
%
% makeplot.sty:
% run: latex makeplot.ins
%
% makeplot.dvi and pdf:
% run: latex makeplot.dtx
%      makeindex -s gind.ist -o makeplot.ind makeplot.idx
%			 makeindex -s gglo.ist -o makeplot.gls makeplot.glo
%      latex makeplot.dtx 
%      pdflatex makeplot.dtx

%<*driver>
\documentclass{ltxdoc}
\usepackage{makeplot}   % to get file info
\EnableCrossrefs
%\DisableCrossrefs     % say \DisableCrossrefs if index is ready
\CodelineIndex
\RecordChanges         % gather update information
%\OnlyDescription      % comment out for implementation details
%\OldMakeindex         % use if your MakeIndex is pre-v2.9
%\setlength\hfuzz{15pt} % dont make so many
%\hbadness=7000         % over and under full box warnings
\begin{document}
  \DocInput{makeplot.dtx}
\end{document}
%</driver>
% \fi
%
% \CheckSum{1234}
%
%% \CharacterTable
%%  {Upper-case    \A\B\C\D\E\F\G\H\I\J\K\L\M\N\O\P\Q\R\S\T\U\V\W\X\Y\Z
%%   Lower-case    \a\b\c\d\e\f\g\h\i\j\k\l\m\n\o\p\q\r\s\t\u\v\w\x\y\z
%%   Digits        \0\1\2\3\4\5\6\7\8\9
%%   Exclamation   \!     Double quote  \"     Hash (number) \#
%%   Dollar        \$     Percent       \%     Ampersand     \&
%%   Acute accent  \'     Left paren    \(     Right paren   \)
%%   Asterisk      \*     Plus          \+     Comma         \,
%%   Minus         \-     Point         \.     Solidus       \/
%%   Colon         \:     Semicolon     \;     Less than     \<
%%   Equals        \=     Greater than  \>     Question mark \?
%%   Commercial at \@     Left bracket  \[     Backslash     \\
%%   Right bracket \]     Circumflex    \^     Underscore    \_
%%   Grave accent  \`     Left brace    \{     Vertical bar  \|
%%   Right brace   \}     Tilde         \~}
%%
%
% \DoNotIndex{\@ifnextchar,\@legend}
% \DoNotIndex{\@legendA,\@legendB,\@legendC,\@legendD,\@legendE,\@legendF,\@legendG,%
%     \@legendAf,\@legendBf,\@legendCf,\@legendDf,\@legendEf,\@legendFf,\@legendGf}
% \DoNotIndex{\@legendDL,\@legendDR,\@legendUL,\@legendUR,\@legendXY,\@legendf,\@legendText}
% \DoNotIndex{\@makeplot,\@plotFile,\@plotFileA,\@plotFileB,\@plotFileC,\@plotFileD,%
%     \@plotFileE,\@plotFileF,\@plotFileG,\@whiteBG}
% \DoNotIndex{\begin,\CurrentOption,\data,\dataplot,\def,\define@choicekey, \define@key,\dif,\factorX,\factorXmakeplot,\factorY,\factorYmakeplot, \drawmargins,\drawmarginsmakeplot,\ff,\fi,\fillcolorWhiteBG, \fillstyleWhiteBG,\footnotesize,\FPadd,\FPdiv,\FPmul,\heightPlotFactor, \heightPlotFactormakeplot,\highY,\hspace,\ifMP@color,\ifMP@drawmargins, \ifnum,\l,\leftX,\linecolorWhiteBG,\linestyleWhiteBG,\linewidthWhiteBG, \llegendXY,\lowY,\makeatletter,\makeatother,\makebox,\MP@colorfalse, \MP@colortrue,\MP@drawmarginsfalse,\MP@drawmarginstrue,\n,\NeedsTeXFormat, \newcommand,\newenvironment,\newif\normalsize,\nr,\nrr,\orgX,\orgXmakeplot, \orgY,\orgYmakeplot,\posx,\posy,\ProvidesPackage,\psframe,\pshlabel, \psline,\psset,\pst@addfams,\psvlabel,\readdata,\relax,\RequirePackage, \rightX,\rotateleft,\rput,\s,\sep,\sepTx,\sepTxlegendXY,\sepY,\sepYleendXY, \sepYy,\slegendXY,\tickHerex,\tickHerey,\ticky,\tickymakeplot,\typeout, \unitX,\unitY,\val,\vall,\var,\w,widthPlot,\widthPlotmakeplot,\ww,\www, \x,\xa,\xaa,\xaaa,\xaDL,\xaDR,\xamakeplot,\xaOrigin,\xaXY,\xb,\xbb,\xc, \xDiff,\xinc,\xmid,\xp,\xUL,\xUR,\xx,\xz,\xzDL,\xzDR,\xzmakeplot,\xzz, \y,\ya,\yaa,\yaaa,\yaDL,\yaDR,\yamakeplot,\yaOrigin,\yDiff,\yinc,\ymid, \yUL,\yUR,\yz,\yzDL,\yzDR,\yzmakeplot,\yzXY,\factorBOUNDARYx, \factorBOUNDARYxmakeplot,\factorBOUNDARYy,\factorBOUNDARYymakeplot, \framearcWhiteBG,\newif,\gridDx,\gridDxmakeplot,\gridDy,\gridDymakeplot, \widthPlot,\sepYlegendXY,\sepTy,\ProcessOptions,\definecolor}
%
% \changes{v1.0.7}{2009/08/26}{minor bugs corrected thanks to Herbert Voss}
% \changes{v1.0.6}{2008/01/18}{minor bugs corrected due to other packages updates and improved slightly documentation}
% \changes{v1.0.5}{2005/12/20}{added option big}
% \changes{v1.0.4}{2005/11/23}{corrected small problem that made labels appear twice in the 'ticky' part}
% \changes{v1.0.3}{2005/10/24}{added optional parameters to readdata, changed fileplot to listplot command (allows more options from pstricks-add) and corrected a bug related to the option tickyplot}
% \changes{v1.0.2}{2005/10/04}{rounded some calculated parameters to avoid overflow (tickherex and tickherey)}
% \changes{v1.0.1}{2005/09/24}{data1.mat and data2.mat files are now generated automatically, thanks to Dan}
% \changes{v1.0}{2005/09/22}{Initial version by Jose-Emilio Vila-Forcen}
%
%
%
%
% \GetFileInfo{makeplot.sty}
%
% \title{A plot package for \LaTeXe\thanks{^^A
%        This file has version number \fileversion,
%        last revised \filedate.}}
% \author{Jose-Emilio Vila-Forcen \\ \texttt{jemilio@gmail.com}}
% \date{\filedate}
% \maketitle
%
% \PrintChanges
%
% \section{Introduction}
%
% The functions described in the present document have been created 
% to help in the drawing of plots from Matlab to \LaTeX. The overall
% objective is to create a easy and fast framework to create plots with
% the same style, proper fonts, and just enough parameters.
%
% \section{Known problems}
% \begin{itemize}
%  \item Sometimes a combination of numbers in the dimensions of the plot might cause an error in latex. That is because the combination of numbers has caused some kind of overflow. The source of the problem might be in the makeplot package or in the pstricks family (supposed to be solved in v2.64 of pstricks-add). Send an email to the package author with all the data and hopefully you will receive an answer with an update of the package: the workaround solution is to use a different number, a close one to the original numbe might work.
%
% \end{itemize}
%
%
% \section{Description}
% The package Startplot is composed of several functions. Due to the 
% author's limited knowledge of \TeX, it is done following \LaTeX\ macros 
% and the robustness of the package is not fully demonstrated.
%
% All positions, total size of the figure and so on are calcul%ated 
% automatically using package FP.
%
%
% \subsection{Definition of an especific style}
% Some functions are provided to change the style of the plots:
% \DescribeMacro{\MPbg}
%  Execute |\MPbg| it in order to use black lines with different styles already predefined.
%  It is by default
%
% \DescribeMacro{\MPcolor}
%  Execute |\MPcolor| it in order to use solid color lines already predefined.
%  It can be selected using the |color| package option or executing the command.
%
% \DescribeMacro{colorLine[ABCDEFG]}
%  If you want to change the colors of the lines, you may change one by one redefining the colors.
%  For example:
%  \begin{quote}
%       |\definecolor{colorLineA}{rgb}{1,1,1}|
%  \end{quote}
%
% \DescribeMacro{\styleofLine[ABCDEFG]}
% It is possible to modify the seven predefined styles one by one: simply recreate a new definition
% for the desired one(s).
%
% \DescribeMacro{Font commands}
% The commands |\fontaxeX|, |\fontaxeY|, |\fonttitleX|, |\fonttitleY| and |\fontlegend| contains
% the definitions of the fonts used in different parts of the legend: axes, labels (or titles) and legend.
% You may change them as, for example:
%  \begin{quote}
%       |\def\fontaxeY{\small}|
%  \end{quote}
%
% \DescribeMacro{\defaultOptionsMakeplot}
% The |\defaultOptionsMakeplot| command is executed at the beginning of the definition of the plot.
% redefine it to your desired code if you need to execute something more for each picture.
%
%
%\subsection{Create the plot}
% The plot is composed of several orders consecutive. Besides the order 
% described here, a lot of them exist in the file which obtain the same 
% result, but using more input parameters. The ones that I describe here 
% are enough for the most of the cases.
%
% \DescribeEnv{makeplot}
% The |makeplot| environment is the starting point to create the plot. The 
% command line is as follows:
%  \begin{quote}
%        |\begin{makeplot}|\oarg{keys}\marg{label in Y}\marg{label in X}
%        .....
%        |\end{makeplot}|
%   \end{quote}
% The optional arguments are the keys from PsTricks, including the ones for |\psaxes| function. Also,
% some new keys are added:
%\begin{description}
%	\item[Dx] Distance between labels in X
%	\item[Dy] Distance between labels in Y
%	\item[width] Width of the plot in mm
%	\item[heightFactor] Factor of height length respect to the width
%	\item[startX] Minimum value in the axis X
%	\item[startY] Minimum value in the axis Y
%	\item[endX] Maximum value in the axis X
%	\item[endY] Maximum value in the axis Y
%	\item[factorBoundaryX] Separation to the border
%	\item[factorBoundaryY] Separation to the border
%	\item[captionX] Separation to the caption in X, over 100
%	\item[captionY] Separation to the caption in Y, over 100
%	\item[changeEndXpos] If it is needed to change the end of the axis X, because of external legend (yes or no)
%	\item[changeEndXsize] How many mm more than the estandard
%	\item[tickyplot] Special ticks (all, x, y, none)
%	\item[drawmargins] Draw margins in this figure (there exist an option in the package to do it globally)
% \item[orgX and orgY] Not working properly, intended to change the origin of the graphic
%\end{description}
% For example, a typical configuration for semilog case it is to use the options:
%\begin{quote}
%     ylogBase=10, logLines=y, subticks=10, xsubticks=1
%\end{quote}
% and LogLines can take the values x, y or all.
%
% \DescribeMacro{\makeplot}
% The |\makeplot| macro works as the environment, but once the plot is finished it is needed to write |\end{pspicture}|
%
% \subsection{Plotting the data from files}
%
% \DescribeMacro{\plotFile}
% The |\plotFile|\oarg{keys}\marg{file} macro takes the data from a file and plot it using the existing
% properties for lines in PsTricks taking into account the optional keys given to the command.
%
% \DescribeMacro{\plotFile[ABCDEFG]}
% These macros have the same configuration as |\plotFile|, but each one uses the line style predefined. It accept the optinal arguments plotNoMax and PlotNo to say how many different plots are in the file and which one is intented to be plot, and nStep, xStart, xEnd, yStart and yEnd in order to select which values of the file should be plot.
%
% \subsection{The legend}
% The |makeplot| package can do very nice legends and without effort. There exists the next keys to control the apearance of the legend:
%\begin{description}
%	\item[fillcolorWhiteBG] Select the color name for the background of the legend
% \item[fillstyleWhiteBG] Select the fillstyle for the background of the legend
% \item[framearcWhiteBG] dimension of the arc of the legend box
% \item[linewidthWhiteBG] linewidth of the legend box
% \item[linestyleWhiteBG] linestyle of the legend box
% \item[linecolorWhiteBG] color of the line of the legend box
%\end{description}
% and
%\begin{description}
%	\item[legendSep] Separation between legends in vertical
% \item[lineLength] Length of the line in the legend area
% \item[lineTextSep] Separation between the line and the text
% \item[borderlineSep] Separation from the border of the box to the line
%\end{description}
%
% The main commands are as follows:
%
%
% \DescribeMacro{\legendXY}
% The |\legendXY| macro usage is as follows:
% \begin{quote}
%    |\legendXY|\oarg{keys}\parg{coordinate X,coordinateY}\marg{number of lines to be in the legend}%
%         \marg{width in mm}
% \end{quote}
%
% \DescribeMacro{\legend[DL,DR,UL,UR]}
% These macros avoid the needed to specify a coordinate and draw the legend in the 
% position down-left, down-right, up-left or up-right of the plot. Therefore, the syntax is:
% \begin{quote}
%    |\legendDL|\oarg{keys}\marg{number of lines to be in the legend}%
%         \marg{width in mm}
% \end{quote}
%
% There exist some different ways to write the legends into the box. The easiest one is to use the
% next commands:
%
% \DescribeMacro{\legend[ABCDEFG]f}
% These macros write the legends into the box in the order of the tex source. Each one has a different
% line style as it was for the plot. The syntax is:
% \begin{quote}
%    |\legendCf|\oarg{keys}\marg{text}%
% \end{quote}
% where the keys might modify the line style.
%
%
% \DescribeMacro{\legendText}
% The |\legendText| macro includes some text in the legend without a line: 
% \begin{quote}
%    |\legendText|\oarg{keys}\marg{text}%
% \end{quote}
% where the keys are not used, but kept in the definition of the macro for paralelism.
%
% \section{Record proper Matlab data}
% \index{Matlab}
% Matlab is one of the most famous data processing software in the 
% research community. Because of its simplicity, the generation of the 
% needed files can be done easily after their generation.
%
% Each plot should be saved in a separate file in ASCII format. Thus, 
%follow the next process for each one of the lines to plot:
% \begin{enumerate}
%	\item Create a matrix Nx2, where N denotes the dimensionality of the 
%    data. Put in the first column the data of the X-axis, and in the  
%    second one the values in the Y-axis corresponding to the previous X 
%    one, let's call this matrix \emph{values};
% \item Save \emph{values} in a file as follows: \verb+ save file.mat values -ascii+.
% \end{enumerate}
%
%\paragraph{Remarks}
%\begin{itemize}
%  \item Editing in a text editor should be possible, and numbers should be visible in ASCII format;
%	\item The data should be real;
%  \item It is possible to comment values in the file with \verb+%+;
%  \item It is possible to modify manually the values in the file.
%\end{itemize}
%
%
%
% \section{Create EPS files}
% \index{EPS}
% It is possible to create EPS files and use them directly in latex. The %
% benefits are faster compilation and it is always good to have them in%
% separate files, for example for distribution.
%
% In order to create EPS files, follow the procedure:
% \begin{enumerate}
% \item Install perl from www.perl.com; \index{perl}
% \item Copy bbox.exe from ps2eps.zip file to \verb+c:\texmf\miktex\bin+. \index{ps2eps}%
%    ps2eps.zip can be found at \\ { \verb+www.telematik.informatik.uni-karlsruhe.de/~bless/ps2eps.html+};
% \item \verb+SET PATHEXT=.pl;%PATHEXT%+ or use the start, settings, control panel,%
%    system, advanced tab, environment variables and edit the PATHEXT entry accordingly;
% \item Rename ps2eps to ps2eps.pl and copy it to \verb+c:\texmf\miktex\bin+.
% \end{enumerate}
%
% Now, we create a batch file to help in the compilation of each time:%
%   Create a bat file something like: \verb+makefigure.bat+, \index{makefigure.bat} save it%
%   in \verb+c:\texmf\miktex\bin+. It should contain the following lines:
% \begin{verbatim}
% echo off
% del figures/%1.eps
% latex %1eps.tex
% dvips -t a4 -P pdf %1eps.dvi
% ps2eps.pl %1eps.ps %1eps.eps
% del %1eps.ps 
% del %1eps.eps
% mkdir figures
% rename %1eps.eps.eps %1.eps
% move %1.eps figures
% \end{verbatim}
% It is possible to change the contain above for different file names.%
%   In my case, I am creating files are follows:
%
% \begin{verbatim}
% \input{headers.tex}
% \begin{document}
% \pagestyle{empty}
% \input{fig1.tex}
% \end{document}
% \end{verbatim}
% with the name \verb+NAMEeps.tex+ where \verb+NAME+ can be changed%
%  to the desired name. The file \verb+headers.tex+ contains all%
%  the headers of my report and therefore all proper fonts and so on%
%  will be used in this file and in the general one.
%
% To create this file, it is possible to make another batch file called \verb+fileForEPS.bat+ such that:
%
% \begin{verbatim}
% echo off
% echo \input{headers.tex} \begin{document} \emph{(continue in the same line)}
%      \pagestyle{empty} \input{%1.tex} \end{document} > %1eps.tex
% \end{verbatim}
%
% Execute \verb+makefigure NAME+ in the proper directory, and the EPS picture file%
%  will be created in the directory \verb+figures/+. If you need to create the%
%  additional tex file, execute \verb+fileForEPS NAME+ and it will be done.
%
%
% The procedure described above allows to use the following code in the main file:
% \begin{verbatim}
% \begin{figure}
% \centering
% %\input{fig.tex}
% \includegraphics{figure/fig.eps}
% \caption{Caption of the figure}
% \label{fig:fig}
% \end{figure}
% \end{verbatim}
%where the commented line will describe which file is used: it is possible%
% to choose between the tex file or (once picture is fixed and in order to%
% improve speed) the eps one.
%
% 
% \clearpage
% \section{An example file}
%    \begin{macrocode}
%<*mptest>
\documentclass[]{article}
\usepackage[color]{makeplot}
\begin{document}
\begin{figure}
\centering

\begin{makeplot}[startX=-10, endX=5, startY=-1, endY=0, Dx = 5,%
   width=40, heightFactor=1, %
   ylogBase=10, logLines=y, subticks=10, xsubticks=1]%
   {$P_e$}{WNR, [dB]}
\plotFileA{data1.mat}
\plotFileB{data2.mat}
\legendDL{24.5}{2}
\legendAf{UDQ-QIM}
\legendBf{UQ-QIM}
\end{makeplot}

\caption{Performance analysis result.}
\label{fig:results}
\end{figure}

\end{document}
%</mptest>
%    \end{macrocode}
% \subsection{Data files:}
% \paragraph{data1.mat}
%    \begin{macrocode}
%<*data1>
 -1.0000000e+001 -3.1186120e-001
 -9.5000000e+000 -3.1382220e-001
 -9.0000000e+000 -3.1610308e-001
 -8.5000000e+000 -3.1874224e-001
 -8.0000000e+000 -3.2177814e-001
 -7.5000000e+000 -3.2524775e-001
 -7.0000000e+000 -3.2918480e-001
 -6.5000000e+000 -3.3361772e-001
 -6.0000000e+000 -3.3856759e-001
 -5.5000000e+000 -3.4404614e-001
 -5.0000000e+000 -3.5005399e-001
 -4.5000000e+000 -3.5657933e-001
 -4.0000000e+000 -3.6359717e-001
 -3.5000000e+000 -3.7106902e-001
 -3.0000000e+000 -3.7894319e-001
 -2.5000000e+000 -3.8715537e-001
 -2.0000000e+000 -3.9562939e-001
 -1.5000000e+000 -4.0427825e-001
 -1.0000000e+000 -4.1300527e-001
 -5.0000000e-001 -4.2170561e-001
  0.0000000e+000 -4.3026847e-001
  5.0000000e-001 -4.5284882e-001
  1.0000000e+000 -4.8524584e-001
  1.5000000e+000 -5.2154969e-001
  2.0000000e+000 -5.6193985e-001
  2.5000000e+000 -6.0664420e-001
  3.0000000e+000 -6.5595407e-001
  3.5000000e+000 -7.1023618e-001
  4.0000000e+000 -7.6994081e-001
  4.5000000e+000 -8.3560655e-001
  5.0000000e+000 -9.0786295e-001
%</data1>
%    \end{macrocode}
% \paragraph{data2.mat}
%    \begin{macrocode}
%<*data2>
 -1.0000000e+001 -3.0170171e-001
 -9.5000000e+000 -3.0182275e-001
 -9.0000000e+000 -3.0197731e-001
 -8.5000000e+000 -3.0217679e-001
 -8.0000000e+000 -3.0243768e-001
 -7.5000000e+000 -3.0278484e-001
 -7.0000000e+000 -3.0325664e-001
 -6.5000000e+000 -3.0391260e-001
 -6.0000000e+000 -3.0484295e-001
 -5.5000000e+000 -3.0617875e-001
 -5.0000000e+000 -3.0809964e-001
 -4.5000000e+000 -3.1083660e-001
 -4.0000000e+000 -3.1466753e-001
 -3.5000000e+000 -3.1990577e-001
 -3.0000000e+000 -3.2688343e-001
 -2.5000000e+000 -3.3593295e-001
 -2.0000000e+000 -3.4737023e-001
 -1.5000000e+000 -3.6148225e-001
 -1.0000000e+000 -3.7852034e-001
 -5.0000000e-001 -3.9869936e-001
  0.0000000e+000 -4.2220206e-001
  5.0000000e-001 -4.4918738e-001
  1.0000000e+000 -4.7980138e-001
  1.5000000e+000 -5.1418989e-001
  2.0000000e+000 -5.5251165e-001
  2.5000000e+000 -5.9495142e-001
  3.0000000e+000 -6.4173231e-001
  3.5000000e+000 -6.9312654e-001
  4.0000000e+000 -7.4946429e-001
  4.5000000e+000 -8.1114012e-001
  5.0000000e+000 -8.7861718e-001
%</data2>
%    \end{macrocode}
%
% \StopEventually{}
% \clearpage
%
% \section{The implementation}
%    \begin{macrocode}
%<*makeplot>
%    \end{macrocode}
%
%
%    \subsection{Headers}
%
%    First we test that we got the right format and name the package, and load external packages.
%    \begin{macrocode}
\NeedsTeXFormat{LaTeX2e}%
\ProvidesPackage{makeplot}[2009/08/26 %
                           v1.0.7 %
                           Plots utility from Jose-Emilio Vila-Forcen]%
\RequirePackage[nomessages]{fp}%
\RequirePackage{pst-plot}%
\RequirePackage{pstricks-add}%
\RequirePackage{xkeyval}%
%    \end{macrocode}
%
%    \begin{macro}{color}
%    The option |color| leads to a redefinition of the line styles
% of the plot and uses colors with solid lines. Otherwise, black lines
% with different line styles are used
%    \begin{macrocode}
\newif\ifMP@color%
\MP@colorfalse%
\DeclareOption{color}{%
  \MP@colortrue%
  \typeout{ ------}%
  \typeout{ ------   makePlot package working in color}%
  \typeout{ ------}}%
%    \end{macrocode}
%    \end{macro}
%
%    \begin{macro}{drawmargins}
% The option | drawmargins | draw the margins of the figues. This option
% is useful in order to check if something is going out of the margins,
% since there is not internal checking for that.
%    \begin{macrocode}
\newif\ifMP@drawmargins%
\MP@drawmarginsfalse%
\DeclareOption{drawmargins}{%
  \MP@drawmarginstrue%
  \typeout{ ------}%
  \typeout{ ------   makePlot package drawing the margins}%
  \typeout{ ------}}%
%    \end{macrocode}
%    \end{macro}
%
%    \begin{macro}{big}
% The option | big | put the lines of the plot larger
%    \begin{macrocode}
\newif\ifMP@big%
\MP@bigfalse%
\DeclareOption{big}{%
	\MP@bigtrue
  \typeout{ ------}%
  \typeout{ ------   makePlot package in big size}%
  \typeout{ ------}}%
%    \end{macrocode}
%    \end{macro}
%
% If an unknown option is used, a warning will be displayed.
%    \begin{macrocode}
\DeclareOption*{%
  \PackageWarning{startPlot}{%
    Unknown option '\CurrentOption'}}%
%    \end{macrocode}
%
%    Now we process the options.
%    \begin{macrocode}
\ProcessOptions\relax%
%    \end{macrocode}
%
% \subsection{Commands to fix the style}
%
%    \begin{macro}{\MPbg}
% Command to use the styles of the lines in black and white, by default:
%    \begin{macrocode}
\newcommand{\MPbg}{%
 \def\styleoflineA{%
 		\psset{linestyle=solid, dash=1pt 0pt 0pt 0pt, dotsep=0pt,%
 		   linecolor=black}}%
 \def\styleoflineB{%
 		\psset{linestyle=dashed, dash=5pt 5pt 0pt 0pt, dotsep=0pt,%
 		   linecolor=black}}%
 \def\styleoflineC{%
 		\psset{linestyle=dotted, dash=3pt 2pt 0pt 0pt, dotsep=1pt,%
 		   linecolor=black}}%
 \def\styleoflineD{%
 		\psset{linestyle=dashed, dash=4pt 1.5pt 1pt 1.5pt, dotsep=3pt,%
 		   linecolor=black}}%
 \def\styleoflineE{%
 		\psset{linestyle=dotted, dash=1pt 4pt 0pt 0pt, dotsep=3pt,%
 		   linecolor=black}}%
 \def\styleoflineF{%
 		\psset{linestyle=dashed, dash=6pt 1.5pt 3pt 1.5pt, dotsep=3pt,%
 		   linecolor=black}}%
 \def\styleoflineG{%
 		\psset{linestyle=dashed, dash=2pt 6pt 0pt 0pt, dotsep=0pt,%
 		   linecolor=black}}}%
%    \end{macrocode}
%    \end{macro}
%
%    \begin{macro}{MPcolor}
% Definition of the standard colors for the lines, using the
% color option of the package. They might be changed by the user:
%    \begin{macrocode}
\definecolor{colorLineA}{rgb}{0,0,1}%
\definecolor{colorLineB}{rgb}{1,0,0}%
\definecolor{colorLineC}{rgb}{0,1,0}%
\definecolor{colorLineD}{rgb}{0,1,1}%
\definecolor{colorLineE}{rgb}{1,0,1}%
\definecolor{colorLineF}{rgb}{1,1,0}%
\definecolor{colorLineG}{rgb}{0.5,0.5,0.5}%
%    \end{macrocode}
%
% Command to set the styles of the lines using colors:
%    \begin{macrocode}
\newcommand{\MPcolor}{%
 \def\styleoflineA{\psset{linestyle=solid, linecolor=colorLineA}}%
 \def\styleoflineB{\psset{linestyle=solid, linecolor=colorLineB}}%
 \def\styleoflineC{\psset{linestyle=solid, linecolor=colorLineC}}%
 \def\styleoflineD{\psset{linestyle=solid, linecolor=colorLineD}}%
 \def\styleoflineE{\psset{linestyle=solid, linecolor=colorLineE}}%
 \def\styleoflineF{\psset{linestyle=solid, linecolor=colorLineF}}%
 \def\styleoflineG{\psset{linestyle=solid, linecolor=colorLineG}}}%
%    \end{macrocode}
%    \end{macro}
%
% According to the package option, one of the styles will be loaded:
%    \begin{macrocode}
\ifMP@color%
 \MPcolor%
\else%
 \MPbg%
\fi%
%    \end{macrocode}
%
%    \begin{macrocode}
\makeatletter%
%    \end{macrocode}
%
%    \begin{macro}{Font sizes}
% Definition of the standard font sizes in the plot. It is possible to redefine them.
%    \begin{macrocode}
\def\fontaxeY{\normalsize}%
\def\fontaxeX{\normalsize}%
\def\fonttitleY{\normalsize}%
\def\fonttitleX{\normalsize}%
\def\fontlegend{\footnotesize}%
%    \end{macrocode}
%    \end{macro}
%
% \subsection{Main macros: makeplot}
% Options for the command |makeplot| and the addition of the family of keys. 
%
%    \begin{macrocode}
\define@key[psset]{makeplot}{Dx}[1]%
   {\def\gridDxmakeplot{#1}}%
\define@key[psset]{makeplot}{Dy}[1]%
   {\def\gridDymakeplot{#1}}%
\define@key[psset]{makeplot}{width}[50]%
   {\def\widthPlotmakeplot{#1}}%
\define@key[psset]{makeplot}{heightFactor}[1]%
   {\def\heightPlotFactormakeplot{#1}}%
\define@key[psset]{makeplot}{startX}[0]%
   {\def\xamakeplot{#1}}%
\define@key[psset]{makeplot}{startY}[0]%
   {\def\yamakeplot{#1}}%
\define@key[psset]{makeplot}{endX}[1]%
   {\def\xzmakeplot{#1}}%
\define@key[psset]{makeplot}{endY}[1]%
   {\FPmul\yzmakeplot{#1}{1.0001}}%
\define@key[psset]{makeplot}{factorBoundaryX}%
   {\FPdiv\factorBOUNDARYxmakeplot{#1}{1.5}}%
\define@key[psset]{makeplot}{factorBoundaryY}%
   {\FPdiv\factorBOUNDARYymakeplot{#1}{1.5}}% 
\define@key[psset]{makeplot}{captionY}%
   {\FPdiv\factorXmakeplot{#1}{8}}% 
\define@key[psset]{makeplot}{captionX}%
   {\FPdiv\factorYmakeplot{#1}{11}}%
\define@choicekey[psset]{makeplot}{changeEndXpos}[\var\nr]{no,yes}%
   {\def\endXamakeplot{\nr}}% 
\define@key[psset]{makeplot}{changeEndXsize}%
   {\def\endXbmakeplot{#1}}% 
\define@choicekey[psset]{makeplot}{tickyplot}[\val\nr]{all,x,y,none}%
   {\def\tickymakeplot{\nr}} % 
\define@choicekey[psset]{makeplot}{drawmargins}[\vall\nrr]{yes,no}%
   {\def\drawmarginsmakeplot{\nrr}} % 
\define@key[psset]{makeplot}{orgX}%
   {\def\orgXmakeplot{#1}} % 
\define@key[psset]{makeplot}{orgY}%
   {\def\orgYmakeplot{#1}} % 
\pst@addfams{makeplot}%
%    \end{macrocode}
%
% Default values for the above parameters
%    \begin{macrocode}
\psset{Dx=1, Dy=1, 
	width=50, heightFactor=1, startX=0, endX=1, startY=0, endY=1,
	showorigin=true, axesstyle=frame, ticks=all, labels=all,
	factorBoundaryX=1, factorBoundaryY=1, captionY=100, captionX=100,
	xsubticks=0, subticksize=1, subtickcolor=black}%
\ifMP@big%
 \psset{tickwidth=0.5pt, subtickwidth=0.5pt, linewidth=1pt}%
\else%
 \psset{tickwidth=0.2pt, subtickwidth=0.2pt, linewidth=0.5pt}%
\fi%
\psset{tickyplot=all}%
\psset{changeEndXsize=0, changeEndXpos=no}%
\ifMP@drawmargins%
 \psset{drawmargins=yes}%
\else%
 \psset{drawmargins=no}%
\fi%
\psset{orgX=314, orgY=314}%
%    \end{macrocode}
%
% \begin{environment}{makeplot}
% The |makeplot| environemnt perfoms most of the tasks needed for the plot. The
% axes are drawn, the legends and everything related with the background of the plot
% besides the legend, that will arrive later.
%    \begin{macrocode}
\newenvironment{makeplot}[3][]%
  {\makeplot[#1]{#2}{#3}}%
  {\end{pspicture}}%
%    \end{macrocode}
% \end{environment}
%
% \begin{macro}{\defaultOptionsMakeplot}
% Default values, it is possible to change this command as desired
% to include some source to be executed in each |makeplot| command
% or enviroment call
%    \begin{macrocode}
	\def\defaultOptionsMakeplot{}%
%    \end{macrocode}
% \end{macro}
%	
% \begin{macro}{\makeplot}
% This is the most important macro, it can be used instead of the 
% makeplot enviroment just adding |\end{pspicture}| at the end of the
% plot.
%    \begin{macrocode}
\def\makeplot{\@ifnextchar[\@makeplot{\@makeplot[]}}%
\def\@makeplot[#1]#2#3{%
  % Use the default options command and process
  % the options of the command line
  \defaultOptionsMakeplot%
  \psset{#1}%
  % Calculate the dimensions of the plot
  \def\gridDx{\gridDxmakeplot}%
  \def\gridDy{\gridDymakeplot}%
  \def\widthPlot{\widthPlotmakeplot}%
  \def\heightPlotFactor{\heightPlotFactormakeplot}%
  \def\xa{\xamakeplot}%
  \def\ya{\yamakeplot}%
  \def\xz{\xzmakeplot}%
  \def\yz{\yzmakeplot}%
  \def\factorBOUNDARYx{\factorBOUNDARYxmakeplot}%
  \def\factorBOUNDARYy{\factorBOUNDARYymakeplot}%
  \def\factorX{\factorXmakeplot}%
  \def\factorY{\factorYmakeplot}%
  \def\endXa{\endXamakeplot}%
  \def\endXb{\endXbmakeplot}%
  \def\ticky{\tickymakeplot}%
  \def\drawmargins{\drawmarginsmakeplot}%
  \def\orgX{\orgXmakeplot}%
  \def\orgY{\orgYmakeplot}%
  %
  % Use the fonts established
  \def\pshlabel{\fontaxeX}%
  \def\psvlabel{\fontaxeY}%
  \def\pshlabel{\fontaxeX}%
  \def\psvlabel{\fontaxeY}%
  %	
  % Calculate the proper units for the plot
  \FPadd\xDiff\xz{-\xa}%
  \FPdiv\unitX\widthPlot\xDiff%
  \FPadd\yDiff\yz{-\ya}%
  \FPmul\ff\widthPlot\heightPlotFactor%
  \FPdiv\unitY\ff\yDiff%
  \psset{xunit=\unitX mm,yunit=\unitY mm}%
  %
  \FPadd\yinc\yz{-\ya}%
  \FPdiv\ymid\yinc{2}%
  \FPadd\ymid\ymid\ya%
  \FPadd\xinc\xz{-\xa}%
  \FPdiv\xmid\xinc{2}%
  \FPadd\xmid\xmid\xa%
  \FPadd\xzz\xz{0}%
  %
  % Related to the position of the legends
  \FPdiv\factorX\factorX\unitX%
  \FPadd\xaa\xa{-\factorX}%
  \FPdiv\factorY\factorY\unitY%
  \FPadd\yaa\ya{-\factorY}%
  %
  % Related to the size of the plot
  \FPdiv\factorBOUNDARYx\factorBOUNDARYx\unitX%
  \FPadd\xaaa\xaa{-\factorBOUNDARYx}%
  \FPdiv\factorBOUNDARYy\factorBOUNDARYy\unitY%
  \FPadd\yaaa\yaa{-\factorBOUNDARYy}%
  %
  % Change of the size of the plot
  % if it is asked in the options,
  % like putting the legend at the right
  \ifnum \endXb>0
    \FPdiv\w{\endXb}\unitX%
    \ifnum \endXa=0
      \def\endXa{\xz}%
    \fi%
    \FPadd\ww{\endXa}{-\xz}%
    \def\sep{0.5}%
    \FPdiv\www\sep\unitX%
    \FPadd\xzz\xz\w%
    \FPadd\xzz\xzz\ww%
    \FPadd\xzz\xzz\www	%
  \fi%
  %
  % Starting the pspicture environment
  \begin{pspicture}(\xaaa,\yaaa)(\xzz,\yz)%
  %
  % Draw a box with the margins if requested
  \ifnum \drawmargins=0
    \psframe(\xaaa,\yaaa)(\xzz,\yz)%
  \fi%
  %
  \FPadd\tickHerex{\yz}{-\ya}%
  \FPmul\tickHerex\tickHerex\unitY%
  \FPadd\tickHerey{\xz}{-\xa}%
  \FPmul\tickHerey\tickHerey\unitX%
  %
  % To change the origin of the plot
  % not working yet
  \def\xaOrigin{\xa}%
  \def\yaOrigin{\ya}%
  \ifnum \orgX=314
  \else%
    \def\xaOrigin{\orgX}%
  \fi%
  \ifnum \orgY=314
  \else%
    \def\yaOrigin{\orgY}%
  \fi	%
  %
  % Put the ticks as given in the options
  \ifnum \ticky=0
    \psaxes[Ox=\xa,Oy=\ya,Dx=\gridDx,Dy=\gridDy, #1,%
        ticksize=-4pt 4pt, subticks=0, subticksize=0,%
        tickwidth=0.5pt, linewidth=0pt,%
        axesstyle=axes, linecolor=white, #1, ticks=all, labels=none]%
        {-}(\xa,\ya)(\xa,\ya)(\xz,\yz)%
  \fi%
  \ifnum \ticky=1
    \psaxes[Ox=\xa,Oy=\ya,Dx=\gridDx,Dy=\gridDy, #1,%
        ticksize=-4pt 4pt, subticks=0, subticksize=0,%
        tickwidth=0.5pt, linewidth=0pt,%
        axesstyle=axes, linecolor=white, #1, ticks=x, labels=none]%
        {-}(\xa,\ya)(\xa,\ya)(\xz,\yz)%
  \fi%
  \ifnum \ticky=2
    \psaxes[Ox=\xa,Oy=\ya,Dx=\gridDx,Dy=\gridDy, #1,%
        ticksize=-4pt 4pt, subticks=0, subticksize=0,%
        tickwidth=0.5pt, linewidth=0pt,%
        axesstyle=axes, linecolor=white, #1, ticks=y, labels=none]%
        {-}(\xa,\ya)(\xa,\ya)(\xz,\yz)%
  \fi%
  \ifnum \ticky=3
  \fi%
  %	
  % Put the main axes and the grid
  \FPround\tickHerex{\tickHerex}{10}%
  \FPround\tickHerey{\tickHerey}{10}%
  \psaxes[xticksize=0mm \tickHerex mm, yticksize=0mm \tickHerey mm,%
	    Ox=\xa,Oy=\ya,Dx=\gridDx,Dy=\gridDy, #1]%
	    {-}(\xaOrigin,\yaOrigin)(\xa,\ya)(\xz,\yz)%
	\def\MP@xa{\xaOrigin}%
	\def\MP@ya{\yaOrigin}%
	\def\MP@xz{\xz}%
	\def\MP@yz{\yz}%
  %
  % Write the legend of the plot
  \rput(\xaa,\ymid){\rotateleft{\fonttitleY #2}}%
  \rput(\xmid,\yaa){\fonttitleX #3}%
  \ifMP@big%
   \psset{linewidth=2pt}%
  \else%
   \psset{linewidth=1pt}%
  \fi%
}%
%    \end{macrocode}
% \end{macro}
%
% \subsection{How to plot the data of a file}
% \begin{macro}{\plotFile}
% Function to plot a file with the linestyle given
%    \begin{macrocode}
\def\plotFile{\@ifnextchar[\@plotFile{\@plotFile[]}}%
\def\@plotFile[#1]#2{%
  \readdata[#1]{\data}{#2}\listplot[#1]{\data}}%
%    \end{macrocode}
% \end{macro}
% And now, a specific function for each linestyle predefined:
% \begin{macro}{\plotFileA}
%    \begin{macrocode}
\def\plotFileA{\@ifnextchar[\@plotFileA{\@plotFileA[]}}%
\def\@plotFileA[#1]#2{%
  \styleoflineA%
  \plotFile[#1]{#2}}%
%    \end{macrocode}
% \end{macro}
% \begin{macro}{\plotFileB}
%    \begin{macrocode}
\def\plotFileB{\@ifnextchar[\@plotFileB{\@plotFileB[]}}%
\def\@plotFileB[#1]#2{%
  \styleoflineB%
  \plotFile[#1]{#2}}%
%    \end{macrocode}
% \end{macro}
% \begin{macro}{\plotFileC}
%    \begin{macrocode}
\def\plotFileC{\@ifnextchar[\@plotFileC{\@plotFileC[]}}%
\def\@plotFileC[#1]#2{%
  \styleoflineC%
  \plotFile[#1]{#2}}%
%    \end{macrocode}
% \end{macro}
% \begin{macro}{\plotFileD}
%    \begin{macrocode}
\def\plotFileD{\@ifnextchar[\@plotFileD{\@plotFileD[]}}%
\def\@plotFileD[#1]#2{%
  \styleoflineD%
  \plotFile[#1]{#2}}%
%    \end{macrocode}
% \end{macro}
% \begin{macro}{\plotFileE}
%    \begin{macrocode}
\def\plotFileE{\@ifnextchar[\@plotFileE{\@plotFileE[]}}%
\def\@plotFileE[#1]#2{%
  \styleoflineE%
  \plotFile[#1]{#2}}%
%    \end{macrocode}
% \end{macro}
% \begin{macro}{\plotFileF}
%    \begin{macrocode}
\def\plotFileF{\@ifnextchar[\@plotFileF{\@plotFileF[]}}%
\def\@plotFileF[#1]#2{%
  \styleoflineF%
  \plotFile[#1]{#2}}%
%    \end{macrocode}
% \end{macro}
% \begin{macro}{\plotFileG}
%    \begin{macrocode}
\def\plotFileG{\@ifnextchar[\@plotFileG{\@plotFileG[]}}%
\def\@plotFileG[#1]#2{%
  \styleoflineG%
  \plotFile[#1]{#2}}%
%    \end{macrocode}
% \end{macro}
%
% \subsection{Legends code}
%
% \begin{macro}{\legend}
% Defines the general |\legend| command, the inputs are:
% the optional parameter with PSTricks keys, the position in X,
% the length of the line, the separation between line and text,
% the position in Y and the text of the legend.
%    \begin{macrocode}
\def\legend{\@ifnextchar[\@legend{\@legendf[]}}%
\def\@legend[#1]#2#3#4#5#6{%
  \def\xb{#2}%
  \def\x{#3}%
  \FPdiv\x\x\unitX%
  \FPadd\xbb\xb\x%
  \def\x{#4}%
  \FPdiv\x\x\unitX%
  \FPadd\xc\xbb\x%
  %
  \def\y{#5}%
  \psline[#1]{-}(\xb,\y)(\xbb,\y)%
  \rput(\xc,\y){\makebox[0 cm][l]{{\fontlegend #6}}}}%
%    \end{macrocode}
% \end{macro}
%
% \begin{macro}{\legend[ABCDEFG]}
% One command for each line style: as the previous command, but
% now using the predefined styles
%    \begin{macrocode}
\def\legendA{\@ifnextchar[\@legendA{\@legendA[]}}%
\def\@legendA[#1]#2#3#4#5{%
  \styleoflineA%
  \legend[#1]{#2}{#3}{#4}{#5}}%
\def\legendB{\@ifnextchar[\@legendB{\@legendB[]}}%
\def\@legendB[#1]#2#3#4#5{%
  \styleoflineB%
  \legend[#1]{#2}{#3}{#4}{#5}}%
\def\legendC{\@ifnextchar[\@legendC{\@legendC[]}}%
\def\@legendC[#1]#2#3#4#5{%
  \styleoflineC%
  \legend[#1]{#2}{#3}{#4}{#5}}%
\def\legendD{\@ifnextchar[\@legendD{\@legendD[]}}%
\def\@legendD[#1]#2#3#4#5{%
  \styleoflineD%
  \legend[#1]{#2}{#3}{#4}{#5}}%
\def\legendE{\@ifnextchar[\@legendE{\@legendE[]}}%
\def\@legendE[#1]#2#3#4#5{%
  \styleoflineE%
  \legend[#1]{#2}{#3}{#4}{#5}}%
\def\legendF{\@ifnextchar[\@legendF{\@legendF[]}}%
\def\@legendF[#1]#2#3#4#5{%
  \styleoflineF%
  \legend[#1]{#2}{#3}{#4}{#5}}%
\def\legendG{\@ifnextchar[\@legendG{\@legendG[]}}%
\def\@legendG[#1]#2#3#4#5{%
  \styleoflineG%
  \legend[#1]{#2}{#3}{#4}{#5}}%
%    \end{macrocode}
% \end{macro}
%
% \begin{macro}{\legend[ABCDEFG]f}
% An automatic legend mode, where the position is automatically
% calculated for each one in consecutive order
%    \begin{macrocode}
\def\legendAf{\@ifnextchar[\@legendAf{\@legendAf[]}}%
\def\@legendAf[#1]#2{%
  \legendA[#1]{\posx}{\l}{\s}{\posy}{#2}%
  \FPadd\posy\posy{-\dif}}%
\def\legendBf{\@ifnextchar[\@legendBf{\@legendBf[]}}%
\def\@legendBf[#1]#2{%
  \legendB[#1]{\posx}{\l}{\s}{\posy}{#2}%
  \FPadd\posy\posy{-\dif}}%
\def\legendCf{\@ifnextchar[\@legendCf{\@legendCf[]}}%
\def\@legendCf[#1]#2{%
  \legendC[#1]{\posx}{\l}{\s}{\posy}{#2}%
  \FPadd\posy\posy{-\dif}}%
\def\legendDf{\@ifnextchar[\@legendDf{\@legendDf[]}}%
\def\@legendDf[#1]#2{%
  \legendD[#1]{\posx}{\l}{\s}{\posy}{#2}%
  \FPadd\posy\posy{-\dif}}%
\def\legendEf{\@ifnextchar[\@legendEf{\@legendEf[]}}%
\def\@legendEf[#1]#2{%
  \legendE[#1]{\posx}{\l}{\s}{\posy}{#2}%
  \FPadd\posy\posy{-\dif}}%
\def\legendFf{\@ifnextchar[\@legendFf{\@legendFf[]}}%
\def\@legendFf[#1]#2{%
  \legendF[#1]{\posx}{\l}{\s}{\posy}{#2}%
  \FPadd\posy\posy{-\dif}}%
\def\legendGf{\@ifnextchar[\@legendGf{\@legendGf[]}}%
\def\@legendGf[#1]#2{%
  \legendG[#1]{\posx}{\l}{\s}{\posy}{#2}%
  \FPadd\posy\posy{-\dif}}%
%    \end{macrocode}
% \end{macro}
%
% \begin{macro}{\legendText}
% A legend without a line: to include some text in the legend box
%    \begin{macrocode}
\def\legendText{\@ifnextchar[\@legendText{\@legendText[]}}%
\def\@legendText[#1]#2{%
  \rput(\posx,\posy){%
    \makebox[0 cm][l]{%
    \hspace{-0.1cm}{\footnotesize #2}}}%
  \FPadd\posy\posy{-\dif}}%
%    \end{macrocode}
% \end{macro}
%
% Keys to be used in the legend environment:
%    \begin{macrocode}
\define@key[psset]{whiteBG}{fillcolorWhiteBG}{\def\fillcolorWhiteBG{#1}}%
\define@key[psset]{whiteBG}{fillstyleWhiteBG}{\def\fillstyleWhiteBG{#1}}%
\define@key[psset]{whiteBG}{framearcWhiteBG}{\def\framearcWhiteBG{#1}}%
\define@key[psset]{whiteBG}{linewidthWhiteBG}{\def\linewidthWhiteBG{#1}}%
\define@key[psset]{whiteBG}{linestyleWhiteBG}{\def\linestyleWhiteBG{#1}}%
\define@key[psset]{whiteBG}{linecolorWhiteBG}{\def\linecolorWhiteBG{#1}}%
\pst@addfams{whiteBG}%
%    \end{macrocode}
% Default values for the above keys:
%    \begin{macrocode}
\psset{fillcolorWhiteBG=white, fillstyleWhiteBG=solid,
   framearcWhiteBG=0.3, linewidthWhiteBG=0.01, 
   linestyleWhiteBG=solid, linecolorWhiteBG=black}%
%    \end{macrocode}
%
% \begin{macro}{\whiteBG}
% Create the box of the legend in the given position
%    \begin{macrocode}
\def\whiteBG{\@ifnextchar[\@whiteBG{\@whiteBG[]}}%
\def\@whiteBG[#1]#2#3#4#5{%
  \psframe[framearc=\framearcWhiteBG, fillcolor=\fillcolorWhiteBG,%
    fillstyle=\fillstyleWhiteBG, linewidth=\linewidthWhiteBG cm,%
    linestyle=\linestyleWhiteBG, linecolor=\linecolorWhiteBG, #1]%
    (#2,#3)(#4,#5)}%
%    \end{macrocode}
% \end{macro}
%
% Keys to control the line length and separation with the text
% in the legend:
%    \begin{macrocode}
\define@key[psset]{legendXY}{legendSep}{\def\sepYlegendXY{#1}} % 
\define@key[psset]{legendXY}{lineLength}{\def\llegendXY{#1}} % 
\define@key[psset]{legendXY}{lineTextSep}{\def\slegendXY{#1}} % 
\define@key[psset]{legendXY}{borderlineSep}{\FPmul\sepTxlegendXY{#1}{2}} % 
\pst@addfams{legendXY}%
%    \end{macrocode}
%
% Default values:
%    \begin{macrocode}
\psset{legendSep=4, lineLength=5, lineTextSep=1, borderlineSep=1}%
%    \end{macrocode}
%
% \begin{macro}{\legendXY}
% Plot a legend box and calculate the parameters of the legend
% in the position given in round brackets, for |#4| different legends
% and with a width of |#5|.
%    \begin{macrocode}
\def\legendXY{\@ifnextchar[\@legendXY{\@legendXY[]}}%
\def\@legendXY[#1](#2,#3)#4#5{%
  \psset{#1}%
  \def\sepY{\sepYlegendXY}%
  \def\l{\llegendXY}%
  \def\s{\slegendXY}%
  \def\sepTx{\sepTxlegendXY}%
  \def\xaXY{#2}%
  \def\yzXY{#3}%
  \FPdiv\sep\sepY{8}% 
  \FPdiv\sepYy\sepY{1.33} % 
  \FPdiv\x\sep\unitX% To separate \sep mm the legend from the axe
  \FPdiv\y\sep\unitY% To separate \sep mm the legend from the axe
  \FPadd\leftX\xaXY{\x}%
  \FPadd\highY\yzXY{-\y}%
  \FPdiv\w{#4}\unitX%
  \FPadd\rightX\leftX{\w}%
  \FPdiv\sepTx\sepTx\unitX%
  \FPadd\posx\leftX\sepTx%
  \FPdiv\sepTy\sepYy\unitY%
  \FPadd\posy\highY{-\sepTy}%
  \FPdiv\dif{\sepY}\unitY% Separation of the legends
  \FPadd\n{#5}{-1}%
  \FPmul\lowY\dif\n%
  \FPadd\lowY\lowY\sepTy%
  \FPadd\lowY\posy{-\lowY}%
  \whiteBG[#1]{\leftX}{\lowY}{\rightX}{\highY}}%
%    \end{macrocode}
% \end{macro}
%
% \begin{macro}{\legend[UL,UR,DL,DR]}
% Put the legend box in the positions upper-left, upper-right,
% down-left and down-right. The input parameters are the optional keys,
% the number of legends and the width.
%    \begin{macrocode}
\def\legendUL{\@ifnextchar[\@legendUL{\@legendUL[]}}%
\def\@legendUL[#1]#2#3{%
  \FPmul\xUL\MP@xa{1}%
  \FPmul\yUL\MP@yz{1}%
  \legendXY[#1](\xUL,\yUL){#2}{#3}}%
  %
  \def\legendUR{\@ifnextchar[\@legendUR{\@legendUR[]}}%
  \def\@legendUR[#1]#2#3{%
  \FPmul\xUR\MP@xz{1}%
  \FPmul\yUR\MP@yz{1}%
  \def\sep{0.5}%
  \FPdiv\xp\sep\unitX% To separate \sep mm the legend from the axe
  \FPmul\xp\xp{2}%
  \FPadd\xp\xUR{-\xp}%
  \FPdiv\xx{#2}\unitX%
  \FPadd\xp\xp{-\xx}%
  \legendXY[#1](\xp,\yUR){#2}{#3}}%
  %
\def\legendDL{\@ifnextchar[\@legendDL{\@legendDL[]}}%
\def\@legendDL[#1]#2#3{%
  \psset{#1}%
  \def\sepY{\sepYlegendXY}%
  \def\sepTx{\sepTxlegendXY}%
  \FPmul\xzDL\MP@xz{1}%
  \FPmul\xaDL\MP@xa{1}%
  \FPmul\yzDL\MP@yz{1}%
  \FPmul\yaDL\MP@ya{1}%
  \FPdiv\sep\sepY{8}%
  \FPdiv\sep\sepY{8}%
  \FPdiv\sepYy\sepY{1.33} % 
  \FPdiv\x\sep\unitX% To separate \sep mm the legend from the axe
  \FPdiv\y\sep\unitY% To separate \sep mm the legend from the axe
  \FPadd\leftX\xaDL{0}%
  \FPdiv\w{#2}\unitX%
  \FPadd\rightX\leftX{\w}%
  \FPdiv\sepTx\sepTx\unitX%
  \FPadd\posx\leftX\sepTx%
  \FPadd\lowY\yaDL{\y}%
  \FPadd\lowY\lowY{\y}%
  \FPdiv\sepTy\sepYy\unitY%
  \FPadd\posy\lowY{\sepTy}%
  \FPdiv\dif\sepY\unitY% Separation of the legends
  \FPadd\n{#3}{-1}%
  \FPmul\highY\dif\n%
  \FPadd\highY\highY\sepTy%
  \FPadd\highY\posy{\highY}%
  \FPadd\posy\highY{-\sepTy}%
  \legendXY[#1](\leftX,\highY){#2}{#3}}%
  %
\def\legendDR{\@ifnextchar[\@legendDR{\@legendDR[]}}%
\def\@legendDR[#1]#2#3{%
  \psset{#1}%
  \def\sepY{\sepYlegendXY}%
  \def\sepTx{\sepTxlegendXY}%
  \FPmul\xzDR\MP@xz{1}%
  \FPmul\xaDR\MP@xa{1}%
  \FPmul\yzDR\MP@yz{1}%
  \FPmul\yaDR\MP@ya{1}%
  \FPdiv\sep\sepY{8}%
  \FPdiv\sepYy\sepY{1.33} % 
  \FPdiv\x\sep\unitX% To separate \sep mm the legend from the axe
  \FPdiv\y\sep\unitY% To separate \sep mm the legend from the axe
  \FPadd\rightX\xzDR{-\x}%
  \FPdiv\w{#2}\unitX%
  \FPadd\leftX\rightX{-\w}%
  \FPadd\leftX\leftX{-\x}%
  \FPdiv\sepTx\sepTx\unitX%
  \FPadd\posx\leftX\sepTx%
  \FPadd\lowY\yaDR{\y}%
  \FPadd\lowY\lowY{\y}%
  \FPdiv\sepTy\sepYy\unitY%
  \FPadd\posy\lowY{\sepTy}%
  \FPdiv\dif{\sepY}\unitY% Separation of the legends
  \FPadd\n{#3}{-1}%
  \FPmul\highY\dif\n%
  \FPadd\highY\highY\sepTy%
  \FPadd\highY\posy{\highY}%
  \FPadd\posy\highY{-\sepTy}%
  \legendXY[#1](\leftX,\highY){#2}{#3}}%
%    \end{macrocode}
% \end{macro}
%
% This is the end
%    \begin{macrocode}
\makeatother%
%</makeplot>
%    \end{macrocode}
% \clearpage
% \PrintIndex
% \Finale
\endinput