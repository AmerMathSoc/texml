\ProvidesFile{blogexec.tex}[2012/12/20 documenting blogexec.sty]
\title{{\huge\pkg{blogexec.sty}}\\---\\Overcoming 
       \pkg{blog.sty}'s Pure Expansion\thanks{This
       document describes version
       \textcolor{blue}{\UseVersionOf{\jobname.sty}}
       of \textsf{\jobname.sty} as of \UseDateOf{\jobname.sty}.}}
% \listfiles
{ \RequirePackage{makedoc} 
  \ProcessLineMessage{} %%% {\message{.}} 
  \MakeJobDoc{17} 
  {\SectionLevelThreeParseInput}  }
\documentclass[fleqn]{article}%% TODO paper dimensions!?
\ProvidesFile{makedoc.cfg}[{2013/03/25 documentation settings}] 
%%
\author{Uwe L\"uck\thanks{%
        \url{http://contact-ednotes.sty.de.vu}}}
%%
%% 'hyperref':
\RequirePackage{ifpdf}
\usepackage[%
  \ifpdf
%     bookmarks=false,                  %% 2010/12/22
%     bookmarksnumbered,
    bookmarksopen,                      %% 2011/01/24!?
    bookmarksopenlevel=2,               %% 2011/01/23
%     pdfpagemode=UseNone,
%     pdfstartpage=10,
    pdfstartview=FitH,                  %% 2012/11/26 again
%     pdfstartview=0 0 100,             %% 2011/08/22
%     pdfstartview={XYZ null null 1},   %% 2011/08/25
%     pdfstartview={XYZ null null null},%% 2011/08/25
%     pdfstartview={XYZ null null .5},    %% 2011/08/26
%     pdffitwindow=true,          %% 2011/08/22
    citebordercolor={ .6 1    .6},
    filebordercolor={1    .6 1},
    linkbordercolor={1    .9  .7},
     urlbordercolor={ .7 1   1},   %% playing 2011/01/24
  \else
    draft
  \fi
]{hyperref}
\hypersetup{% 
    pdfauthor={Uwe L\374ck}% 
}
%% metadata, |\MDkeywords{<text>}|, |\MDkeywordsstring|:
%% %% 2011/08/22:
\makeatletter
  \newcommand*{\MDkeywords}[1]{%
    \gdef\MDkeywordsstring{#1}%
    \hypersetup{pdfkeywords=\MDkeywordsstring}%% TODO!?
  }
  \@onlypreamble\MDkeywords
%% |\MDaddtoabstract{<par-head>}|, `:' added:
  \newcommand*{\MDaddtoabstract}[1]{%           %% 2012/05/10
    \par\smallskip\noindent
    \strong{#1:}\quad\ignorespaces}
%% \pagebreak[2]
%% |\printMDkeywords|:
  \newcommand*{\printMDkeywords}{%
    \MDaddtoabstract{Keywords}%
    \MDkeywordsstring 
%     \global\let\MDkeywordsstring\relax    %% `%' 2012/11/12
  }
%% The previous definitions mainly are useful with a variant 
%% |\begin{MDabstract}| of \LaTeX's `{abstract}' environment:
  \newenvironment{MDabstract}
                 {\abstract\noindent
                  \hspace{1sp}%% for niceverb
                  \ignorespaces}
                 {\@ifundefined{MDkeywordsstring}%
                               {}%
                               {\printMDkeywords}%
                  \global\let\MDabstract\relax    %% 2012/11/12
                  \global\let\endMDabstract\relax %% 2012/11/12
                  \endabstract}
%% |\[MD]docnewline| 2012/11/12 from `readprov.tex':
  \newcommand*{\MDdocnewline}{\leavevmode\@normalcr[\topsep]}
%% <- `\leavevmode' for use with `\paragraph'.
%%    Sometimes needs to be preceded by a space.
%% 
%% |\MDfinaldatechecks[<tex-script>]| with \ctanpkgref{filedate}:
  \newcommand*{\MDfinaldatechecks}[1][fdatechk]{%
    \AtEndDocument{%
%       \clearpage %% 2013/03/25 no avail -- with `filedate'!
      \def\@pkgextension{sty}%
      \def\NeedsTeXFormat##1[##2]{}%
      \noNiceVerb                       %% 2013/03/22
      \input{#1}%
    }}
  \@onlypreamble\MDfinaldatechecks
\makeatother
%% Use other packages:
\RequirePackage{niceverb}[2011/01/24] 
\RequirePackage{readprov}               %% 2010/12/08
\RequirePackage{hypertoc}               %% 2011/01/23
\RequirePackage{texlinks}               %% 2011/01/24
\RequirePackage{relsize}                %% 2011/06/27
\RequirePackage{color}                  %% 2011/08/06
\RequirePackage{lmodern}                %% 2012/10/29
\RequirePackage{filedate}               %% 2012/11/12
\RequirePackage{filesdo}                %% 2013/03/22 
%% \pagebreak[3]
%% Logical markup:\qquad  |\strong{<chars>}|, |\meta{<chars>}|, 
%% |\acro{<chars>}|, |\pkg{<chars>}|, 
%% |\code{<chars>}|, |\file{<chars>}|:{\sloppy\par}
\makeatletter
  \def\do#1#2{\@ifdefinable#1{\let#1#2}}%% 2012/07/13
  \do\strong\textbf \do\file\texttt \do\acro\textsmaller 
  %% <- wrong tests before 2012/07/13
  \do\meta\textit   \do \pkg\textsf \do\code\texttt
  \ifpdf
    \pdfstringdefDisableCommands{%
        \let\acro\textrm 
        \let\file\textrm                            %% 2011/11/09
        \let\code\textrm                            %% 2011/11/20
        \let\pkg \textrm                            %% 2012/03/23
    }
  \fi
  %% TODO 2011/07/22 -> `htlogml.sty'
\makeatother
%% |\qtdcode{<text>}|: 2012/10/24:
    \newcommand*{\qtdcode}[1]{`\code{#1}'} 
%% |\pkgtitle{<package-name>}{<caption>}| 
\newcommand*{\pkgtitle}[2]{%            %% 2012/07/13
    \global\let\pkgtitle\relax
    \pkg{\huge #1}\\---\\#2\thanks{This 
       document describes version 
       \textcolor{blue}{\UseVersionOf{\jobname.sty}} 
       of \textsf{\jobname.sty} as of \UseDateOf{\jobname.sty}.}}
%% TODO: %% |\TODO| bad with `mdoccorr.cfg'
\newcommand*{\TODO}{\textcolor{blue}{\acro{TODO}}}  %% 2012/11/06
%% `\MDsampleinput[{<file>}' was added 2012/11/06. 
%% Problems with `myfilist.tex' were due to 'parskip.sty'
%% there. On 2012/11/12, we change the former simple macro to a 
%% much more complex
%% |\MDsamplecodeinput[<add-hfuss>]{<file>}| 
\newcommand*{\MDsamplecodeinput}[2][]{%
    \begingroup
        \vskip\bigskipamount \hrule
        \nobreak\vskip-\parskip 
%         \nobreak\vskip\medskipamount
%% Previous mistake (same below) due to manual change 
%% of `\topsep' in the file `myfilist.tex' (2012/11/30).
        \ifx\\#1\\\else
            \hfuzz=\textwidth \advance\hfuzz#1\relax
        \fi
        \noNiceVerb \verbatiminput{#2}%
%         \nobreak\vskip\medskipamount 
        \hrule \vskip-\parskip 
        \bigskip %%% \bigbreak
%% `\bigbreak' made much larger space in `myfilist.tex'.
    \endgroup
}
%% |\ctanpkgdref{<pkg-id>}| adds the printed link to 
%% `ctan.org/pkg' as a footnote. There is a little space 
%% for coloured link borders:
\newcommand*{\ctanpkgdref}[1]{%
    \ctanpkgref{#1}\,\urlfoot{CtanPkgRef}{#1}}
\errorcontextlines=4
\pagestyle{headings}

\endinput 
 %% shared formatting settings
\usepackage{filesdo} \MDfinaldatechecks     %% 2012/12/20
\providecommand*{\HTML}{\acro{HTML}}
\ReadPackageInfos{blogexec}
\sloppy
\begin{document}
\maketitle
\begin{abstract}\noindent
'blog.sty' before v0.7 generated \HTML\ by pure macro expansion 
and could use \LaTeX\ macros (redefined) only to a very limited 
extent. On adding 'blogexec.sty', some macros (configurable) are 
``intercepted" before expansion in a 'blog' run for 
``running" some code, basically `\begin', `\end', 
and a new general `\EXECUTE'. 
A table environment with active characters inside only is 
provided---perhaps ``nicer than \LaTeX."
\end{abstract}
\tableofcontents

%   \newpage
\section{Features and Usage}

% \subsection{Installing and Calling}
The file 'blogexec.sty' is provided ready, installation only requires
putting it somewhere where \TeX\ finds it
(which may need updating the filename data
 base).\urlfoot{ukfaqref}{inst-wlcf}           %% corr. 2011/02/08

'blogexec.sty' may be loaded by
\begin{verbatim}
  \RequirePackage{blogexec}
\end{verbatim}
in a driver file for 'blog.sty'. \strong{Alternatively}, 
the following commands in a 'blog' driver file 
(in a certain way even in a \emph{source} file)
load 'blogexec.sty' and then are carried out according to their 
definitions in 'blogexec':
\begin{description}
  \cmdboxitem|\BlogInterceptExecute| \ intercepts |\EXECUTE| only. 
  \cmdboxitem|\BlogInterceptEnvironments| \ intercepts |\EXECUTE|,
    |\begin|, and |\end| only; the latter two then work much 
    as with \LaTeX. They expand to \HTML\ code as with 'blog'; 
    `\begin{<env>}' additionally executes commands according 
    to an (optional) 
    \[|\MakeBlogBeginRun{<env>}<arguments>{<begin-code>}|.\]
  \cmdboxitem|\BlogInterceptExtra| \ intercepts all the commands 
    in certain lists (using the \ctanpkgref{dowith} package), 
    including `\EXECUTE', `\begin', `\end'. E.g., 
    \[|\MakeBlogOneArgInterception{<cmd>}{<run>}{<write>}|\]
    adds <cmd> to such a list and tells that <run> should be 
    carried out and that <cmd><one-argument> should be replaced 
    by <write> in a line containing <cmd> 
    (not hidden in braces, and there better should not be much more 
     in the line).
  \cmdboxitem|\BlogInterceptHash| \ does \emph{not} choose an 
    \strong{``interception level"} as the previous commands do, 
    but may be necessary for allowing parameters in macro 
    definitions to be run in the course of an interception.
    It is automatically (``implicitly") envoked by the star forms 
    of the above commands, i.e., by either of
    \begin{itemize}
      \item |\BlogInterceptExecute*|
      \item |\BlogInterceptEnvironments*|
      \item |\BlogInterceptExtra*|
    \end{itemize}
\end{description}
The reader may find additional details in the following sections 
near the code implementing the commands.

'blogexec.sty' also modifies 'blog.sty''s (v0.7) 
|{stdallrulestable}| environment as follows:
\begin{itemize}
  \item[\InlineCmdBox{\texttt{&|}}] 
    The vertical stroke becomes an active character 
    that closes a table cell and opens another one 
    (being an alias for 'blog.sty''s v0.7 `\endcell')---just
    as |&| does it with \TeX/\LaTeX. 
  \cmdboxitem|&| The ampersand becomes an active character 
    that---\strong{differently to \TeX\slash\LaTeX}---as an alias for 
    'blog.sty''s |\figurespace| produces the Unicode figure space 
    for alignment of figures. 
\end{itemize}
\emph{Outside} the `{stdallrules}' environment, both characters 
have their ``usual" meaning, i.e., |&| may be used for accessing 
\HTML\ entities (as 'blog.sty' allows it). |\cr| (and |\endline|, 
provided by 'blog.sty' v0.7) ends a table row and starts a new one. 
|\\| is not touched---a \strong{difference to \LaTeX} and may still be 
used for breaking a line within a table cell. 

% \section{Example}

%   \pagebreak
% \section{Implementation}
\section{Package File Header (Legalize)}
\ProvidesFile{blogexec.tex}[2012/12/20 documenting blogexec.sty]
\title{{\huge\pkg{blogexec.sty}}\\---\\Overcoming 
       \pkg{blog.sty}'s Pure Expansion\thanks{This
       document describes version
       \textcolor{blue}{\UseVersionOf{\jobname.sty}}
       of \textsf{\jobname.sty} as of \UseDateOf{\jobname.sty}.}}
% \listfiles
{ \RequirePackage{makedoc} 
  \ProcessLineMessage{} %%% {\message{.}} 
  \MakeJobDoc{17} 
  {\SectionLevelThreeParseInput}  }
\documentclass[fleqn]{article}%% TODO paper dimensions!?
\ProvidesFile{makedoc.cfg}[{2013/03/25 documentation settings}] 
%%
\author{Uwe L\"uck\thanks{%
        \url{http://contact-ednotes.sty.de.vu}}}
%%
%% 'hyperref':
\RequirePackage{ifpdf}
\usepackage[%
  \ifpdf
%     bookmarks=false,                  %% 2010/12/22
%     bookmarksnumbered,
    bookmarksopen,                      %% 2011/01/24!?
    bookmarksopenlevel=2,               %% 2011/01/23
%     pdfpagemode=UseNone,
%     pdfstartpage=10,
    pdfstartview=FitH,                  %% 2012/11/26 again
%     pdfstartview=0 0 100,             %% 2011/08/22
%     pdfstartview={XYZ null null 1},   %% 2011/08/25
%     pdfstartview={XYZ null null null},%% 2011/08/25
%     pdfstartview={XYZ null null .5},    %% 2011/08/26
%     pdffitwindow=true,          %% 2011/08/22
    citebordercolor={ .6 1    .6},
    filebordercolor={1    .6 1},
    linkbordercolor={1    .9  .7},
     urlbordercolor={ .7 1   1},   %% playing 2011/01/24
  \else
    draft
  \fi
]{hyperref}
\hypersetup{% 
    pdfauthor={Uwe L\374ck}% 
}
%% metadata, |\MDkeywords{<text>}|, |\MDkeywordsstring|:
%% %% 2011/08/22:
\makeatletter
  \newcommand*{\MDkeywords}[1]{%
    \gdef\MDkeywordsstring{#1}%
    \hypersetup{pdfkeywords=\MDkeywordsstring}%% TODO!?
  }
  \@onlypreamble\MDkeywords
%% |\MDaddtoabstract{<par-head>}|, `:' added:
  \newcommand*{\MDaddtoabstract}[1]{%           %% 2012/05/10
    \par\smallskip\noindent
    \strong{#1:}\quad\ignorespaces}
%% \pagebreak[2]
%% |\printMDkeywords|:
  \newcommand*{\printMDkeywords}{%
    \MDaddtoabstract{Keywords}%
    \MDkeywordsstring 
%     \global\let\MDkeywordsstring\relax    %% `%' 2012/11/12
  }
%% The previous definitions mainly are useful with a variant 
%% |\begin{MDabstract}| of \LaTeX's `{abstract}' environment:
  \newenvironment{MDabstract}
                 {\abstract\noindent
                  \hspace{1sp}%% for niceverb
                  \ignorespaces}
                 {\@ifundefined{MDkeywordsstring}%
                               {}%
                               {\printMDkeywords}%
                  \global\let\MDabstract\relax    %% 2012/11/12
                  \global\let\endMDabstract\relax %% 2012/11/12
                  \endabstract}
%% |\[MD]docnewline| 2012/11/12 from `readprov.tex':
  \newcommand*{\MDdocnewline}{\leavevmode\@normalcr[\topsep]}
%% <- `\leavevmode' for use with `\paragraph'.
%%    Sometimes needs to be preceded by a space.
%% 
%% |\MDfinaldatechecks[<tex-script>]| with \ctanpkgref{filedate}:
  \newcommand*{\MDfinaldatechecks}[1][fdatechk]{%
    \AtEndDocument{%
%       \clearpage %% 2013/03/25 no avail -- with `filedate'!
      \def\@pkgextension{sty}%
      \def\NeedsTeXFormat##1[##2]{}%
      \noNiceVerb                       %% 2013/03/22
      \input{#1}%
    }}
  \@onlypreamble\MDfinaldatechecks
\makeatother
%% Use other packages:
\RequirePackage{niceverb}[2011/01/24] 
\RequirePackage{readprov}               %% 2010/12/08
\RequirePackage{hypertoc}               %% 2011/01/23
\RequirePackage{texlinks}               %% 2011/01/24
\RequirePackage{relsize}                %% 2011/06/27
\RequirePackage{color}                  %% 2011/08/06
\RequirePackage{lmodern}                %% 2012/10/29
\RequirePackage{filedate}               %% 2012/11/12
\RequirePackage{filesdo}                %% 2013/03/22 
%% \pagebreak[3]
%% Logical markup:\qquad  |\strong{<chars>}|, |\meta{<chars>}|, 
%% |\acro{<chars>}|, |\pkg{<chars>}|, 
%% |\code{<chars>}|, |\file{<chars>}|:{\sloppy\par}
\makeatletter
  \def\do#1#2{\@ifdefinable#1{\let#1#2}}%% 2012/07/13
  \do\strong\textbf \do\file\texttt \do\acro\textsmaller 
  %% <- wrong tests before 2012/07/13
  \do\meta\textit   \do \pkg\textsf \do\code\texttt
  \ifpdf
    \pdfstringdefDisableCommands{%
        \let\acro\textrm 
        \let\file\textrm                            %% 2011/11/09
        \let\code\textrm                            %% 2011/11/20
        \let\pkg \textrm                            %% 2012/03/23
    }
  \fi
  %% TODO 2011/07/22 -> `htlogml.sty'
\makeatother
%% |\qtdcode{<text>}|: 2012/10/24:
    \newcommand*{\qtdcode}[1]{`\code{#1}'} 
%% |\pkgtitle{<package-name>}{<caption>}| 
\newcommand*{\pkgtitle}[2]{%            %% 2012/07/13
    \global\let\pkgtitle\relax
    \pkg{\huge #1}\\---\\#2\thanks{This 
       document describes version 
       \textcolor{blue}{\UseVersionOf{\jobname.sty}} 
       of \textsf{\jobname.sty} as of \UseDateOf{\jobname.sty}.}}
%% TODO: %% |\TODO| bad with `mdoccorr.cfg'
\newcommand*{\TODO}{\textcolor{blue}{\acro{TODO}}}  %% 2012/11/06
%% `\MDsampleinput[{<file>}' was added 2012/11/06. 
%% Problems with `myfilist.tex' were due to 'parskip.sty'
%% there. On 2012/11/12, we change the former simple macro to a 
%% much more complex
%% |\MDsamplecodeinput[<add-hfuss>]{<file>}| 
\newcommand*{\MDsamplecodeinput}[2][]{%
    \begingroup
        \vskip\bigskipamount \hrule
        \nobreak\vskip-\parskip 
%         \nobreak\vskip\medskipamount
%% Previous mistake (same below) due to manual change 
%% of `\topsep' in the file `myfilist.tex' (2012/11/30).
        \ifx\\#1\\\else
            \hfuzz=\textwidth \advance\hfuzz#1\relax
        \fi
        \noNiceVerb \verbatiminput{#2}%
%         \nobreak\vskip\medskipamount 
        \hrule \vskip-\parskip 
        \bigskip %%% \bigbreak
%% `\bigbreak' made much larger space in `myfilist.tex'.
    \endgroup
}
%% |\ctanpkgdref{<pkg-id>}| adds the printed link to 
%% `ctan.org/pkg' as a footnote. There is a little space 
%% for coloured link borders:
\newcommand*{\ctanpkgdref}[1]{%
    \ctanpkgref{#1}\,\urlfoot{CtanPkgRef}{#1}}
\errorcontextlines=4
\pagestyle{headings}

\endinput 
 %% shared formatting settings
\usepackage{filesdo} \MDfinaldatechecks     %% 2012/12/20
\providecommand*{\HTML}{\acro{HTML}}
\ReadPackageInfos{blogexec}
\sloppy
\begin{document}
\maketitle
\begin{abstract}\noindent
'blog.sty' before v0.7 generated \HTML\ by pure macro expansion 
and could use \LaTeX\ macros (redefined) only to a very limited 
extent. On adding 'blogexec.sty', some macros (configurable) are 
``intercepted" before expansion in a 'blog' run for 
``running" some code, basically `\begin', `\end', 
and a new general `\EXECUTE'. 
A table environment with active characters inside only is 
provided---perhaps ``nicer than \LaTeX."
\end{abstract}
\tableofcontents

%   \newpage
\section{Features and Usage}

% \subsection{Installing and Calling}
The file 'blogexec.sty' is provided ready, installation only requires
putting it somewhere where \TeX\ finds it
(which may need updating the filename data
 base).\urlfoot{ukfaqref}{inst-wlcf}           %% corr. 2011/02/08

'blogexec.sty' may be loaded by
\begin{verbatim}
  \RequirePackage{blogexec}
\end{verbatim}
in a driver file for 'blog.sty'. \strong{Alternatively}, 
the following commands in a 'blog' driver file 
(in a certain way even in a \emph{source} file)
load 'blogexec.sty' and then are carried out according to their 
definitions in 'blogexec':
\begin{description}
  \cmdboxitem|\BlogInterceptExecute| \ intercepts |\EXECUTE| only. 
  \cmdboxitem|\BlogInterceptEnvironments| \ intercepts |\EXECUTE|,
    |\begin|, and |\end| only; the latter two then work much 
    as with \LaTeX. They expand to \HTML\ code as with 'blog'; 
    `\begin{<env>}' additionally executes commands according 
    to an (optional) 
    \[|\MakeBlogBeginRun{<env>}<arguments>{<begin-code>}|.\]
  \cmdboxitem|\BlogInterceptExtra| \ intercepts all the commands 
    in certain lists (using the \ctanpkgref{dowith} package), 
    including `\EXECUTE', `\begin', `\end'. E.g., 
    \[|\MakeBlogOneArgInterception{<cmd>}{<run>}{<write>}|\]
    adds <cmd> to such a list and tells that <run> should be 
    carried out and that <cmd><one-argument> should be replaced 
    by <write> in a line containing <cmd> 
    (not hidden in braces, and there better should not be much more 
     in the line).
  \cmdboxitem|\BlogInterceptHash| \ does \emph{not} choose an 
    \strong{``interception level"} as the previous commands do, 
    but may be necessary for allowing parameters in macro 
    definitions to be run in the course of an interception.
    It is automatically (``implicitly") envoked by the star forms 
    of the above commands, i.e., by either of
    \begin{itemize}
      \item |\BlogInterceptExecute*|
      \item |\BlogInterceptEnvironments*|
      \item |\BlogInterceptExtra*|
    \end{itemize}
\end{description}
The reader may find additional details in the following sections 
near the code implementing the commands.

'blogexec.sty' also modifies 'blog.sty''s (v0.7) 
|{stdallrulestable}| environment as follows:
\begin{itemize}
  \item[\InlineCmdBox{\texttt{&|}}] 
    The vertical stroke becomes an active character 
    that closes a table cell and opens another one 
    (being an alias for 'blog.sty''s v0.7 `\endcell')---just
    as |&| does it with \TeX/\LaTeX. 
  \cmdboxitem|&| The ampersand becomes an active character 
    that---\strong{differently to \TeX\slash\LaTeX}---as an alias for 
    'blog.sty''s |\figurespace| produces the Unicode figure space 
    for alignment of figures. 
\end{itemize}
\emph{Outside} the `{stdallrules}' environment, both characters 
have their ``usual" meaning, i.e., |&| may be used for accessing 
\HTML\ entities (as 'blog.sty' allows it). |\cr| (and |\endline|, 
provided by 'blog.sty' v0.7) ends a table row and starts a new one. 
|\\| is not touched---a \strong{difference to \LaTeX} and may still be 
used for breaking a line within a table cell. 

% \section{Example}

%   \pagebreak
% \section{Implementation}
\section{Package File Header (Legalize)}
\ProvidesFile{blogexec.tex}[2012/12/20 documenting blogexec.sty]
\title{{\huge\pkg{blogexec.sty}}\\---\\Overcoming 
       \pkg{blog.sty}'s Pure Expansion\thanks{This
       document describes version
       \textcolor{blue}{\UseVersionOf{\jobname.sty}}
       of \textsf{\jobname.sty} as of \UseDateOf{\jobname.sty}.}}
% \listfiles
{ \RequirePackage{makedoc} 
  \ProcessLineMessage{} %%% {\message{.}} 
  \MakeJobDoc{17} 
  {\SectionLevelThreeParseInput}  }
\documentclass[fleqn]{article}%% TODO paper dimensions!?
\ProvidesFile{makedoc.cfg}[{2013/03/25 documentation settings}] 
%%
\author{Uwe L\"uck\thanks{%
        \url{http://contact-ednotes.sty.de.vu}}}
%%
%% 'hyperref':
\RequirePackage{ifpdf}
\usepackage[%
  \ifpdf
%     bookmarks=false,                  %% 2010/12/22
%     bookmarksnumbered,
    bookmarksopen,                      %% 2011/01/24!?
    bookmarksopenlevel=2,               %% 2011/01/23
%     pdfpagemode=UseNone,
%     pdfstartpage=10,
    pdfstartview=FitH,                  %% 2012/11/26 again
%     pdfstartview=0 0 100,             %% 2011/08/22
%     pdfstartview={XYZ null null 1},   %% 2011/08/25
%     pdfstartview={XYZ null null null},%% 2011/08/25
%     pdfstartview={XYZ null null .5},    %% 2011/08/26
%     pdffitwindow=true,          %% 2011/08/22
    citebordercolor={ .6 1    .6},
    filebordercolor={1    .6 1},
    linkbordercolor={1    .9  .7},
     urlbordercolor={ .7 1   1},   %% playing 2011/01/24
  \else
    draft
  \fi
]{hyperref}
\hypersetup{% 
    pdfauthor={Uwe L\374ck}% 
}
%% metadata, |\MDkeywords{<text>}|, |\MDkeywordsstring|:
%% %% 2011/08/22:
\makeatletter
  \newcommand*{\MDkeywords}[1]{%
    \gdef\MDkeywordsstring{#1}%
    \hypersetup{pdfkeywords=\MDkeywordsstring}%% TODO!?
  }
  \@onlypreamble\MDkeywords
%% |\MDaddtoabstract{<par-head>}|, `:' added:
  \newcommand*{\MDaddtoabstract}[1]{%           %% 2012/05/10
    \par\smallskip\noindent
    \strong{#1:}\quad\ignorespaces}
%% \pagebreak[2]
%% |\printMDkeywords|:
  \newcommand*{\printMDkeywords}{%
    \MDaddtoabstract{Keywords}%
    \MDkeywordsstring 
%     \global\let\MDkeywordsstring\relax    %% `%' 2012/11/12
  }
%% The previous definitions mainly are useful with a variant 
%% |\begin{MDabstract}| of \LaTeX's `{abstract}' environment:
  \newenvironment{MDabstract}
                 {\abstract\noindent
                  \hspace{1sp}%% for niceverb
                  \ignorespaces}
                 {\@ifundefined{MDkeywordsstring}%
                               {}%
                               {\printMDkeywords}%
                  \global\let\MDabstract\relax    %% 2012/11/12
                  \global\let\endMDabstract\relax %% 2012/11/12
                  \endabstract}
%% |\[MD]docnewline| 2012/11/12 from `readprov.tex':
  \newcommand*{\MDdocnewline}{\leavevmode\@normalcr[\topsep]}
%% <- `\leavevmode' for use with `\paragraph'.
%%    Sometimes needs to be preceded by a space.
%% 
%% |\MDfinaldatechecks[<tex-script>]| with \ctanpkgref{filedate}:
  \newcommand*{\MDfinaldatechecks}[1][fdatechk]{%
    \AtEndDocument{%
%       \clearpage %% 2013/03/25 no avail -- with `filedate'!
      \def\@pkgextension{sty}%
      \def\NeedsTeXFormat##1[##2]{}%
      \noNiceVerb                       %% 2013/03/22
      \input{#1}%
    }}
  \@onlypreamble\MDfinaldatechecks
\makeatother
%% Use other packages:
\RequirePackage{niceverb}[2011/01/24] 
\RequirePackage{readprov}               %% 2010/12/08
\RequirePackage{hypertoc}               %% 2011/01/23
\RequirePackage{texlinks}               %% 2011/01/24
\RequirePackage{relsize}                %% 2011/06/27
\RequirePackage{color}                  %% 2011/08/06
\RequirePackage{lmodern}                %% 2012/10/29
\RequirePackage{filedate}               %% 2012/11/12
\RequirePackage{filesdo}                %% 2013/03/22 
%% \pagebreak[3]
%% Logical markup:\qquad  |\strong{<chars>}|, |\meta{<chars>}|, 
%% |\acro{<chars>}|, |\pkg{<chars>}|, 
%% |\code{<chars>}|, |\file{<chars>}|:{\sloppy\par}
\makeatletter
  \def\do#1#2{\@ifdefinable#1{\let#1#2}}%% 2012/07/13
  \do\strong\textbf \do\file\texttt \do\acro\textsmaller 
  %% <- wrong tests before 2012/07/13
  \do\meta\textit   \do \pkg\textsf \do\code\texttt
  \ifpdf
    \pdfstringdefDisableCommands{%
        \let\acro\textrm 
        \let\file\textrm                            %% 2011/11/09
        \let\code\textrm                            %% 2011/11/20
        \let\pkg \textrm                            %% 2012/03/23
    }
  \fi
  %% TODO 2011/07/22 -> `htlogml.sty'
\makeatother
%% |\qtdcode{<text>}|: 2012/10/24:
    \newcommand*{\qtdcode}[1]{`\code{#1}'} 
%% |\pkgtitle{<package-name>}{<caption>}| 
\newcommand*{\pkgtitle}[2]{%            %% 2012/07/13
    \global\let\pkgtitle\relax
    \pkg{\huge #1}\\---\\#2\thanks{This 
       document describes version 
       \textcolor{blue}{\UseVersionOf{\jobname.sty}} 
       of \textsf{\jobname.sty} as of \UseDateOf{\jobname.sty}.}}
%% TODO: %% |\TODO| bad with `mdoccorr.cfg'
\newcommand*{\TODO}{\textcolor{blue}{\acro{TODO}}}  %% 2012/11/06
%% `\MDsampleinput[{<file>}' was added 2012/11/06. 
%% Problems with `myfilist.tex' were due to 'parskip.sty'
%% there. On 2012/11/12, we change the former simple macro to a 
%% much more complex
%% |\MDsamplecodeinput[<add-hfuss>]{<file>}| 
\newcommand*{\MDsamplecodeinput}[2][]{%
    \begingroup
        \vskip\bigskipamount \hrule
        \nobreak\vskip-\parskip 
%         \nobreak\vskip\medskipamount
%% Previous mistake (same below) due to manual change 
%% of `\topsep' in the file `myfilist.tex' (2012/11/30).
        \ifx\\#1\\\else
            \hfuzz=\textwidth \advance\hfuzz#1\relax
        \fi
        \noNiceVerb \verbatiminput{#2}%
%         \nobreak\vskip\medskipamount 
        \hrule \vskip-\parskip 
        \bigskip %%% \bigbreak
%% `\bigbreak' made much larger space in `myfilist.tex'.
    \endgroup
}
%% |\ctanpkgdref{<pkg-id>}| adds the printed link to 
%% `ctan.org/pkg' as a footnote. There is a little space 
%% for coloured link borders:
\newcommand*{\ctanpkgdref}[1]{%
    \ctanpkgref{#1}\,\urlfoot{CtanPkgRef}{#1}}
\errorcontextlines=4
\pagestyle{headings}

\endinput 
 %% shared formatting settings
\usepackage{filesdo} \MDfinaldatechecks     %% 2012/12/20
\providecommand*{\HTML}{\acro{HTML}}
\ReadPackageInfos{blogexec}
\sloppy
\begin{document}
\maketitle
\begin{abstract}\noindent
'blog.sty' before v0.7 generated \HTML\ by pure macro expansion 
and could use \LaTeX\ macros (redefined) only to a very limited 
extent. On adding 'blogexec.sty', some macros (configurable) are 
``intercepted" before expansion in a 'blog' run for 
``running" some code, basically `\begin', `\end', 
and a new general `\EXECUTE'. 
A table environment with active characters inside only is 
provided---perhaps ``nicer than \LaTeX."
\end{abstract}
\tableofcontents

%   \newpage
\section{Features and Usage}

% \subsection{Installing and Calling}
The file 'blogexec.sty' is provided ready, installation only requires
putting it somewhere where \TeX\ finds it
(which may need updating the filename data
 base).\urlfoot{ukfaqref}{inst-wlcf}           %% corr. 2011/02/08

'blogexec.sty' may be loaded by
\begin{verbatim}
  \RequirePackage{blogexec}
\end{verbatim}
in a driver file for 'blog.sty'. \strong{Alternatively}, 
the following commands in a 'blog' driver file 
(in a certain way even in a \emph{source} file)
load 'blogexec.sty' and then are carried out according to their 
definitions in 'blogexec':
\begin{description}
  \cmdboxitem|\BlogInterceptExecute| \ intercepts |\EXECUTE| only. 
  \cmdboxitem|\BlogInterceptEnvironments| \ intercepts |\EXECUTE|,
    |\begin|, and |\end| only; the latter two then work much 
    as with \LaTeX. They expand to \HTML\ code as with 'blog'; 
    `\begin{<env>}' additionally executes commands according 
    to an (optional) 
    \[|\MakeBlogBeginRun{<env>}<arguments>{<begin-code>}|.\]
  \cmdboxitem|\BlogInterceptExtra| \ intercepts all the commands 
    in certain lists (using the \ctanpkgref{dowith} package), 
    including `\EXECUTE', `\begin', `\end'. E.g., 
    \[|\MakeBlogOneArgInterception{<cmd>}{<run>}{<write>}|\]
    adds <cmd> to such a list and tells that <run> should be 
    carried out and that <cmd><one-argument> should be replaced 
    by <write> in a line containing <cmd> 
    (not hidden in braces, and there better should not be much more 
     in the line).
  \cmdboxitem|\BlogInterceptHash| \ does \emph{not} choose an 
    \strong{``interception level"} as the previous commands do, 
    but may be necessary for allowing parameters in macro 
    definitions to be run in the course of an interception.
    It is automatically (``implicitly") envoked by the star forms 
    of the above commands, i.e., by either of
    \begin{itemize}
      \item |\BlogInterceptExecute*|
      \item |\BlogInterceptEnvironments*|
      \item |\BlogInterceptExtra*|
    \end{itemize}
\end{description}
The reader may find additional details in the following sections 
near the code implementing the commands.

'blogexec.sty' also modifies 'blog.sty''s (v0.7) 
|{stdallrulestable}| environment as follows:
\begin{itemize}
  \item[\InlineCmdBox{\texttt{&|}}] 
    The vertical stroke becomes an active character 
    that closes a table cell and opens another one 
    (being an alias for 'blog.sty''s v0.7 `\endcell')---just
    as |&| does it with \TeX/\LaTeX. 
  \cmdboxitem|&| The ampersand becomes an active character 
    that---\strong{differently to \TeX\slash\LaTeX}---as an alias for 
    'blog.sty''s |\figurespace| produces the Unicode figure space 
    for alignment of figures. 
\end{itemize}
\emph{Outside} the `{stdallrules}' environment, both characters 
have their ``usual" meaning, i.e., |&| may be used for accessing 
\HTML\ entities (as 'blog.sty' allows it). |\cr| (and |\endline|, 
provided by 'blog.sty' v0.7) ends a table row and starts a new one. 
|\\| is not touched---a \strong{difference to \LaTeX} and may still be 
used for breaking a line within a table cell. 

% \section{Example}

%   \pagebreak
% \section{Implementation}
\section{Package File Header (Legalize)}
\ProvidesFile{blogexec.tex}[2012/12/20 documenting blogexec.sty]
\title{{\huge\pkg{blogexec.sty}}\\---\\Overcoming 
       \pkg{blog.sty}'s Pure Expansion\thanks{This
       document describes version
       \textcolor{blue}{\UseVersionOf{\jobname.sty}}
       of \textsf{\jobname.sty} as of \UseDateOf{\jobname.sty}.}}
% \listfiles
{ \RequirePackage{makedoc} 
  \ProcessLineMessage{} %%% {\message{.}} 
  \MakeJobDoc{17} 
  {\SectionLevelThreeParseInput}  }
\documentclass[fleqn]{article}%% TODO paper dimensions!?
\input{makedoc.cfg} %% shared formatting settings
\usepackage{filesdo} \MDfinaldatechecks     %% 2012/12/20
\providecommand*{\HTML}{\acro{HTML}}
\ReadPackageInfos{blogexec}
\sloppy
\begin{document}
\maketitle
\begin{abstract}\noindent
'blog.sty' before v0.7 generated \HTML\ by pure macro expansion 
and could use \LaTeX\ macros (redefined) only to a very limited 
extent. On adding 'blogexec.sty', some macros (configurable) are 
``intercepted" before expansion in a 'blog' run for 
``running" some code, basically `\begin', `\end', 
and a new general `\EXECUTE'. 
A table environment with active characters inside only is 
provided---perhaps ``nicer than \LaTeX."
\end{abstract}
\tableofcontents

%   \newpage
\section{Features and Usage}

% \subsection{Installing and Calling}
The file 'blogexec.sty' is provided ready, installation only requires
putting it somewhere where \TeX\ finds it
(which may need updating the filename data
 base).\urlfoot{ukfaqref}{inst-wlcf}           %% corr. 2011/02/08

'blogexec.sty' may be loaded by
\begin{verbatim}
  \RequirePackage{blogexec}
\end{verbatim}
in a driver file for 'blog.sty'. \strong{Alternatively}, 
the following commands in a 'blog' driver file 
(in a certain way even in a \emph{source} file)
load 'blogexec.sty' and then are carried out according to their 
definitions in 'blogexec':
\begin{description}
  \cmdboxitem|\BlogInterceptExecute| \ intercepts |\EXECUTE| only. 
  \cmdboxitem|\BlogInterceptEnvironments| \ intercepts |\EXECUTE|,
    |\begin|, and |\end| only; the latter two then work much 
    as with \LaTeX. They expand to \HTML\ code as with 'blog'; 
    `\begin{<env>}' additionally executes commands according 
    to an (optional) 
    \[|\MakeBlogBeginRun{<env>}<arguments>{<begin-code>}|.\]
  \cmdboxitem|\BlogInterceptExtra| \ intercepts all the commands 
    in certain lists (using the \ctanpkgref{dowith} package), 
    including `\EXECUTE', `\begin', `\end'. E.g., 
    \[|\MakeBlogOneArgInterception{<cmd>}{<run>}{<write>}|\]
    adds <cmd> to such a list and tells that <run> should be 
    carried out and that <cmd><one-argument> should be replaced 
    by <write> in a line containing <cmd> 
    (not hidden in braces, and there better should not be much more 
     in the line).
  \cmdboxitem|\BlogInterceptHash| \ does \emph{not} choose an 
    \strong{``interception level"} as the previous commands do, 
    but may be necessary for allowing parameters in macro 
    definitions to be run in the course of an interception.
    It is automatically (``implicitly") envoked by the star forms 
    of the above commands, i.e., by either of
    \begin{itemize}
      \item |\BlogInterceptExecute*|
      \item |\BlogInterceptEnvironments*|
      \item |\BlogInterceptExtra*|
    \end{itemize}
\end{description}
The reader may find additional details in the following sections 
near the code implementing the commands.

'blogexec.sty' also modifies 'blog.sty''s (v0.7) 
|{stdallrulestable}| environment as follows:
\begin{itemize}
  \item[\InlineCmdBox{\texttt{&|}}] 
    The vertical stroke becomes an active character 
    that closes a table cell and opens another one 
    (being an alias for 'blog.sty''s v0.7 `\endcell')---just
    as |&| does it with \TeX/\LaTeX. 
  \cmdboxitem|&| The ampersand becomes an active character 
    that---\strong{differently to \TeX\slash\LaTeX}---as an alias for 
    'blog.sty''s |\figurespace| produces the Unicode figure space 
    for alignment of figures. 
\end{itemize}
\emph{Outside} the `{stdallrules}' environment, both characters 
have their ``usual" meaning, i.e., |&| may be used for accessing 
\HTML\ entities (as 'blog.sty' allows it). |\cr| (and |\endline|, 
provided by 'blog.sty' v0.7) ends a table row and starts a new one. 
|\\| is not touched---a \strong{difference to \LaTeX} and may still be 
used for breaking a line within a table cell. 

% \section{Example}

%   \pagebreak
% \section{Implementation}
\section{Package File Header (Legalize)}
\input{blogexec.doc}
\end{document}

VERSION HISTORY

2011/11/04  for v0.1    very first
2011/11/09f.            something more
2011/11/20              star forms
2012/12/20  for v0.21   date checks



\end{document}

VERSION HISTORY

2011/11/04  for v0.1    very first
2011/11/09f.            something more
2011/11/20              star forms
2012/12/20  for v0.21   date checks



\end{document}

VERSION HISTORY

2011/11/04  for v0.1    very first
2011/11/09f.            something more
2011/11/20              star forms
2012/12/20  for v0.21   date checks



\end{document}

VERSION HISTORY

2011/11/04  for v0.1    very first
2011/11/09f.            something more
2011/11/20              star forms
2012/12/20  for v0.21   date checks


