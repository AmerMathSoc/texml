\def\exposedate{2011/09/04}
\ProvidesFile{expose.tex}[\exposedate]
\documentclass[a4paper,11pt]{article}
\usepackage{german}
\newcommand*{\DQ}[1]{"`#1"'}
\usepackage{parskip}
\parskip=.84\parskip
\usepackage[atari]{umlaute}
\ProvidesFile{makedoc.cfg}[{2013/03/25 documentation settings}] 
%%
\author{Uwe L\"uck\thanks{%
        \url{http://contact-ednotes.sty.de.vu}}}
%%
%% 'hyperref':
\RequirePackage{ifpdf}
\usepackage[%
  \ifpdf
%     bookmarks=false,                  %% 2010/12/22
%     bookmarksnumbered,
    bookmarksopen,                      %% 2011/01/24!?
    bookmarksopenlevel=2,               %% 2011/01/23
%     pdfpagemode=UseNone,
%     pdfstartpage=10,
    pdfstartview=FitH,                  %% 2012/11/26 again
%     pdfstartview=0 0 100,             %% 2011/08/22
%     pdfstartview={XYZ null null 1},   %% 2011/08/25
%     pdfstartview={XYZ null null null},%% 2011/08/25
%     pdfstartview={XYZ null null .5},    %% 2011/08/26
%     pdffitwindow=true,          %% 2011/08/22
    citebordercolor={ .6 1    .6},
    filebordercolor={1    .6 1},
    linkbordercolor={1    .9  .7},
     urlbordercolor={ .7 1   1},   %% playing 2011/01/24
  \else
    draft
  \fi
]{hyperref}
\hypersetup{% 
    pdfauthor={Uwe L\374ck}% 
}
%% metadata, |\MDkeywords{<text>}|, |\MDkeywordsstring|:
%% %% 2011/08/22:
\makeatletter
  \newcommand*{\MDkeywords}[1]{%
    \gdef\MDkeywordsstring{#1}%
    \hypersetup{pdfkeywords=\MDkeywordsstring}%% TODO!?
  }
  \@onlypreamble\MDkeywords
%% |\MDaddtoabstract{<par-head>}|, `:' added:
  \newcommand*{\MDaddtoabstract}[1]{%           %% 2012/05/10
    \par\smallskip\noindent
    \strong{#1:}\quad\ignorespaces}
%% \pagebreak[2]
%% |\printMDkeywords|:
  \newcommand*{\printMDkeywords}{%
    \MDaddtoabstract{Keywords}%
    \MDkeywordsstring 
%     \global\let\MDkeywordsstring\relax    %% `%' 2012/11/12
  }
%% The previous definitions mainly are useful with a variant 
%% |\begin{MDabstract}| of \LaTeX's `{abstract}' environment:
  \newenvironment{MDabstract}
                 {\abstract\noindent
                  \hspace{1sp}%% for niceverb
                  \ignorespaces}
                 {\@ifundefined{MDkeywordsstring}%
                               {}%
                               {\printMDkeywords}%
                  \global\let\MDabstract\relax    %% 2012/11/12
                  \global\let\endMDabstract\relax %% 2012/11/12
                  \endabstract}
%% |\[MD]docnewline| 2012/11/12 from `readprov.tex':
  \newcommand*{\MDdocnewline}{\leavevmode\@normalcr[\topsep]}
%% <- `\leavevmode' for use with `\paragraph'.
%%    Sometimes needs to be preceded by a space.
%% 
%% |\MDfinaldatechecks[<tex-script>]| with \ctanpkgref{filedate}:
  \newcommand*{\MDfinaldatechecks}[1][fdatechk]{%
    \AtEndDocument{%
%       \clearpage %% 2013/03/25 no avail -- with `filedate'!
      \def\@pkgextension{sty}%
      \def\NeedsTeXFormat##1[##2]{}%
      \noNiceVerb                       %% 2013/03/22
      \input{#1}%
    }}
  \@onlypreamble\MDfinaldatechecks
\makeatother
%% Use other packages:
\RequirePackage{niceverb}[2011/01/24] 
\RequirePackage{readprov}               %% 2010/12/08
\RequirePackage{hypertoc}               %% 2011/01/23
\RequirePackage{texlinks}               %% 2011/01/24
\RequirePackage{relsize}                %% 2011/06/27
\RequirePackage{color}                  %% 2011/08/06
\RequirePackage{lmodern}                %% 2012/10/29
\RequirePackage{filedate}               %% 2012/11/12
\RequirePackage{filesdo}                %% 2013/03/22 
%% \pagebreak[3]
%% Logical markup:\qquad  |\strong{<chars>}|, |\meta{<chars>}|, 
%% |\acro{<chars>}|, |\pkg{<chars>}|, 
%% |\code{<chars>}|, |\file{<chars>}|:{\sloppy\par}
\makeatletter
  \def\do#1#2{\@ifdefinable#1{\let#1#2}}%% 2012/07/13
  \do\strong\textbf \do\file\texttt \do\acro\textsmaller 
  %% <- wrong tests before 2012/07/13
  \do\meta\textit   \do \pkg\textsf \do\code\texttt
  \ifpdf
    \pdfstringdefDisableCommands{%
        \let\acro\textrm 
        \let\file\textrm                            %% 2011/11/09
        \let\code\textrm                            %% 2011/11/20
        \let\pkg \textrm                            %% 2012/03/23
    }
  \fi
  %% TODO 2011/07/22 -> `htlogml.sty'
\makeatother
%% |\qtdcode{<text>}|: 2012/10/24:
    \newcommand*{\qtdcode}[1]{`\code{#1}'} 
%% |\pkgtitle{<package-name>}{<caption>}| 
\newcommand*{\pkgtitle}[2]{%            %% 2012/07/13
    \global\let\pkgtitle\relax
    \pkg{\huge #1}\\---\\#2\thanks{This 
       document describes version 
       \textcolor{blue}{\UseVersionOf{\jobname.sty}} 
       of \textsf{\jobname.sty} as of \UseDateOf{\jobname.sty}.}}
%% TODO: %% |\TODO| bad with `mdoccorr.cfg'
\newcommand*{\TODO}{\textcolor{blue}{\acro{TODO}}}  %% 2012/11/06
%% `\MDsampleinput[{<file>}' was added 2012/11/06. 
%% Problems with `myfilist.tex' were due to 'parskip.sty'
%% there. On 2012/11/12, we change the former simple macro to a 
%% much more complex
%% |\MDsamplecodeinput[<add-hfuss>]{<file>}| 
\newcommand*{\MDsamplecodeinput}[2][]{%
    \begingroup
        \vskip\bigskipamount \hrule
        \nobreak\vskip-\parskip 
%         \nobreak\vskip\medskipamount
%% Previous mistake (same below) due to manual change 
%% of `\topsep' in the file `myfilist.tex' (2012/11/30).
        \ifx\\#1\\\else
            \hfuzz=\textwidth \advance\hfuzz#1\relax
        \fi
        \noNiceVerb \verbatiminput{#2}%
%         \nobreak\vskip\medskipamount 
        \hrule \vskip-\parskip 
        \bigskip %%% \bigbreak
%% `\bigbreak' made much larger space in `myfilist.tex'.
    \endgroup
}
%% |\ctanpkgdref{<pkg-id>}| adds the printed link to 
%% `ctan.org/pkg' as a footnote. There is a little space 
%% for coloured link borders:
\newcommand*{\ctanpkgdref}[1]{%
    \ctanpkgref{#1}\,\urlfoot{CtanPkgRef}{#1}}
\errorcontextlines=4
\pagestyle{headings}

\endinput 

\providecommand{\pkg}{\pkgnamefmt}
\providecommand{\code}{\texttt}
\title{Paketdokumentation und Webseitenpflege\\ 
       mit (\pkg{niceverb.sty} und) \pkg{fifinddo.sty}}
\author{\acro{DANTE}-Herbsttagung 2011\\Uwe L�ck}
\date{Entwurf \exposedate}
\begin{document}
\maketitle
\thispagestyle{empty}

Das 'nicetext'-B�ndel\urlpkgfoot{nicetext} 
versucht sich an
\wikideref{Syntaktischer Zucker}{\strong{%
           \DQ{syntaktischem Zucker}}:} 
Schriftsatz in \TeX-Qualit�t 
bei m�glichst wenig \TeX-artiger Auszeichnung. 
% Grunds�tzlich sehe ich daf�r zwei Methoden: 
Das kann durch 
(MS)~Zeichenkettenersetzung %%%, 
oder (auch) durch
(MA)~aktive Zeichen 
(die einige Fallunterscheidungen ausf�hren) %%%.
% Sie schlie�en sich keineswegs gegenseitig aus.
% Die praktische Bedeutung des B�ndels liegt f�r mich zur Zeit 
% ganz in der Paketdokumentation. 
erreicht werden.

Die \DQ{herrschende Lehre} f�r die \strong{Dokumentation} von \LaTeX-Paketen 
ist die \DQ{\file{.dtx}}-Methode, die z.\,B. eine spezielle 
Markierung des Paketcodes erfordert. 
'makedoc.sty' aus dem 'nicetext'-B�ndel dagegen 
setzt die gute alte Idee um, dass man Kommentarzeilen schon 
% dadurch dadurch von Codezeilen unterscheiden kann, 
% dass erstere mit einem Kommentarzeichen (\lq\code{\%}\rq) beginnen! 
am Kommentarzeichen (\lq`%'\rq) erkennt! 

Mit derselben Absicht verwendete Stephan B�ttcher die 
\strong{Skriptsprache} \meta{\Wikideref{awk}} 
% f�r die Dokumentation von 
zur Kommentierung von 'lineno.sty'.\urlpkgfoot{lineno}
'makedoc.sty' bietet % dem gegen�ber 
stattdessen eine Skriptsprache, % deren Interpretationsengine 
die wie bei 'docstrip.tex'\urlpkgfoot{docstrip} 
durch die \TeX-Engine selbst verarbeitet wird, 
zusammen mit besonderen Makros. 
'makedoc' wandelt die Paketdatei in eine Dokumentationsdatei um, 
indem Kommentarzeichen entfernt und Codezeilen in Listing-Umgebungen 
eingebettet werden. Bei dieser Gelegenheit kann auch gem�� Methode (MS) 
\TeX-Code f�r typografische Feinheiten eingef�gt und die 
\Wikideref{MediaWiki}-\wikideref{Hilfe:Inhaltsverzeichnis}{Gliederung} 
umgesetzt werden.

'niceverb.sty' (ebenfalls in 'nicetext') % bietet weitere Feinheiten 
erm�glicht weiter
nach der Methode (MA) 
% f�r 
eine geradezu \acro{WYS\-I\-WYG}-artige \strong{Syntaxbeschreibung} 
f�r Paketmakros. \linebreak
% Sogar Pakete anderer Autoren (z.\,B. Donald Arseneau) 
% k�nnen in \TeX-Qualit�t aus der reinen ASCII-Dokumentation 
% in der Paketdatei gesetzt werden, ohne �nderungen an dieser Datei 
% vorzunehmen. 
% Bei eigenen Paketen erfordert die Kommentierung 
% wenig mehr Aufwand als Kommentarzeichen, und der Kommentarcode 
% ist so leserlich wie �berhaupt nur m�glich.
Makros k�n\-nen \DQ{im Handumdrehen} kommentiert werden. 
% (oder man schreibt zuerst den Kommentar als Plan)~\dots
Vorhandene \linebreak
\acro{ASCII}-Dokumentation kann \DQ{automatisch} 
in \TeX-Qualit�t gesetzt werden.

% \enlargethispage{1\baselineskip}

'makedoc.sty' erweitert die \TeX-Skriptsprache aus 'fifinddo.sty'
zum Verarbeiten von Text-Dateien zu Text-Dateien. 
'blog.sty' aus dem \ctanpkgref{morehype}-B�ndel 
setzt ebenfalls auf 'fifinddo' auf. 
Im Gegensatz zu 'makedoc' werden dabei die Makros der Quelldateien 
\emph{expandiert} -- in \strong{\acro{HTML}}-Tags und -Zeichen.

Der Vortrag vergleicht diese Ans�tze mit denen 
\httpref{www.webdesign-bu.de/uwe\string_lueck/heyctan.htm\#mine-related}
        {\strong{anderer}} Autoren.

\end{document}
