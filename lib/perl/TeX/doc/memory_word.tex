\documentclass{amsart}

\usepackage{web2c}

\usepackage{fullpage}

\begin{document}

\newnode{memory_word}{}{}{}

\begin{tabular}{N|Q|Q|Q|Q|l}
\CLINE
1:  & \W{\texttt{int} \rlap{($\equiv\,\mathtt{sc}$)}}
    & \texttt{integer}\\\CLINE
\SP
2:  & \W{\texttt{gr}}& \texttt{glue_ratio}\\\CLINE
\SP
3a: & \H{\X{lh} (\texttt{info}, \texttt{llink})}
    & \H{\X{rh} (\texttt{link}, \texttt{rlink})}
    & \texttt{two_halves}\\\CLINE
\SP
3b: & \X{hh.b0} (\texttt{type})
    & \X{hh.b1} (\texttt{subtype}) 
    & \H{\X{hh.rh} (\texttt{link})}
    & \texttt{two_halves}\\\CLINE
\SP
4:  & \X{qqqq.b0}
    & \X{qqqq.b1}
    & \X{qqqq.b2}
    & \X{qqqq.b3} & \texttt{four_quarters}\\
\CLINE
\end{tabular}

\bigskip

\noindent{\Large WEB2C:}

\begin{verbatim}
typedef struct {
  struct {
    unsigned short B0, B1, B2, B3;
  } u;
} fourquarters;

typedef union {
  struct {
    integer RH, LH;
  } v;

  struct { /* Make B0, B1 overlap the most significant bytes of LH.  */
    integer junk;
    short B0, B1;
  } u;
} twohalves;

typedef union {
  double gr;
  twohalves hh;
  void * ptr;
  integer cint;
  fourquarters qqqq;
} memoryword;
\end{verbatim}

\end{document}

HASH

@d next(#) == hash[#].lh { link for coalesced lists }
@d text(#) == hash[#].rh { string number for control sequence name }

EQTB

@d eq_level_field(#) == #.hh.b1
@d eq_type_field(#)  == #.hh.b0
@d equiv_field(#)    == #.hh.rh

@d eq_level(#) == eq_level_field(eqtb[#]) { level of definition }
@d eq_type(#)  == eq_type_field (eqtb[#]) { command code for equivalent }
@d equiv(#)    == equiv_field   (eqtb[#]) { equivalent value }

@d skip(#)      == equiv(skip_base    + #) { mem location of glue specification }
@d mu_skip(#)   == equiv(mu_skip_base + #) { mem location of math glue spec }
@d glue_par(#)  == equiv(glue_base    + #) { mem location of glue specification }

@d del_code(#)  == eqtb[del_code_base + #].int
@d count(#)     == eqtb[count_base    + #].int
@d int_par(#)   == eqtb[int_base      + #].int { an integer parameter }

@d dimen(#)     == eqtb[scaled_base + #].sc
@d dimen_par(#) == eqtb[dimen_base  + #].sc { a scaled quantity }

@d prim_eq_level(#) == prim_eq_level_field(eqtb[prim_eqtb_base + #]) { level of definition }
@d prim_eq_type(#)  == prim_eq_type_field (eqtb[prim_eqtb_base + #]) { command code for equivalent }
@d prim_equiv(#)    == prim_equiv_field   (eqtb[prim_eqtb_base + #]) { equivalent value }

@d eTeX_state(#)   == eqtb[eTeX_state_base + #].int { an eTeX state variable }

MEM

@d link(#)         == mem[#].hh.rh { the link field of a memory word }
@d info(#)         == mem[#].hh.lh { the info field of a memory word }

@d type(#)         == mem[#].hh.b0 { identifies what kind of node this is }
@d subtype(#)      == mem[#].hh.b1 { secondary identification in some cases }
@d width(#)        == mem[# + width_offset].sc  { width of the box, in sp }
@d depth(#)        == mem[# + depth_offset].sc  { depth of the box, in sp }
@d height(#)       == mem[# + height_offset].sc { height of the box, in sp }
@d shift_amount(#) == mem[# + 4].sc { repositioning distance, in sp }
@d glue_set(#)     == mem[# + glue_offset].gr
@d float_cost(#)   == mem[# + 1].int   { the floating_penalty to be used }
@d adjust_ptr(#)   == mem[# + 1].int
@d native_size(#)           == mem[# + 4].qqqq.b0
@d native_font(#)           == mem[# + 4].qqqq.b1
@d native_length(#)         == mem[# + 4].qqqq.b2
@d native_glyph_count(#)    == mem[# + 4].qqqq.b3
@d native_glyph_info_ptr(#) == mem[# + 5].ptr
@d pic_path_length(#) == mem[# + 4].hh.b0
@d pic_page(#)        == mem[# + 4].hh.b1
@d pic_transform1(#)  == mem[# + 5].hh.lh
@d pic_transform2(#)  == mem[# + 5].hh.rh
@d pic_transform3(#)  == mem[# + 6].hh.lh
@d pic_transform4(#)  == mem[# + 6].hh.rh
@d pic_transform5(#)  == mem[# + 7].hh.lh
@d pic_transform6(#)  == mem[# + 7].hh.rh
@d pic_pdf_box(#)     == mem[# + 8].hh.b0
@d stretch(#)        == mem[# + 2].sc { the stretchability of this glob of glue }
@d shrink(#)         == mem[# + 3].sc { the shrinkability of this glob of glue }
@d penalty(#)    == mem[# + 1].int { the added cost of breaking a list here }
@d glue_stretch(#)  == mem[# + glue_offset].sc { total stretch in an unset node }

@d if_line_field(#) == mem[# + 1].int
@d location(#)==mem[# + 2].int { DVI byte number for a movement command }
@d math_type            == link { a halfword in mem }
@d plane_and_fam_field  == font { a quarterword in mem }
@d radical_noad_size=5 { number of mem words in a radical noad }
@d fraction_noad_size=6 { number of mem words in a fraction noad }
@d small_fam(#)==(mem[#].qqqq.b0 mod "100) { fam for ``small'' delimiter }
@d small_char(#)==(mem[#].qqqq.b1 + (mem[#].qqqq.b0 div "100) * "10000) { character for ``small'' delimiter }
@d large_fam(#)==(mem[#].qqqq.b2 mod "100) { fam for ``large'' delimiter }
@d large_char(#)==(mem[#].qqqq.b3 + (mem[#].qqqq.b2 div "100) * "10000) { character for ``large'' delimiter }
@d small_plane_and_fam_field(#)==mem[#].qqqq.b0
@d small_char_field(#)==mem[#].qqqq.b1
@d large_plane_and_fam_field(#)==mem[#].qqqq.b2
@d large_char_field(#)==mem[#].qqqq.b3
@d accent_noad_size  =     5 { number of mem words in an accent noad }
@d new_hlist(#) == mem[nucleus(#)].int { the translation of an mlist }
@d u_part(#)     == mem[# + height_offset].int { pointer to <u_j> token list }
@d v_part(#)     == mem[# + depth_offset].int  { pointer to <v_j> token list }
@d extra_info(#) == info(# + list_offset) { info to remember during template }
@d align_stack_node_size = 6 { number of mem words to save alignment states }
@d span_node_size = 2 { number of mem words for a span node }
@d total_demerits(#) == mem[# + 2].int { the quantity that TeX minimizes }
@d update_active(#) == active_width[#] := active_width[#] + mem[r + #].sc

@d total_pic_node_size(#) == (pic_node_size + (pic_path_length(#) + sizeof(memory_word) - 1) div sizeof(memory_word))

@d expr_e_field(#) == mem[# + 1].int { saved expression so far }
@d expr_t_field(#) == mem[# + 2].int { saved term so far }
@d expr_n_field(#) == mem[# + 3].int { saved numerator }
@d sa_int(#) == mem[# + 2].int { an integer }
@d sa_dim(#) == mem[# + 2].sc  { a dimension (a somewhat esotheric distinction) }
@d active_short(#) == mem[# + 3].sc { shortfall of this line }
@d active_glue(#)  == mem[# + 4].sc { corresponding glue stretch or shrink }
