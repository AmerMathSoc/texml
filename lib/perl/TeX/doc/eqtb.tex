\documentclass{amsart}

\setcounter{secnumdepth}{-1}

\usepackage{web2c}

\usepackage{fullpage}

\usepackage{longtable}

\renewcommand{\T}{\relax\ifmmode\expandafter\mathtt\else\expandafter\texttt\fi}

\let\equivalent\equiv

\newcommand{\eqtb}[1]{\ensuremath{\T{eqtb}[#1]}}
\newcommand{\eqtype}[1]{\T{eq_type}[#1]}
\newcommand{\eqlevel}[1]{\T{eq_level}[#1]}
\renewcommand{\equiv}[1]{\T{equiv}[#1]}

\newcommand{\id}[1]{\T{id_lookup}(\text{\texttt{"#1"}})}

\def\<#1>{$\langle\text{#1}\rangle$}

\newenvironment{region}[1][]{%
%    \LTleft0pt
    \longtable{|l|>{\ttfamily}l|l}
\caption{#1}\\
    \1
    \endhead
    \1
    \endfoot
Region 0 & frozen_control_sequence = frozen_protection \kill
}{%
    \endlongtable
}

\title{\T{eqtb}}

\begin{document}

\maketitle

Quoting \T{tex.web} (part 17, \S\S220--221)
\begin{quotation}

The biggest and most important such table is called \T{eqtb}.  It
holds the current ``equivalents'' of things; i.e., it explains what
things mean or what their current values are, for all quantities that
are subject to the nesting structure provided by TeX's grouping
mechanism.  There are six parts to \T{eqtb}:
\begin{enumerate}
\item \T{eqtb[active_base..($\T{hash_base} - 1$)]} holds the current
equivalents of single-character control sequences.

\item
\T{eqtb[hash_base..($\T{glue_base} - 1$)]} holds the current equivalents of
multiletter control sequences.

\item
\T{eqtb[glue_base..($\T{local_base} - 1$)]} holds the current equivalents of
glue parameters like the current baselineskip.

\item
\T{eqtb[local_base..($\T{int_base} - 1$)]} holds the current equivalents of
local halfword quantities like the current box registers, the current
``catcodes,'' the current font, and a pointer to the current paragraph
shape.

\item
\T{eqtb[int_base..($\T{dimen_base} - 1$)]} holds the current equivalents of
fullword integer parameters like the current hyphenation penalty.

\item
\T{eqtb[dimen_base..eqtb_size]} holds the current equivalents of
fullword dimension parameters like the current hsize or amount of
hanging indentation.
\end{enumerate}

Note that, for example, the current amount of baselineskip glue is
determined by the setting of a particular location in region~3 of
eqtb, while the current meaning of the control sequence
`\cs{baselineskip}' (which might have been changed by \cs{def} or
\cs{let}) appears in region~2.

Each entry in \T{eqtb} is a \T{memory_word}.  Most of these words are
of type \T{two_halves}, and subdivided into three fields:
\begin{enumerate}
\item The \T{eq_level} (a quarterword) is the level of grouping at
  which this equivalent was defined. If the level is \T{level_zero},
  the equivalent has never been defined; \T{level_one} refers to the
  outer level (outside of all groups), and this level is also used for
  global definitions that never go away. Higher levels are for
  equivalents that will disappear at the end of their group.

\item The \T{eq_type} (another quarterword) specifies what kind of
  entry this is. There are many types, since each \TeX\ primitive like
  \cs{hbox}, \cs{def}, etc., has its own special code. The list of
  command codes above includes all possible settings of the
  \T{eq_type} field.

\item The \T{equiv} (a halfword) is the current equivalent value.
  This may be a font number, a pointer into \T{mem}, or a variety of
  other things.\footnote{That is, the meaning of the \T{equiv} field
  varies depending on the value of \T{equiv_type}.  This will be
  explained further below.---dmj}

\end{enumerate}
\end{quotation}

The goal of this document is to explain what this means in more
detail.

\clearpage

\section{Data Types and Structures}

\newnode{two_halves}{}{}{}

\T{two_halves} is a union type that can take one of two forms:

\begin{tabular}{N|Q|Q|Q|Q|l}
\CLINE
1: & \H{\X{lh}}
   & \H{\X{rh}}
   & \texttt{two_halves}\\\CLINE
\SP
2: & \X{b0}
   & \X{b1}
   & \H{\X{rh}}
   & \texttt{two_halves}\\\CLINE
\CLINE
\end{tabular}

That is, it contains a halfword named \T{rh} and either another
halfword named \T{lh} \emph{or} two quarter words named \T{b0} and
\T{b1}.  The interpretation of the subfields depends on how the word
is being used.

\newnode{memory_word}{}{}{}

\T{memory_word} is a union type that can take one of 5 forms:

\begin{tabular}{N|Q|Q|Q|Q|l}
\CLINE
1:   & \W{\texttt{int} \rlap{($\equivalent\,\mathtt{sc}$)}}
     & \texttt{integer}\\\CLINE
\SP
2:   & \W{\texttt{gr}}& \texttt{glue_ratio}\\\CLINE
\SP
3.1: & \H{\X{hh.lh}}
     & \H{\X{hh.rh}}
     & \texttt{two_halves}\\\CLINE
\SP
3.2: & \X{hh.b0}
     & \X{hh.b1}
     & \H{\X{hh.rh}}
     & \texttt{two_halves}\\\CLINE
\SP
4:   & \X{qqqq.b0}
     & \X{qqqq.b1}
     & \X{qqqq.b2}
     & \X{qqqq.b3} & \texttt{four_quarters}\\
\CLINE
\end{tabular}

\newnode{eqtb}{}{}{}

\T{eqtb} is an array of \T{memory_word}:

\begin{verbatim}
        eqtb: array[active_base..eqtb_size] of memory_word;
\end{verbatim}

Most words in \T{eqtb} have type~3.2, but those in regions 5 and~6
have type~1:

\begin{tabular}{N|Q|Q|Q|Q|l}
\CLINE
1:   & \W{\texttt{int} \rlap{($\equiv\,\mathtt{sc}$)}}
     & \texttt{integer}\\\CLINE
\SP
3.2: & \X{hh.b0} (\texttt{eq_type})
     & \X{hh.b1} (\texttt{eq_level}) 
     & \H{\X{hh.rh} (\texttt{equiv})}
     & \texttt{two_halves}\\\CLINE
\CLINE
\end{tabular}

\newnode{mem}{}{}{}

\T{mem} (the \emph{dynamic} memory) is also an array of
\T{memory_word}:

\begin{verbatim}
        mem:  array[mem_min..mem_max] of memory_word;
\end{verbatim}

Words in \T{mem} can have any of the 5 forms depending on how they are
being used.

\newnode{hash}{}{}{}

\T{hash} is an array of \T{two_halves}.  All words in \T{hash} have
type~1:

\begin{verbatim}
        hash: array[hash_base..undefined_control_sequence - 1] of two_halves;
\end{verbatim}

\begin{tabular}{N|Q|Q|Q|Q|l}
\CLINE
   & \H{\X{lh} (\texttt{next})}
   & \H{\X{rh} (\texttt{text})}
   & \texttt{two_halves}\\\CLINE
\CLINE
\end{tabular}

Note that there is a 1-1 correspondence between \T{hash} and region~2
of \T{eqtb}\footnote{Except there is no entry in \T{hash}
corresponding to the very last entry of region~2, which corresponds to
the permanently undefined control sequence.}.  In brief, for $c$ in
the range of \T{hash}, \T{hash[$c$]} contains the name of the control
sequence and \T{eqtb[$c$]} contains the current meaning.

\clearpage

\section{eqtb (XeTeX 3.14159265-2.6-0.999992)}

\subsection{Regions 1 and 2}

Equivalents for active characters and control sequences.

Region~1 has two subregions:
\begin{enumerate}
    \item $\T{active_base}\dots\T{single_base} - 1$: Active
      characters.  The meaning of the character with Unicode code
      point~$c$ is stored in \eqtb{\T{active_base} + c}.  For
      example, since the code point for \T{\textasciitilde}
      is~126, the meaning of the active
      character~\T{\textasciitilde} is stored in
      $\eqtb{\T{active_base} + 126} = \eqtb{127}$.

    \item $\T{single_base}\dots\T{null_cs} - 1$:
      Single-character control sequences; The meaning of \cs{~} is
      stored in $\eqtb{\T{single_base} + 126} = \eqtb{1\,114\,239}$.
\end{enumerate}
Region~1 ends with the special value \T{null_cs}, which holds the
meaning of the empty control sequence
\cs{csname}\cs{endcsname}.\footnote{Yes, you can define that.
  Really.}

Region 2 has 3 subregions:
\begin{enumerate}

\item $\T{hash_base}\dots\T{hash_base} + \T{hash_size} - 1$:
  equivalents of multi-character control sequences, the names of which
  are stored in the \emph{hash table} \T{hash}.  If $s$ is a string
  with $\operatorname{length}(s) > 1$, the meaning of~\T{\bslchar}$s$
  will be stored in \eqtb{\T{id_lookup}(s)}.\footnote{Note that since
  this is a hash table, the exact location will depend not only on
  what other control names have have been defined, but the order in
  which they were defined.}  This includes primitives as well as
  definitions made via \cs{def}, \cs{chardef}, \cs{let}, etc.

\item $\T{frozen_control_sequence} \dots \T{frozen_primitive}$:
  Immutable pointers to primitives that the \TeX\ engine needs direct
  access to.  For example, when \TeX\ starts up, we have
  $\eqtb{\T{frozen_cr}} = \eqtb{\id{cr}}$, but if the user redefines
  \cs{cr} (this is \emph{not} recommended), subsequent uses of \cs{cr}
  will refer to the user's definition, but the \TeX\ engine can still
  synthesize tokens denoting the primitive meaning of \cs{cr} by using
  the value of \eqtb{\T{frozen_cr}}.  Also included are a couple of
  special ``hidden'' primitives like \T{frozen_endv} and \T{end_write}
  that can't be accessed from user code.

\item $\T{font_id_base} \dots \T{font_id_base} + 256$:
  Frozen equivalents of user-defined fonts.

\end{enumerate}
Region~2 ends with the \T{undefined_control_sequence}, which
denotes a permanently undefined control sequence.  (That is,
$\eqtype{\T{undefined_control_sequence}} =
\T{undefined_cs}$.)

When processing a token, \TeX's internal global variable \T{cur_cs}
will be 0 for non--control-sequence tokens and a pointer to region 1
or~2 for control sequences.

Depending on the command type stored in \T{eq_type} (which will be the
value of \T{cur_cmd} when \TeX\ is processing a token), \T{equiv} (aka
\T{cur_chr}) will be either a ``command modifier'' or a pointer to
related information.

For example, after
\begin{verbatim}
    primitive("hbox", make_box, vtop_code + hmode);
\end{verbatim}
\T{cur_val} will be set equal to $\id{hbox}$ and
\begin{verbatim}
    eq_level(cur_val) = level_one
    eq_type(cur_val)  = make_box
    equiv(cur_val)    = vtop_code + hmode
    text(cur_val)     = "hbox"
\end{verbatim}
so when \TeX\ is processing the token \cs{hbox}, we will have
\begin{verbatim}
    cur_cs  = id_lookup("hbox")
    cur_cmd = make_box
    cur_chr = vtop_code + hmode
\end{verbatim}

Macro definitions create entries in region~1 with \T{eq_type} equal to
one of \T{call}, \T{long_call}, \T{outer_call} or
\T{long_outher_call}.  Assuming $\cs{globaldefs} = 0$ and $\T{p} =
\id{foo}$, after
\begin{verbatim}
    \def\foo{...}
\end{verbatim}
we will have
\begin{verbatim}
    eq_level(p) = cur_level
    eq_type(p)  = call
    equiv(p)    = def_ref
\end{verbatim}
where \T{def_ref} is a pointer to a token list, stored in \T{mem},
that will include both the parameter text and the replacement text.

A font definition
\begin{verbatim}
    \font\bar = ...
\end{verbatim}
results in (where $\T{u} = \id{bar}$)
\begin{verbatim}
    eq_level(u) = cur_level;
    eq_type(u)  = set_font;
    equiv(u)    = f;

    eqtb[font_id_base + f] := eqtb[u].
\end{verbatim}
where \T{f} is an \T{internal_font_number}, i.e., an index into the
various font-related arrays (\T{char_base}, etc.).

Incidentally, a character token is represented by
\begin{verbatim}
    cur_cs  = 0;
    cur_cmd = 1..13 (left_brace..active_char);
    chr_chr = Unicode code point.
\end{verbatim}
For example the letter 'A' would be represented internally as
\begin{verbatim}
    cur_cs  = 0;
    cur_cmd = 11 (letter);
    chr_chr = 65.
\end{verbatim}
The active character \verb+~+ would be represented internally as
\begin{verbatim}
    cur_cs  = 0;
    cur_cmd = 13 (active);
    chr_chr = 176.
\end{verbatim}

\clearpage

\begin{region}[Regions 1 and 2]
Region 1
& "0000 & \T{active_base}\\ \2
& \VC \\ \2
& "10FFFF\\ \2
& \cs{"0000} & \T{single_base} \\ \2
& \VC \\ \2
& \cs{"10FFFF}\\ \2
& null_cs      \\ \1

Region 2
& hash_base & \T{hash_base} \\ \2
& \VCa{$\T{hash_size} - 1$}\\ \2
& frozen_control_sequence = frozen_protection \\ \2
& frozen_cr             \\ \2
& frozen_end_group      \\ \2
& frozen_right          \\ \2
& frozen_fi             \\ \2
& frozen_end_template   \\ \2
& frozen_endv           \\ \2
& frozen_relax          \\ \2
& end_write             \\ \2
& frozen_dont_expand    \\ \2
& frozen_null_font     \hfill (\cs{fnt}0) & \T{font_id_base}\\ \2
& frozen_primitive     \hfill (\cs{fnt}1) \\ \2
& prim_eqtb_base       \hfill (\cs{fnt}2) & \T{prim_eqtb_base}\\ \2
& \cs{fnt}3 \\ \2
& \VC \\ \2
& \cs{fnt}256 \\ \2
& undefined_control_sequence \\ \1
\end{region}

\clearpage

\subsection{Region 3}

Region~3 contains \<glue parameter>s, \<muglue parameters> and glue
and muglue registers.  

It consists of three subregions:
\begin{enumerate}
    \item $\T{glue_base} \dots \T{skip_base} - 1$: \<glue
      parameter>s and \<muglue parameters>.

    \item $\T{skip_base} \dots \T{skip_base} + 255$:
      \cs{skip} registers.

    \item $\T{mu_skip_base} \dots \T{mu_skip_base} + 255$:
      \cs{muskip} registers.
\end{enumerate}

For these, the \T{equiv} field is a pointer to a glue
specification in the \T{mem} array.

\begin{region}[Region 3]
Region 3
& \cs{lineskip}     & \T{glue_base}\\ \2
& \cs{baselineskip} \\ \2
& \cs{parskip} \\ \2
& \cs{abovedisplayskip} \\ \2
& \cs{belowdisplayskip} \\ \2
& \cs{abovedisplayshortskip} \\ \2
& \cs{belowdisplayshortskip} \\ \2
& \cs{leftskip} \\ \2
& \cs{rightskip} \\ \2
& \cs{topskip} \\ \2
& \cs{splittopskip} \\ \2
& \cs{tabskip} \\ \2
& \cs{spaceskip} \\ \2
& \cs{xspaceskip} \\ \2
& \cs{parfillskip} \\ \2
& \cs{XeTeXlinebreakskip} \\ \2
& \cs{thinmuskip} \\ \2
& \cs{medmuskip} \\ \2
& \cs{thickmuskip} \\ \2
& \cs{skip}0        & $\T{skip_base}$ \\ \2
& \cs{skip}1\\ \2
& \VC\\ \2
& \cs{skip}255\\ \2
& \cs{muskip}0      & $\T{mu_skip_base}$ \\ \2
& \cs{muskip}1 \\ \2
& \VC \\ \2
& \cs{muskip}255 \\ \2
\1
\end{region}

\clearpage

\subsection{Region 4}

``Halfword'' quantities:
\begin{enumerate}

\item \cs{parshape}, \cs{interlinepenalties}, \cs{clubpenalties},
  \cs{widowpenalties}, \cs{displaywidowpenalties}: \T{equiv} is a
  pointer to \T{mem}, which is the start of an array of integers
  (the first word of which is nonsense?).

\item  \<token parameter>s and token registers: \T{equiv} is a
  pointer to a token list in \T{mem}?

\item box registers: \T{equiv} is a pointer to a box (node list)
  in \T{mem}

\item fonts: \T{equiv} is a pointer into \T{font_id_base}?

\item \cs{catcode}s. \cs{lccode}s, \cs{uccode}s, \cs{sfcode}s, and
  \cs{mathcode}s: \T{equiv} is the code (a halfword integer).

\end{enumerate}

\begin{region}[Region 4]
Region 4
& \cs{parshape}             & \T{local_base} \\ \2
& \cs{output}          \\ \2
& \cs{everypar}        \\ \2
& \cs{everymath}       \\ \2
& \cs{everydisplay}    \\ \2
& \cs{everyhbox}       \\ \2
& \cs{everyvbox}       \\ \2
& \cs{everyjob}        \\ \2
& \cs{everycr}         \\ \2
& \cs{errhelp}         \\ \2
& \cs{everyeof}             & \T{etex_toks_base} \\ \2
& \cs{XeTeXinterchartoks} \\ \2
& \cs{toks}0                & \T{toks_base} \\ \2
& \cs{toks1}\\ \2
& \VC\\ \2
& \cs{toks255}\\ \2
& \cs{interlinepenalties}   & \T{etex_pen_base} \\ \2
& \cs{clubpenalties}   \\ \2
& \cs{widowpenalties}  \\ \2
& \cs{displaywidowpenalties} \\ \2
& \cs{box}0                 & \T{box_base} \\ \2
& \cs{box}1 \\ \2
& \VC \\ \2
& \cs{box}255 \\ \2
& cur_font  \\ \2
& \cs{textfont}0            & \T{math_font_base} \\ \2
& \VC \\ \2
& \cs{texfont}255   \\ \2
& \cs{scriptfont}0    \\ \2
& \VC \\ \2
& \cs{scriptfont}255 \\ \2
& \cs{scriptscriptfont}0    \\ \2
& \VC \\ \2
& \cs{scriptscriptfont}255 \\ \2
& \cs{catcode}0             & \T{cat_code_base} \\ \2
& \VC                   \\ \2
& \cs{catcode}"10FFFF   \\ \2
& \cs{lccode}0              & \T{lc_code_base} \\ \2
& \VC                   \\ \2
& \cs{lccode}"10FFFF   \\ \2
& \cs{uccode}0              & \T{uc_code_base} \\ \2
& \VC   \\ \2
& \cs{uccode}"10FFFF   \\ \2
& \cs{sfcode}              & \T{sf_code_base} \\ \2
& \VC                   \\ \2
& \cs{sfcode}"10FFFF    \\ \2
& \cs{mathcode}0            & \T{math_code_base} \\ \2
& \VC                   \\ \2
& \cs{mathcode}"10FFFF  \\ \2
\1
\end{region}

\clearpage

\subsection{Regions 5 and~6}

Words in regions 5 and~6 contain a single integer (or scaled integer)
field.
\begin{table}[H]
\begin{tabular}{N|Q|Q|Q|Q|l}
\CLINE
    & \W{\texttt{int} \rlap{($\equivalent\,\mathtt{sc}$)}}
    & \texttt{integer}\\\CLINE
\CLINE
\end{tabular}
\end{table}
\unskip
and the the \T{eq_level} is stored in an auxiliary array:
\begin{verbatim}
        xeq_level: array[int_base..eqtb_size] of quarterword;
\end{verbatim}

Region~5 contains \<integer parameter>s, \cs{count} registers, and
\cs{delcode}s.

\begin{region}[Region 5]
Region 5
& \cs{pretolerance}         & \T{int_base} \\ \2
& \cs{tolerance} \\ \2
& \cs{linepenalty} \\ \2
& \cs{hyphenpenalty} \\ \2
& \cs{exhyphenpenalty} \\ \2
& \cs{clubpenalty} \\ \2
& \cs{widowpenalty} \\ \2
& \cs{displaywidowpenalty} \\ \2
& \cs{brokenpenalty} \\ \2
& \cs{binoppenalty} \\ \2
& \cs{relpenalty} \\ \2
& \cs{predisplaypenalty} \\ \2
& \cs{postdisplaypenalty} \\ \2
& \cs{interlinepenalty} \\ \2
& \cs{doublehyphendemerits} \\ \2
& \cs{finalhyphendemerits} \\ \2
& \cs{adjdemerits} \\ \2
& \cs{mag} \\ \2
& \cs{delimiterfactor} \\ \2
& \cs{looseness} \\ \2
& \cs{time} \\ \2
& \cs{day} \\ \2
& \cs{month} \\ \2
& \cs{year} \\ \2
& \cs{showboxbreadth} \\ \2
& \cs{showboxdepth} \\ \2
& \cs{hbadness} \\ \2
& \cs{vbadness} \\ \2
& \cs{pausing} \\ \2
& \cs{tracingonline} \\ \2
& \cs{tracingmacros} \\ \2
& \cs{tracingstats} \\ \2
& \cs{tracingparagraphs} \\ \2
& \cs{tracingpages} \\ \2
& \cs{tracingoutput} \\ \2
& \cs{tracinglost_chars} \\ \2
& \cs{tracingcommands} \\ \2
& \cs{tracingrestores} \\ \2
& \cs{uchyph} \\ \2
& \cs{outputpenalty} \\ \2
& \cs{maxdeadcycles} \\ \2
& \cs{hangafter} \\ \2
& \cs{floatingpenalty} \\ \2
& \cs{globaldefs} \\ \2
& \cs{fam} \\ \2
& \cs{escapechar} \\ \2
& \cs{defaulthyphenchar} \\ \2
& \cs{defaultskewchar} \\ \2
& \cs{endlinechar} \\ \2
& \cs{newlinechar} \\ \2
& \cs{language} \\ \2
& \cs{lefthyphenmin} \\ \2
& \cs{righthyphenmin} \\ \2
& \cs{holdinginserts} \\ \2
& \cs{errorcontextlines} \\ \2
& \cs{tracingassigns}       & \T{etex_int_base} \\ \2
& \cs{tracinggroups} \\ \2
& \cs{tracingifs} \\ \2
& \cs{tracingscantokens} \\ \2
& \cs{tracingnesting} \\ \2
& \cs{predisplaydirection} \\ \2
& \cs{lastlinefit} \\ \2
& \cs{savingvdiscards} \\ \2
& \cs{savinghyphs} \\ \2
& \cs{suppressfontnotfounderror} \\ \2
& \cs{XeTeXlinebreaklocale} \\ \2
& \cs{XeTeXlinebreakpenalty} \\ \2
& \cs{XeTeXprotrudechars} \\ \2

& \cs{count}0               & \T{count_base}\\ \2
& \cs{count}1 \\ \2
& \VC \\ \2
& \cs{count}255 \\ \2
& del_code_base             & \T{del_code_base} \\ \2
& \VCa{$\T{number_usvs} - 1$}\\ \2
\1
\end{region}

Region~6 contains \<dimen parameter>s and \cs{dimen} registers.

\begin{region}[Region 6]
Region 6
& \cs{parident}             & \T{dimen_base} \\ \2
& \cs{mathsurround} \\ \2
& \cs{lineskiplimit} \\ \2
& \cs{hsize} \\ \2
& \cs{vsize} \\ \2
& \cs{maxdepth} \\ \2
& \cs{splitmaxdepth} \\ \2
& \cs{boxmaxdepth} \\ \2
& \cs{hfuzz} \\ \2
& \cs{vfuzz} \\ \2
& \cs{delimitershortfall} \\ \2
& \cs{nulldelimiterspace} \\ \2
& \cs{scriptspace} \\ \2
& \cs{predisplaysize} \\ \2
& \cs{displaywidth} \\ \2
& \cs{displayindent} \\ \2
& \cs{overfullrule} \\ \2
& \cs{hangindent} \\ \2
& \cs{hoffset} \\ \2
& \cs{voffset} \\ \2
& \cs{emergencystretch} \\ \2
& \cs{pdfpagewidth} \\ \2
& \cs{pdfpageheight} \\ \2
& \cs{dimen}0               & \T{scaled_base} \\ \2
& \cs{dimen}1 \\ \2
& \VC \\ \2
& \cs{dimen}255 \\ \2
& eqtb_size \\ \1
\end{region}

\end{document}

HASH

@d next(#) == hash[#].lh { link for coalesced lists }
@d text(#) == hash[#].rh { string number for control sequence name }

EQTB

@d eq_level_field(#) == #.hh.b1
@d eq_type_field(#)  == #.hh.b0
@d equiv_field(#)    == #.hh.rh

@d eq_level(#) == eq_level_field(eqtb[#]) { level of definition }
@d eq_type(#)  == eq_type_field (eqtb[#]) { command code for equivalent }
@d equiv(#)    == equiv_field   (eqtb[#]) { equivalent value }

@d skip(#)      == equiv(skip_base    + #) { mem location of glue specification }
@d mu_skip(#)   == equiv(mu_skip_base + #) { mem location of math glue spec }
@d glue_par(#)  == equiv(glue_base    + #) { mem location of glue specification }

@d del_code(#)  == eqtb[del_code_base + #].int
@d count(#)     == eqtb[count_base    + #].int
@d int_par(#)   == eqtb[int_base      + #].int { an integer parameter }

@d dimen(#)     == eqtb[scaled_base + #].sc
@d dimen_par(#) == eqtb[dimen_base  + #].sc { a scaled quantity }

@d prim_eq_level(#) == prim_eq_level_field(eqtb[prim_eqtb_base + #]) { level of definition }
@d prim_eq_type(#)  == prim_eq_type_field (eqtb[prim_eqtb_base + #]) { command code for equivalent }
@d prim_equiv(#)    == prim_equiv_field   (eqtb[prim_eqtb_base + #]) { equivalent value }

@d eTeX_state(#)   == eqtb[eTeX_state_base + #].int { an eTeX state variable }

MEM

@d link(#)         == mem[#].hh.rh { the link field of a memory word }
@d info(#)         == mem[#].hh.lh { the info field of a memory word }

@d type(#)         == mem[#].hh.b0 { identifies what kind of node this is }
@d subtype(#)      == mem[#].hh.b1 { secondary identification in some cases }
@d width(#)        == mem[# + width_offset].sc  { width of the box, in sp }
@d depth(#)        == mem[# + depth_offset].sc  { depth of the box, in sp }
@d height(#)       == mem[# + height_offset].sc { height of the box, in sp }
@d shift_amount(#) == mem[# + 4].sc { repositioning distance, in sp }
@d glue_set(#)     == mem[# + glue_offset].gr
@d float_cost(#)   == mem[# + 1].int   { the floating_penalty to be used }
@d adjust_ptr(#)   == mem[# + 1].int
@d native_size(#)           == mem[# + 4].qqqq.b0
@d native_font(#)           == mem[# + 4].qqqq.b1
@d native_length(#)         == mem[# + 4].qqqq.b2
@d native_glyph_count(#)    == mem[# + 4].qqqq.b3
@d native_glyph_info_ptr(#) == mem[# + 5].ptr
@d pic_path_length(#) == mem[# + 4].hh.b0
@d pic_page(#)        == mem[# + 4].hh.b1
@d pic_transform1(#)  == mem[# + 5].hh.lh
@d pic_transform2(#)  == mem[# + 5].hh.rh
@d pic_transform3(#)  == mem[# + 6].hh.lh
@d pic_transform4(#)  == mem[# + 6].hh.rh
@d pic_transform5(#)  == mem[# + 7].hh.lh
@d pic_transform6(#)  == mem[# + 7].hh.rh
@d pic_pdf_box(#)     == mem[# + 8].hh.b0
@d stretch(#)        == mem[# + 2].sc { the stretchability of this glob of glue }
@d shrink(#)         == mem[# + 3].sc { the shrinkability of this glob of glue }
@d penalty(#)    == mem[# + 1].int { the added cost of breaking a list here }
@d glue_stretch(#)  == mem[# + glue_offset].sc { total stretch in an unset node }

@d if_line_field(#) == mem[# + 1].int
@d location(#)==mem[# + 2].int { DVI byte number for a movement command }
@d math_type            == link { a halfword in mem }
@d plane_and_fam_field  == font { a quarterword in mem }
@d radical_noad_size=5 { number of mem words in a radical noad }
@d fraction_noad_size=6 { number of mem words in a fraction noad }
@d small_fam(#)==(mem[#].qqqq.b0 mod "100) { fam for ``small'' delimiter }
@d small_char(#)==(mem[#].qqqq.b1 + (mem[#].qqqq.b0 div "100) * "10000) { character for ``small'' delimiter }
@d large_fam(#)==(mem[#].qqqq.b2 mod "100) { fam for ``large'' delimiter }
@d large_char(#)==(mem[#].qqqq.b3 + (mem[#].qqqq.b2 div "100) * "10000) { character for ``large'' delimiter }
@d small_plane_and_fam_field(#)==mem[#].qqqq.b0
@d small_char_field(#)==mem[#].qqqq.b1
@d large_plane_and_fam_field(#)==mem[#].qqqq.b2
@d large_char_field(#)==mem[#].qqqq.b3
@d accent_noad_size  =     5 { number of mem words in an accent noad }
@d new_hlist(#) == mem[nucleus(#)].int { the translation of an mlist }
@d u_part(#)     == mem[# + height_offset].int { pointer to <u_j> token list }
@d v_part(#)     == mem[# + depth_offset].int  { pointer to <v_j> token list }
@d extra_info(#) == info(# + list_offset) { info to remember during template }
@d align_stack_node_size = 6 { number of mem words to save alignment states }
@d span_node_size = 2 { number of mem words for a span node }
@d total_demerits(#) == mem[# + 2].int { the quantity that TeX minimizes }
@d update_active(#) == active_width[#] := active_width[#] + mem[r + #].sc

@d total_pic_node_size(#) == (pic_node_size + (pic_path_length(#) + sizeof(memory_word) - 1) div sizeof(memory_word))

@d expr_e_field(#) == mem[# + 1].int { saved expression so far }
@d expr_t_field(#) == mem[# + 2].int { saved term so far }
@d expr_n_field(#) == mem[# + 3].int { saved numerator }
@d sa_int(#) == mem[# + 2].int { an integer }
@d sa_dim(#) == mem[# + 2].sc  { a dimension (a somewhat esotheric distinction) }
@d active_short(#) == mem[# + 3].sc { shortfall of this line }
@d active_glue(#)  == mem[# + 4].sc { corresponding glue stretch or shrink }
