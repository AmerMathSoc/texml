\documentclass{amsart}

\usepackage{web2c}

\usepackage{longtable}

\begin{document}

Quoting \texttt{tex.web} (part 17, \S221)
\begin{quotation}

Each entry in \texttt{eqtb} is a \texttt{memory_word}. Most of these
words are of type \texttt{two_halves}, and subdivided into three
fields:
\begin{enumerate}
\item The \texttt{eq_level} (a quarterword) is the level of grouping
  at which this equivalent was defined. If the level is
  \texttt{level_zero}, the equivalent has never been defined;
  \texttt{level_one} refers to the outer level (outside of all
  groups), and this level is also used for global definitions that
  never go away. Higher levels are for equivalents that will disappear
  at the end of their group.

\item The \texttt{eq_type} (another quarterword) speci<fies what kind
  of entry this is. There are many types, since each \TeX\ primitive
  like \cs{hbox}, \cs{def}, etc., has its own special
  code. The list of command codes above includes all possible settings
  of the \texttt{eq_type} field.

\item The \texttt{equiv} (a halfword) is the current equivalent value.
  This may be a font number, a pointer into \texttt{mem}, or a variety
  of other things.

\end{enumerate}
\end{quotation}

\newnode{two_halves}{}{}{}

\begin{tabular}{N|Q|Q|Q|Q|l}
\CLINE
    & \X{eq_type}
    & \X{eq_level}
    & \H{\X{equiv}}\\
\CLINE
\end{tabular}

For example, after
\begin{verbatim}
    primitive("hbox", make_box, vtop_code + hmode);
\end{verbatim}
\texttt{cur_val} will be the location of the string ``hbox'' in the
hash table (i.e., between \texttt{hash_base} and \texttt{hash_base} +
\texttt{hash_size}) and
\begin{verbatim}
    eq_level(cur_val) = level_one
    eq_type(cur_val)  = make_box
    equiv(cur_val)    = vtop_code + hmode
\end{verbatim}
And after
\begin{verbatim}
    \def\foo{...}
\end{verbatim}
assuming \cs{globaldefs} = 0, we have
\begin{verbatim}
    p           = id_lookup("foo")
    eq_level(p) = cur_level
    eq_type(p)  = call
    equiv(p)    = def_ref
\end{verbatim}

\clearpage

\newnode{eqtb (XeTeX 3.14159265-2.6-0.999992)}{}{}{}

\begin{longtable}{|l|>{\ttfamily}l|l}
\1
\endhead
\1
\endfoot
& & 0\\ \1

Region 1
& active_base & 1\\ \2
& \VCa{$\mathtt{number_usvs} - 1$}\\ \2
& single_base \\ \2
& \VCa{$\mathtt{number_usvs} - 1$}\\ \2
& null_cs      \\ \1

Region 2
& hash_base \\ \2
& \VCa{$\mathtt{hash_size} - 1$}\\ \2
& frozen_control_sequence = frozen_protection \\ \2
& frozen_cr             \\ \2
& frozen_end_group      \\ \2
& frozen_right          \\ \2
& frozen_fi             \\ \2
& frozen_end_template   \\ \2
& frozen_endv           \\ \2
& frozen_relax          \\ \2
& end_write             \\ \2
& frozen_dont_expand    \\ \2
& frozen_null_font = font_id_base \hfill (\cs{fnt}0)\\ \2
& frozen_primitive     \hfill (\cs{fnt}1) \\ \2
& prim_eqtb_base        \hfill (\cs{fnt}2)\\ \2
& \cs{fnt}3 \\ \2
& \VC \\ \2
& \cs{fnt}256 \\ \2
& undefined_control_sequence \\ \1

Region 3
& \cs{lineskip} & \texttt{glue_base}\\ \2
& \cs{baselineskip} \\ \2
& \cs{parskip} \\ \2
& \cs{abovedisplayskip} \\ \2
& \cs{belowdisplayskip} \\ \2
& \cs{abovedisplayshortskip} \\ \2
& \cs{belowdisplayshortskip} \\ \2
& \cs{leftskip} \\ \2
& \cs{rightskip} \\ \2
& \cs{topskip} \\ \2
& \cs{splittopskip} \\ \2
& \cs{tabskip} \\ \2
& \cs{spaceskip} \\ \2
& \cs{xspaceskip} \\ \2
& \cs{parfillskip} \\ \2
& \cs{XeTeXlinebreakskip} \\ \2
& \cs{thinmuskip} \\ \2
& \cs{medmuskip} \\ \2
& \cs{thickmuskip} \\ \2
& \cs{skip}0 & $\mathtt{skip_base}$ \\ \2
& \cs{skip}1\\ \2
& \VC\\ \2
& \cs{skip}255\\ \2
& \cs{muskip}0 & $\mathtt{mu_skip_base}$ \\ \2
& \cs{muskip}1 \\ \2
& \VC \\ \2
& \cs{muskip}255 \\ \2
\1

Region 4
& \cs{parshape}      & \texttt{local_base} \\ \2
& \cs{output}   \\ \2
& \cs{everypar}        \\ \2
& \cs{everymath}       \\ \2
& \cs{everydisplay}    \\ \2
& \cs{everyhbox}       \\ \2
& \cs{everyvbox}       \\ \2
& \cs{everyjob}        \\ \2
& \cs{everycr}         \\ \2
& \cs{errhelp}         \\ \2
& \cs{everyeof}  & \texttt{etex_toks_base} \\ \2
& \cs{XeTeXinterchartoks} \\ \2
& \cs{toks}0 & \texttt{toks_base} \\ \2
& \cs{toks1}\\ \2
& \VC\\ \2
& \cs{toks255}\\ \2
& \cs{interlinepenalties} & \texttt{etex_pen_base} \\ \2
& \cs{clubpenalties}   \\ \2
& \cs{widowpenalties}  \\ \2
& \cs{displaywidowpenalties} \\ \2
& \cs{box}0 & \texttt{box_base} \\ \2
& \cs{box}1 \\ \2
& \VC \\ \2
& \cs{box}255 \\ \2
& cur_font  \\ \2
& math_font_base \\ \2
& \VCa{$3 * 256 - 1$} \\ \2
& cat_code_base\\ \2
& \VCa{$\mathtt{number_usvs} - 1$}\\ \2
& lc_code_base \\ \2
& \VCa{$\mathtt{number_usvs} - 1$}\\ \2
& uc_code_base \\ \2
& \VCa{$\mathtt{number_usvs} - 1$}\\ \2
& sf_code_base \\ \2
& \VCa{$\mathtt{number_usvs} - 1$}\\ \2
& math_code_base \\ \2
& \VCa{$\mathtt{number_usvs} - 1$}\\ \2
\1

Region 5
& \cs{pretolerance} & \texttt{int_base} \\ \2
& \cs{tolerance} \\ \2
& \cs{linepenalty} \\ \2
& \cs{hyphenpenalty} \\ \2
& \cs{exhyphenpenalty} \\ \2
& \cs{clubpenalty} \\ \2
& \cs{widowpenalty} \\ \2
& \cs{displaywidowpenalty} \\ \2
& \cs{brokenpenalty} \\ \2
& \cs{binoppenalty} \\ \2
& \cs{relpenalty} \\ \2
& \cs{predisplaypenalty} \\ \2
& \cs{postdisplaypenalty} \\ \2
& \cs{interlinepenalty} \\ \2
& \cs{doublehyphendemerits} \\ \2
& \cs{finalhyphendemerits} \\ \2
& \cs{adjdemerits} \\ \2
& \cs{mag} \\ \2
& \cs{delimiterfactor} \\ \2
& \cs{looseness} \\ \2
& \cs{time} \\ \2
& \cs{day} \\ \2
& \cs{month} \\ \2
& \cs{year} \\ \2
& \cs{showboxbreadth} \\ \2
& \cs{showboxdepth} \\ \2
& \cs{hbadness} \\ \2
& \cs{vbadness} \\ \2
& \cs{pausing} \\ \2
& \cs{tracingonline} \\ \2
& \cs{tracingmacros} \\ \2
& \cs{tracingstats} \\ \2
& \cs{tracingparagraphs} \\ \2
& \cs{tracingpages} \\ \2
& \cs{tracingoutput} \\ \2
& \cs{tracinglost_chars} \\ \2
& \cs{tracingcommands} \\ \2
& \cs{tracingrestores} \\ \2
& \cs{uchyph} \\ \2
& \cs{outputpenalty} \\ \2
& \cs{maxdeadcycles} \\ \2
& \cs{hangafter} \\ \2
& \cs{floatingpenalty} \\ \2
& \cs{globaldefs} \\ \2
& \cs{fam} \\ \2
& \cs{escapechar} \\ \2
& \cs{defaulthyphenchar} \\ \2
& \cs{defaultskewchar} \\ \2
& \cs{endlinechar} \\ \2
& \cs{newlinechar} \\ \2
& \cs{language} \\ \2
& \cs{lefthyphenmin} \\ \2
& \cs{righthyphenmin} \\ \2
& \cs{holdinginserts} \\ \2
& \cs{errorcontextlines} \\ \2
& \cs{tracingassigns} & \texttt{etex_int_base} \\ \2
& \cs{tracinggroups} \\ \2
& \cs{tracingifs} \\ \2
& \cs{tracingscantokens} \\ \2
& \cs{tracingnesting} \\ \2
& \cs{predisplaydirection} \\ \2
& \cs{lastlinefit} \\ \2
& \cs{savingvdiscards} \\ \2
& \cs{savinghyphs} \\ \2
& \cs{suppressfontnotfounderror} \\ \2
& \cs{XeTeXlinebreaklocale} \\ \2
& \cs{XeTeXlinebreakpenalty} \\ \2
& \cs{XeTeXprotrudechars} \\ \2

& \cs{count}0 & \texttt{count_base}\\ \2
& \cs{count}1 \\ \2
& \VC \\ \2
& \cs{count}255 \\ \2
& del_code_base \\ \2
& \VCa{$\mathtt{number_usvs} - 1$}\\ \2
\1

Region 6
& \cs{parident}    & \texttt{dimen_base} \\ \2
& \cs{mathsurround} \\ \2
& \cs{lineskiplimit} \\ \2
& \cs{hsize} \\ \2
& \cs{vsize} \\ \2
& \cs{maxdepth} \\ \2
& \cs{splitmaxdepth} \\ \2
& \cs{boxmaxdepth} \\ \2
& \cs{hfuzz} \\ \2
& \cs{vfuzz} \\ \2
& \cs{delimitershortfall} \\ \2
& \cs{nulldelimiterspace} \\ \2
& \cs{scriptspace} \\ \2
& \cs{predisplaysize} \\ \2
& \cs{displaywidth} \\ \2
& \cs{displayindent} \\ \2
& \cs{overfullrule} \\ \2
& \cs{hangindent} \\ \2
& \cs{hoffset} \\ \2
& \cs{voffset} \\ \2
& \cs{emergencystretch} \\ \2
& \cs{pdfpagewidth} \\ \2
& \cs{pdfpageheight} \\ \2
& \cs{dimen}0   & \texttt{scaled_base} \\ \2
& \cs{dimen}1 \\ \2
& \VC \\ \2
& \cs{dimen}255 \\ \2
& eqtb_size \\ \1
\end{longtable}

\end{document}
